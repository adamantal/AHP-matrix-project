\documentclass{article}
\usepackage[latin1]{inputenc}
\usepackage{amssymb, amsmath, amsthm}
\usepackage[a4paper]{geometry}
%\usepackage{authblk} % for headings
%\usepackage{pifont}
%\usepackage{graphicx}
%\usepackage{xtab} % tackle the long tables
%\usepackage{longtable} % tackle the long tables
\usepackage{footnote}
\makesavenoteenv{tabular}
\usepackage{tabularx}
%\usepackage{tabu}
\usepackage{rotating}
\usepackage{color}

\theoremstyle{plain}
\newtheorem{conjecture}{Conjecture}
\newtheorem{corollary}{Corollary}
\newtheorem{definition}{Definition}[section]
\newtheorem{example}{Example}[section]
\newtheorem{lemma}{Lemma}[section]
\newtheorem{problem}{Problem}
\newtheorem{proposition}{Proposition}[section]
\newtheorem{remark}{Remark}
\newtheorem{theorem}{Theorem}[section]
\newcommand{\aproof}{\hfill{\ding{111}}}

\def\keywords{\vspace{.5em} % Add keywords
{\textit{Keywords}:\,\relax%
}}
\def\endkeywords{\par}

\begin{document}
%\pagenumbering{}
\begin{center}
\large{\textbf{The logarithmic least squares optimality of the geometric mean of weight vectors
calculated from all spanning trees for (in)complete pairwise comparison matrices}} \\[1cm]
{S\'andor Boz\'oki$^{1,2}$,}  {Vitaliy Tsyganok$^{3,4}$}   \\[1cm]
\end{center}
\footnotetext[1]{Laboratory on Engineering and Management
Intelligence, Research Group of Operations Research and Decision
Systems, Institute for Computer Science and Control, Hungarian
Academy of Sciences; Mail: 1518 Budapest, P.O.~Box 63, Hungary. E-mail: bozoki.sandor@sztaki.mta.hu}
\footnotetext[2]{Department of Operations Research and Actuarial Sciences, Corvinus
University of Budapest, Hungary}
\footnotetext[3]{Laboratory for Decision Support Systems,
Institute for Information Recording of National Academy of Sciences of Ukraine; Mail:
2, Shpak str., Kyiv, 03113, Ukraine. E-mail: tsyganok@ipri.kiev.ua}
\footnotetext[4]{Department of System Analysis, State University of Telecommunications, Ukraine}

\date{}

\begin{abstract}
Pairwise comparison matrices, a method for preference modelling and quantif{\kern0pt}ication
in multi-attribute decision making and ranking problems, are naturally extended to the incomplete case,
of{\kern0pt}fering a wider range of applicability. The weighting problem is to f{\kern0pt}ind a weight vector
that ref{\kern0pt}lects the decision maker's preferences in the best possible way. 
The logarithmic least squares problem has a unique and easily computable solution.
The spanning tree approach does not assume any metric in advance, instead, it
goes through all minimal suf{\kern0pt}f{\kern0pt}icient subsets (spanning trees) of the set of pairwise comparisons,
 and f{\kern0pt}inally the weight vectors are aggregated.
It is shown that the geometric mean of weight vectors, calculated from all spanning trees,
is the optimal solution of the well known logarithmic least squares problem, not only for
complete, as it was recently proved by Lundy, Siraj and Greco, but
for incomplete pairwise comparison matrices as well.
\end{abstract}

\section{Incomplete pairwise comparison matrices}

Cardinal preferences of decision makers are often modelled and calculated by pairwise comparison matrices
\cite{Saaty1977}.
Questions 'How many times is a criterion more important than another one?'
or 'How many times is a given alternative better than another one with
respect to a f{\kern0pt}ixed criterion?'
are typical in multi-attribute decision problems.
The numerical answers are collected into a pairwise comparison matrix $\mathbf{A}=[a_{ij}]_{i,j=1 \ldots n}$
having reciprocity, i.e., $a_{ij} = 1/a_{ji}.$
A pairwise comparison matrix can be complete, as in a classical AHP model \cite{Saaty1977},
or incomplete \cite{Harker1987b,Kwiesielewicz1996}. In this study
incomplete means 'not necessarily complete', in other words, the number of missing elements is allowed to be zero.

Incomplete pairwise comparison matrices are applied not only in the same decision situations
in which the complete matrices arise, but also to larger decision and ranking problems.
Boz\'oki, Csat\'o and Temesi \cite{BozokiCsatoTemesi2016} proposed a
ranking method for top tennis players based on their pairwise results, where incompleteness occurs in a natural way.
Csat\'o \cite{Csato2013} constructed a $149 \times 149$ incomplete pairwise comparison matrix to rank the teams
of the 39th Chess Olympiad 2010.

\begin{example} \label{6x6example} % \ref{6x6example}
Let $\mathbf{A}$ be a $6 \times 6$ incomplete pairwise comparison matrix as follows:
\[
\mathbf{A} =
\begin{pmatrix}
     1    &   a_{12}     &            &     a_{14}   &    a_{15}    &     a_{16}        \\
  a_{21}  &     1        &    a_{23}  &              &              &                   \\
          &   a_{32}     &      1     &     a_{34}   &              &                   \\
  a_{41}  &              &    a_{43}  &      1       &    a_{45}    &                   \\
  a_{51}  &              &            &     a_{54}   &       1      &                   \\
  a_{61}  &              &            &              &              &         1
\end{pmatrix}.
\]
\end{example}

\section{The logarithmic least squares (LLS) problem}

The basic problem of f{\kern0pt}inding the best weight vectors usually includes an additional
information on how closeness is def{\kern0pt}ined or specif{\kern0pt}ied.
The classical approaches apply metrics based on least squares
\cite{ChuKalabaSpingarn1979}, weighted least squares \cite{ChuKalabaSpingarn1979},
logarithmic least squares \cite{CrawfordWilliams1985,deJong1984}, just to name a few.
 Further weighting methods are discussed by Golany and Kress \cite{GolanyKress1993}
and by Choo and Wedley \cite{ChooWedley2004}.
Even the well-known eigenvector method \cite{Saaty1977} is proved to be a
distance minimizing method \cite{Fichtner1984,Fichtner1986}, although its
metric seems to be rather artif{\kern0pt}icial.\\

\begin{definition}
The Logarithmic Least Squares (LLS) problem \cite{Kwiesielewicz1996} is def{\kern0pt}ined as follows:
\begin{align}
&\min \sum \limits_{\scriptsize{
             \begin{array}{c}
              i,j:  \\
              a_{ij} \text{ is known} \\
             \end{array}}}
\left[\log a_{ij}
-\log\left(\frac{w_{i}}{w_{j}}\right)\right]^2
                                                  \label{eq:IncompleteLLSMProblem-ObjFunction}  \\
%                                    %(\ref{eq:IncompleteLLSMProblem-ObjFunction})
&\text{subject to } \qquad w_{i} > 0, \qquad i=1,2,\dotsc,n.
    \label{eq:IncompleteLLSMProblem-Positivity}
%                                    %(\ref{eq:IncompleteLLSMProblem-Positivity})
\end{align}
\end{definition}

Originally, the LLS problem was def{\kern0pt}ined for complete pairwise comparison matrices,
i.e., the sum in the objective function is taken for all $i,j$. In this special case,
the LLS optimal solution is unique and it can be explicitly computed by taking  the
row-wise geometric mean \cite{CrawfordWilliams1985,deJong1984}. \\

The most common normalizations are $\sum\limits_{i=1}^{n}w_{i} = 1$ and
$\prod\limits_{i=1}^{n}w_{i} = 1.$ Normalization $w_{1} = 1$ (called
\emph{ideal-mode} in Lundy, Siraj and Greco \cite{LundySirajGreco2017}),
can also be interpreted in the following way:
the f{\kern0pt}irst object (criterion, alternative) is considered
a reference point and all the others are expressed according to it.

Given an (in)complete pairwise comparison matrix $\mathbf{A}$ of
size $n \times n$, an undirected graph $G(V,E)$ is def{\kern0pt}ined as follows: $G$ has $n$ nodes and
the edge between nodes $i$ and $j$ is drawn if and only if the matrix element
$a_{ij}$ is known.\\

The incomplete LLS problem can be solved by the following theorem.
\begin{theorem} \label{BozokiFulopRonyai2010theorem} % \ref{BozokiFulopRonyai2010theorem}
(Boz\'oki, F\"ul\"op, R\'onyai \cite{BozokiFulopRonyai2010})
Let $\mathbf{A}$ be an incomplete or complete pairwise comparison matrix such that its
associated graph $G$ is connected. Then the optimal solution $\mathbf{w} = \exp \mathbf{y}$
of the logarithmic least squares problem is the unique solution of the following system of linear equations:
\begin{align}
\left( \mathbf{L}  \mathbf{y}          \right)_i  &=    \sum\limits_{k: e(i,k) \in E(G)} \log a_{ik}
\qquad \qquad \text{ for all } i=1,2,\ldots,n-1,n,  \label{equationLaplacian} \\
                                                                                    %  \ref{equationLaplacian}
                                    y_1  &= 0.        \label{normalizationy1=0}      % \ref{normalizationy1=0}
\end{align}
where $\mathbf{L}$ denotes the Laplacian matrix of $G$ ($\ell_{ii}$ is the degree of node $i$ and $\ell_{ij}=-1$
if nodes $i$ and $j$ are adjacent).
\end{theorem}
$\mathbf{L}$ has rank $n-1$.
Normalization (\ref{normalizationy1=0}), being equivalent to $w_1 = 1,$ plays a technical role only.
It can be replaced by, e.g., the commonly used $\prod_{i=1}^n w_i = 1 \, \, (\Leftrightarrow \sum_{i=1}^n y_i = 0).$
The computational complexity of solving the system of $n$ linear
equations (\ref{equationLaplacian})-(\ref{normalizationy1=0}) is at most $O(n^3)$ \cite[Chapter 1]{Watkins2002}.


\begin{example} \label{6x6exampleLaplacian} % \ref{6x6exampleLaplacian}
Let incomplete pairwise comparison matrix $\mathbf{A}$ be the same as in Example \ref{6x6example}.
 The equations in Eq.~(\ref{equationLaplacian})
  for $i=1,2,\ldots,6$ form the following system of linear equations:
\[
\begin{pmatrix}
     4    &     -1       &      0     &     -1       &      -1      &        -1         \\
    -1    &      2       &     -1     &      0       &       0      &         0         \\
     0    &     -1       &      2     &     -1       &       0      &         0         \\
    -1    &      0       &     -1     &      3       &      -1      &         0         \\
    -1    &      0       &      0     &     -1       &       2      &         0         \\
    -1    &      0       &      0     &      0       &       0      &         1
\end{pmatrix}
\begin{pmatrix}
    y_1    \\
    y_2    \\
    y_3    \\
    y_4    \\
    y_5    \\
    y_6
\end{pmatrix} =
\begin{pmatrix}
   \log a_{12}  + \log a_{14}  +\log a_{15} + \log  a_{16}    \\
   \log a_{21}  + \log a_{23}    \\
   \log a_{32}  + \log a_{34}    \\
   \log a_{41}  + \log a_{43}  +\log a_{45}   \\
   \log a_{51}  + \log a_{54}    \\
   \log a_{61}
\end{pmatrix}.
\]
\end{example}


\section{Aggregations of weight vectors calculated from all spanning trees}


The spanning tree approach by Tsyganok \cite{Tsyganok2000,Tsyganok2010}
does not assume any distance function or measure of closeness. The basic idea
is that the set of pairwise comparisons is considered as the union of minimal, connected
subsets, or, in graph theoretical terms, spanning trees.
Let $S$ denote the number of all spanning trees of graph $G$.
Every spanning tree determines a unique
weight vector f{\kern0pt}itting on the corresponding subset of matrix elements perfectly.
Given a spanning tree, the calculation of its associated weight vector requires $O(n)$ steps.


The number of spanning trees can be very large.
In the special case of complete pairwise comparison matrices, the number of all spanning trees is
$S= n^{n-2}$ by Cayley's theorem.
Another extremal case is when the graph of the incomplete pairwise comparison
matrix is itself a tree ($S=1$).
The enumeration of all spanning trees with the algorithm of Gabow and Myers \cite{GabowMyers1978},
requires $O(n+m+nS)$ steps, where $m$ denotes the number of edges in $G$.


The computational complexity of calculating all weight vectors, associated to the spanning trees,
is $\max\{O(nS), O(n+m+nS) \}$ steps, where $S$,
the number of spanning trees, is between 1 and $n^{n-2}$.


The most natural candidates for the aggregation of weight vectors calculated from all spanning trees
are the arithmetic
 \cite{SirajMikhailovKeane2012a,SirajMikhailovKeane2012b,Tsyganok2000,Tsyganok2010}
and the geometric means \cite{LundySirajGreco2017,TsyganokKadenkoAndriichuk2015}.


The following theorem connects two weighting methods.
\begin{theorem}  \label{LundySirajGreco2017theorem} % \ref{LundySirajGreco2017theorem}
(Lundy, Siraj and Greco \cite{LundySirajGreco2017})
The geometric mean of weight vectors calculated from all spanning trees
is logarithmic least squares optimal in case of complete pairwise comparison matrices.
\end{theorem}

The rest of this paper provides the generalization of this result,
it is shown that it holds for incomplete matrices as well.


\section{Main result: the geometric mean of weight vectors calculated from all spanning trees
is logarithmic least squares optimal}

\begin{theorem} \label{maintheorem} % \ref{maintheorem}
Let $\mathbf{A}$ be an incomplete or complete pairwise comparison matrix such that its
associated graph is connected.
Then the optimal solution of the logarithmic least squares problem
is equal, up to a scalar multiplier, to the geometric mean of weight vectors
calculated from all spanning trees.
\end{theorem}

\begin{proof}
Let $G$ be the connected graph associated with the (in)complete pairwise comparison matrix $\mathbf{A}$
and let $E(G)$ denote the set of edges. The edge between nodes $i$ and $j$ is denoted by $e(i,j)$.
The Laplacian matrix of graph $G$ is denoted by $\mathbf{L}$.
Let $T^1, T^2, \ldots, T^s, \ldots, T^S$ denote the spanning trees of $G$, where $S$ denotes the number of spanning trees.
$E(T^s)$ denotes the set of edges in $T^s$.
Hereafter, upper index $s$ is also used for indexing a weight vector
or a pairwise comparison matrix, associated to spanning tree $T^s$.
Let $\mathbf{w}^s, s=1,2,\ldots,S,$ denote the weight vector calculated from spanning tree $T^s$.
Weight vector $\mathbf{w}^s$ is unique up to a scalar multiplier.
For sake of simplicity we can
assume that $w_1^s = 1$, but other ways of normalization, e.g., $\prod w_i = 1$ can also be chosen.
Let $\mathbf{y}^s := \log \mathbf{w}^s, \, s=1,2,\ldots,S$, where the logarithm is taken element-wise.
Let $\mathbf{w}^{LLS}$ denote the optimal solution to the Logarithmic Least Squares problem
(normalized by $w_1^{LLS} = 1$) and $\mathbf{y}^{LLS} := \log \mathbf{w}^{LLS}$.
 By Theorem \ref{BozokiFulopRonyai2010theorem},
\begin{equation*}
\left( \mathbf{L} \mathbf{y}^{LLS} \right)_i = \sum\limits_{k: e(i,k) \in E(G)} b_{ik}
\qquad \qquad \text{ for all } i=1,2,\ldots,n,
\end{equation*}
where $b_{ik} = \log a_{ik}$ for all $(i,k) \in E(G).$

It is suf{\kern0pt}f{\kern0pt}icient to show that
\begin{equation} \label{equation:goal} % \ref{equation:goal}
\left( \mathbf{L} \frac{1}{S} \sum\limits_{s=1}^{S}\mathbf{y}^s \right)_i = \sum\limits_{k: e(i,k) \in E(G)} b_{ik}
\qquad \qquad \text{ for all } i=1,2,\ldots,n.
\end{equation}

Consider an arbitrary spanning tree $T^s$. Then $\frac{w^s_i}{w^s_j} = a_{ij}$ for all
$e(i,j) \in E(T^s)$. Introduce the incomplete pairwise comparison matrix $\mathbf{A}^s$
 by $a^s_{ij} := a_{ij}$ for all $e(i,j) \in E(T^s)$ and
$a^s_{ij} := \frac{w^s_i}{w^s_j}$ for all $e(i,j) \in E(G) \backslash E(T^s)$.
Again, $b^s_{ij} := \log a^s_{ij} (= y^s_i - y^s_j)$.
Note that the Laplacian matrices of $\mathbf{A}$ and $\mathbf{A}^s$ are the same ($\mathbf{L}$).
Since the weight vector $\mathbf{w}^s$ is generated by the matrix elements
belonging to spanning tree $T^s$, it is also the optimal solution of the
 LLS problem regarding
$\mathbf{A}^s$.
Equivalently, the following system of linear equations holds.
\begin{equation}
\left( \mathbf{L} \mathbf{y}^s \right)_i=
\sum\limits_{k: e(i,k) \in E(T^s)} b_{ik} + \sum\limits_{k: e(i,k) \in E(G) \backslash E(T^s)} b^s_{ik}
\qquad \qquad
\text{ for all } i=1,2,\ldots,n. \label{equation:Ly^s} % \ref{equation:Ly^s}
\end{equation}

\begin{lemma} \label{lemma:average} % \ref{lemma:average}
\begin{equation}
\sum\limits_{s=1}^{S}
\left(\sum\limits_{k: e(i,k) \in E(T^s)} b_{ik} + \sum\limits_{k: e(i,k) \in E(G) \backslash E(T^s)} b^s_{ik} \right) =
S \sum\limits_{k: e(i,k) \in E(G)} b_{ik}. \label{equationlemma:average} % \ref{equationlemma:average}
\end{equation}
\end{lemma}
\begin{proof} % of Lemma \ref{lemma:average}.
Let $i$ be f{\kern0pt}ixed arbitrarily and consider node $i$ in all spanning trees.
There is nothing to do with edges $e(i,k) \in E(T^s)$.
Since $T^s$ is a spanning tree,
for every edge $e(i,k) \in E(G) \backslash E(T^s)$ there exists a unique path \\
$ P= \{ e(i,k_1), e(k_1,k_2), \ldots , e(k_{\ell},k) \} \subseteq E(T^s)$.
$P \cup e(i,k)$ is a cycle and
\begin{equation} \label{equation:bsik} % \ref{equation:bsik}
b^s_{i k} = b_{i k_1} + b_{k_1 k_2} + \ldots + b_{k_{\ell} k}.
\end{equation}
Consider the following spanning tree:
$T^{s_{i,k,k_1}^{\prime}} := (T^s \backslash e(i,k_1) ) \cup e(i,k)$
 as in Figure 1.
% \newpage
\unitlength 1mm
\begin{center}
\begin{picture}(100,100)
\put(-26,30){\resizebox{150mm}{!}{\rotatebox{0}{
\includegraphics{Figure1.eps}}}}  %
\put(10,60){\makebox{$T^s$}}
\put(90,60){\makebox{$T^{s_{i,k,k_1}^{\prime}}$}}
\put(-13,20){\makebox{Figure 1. The replacement of edge $e(i,k_1)$ in spanning tree $T^s$ by edge $e(i,k)$}}
\put(25,15){\makebox{results in spanning tree $T^{s_{i,k,k_1}^{\prime}}$.}}
\end{picture}
\end{center}

\noindent
Spanning trees $T^s$ and $T^{s_{i,k,k_1}^{\prime}}$ dif{\kern0pt}fer in one edge only and
\begin{equation} \label{equation:bsprimeik1} % \ref{equation:bsprimeik1}
b^{s_{i,k,k_1}^{\prime}}_{i k_1} = b_{i k} + b_{k k_{\ell}} + \ldots + b_{k_2 k_1}.
\end{equation}
Adding up equations (\ref{equation:bsik}) and (\ref{equation:bsprimeik1})
results in
\begin{equation} \label{equation:bsik+bsprimeik1} % \ref{equation:bsik+bsprimeik1}
b^s_{i k} + b^{s_{i,k,k_1}^{\prime}}_{i k_1} = b_{i k} + b_{i k_1},
\end{equation}
all intermediate terms vanish due to the reciprocal property of pairwise comparison matrices.
Now let us continue this process and go through all edges $e(i,k) \in E(G) \backslash E(T^s)$
for all $k$ and $s$. The remarkable symmetry of the set of all spanning trees implies that
every edge occurs in exactly one pair.
Summing all these equations like (\ref{equation:bsik+bsprimeik1}), the statement
 of Lemma \ref{lemma:average} follows.
An illustrative example is given below in Example \ref{lemmaexample}.
\end{proof}

Finally, summarizing equations in Eq.~(\ref{equation:Ly^s}) for all $s=1,2,\ldots,S$ and
applying Lemma \ref{lemma:average} complete the proof of Theorem \ref{maintheorem}, concluding
$\mathbf{y}^{LLS} = \frac{1}{S} \sum\limits_{s=1}^{S}\mathbf{y}^s. $
\end{proof}

\textbf{Remark.} Complete pairwise comparison matrices ($S=n^{n-2}$)
are included in Theorem \ref{maintheorem} as a special case.
The proof of Theorem \ref{maintheorem} can also be considered as a
second and shorter proof of Theorem \ref{LundySirajGreco2017theorem}.

\newpage
\begin{example} \label{lemmaexample} % \ref{lemmaexample}
(An illustration of Lemma \ref{lemma:average})
Let incomplete pairwise comparison matrix $\mathbf{A}$ be the same as in Example \ref{6x6example}.
The associated graph $G$ and its spanning trees $T^1, T^2, \ldots, T^{11}$ are shown in Figure 2.

Let us focus on node $i=1$. Edges departing from node 1 are missing 12 times (and they are not missing
32 times) in the whole set of spanning trees, hence we can identify 6 pairs.
They induce 6 pairs of equations, that are labelled in Figure 2. In tree $T^1$,
\begin{equation} \label{11}
b_{12}^{1} = b_{15} + b_{54} + b_{43} + b_{32}.
\end{equation}
Note that equation (11), as well as the forthcoming ones, is
labelled on the corresponding edges in Figure 2.
Now $s=1, k=2, k_1=5$ and ${s_{1,2,5}^{\prime}} = 4$, because the replacement of edge $e(1,5)$ in tree $T^1$
by edge $e(1,2)$ results in tree $T^4$. Here
\begin{equation} \label{12}
b_{15}^{4} = b_{12} + b_{23} + b_{34} + b_{45}.
\end{equation}

The sum of equations (11) and (12) conf{\kern0pt}irms (\ref{equation:bsik+bsprimeik1}).

Let us continue by edge $e(1,4)$ in tree $T^1$.
\begin{eqnarray}
b_{14}^{1} = b_{15} + b_{54}, \label{13} \\
b_{15}^{2} = b_{14} + b_{45}. \label{14}
\end{eqnarray}

The remaining four pairs of edges and their equations are listed below.
\begin{eqnarray}
b_{12}^{2} = b_{14} + b_{43} + b_{32}, \label{15} \\
b_{14}^{4} = b_{12} + b_{23} + b_{34}, \label{16}
\end{eqnarray}
\begin{eqnarray}
b_{12}^{3} = b_{14} + b_{43} + b_{32}, \label{17} \\
b_{14}^{7} = b_{12} + b_{23} + b_{34}, \label{18}
\end{eqnarray}
\begin{eqnarray}
b_{14}^{5} = b_{15} + b_{54},          \label{19} \\
b_{15}^{8} = b_{14} + b_{45},          \label{20}
\end{eqnarray}
\begin{eqnarray}
b_{14}^{6} = b_{15} + b_{54},          \label{21} \\
b_{15}^{9} = b_{14} + b_{45}.          \label{22}
\end{eqnarray}

Lemma \ref{lemma:average} is now conf{\kern0pt}irmed for $i=1$:
\begin{equation*}
\sum\limits_{s=1}^{11}
\left(\sum\limits_{k: e(1,k) \in E(T^s)} b_{1k} + \sum\limits_{k: e(1,k) \in E(G) \backslash E(T^s)} b^s_{1k} \right) =
11 \sum\limits_{k: e(1,k) \in E(G)} b_{1k} = 11 (b_{12}+b_{14}+b_{15}+b_{16}).
\end{equation*}

\newpage
\unitlength 1mm
\begin{center}
\begin{picture}(100,200)
\put(-26,10){\resizebox{150mm}{!}{\rotatebox{0}{
\includegraphics{Figure2.eps}}}}
\put(-8,0){\makebox{Figure 2. Graph $G$ of Example \ref{lemmaexample}
and its spanning trees $T^1, T^2, \ldots, T^{11}$ }}
\end{picture}
\end{center}

\newpage
Let us move to node $2$. Three pairs of equations can be obtained:
\begin{eqnarray}
b_{21}^{1} = b_{23} + b_{34} + b_{45} + b_{51},      \label{23} \\
b_{23}^{5} = b_{21} + b_{15} + b_{54} + b_{43},      \label{24}
\end{eqnarray}
\begin{eqnarray}
b_{21}^{2} = b_{23} + b_{34} + b_{41},               \label{25} \\
b_{23}^{8} = b_{21} + b_{14} + b_{43},               \label{26}
\end{eqnarray}
\begin{eqnarray}
b_{21}^{3} = b_{23} + b_{34} + b_{41},               \label{27} \\
b_{23}^{10}= b_{21} + b_{14} + b_{43}.               \label{28}
\end{eqnarray}

Lemma \ref{lemma:average} is now conf{\kern0pt}irmed for $i=2$:
\begin{equation*}
\sum\limits_{s=1}^{11}
\left(\sum\limits_{k: e(2,k) \in E(T^s)} b_{2k} + \sum\limits_{k: e(2,k) \in E(G) \backslash E(T^s)} b^s_{2k} \right) =
11 \sum\limits_{k: e(2,k) \in E(G)} b_{2k} = 11 (b_{21}+b_{23} ).
\end{equation*}

Cases related to the remaining nodes can be treated likewise.
\end{example}

\section{Conclusions}
It was shown in this paper that two weighting methods,
based on rather dif{\kern0pt}ferent principles and approaches,
 are equivalent not only for complete pairwise comparison matrices,
as it was recently proved by Lundy, Siraj and Greco \cite{LundySirajGreco2017},
but also for incomplete ones.
The geometric mean of weight vectors calculated from all spanning trees was proved to be
logarithmic least squares optimal. The advantages rooted in the def{\kern0pt}inition of the two methods,
namely the clear interpretation of taking all spanning trees into account and
the optimality by a widely analyzed objective function (LLS), are united.


There is a signif{\kern0pt}icant dif{\kern0pt}ference in computational complexity.
The logarithmic least squares problem can be solved
from a single system of linear equations (the coef{\kern0pt}f{\kern0pt}icient matrix is the Laplacian),
requiring at most $O(n^3)$ steps.
The spanning tree approach requires $\max\{O(nS), O(n+m+nS) \}$ steps,
where $S$, the number of spanning trees is between 1 and $n^{n-2}$.
As soon as $S$ exceeds $O(n^2)$, the logarithmic least squares problem is faster to solve.


An important consequence of the paper is that future analyses of weighting methods should not
distinguish between the incomplete LLS and the geometric mean of weight vectors from all spanning trees.

Certain applications apply the spanning trees enumeration, but not necessarily
together with the aggregation by the geometric mean.
The approach of spanning trees enumeration is used in determining the consistency
to build the distribution of expert estimates based on the matrix \cite{OlenkoTsyganok2016}.
Such problems of{\kern0pt}fer further research possibilities.

The possible equivalence of the arithmetic (or other but not geometric) mean of weight vectors,
calculated from all spanning trees and other weighting methods, is still an open problem.

\section*{Acknowledgements}
S.~Boz\'oki acknowledges the support of the J\'anos Bolyai Research Fellowship
 of the Hungarian Academy of Sciences (no.~BO/00154/16/3)
and the Hungarian Scientif{\kern0pt}ic Research Fund (OTKA), grant no.~K111797.
 Orsolya Csisz�r is greatly acknowledged for her careful proofreading.

\begin{thebibliography}{99}

\bibitem{BozokiCsatoTemesi2016}
Boz\'oki, S., Csat\'o, L., Temesi, J. (2016):
An application of incomplete pairwise comparison matrices for ranking top tennis players,
European Journal of Operational Research,
248(1) 211--218
% DOI 10.1016/j.ejor.2015.06.069
% http://www.sciencedirect.com/science/article/pii/S0377221715006220

\bibitem{BozokiFulopRonyai2010}
Boz\'oki, S., F\"ul\"op, J., R\'onyai, L. (2010):
On optimal completion of incomplete pairwise comparison matrices,
Mathematical and Computer Modelling, 52(1-2) 318--333
% DOI 10.1016/j.mcm.2010.02.047
% http://www.sciencedirect.com/science/article/pii/S0895717710001159

\bibitem{ChooWedley2004}
Choo, E.U., Wedley, W.C. (2004):
A common framework for deriving preference values from pairwise comparison matrices,
Computers \& Operations Research,
31(6) 893--908
% DOI 10.1016/S0305-0548(03)00042-X
% http://www.sciencedirect.com/science/article/pii/S030505480300042X

\bibitem{ChuKalabaSpingarn1979}
Chu, A.T.W., Kalaba, R.E., Spingarn, K. (1979):
A comparison of two methods for determining the weights of belonging to fuzzy sets,
Journal of Optimization Theory and Applications,
27(4) 531--538
% DOI 10.1007/BF00933438
% http://link.springer.com/article/10.1007/BF00933438

\bibitem{CrawfordWilliams1985}
Crawford, G., Williams, C. (1985):
A note on the analysis of subjective judgment matrices,
Journal of Mathematical Psychology
29(4) 387--405
% DOI 10.1016/0022-2496(85)90002-1
% http://www.sciencedirect.com/science/article/pii/0022249685900021

\bibitem{Csato2013}
Csat\'o, L. (2013):
Ranking by pairwise comparisons for Swiss-system tournaments,
Central European Journal of Operations Research
21(4) 783--803
% DOI 10.1007/s10100-012-0261-8
% http://link.springer.com/article/10.1007/s10100-012-0261-8

\bibitem{Fichtner1984}
Fichtner, J. (1984).
Some thoughts about the Mathematics of the Analytic Hierarchy Process.
Report 8403,
Universit\"at der Bundeswehr M\"unchen,
Fakult\"at f\"ur Informatik,
Institut f\"ur Angewandte Systemforschung und Operations Research,
Werner-Heisenberg-Weg 39, D-8014 Neubiberg, F.R.G.
1984.

\bibitem{Fichtner1986}
Fichtner, J. (1986).
On deriving priority vectors from matrices of pairwise comparisons.
Socio-Economic Planning Sciences,
20(6) 341--345
% DOI 10.1016/0038-0121(86)90045-5
% http://www.sciencedirect.com/science/article/pii/0038012186900455


\bibitem{GabowMyers1978}
% Harold N. Gabow and Eugene W. Myers
Gabow, H.N., Myers, E.W. (1978):
Finding all spanning trees of directed and undirected graphs,
SIAM Journal on Computing, 7(3) 280--287
% DOI 10.1137/0207024
% http://dx.doi.org/10.1137/0207024
%  http://epubs.siam.org/doi/abs/10.1137/0207024


\bibitem{GolanyKress1993}
Golany, B., Kress, M. (1993):
A multicriteria evaluation of methods for obtaining weights from ratio-scale matrices,
European Journal of Operational Research,
69(2) 210--220
% DOI 10.1016/0377-2217(93)90165-J
% http://www.sciencedirect.com/science/article/pii/037722179390165J

\bibitem{Harker1987b}
Harker, P.T. (1987):
Incomplete pairwise comparisons in the analytic hierarchy process,
Mathematical Modelling
{9}(11) 837--848
% DOI 10.1016/0270-0255(87)90503-3
% http://www.sciencedirect.com/science/article/pii/0270025587905033

\bibitem{deJong1984}
de Jong, P. (1984):
A statistical approach to Saaty's scaling methods for priorities,
Journal of Mathematical Psychology
28(4) 467--478
% DOI 10.1016/0022-2496(84)90013-0
% http://www.sciencedirect.com/science/article/pii/0022249684900130

\bibitem{Kwiesielewicz1996}
Kwiesielewicz, M. (1996):
The logarithmic least squares and the generalised pseudoinverse in estimating ratios,
European Journal of Operational Research
93(3) 611--619
% DOI 10.1016/0377-2217(95)00079-8
% http://www.sciencedirect.com/science/article/pii/0377221795000798

\bibitem{LundySirajGreco2017}
Lundy, M., Siraj, S., Greco, S. (2017):
The mathematical equivalence of the ``spanning tree''
and row geometric mean preference vectors and its implications for preference analysis,
European Journal of Operational Research 257(1) 197--208
% DOI 10.1016/j.ejor.2016.07.042
% http://www.sciencedirect.com/science/article/pii/S0377221716305975

\bibitem{OlenkoTsyganok2016}
Olenko, A., Tsyganok, V. (2016):
Double Entropy Inter-Rater Agreement Indices,
Applied Psychological Measurement
40(1) 37--55
% doi: 10.1177/0146621615592718.
% http://journals.sagepub.com/doi/abs/10.1177/0146621615592718

\bibitem{Saaty1977}
Saaty, T.L. (1977):
A scaling method for priorities in hierarchical structures,
Journal of Mathematical Psychology
15(3) 234--281
% doi:10.1016/0022-2496(77)90033-5
% http://www.sciencedirect.com/science/article/pii/0022249677900335

\bibitem{SirajMikhailovKeane2012a}
Siraj, S., Mikhailov, L., Keane, J.A. (2012):
Enumerating all spanning trees for pairwise comparisons,
Computers \& Operations Research,
39(2) 191--199
% doi:10.1016/j.cor.2011.03.010
% http://www.sciencedirect.com/science/article/pii/S0305054811000839

\bibitem{SirajMikhailovKeane2012b}
Siraj, S., Mikhailov, L., Keane, J.A. (2012):
Corrigendum to ``Enumerating all spanning trees for pairwise comparisons [Comput.~Oper.~Res.~39(2012) 191-199]'',
Computers \& Operations Research,
39(9) page 2265
% doi:10.1016/j.cor.2011.11.010
% http://www.sciencedirect.com/science/article/pii/S0305054811003352

\bibitem{Tsyganok2000}
Tsyganok, V. (2000):
Combinatorial method of pairwise comparisons with feedback,
Data Recording, Storage \& Processing
2:92--102 (in Ukrainian).


\bibitem{Tsyganok2010}
Tsyganok, V. (2010):
Investigation of the aggregation ef{\kern0pt}fectiveness of expert estimates obtained by the pairwise comparison method,
Mathematical and Computer Modelling,
52(3-4) 538--54
% doi: 10.1016/j.mcm.2010.03.052
% http://www.sciencedirect.com/science/article/pii/S0895717710001706


\bibitem{TsyganokKadenkoAndriichuk2015}
Tsyganok, V.V., Kadenko, S.V., Andriichuk, O.V. (2015):
Using dif{\kern0pt}ferent pair-wise comparison scales for developing industrial strategies,
International Journal of Management and Decision Making, 14(3) 224--250
% DOI 10.1504/IJMDM.2015.070760
% http://www.inderscienceonline.com/doi/abs/10.1504/IJMDM.2015.070760

\bibitem{Watkins2002}
Watkins, D.S. (2002):
Fundamentals of Matrix Computations,
Second Edition. John Wiley \& Sons, Inc., New York
% ISBN 0-471-21394-2


\end{thebibliography}
\end{document}
