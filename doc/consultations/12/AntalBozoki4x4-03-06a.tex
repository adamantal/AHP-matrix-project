\documentclass{article}
\usepackage[latin1]{inputenc}
\usepackage{amssymb, amsmath, amsthm}
\usepackage[a4paper]{geometry}

%\usepackage{authblk} % for headings
%\usepackage{pifont}
%\usepackage{graphicx}
%\usepackage{xtab} % tackle the long tables
%\usepackage{longtable} % tackle the long tables
\usepackage{footnote}
\makesavenoteenv{tabular}
\usepackage{tabularx}
%\usepackage{tabu}
\usepackage{rotating}

\def\url#1{\expandafter\string\csname #1\endcsname}

\usepackage{enumitem}
\usepackage{color}
\usepackage{amsmath}% http://ctan.org/pkg/amsmath
\newcommand{\abs}[1]{\lvert#1\rvert}
\newcommand{\Abs}[1]{\left\lvert#1\right\rvert}

% \highlight[<colour>]{<stuff>}
\newcommand{\highlight}[2][yellow]{\mathchoice%
  {\colorbox{#1}{$\displaystyle#2$}}%
  {\colorbox{#1}{$\textstyle#2$}}%
  {\colorbox{#1}{$\scriptstyle#2$}}%
  {\colorbox{#1}{$\scriptscriptstyle#2$}}}%
%-----------------------------------------------------------

\theoremstyle{plain}
\newtheorem{conjecture}{Conjecture}
\newtheorem{corollary}{Corollary}
\newtheorem{definition}{Definition}[section]
\newtheorem{example}{Example}[section]
\newtheorem*{example*}{Example}
\newtheorem{lemma}{Lemma}[section]
\newtheorem{problem}{Problem}
\newtheorem{openproblem}{Open problem}
\newtheorem{proposition}{Proposition}[section]
\newtheorem{remark}{Remark}
\newtheorem{theorem}{Theorem}[section]
\newcommand{\aproof}{\hfill{\ding{111}}}

\def\keywords{\vspace{.5em} % Add keywords
{\textit{Keywords}:\,\relax%
}}
\def\endkeywords{\par}

\usepackage{color}
\definecolor{gr}{rgb}{0,0.5,0}
\usepackage{hyperref}
%\usepackage{url}



\begin{document}

\begin{center}
\color{red}
\textbf{Please print carefully. The paper on the f{\kern0pt}irst 8 pages is followed by an extremely long appendix.} \\[2cm]
\color{black}
\end{center}



\begin{center}
\Large{Ef{\kern0pt}f{\kern0pt}iciency test of priority vectors derived from $4\times4$ pairwise comparison matrices} \\
\end{center}

\begin{center}
\'Ad\'am ANTAL$^{\,\,1}$,
\footnotetext[1]{E�tv�s Lor�nd University (ELTE), Budapest, Hungary \textit{E-mail: antaladam@yahoo.com}}
S\'andor BOZ\'OKI$^{\,\,2,3,4}$
\footnotetext[2]{Laboratory on Engineering and Management
Intelligence, Research Group of Operations Research and Decision
Systems, Institute for Computer Science and Control, Hungarian
Academy of Sciences (MTA SZTAKI); Mail: 1518 Budapest, P.O.~Box 63, Hungary.}
\footnotetext[3]{Department of Operations Research and Actuarial Sciences, Corvinus University
of Budapest, Hungary
 \textit{E-mail: bozoki.sandor@sztaki.mta.hu}}
 \footnotetext[4]{corresponding author}

\end{center}
\begin{abstract}
Three weighting methods, the eigenvector, the arithmetic mean of all spanning trees' weight vectors (AMAST) and the
cosine maximization have been investigated in case of $4 \times 4$ pairwise comparison matrices,
with elements chosen from the usual ratio scale $1,2,\ldots,9,1/2,$ $1/3,\ldots,1/9$.
Out of the $32\,157$ permutation f{\kern0pt}iltered matrices fulf{\kern0pt}illing the rule of acceptable inconsistency ($CR \leq 0.1$),
591 (1.84\%); 197 (0.61\%) and 602 (1.87\%) have inef{\kern0pt}f{\kern0pt}icient eigenvector, AMAST and cosine maximizing weight vector,
respectively. All these examples are listed in the appendix.

\keywords{multiple criteria analysis, decision support, pairwise comparison matrix, Pareto
optimality, ef{\kern0pt}f{\kern0pt}iciency, eigenvector, spanning trees, cosine similarity}
\end{abstract}
\section{Introduction} \label{section:1} % \ref{section:1}
\label{intro}


\subsection{Ef{\kern0pt}f{\kern0pt}iciency of weight vectors derived from pairwise comparison matrices}

Preference modelling, especially the quantif{\kern0pt}ication of decision maker's preferences
is fundamental in decision theory and decision support. We focus on decision models based
on cardinal information originated from comparisons of two objects at a time.
A positive and reciprocal ($a_{ij}=1/a_{ji}$ for all $i,j$) matrix
is called a pairwise comparison matrix \cite{Saaty1977}. The matrix element $a_{ij}$ ref{\kern0pt}lects the decision maker's
preference on a ratio scale when item (typically the importance of a criterion, or the performance of action) $i$ is
compared to item $j$.


Let $\mathbf{A}$ be a pairwise comparison matrix of size $n \times n$
and $\mathbf{w}, \mathbf{w^{\prime}} \in \mathbb{R}^n$ be positive weight vectors.


\begin{definition} \label{def:DefinitionEfficient}  % \ref{def:DefinitionEfficient}
Weight vector $\mathbf{w^{\prime}} = (w^{\prime}_1, w^{\prime}_2, \ldots, w^{\prime}_n)^{\top}$
\emph{dominates} weight vector $\mathbf{w}$ if
\begin{align}
 \left|a_{ij} - \frac{w^{\prime}_i}{w^{\prime}_j} \right| &\leq \left|a_{ij} - \frac{w_i}{w_j} \right| \qquad \text{ for all } 1 \leq i,j \leq n, \label{def:DefinitionEfficientProperty1}  \\   %(\ref{def:DefinitionEfficientProperty1})
 \left|a_{k{\ell}} - \frac{w^{\prime}_k}{w^{\prime}_{\ell}} \right| &<  \left|a_{k{\ell}} - \frac{w_k}{w_{\ell}} \right|  \qquad \text{ for some } 1 \leq k,\ell \leq n.  \label{def:DefinitionEfficientProperty2}    %(\ref{def:DefinitionEfficientProperty2})
\end{align}
\end{definition}

If weight vector $\mathbf{w}$ cannot be dominated, then it is called \emph{ef{\kern0pt}f{\kern0pt}icient},
otherwise it is called \emph{inef{\kern0pt}f{\kern0pt}icient}.

\begin{definition} \label{def:DefinitionInternallyEfficient} %(\ref{def:DefinitionInternallyEfficient})
Weight vector $\mathbf{w^{\prime}} = (w^{\prime}_1, w^{\prime}_2, \ldots, w^{\prime}_n)^{\top}$
\emph{internally} dominates weight vector $\mathbf{w}$ if
\begin{align}
\left.
\begin{array}{ccc}
a_{ij} \leq \frac{w_i}{w_j} & \Longrightarrow &
a_{ij} \leq \frac{w^{\prime}_i}{w^{\prime}_j} \leq \frac{w_i}{w_j} \\
a_{ij} \geq \frac{w_i}{w_j} & \Longrightarrow &
a_{ij} \geq \frac{w^{\prime}_i}{w^{\prime}_j} \geq \frac{w_i}{w_j}
\end{array}
\right\}
\qquad
\text{ for all } 1 \leq i,j \leq n, \label{def:DefinitionInternallyEfficientProperty1}  \\   %(\ref{def:DefinitionInternallyEfficientProperty1})
\left.
\begin{array}{ccc}
a_{k{\ell}} \leq \frac{w_k}{w_{\ell}} & \Longrightarrow &
\frac{w^{\prime}_k}{w^{\prime}_{\ell}} < \frac{w_k}{w_{\ell}} \\
a_{k{\ell}} \geq \frac{w_k}{w_{\ell}} & \Longrightarrow &
\frac{w^{\prime}_k}{w^{\prime}_{\ell}} > \frac{w_k}{w_{\ell}}
\end{array}
\right\}
\qquad
\text{ for some } 1 \leq k,\ell \leq n.  \label{def:DefinitionInternallyEfficientProperty2}    %(\ref{def:DefinitionInternallyEfficientProperty2})
\end{align}
\end{definition}
If weight vector $\mathbf{w}$ cannot be internally dominated, then it is called
\emph{internally ef{\kern0pt}f{\kern0pt}icient}, otherwise it is called \emph{internally inef{\kern0pt}f{\kern0pt}icient}.

Ef{\kern0pt}f{\kern0pt}iciency implies internal ef{\kern0pt}f{\kern0pt}iciency by def{\kern0pt}inition, but they are in fact equivalent
\cite[Corollary 1]{BozokiFulop2018}.
Def{\kern0pt}initions also imply that ef{\kern0pt}f{\kern0pt}iciency is scale invariant:
weight vector $\mathbf{w}$ is ef{\kern0pt}f{\kern0pt}icient if and only if $c\mathbf{w}$ is ef{\kern0pt}f{\kern0pt}icient, where $c > 0$
is arbitrary. Ratios $\frac{w_i}{w_j}$ and $\frac{cw_i}{cw_j}$ are clearly equal.

A dominating weight vector $\mathbf{w}^{\prime}$, if it exists,  can be found by solving
the following linear program, developed by Boz\'oki and F\"ul\"op \cite[formulas (26)-(31)]{BozokiFulop2018}:

\begin{align}
\min  \sum\limits_{(i,j): a_{ij} < \frac{w_i}{w_j} } - s_{ij}
     &  &  \label{LP1}  \\
y_j - y_i  & \leq - \log a_{ij}    &\text{ for all $(i,j)$ such that $ a_{ij} < \frac{w_i}{w_j}$ ,} \label{LP2}  \\
y_i - y_j + s_{ij} & \leq  \log w_i - \log w_j  &\text{ for all $(i,j)$ such that $ a_{ij} < \frac{w_i}{w_j}$ ,} \label{LP3} \\
y_i - y_j  & = \log a_{ij}    &\text{ for all $(i,j)$ such that $ a_{ij} = \frac{w_i}{w_j}$ ,} \label{LP4} \\
s_{ij}  & \geq 0  &\text{ for all $(i,j)$ such that $ a_{ij} < \frac{w_i}{w_j}$ ,} \label{LP5} \\
                   y_1 & = 0    & \label{LP6}
\end{align}
Variables are $y_i, \,  1 \leq i \leq n$ and $s_{ij}$ for all $(i,j)$ such that $ a_{ij} < \frac{w_i}{w_j}$.

According to Theorem 4.1 in \cite{BozokiFulop2018},
the optimum value of the linear program (\ref{LP1})-(\ref{LP6}) is 0
if and only if weight vector $\mathbf{w}$ is ef{\kern0pt}f{\kern0pt}icient.
Let $(\mathbf{y}^{\ast},\mathbf{s}^{\ast})$ denote the optimal solution to (\ref{LP1})-(\ref{LP6}).
If weight vector $\mathbf{w}$ is inef{\kern0pt}f{\kern0pt}icient, then
weight vector $\mathbf{w}^{\prime}=\exp({\mathbf{y}}^{\ast})$
is ef{\kern0pt}f{\kern0pt}icient and dominates $\mathbf{w}$ internally.

\subsection{Weighting methods}

Several weighting methods have been proposed to f{\kern0pt}ind a weight vector $\mathbf{w}$
from a pairwise comparison matrix $\mathbf{A}$. An inevitably oversimplif{\kern0pt}ied conclusion of the comprehensive studies of
Golany and Kress \cite{GolanyKress1993},
Choo and Wedley \cite{ChooWedley2004},
Lin \cite{Lin2007},
Bajwa, Choo and Wedley \cite{BajwaChooWedley2008},
Fedrizzi and Brunelli \cite{FedrizziBrunelli2010},
is that there is no universal weighting method that outperforms the others.

Several distance minimizing methods, such as the least squares method,
where the metric is strictly monotonic \cite{Fedrizzi2013}, always result in ef{\kern0pt}f{\kern0pt}icient weight vectors.
Interestingly, not all distance minimizing methods generate ef{\kern0pt}f{\kern0pt}icient weights. The most remarkable
example is the eigenvector method, deriving weight from the principal right eigenvector of $\mathbf{A}$,
which minimizes a special (neither continuous, nor strictly monotonic)
 metric found by Fichtner \cite{Fichtner1984,Fichtner1986}.
Blanquero, Carrizosa and Conde \cite{BlanqueroCarrizosaConde2006} observed f{\kern0pt}irst that the eigenvector is not
always ef{\kern0pt}f{\kern0pt}icient. However, a necessary and suf{\kern0pt}f{\kern0pt}icient condition of the ef{\kern0pt}f{\kern0pt}iciency of the eigenvector has not been
found.

We show that another two weighting methods, namely the arithmetic mean of weight vectors
calculated from all spanning trees and the weight vector calculated by the cosine maximization method
can also be inef{\kern0pt}f{\kern0pt}icient. The spanning tree approach was proposed by Tsyganok \cite{Tsyganok2000,Tsyganok2010}.
It takes into account all the connected subsets of cardinality $n-1$ of the set of all comparisons, i.e., all the
$n^{n-2}$ spanning trees if the matrix elements are represented by edges in a graph on $n$ vertices.
Every spanning tree determines a unique weight vector, and two natural ways of their aggregation are the
arithmetic mean \cite{SirajMikhailovKeane2012a,SirajMikhailovKeane2012b,Tsyganok2000,Tsyganok2010}, denoted here by
AMAST (arithmetic mean of all spanning tree weight vectors), and the geometric mean
\cite{LundySirajGreco2017,BozokiTsyganok2018}. However, the geometric mean of all spanning tree weight vectors
has recently been proved to be equivalent to the
Logarithmic Least Squares Method \cite{CrawfordWilliams1980,CrawfordWilliams1985,deGraan1980,Rabinowitz1976}, which always
provides an ef{\kern0pt}f{\kern0pt}icient weight vector \cite[Corollary 7]{BlanqueroCarrizosaConde2006}.
The eigenvector and the logarithmic least squares method are equivalent for $3 \times 3$ pairwise
comparison matrices \cite{CrawfordWilliams1980,CrawfordWilliams1985}. Consequently, the smallest
matrix size, when the eigenvector can be inef{\kern0pt}f{\kern0pt}icient, is $4 \times 4.$

The cosine maximization method \cite{KouLin2014} is based on a geometric intuition of
that vectors can be considered similar to each other if their angle is small (cosine of the angle is high),
or, equivalently, their dot product is high. The simply computable and unique weight
vector \cite[Theorem 2]{KouLin2014}, denoted
by $\mathbf{w}^{\cos}$, maximizes the sum of cosine similarity measure for all column vectors of the pairwise
comparison matrix. Despite the geometric intuition behind, weight vector $\mathbf{w}^{\cos}$ can be inef{\kern0pt}f{\kern0pt}icient
as it is presented in the next section.



\section{Results} \label{section:2} % \ref{section:2}


There are $1\,007\,097$ pairwise comparison matrices of size $4 \times 4$ such that
all elements are from the ratio scale $1,2,\ldots,9,1/2,1/3,\ldots,1/9$,
and no pair of matrices can be transformed into each other by row/column permutations
(without permutation f{\kern0pt}iltering there would be $17^6 = 24\,137\,569$ matrices).

\subsection{Matrices with acceptable inconsistency ($CR \leq 0.1$)}
The $CR$ inconsistency is below $0.1$ \cite{Saaty1977} for $32\,157$ out of all
permutation f{\kern0pt}iltered matrices, resulting in a ratio 3.19\%.
It is similar to the frequency 3.15\% experienced by Boz\'oki and Rapcs\'ak \cite[Table 4]{BozokiRapcsak2008},
although their matrices were generated randomly.

Out of the $32\,157$ permutation f{\kern0pt}iltered matrices fulf{\kern0pt}illing $CR \leq 0.1$,
\begin{itemize}
\item 591 (1.84\%) have inef{\kern0pt}f{\kern0pt}icient eigenvector;
\item 197 (0.61\%) have inef{\kern0pt}f{\kern0pt}icient weight vector calculated by the spanning trees' arithmetic mean;
\item 602 (1.87\%) have inef{\kern0pt}f{\kern0pt}icient weight vector calculated by the cosine maximization method.
\end{itemize}

They are listed in the Appendix (Examples A.1--591, B.1--197, C.1--602, respectively),
an illustrative example is given below.

\begin{example*} (Example A.73 in the Appendix)
\begin{equation*}
\mathbf{A} =
\begin{pmatrix}
$\,\,$ 1 $\,\,$ & $\,\,$2$\,\,$ & $\,\,$6$\,\,$ & $\,\,$7 $\,\,$ \\
$\,\,$ 1/2$\,\,$ & $\,\,$ 1 $\,\,$ & $\,\,$5$\,\,$ & $\,\,$2 $\,\,$ \\
$\,\,$ 1/6$\,\,$ & $\,\,$ 1/5$\,\,$ & $\,\,$ 1 $\,\,$ & $\,\,$2 $\,\,$ \\
$\,\,$ 1/7$\,\,$ & $\,\,$ 1/2$\,\,$ & $\,\,$ 1/2$\,\,$ & $\,\,$ 1  $\,\,$ \\
\end{pmatrix},
\qquad
\lambda_{\max} =
4.2251,
\qquad
CR = 0.0849
\end{equation*}

\begin{equation*}
\mathbf{w}^{EM} =
\begin{pmatrix}
\color{red} 0.536063\color{black} \\
0.284402\\
0.096706\\
0.082830
\end{pmatrix}\end{equation*}
\begin{equation*}
\left[ \frac{{w}^{EM}_i}{{w}^{EM}_j} \right] =
\begin{pmatrix}
$\,\,$ 1 $\,\,$ & $\,\,$\color{red} 1.8849\color{black} $\,\,$ & $\,\,$\color{red} 5.5432\color{black} $\,\,$ & $\,\,$\color{red} 6.4718\color{black} $\,\,$ \\
$\,\,$\color{red} 0.5305\color{black} $\,\,$ & $\,\,$ 1 $\,\,$ & $\,\,$2.9409$\,\,$ & $\,\,$3.4336  $\,\,$ \\
$\,\,$\color{red} 0.1804\color{black} $\,\,$ & $\,\,$0.3400$\,\,$ & $\,\,$ 1 $\,\,$ & $\,\,$1.1675 $\,\,$ \\
$\,\,$\color{red} 0.1545\color{black} $\,\,$ & $\,\,$0.2912$\,\,$ & $\,\,$0.8565$\,\,$ & $\,\,$ 1  $\,\,$ \\
\end{pmatrix},
\end{equation*}

It can be seen that all non-diagonal elements of the f{\kern0pt}irst row/column are under/overestimated,
indicating that the eigenvector is inef{\kern0pt}f{\kern0pt}icient.
Indeed, the ef{\kern0pt}f{\kern0pt}icient and internally
dominating weight vector $\mathbf{w}^{\prime}$, found by (\ref{LP1})-(\ref{LP6}) is as follows:
\[
\mathbf{w}^{\prime} =
\begin{pmatrix}
0.550771\\
0.275385\\
0.093640\\
0.080204
\end{pmatrix}.
\]
Its relation to the eigenvector $\mathbf{w}^{EM}$ can be made more visible by an appropriate
rescaling
\begin{equation*}
\mathbf{w}^{\prime} =
0.968297\cdot
\begin{pmatrix}
\color{gr} 0.568803\color{black} \\
0.284402\\
0.096706\\
0.082830
\end{pmatrix},
\end{equation*}
which shows that the last three coordinates have not changed,
the f{\kern0pt}irst coordinate has been increased as we expected.
Furthermore, the f{\kern0pt}irst coordinate has been increased such that
$a_{12} = \frac{w^{\prime}_1}{w^{\prime}_2} = 2$, i.e., not only
$\left| \frac{w^{\prime}_1}{w^{\prime}_2} - a_{12} \right| <
 \left| \frac{w_1}{w_2} - a_{12} \right| $ holds but the left hand side of the inequality equals to zero.

\begin{equation*}
\left[ \frac{{w}^{\prime}_i}{{w}^{\prime}_j} \right] =
\begin{pmatrix}
$\,\,$ 1 $\,\,$ & $\,\,$\color{gr} \color{blue} 2\color{black} $\,\,$ & $\,\,$\color{gr} 5.8818\color{black} $\,\,$ & $\,\,$\color{gr} 6.8671\color{black} $\,\,$ \\
$\,\,$\color{gr} \color{blue}  1/2\color{black} $\,\,$ & $\,\,$ 1 $\,\,$ & $\,\,$2.9409$\,\,$ & $\,\,$3.4336  $\,\,$ \\
$\,\,$\color{gr} 0.1700\color{black} $\,\,$ & $\,\,$0.3400$\,\,$ & $\,\,$ 1 $\,\,$ & $\,\,$1.1675 $\,\,$ \\
$\,\,$\color{gr} 0.1456\color{black} $\,\,$ & $\,\,$0.2912$\,\,$ & $\,\,$0.8565$\,\,$ & $\,\,$ 1  $\,\,$ \\
\end{pmatrix},
\end{equation*}
\end{example*}

We would also like to draw the attention to a special example.
\begin{example}  (Example A.116 in the Appendix) % Example 1.116
\begin{equation*}
\mathbf{A} =
\begin{pmatrix}
$\,\,$ 1 $\,\,$ & $\,\,$2$\,\,$ & $\,\,$9$\,\,$ & $\,\,$8 $\,\,$ \\
$\,\,$ 1/2$\,\,$ & $\,\,$ 1 $\,\,$ & $\,\,$3$\,\,$ & $\,\,$6 $\,\,$ \\
$\,\,$ 1/9$\,\,$ & $\,\,$ 1/3$\,\,$ & $\,\,$ 1 $\,\,$ & $\,\,$3 $\,\,$ \\
$\,\,$ 1/8$\,\,$ & $\,\,$ 1/6$\,\,$ & $\,\,$ 1/3$\,\,$ & $\,\,$ 1  $\,\,$ \\
\end{pmatrix},
\qquad
\lambda_{\max} =
4.1263,
\qquad
CR = 0.0476
\end{equation*}

\begin{equation*}
\mathbf{w}^{EM} =
\begin{pmatrix}
0.578100\\
\color{red} 0.277375\color{black} \\
0.096350\\
0.048175
\end{pmatrix}\end{equation*}
\begin{equation*}
\left[ \frac{{w}^{EM}_i}{{w}^{EM}_j} \right] =
\begin{pmatrix}
$\,\,$ 1 $\,\,$ & $\,\,$\color{red} 2.0842\color{black} $\,\,$ & $\,\,$6$\,\,$ & $\,\,$12$\,\,$ \\
$\,\,$\color{red} 0.4798\color{black} $\,\,$ & $\,\,$ 1 $\,\,$ & $\,\,$\color{red} 2.8788\color{black} $\,\,$ & $\,\,$\color{red} 5.7577\color{black}   $\,\,$ \\
$\,\,$1/6$\,\,$ & $\,\,$\color{red} 0.3474\color{black} $\,\,$ & $\,\,$ 1 $\,\,$ & $\,\,$2 $\,\,$ \\
$\,\,$1/12$\,\,$ & $\,\,$\color{red} 0.1737\color{black} $\,\,$ & $\,\,$1/2$\,\,$ & $\,\,$ 1  $\,\,$ \\
\end{pmatrix},
\end{equation*}

\begin{equation*}
\mathbf{w}^{\prime} =
\begin{pmatrix}
0.571429\\
0.285714\\
0.095238\\
0.047619
\end{pmatrix} =
0.988460\cdot
\begin{pmatrix}
0.578100\\
\color{gr} 0.289050\color{black} \\
0.096350\\
0.048175
\end{pmatrix},
\end{equation*}
\begin{equation*}
\left[ \frac{{w}^{\prime}_i}{{w}^{\prime}_j} \right] =
\begin{pmatrix}
$\,\,$ 1 $\,\,$ & $\,\,$\color{blue} 2\color{black} $\,\,$ & $\,\,$6$\,\,$ & $\,\,$12$\,\,$ \\
$\,\,$\color{blue} 1/2\color{black} $\,\,$ & $\,\,$ 1 $\,\,$ & $\,\,$\color{blue} 3\color{black} $\,\,$ & $\,\,$\color{gr} \color{blue} 6\color{black}   $\,\,$ \\
$\,\,$1/6$\,\,$ & $\,\,$\color{blue} 1/3\color{black} $\,\,$ & $\,\,$ 1 $\,\,$ & $\,\,$2 $\,\,$ \\
$\,\,$1/12$\,\,$ & $\,\,$\color{gr} \color{blue}  1/6\color{black} $\,\,$ & $\,\,$1/2$\,\,$ & $\,\,$ 1  $\,\,$ \\
\end{pmatrix},
\end{equation*}
The ratios calculated from the dominating weight vector give
not only an improved, but a perfect approximation of the original
pairwise comparison matrix in all entries of a row/column.
Examples A.116, A.139, A.162., A.200, A.374 and A.553 have this particular property.
\end{example}


\subsection{Matrices with arbitrary $CR$ inconsistency}
Since the rule of thumb $CR \leq 0.1$ has been applied in a wide range of decision problems
\cite{SubramanianRamanathan2012,VaidyaKumar2006},
its theoretical foundation is debated.
We have considered all the $1\,007\,097$ permutation f{\kern0pt}iltered
pairwise comparison matrices of size $4 \times 4$ and plotted the
frequencies of inef{\kern0pt}f{\kern0pt}icient eigenvector, arithmetic mean of all spanning tree weight vectors (AMAST),
and the weight vector from cosine maximization, as functions of $CR$ inconsistency.
$CR = 0$ represents the consistent case. The upper bound of $CR$ is 3.645
since pairwise comparison matrix
\[
\begin{pmatrix}
   1      &      9       &     1/9      &         9            \\
  1/9     &      1       &     9        &        1/9           \\
   9      &     1/9      &     1        &         9            \\
  1/9     &      9       &    1/9       &         1
\end{pmatrix}
\]
has the largest $\lambda_{\max} = 13.6\dot{6}$, consequently the largest $CR$ inconsistency
\cite{AupetitGenest1993}.


The number of pairwise comparison matrices as a function of $CR$ (on the left in Figure 1)
is similar to the one in Boz\'oki, Rapcs\'ak \cite[Figure 3b]{BozokiRapcsak2008}, but their matrices were
generated randomly.

The right part of Figure 1 shows that the ratio of inef{\kern0pt}f{\kern0pt}icient EM, AMAST and $\cos$ weight vectors is maximal
(29\%; 28\% and 14\%, respectively) at around $CR = 1.2$.
All the three methods provide ef{\kern0pt}f{\kern0pt}icient priority vectors
for matrices with extremely high inconsistency ($CR > 2.6$).

\unitlength 1mm
\begin{center}
\begin{picture}(150,70)
\put(10,10){\resizebox{140mm}{!}{\rotatebox{0}{\includegraphics{abra3.eps}}}}
\put(45,9){(a)}
\put(110,9){(b)}
\put(20,3){\makebox{\textbf{Figure 1.} (a) distribution of inconsistency index $CR$ among $4 \times 4 $ matrices }}
\put(5,-1){\makebox{(b) frequencies of inef{\kern0pt}f{\kern0pt}icient eigenvectors, AMAST and cosine maximization weight vectors }}
\end{picture}
\end{center}
% \ref{example1}


\section{Conclusions}
We have found that three weighting methods, the eigenvector, the arithmetic mean of all
spanning trees's weight vectors and the cosine maximization provide inef{\kern0pt}f{\kern0pt}icient priority
vectors with a small but not negligible frequency if $CR \leq 0.1$. Hundreds of examples indicate that
it is not only a theoretical phenomenon, but it may have ef{\kern0pt}fect on the ranking itself, too.
Our opinion is that inef{\kern0pt}f{\kern0pt}icient priority vectors are not acceptable in any decision problem.
Consequently, the test of ef{\kern0pt}f{\kern0pt}iciency, and, in case of inef{\kern0pt}f{\kern0pt}iciency, f{\kern0pt}inding an ef{\kern0pt}f{\kern0pt}icient dominating
weight vector is necessary.

\section*{Acknowledgements}
Research was supported in part by the Hungarian
Scientif{\kern0pt}ic Research Fund (OTKA) grant no.~K111797.
S.~Boz\'oki acknowledges the support of the J\'anos Bolyai
Research Fellowship of the Hungarian Academy of Sciences
(no.~BO/00154/16).
LEMON -- Library for Ef{\kern0pt}f{\kern0pt}icient Modeling and Optimization in Networks
C++ optimization library (http://lemon.cs.elte.hu) has been used in part
in the calculations.

\begin{thebibliography}{99}
\bibitem{AupetitGenest1993}
Aupetit, B., Genest, C. (1993)
On some useful properties of the Perron eigenvalue of a
positive reciprocal matrix in the context of the analytic hierarchy process.
European Journal of Operational Research
70(2):263--268
% DOI 10.1016/0377-2217(93)90044-N
% http://www.sciencedirect.com/science/article/pii/037722179390044N

\bibitem{BajwaChooWedley2008}
Bajwa, G., Choo, E.U., Wedley, W.C. (2008)
Ef{\kern0pt}fectiveness analysis of deriving priority vectors from reciprocal pairwise comparison matrices.
Asia-Pacif{\kern0pt}ic Journal of Operational Research,
25(3):279--299.
% DOI 10.1142/S0217595908001754
% http://www.worldscientific.com/doi/pdf/10.1142/S0217595908001754

\bibitem{BlanqueroCarrizosaConde2006}
Blanquero, R., Carrizosa, E., Conde, E. (2006)  Inferring
ef{\kern0pt}f{\kern0pt}icient weights from pairwise comparison
matrices. Mathematical Methods of Operations Research
64(2):271--284
% DOI 10.1007/s00186-006-0077-1
% http://link.springer.com/article/10.1007/s00186-006-0077-1#

\bibitem{BozokiFulop2018}
Boz\'oki, S., F\"ul\"op, J. (2018)
Ef{\kern0pt}f{\kern0pt}icient weight vectors from pairwise comparison matrices.
European Journal of Operational Research
 264(2):419-427
% DOI 10.1016/j.ejor.2017.06.033
% http://www.sciencedirect.com/science/article/pii/S0377221717305726

\bibitem{BozokiRapcsak2008}
Boz\'oki, S., Rapcs\'ak, T. (2008)
On Saaty's and Koczkodaj's inconsistencies of pairwise comparison matrices.
Journal of Global Optimization
42(2):157--175.
% DOI 10.1007/s10898-007-9236-z
% https://link.springer.com/article/10.1007%2Fs10898-007-9236-z

\bibitem{BozokiTsyganok2018}
Boz�ki, S., Tsyganok, V. ($\geq$ 2018)
The logarithmic least squares optimality of the geometric mean of weight vectors
calculated from all spanning trees for (in)complete pairwise comparison matrices.
Under review,
https://arxiv.org/abs/1701.04265
%    https://arxiv.org/abs/1701.04265
% DOI
%

\bibitem{ChooWedley2004}
Choo, E.U., Wedley, W.C. (2004)
A common framework for deriving preference values from pairwise comparison matrices.
Computers \& Operations Research 31(6):893--908
% DOI 10.1016/S0305-0548(03)00042-X
% http://www.sciencedirect.com/science/article/pii/S030505480300042X

\bibitem{CrawfordWilliams1980}
Crawford, G., Williams, C. (1980)
Analysis of subjective judgment matrices. The Rand Corporation, Of{\kern0pt}f{\kern0pt}ice
of the Secretary of Defense USA, R-2572-AF

\bibitem{CrawfordWilliams1985}
Crawford, G., Williams, C. (1985)
A note on the analysis of subjective judgment matrices.
Journal of Mathematical Psychology
29(4):387--405
% DOI 10.1016/0022-2496(85)90002-1
% http://www.sciencedirect.com/science/article/pii/0022249685900021

\bibitem{deGraan1980}
de Graan, J.G. (1980)
Extensions of the multiple criteria analysis method of T.L.~Saaty.
Presented at EURO IV Conference, Cambridge, July 22-25, 1980

\bibitem{Fedrizzi2013}
Fedrizzi, M. (2013)
Obtaining non-dominated weights from preference relations through norm-induced distances.
Proceedings of the
XXXVII Meeting of the Italian Association for Mathematics Applied to Economic and Social Sciences (AMASES),
Stresa, Italy, September 5-7, 2013.
% https://www.eco.uninsubria.it/site/xxxvii-meeting-amases/

\bibitem{FedrizziBrunelli2010}
Fedrizzi, M., Brunelli, M. (2010)
On the priority vector associated with a reciprocal relation and a pairwise comparison matrix.
Soft Computing 14(6):639--645
% http://link.springer.com/article/10.1007%2Fs00500-009-0432-2

\bibitem{Fichtner1984}
Fichtner, J. (1984)
Some thoughts about the Mathematics of the Analytic Hierarchy Process.
Report 8403,
Universit\"at der Bundeswehr M\"unchen,
Fakult\"at f\"ur Informatik,
Institut f\"ur Angewandte Systemforschung und Operations Research,
Werner-Heisenberg-Weg 39, D-8014 Neubiberg, F.R.G.
1984.

\bibitem{Fichtner1986}
Fichtner, J. (1986)
On deriving priority vectors from matrices of pairwise comparisons.
Socio-Economic Planning Sciences
20(6):341--345
% DOI 10.1016/0038-0121(86)90045-5
% http://www.sciencedirect.com/science/article/pii/0038012186900455

\bibitem{GolanyKress1993}
Golany, B., Kress, M. (1993)
A multicriteria evaluation of methods for obtaining weights from ratio-scale matrices.
European Journal of Operational Research
69(2):210--220
% doi:10.1016/0377-2217(93)90165-J
% http://www.sciencedirect.com/science/article/pii/037722179390165J

\bibitem{KouLin2014}
Kou, G., Lin, C. (2014)
A cosine maximization method for the priority vector derivation in AHP.
European Journal of Operational Research
235(1):225--232
% DOI 10.1016/j.ejor.2013.10.019
% http://www.sciencedirect.com/science/article/pii/S0377221713008424

\bibitem{Lin2007}
Lin, C.-C. (2007)
A revised framework for deriving preference values from pairwise comparison matrices.
European Journal of Operational Research
176(2):1145--1150

\bibitem{LundySirajGreco2017}
Lundy, M., Siraj, S., Greco, S. (2017)
The mathematical equivalence of the ``spanning tree''
and row geometric mean preference vectors and its implications for preference analysis.
European Journal of Operational Research 257(1):197--208
% DOI 10.1016/j.ejor.2016.07.042
% http://www.sciencedirect.com/science/article/pii/S0377221716305975

\bibitem{Rabinowitz1976}
Rabinowitz, G. (1976)
Some comments on measuring world inf{\kern0pt}luence.
Journal of Peace Science
2(1):49--55
% DOI 10.1177/073889427600200104
% https://doi.org/10.1177/073889427600200104
% http://journals.sagepub.com/doi/abs/10.1177/073889427600200104?journalCode=cmpa

\bibitem{Saaty1977}
Saaty, T.L. (1977)
A scaling method for priorities in hierarchical structures.
Journal of Mathematical Psychology
15(3):234--281
% doi:10.1016/0022-2496(77)90033-5
% http://www.sciencedirect.com/science/article/pii/0022249677900335

\bibitem{SirajMikhailovKeane2012a}
Siraj, S., Mikhailov, L., Keane, J.A. (2012)
Enumerating all spanning trees for pairwise comparisons.
Computers \& Operations Research
39(2):191--199
% doi:10.1016/j.cor.2011.03.010
% http://www.sciencedirect.com/science/article/pii/S0305054811000839

\bibitem{SirajMikhailovKeane2012b}
Siraj, S., Mikhailov, L., Keane, J.A. (2012)
Corrigendum to ``Enumerating all spanning trees for pairwise comparisons [Comput. Oper. Res. 39 (2012) 191-199]''.
Computers \& Operations Research
39(9) page 2265
% doi:10.1016/j.cor.2011.11.010
% http://www.sciencedirect.com/science/article/pii/S0305054811003352

\bibitem{SubramanianRamanathan2012}
Subramanian, N., Ramanathan, R. (2012)
A review of applications of Analytic Hierarchy Process in operations management.
International Journal of Production Economics
138(2):215--241
% DOI 10.1016/j.ijpe.2012.03.036
% http://www.sciencedirect.com/science/article/pii/S0925527312001442

\bibitem{Tsyganok2000}
Tsyganok, V. (2000)
Combinatorial method of pairwise comparisons with feedback.
Data Recording, Storage \& Processing
2:92--102 (in Ukrainian)

\bibitem{Tsyganok2010}
Tsyganok, V. (2010)
Investigation of the aggregation ef{\kern0pt}fectiveness of expert estimates obtained by the pairwise comparison method.
Mathematical and Computer Modelling
52(3-4):538--544
% doi: 10.1016/j.mcm.2010.03.052
% http://www.sciencedirect.com/science/article/pii/S0895717710001706

\bibitem{VaidyaKumar2006}
Vaidya, O.S., Kumar, S. (2006)
Analytic hierarchy process: An overview of applications.
European Journal of Operational Research
169(1):1--29
% DOI 10.1016/j.ejor.2004.04.028
% http://www.sciencedirect.com/science/article/pii/S0377221704003054

\end{thebibliography}

\newpage
\appendix
\section*{Online appendix}
\setcounter{section}{0}
\section{Inef{\kern0pt}f{\kern0pt}icient eigenvector}
\begin{example}
\begin{equation*}
\mathbf{A} =
\begin{pmatrix}
$\,\,$ 1 $\,\,$ & $\,\,$1$\,\,$ & $\,\,$3$\,\,$ & $\,\,$2 $\,\,$ \\
$\,\,$ 1 $\,\,$ & $\,\,$ 1 $\,\,$ & $\,\,$4$\,\,$ & $\,\,$6 $\,\,$ \\
$\,\,$ 1/3$\,\,$ & $\,\,$ 1/4$\,\,$ & $\,\,$ 1 $\,\,$ & $\,\,$1 $\,\,$ \\
$\,\,$ 1/2$\,\,$ & $\,\,$ 1/6$\,\,$ & $\,\,$ 1 $\,\,$ & $\,\,$ 1  $\,\,$ \\
\end{pmatrix},
\qquad
\lambda_{\max} =
4.1031,
\qquad
CR = 0.0389
\end{equation*}

\begin{equation*}
\mathbf{w}^{EM} =
\begin{pmatrix}
0.323239\\
0.457955\\
\color{red} 0.107489\color{black} \\
0.111318
\end{pmatrix}\end{equation*}
\begin{equation*}
\left[ \frac{{w}^{EM}_i}{{w}^{EM}_j} \right] =
\begin{pmatrix}
$\,\,$ 1 $\,\,$ & $\,\,$0.7058$\,\,$ & $\,\,$\color{red} 3.0072\color{black} $\,\,$ & $\,\,$2.9038$\,\,$ \\
$\,\,$1.4168$\,\,$ & $\,\,$ 1 $\,\,$ & $\,\,$\color{red} 4.2605\color{black} $\,\,$ & $\,\,$4.1140  $\,\,$ \\
$\,\,$\color{red} 0.3325\color{black} $\,\,$ & $\,\,$\color{red} 0.2347\color{black} $\,\,$ & $\,\,$ 1 $\,\,$ & $\,\,$\color{red} 0.9656\color{black}  $\,\,$ \\
$\,\,$0.3444$\,\,$ & $\,\,$0.2431$\,\,$ & $\,\,$\color{red} 1.0356\color{black} $\,\,$ & $\,\,$ 1  $\,\,$ \\
\end{pmatrix},
\end{equation*}

\begin{equation*}
\mathbf{w}^{\prime} =
\begin{pmatrix}
0.323156\\
0.457837\\
0.107719\\
0.111289
\end{pmatrix} =
0.999742\cdot
\begin{pmatrix}
0.323239\\
0.457955\\
\color{gr} 0.107746\color{black} \\
0.111318
\end{pmatrix},
\end{equation*}
\begin{equation*}
\left[ \frac{{w}^{\prime}_i}{{w}^{\prime}_j} \right] =
\begin{pmatrix}
$\,\,$ 1 $\,\,$ & $\,\,$0.7058$\,\,$ & $\,\,$\color{gr} \color{blue} 3\color{black} $\,\,$ & $\,\,$2.9038$\,\,$ \\
$\,\,$1.4168$\,\,$ & $\,\,$ 1 $\,\,$ & $\,\,$\color{gr} 4.2503\color{black} $\,\,$ & $\,\,$4.1140  $\,\,$ \\
$\,\,$\color{gr} \color{blue}  1/3\color{black} $\,\,$ & $\,\,$\color{gr} 0.2353\color{black} $\,\,$ & $\,\,$ 1 $\,\,$ & $\,\,$\color{gr} 0.9679\color{black}  $\,\,$ \\
$\,\,$0.3444$\,\,$ & $\,\,$0.2431$\,\,$ & $\,\,$\color{gr} 1.0331\color{black} $\,\,$ & $\,\,$ 1  $\,\,$ \\
\end{pmatrix},
\end{equation*}
\end{example}
\newpage
\begin{example}
\begin{equation*}
\mathbf{A} =
\begin{pmatrix}
$\,\,$ 1 $\,\,$ & $\,\,$1$\,\,$ & $\,\,$3$\,\,$ & $\,\,$2 $\,\,$ \\
$\,\,$ 1 $\,\,$ & $\,\,$ 1 $\,\,$ & $\,\,$4$\,\,$ & $\,\,$7 $\,\,$ \\
$\,\,$ 1/3$\,\,$ & $\,\,$ 1/4$\,\,$ & $\,\,$ 1 $\,\,$ & $\,\,$1 $\,\,$ \\
$\,\,$ 1/2$\,\,$ & $\,\,$ 1/7$\,\,$ & $\,\,$ 1 $\,\,$ & $\,\,$ 1  $\,\,$ \\
\end{pmatrix},
\qquad
\lambda_{\max} =
4.1365,
\qquad
CR = 0.0515
\end{equation*}

\begin{equation*}
\mathbf{w}^{EM} =
\begin{pmatrix}
0.318086\\
0.471201\\
\color{red} 0.105051\color{black} \\
0.105662
\end{pmatrix}\end{equation*}
\begin{equation*}
\left[ \frac{{w}^{EM}_i}{{w}^{EM}_j} \right] =
\begin{pmatrix}
$\,\,$ 1 $\,\,$ & $\,\,$0.6751$\,\,$ & $\,\,$\color{red} 3.0279\color{black} $\,\,$ & $\,\,$3.0104$\,\,$ \\
$\,\,$1.4814$\,\,$ & $\,\,$ 1 $\,\,$ & $\,\,$\color{red} 4.4855\color{black} $\,\,$ & $\,\,$4.4595  $\,\,$ \\
$\,\,$\color{red} 0.3303\color{black} $\,\,$ & $\,\,$\color{red} 0.2229\color{black} $\,\,$ & $\,\,$ 1 $\,\,$ & $\,\,$\color{red} 0.9942\color{black}  $\,\,$ \\
$\,\,$0.3322$\,\,$ & $\,\,$0.2242$\,\,$ & $\,\,$\color{red} 1.0058\color{black} $\,\,$ & $\,\,$ 1  $\,\,$ \\
\end{pmatrix},
\end{equation*}

\begin{equation*}
\mathbf{w}^{\prime} =
\begin{pmatrix}
0.317892\\
0.470913\\
0.105597\\
0.105597
\end{pmatrix} =
0.999389\cdot
\begin{pmatrix}
0.318086\\
0.471201\\
\color{gr} 0.105662\color{black} \\
0.105662
\end{pmatrix},
\end{equation*}
\begin{equation*}
\left[ \frac{{w}^{\prime}_i}{{w}^{\prime}_j} \right] =
\begin{pmatrix}
$\,\,$ 1 $\,\,$ & $\,\,$0.6751$\,\,$ & $\,\,$\color{gr} 3.0104\color{black} $\,\,$ & $\,\,$3.0104$\,\,$ \\
$\,\,$1.4814$\,\,$ & $\,\,$ 1 $\,\,$ & $\,\,$\color{gr} 4.4595\color{black} $\,\,$ & $\,\,$4.4595  $\,\,$ \\
$\,\,$\color{gr} 0.3322\color{black} $\,\,$ & $\,\,$\color{gr} 0.2242\color{black} $\,\,$ & $\,\,$ 1 $\,\,$ & $\,\,$\color{gr} \color{blue} 1\color{black}  $\,\,$ \\
$\,\,$0.3322$\,\,$ & $\,\,$0.2242$\,\,$ & $\,\,$\color{gr} \color{blue} 1\color{black} $\,\,$ & $\,\,$ 1  $\,\,$ \\
\end{pmatrix},
\end{equation*}
\end{example}
\newpage
\begin{example}
\begin{equation*}
\mathbf{A} =
\begin{pmatrix}
$\,\,$ 1 $\,\,$ & $\,\,$1$\,\,$ & $\,\,$3$\,\,$ & $\,\,$2 $\,\,$ \\
$\,\,$ 1 $\,\,$ & $\,\,$ 1 $\,\,$ & $\,\,$5$\,\,$ & $\,\,$8 $\,\,$ \\
$\,\,$ 1/3$\,\,$ & $\,\,$ 1/5$\,\,$ & $\,\,$ 1 $\,\,$ & $\,\,$1 $\,\,$ \\
$\,\,$ 1/2$\,\,$ & $\,\,$ 1/8$\,\,$ & $\,\,$ 1 $\,\,$ & $\,\,$ 1  $\,\,$ \\
\end{pmatrix},
\qquad
\lambda_{\max} =
4.1655,
\qquad
CR = 0.0624
\end{equation*}

\begin{equation*}
\mathbf{w}^{EM} =
\begin{pmatrix}
0.309383\\
0.496976\\
\color{red} 0.095105\color{black} \\
0.098536
\end{pmatrix}\end{equation*}
\begin{equation*}
\left[ \frac{{w}^{EM}_i}{{w}^{EM}_j} \right] =
\begin{pmatrix}
$\,\,$ 1 $\,\,$ & $\,\,$0.6225$\,\,$ & $\,\,$\color{red} 3.2531\color{black} $\,\,$ & $\,\,$3.1398$\,\,$ \\
$\,\,$1.6063$\,\,$ & $\,\,$ 1 $\,\,$ & $\,\,$\color{red} 5.2255\color{black} $\,\,$ & $\,\,$5.0436  $\,\,$ \\
$\,\,$\color{red} 0.3074\color{black} $\,\,$ & $\,\,$\color{red} 0.1914\color{black} $\,\,$ & $\,\,$ 1 $\,\,$ & $\,\,$\color{red} 0.9652\color{black}  $\,\,$ \\
$\,\,$0.3185$\,\,$ & $\,\,$0.1983$\,\,$ & $\,\,$\color{red} 1.0361\color{black} $\,\,$ & $\,\,$ 1  $\,\,$ \\
\end{pmatrix},
\end{equation*}

\begin{equation*}
\mathbf{w}^{\prime} =
\begin{pmatrix}
0.308325\\
0.495277\\
0.098199\\
0.098199
\end{pmatrix} =
0.996581\cdot
\begin{pmatrix}
0.309383\\
0.496976\\
\color{gr} 0.098536\color{black} \\
0.098536
\end{pmatrix},
\end{equation*}
\begin{equation*}
\left[ \frac{{w}^{\prime}_i}{{w}^{\prime}_j} \right] =
\begin{pmatrix}
$\,\,$ 1 $\,\,$ & $\,\,$0.6225$\,\,$ & $\,\,$\color{gr} 3.1398\color{black} $\,\,$ & $\,\,$3.1398$\,\,$ \\
$\,\,$1.6063$\,\,$ & $\,\,$ 1 $\,\,$ & $\,\,$\color{gr} 5.0436\color{black} $\,\,$ & $\,\,$5.0436  $\,\,$ \\
$\,\,$\color{gr} 0.3185\color{black} $\,\,$ & $\,\,$\color{gr} 0.1983\color{black} $\,\,$ & $\,\,$ 1 $\,\,$ & $\,\,$\color{gr} \color{blue} 1\color{black}  $\,\,$ \\
$\,\,$0.3185$\,\,$ & $\,\,$0.1983$\,\,$ & $\,\,$\color{gr} \color{blue} 1\color{black} $\,\,$ & $\,\,$ 1  $\,\,$ \\
\end{pmatrix},
\end{equation*}
\end{example}
\newpage
\begin{example}
\begin{equation*}
\mathbf{A} =
\begin{pmatrix}
$\,\,$ 1 $\,\,$ & $\,\,$1$\,\,$ & $\,\,$3$\,\,$ & $\,\,$2 $\,\,$ \\
$\,\,$ 1 $\,\,$ & $\,\,$ 1 $\,\,$ & $\,\,$5$\,\,$ & $\,\,$9 $\,\,$ \\
$\,\,$ 1/3$\,\,$ & $\,\,$ 1/5$\,\,$ & $\,\,$ 1 $\,\,$ & $\,\,$1 $\,\,$ \\
$\,\,$ 1/2$\,\,$ & $\,\,$ 1/9$\,\,$ & $\,\,$ 1 $\,\,$ & $\,\,$ 1  $\,\,$ \\
\end{pmatrix},
\qquad
\lambda_{\max} =
4.1966,
\qquad
CR = 0.0741
\end{equation*}

\begin{equation*}
\mathbf{w}^{EM} =
\begin{pmatrix}
0.305188\\
0.507193\\
\color{red} 0.093121\color{black} \\
0.094498
\end{pmatrix}\end{equation*}
\begin{equation*}
\left[ \frac{{w}^{EM}_i}{{w}^{EM}_j} \right] =
\begin{pmatrix}
$\,\,$ 1 $\,\,$ & $\,\,$0.6017$\,\,$ & $\,\,$\color{red} 3.2773\color{black} $\,\,$ & $\,\,$3.2296$\,\,$ \\
$\,\,$1.6619$\,\,$ & $\,\,$ 1 $\,\,$ & $\,\,$\color{red} 5.4466\color{black} $\,\,$ & $\,\,$5.3672  $\,\,$ \\
$\,\,$\color{red} 0.3051\color{black} $\,\,$ & $\,\,$\color{red} 0.1836\color{black} $\,\,$ & $\,\,$ 1 $\,\,$ & $\,\,$\color{red} 0.9854\color{black}  $\,\,$ \\
$\,\,$0.3096$\,\,$ & $\,\,$0.1863$\,\,$ & $\,\,$\color{red} 1.0148\color{black} $\,\,$ & $\,\,$ 1  $\,\,$ \\
\end{pmatrix},
\end{equation*}

\begin{equation*}
\mathbf{w}^{\prime} =
\begin{pmatrix}
0.304768\\
0.506496\\
0.094368\\
0.094368
\end{pmatrix} =
0.998624\cdot
\begin{pmatrix}
0.305188\\
0.507193\\
\color{gr} 0.094498\color{black} \\
0.094498
\end{pmatrix},
\end{equation*}
\begin{equation*}
\left[ \frac{{w}^{\prime}_i}{{w}^{\prime}_j} \right] =
\begin{pmatrix}
$\,\,$ 1 $\,\,$ & $\,\,$0.6017$\,\,$ & $\,\,$\color{gr} 3.2296\color{black} $\,\,$ & $\,\,$3.2296$\,\,$ \\
$\,\,$1.6619$\,\,$ & $\,\,$ 1 $\,\,$ & $\,\,$\color{gr} 5.3672\color{black} $\,\,$ & $\,\,$5.3672  $\,\,$ \\
$\,\,$\color{gr} 0.3096\color{black} $\,\,$ & $\,\,$\color{gr} 0.1863\color{black} $\,\,$ & $\,\,$ 1 $\,\,$ & $\,\,$\color{gr} \color{blue} 1\color{black}  $\,\,$ \\
$\,\,$0.3096$\,\,$ & $\,\,$0.1863$\,\,$ & $\,\,$\color{gr} \color{blue} 1\color{black} $\,\,$ & $\,\,$ 1  $\,\,$ \\
\end{pmatrix},
\end{equation*}
\end{example}
\newpage
\begin{example}
\begin{equation*}
\mathbf{A} =
\begin{pmatrix}
$\,\,$ 1 $\,\,$ & $\,\,$1$\,\,$ & $\,\,$4$\,\,$ & $\,\,$3 $\,\,$ \\
$\,\,$ 1 $\,\,$ & $\,\,$ 1 $\,\,$ & $\,\,$5$\,\,$ & $\,\,$7 $\,\,$ \\
$\,\,$ 1/4$\,\,$ & $\,\,$ 1/5$\,\,$ & $\,\,$ 1 $\,\,$ & $\,\,$1 $\,\,$ \\
$\,\,$ 1/3$\,\,$ & $\,\,$ 1/7$\,\,$ & $\,\,$ 1 $\,\,$ & $\,\,$ 1  $\,\,$ \\
\end{pmatrix},
\qquad
\lambda_{\max} =
4.0609,
\qquad
CR = 0.0230
\end{equation*}

\begin{equation*}
\mathbf{w}^{EM} =
\begin{pmatrix}
0.355882\\
0.465883\\
\color{red} 0.088744\color{black} \\
0.089491
\end{pmatrix}\end{equation*}
\begin{equation*}
\left[ \frac{{w}^{EM}_i}{{w}^{EM}_j} \right] =
\begin{pmatrix}
$\,\,$ 1 $\,\,$ & $\,\,$0.7639$\,\,$ & $\,\,$\color{red} 4.0102\color{black} $\,\,$ & $\,\,$3.9767$\,\,$ \\
$\,\,$1.3091$\,\,$ & $\,\,$ 1 $\,\,$ & $\,\,$\color{red} 5.2498\color{black} $\,\,$ & $\,\,$5.2059  $\,\,$ \\
$\,\,$\color{red} 0.2494\color{black} $\,\,$ & $\,\,$\color{red} 0.1905\color{black} $\,\,$ & $\,\,$ 1 $\,\,$ & $\,\,$\color{red} 0.9916\color{black}  $\,\,$ \\
$\,\,$0.2515$\,\,$ & $\,\,$0.1921$\,\,$ & $\,\,$\color{red} 1.0084\color{black} $\,\,$ & $\,\,$ 1  $\,\,$ \\
\end{pmatrix},
\end{equation*}

\begin{equation*}
\mathbf{w}^{\prime} =
\begin{pmatrix}
0.355801\\
0.465778\\
0.088950\\
0.089471
\end{pmatrix} =
0.999773\cdot
\begin{pmatrix}
0.355882\\
0.465883\\
\color{gr} 0.088971\color{black} \\
0.089491
\end{pmatrix},
\end{equation*}
\begin{equation*}
\left[ \frac{{w}^{\prime}_i}{{w}^{\prime}_j} \right] =
\begin{pmatrix}
$\,\,$ 1 $\,\,$ & $\,\,$0.7639$\,\,$ & $\,\,$\color{gr} \color{blue} 4\color{black} $\,\,$ & $\,\,$3.9767$\,\,$ \\
$\,\,$1.3091$\,\,$ & $\,\,$ 1 $\,\,$ & $\,\,$\color{gr} 5.2364\color{black} $\,\,$ & $\,\,$5.2059  $\,\,$ \\
$\,\,$\color{gr} \color{blue}  1/4\color{black} $\,\,$ & $\,\,$\color{gr} 0.1910\color{black} $\,\,$ & $\,\,$ 1 $\,\,$ & $\,\,$\color{gr} 0.9942\color{black}  $\,\,$ \\
$\,\,$0.2515$\,\,$ & $\,\,$0.1921$\,\,$ & $\,\,$\color{gr} 1.0058\color{black} $\,\,$ & $\,\,$ 1  $\,\,$ \\
\end{pmatrix},
\end{equation*}
\end{example}
\newpage
\begin{example}
\begin{equation*}
\mathbf{A} =
\begin{pmatrix}
$\,\,$ 1 $\,\,$ & $\,\,$1$\,\,$ & $\,\,$4$\,\,$ & $\,\,$3 $\,\,$ \\
$\,\,$ 1 $\,\,$ & $\,\,$ 1 $\,\,$ & $\,\,$6$\,\,$ & $\,\,$9 $\,\,$ \\
$\,\,$ 1/4$\,\,$ & $\,\,$ 1/6$\,\,$ & $\,\,$ 1 $\,\,$ & $\,\,$1 $\,\,$ \\
$\,\,$ 1/3$\,\,$ & $\,\,$ 1/9$\,\,$ & $\,\,$ 1 $\,\,$ & $\,\,$ 1  $\,\,$ \\
\end{pmatrix},
\qquad
\lambda_{\max} =
4.1031,
\qquad
CR = 0.0389
\end{equation*}

\begin{equation*}
\mathbf{w}^{EM} =
\begin{pmatrix}
0.341478\\
0.498030\\
\color{red} 0.080150\color{black} \\
0.080342
\end{pmatrix}\end{equation*}
\begin{equation*}
\left[ \frac{{w}^{EM}_i}{{w}^{EM}_j} \right] =
\begin{pmatrix}
$\,\,$ 1 $\,\,$ & $\,\,$0.6857$\,\,$ & $\,\,$\color{red} 4.2605\color{black} $\,\,$ & $\,\,$4.2503$\,\,$ \\
$\,\,$1.4585$\,\,$ & $\,\,$ 1 $\,\,$ & $\,\,$\color{red} 6.2137\color{black} $\,\,$ & $\,\,$6.1989  $\,\,$ \\
$\,\,$\color{red} 0.2347\color{black} $\,\,$ & $\,\,$\color{red} 0.1609\color{black} $\,\,$ & $\,\,$ 1 $\,\,$ & $\,\,$\color{red} 0.9976\color{black}  $\,\,$ \\
$\,\,$0.2353$\,\,$ & $\,\,$0.1613$\,\,$ & $\,\,$\color{red} 1.0024\color{black} $\,\,$ & $\,\,$ 1  $\,\,$ \\
\end{pmatrix},
\end{equation*}

\begin{equation*}
\mathbf{w}^{\prime} =
\begin{pmatrix}
0.341413\\
0.497934\\
0.080327\\
0.080327
\end{pmatrix} =
0.999808\cdot
\begin{pmatrix}
0.341478\\
0.498030\\
\color{gr} 0.080342\color{black} \\
0.080342
\end{pmatrix},
\end{equation*}
\begin{equation*}
\left[ \frac{{w}^{\prime}_i}{{w}^{\prime}_j} \right] =
\begin{pmatrix}
$\,\,$ 1 $\,\,$ & $\,\,$0.6857$\,\,$ & $\,\,$\color{gr} 4.2503\color{black} $\,\,$ & $\,\,$4.2503$\,\,$ \\
$\,\,$1.4585$\,\,$ & $\,\,$ 1 $\,\,$ & $\,\,$\color{gr} 6.1989\color{black} $\,\,$ & $\,\,$6.1989  $\,\,$ \\
$\,\,$\color{gr} 0.2353\color{black} $\,\,$ & $\,\,$\color{gr} 0.1613\color{black} $\,\,$ & $\,\,$ 1 $\,\,$ & $\,\,$\color{gr} \color{blue} 1\color{black}  $\,\,$ \\
$\,\,$0.2353$\,\,$ & $\,\,$0.1613$\,\,$ & $\,\,$\color{gr} \color{blue} 1\color{black} $\,\,$ & $\,\,$ 1  $\,\,$ \\
\end{pmatrix},
\end{equation*}
\end{example}
\newpage
\begin{example}
\begin{equation*}
\mathbf{A} =
\begin{pmatrix}
$\,\,$ 1 $\,\,$ & $\,\,$1$\,\,$ & $\,\,$5$\,\,$ & $\,\,$1 $\,\,$ \\
$\,\,$ 1 $\,\,$ & $\,\,$ 1 $\,\,$ & $\,\,$3$\,\,$ & $\,\,$2 $\,\,$ \\
$\,\,$ 1/5$\,\,$ & $\,\,$ 1/3$\,\,$ & $\,\,$ 1 $\,\,$ & $\,\,$1 $\,\,$ \\
$\,\,$ 1 $\,\,$ & $\,\,$ 1/2$\,\,$ & $\,\,$ 1 $\,\,$ & $\,\,$ 1  $\,\,$ \\
\end{pmatrix},
\qquad
\lambda_{\max} =
4.2277,
\qquad
CR = 0.0859
\end{equation*}

\begin{equation*}
\mathbf{w}^{EM} =
\begin{pmatrix}
0.347588\\
\color{red} 0.338539\color{black} \\
0.117377\\
0.196496
\end{pmatrix}\end{equation*}
\begin{equation*}
\left[ \frac{{w}^{EM}_i}{{w}^{EM}_j} \right] =
\begin{pmatrix}
$\,\,$ 1 $\,\,$ & $\,\,$\color{red} 1.0267\color{black} $\,\,$ & $\,\,$2.9613$\,\,$ & $\,\,$1.7689$\,\,$ \\
$\,\,$\color{red} 0.9740\color{black} $\,\,$ & $\,\,$ 1 $\,\,$ & $\,\,$\color{red} 2.8842\color{black} $\,\,$ & $\,\,$\color{red} 1.7229\color{black}   $\,\,$ \\
$\,\,$0.3377$\,\,$ & $\,\,$\color{red} 0.3467\color{black} $\,\,$ & $\,\,$ 1 $\,\,$ & $\,\,$0.5973 $\,\,$ \\
$\,\,$0.5653$\,\,$ & $\,\,$\color{red} 0.5804\color{black} $\,\,$ & $\,\,$1.6741$\,\,$ & $\,\,$ 1  $\,\,$ \\
\end{pmatrix},
\end{equation*}

\begin{equation*}
\mathbf{w}^{\prime} =
\begin{pmatrix}
0.344471\\
0.344471\\
0.116324\\
0.194734
\end{pmatrix} =
0.991032\cdot
\begin{pmatrix}
0.347588\\
\color{gr} 0.347588\color{black} \\
0.117377\\
0.196496
\end{pmatrix},
\end{equation*}
\begin{equation*}
\left[ \frac{{w}^{\prime}_i}{{w}^{\prime}_j} \right] =
\begin{pmatrix}
$\,\,$ 1 $\,\,$ & $\,\,$\color{gr} \color{blue} 1\color{black} $\,\,$ & $\,\,$2.9613$\,\,$ & $\,\,$1.7689$\,\,$ \\
$\,\,$\color{gr} \color{blue} 1\color{black} $\,\,$ & $\,\,$ 1 $\,\,$ & $\,\,$\color{gr} 2.9613\color{black} $\,\,$ & $\,\,$\color{gr} 1.7689\color{black}   $\,\,$ \\
$\,\,$0.3377$\,\,$ & $\,\,$\color{gr} 0.3377\color{black} $\,\,$ & $\,\,$ 1 $\,\,$ & $\,\,$0.5973 $\,\,$ \\
$\,\,$0.5653$\,\,$ & $\,\,$\color{gr} 0.5653\color{black} $\,\,$ & $\,\,$1.6741$\,\,$ & $\,\,$ 1  $\,\,$ \\
\end{pmatrix},
\end{equation*}
\end{example}
\newpage
\begin{example}
\begin{equation*}
\mathbf{A} =
\begin{pmatrix}
$\,\,$ 1 $\,\,$ & $\,\,$1$\,\,$ & $\,\,$5$\,\,$ & $\,\,$2 $\,\,$ \\
$\,\,$ 1 $\,\,$ & $\,\,$ 1 $\,\,$ & $\,\,$3$\,\,$ & $\,\,$4 $\,\,$ \\
$\,\,$ 1/5$\,\,$ & $\,\,$ 1/3$\,\,$ & $\,\,$ 1 $\,\,$ & $\,\,$2 $\,\,$ \\
$\,\,$ 1/2$\,\,$ & $\,\,$ 1/4$\,\,$ & $\,\,$ 1/2$\,\,$ & $\,\,$ 1  $\,\,$ \\
\end{pmatrix},
\qquad
\lambda_{\max} =
4.2277,
\qquad
CR = 0.0859
\end{equation*}

\begin{equation*}
\mathbf{w}^{EM} =
\begin{pmatrix}
0.385459\\
\color{red} 0.375424\color{black} \\
0.130165\\
0.108952
\end{pmatrix}\end{equation*}
\begin{equation*}
\left[ \frac{{w}^{EM}_i}{{w}^{EM}_j} \right] =
\begin{pmatrix}
$\,\,$ 1 $\,\,$ & $\,\,$\color{red} 1.0267\color{black} $\,\,$ & $\,\,$2.9613$\,\,$ & $\,\,$3.5379$\,\,$ \\
$\,\,$\color{red} 0.9740\color{black} $\,\,$ & $\,\,$ 1 $\,\,$ & $\,\,$\color{red} 2.8842\color{black} $\,\,$ & $\,\,$\color{red} 3.4458\color{black}   $\,\,$ \\
$\,\,$0.3377$\,\,$ & $\,\,$\color{red} 0.3467\color{black} $\,\,$ & $\,\,$ 1 $\,\,$ & $\,\,$1.1947 $\,\,$ \\
$\,\,$0.2827$\,\,$ & $\,\,$\color{red} 0.2902\color{black} $\,\,$ & $\,\,$0.8370$\,\,$ & $\,\,$ 1  $\,\,$ \\
\end{pmatrix},
\end{equation*}

\begin{equation*}
\mathbf{w}^{\prime} =
\begin{pmatrix}
0.381629\\
0.381629\\
0.128872\\
0.107870
\end{pmatrix} =
0.990065\cdot
\begin{pmatrix}
0.385459\\
\color{gr} 0.385459\color{black} \\
0.130165\\
0.108952
\end{pmatrix},
\end{equation*}
\begin{equation*}
\left[ \frac{{w}^{\prime}_i}{{w}^{\prime}_j} \right] =
\begin{pmatrix}
$\,\,$ 1 $\,\,$ & $\,\,$\color{gr} \color{blue} 1\color{black} $\,\,$ & $\,\,$2.9613$\,\,$ & $\,\,$3.5379$\,\,$ \\
$\,\,$\color{gr} \color{blue} 1\color{black} $\,\,$ & $\,\,$ 1 $\,\,$ & $\,\,$\color{gr} 2.9613\color{black} $\,\,$ & $\,\,$\color{gr} 3.5379\color{black}   $\,\,$ \\
$\,\,$0.3377$\,\,$ & $\,\,$\color{gr} 0.3377\color{black} $\,\,$ & $\,\,$ 1 $\,\,$ & $\,\,$1.1947 $\,\,$ \\
$\,\,$0.2827$\,\,$ & $\,\,$\color{gr} 0.2827\color{black} $\,\,$ & $\,\,$0.8370$\,\,$ & $\,\,$ 1  $\,\,$ \\
\end{pmatrix},
\end{equation*}
\end{example}
\newpage
\begin{example}
\begin{equation*}
\mathbf{A} =
\begin{pmatrix}
$\,\,$ 1 $\,\,$ & $\,\,$1$\,\,$ & $\,\,$5$\,\,$ & $\,\,$3 $\,\,$ \\
$\,\,$ 1 $\,\,$ & $\,\,$ 1 $\,\,$ & $\,\,$3$\,\,$ & $\,\,$6 $\,\,$ \\
$\,\,$ 1/5$\,\,$ & $\,\,$ 1/3$\,\,$ & $\,\,$ 1 $\,\,$ & $\,\,$3 $\,\,$ \\
$\,\,$ 1/3$\,\,$ & $\,\,$ 1/6$\,\,$ & $\,\,$ 1/3$\,\,$ & $\,\,$ 1  $\,\,$ \\
\end{pmatrix},
\qquad
\lambda_{\max} =
4.2277,
\qquad
CR = 0.0859
\end{equation*}

\begin{equation*}
\mathbf{w}^{EM} =
\begin{pmatrix}
0.399985\\
\color{red} 0.389572\color{black} \\
0.135071\\
0.075372
\end{pmatrix}\end{equation*}
\begin{equation*}
\left[ \frac{{w}^{EM}_i}{{w}^{EM}_j} \right] =
\begin{pmatrix}
$\,\,$ 1 $\,\,$ & $\,\,$\color{red} 1.0267\color{black} $\,\,$ & $\,\,$2.9613$\,\,$ & $\,\,$5.3068$\,\,$ \\
$\,\,$\color{red} 0.9740\color{black} $\,\,$ & $\,\,$ 1 $\,\,$ & $\,\,$\color{red} 2.8842\color{black} $\,\,$ & $\,\,$\color{red} 5.1686\color{black}   $\,\,$ \\
$\,\,$0.3377$\,\,$ & $\,\,$\color{red} 0.3467\color{black} $\,\,$ & $\,\,$ 1 $\,\,$ & $\,\,$1.7920 $\,\,$ \\
$\,\,$0.1884$\,\,$ & $\,\,$\color{red} 0.1935\color{black} $\,\,$ & $\,\,$0.5580$\,\,$ & $\,\,$ 1  $\,\,$ \\
\end{pmatrix},
\end{equation*}

\begin{equation*}
\mathbf{w}^{\prime} =
\begin{pmatrix}
0.395863\\
0.395863\\
0.133679\\
0.074595
\end{pmatrix} =
0.989694\cdot
\begin{pmatrix}
0.399985\\
\color{gr} 0.399985\color{black} \\
0.135071\\
0.075372
\end{pmatrix},
\end{equation*}
\begin{equation*}
\left[ \frac{{w}^{\prime}_i}{{w}^{\prime}_j} \right] =
\begin{pmatrix}
$\,\,$ 1 $\,\,$ & $\,\,$\color{gr} \color{blue} 1\color{black} $\,\,$ & $\,\,$2.9613$\,\,$ & $\,\,$5.3068$\,\,$ \\
$\,\,$\color{gr} \color{blue} 1\color{black} $\,\,$ & $\,\,$ 1 $\,\,$ & $\,\,$\color{gr} 2.9613\color{black} $\,\,$ & $\,\,$\color{gr} 5.3068\color{black}   $\,\,$ \\
$\,\,$0.3377$\,\,$ & $\,\,$\color{gr} 0.3377\color{black} $\,\,$ & $\,\,$ 1 $\,\,$ & $\,\,$1.7920 $\,\,$ \\
$\,\,$0.1884$\,\,$ & $\,\,$\color{gr} 0.1884\color{black} $\,\,$ & $\,\,$0.5580$\,\,$ & $\,\,$ 1  $\,\,$ \\
\end{pmatrix},
\end{equation*}
\end{example}
\newpage
\begin{example}
\begin{equation*}
\mathbf{A} =
\begin{pmatrix}
$\,\,$ 1 $\,\,$ & $\,\,$1$\,\,$ & $\,\,$6$\,\,$ & $\,\,$2 $\,\,$ \\
$\,\,$ 1 $\,\,$ & $\,\,$ 1 $\,\,$ & $\,\,$4$\,\,$ & $\,\,$3 $\,\,$ \\
$\,\,$ 1/6$\,\,$ & $\,\,$ 1/4$\,\,$ & $\,\,$ 1 $\,\,$ & $\,\,$1 $\,\,$ \\
$\,\,$ 1/2$\,\,$ & $\,\,$ 1/3$\,\,$ & $\,\,$ 1 $\,\,$ & $\,\,$ 1  $\,\,$ \\
\end{pmatrix},
\qquad
\lambda_{\max} =
4.1031,
\qquad
CR = 0.0389
\end{equation*}

\begin{equation*}
\mathbf{w}^{EM} =
\begin{pmatrix}
0.392093\\
\color{red} 0.378607\color{black} \\
0.094879\\
0.134421
\end{pmatrix}\end{equation*}
\begin{equation*}
\left[ \frac{{w}^{EM}_i}{{w}^{EM}_j} \right] =
\begin{pmatrix}
$\,\,$ 1 $\,\,$ & $\,\,$\color{red} 1.0356\color{black} $\,\,$ & $\,\,$4.1326$\,\,$ & $\,\,$2.9169$\,\,$ \\
$\,\,$\color{red} 0.9656\color{black} $\,\,$ & $\,\,$ 1 $\,\,$ & $\,\,$\color{red} 3.9904\color{black} $\,\,$ & $\,\,$\color{red} 2.8166\color{black}   $\,\,$ \\
$\,\,$0.2420$\,\,$ & $\,\,$\color{red} 0.2506\color{black} $\,\,$ & $\,\,$ 1 $\,\,$ & $\,\,$0.7058 $\,\,$ \\
$\,\,$0.3428$\,\,$ & $\,\,$\color{red} 0.3550\color{black} $\,\,$ & $\,\,$1.4168$\,\,$ & $\,\,$ 1  $\,\,$ \\
\end{pmatrix},
\end{equation*}

\begin{equation*}
\mathbf{w}^{\prime} =
\begin{pmatrix}
0.391738\\
0.379170\\
0.094793\\
0.134299
\end{pmatrix} =
0.999093\cdot
\begin{pmatrix}
0.392093\\
\color{gr} 0.379514\color{black} \\
0.094879\\
0.134421
\end{pmatrix},
\end{equation*}
\begin{equation*}
\left[ \frac{{w}^{\prime}_i}{{w}^{\prime}_j} \right] =
\begin{pmatrix}
$\,\,$ 1 $\,\,$ & $\,\,$\color{gr} 1.0331\color{black} $\,\,$ & $\,\,$4.1326$\,\,$ & $\,\,$2.9169$\,\,$ \\
$\,\,$\color{gr} 0.9679\color{black} $\,\,$ & $\,\,$ 1 $\,\,$ & $\,\,$\color{gr} \color{blue} 4\color{black} $\,\,$ & $\,\,$\color{gr} 2.8233\color{black}   $\,\,$ \\
$\,\,$0.2420$\,\,$ & $\,\,$\color{gr} \color{blue}  1/4\color{black} $\,\,$ & $\,\,$ 1 $\,\,$ & $\,\,$0.7058 $\,\,$ \\
$\,\,$0.3428$\,\,$ & $\,\,$\color{gr} 0.3542\color{black} $\,\,$ & $\,\,$1.4168$\,\,$ & $\,\,$ 1  $\,\,$ \\
\end{pmatrix},
\end{equation*}
\end{example}
\newpage
\begin{example}
\begin{equation*}
\mathbf{A} =
\begin{pmatrix}
$\,\,$ 1 $\,\,$ & $\,\,$1$\,\,$ & $\,\,$6$\,\,$ & $\,\,$3 $\,\,$ \\
$\,\,$ 1 $\,\,$ & $\,\,$ 1 $\,\,$ & $\,\,$4$\,\,$ & $\,\,$5 $\,\,$ \\
$\,\,$ 1/6$\,\,$ & $\,\,$ 1/4$\,\,$ & $\,\,$ 1 $\,\,$ & $\,\,$2 $\,\,$ \\
$\,\,$ 1/3$\,\,$ & $\,\,$ 1/5$\,\,$ & $\,\,$ 1/2$\,\,$ & $\,\,$ 1  $\,\,$ \\
\end{pmatrix},
\qquad
\lambda_{\max} =
4.1655,
\qquad
CR = 0.0624
\end{equation*}

\begin{equation*}
\mathbf{w}^{EM} =
\begin{pmatrix}
0.409202\\
\color{red} 0.398791\color{black} \\
0.106845\\
0.085161
\end{pmatrix}\end{equation*}
\begin{equation*}
\left[ \frac{{w}^{EM}_i}{{w}^{EM}_j} \right] =
\begin{pmatrix}
$\,\,$ 1 $\,\,$ & $\,\,$\color{red} 1.0261\color{black} $\,\,$ & $\,\,$3.8299$\,\,$ & $\,\,$4.8050$\,\,$ \\
$\,\,$\color{red} 0.9746\color{black} $\,\,$ & $\,\,$ 1 $\,\,$ & $\,\,$\color{red} 3.7324\color{black} $\,\,$ & $\,\,$\color{red} 4.6828\color{black}   $\,\,$ \\
$\,\,$0.2611$\,\,$ & $\,\,$\color{red} 0.2679\color{black} $\,\,$ & $\,\,$ 1 $\,\,$ & $\,\,$1.2546 $\,\,$ \\
$\,\,$0.2081$\,\,$ & $\,\,$\color{red} 0.2135\color{black} $\,\,$ & $\,\,$0.7971$\,\,$ & $\,\,$ 1  $\,\,$ \\
\end{pmatrix},
\end{equation*}

\begin{equation*}
\mathbf{w}^{\prime} =
\begin{pmatrix}
0.404986\\
0.404986\\
0.105744\\
0.084284
\end{pmatrix} =
0.989696\cdot
\begin{pmatrix}
0.409202\\
\color{gr} 0.409202\color{black} \\
0.106845\\
0.085161
\end{pmatrix},
\end{equation*}
\begin{equation*}
\left[ \frac{{w}^{\prime}_i}{{w}^{\prime}_j} \right] =
\begin{pmatrix}
$\,\,$ 1 $\,\,$ & $\,\,$\color{gr} \color{blue} 1\color{black} $\,\,$ & $\,\,$3.8299$\,\,$ & $\,\,$4.8050$\,\,$ \\
$\,\,$\color{gr} \color{blue} 1\color{black} $\,\,$ & $\,\,$ 1 $\,\,$ & $\,\,$\color{gr} 3.8299\color{black} $\,\,$ & $\,\,$\color{gr} 4.8050\color{black}   $\,\,$ \\
$\,\,$0.2611$\,\,$ & $\,\,$\color{gr} 0.2611\color{black} $\,\,$ & $\,\,$ 1 $\,\,$ & $\,\,$1.2546 $\,\,$ \\
$\,\,$0.2081$\,\,$ & $\,\,$\color{gr} 0.2081\color{black} $\,\,$ & $\,\,$0.7971$\,\,$ & $\,\,$ 1  $\,\,$ \\
\end{pmatrix},
\end{equation*}
\end{example}
\newpage
\begin{example}
\begin{equation*}
\mathbf{A} =
\begin{pmatrix}
$\,\,$ 1 $\,\,$ & $\,\,$1$\,\,$ & $\,\,$6$\,\,$ & $\,\,$4 $\,\,$ \\
$\,\,$ 1 $\,\,$ & $\,\,$ 1 $\,\,$ & $\,\,$4$\,\,$ & $\,\,$7 $\,\,$ \\
$\,\,$ 1/6$\,\,$ & $\,\,$ 1/4$\,\,$ & $\,\,$ 1 $\,\,$ & $\,\,$3 $\,\,$ \\
$\,\,$ 1/4$\,\,$ & $\,\,$ 1/7$\,\,$ & $\,\,$ 1/3$\,\,$ & $\,\,$ 1  $\,\,$ \\
\end{pmatrix},
\qquad
\lambda_{\max} =
4.1964,
\qquad
CR = 0.0741
\end{equation*}

\begin{equation*}
\mathbf{w}^{EM} =
\begin{pmatrix}
0.416938\\
\color{red} 0.408107\color{black} \\
0.112385\\
0.062570
\end{pmatrix}\end{equation*}
\begin{equation*}
\left[ \frac{{w}^{EM}_i}{{w}^{EM}_j} \right] =
\begin{pmatrix}
$\,\,$ 1 $\,\,$ & $\,\,$\color{red} 1.0216\color{black} $\,\,$ & $\,\,$3.7099$\,\,$ & $\,\,$6.6636$\,\,$ \\
$\,\,$\color{red} 0.9788\color{black} $\,\,$ & $\,\,$ 1 $\,\,$ & $\,\,$\color{red} 3.6313\color{black} $\,\,$ & $\,\,$\color{red} 6.5224\color{black}   $\,\,$ \\
$\,\,$0.2695$\,\,$ & $\,\,$\color{red} 0.2754\color{black} $\,\,$ & $\,\,$ 1 $\,\,$ & $\,\,$1.7962 $\,\,$ \\
$\,\,$0.1501$\,\,$ & $\,\,$\color{red} 0.1533\color{black} $\,\,$ & $\,\,$0.5567$\,\,$ & $\,\,$ 1  $\,\,$ \\
\end{pmatrix},
\end{equation*}

\begin{equation*}
\mathbf{w}^{\prime} =
\begin{pmatrix}
0.413288\\
0.413288\\
0.111401\\
0.062022
\end{pmatrix} =
0.991246\cdot
\begin{pmatrix}
0.416938\\
\color{gr} 0.416938\color{black} \\
0.112385\\
0.062570
\end{pmatrix},
\end{equation*}
\begin{equation*}
\left[ \frac{{w}^{\prime}_i}{{w}^{\prime}_j} \right] =
\begin{pmatrix}
$\,\,$ 1 $\,\,$ & $\,\,$\color{gr} \color{blue} 1\color{black} $\,\,$ & $\,\,$3.7099$\,\,$ & $\,\,$6.6636$\,\,$ \\
$\,\,$\color{gr} \color{blue} 1\color{black} $\,\,$ & $\,\,$ 1 $\,\,$ & $\,\,$\color{gr} 3.7099\color{black} $\,\,$ & $\,\,$\color{gr} 6.6636\color{black}   $\,\,$ \\
$\,\,$0.2695$\,\,$ & $\,\,$\color{gr} 0.2695\color{black} $\,\,$ & $\,\,$ 1 $\,\,$ & $\,\,$1.7962 $\,\,$ \\
$\,\,$0.1501$\,\,$ & $\,\,$\color{gr} 0.1501\color{black} $\,\,$ & $\,\,$0.5567$\,\,$ & $\,\,$ 1  $\,\,$ \\
\end{pmatrix},
\end{equation*}
\end{example}
\newpage
\begin{example}
\begin{equation*}
\mathbf{A} =
\begin{pmatrix}
$\,\,$ 1 $\,\,$ & $\,\,$1$\,\,$ & $\,\,$7$\,\,$ & $\,\,$2 $\,\,$ \\
$\,\,$ 1 $\,\,$ & $\,\,$ 1 $\,\,$ & $\,\,$5$\,\,$ & $\,\,$3 $\,\,$ \\
$\,\,$ 1/7$\,\,$ & $\,\,$ 1/5$\,\,$ & $\,\,$ 1 $\,\,$ & $\,\,$1 $\,\,$ \\
$\,\,$ 1/2$\,\,$ & $\,\,$ 1/3$\,\,$ & $\,\,$ 1 $\,\,$ & $\,\,$ 1  $\,\,$ \\
\end{pmatrix},
\qquad
\lambda_{\max} =
4.1351,
\qquad
CR = 0.0509
\end{equation*}

\begin{equation*}
\mathbf{w}^{EM} =
\begin{pmatrix}
0.396520\\
\color{red} 0.387339\color{black} \\
0.084703\\
0.131439
\end{pmatrix}\end{equation*}
\begin{equation*}
\left[ \frac{{w}^{EM}_i}{{w}^{EM}_j} \right] =
\begin{pmatrix}
$\,\,$ 1 $\,\,$ & $\,\,$\color{red} 1.0237\color{black} $\,\,$ & $\,\,$4.6813$\,\,$ & $\,\,$3.0168$\,\,$ \\
$\,\,$\color{red} 0.9768\color{black} $\,\,$ & $\,\,$ 1 $\,\,$ & $\,\,$\color{red} 4.5729\color{black} $\,\,$ & $\,\,$\color{red} 2.9469\color{black}   $\,\,$ \\
$\,\,$0.2136$\,\,$ & $\,\,$\color{red} 0.2187\color{black} $\,\,$ & $\,\,$ 1 $\,\,$ & $\,\,$0.6444 $\,\,$ \\
$\,\,$0.3315$\,\,$ & $\,\,$\color{red} 0.3393\color{black} $\,\,$ & $\,\,$1.5518$\,\,$ & $\,\,$ 1  $\,\,$ \\
\end{pmatrix},
\end{equation*}

\begin{equation*}
\mathbf{w}^{\prime} =
\begin{pmatrix}
0.393772\\
0.391584\\
0.084116\\
0.130528
\end{pmatrix} =
0.993070\cdot
\begin{pmatrix}
0.396520\\
\color{gr} 0.394317\color{black} \\
0.084703\\
0.131439
\end{pmatrix},
\end{equation*}
\begin{equation*}
\left[ \frac{{w}^{\prime}_i}{{w}^{\prime}_j} \right] =
\begin{pmatrix}
$\,\,$ 1 $\,\,$ & $\,\,$\color{gr} 1.0056\color{black} $\,\,$ & $\,\,$4.6813$\,\,$ & $\,\,$3.0168$\,\,$ \\
$\,\,$\color{gr} 0.9944\color{black} $\,\,$ & $\,\,$ 1 $\,\,$ & $\,\,$\color{gr} 4.6553\color{black} $\,\,$ & $\,\,$\color{gr} \color{blue} 3\color{black}   $\,\,$ \\
$\,\,$0.2136$\,\,$ & $\,\,$\color{gr} 0.2148\color{black} $\,\,$ & $\,\,$ 1 $\,\,$ & $\,\,$0.6444 $\,\,$ \\
$\,\,$0.3315$\,\,$ & $\,\,$\color{gr} \color{blue}  1/3\color{black} $\,\,$ & $\,\,$1.5518$\,\,$ & $\,\,$ 1  $\,\,$ \\
\end{pmatrix},
\end{equation*}
\end{example}
\newpage
\begin{example}
\begin{equation*}
\mathbf{A} =
\begin{pmatrix}
$\,\,$ 1 $\,\,$ & $\,\,$1$\,\,$ & $\,\,$7$\,\,$ & $\,\,$3 $\,\,$ \\
$\,\,$ 1 $\,\,$ & $\,\,$ 1 $\,\,$ & $\,\,$4$\,\,$ & $\,\,$5 $\,\,$ \\
$\,\,$ 1/7$\,\,$ & $\,\,$ 1/4$\,\,$ & $\,\,$ 1 $\,\,$ & $\,\,$2 $\,\,$ \\
$\,\,$ 1/3$\,\,$ & $\,\,$ 1/5$\,\,$ & $\,\,$ 1/2$\,\,$ & $\,\,$ 1  $\,\,$ \\
\end{pmatrix},
\qquad
\lambda_{\max} =
4.2057,
\qquad
CR = 0.0776
\end{equation*}

\begin{equation*}
\mathbf{w}^{EM} =
\begin{pmatrix}
0.423221\\
\color{red} 0.390610\color{black} \\
0.101899\\
0.084271
\end{pmatrix}\end{equation*}
\begin{equation*}
\left[ \frac{{w}^{EM}_i}{{w}^{EM}_j} \right] =
\begin{pmatrix}
$\,\,$ 1 $\,\,$ & $\,\,$\color{red} 1.0835\color{black} $\,\,$ & $\,\,$4.1534$\,\,$ & $\,\,$5.0222$\,\,$ \\
$\,\,$\color{red} 0.9229\color{black} $\,\,$ & $\,\,$ 1 $\,\,$ & $\,\,$\color{red} 3.8333\color{black} $\,\,$ & $\,\,$\color{red} 4.6352\color{black}   $\,\,$ \\
$\,\,$0.2408$\,\,$ & $\,\,$\color{red} 0.2609\color{black} $\,\,$ & $\,\,$ 1 $\,\,$ & $\,\,$1.2092 $\,\,$ \\
$\,\,$0.1991$\,\,$ & $\,\,$\color{red} 0.2157\color{black} $\,\,$ & $\,\,$0.8270$\,\,$ & $\,\,$ 1  $\,\,$ \\
\end{pmatrix},
\end{equation*}

\begin{equation*}
\mathbf{w}^{\prime} =
\begin{pmatrix}
0.416153\\
0.400787\\
0.100197\\
0.082863
\end{pmatrix} =
0.983299\cdot
\begin{pmatrix}
0.423221\\
\color{gr} 0.407594\color{black} \\
0.101899\\
0.084271
\end{pmatrix},
\end{equation*}
\begin{equation*}
\left[ \frac{{w}^{\prime}_i}{{w}^{\prime}_j} \right] =
\begin{pmatrix}
$\,\,$ 1 $\,\,$ & $\,\,$\color{gr} 1.0383\color{black} $\,\,$ & $\,\,$4.1534$\,\,$ & $\,\,$5.0222$\,\,$ \\
$\,\,$\color{gr} 0.9631\color{black} $\,\,$ & $\,\,$ 1 $\,\,$ & $\,\,$\color{gr} \color{blue} 4\color{black} $\,\,$ & $\,\,$\color{gr} 4.8367\color{black}   $\,\,$ \\
$\,\,$0.2408$\,\,$ & $\,\,$\color{gr} \color{blue}  1/4\color{black} $\,\,$ & $\,\,$ 1 $\,\,$ & $\,\,$1.2092 $\,\,$ \\
$\,\,$0.1991$\,\,$ & $\,\,$\color{gr} 0.2068\color{black} $\,\,$ & $\,\,$0.8270$\,\,$ & $\,\,$ 1  $\,\,$ \\
\end{pmatrix},
\end{equation*}
\end{example}
\newpage
\begin{example}
\begin{equation*}
\mathbf{A} =
\begin{pmatrix}
$\,\,$ 1 $\,\,$ & $\,\,$1$\,\,$ & $\,\,$7$\,\,$ & $\,\,$3 $\,\,$ \\
$\,\,$ 1 $\,\,$ & $\,\,$ 1 $\,\,$ & $\,\,$5$\,\,$ & $\,\,$5 $\,\,$ \\
$\,\,$ 1/7$\,\,$ & $\,\,$ 1/5$\,\,$ & $\,\,$ 1 $\,\,$ & $\,\,$2 $\,\,$ \\
$\,\,$ 1/3$\,\,$ & $\,\,$ 1/5$\,\,$ & $\,\,$ 1/2$\,\,$ & $\,\,$ 1  $\,\,$ \\
\end{pmatrix},
\qquad
\lambda_{\max} =
4.2095,
\qquad
CR = 0.0790
\end{equation*}

\begin{equation*}
\mathbf{w}^{EM} =
\begin{pmatrix}
0.413489\\
\color{red} 0.407583\color{black} \\
0.095680\\
0.083249
\end{pmatrix}\end{equation*}
\begin{equation*}
\left[ \frac{{w}^{EM}_i}{{w}^{EM}_j} \right] =
\begin{pmatrix}
$\,\,$ 1 $\,\,$ & $\,\,$\color{red} 1.0145\color{black} $\,\,$ & $\,\,$4.3216$\,\,$ & $\,\,$4.9669$\,\,$ \\
$\,\,$\color{red} 0.9857\color{black} $\,\,$ & $\,\,$ 1 $\,\,$ & $\,\,$\color{red} 4.2599\color{black} $\,\,$ & $\,\,$\color{red} 4.8960\color{black}   $\,\,$ \\
$\,\,$0.2314$\,\,$ & $\,\,$\color{red} 0.2347\color{black} $\,\,$ & $\,\,$ 1 $\,\,$ & $\,\,$1.1493 $\,\,$ \\
$\,\,$0.2013$\,\,$ & $\,\,$\color{red} 0.2043\color{black} $\,\,$ & $\,\,$0.8701$\,\,$ & $\,\,$ 1  $\,\,$ \\
\end{pmatrix},
\end{equation*}

\begin{equation*}
\mathbf{w}^{\prime} =
\begin{pmatrix}
0.411061\\
0.411061\\
0.095118\\
0.082760
\end{pmatrix} =
0.994128\cdot
\begin{pmatrix}
0.413489\\
\color{gr} 0.413489\color{black} \\
0.095680\\
0.083249
\end{pmatrix},
\end{equation*}
\begin{equation*}
\left[ \frac{{w}^{\prime}_i}{{w}^{\prime}_j} \right] =
\begin{pmatrix}
$\,\,$ 1 $\,\,$ & $\,\,$\color{gr} \color{blue} 1\color{black} $\,\,$ & $\,\,$4.3216$\,\,$ & $\,\,$4.9669$\,\,$ \\
$\,\,$\color{gr} \color{blue} 1\color{black} $\,\,$ & $\,\,$ 1 $\,\,$ & $\,\,$\color{gr} 4.3216\color{black} $\,\,$ & $\,\,$\color{gr} 4.9669\color{black}   $\,\,$ \\
$\,\,$0.2314$\,\,$ & $\,\,$\color{gr} 0.2314\color{black} $\,\,$ & $\,\,$ 1 $\,\,$ & $\,\,$1.1493 $\,\,$ \\
$\,\,$0.2013$\,\,$ & $\,\,$\color{gr} 0.2013\color{black} $\,\,$ & $\,\,$0.8701$\,\,$ & $\,\,$ 1  $\,\,$ \\
\end{pmatrix},
\end{equation*}
\end{example}
\newpage
\begin{example}
\begin{equation*}
\mathbf{A} =
\begin{pmatrix}
$\,\,$ 1 $\,\,$ & $\,\,$1$\,\,$ & $\,\,$7$\,\,$ & $\,\,$4 $\,\,$ \\
$\,\,$ 1 $\,\,$ & $\,\,$ 1 $\,\,$ & $\,\,$4$\,\,$ & $\,\,$8 $\,\,$ \\
$\,\,$ 1/7$\,\,$ & $\,\,$ 1/4$\,\,$ & $\,\,$ 1 $\,\,$ & $\,\,$3 $\,\,$ \\
$\,\,$ 1/4$\,\,$ & $\,\,$ 1/8$\,\,$ & $\,\,$ 1/3$\,\,$ & $\,\,$ 1  $\,\,$ \\
\end{pmatrix},
\qquad
\lambda_{\max} =
4.2421,
\qquad
CR = 0.0913
\end{equation*}

\begin{equation*}
\mathbf{w}^{EM} =
\begin{pmatrix}
0.426790\\
\color{red} 0.408385\color{black} \\
0.105339\\
0.059485
\end{pmatrix}\end{equation*}
\begin{equation*}
\left[ \frac{{w}^{EM}_i}{{w}^{EM}_j} \right] =
\begin{pmatrix}
$\,\,$ 1 $\,\,$ & $\,\,$\color{red} 1.0451\color{black} $\,\,$ & $\,\,$4.0516$\,\,$ & $\,\,$7.1747$\,\,$ \\
$\,\,$\color{red} 0.9569\color{black} $\,\,$ & $\,\,$ 1 $\,\,$ & $\,\,$\color{red} 3.8768\color{black} $\,\,$ & $\,\,$\color{red} 6.8653\color{black}   $\,\,$ \\
$\,\,$0.2468$\,\,$ & $\,\,$\color{red} 0.2579\color{black} $\,\,$ & $\,\,$ 1 $\,\,$ & $\,\,$1.7708 $\,\,$ \\
$\,\,$0.1394$\,\,$ & $\,\,$\color{red} 0.1457\color{black} $\,\,$ & $\,\,$0.5647$\,\,$ & $\,\,$ 1  $\,\,$ \\
\end{pmatrix},
\end{equation*}

\begin{equation*}
\mathbf{w}^{\prime} =
\begin{pmatrix}
0.421324\\
0.415962\\
0.103990\\
0.058724
\end{pmatrix} =
0.987193\cdot
\begin{pmatrix}
0.426790\\
\color{gr} 0.421358\color{black} \\
0.105339\\
0.059485
\end{pmatrix},
\end{equation*}
\begin{equation*}
\left[ \frac{{w}^{\prime}_i}{{w}^{\prime}_j} \right] =
\begin{pmatrix}
$\,\,$ 1 $\,\,$ & $\,\,$\color{gr} 1.0129\color{black} $\,\,$ & $\,\,$4.0516$\,\,$ & $\,\,$7.1747$\,\,$ \\
$\,\,$\color{gr} 0.9873\color{black} $\,\,$ & $\,\,$ 1 $\,\,$ & $\,\,$\color{gr} \color{blue} 4\color{black} $\,\,$ & $\,\,$\color{gr} 7.0834\color{black}   $\,\,$ \\
$\,\,$0.2468$\,\,$ & $\,\,$\color{gr} \color{blue}  1/4\color{black} $\,\,$ & $\,\,$ 1 $\,\,$ & $\,\,$1.7708 $\,\,$ \\
$\,\,$0.1394$\,\,$ & $\,\,$\color{gr} 0.1412\color{black} $\,\,$ & $\,\,$0.5647$\,\,$ & $\,\,$ 1  $\,\,$ \\
\end{pmatrix},
\end{equation*}
\end{example}
\newpage
\begin{example}
\begin{equation*}
\mathbf{A} =
\begin{pmatrix}
$\,\,$ 1 $\,\,$ & $\,\,$1$\,\,$ & $\,\,$7$\,\,$ & $\,\,$4 $\,\,$ \\
$\,\,$ 1 $\,\,$ & $\,\,$ 1 $\,\,$ & $\,\,$5$\,\,$ & $\,\,$6 $\,\,$ \\
$\,\,$ 1/7$\,\,$ & $\,\,$ 1/5$\,\,$ & $\,\,$ 1 $\,\,$ & $\,\,$2 $\,\,$ \\
$\,\,$ 1/4$\,\,$ & $\,\,$ 1/6$\,\,$ & $\,\,$ 1/2$\,\,$ & $\,\,$ 1  $\,\,$ \\
\end{pmatrix},
\qquad
\lambda_{\max} =
4.1351,
\qquad
CR = 0.0509
\end{equation*}

\begin{equation*}
\mathbf{w}^{EM} =
\begin{pmatrix}
0.424412\\
\color{red} 0.414585\color{black} \\
0.090661\\
0.070342
\end{pmatrix}\end{equation*}
\begin{equation*}
\left[ \frac{{w}^{EM}_i}{{w}^{EM}_j} \right] =
\begin{pmatrix}
$\,\,$ 1 $\,\,$ & $\,\,$\color{red} 1.0237\color{black} $\,\,$ & $\,\,$4.6813$\,\,$ & $\,\,$6.0335$\,\,$ \\
$\,\,$\color{red} 0.9768\color{black} $\,\,$ & $\,\,$ 1 $\,\,$ & $\,\,$\color{red} 4.5729\color{black} $\,\,$ & $\,\,$\color{red} 5.8938\color{black}   $\,\,$ \\
$\,\,$0.2136$\,\,$ & $\,\,$\color{red} 0.2187\color{black} $\,\,$ & $\,\,$ 1 $\,\,$ & $\,\,$1.2889 $\,\,$ \\
$\,\,$0.1657$\,\,$ & $\,\,$\color{red} 0.1697\color{black} $\,\,$ & $\,\,$0.7759$\,\,$ & $\,\,$ 1  $\,\,$ \\
\end{pmatrix},
\end{equation*}

\begin{equation*}
\mathbf{w}^{\prime} =
\begin{pmatrix}
0.421266\\
0.418925\\
0.089989\\
0.069821
\end{pmatrix} =
0.992587\cdot
\begin{pmatrix}
0.424412\\
\color{gr} 0.422054\color{black} \\
0.090661\\
0.070342
\end{pmatrix},
\end{equation*}
\begin{equation*}
\left[ \frac{{w}^{\prime}_i}{{w}^{\prime}_j} \right] =
\begin{pmatrix}
$\,\,$ 1 $\,\,$ & $\,\,$\color{gr} 1.0056\color{black} $\,\,$ & $\,\,$4.6813$\,\,$ & $\,\,$6.0335$\,\,$ \\
$\,\,$\color{gr} 0.9944\color{black} $\,\,$ & $\,\,$ 1 $\,\,$ & $\,\,$\color{gr} 4.6553\color{black} $\,\,$ & $\,\,$\color{gr} \color{blue} 6\color{black}   $\,\,$ \\
$\,\,$0.2136$\,\,$ & $\,\,$\color{gr} 0.2148\color{black} $\,\,$ & $\,\,$ 1 $\,\,$ & $\,\,$1.2889 $\,\,$ \\
$\,\,$0.1657$\,\,$ & $\,\,$\color{gr} \color{blue}  1/6\color{black} $\,\,$ & $\,\,$0.7759$\,\,$ & $\,\,$ 1  $\,\,$ \\
\end{pmatrix},
\end{equation*}
\end{example}
\newpage
\begin{example}
\begin{equation*}
\mathbf{A} =
\begin{pmatrix}
$\,\,$ 1 $\,\,$ & $\,\,$1$\,\,$ & $\,\,$7$\,\,$ & $\,\,$4 $\,\,$ \\
$\,\,$ 1 $\,\,$ & $\,\,$ 1 $\,\,$ & $\,\,$5$\,\,$ & $\,\,$7 $\,\,$ \\
$\,\,$ 1/7$\,\,$ & $\,\,$ 1/5$\,\,$ & $\,\,$ 1 $\,\,$ & $\,\,$3 $\,\,$ \\
$\,\,$ 1/4$\,\,$ & $\,\,$ 1/7$\,\,$ & $\,\,$ 1/3$\,\,$ & $\,\,$ 1  $\,\,$ \\
\end{pmatrix},
\qquad
\lambda_{\max} =
4.2453,
\qquad
CR = 0.0925
\end{equation*}

\begin{equation*}
\mathbf{w}^{EM} =
\begin{pmatrix}
0.421175\\
\color{red} 0.416917\color{black} \\
0.100760\\
0.061147
\end{pmatrix}\end{equation*}
\begin{equation*}
\left[ \frac{{w}^{EM}_i}{{w}^{EM}_j} \right] =
\begin{pmatrix}
$\,\,$ 1 $\,\,$ & $\,\,$\color{red} 1.0102\color{black} $\,\,$ & $\,\,$4.1800$\,\,$ & $\,\,$6.8879$\,\,$ \\
$\,\,$\color{red} 0.9899\color{black} $\,\,$ & $\,\,$ 1 $\,\,$ & $\,\,$\color{red} 4.1377\color{black} $\,\,$ & $\,\,$\color{red} 6.8182\color{black}   $\,\,$ \\
$\,\,$0.2392$\,\,$ & $\,\,$\color{red} 0.2417\color{black} $\,\,$ & $\,\,$ 1 $\,\,$ & $\,\,$1.6478 $\,\,$ \\
$\,\,$0.1452$\,\,$ & $\,\,$\color{red} 0.1467\color{black} $\,\,$ & $\,\,$0.6069$\,\,$ & $\,\,$ 1  $\,\,$ \\
\end{pmatrix},
\end{equation*}

\begin{equation*}
\mathbf{w}^{\prime} =
\begin{pmatrix}
0.419389\\
0.419389\\
0.100333\\
0.060888
\end{pmatrix} =
0.995760\cdot
\begin{pmatrix}
0.421175\\
\color{gr} 0.421175\color{black} \\
0.100760\\
0.061147
\end{pmatrix},
\end{equation*}
\begin{equation*}
\left[ \frac{{w}^{\prime}_i}{{w}^{\prime}_j} \right] =
\begin{pmatrix}
$\,\,$ 1 $\,\,$ & $\,\,$\color{gr} \color{blue} 1\color{black} $\,\,$ & $\,\,$4.1800$\,\,$ & $\,\,$6.8879$\,\,$ \\
$\,\,$\color{gr} \color{blue} 1\color{black} $\,\,$ & $\,\,$ 1 $\,\,$ & $\,\,$\color{gr} 4.1800\color{black} $\,\,$ & $\,\,$\color{gr} 6.8879\color{black}   $\,\,$ \\
$\,\,$0.2392$\,\,$ & $\,\,$\color{gr} 0.2392\color{black} $\,\,$ & $\,\,$ 1 $\,\,$ & $\,\,$1.6478 $\,\,$ \\
$\,\,$0.1452$\,\,$ & $\,\,$\color{gr} 0.1452\color{black} $\,\,$ & $\,\,$0.6069$\,\,$ & $\,\,$ 1  $\,\,$ \\
\end{pmatrix},
\end{equation*}
\end{example}
\newpage
\begin{example}
\begin{equation*}
\mathbf{A} =
\begin{pmatrix}
$\,\,$ 1 $\,\,$ & $\,\,$1$\,\,$ & $\,\,$7$\,\,$ & $\,\,$5 $\,\,$ \\
$\,\,$ 1 $\,\,$ & $\,\,$ 1 $\,\,$ & $\,\,$5$\,\,$ & $\,\,$8 $\,\,$ \\
$\,\,$ 1/7$\,\,$ & $\,\,$ 1/5$\,\,$ & $\,\,$ 1 $\,\,$ & $\,\,$3 $\,\,$ \\
$\,\,$ 1/5$\,\,$ & $\,\,$ 1/8$\,\,$ & $\,\,$ 1/3$\,\,$ & $\,\,$ 1  $\,\,$ \\
\end{pmatrix},
\qquad
\lambda_{\max} =
4.1801,
\qquad
CR = 0.0679
\end{equation*}

\begin{equation*}
\mathbf{w}^{EM} =
\begin{pmatrix}
0.428823\\
\color{red} 0.421213\color{black} \\
0.096340\\
0.053624
\end{pmatrix}\end{equation*}
\begin{equation*}
\left[ \frac{{w}^{EM}_i}{{w}^{EM}_j} \right] =
\begin{pmatrix}
$\,\,$ 1 $\,\,$ & $\,\,$\color{red} 1.0181\color{black} $\,\,$ & $\,\,$4.4511$\,\,$ & $\,\,$7.9969$\,\,$ \\
$\,\,$\color{red} 0.9823\color{black} $\,\,$ & $\,\,$ 1 $\,\,$ & $\,\,$\color{red} 4.3721\color{black} $\,\,$ & $\,\,$\color{red} 7.8550\color{black}   $\,\,$ \\
$\,\,$0.2247$\,\,$ & $\,\,$\color{red} 0.2287\color{black} $\,\,$ & $\,\,$ 1 $\,\,$ & $\,\,$1.7966 $\,\,$ \\
$\,\,$0.1250$\,\,$ & $\,\,$\color{red} 0.1273\color{black} $\,\,$ & $\,\,$0.5566$\,\,$ & $\,\,$ 1  $\,\,$ \\
\end{pmatrix},
\end{equation*}

\begin{equation*}
\mathbf{w}^{\prime} =
\begin{pmatrix}
0.425584\\
0.425584\\
0.095613\\
0.053219
\end{pmatrix} =
0.992448\cdot
\begin{pmatrix}
0.428823\\
\color{gr} 0.428823\color{black} \\
0.096340\\
0.053624
\end{pmatrix},
\end{equation*}
\begin{equation*}
\left[ \frac{{w}^{\prime}_i}{{w}^{\prime}_j} \right] =
\begin{pmatrix}
$\,\,$ 1 $\,\,$ & $\,\,$\color{gr} \color{blue} 1\color{black} $\,\,$ & $\,\,$4.4511$\,\,$ & $\,\,$7.9969$\,\,$ \\
$\,\,$\color{gr} \color{blue} 1\color{black} $\,\,$ & $\,\,$ 1 $\,\,$ & $\,\,$\color{gr} 4.4511\color{black} $\,\,$ & $\,\,$\color{gr} 7.9969\color{black}   $\,\,$ \\
$\,\,$0.2247$\,\,$ & $\,\,$\color{gr} 0.2247\color{black} $\,\,$ & $\,\,$ 1 $\,\,$ & $\,\,$1.7966 $\,\,$ \\
$\,\,$0.1250$\,\,$ & $\,\,$\color{gr} 0.1250\color{black} $\,\,$ & $\,\,$0.5566$\,\,$ & $\,\,$ 1  $\,\,$ \\
\end{pmatrix},
\end{equation*}
\end{example}
\newpage
\begin{example}
\begin{equation*}
\mathbf{A} =
\begin{pmatrix}
$\,\,$ 1 $\,\,$ & $\,\,$1$\,\,$ & $\,\,$8$\,\,$ & $\,\,$2 $\,\,$ \\
$\,\,$ 1 $\,\,$ & $\,\,$ 1 $\,\,$ & $\,\,$5$\,\,$ & $\,\,$3 $\,\,$ \\
$\,\,$ 1/8$\,\,$ & $\,\,$ 1/5$\,\,$ & $\,\,$ 1 $\,\,$ & $\,\,$1 $\,\,$ \\
$\,\,$ 1/2$\,\,$ & $\,\,$ 1/3$\,\,$ & $\,\,$ 1 $\,\,$ & $\,\,$ 1  $\,\,$ \\
\end{pmatrix},
\qquad
\lambda_{\max} =
4.1655,
\qquad
CR = 0.0624
\end{equation*}

\begin{equation*}
\mathbf{w}^{EM} =
\begin{pmatrix}
0.407909\\
\color{red} 0.380623\color{black} \\
0.081282\\
0.130187
\end{pmatrix}\end{equation*}
\begin{equation*}
\left[ \frac{{w}^{EM}_i}{{w}^{EM}_j} \right] =
\begin{pmatrix}
$\,\,$ 1 $\,\,$ & $\,\,$\color{red} 1.0717\color{black} $\,\,$ & $\,\,$5.0185$\,\,$ & $\,\,$3.1333$\,\,$ \\
$\,\,$\color{red} 0.9331\color{black} $\,\,$ & $\,\,$ 1 $\,\,$ & $\,\,$\color{red} 4.6828\color{black} $\,\,$ & $\,\,$\color{red} 2.9237\color{black}   $\,\,$ \\
$\,\,$0.1993$\,\,$ & $\,\,$\color{red} 0.2135\color{black} $\,\,$ & $\,\,$ 1 $\,\,$ & $\,\,$0.6243 $\,\,$ \\
$\,\,$0.3192$\,\,$ & $\,\,$\color{red} 0.3420\color{black} $\,\,$ & $\,\,$1.6017$\,\,$ & $\,\,$ 1  $\,\,$ \\
\end{pmatrix},
\end{equation*}

\begin{equation*}
\mathbf{w}^{\prime} =
\begin{pmatrix}
0.403895\\
0.386717\\
0.080482\\
0.128906
\end{pmatrix} =
0.990161\cdot
\begin{pmatrix}
0.407909\\
\color{gr} 0.390560\color{black} \\
0.081282\\
0.130187
\end{pmatrix},
\end{equation*}
\begin{equation*}
\left[ \frac{{w}^{\prime}_i}{{w}^{\prime}_j} \right] =
\begin{pmatrix}
$\,\,$ 1 $\,\,$ & $\,\,$\color{gr} 1.0444\color{black} $\,\,$ & $\,\,$5.0185$\,\,$ & $\,\,$3.1333$\,\,$ \\
$\,\,$\color{gr} 0.9575\color{black} $\,\,$ & $\,\,$ 1 $\,\,$ & $\,\,$\color{gr} 4.8050\color{black} $\,\,$ & $\,\,$\color{gr} \color{blue} 3\color{black}   $\,\,$ \\
$\,\,$0.1993$\,\,$ & $\,\,$\color{gr} 0.2081\color{black} $\,\,$ & $\,\,$ 1 $\,\,$ & $\,\,$0.6243 $\,\,$ \\
$\,\,$0.3192$\,\,$ & $\,\,$\color{gr} \color{blue}  1/3\color{black} $\,\,$ & $\,\,$1.6017$\,\,$ & $\,\,$ 1  $\,\,$ \\
\end{pmatrix},
\end{equation*}
\end{example}
\newpage
\begin{example}
\begin{equation*}
\mathbf{A} =
\begin{pmatrix}
$\,\,$ 1 $\,\,$ & $\,\,$1$\,\,$ & $\,\,$8$\,\,$ & $\,\,$3 $\,\,$ \\
$\,\,$ 1 $\,\,$ & $\,\,$ 1 $\,\,$ & $\,\,$4$\,\,$ & $\,\,$5 $\,\,$ \\
$\,\,$ 1/8$\,\,$ & $\,\,$ 1/4$\,\,$ & $\,\,$ 1 $\,\,$ & $\,\,$2 $\,\,$ \\
$\,\,$ 1/3$\,\,$ & $\,\,$ 1/5$\,\,$ & $\,\,$ 1/2$\,\,$ & $\,\,$ 1  $\,\,$ \\
\end{pmatrix},
\qquad
\lambda_{\max} =
4.2460,
\qquad
CR = 0.0928
\end{equation*}

\begin{equation*}
\mathbf{w}^{EM} =
\begin{pmatrix}
0.435826\\
\color{red} 0.383095\color{black} \\
0.097675\\
0.083404
\end{pmatrix}\end{equation*}
\begin{equation*}
\left[ \frac{{w}^{EM}_i}{{w}^{EM}_j} \right] =
\begin{pmatrix}
$\,\,$ 1 $\,\,$ & $\,\,$\color{red} 1.1376\color{black} $\,\,$ & $\,\,$4.4620$\,\,$ & $\,\,$5.2255$\,\,$ \\
$\,\,$\color{red} 0.8790\color{black} $\,\,$ & $\,\,$ 1 $\,\,$ & $\,\,$\color{red} 3.9221\color{black} $\,\,$ & $\,\,$\color{red} 4.5933\color{black}   $\,\,$ \\
$\,\,$0.2241$\,\,$ & $\,\,$\color{red} 0.2550\color{black} $\,\,$ & $\,\,$ 1 $\,\,$ & $\,\,$1.1711 $\,\,$ \\
$\,\,$0.1914$\,\,$ & $\,\,$\color{red} 0.2177\color{black} $\,\,$ & $\,\,$0.8539$\,\,$ & $\,\,$ 1  $\,\,$ \\
\end{pmatrix},
\end{equation*}

\begin{equation*}
\mathbf{w}^{\prime} =
\begin{pmatrix}
0.432535\\
0.387752\\
0.096938\\
0.082774
\end{pmatrix} =
0.992451\cdot
\begin{pmatrix}
0.435826\\
\color{gr} 0.390702\color{black} \\
0.097675\\
0.083404
\end{pmatrix},
\end{equation*}
\begin{equation*}
\left[ \frac{{w}^{\prime}_i}{{w}^{\prime}_j} \right] =
\begin{pmatrix}
$\,\,$ 1 $\,\,$ & $\,\,$\color{gr} 1.1155\color{black} $\,\,$ & $\,\,$4.4620$\,\,$ & $\,\,$5.2255$\,\,$ \\
$\,\,$\color{gr} 0.8965\color{black} $\,\,$ & $\,\,$ 1 $\,\,$ & $\,\,$\color{gr} \color{blue} 4\color{black} $\,\,$ & $\,\,$\color{gr} 4.6845\color{black}   $\,\,$ \\
$\,\,$0.2241$\,\,$ & $\,\,$\color{gr} \color{blue}  1/4\color{black} $\,\,$ & $\,\,$ 1 $\,\,$ & $\,\,$1.1711 $\,\,$ \\
$\,\,$0.1914$\,\,$ & $\,\,$\color{gr} 0.2135\color{black} $\,\,$ & $\,\,$0.8539$\,\,$ & $\,\,$ 1  $\,\,$ \\
\end{pmatrix},
\end{equation*}
\end{example}
\newpage
\begin{example}
\begin{equation*}
\mathbf{A} =
\begin{pmatrix}
$\,\,$ 1 $\,\,$ & $\,\,$1$\,\,$ & $\,\,$8$\,\,$ & $\,\,$3 $\,\,$ \\
$\,\,$ 1 $\,\,$ & $\,\,$ 1 $\,\,$ & $\,\,$5$\,\,$ & $\,\,$5 $\,\,$ \\
$\,\,$ 1/8$\,\,$ & $\,\,$ 1/5$\,\,$ & $\,\,$ 1 $\,\,$ & $\,\,$2 $\,\,$ \\
$\,\,$ 1/3$\,\,$ & $\,\,$ 1/5$\,\,$ & $\,\,$ 1/2$\,\,$ & $\,\,$ 1  $\,\,$ \\
\end{pmatrix},
\qquad
\lambda_{\max} =
4.2460,
\qquad
CR = 0.0928
\end{equation*}

\begin{equation*}
\mathbf{w}^{EM} =
\begin{pmatrix}
0.425832\\
\color{red} 0.399788\color{black} \\
0.091868\\
0.082511
\end{pmatrix}\end{equation*}
\begin{equation*}
\left[ \frac{{w}^{EM}_i}{{w}^{EM}_j} \right] =
\begin{pmatrix}
$\,\,$ 1 $\,\,$ & $\,\,$\color{red} 1.0651\color{black} $\,\,$ & $\,\,$4.6352$\,\,$ & $\,\,$5.1609$\,\,$ \\
$\,\,$\color{red} 0.9388\color{black} $\,\,$ & $\,\,$ 1 $\,\,$ & $\,\,$\color{red} 4.3517\color{black} $\,\,$ & $\,\,$\color{red} 4.8452\color{black}   $\,\,$ \\
$\,\,$0.2157$\,\,$ & $\,\,$\color{red} 0.2298\color{black} $\,\,$ & $\,\,$ 1 $\,\,$ & $\,\,$1.1134 $\,\,$ \\
$\,\,$0.1938$\,\,$ & $\,\,$\color{red} 0.2064\color{black} $\,\,$ & $\,\,$0.8981$\,\,$ & $\,\,$ 1  $\,\,$ \\
\end{pmatrix},
\end{equation*}

\begin{equation*}
\mathbf{w}^{\prime} =
\begin{pmatrix}
0.420463\\
0.407356\\
0.090710\\
0.081471
\end{pmatrix} =
0.987392\cdot
\begin{pmatrix}
0.425832\\
\color{gr} 0.412557\color{black} \\
0.091868\\
0.082511
\end{pmatrix},
\end{equation*}
\begin{equation*}
\left[ \frac{{w}^{\prime}_i}{{w}^{\prime}_j} \right] =
\begin{pmatrix}
$\,\,$ 1 $\,\,$ & $\,\,$\color{gr} 1.0322\color{black} $\,\,$ & $\,\,$4.6352$\,\,$ & $\,\,$5.1609$\,\,$ \\
$\,\,$\color{gr} 0.9688\color{black} $\,\,$ & $\,\,$ 1 $\,\,$ & $\,\,$\color{gr} 4.4907\color{black} $\,\,$ & $\,\,$\color{gr} \color{blue} 5\color{black}   $\,\,$ \\
$\,\,$0.2157$\,\,$ & $\,\,$\color{gr} 0.2227\color{black} $\,\,$ & $\,\,$ 1 $\,\,$ & $\,\,$1.1134 $\,\,$ \\
$\,\,$0.1938$\,\,$ & $\,\,$\color{gr} \color{blue}  1/5\color{black} $\,\,$ & $\,\,$0.8981$\,\,$ & $\,\,$ 1  $\,\,$ \\
\end{pmatrix},
\end{equation*}
\end{example}
\newpage
\begin{example}
\begin{equation*}
\mathbf{A} =
\begin{pmatrix}
$\,\,$ 1 $\,\,$ & $\,\,$1$\,\,$ & $\,\,$8$\,\,$ & $\,\,$3 $\,\,$ \\
$\,\,$ 1 $\,\,$ & $\,\,$ 1 $\,\,$ & $\,\,$5$\,\,$ & $\,\,$6 $\,\,$ \\
$\,\,$ 1/8$\,\,$ & $\,\,$ 1/5$\,\,$ & $\,\,$ 1 $\,\,$ & $\,\,$2 $\,\,$ \\
$\,\,$ 1/3$\,\,$ & $\,\,$ 1/6$\,\,$ & $\,\,$ 1/2$\,\,$ & $\,\,$ 1  $\,\,$ \\
\end{pmatrix},
\qquad
\lambda_{\max} =
4.2460,
\qquad
CR = 0.0928
\end{equation*}

\begin{equation*}
\mathbf{w}^{EM} =
\begin{pmatrix}
0.420183\\
\color{red} 0.412002\color{black} \\
0.089697\\
0.078118
\end{pmatrix}\end{equation*}
\begin{equation*}
\left[ \frac{{w}^{EM}_i}{{w}^{EM}_j} \right] =
\begin{pmatrix}
$\,\,$ 1 $\,\,$ & $\,\,$\color{red} 1.0199\color{black} $\,\,$ & $\,\,$4.6845$\,\,$ & $\,\,$5.3788$\,\,$ \\
$\,\,$\color{red} 0.9805\color{black} $\,\,$ & $\,\,$ 1 $\,\,$ & $\,\,$\color{red} 4.5933\color{black} $\,\,$ & $\,\,$\color{red} 5.2741\color{black}   $\,\,$ \\
$\,\,$0.2135$\,\,$ & $\,\,$\color{red} 0.2177\color{black} $\,\,$ & $\,\,$ 1 $\,\,$ & $\,\,$1.1482 $\,\,$ \\
$\,\,$0.1859$\,\,$ & $\,\,$\color{red} 0.1896\color{black} $\,\,$ & $\,\,$0.8709$\,\,$ & $\,\,$ 1  $\,\,$ \\
\end{pmatrix},
\end{equation*}

\begin{equation*}
\mathbf{w}^{\prime} =
\begin{pmatrix}
0.416773\\
0.416773\\
0.088969\\
0.077485
\end{pmatrix} =
0.991886\cdot
\begin{pmatrix}
0.420183\\
\color{gr} 0.420183\color{black} \\
0.089697\\
0.078118
\end{pmatrix},
\end{equation*}
\begin{equation*}
\left[ \frac{{w}^{\prime}_i}{{w}^{\prime}_j} \right] =
\begin{pmatrix}
$\,\,$ 1 $\,\,$ & $\,\,$\color{gr} \color{blue} 1\color{black} $\,\,$ & $\,\,$4.6845$\,\,$ & $\,\,$5.3788$\,\,$ \\
$\,\,$\color{gr} \color{blue} 1\color{black} $\,\,$ & $\,\,$ 1 $\,\,$ & $\,\,$\color{gr} 4.6845\color{black} $\,\,$ & $\,\,$\color{gr} 5.3788\color{black}   $\,\,$ \\
$\,\,$0.2135$\,\,$ & $\,\,$\color{gr} 0.2135\color{black} $\,\,$ & $\,\,$ 1 $\,\,$ & $\,\,$1.1482 $\,\,$ \\
$\,\,$0.1859$\,\,$ & $\,\,$\color{gr} 0.1859\color{black} $\,\,$ & $\,\,$0.8709$\,\,$ & $\,\,$ 1  $\,\,$ \\
\end{pmatrix},
\end{equation*}
\end{example}
\newpage
\begin{example}
\begin{equation*}
\mathbf{A} =
\begin{pmatrix}
$\,\,$ 1 $\,\,$ & $\,\,$1$\,\,$ & $\,\,$8$\,\,$ & $\,\,$4 $\,\,$ \\
$\,\,$ 1 $\,\,$ & $\,\,$ 1 $\,\,$ & $\,\,$5$\,\,$ & $\,\,$6 $\,\,$ \\
$\,\,$ 1/8$\,\,$ & $\,\,$ 1/5$\,\,$ & $\,\,$ 1 $\,\,$ & $\,\,$2 $\,\,$ \\
$\,\,$ 1/4$\,\,$ & $\,\,$ 1/6$\,\,$ & $\,\,$ 1/2$\,\,$ & $\,\,$ 1  $\,\,$ \\
\end{pmatrix},
\qquad
\lambda_{\max} =
4.1655,
\qquad
CR = 0.0624
\end{equation*}

\begin{equation*}
\mathbf{w}^{EM} =
\begin{pmatrix}
0.436309\\
\color{red} 0.407124\color{black} \\
0.086941\\
0.069625
\end{pmatrix}\end{equation*}
\begin{equation*}
\left[ \frac{{w}^{EM}_i}{{w}^{EM}_j} \right] =
\begin{pmatrix}
$\,\,$ 1 $\,\,$ & $\,\,$\color{red} 1.0717\color{black} $\,\,$ & $\,\,$5.0185$\,\,$ & $\,\,$6.2665$\,\,$ \\
$\,\,$\color{red} 0.9331\color{black} $\,\,$ & $\,\,$ 1 $\,\,$ & $\,\,$\color{red} 4.6828\color{black} $\,\,$ & $\,\,$\color{red} 5.8473\color{black}   $\,\,$ \\
$\,\,$0.1993$\,\,$ & $\,\,$\color{red} 0.2135\color{black} $\,\,$ & $\,\,$ 1 $\,\,$ & $\,\,$1.2487 $\,\,$ \\
$\,\,$0.1596$\,\,$ & $\,\,$\color{red} 0.1710\color{black} $\,\,$ & $\,\,$0.8008$\,\,$ & $\,\,$ 1  $\,\,$ \\
\end{pmatrix},
\end{equation*}

\begin{equation*}
\mathbf{w}^{\prime} =
\begin{pmatrix}
0.431721\\
0.413359\\
0.086027\\
0.068893
\end{pmatrix} =
0.989483\cdot
\begin{pmatrix}
0.436309\\
\color{gr} 0.417753\color{black} \\
0.086941\\
0.069625
\end{pmatrix},
\end{equation*}
\begin{equation*}
\left[ \frac{{w}^{\prime}_i}{{w}^{\prime}_j} \right] =
\begin{pmatrix}
$\,\,$ 1 $\,\,$ & $\,\,$\color{gr} 1.0444\color{black} $\,\,$ & $\,\,$5.0185$\,\,$ & $\,\,$6.2665$\,\,$ \\
$\,\,$\color{gr} 0.9575\color{black} $\,\,$ & $\,\,$ 1 $\,\,$ & $\,\,$\color{gr} 4.8050\color{black} $\,\,$ & $\,\,$\color{gr} \color{blue} 6\color{black}   $\,\,$ \\
$\,\,$0.1993$\,\,$ & $\,\,$\color{gr} 0.2081\color{black} $\,\,$ & $\,\,$ 1 $\,\,$ & $\,\,$1.2487 $\,\,$ \\
$\,\,$0.1596$\,\,$ & $\,\,$\color{gr} \color{blue}  1/6\color{black} $\,\,$ & $\,\,$0.8008$\,\,$ & $\,\,$ 1  $\,\,$ \\
\end{pmatrix},
\end{equation*}
\end{example}
\newpage
\begin{example}
\begin{equation*}
\mathbf{A} =
\begin{pmatrix}
$\,\,$ 1 $\,\,$ & $\,\,$1$\,\,$ & $\,\,$8$\,\,$ & $\,\,$4 $\,\,$ \\
$\,\,$ 1 $\,\,$ & $\,\,$ 1 $\,\,$ & $\,\,$5$\,\,$ & $\,\,$7 $\,\,$ \\
$\,\,$ 1/8$\,\,$ & $\,\,$ 1/5$\,\,$ & $\,\,$ 1 $\,\,$ & $\,\,$2 $\,\,$ \\
$\,\,$ 1/4$\,\,$ & $\,\,$ 1/7$\,\,$ & $\,\,$ 1/2$\,\,$ & $\,\,$ 1  $\,\,$ \\
\end{pmatrix},
\qquad
\lambda_{\max} =
4.1665,
\qquad
CR = 0.0628
\end{equation*}

\begin{equation*}
\mathbf{w}^{EM} =
\begin{pmatrix}
0.431023\\
\color{red} 0.417383\color{black} \\
0.085269\\
0.066325
\end{pmatrix}\end{equation*}
\begin{equation*}
\left[ \frac{{w}^{EM}_i}{{w}^{EM}_j} \right] =
\begin{pmatrix}
$\,\,$ 1 $\,\,$ & $\,\,$\color{red} 1.0327\color{black} $\,\,$ & $\,\,$5.0549$\,\,$ & $\,\,$6.4987$\,\,$ \\
$\,\,$\color{red} 0.9684\color{black} $\,\,$ & $\,\,$ 1 $\,\,$ & $\,\,$\color{red} 4.8949\color{black} $\,\,$ & $\,\,$\color{red} 6.2930\color{black}   $\,\,$ \\
$\,\,$0.1978$\,\,$ & $\,\,$\color{red} 0.2043\color{black} $\,\,$ & $\,\,$ 1 $\,\,$ & $\,\,$1.2856 $\,\,$ \\
$\,\,$0.1539$\,\,$ & $\,\,$\color{red} 0.1589\color{black} $\,\,$ & $\,\,$0.7778$\,\,$ & $\,\,$ 1  $\,\,$ \\
\end{pmatrix},
\end{equation*}

\begin{equation*}
\mathbf{w}^{\prime} =
\begin{pmatrix}
0.427195\\
0.422558\\
0.084512\\
0.065735
\end{pmatrix} =
0.991118\cdot
\begin{pmatrix}
0.431023\\
\color{gr} 0.426345\color{black} \\
0.085269\\
0.066325
\end{pmatrix},
\end{equation*}
\begin{equation*}
\left[ \frac{{w}^{\prime}_i}{{w}^{\prime}_j} \right] =
\begin{pmatrix}
$\,\,$ 1 $\,\,$ & $\,\,$\color{gr} 1.0110\color{black} $\,\,$ & $\,\,$5.0549$\,\,$ & $\,\,$6.4987$\,\,$ \\
$\,\,$\color{gr} 0.9891\color{black} $\,\,$ & $\,\,$ 1 $\,\,$ & $\,\,$\color{gr} \color{blue} 5\color{black} $\,\,$ & $\,\,$\color{gr} 6.4282\color{black}   $\,\,$ \\
$\,\,$0.1978$\,\,$ & $\,\,$\color{gr} \color{blue}  1/5\color{black} $\,\,$ & $\,\,$ 1 $\,\,$ & $\,\,$1.2856 $\,\,$ \\
$\,\,$0.1539$\,\,$ & $\,\,$\color{gr} 0.1556\color{black} $\,\,$ & $\,\,$0.7778$\,\,$ & $\,\,$ 1  $\,\,$ \\
\end{pmatrix},
\end{equation*}
\end{example}
\newpage
\begin{example}
\begin{equation*}
\mathbf{A} =
\begin{pmatrix}
$\,\,$ 1 $\,\,$ & $\,\,$1$\,\,$ & $\,\,$8$\,\,$ & $\,\,$5 $\,\,$ \\
$\,\,$ 1 $\,\,$ & $\,\,$ 1 $\,\,$ & $\,\,$5$\,\,$ & $\,\,$9 $\,\,$ \\
$\,\,$ 1/8$\,\,$ & $\,\,$ 1/5$\,\,$ & $\,\,$ 1 $\,\,$ & $\,\,$3 $\,\,$ \\
$\,\,$ 1/5$\,\,$ & $\,\,$ 1/9$\,\,$ & $\,\,$ 1/3$\,\,$ & $\,\,$ 1  $\,\,$ \\
\end{pmatrix},
\qquad
\lambda_{\max} =
4.2138,
\qquad
CR = 0.0806
\end{equation*}

\begin{equation*}
\mathbf{w}^{EM} =
\begin{pmatrix}
0.437016\\
\color{red} 0.420838\color{black} \\
0.090965\\
0.051181
\end{pmatrix}\end{equation*}
\begin{equation*}
\left[ \frac{{w}^{EM}_i}{{w}^{EM}_j} \right] =
\begin{pmatrix}
$\,\,$ 1 $\,\,$ & $\,\,$\color{red} 1.0384\color{black} $\,\,$ & $\,\,$4.8042$\,\,$ & $\,\,$8.5386$\,\,$ \\
$\,\,$\color{red} 0.9630\color{black} $\,\,$ & $\,\,$ 1 $\,\,$ & $\,\,$\color{red} 4.6264\color{black} $\,\,$ & $\,\,$\color{red} 8.2225\color{black}   $\,\,$ \\
$\,\,$0.2081$\,\,$ & $\,\,$\color{red} 0.2162\color{black} $\,\,$ & $\,\,$ 1 $\,\,$ & $\,\,$1.7773 $\,\,$ \\
$\,\,$0.1171$\,\,$ & $\,\,$\color{red} 0.1216\color{black} $\,\,$ & $\,\,$0.5627$\,\,$ & $\,\,$ 1  $\,\,$ \\
\end{pmatrix},
\end{equation*}

\begin{equation*}
\mathbf{w}^{\prime} =
\begin{pmatrix}
0.430058\\
0.430058\\
0.089517\\
0.050367
\end{pmatrix} =
0.984080\cdot
\begin{pmatrix}
0.437016\\
\color{gr} 0.437016\color{black} \\
0.090965\\
0.051181
\end{pmatrix},
\end{equation*}
\begin{equation*}
\left[ \frac{{w}^{\prime}_i}{{w}^{\prime}_j} \right] =
\begin{pmatrix}
$\,\,$ 1 $\,\,$ & $\,\,$\color{gr} \color{blue} 1\color{black} $\,\,$ & $\,\,$4.8042$\,\,$ & $\,\,$8.5386$\,\,$ \\
$\,\,$\color{gr} \color{blue} 1\color{black} $\,\,$ & $\,\,$ 1 $\,\,$ & $\,\,$\color{gr} 4.8042\color{black} $\,\,$ & $\,\,$\color{gr} 8.5386\color{black}   $\,\,$ \\
$\,\,$0.2081$\,\,$ & $\,\,$\color{gr} 0.2081\color{black} $\,\,$ & $\,\,$ 1 $\,\,$ & $\,\,$1.7773 $\,\,$ \\
$\,\,$0.1171$\,\,$ & $\,\,$\color{gr} 0.1171\color{black} $\,\,$ & $\,\,$0.5627$\,\,$ & $\,\,$ 1  $\,\,$ \\
\end{pmatrix},
\end{equation*}
\end{example}
\newpage
\begin{example}
\begin{equation*}
\mathbf{A} =
\begin{pmatrix}
$\,\,$ 1 $\,\,$ & $\,\,$1$\,\,$ & $\,\,$9$\,\,$ & $\,\,$2 $\,\,$ \\
$\,\,$ 1 $\,\,$ & $\,\,$ 1 $\,\,$ & $\,\,$5$\,\,$ & $\,\,$3 $\,\,$ \\
$\,\,$ 1/9$\,\,$ & $\,\,$ 1/5$\,\,$ & $\,\,$ 1 $\,\,$ & $\,\,$1 $\,\,$ \\
$\,\,$ 1/2$\,\,$ & $\,\,$ 1/3$\,\,$ & $\,\,$ 1 $\,\,$ & $\,\,$ 1  $\,\,$ \\
\end{pmatrix},
\qquad
\lambda_{\max} =
4.1966,
\qquad
CR = 0.0741
\end{equation*}

\begin{equation*}
\mathbf{w}^{EM} =
\begin{pmatrix}
0.418313\\
\color{red} 0.374401\color{black} \\
0.078313\\
0.128973
\end{pmatrix}\end{equation*}
\begin{equation*}
\left[ \frac{{w}^{EM}_i}{{w}^{EM}_j} \right] =
\begin{pmatrix}
$\,\,$ 1 $\,\,$ & $\,\,$\color{red} 1.1173\color{black} $\,\,$ & $\,\,$5.3416$\,\,$ & $\,\,$3.2434$\,\,$ \\
$\,\,$\color{red} 0.8950\color{black} $\,\,$ & $\,\,$ 1 $\,\,$ & $\,\,$\color{red} 4.7808\color{black} $\,\,$ & $\,\,$\color{red} 2.9029\color{black}   $\,\,$ \\
$\,\,$0.1872$\,\,$ & $\,\,$\color{red} 0.2092\color{black} $\,\,$ & $\,\,$ 1 $\,\,$ & $\,\,$0.6072 $\,\,$ \\
$\,\,$0.3083$\,\,$ & $\,\,$\color{red} 0.3445\color{black} $\,\,$ & $\,\,$1.6469$\,\,$ & $\,\,$ 1  $\,\,$ \\
\end{pmatrix},
\end{equation*}

\begin{equation*}
\mathbf{w}^{\prime} =
\begin{pmatrix}
0.413142\\
0.382135\\
0.077345\\
0.127378
\end{pmatrix} =
0.987637\cdot
\begin{pmatrix}
0.418313\\
\color{gr} 0.386919\color{black} \\
0.078313\\
0.128973
\end{pmatrix},
\end{equation*}
\begin{equation*}
\left[ \frac{{w}^{\prime}_i}{{w}^{\prime}_j} \right] =
\begin{pmatrix}
$\,\,$ 1 $\,\,$ & $\,\,$\color{gr} 1.0811\color{black} $\,\,$ & $\,\,$5.3416$\,\,$ & $\,\,$3.2434$\,\,$ \\
$\,\,$\color{gr} 0.9250\color{black} $\,\,$ & $\,\,$ 1 $\,\,$ & $\,\,$\color{gr} 4.9407\color{black} $\,\,$ & $\,\,$\color{gr} \color{blue} 3\color{black}   $\,\,$ \\
$\,\,$0.1872$\,\,$ & $\,\,$\color{gr} 0.2024\color{black} $\,\,$ & $\,\,$ 1 $\,\,$ & $\,\,$0.6072 $\,\,$ \\
$\,\,$0.3083$\,\,$ & $\,\,$\color{gr} \color{blue}  1/3\color{black} $\,\,$ & $\,\,$1.6469$\,\,$ & $\,\,$ 1  $\,\,$ \\
\end{pmatrix},
\end{equation*}
\end{example}
\newpage
\begin{example}
\begin{equation*}
\mathbf{A} =
\begin{pmatrix}
$\,\,$ 1 $\,\,$ & $\,\,$1$\,\,$ & $\,\,$9$\,\,$ & $\,\,$3 $\,\,$ \\
$\,\,$ 1 $\,\,$ & $\,\,$ 1 $\,\,$ & $\,\,$6$\,\,$ & $\,\,$4 $\,\,$ \\
$\,\,$ 1/9$\,\,$ & $\,\,$ 1/6$\,\,$ & $\,\,$ 1 $\,\,$ & $\,\,$1 $\,\,$ \\
$\,\,$ 1/3$\,\,$ & $\,\,$ 1/4$\,\,$ & $\,\,$ 1 $\,\,$ & $\,\,$ 1  $\,\,$ \\
\end{pmatrix},
\qquad
\lambda_{\max} =
4.1031,
\qquad
CR = 0.0389
\end{equation*}

\begin{equation*}
\mathbf{w}^{EM} =
\begin{pmatrix}
0.428048\\
\color{red} 0.401877\color{black} \\
0.069365\\
0.100710
\end{pmatrix}\end{equation*}
\begin{equation*}
\left[ \frac{{w}^{EM}_i}{{w}^{EM}_j} \right] =
\begin{pmatrix}
$\,\,$ 1 $\,\,$ & $\,\,$\color{red} 1.0651\color{black} $\,\,$ & $\,\,$6.1709$\,\,$ & $\,\,$4.2503$\,\,$ \\
$\,\,$\color{red} 0.9389\color{black} $\,\,$ & $\,\,$ 1 $\,\,$ & $\,\,$\color{red} 5.7936\color{black} $\,\,$ & $\,\,$\color{red} 3.9904\color{black}   $\,\,$ \\
$\,\,$0.1621$\,\,$ & $\,\,$\color{red} 0.1726\color{black} $\,\,$ & $\,\,$ 1 $\,\,$ & $\,\,$0.6888 $\,\,$ \\
$\,\,$0.2353$\,\,$ & $\,\,$\color{red} 0.2506\color{black} $\,\,$ & $\,\,$1.4519$\,\,$ & $\,\,$ 1  $\,\,$ \\
\end{pmatrix},
\end{equation*}

\begin{equation*}
\mathbf{w}^{\prime} =
\begin{pmatrix}
0.427636\\
0.402452\\
0.069299\\
0.100613
\end{pmatrix} =
0.999038\cdot
\begin{pmatrix}
0.428048\\
\color{gr} 0.402840\color{black} \\
0.069365\\
0.100710
\end{pmatrix},
\end{equation*}
\begin{equation*}
\left[ \frac{{w}^{\prime}_i}{{w}^{\prime}_j} \right] =
\begin{pmatrix}
$\,\,$ 1 $\,\,$ & $\,\,$\color{gr} 1.0626\color{black} $\,\,$ & $\,\,$6.1709$\,\,$ & $\,\,$4.2503$\,\,$ \\
$\,\,$\color{gr} 0.9411\color{black} $\,\,$ & $\,\,$ 1 $\,\,$ & $\,\,$\color{gr} 5.8075\color{black} $\,\,$ & $\,\,$\color{gr} \color{blue} 4\color{black}   $\,\,$ \\
$\,\,$0.1621$\,\,$ & $\,\,$\color{gr} 0.1722\color{black} $\,\,$ & $\,\,$ 1 $\,\,$ & $\,\,$0.6888 $\,\,$ \\
$\,\,$0.2353$\,\,$ & $\,\,$\color{gr} \color{blue}  1/4\color{black} $\,\,$ & $\,\,$1.4519$\,\,$ & $\,\,$ 1  $\,\,$ \\
\end{pmatrix},
\end{equation*}
\end{example}
\newpage
\begin{example}
\begin{equation*}
\mathbf{A} =
\begin{pmatrix}
$\,\,$ 1 $\,\,$ & $\,\,$1$\,\,$ & $\,\,$9$\,\,$ & $\,\,$4 $\,\,$ \\
$\,\,$ 1 $\,\,$ & $\,\,$ 1 $\,\,$ & $\,\,$5$\,\,$ & $\,\,$6 $\,\,$ \\
$\,\,$ 1/9$\,\,$ & $\,\,$ 1/5$\,\,$ & $\,\,$ 1 $\,\,$ & $\,\,$2 $\,\,$ \\
$\,\,$ 1/4$\,\,$ & $\,\,$ 1/6$\,\,$ & $\,\,$ 1/2$\,\,$ & $\,\,$ 1  $\,\,$ \\
\end{pmatrix},
\qquad
\lambda_{\max} =
4.1966,
\qquad
CR = 0.0741
\end{equation*}

\begin{equation*}
\mathbf{w}^{EM} =
\begin{pmatrix}
0.447148\\
\color{red} 0.400209\color{black} \\
0.083711\\
0.068932
\end{pmatrix}\end{equation*}
\begin{equation*}
\left[ \frac{{w}^{EM}_i}{{w}^{EM}_j} \right] =
\begin{pmatrix}
$\,\,$ 1 $\,\,$ & $\,\,$\color{red} 1.1173\color{black} $\,\,$ & $\,\,$5.3416$\,\,$ & $\,\,$6.4868$\,\,$ \\
$\,\,$\color{red} 0.8950\color{black} $\,\,$ & $\,\,$ 1 $\,\,$ & $\,\,$\color{red} 4.7808\color{black} $\,\,$ & $\,\,$\color{red} 5.8059\color{black}   $\,\,$ \\
$\,\,$0.1872$\,\,$ & $\,\,$\color{red} 0.2092\color{black} $\,\,$ & $\,\,$ 1 $\,\,$ & $\,\,$1.2144 $\,\,$ \\
$\,\,$0.1542$\,\,$ & $\,\,$\color{red} 0.1722\color{black} $\,\,$ & $\,\,$0.8234$\,\,$ & $\,\,$ 1  $\,\,$ \\
\end{pmatrix},
\end{equation*}

\begin{equation*}
\mathbf{w}^{\prime} =
\begin{pmatrix}
0.441244\\
0.408129\\
0.082606\\
0.068021
\end{pmatrix} =
0.986796\cdot
\begin{pmatrix}
0.447148\\
\color{gr} 0.413590\color{black} \\
0.083711\\
0.068932
\end{pmatrix},
\end{equation*}
\begin{equation*}
\left[ \frac{{w}^{\prime}_i}{{w}^{\prime}_j} \right] =
\begin{pmatrix}
$\,\,$ 1 $\,\,$ & $\,\,$\color{gr} 1.0811\color{black} $\,\,$ & $\,\,$5.3416$\,\,$ & $\,\,$6.4868$\,\,$ \\
$\,\,$\color{gr} 0.9250\color{black} $\,\,$ & $\,\,$ 1 $\,\,$ & $\,\,$\color{gr} 4.9407\color{black} $\,\,$ & $\,\,$\color{gr} \color{blue} 6\color{black}   $\,\,$ \\
$\,\,$0.1872$\,\,$ & $\,\,$\color{gr} 0.2024\color{black} $\,\,$ & $\,\,$ 1 $\,\,$ & $\,\,$1.2144 $\,\,$ \\
$\,\,$0.1542$\,\,$ & $\,\,$\color{gr} \color{blue}  1/6\color{black} $\,\,$ & $\,\,$0.8234$\,\,$ & $\,\,$ 1  $\,\,$ \\
\end{pmatrix},
\end{equation*}
\end{example}
\newpage
\begin{example}
\begin{equation*}
\mathbf{A} =
\begin{pmatrix}
$\,\,$ 1 $\,\,$ & $\,\,$1$\,\,$ & $\,\,$9$\,\,$ & $\,\,$4 $\,\,$ \\
$\,\,$ 1 $\,\,$ & $\,\,$ 1 $\,\,$ & $\,\,$5$\,\,$ & $\,\,$7 $\,\,$ \\
$\,\,$ 1/9$\,\,$ & $\,\,$ 1/5$\,\,$ & $\,\,$ 1 $\,\,$ & $\,\,$2 $\,\,$ \\
$\,\,$ 1/4$\,\,$ & $\,\,$ 1/7$\,\,$ & $\,\,$ 1/2$\,\,$ & $\,\,$ 1  $\,\,$ \\
\end{pmatrix},
\qquad
\lambda_{\max} =
4.1975,
\qquad
CR = 0.0745
\end{equation*}

\begin{equation*}
\mathbf{w}^{EM} =
\begin{pmatrix}
0.441726\\
\color{red} 0.410432\color{black} \\
0.082126\\
0.065716
\end{pmatrix}\end{equation*}
\begin{equation*}
\left[ \frac{{w}^{EM}_i}{{w}^{EM}_j} \right] =
\begin{pmatrix}
$\,\,$ 1 $\,\,$ & $\,\,$\color{red} 1.0762\color{black} $\,\,$ & $\,\,$5.3787$\,\,$ & $\,\,$6.7218$\,\,$ \\
$\,\,$\color{red} 0.9292\color{black} $\,\,$ & $\,\,$ 1 $\,\,$ & $\,\,$\color{red} 4.9976\color{black} $\,\,$ & $\,\,$\color{red} 6.2456\color{black}   $\,\,$ \\
$\,\,$0.1859$\,\,$ & $\,\,$\color{red} 0.2001\color{black} $\,\,$ & $\,\,$ 1 $\,\,$ & $\,\,$1.2497 $\,\,$ \\
$\,\,$0.1488$\,\,$ & $\,\,$\color{red} 0.1601\color{black} $\,\,$ & $\,\,$0.8002$\,\,$ & $\,\,$ 1  $\,\,$ \\
\end{pmatrix},
\end{equation*}

\begin{equation*}
\mathbf{w}^{\prime} =
\begin{pmatrix}
0.441639\\
0.410548\\
0.082110\\
0.065703
\end{pmatrix} =
0.999804\cdot
\begin{pmatrix}
0.441726\\
\color{gr} 0.410629\color{black} \\
0.082126\\
0.065716
\end{pmatrix},
\end{equation*}
\begin{equation*}
\left[ \frac{{w}^{\prime}_i}{{w}^{\prime}_j} \right] =
\begin{pmatrix}
$\,\,$ 1 $\,\,$ & $\,\,$\color{gr} 1.0757\color{black} $\,\,$ & $\,\,$5.3787$\,\,$ & $\,\,$6.7218$\,\,$ \\
$\,\,$\color{gr} 0.9296\color{black} $\,\,$ & $\,\,$ 1 $\,\,$ & $\,\,$\color{gr} \color{blue} 5\color{black} $\,\,$ & $\,\,$\color{gr} 6.2486\color{black}   $\,\,$ \\
$\,\,$0.1859$\,\,$ & $\,\,$\color{gr} \color{blue}  1/5\color{black} $\,\,$ & $\,\,$ 1 $\,\,$ & $\,\,$1.2497 $\,\,$ \\
$\,\,$0.1488$\,\,$ & $\,\,$\color{gr} 0.1600\color{black} $\,\,$ & $\,\,$0.8002$\,\,$ & $\,\,$ 1  $\,\,$ \\
\end{pmatrix},
\end{equation*}
\end{example}
\newpage
\begin{example}
\begin{equation*}
\mathbf{A} =
\begin{pmatrix}
$\,\,$ 1 $\,\,$ & $\,\,$1$\,\,$ & $\,\,$9$\,\,$ & $\,\,$4 $\,\,$ \\
$\,\,$ 1 $\,\,$ & $\,\,$ 1 $\,\,$ & $\,\,$6$\,\,$ & $\,\,$7 $\,\,$ \\
$\,\,$ 1/9$\,\,$ & $\,\,$ 1/6$\,\,$ & $\,\,$ 1 $\,\,$ & $\,\,$2 $\,\,$ \\
$\,\,$ 1/4$\,\,$ & $\,\,$ 1/7$\,\,$ & $\,\,$ 1/2$\,\,$ & $\,\,$ 1  $\,\,$ \\
\end{pmatrix},
\qquad
\lambda_{\max} =
4.1964,
\qquad
CR = 0.0741
\end{equation*}

\begin{equation*}
\mathbf{w}^{EM} =
\begin{pmatrix}
0.433165\\
\color{red} 0.423990\color{black} \\
0.077839\\
0.065005
\end{pmatrix}\end{equation*}
\begin{equation*}
\left[ \frac{{w}^{EM}_i}{{w}^{EM}_j} \right] =
\begin{pmatrix}
$\,\,$ 1 $\,\,$ & $\,\,$\color{red} 1.0216\color{black} $\,\,$ & $\,\,$5.5649$\,\,$ & $\,\,$6.6636$\,\,$ \\
$\,\,$\color{red} 0.9788\color{black} $\,\,$ & $\,\,$ 1 $\,\,$ & $\,\,$\color{red} 5.4470\color{black} $\,\,$ & $\,\,$\color{red} 6.5224\color{black}   $\,\,$ \\
$\,\,$0.1797$\,\,$ & $\,\,$\color{red} 0.1836\color{black} $\,\,$ & $\,\,$ 1 $\,\,$ & $\,\,$1.1974 $\,\,$ \\
$\,\,$0.1501$\,\,$ & $\,\,$\color{red} 0.1533\color{black} $\,\,$ & $\,\,$0.8351$\,\,$ & $\,\,$ 1  $\,\,$ \\
\end{pmatrix},
\end{equation*}

\begin{equation*}
\mathbf{w}^{\prime} =
\begin{pmatrix}
0.429227\\
0.429227\\
0.077132\\
0.064414
\end{pmatrix} =
0.990908\cdot
\begin{pmatrix}
0.433165\\
\color{gr} 0.433165\color{black} \\
0.077839\\
0.065005
\end{pmatrix},
\end{equation*}
\begin{equation*}
\left[ \frac{{w}^{\prime}_i}{{w}^{\prime}_j} \right] =
\begin{pmatrix}
$\,\,$ 1 $\,\,$ & $\,\,$\color{gr} \color{blue} 1\color{black} $\,\,$ & $\,\,$5.5649$\,\,$ & $\,\,$6.6636$\,\,$ \\
$\,\,$\color{gr} \color{blue} 1\color{black} $\,\,$ & $\,\,$ 1 $\,\,$ & $\,\,$\color{gr} 5.5649\color{black} $\,\,$ & $\,\,$\color{gr} 6.6636\color{black}   $\,\,$ \\
$\,\,$0.1797$\,\,$ & $\,\,$\color{gr} 0.1797\color{black} $\,\,$ & $\,\,$ 1 $\,\,$ & $\,\,$1.1974 $\,\,$ \\
$\,\,$0.1501$\,\,$ & $\,\,$\color{gr} 0.1501\color{black} $\,\,$ & $\,\,$0.8351$\,\,$ & $\,\,$ 1  $\,\,$ \\
\end{pmatrix},
\end{equation*}
\end{example}
\newpage
\begin{example}
\begin{equation*}
\mathbf{A} =
\begin{pmatrix}
$\,\,$ 1 $\,\,$ & $\,\,$1$\,\,$ & $\,\,$9$\,\,$ & $\,\,$5 $\,\,$ \\
$\,\,$ 1 $\,\,$ & $\,\,$ 1 $\,\,$ & $\,\,$6$\,\,$ & $\,\,$8 $\,\,$ \\
$\,\,$ 1/9$\,\,$ & $\,\,$ 1/6$\,\,$ & $\,\,$ 1 $\,\,$ & $\,\,$2 $\,\,$ \\
$\,\,$ 1/5$\,\,$ & $\,\,$ 1/8$\,\,$ & $\,\,$ 1/2$\,\,$ & $\,\,$ 1  $\,\,$ \\
\end{pmatrix},
\qquad
\lambda_{\max} =
4.1406,
\qquad
CR = 0.0530
\end{equation*}

\begin{equation*}
\mathbf{w}^{EM} =
\begin{pmatrix}
0.440614\\
\color{red} 0.427856\color{black} \\
0.074569\\
0.056961
\end{pmatrix}\end{equation*}
\begin{equation*}
\left[ \frac{{w}^{EM}_i}{{w}^{EM}_j} \right] =
\begin{pmatrix}
$\,\,$ 1 $\,\,$ & $\,\,$\color{red} 1.0298\color{black} $\,\,$ & $\,\,$5.9088$\,\,$ & $\,\,$7.7354$\,\,$ \\
$\,\,$\color{red} 0.9710\color{black} $\,\,$ & $\,\,$ 1 $\,\,$ & $\,\,$\color{red} 5.7377\color{black} $\,\,$ & $\,\,$\color{red} 7.5114\color{black}   $\,\,$ \\
$\,\,$0.1692$\,\,$ & $\,\,$\color{red} 0.1743\color{black} $\,\,$ & $\,\,$ 1 $\,\,$ & $\,\,$1.3091 $\,\,$ \\
$\,\,$0.1293$\,\,$ & $\,\,$\color{red} 0.1331\color{black} $\,\,$ & $\,\,$0.7639$\,\,$ & $\,\,$ 1  $\,\,$ \\
\end{pmatrix},
\end{equation*}

\begin{equation*}
\mathbf{w}^{\prime} =
\begin{pmatrix}
0.435064\\
0.435064\\
0.073629\\
0.056243
\end{pmatrix} =
0.987403\cdot
\begin{pmatrix}
0.440614\\
\color{gr} 0.440614\color{black} \\
0.074569\\
0.056961
\end{pmatrix},
\end{equation*}
\begin{equation*}
\left[ \frac{{w}^{\prime}_i}{{w}^{\prime}_j} \right] =
\begin{pmatrix}
$\,\,$ 1 $\,\,$ & $\,\,$\color{gr} \color{blue} 1\color{black} $\,\,$ & $\,\,$5.9088$\,\,$ & $\,\,$7.7354$\,\,$ \\
$\,\,$\color{gr} \color{blue} 1\color{black} $\,\,$ & $\,\,$ 1 $\,\,$ & $\,\,$\color{gr} 5.9088\color{black} $\,\,$ & $\,\,$\color{gr} 7.7354\color{black}   $\,\,$ \\
$\,\,$0.1692$\,\,$ & $\,\,$\color{gr} 0.1692\color{black} $\,\,$ & $\,\,$ 1 $\,\,$ & $\,\,$1.3091 $\,\,$ \\
$\,\,$0.1293$\,\,$ & $\,\,$\color{gr} 0.1293\color{black} $\,\,$ & $\,\,$0.7639$\,\,$ & $\,\,$ 1  $\,\,$ \\
\end{pmatrix},
\end{equation*}
\end{example}
\newpage
\begin{example}
\begin{equation*}
\mathbf{A} =
\begin{pmatrix}
$\,\,$ 1 $\,\,$ & $\,\,$1$\,\,$ & $\,\,$9$\,\,$ & $\,\,$5 $\,\,$ \\
$\,\,$ 1 $\,\,$ & $\,\,$ 1 $\,\,$ & $\,\,$6$\,\,$ & $\,\,$9 $\,\,$ \\
$\,\,$ 1/9$\,\,$ & $\,\,$ 1/6$\,\,$ & $\,\,$ 1 $\,\,$ & $\,\,$2 $\,\,$ \\
$\,\,$ 1/5$\,\,$ & $\,\,$ 1/9$\,\,$ & $\,\,$ 1/2$\,\,$ & $\,\,$ 1  $\,\,$ \\
\end{pmatrix},
\qquad
\lambda_{\max} =
4.1433,
\qquad
CR = 0.0540
\end{equation*}

\begin{equation*}
\mathbf{w}^{EM} =
\begin{pmatrix}
0.436000\\
\color{red} 0.435779\color{black} \\
0.073399\\
0.054821
\end{pmatrix}\end{equation*}
\begin{equation*}
\left[ \frac{{w}^{EM}_i}{{w}^{EM}_j} \right] =
\begin{pmatrix}
$\,\,$ 1 $\,\,$ & $\,\,$\color{red} 1.0005\color{black} $\,\,$ & $\,\,$5.9401$\,\,$ & $\,\,$7.9531$\,\,$ \\
$\,\,$\color{red} 0.9995\color{black} $\,\,$ & $\,\,$ 1 $\,\,$ & $\,\,$\color{red} 5.9371\color{black} $\,\,$ & $\,\,$\color{red} 7.9491\color{black}   $\,\,$ \\
$\,\,$0.1683$\,\,$ & $\,\,$\color{red} 0.1684\color{black} $\,\,$ & $\,\,$ 1 $\,\,$ & $\,\,$1.3389 $\,\,$ \\
$\,\,$0.1257$\,\,$ & $\,\,$\color{red} 0.1258\color{black} $\,\,$ & $\,\,$0.7469$\,\,$ & $\,\,$ 1  $\,\,$ \\
\end{pmatrix},
\end{equation*}

\begin{equation*}
\mathbf{w}^{\prime} =
\begin{pmatrix}
0.435904\\
0.435904\\
0.073383\\
0.054809
\end{pmatrix} =
0.999780\cdot
\begin{pmatrix}
0.436000\\
\color{gr} 0.436000\color{black} \\
0.073399\\
0.054821
\end{pmatrix},
\end{equation*}
\begin{equation*}
\left[ \frac{{w}^{\prime}_i}{{w}^{\prime}_j} \right] =
\begin{pmatrix}
$\,\,$ 1 $\,\,$ & $\,\,$\color{gr} \color{blue} 1\color{black} $\,\,$ & $\,\,$5.9401$\,\,$ & $\,\,$7.9531$\,\,$ \\
$\,\,$\color{gr} \color{blue} 1\color{black} $\,\,$ & $\,\,$ 1 $\,\,$ & $\,\,$\color{gr} 5.9401\color{black} $\,\,$ & $\,\,$\color{gr} 7.9531\color{black}   $\,\,$ \\
$\,\,$0.1683$\,\,$ & $\,\,$\color{gr} 0.1683\color{black} $\,\,$ & $\,\,$ 1 $\,\,$ & $\,\,$1.3389 $\,\,$ \\
$\,\,$0.1257$\,\,$ & $\,\,$\color{gr} 0.1257\color{black} $\,\,$ & $\,\,$0.7469$\,\,$ & $\,\,$ 1  $\,\,$ \\
\end{pmatrix},
\end{equation*}
\end{example}
\newpage
\begin{example}
\begin{equation*}
\mathbf{A} =
\begin{pmatrix}
$\,\,$ 1 $\,\,$ & $\,\,$1$\,\,$ & $\,\,$9$\,\,$ & $\,\,$5 $\,\,$ \\
$\,\,$ 1 $\,\,$ & $\,\,$ 1 $\,\,$ & $\,\,$6$\,\,$ & $\,\,$9 $\,\,$ \\
$\,\,$ 1/9$\,\,$ & $\,\,$ 1/6$\,\,$ & $\,\,$ 1 $\,\,$ & $\,\,$3 $\,\,$ \\
$\,\,$ 1/5$\,\,$ & $\,\,$ 1/9$\,\,$ & $\,\,$ 1/3$\,\,$ & $\,\,$ 1  $\,\,$ \\
\end{pmatrix},
\qquad
\lambda_{\max} =
4.2507,
\qquad
CR = 0.0946
\end{equation*}

\begin{equation*}
\mathbf{w}^{EM} =
\begin{pmatrix}
0.439075\\
\color{red} 0.427544\color{black} \\
0.083220\\
0.050161
\end{pmatrix}\end{equation*}
\begin{equation*}
\left[ \frac{{w}^{EM}_i}{{w}^{EM}_j} \right] =
\begin{pmatrix}
$\,\,$ 1 $\,\,$ & $\,\,$\color{red} 1.0270\color{black} $\,\,$ & $\,\,$5.2761$\,\,$ & $\,\,$8.7534$\,\,$ \\
$\,\,$\color{red} 0.9737\color{black} $\,\,$ & $\,\,$ 1 $\,\,$ & $\,\,$\color{red} 5.1375\color{black} $\,\,$ & $\,\,$\color{red} 8.5235\color{black}   $\,\,$ \\
$\,\,$0.1895$\,\,$ & $\,\,$\color{red} 0.1946\color{black} $\,\,$ & $\,\,$ 1 $\,\,$ & $\,\,$1.6591 $\,\,$ \\
$\,\,$0.1142$\,\,$ & $\,\,$\color{red} 0.1173\color{black} $\,\,$ & $\,\,$0.6028$\,\,$ & $\,\,$ 1  $\,\,$ \\
\end{pmatrix},
\end{equation*}

\begin{equation*}
\mathbf{w}^{\prime} =
\begin{pmatrix}
0.434070\\
0.434070\\
0.082271\\
0.049589
\end{pmatrix} =
0.988600\cdot
\begin{pmatrix}
0.439075\\
\color{gr} 0.439075\color{black} \\
0.083220\\
0.050161
\end{pmatrix},
\end{equation*}
\begin{equation*}
\left[ \frac{{w}^{\prime}_i}{{w}^{\prime}_j} \right] =
\begin{pmatrix}
$\,\,$ 1 $\,\,$ & $\,\,$\color{gr} \color{blue} 1\color{black} $\,\,$ & $\,\,$5.2761$\,\,$ & $\,\,$8.7534$\,\,$ \\
$\,\,$\color{gr} \color{blue} 1\color{black} $\,\,$ & $\,\,$ 1 $\,\,$ & $\,\,$\color{gr} 5.2761\color{black} $\,\,$ & $\,\,$\color{gr} 8.7534\color{black}   $\,\,$ \\
$\,\,$0.1895$\,\,$ & $\,\,$\color{gr} 0.1895\color{black} $\,\,$ & $\,\,$ 1 $\,\,$ & $\,\,$1.6591 $\,\,$ \\
$\,\,$0.1142$\,\,$ & $\,\,$\color{gr} 0.1142\color{black} $\,\,$ & $\,\,$0.6028$\,\,$ & $\,\,$ 1  $\,\,$ \\
\end{pmatrix},
\end{equation*}
\end{example}
\newpage
\begin{example}
\begin{equation*}
\mathbf{A} =
\begin{pmatrix}
$\,\,$ 1 $\,\,$ & $\,\,$2$\,\,$ & $\,\,$1$\,\,$ & $\,\,$5 $\,\,$ \\
$\,\,$ 1/2$\,\,$ & $\,\,$ 1 $\,\,$ & $\,\,$2$\,\,$ & $\,\,$4 $\,\,$ \\
$\,\,$ 1 $\,\,$ & $\,\,$ 1/2$\,\,$ & $\,\,$ 1 $\,\,$ & $\,\,$3 $\,\,$ \\
$\,\,$ 1/5$\,\,$ & $\,\,$ 1/4$\,\,$ & $\,\,$ 1/3$\,\,$ & $\,\,$ 1  $\,\,$ \\
\end{pmatrix},
\qquad
\lambda_{\max} =
4.1655,
\qquad
CR = 0.0624
\end{equation*}

\begin{equation*}
\mathbf{w}^{EM} =
\begin{pmatrix}
0.383585\\
0.304215\\
0.238794\\
\color{red} 0.073406\color{black}
\end{pmatrix}\end{equation*}
\begin{equation*}
\left[ \frac{{w}^{EM}_i}{{w}^{EM}_j} \right] =
\begin{pmatrix}
$\,\,$ 1 $\,\,$ & $\,\,$1.2609$\,\,$ & $\,\,$1.6063$\,\,$ & $\,\,$\color{red} 5.2255\color{black} $\,\,$ \\
$\,\,$0.7931$\,\,$ & $\,\,$ 1 $\,\,$ & $\,\,$1.2740$\,\,$ & $\,\,$\color{red} 4.1443\color{black}   $\,\,$ \\
$\,\,$0.6225$\,\,$ & $\,\,$0.7850$\,\,$ & $\,\,$ 1 $\,\,$ & $\,\,$\color{red} 3.2531\color{black}  $\,\,$ \\
$\,\,$\color{red} 0.1914\color{black} $\,\,$ & $\,\,$\color{red} 0.2413\color{black} $\,\,$ & $\,\,$\color{red} 0.3074\color{black} $\,\,$ & $\,\,$ 1  $\,\,$ \\
\end{pmatrix},
\end{equation*}

\begin{equation*}
\mathbf{w}^{\prime} =
\begin{pmatrix}
0.382572\\
0.303412\\
0.238163\\
0.075853
\end{pmatrix} =
0.997359\cdot
\begin{pmatrix}
0.383585\\
0.304215\\
0.238794\\
\color{gr} 0.076054\color{black}
\end{pmatrix},
\end{equation*}
\begin{equation*}
\left[ \frac{{w}^{\prime}_i}{{w}^{\prime}_j} \right] =
\begin{pmatrix}
$\,\,$ 1 $\,\,$ & $\,\,$1.2609$\,\,$ & $\,\,$1.6063$\,\,$ & $\,\,$\color{gr} 5.0436\color{black} $\,\,$ \\
$\,\,$0.7931$\,\,$ & $\,\,$ 1 $\,\,$ & $\,\,$1.2740$\,\,$ & $\,\,$\color{gr} \color{blue} 4\color{black}   $\,\,$ \\
$\,\,$0.6225$\,\,$ & $\,\,$0.7850$\,\,$ & $\,\,$ 1 $\,\,$ & $\,\,$\color{gr} 3.1398\color{black}  $\,\,$ \\
$\,\,$\color{gr} 0.1983\color{black} $\,\,$ & $\,\,$\color{gr} \color{blue}  1/4\color{black} $\,\,$ & $\,\,$\color{gr} 0.3185\color{black} $\,\,$ & $\,\,$ 1  $\,\,$ \\
\end{pmatrix},
\end{equation*}
\end{example}
\newpage
\begin{example}
\begin{equation*}
\mathbf{A} =
\begin{pmatrix}
$\,\,$ 1 $\,\,$ & $\,\,$2$\,\,$ & $\,\,$1$\,\,$ & $\,\,$6 $\,\,$ \\
$\,\,$ 1/2$\,\,$ & $\,\,$ 1 $\,\,$ & $\,\,$2$\,\,$ & $\,\,$5 $\,\,$ \\
$\,\,$ 1 $\,\,$ & $\,\,$ 1/2$\,\,$ & $\,\,$ 1 $\,\,$ & $\,\,$4 $\,\,$ \\
$\,\,$ 1/6$\,\,$ & $\,\,$ 1/5$\,\,$ & $\,\,$ 1/4$\,\,$ & $\,\,$ 1  $\,\,$ \\
\end{pmatrix},
\qquad
\lambda_{\max} =
4.1655,
\qquad
CR = 0.0624
\end{equation*}

\begin{equation*}
\mathbf{w}^{EM} =
\begin{pmatrix}
0.385526\\
0.309644\\
0.245574\\
\color{red} 0.059256\color{black}
\end{pmatrix}\end{equation*}
\begin{equation*}
\left[ \frac{{w}^{EM}_i}{{w}^{EM}_j} \right] =
\begin{pmatrix}
$\,\,$ 1 $\,\,$ & $\,\,$1.2451$\,\,$ & $\,\,$1.5699$\,\,$ & $\,\,$\color{red} 6.5061\color{black} $\,\,$ \\
$\,\,$0.8032$\,\,$ & $\,\,$ 1 $\,\,$ & $\,\,$1.2609$\,\,$ & $\,\,$\color{red} 5.2255\color{black}   $\,\,$ \\
$\,\,$0.6370$\,\,$ & $\,\,$0.7931$\,\,$ & $\,\,$ 1 $\,\,$ & $\,\,$\color{red} 4.1443\color{black}  $\,\,$ \\
$\,\,$\color{red} 0.1537\color{black} $\,\,$ & $\,\,$\color{red} 0.1914\color{black} $\,\,$ & $\,\,$\color{red} 0.2413\color{black} $\,\,$ & $\,\,$ 1  $\,\,$ \\
\end{pmatrix},
\end{equation*}

\begin{equation*}
\mathbf{w}^{\prime} =
\begin{pmatrix}
0.384704\\
0.308984\\
0.245050\\
0.061262
\end{pmatrix} =
0.997867\cdot
\begin{pmatrix}
0.385526\\
0.309644\\
0.245574\\
\color{gr} 0.061393\color{black}
\end{pmatrix},
\end{equation*}
\begin{equation*}
\left[ \frac{{w}^{\prime}_i}{{w}^{\prime}_j} \right] =
\begin{pmatrix}
$\,\,$ 1 $\,\,$ & $\,\,$1.2451$\,\,$ & $\,\,$1.5699$\,\,$ & $\,\,$\color{gr} 6.2796\color{black} $\,\,$ \\
$\,\,$0.8032$\,\,$ & $\,\,$ 1 $\,\,$ & $\,\,$1.2609$\,\,$ & $\,\,$\color{gr} 5.0436\color{black}   $\,\,$ \\
$\,\,$0.6370$\,\,$ & $\,\,$0.7931$\,\,$ & $\,\,$ 1 $\,\,$ & $\,\,$\color{gr} \color{blue} 4\color{black}  $\,\,$ \\
$\,\,$\color{gr} 0.1592\color{black} $\,\,$ & $\,\,$\color{gr} 0.1983\color{black} $\,\,$ & $\,\,$\color{gr} \color{blue}  1/4\color{black} $\,\,$ & $\,\,$ 1  $\,\,$ \\
\end{pmatrix},
\end{equation*}
\end{example}
\newpage
\begin{example}
\begin{equation*}
\mathbf{A} =
\begin{pmatrix}
$\,\,$ 1 $\,\,$ & $\,\,$2$\,\,$ & $\,\,$1$\,\,$ & $\,\,$7 $\,\,$ \\
$\,\,$ 1/2$\,\,$ & $\,\,$ 1 $\,\,$ & $\,\,$2$\,\,$ & $\,\,$5 $\,\,$ \\
$\,\,$ 1 $\,\,$ & $\,\,$ 1/2$\,\,$ & $\,\,$ 1 $\,\,$ & $\,\,$4 $\,\,$ \\
$\,\,$ 1/7$\,\,$ & $\,\,$ 1/5$\,\,$ & $\,\,$ 1/4$\,\,$ & $\,\,$ 1  $\,\,$ \\
\end{pmatrix},
\qquad
\lambda_{\max} =
4.1665,
\qquad
CR = 0.0628
\end{equation*}

\begin{equation*}
\mathbf{w}^{EM} =
\begin{pmatrix}
0.394416\\
0.305311\\
0.243936\\
\color{red} 0.056337\color{black}
\end{pmatrix}\end{equation*}
\begin{equation*}
\left[ \frac{{w}^{EM}_i}{{w}^{EM}_j} \right] =
\begin{pmatrix}
$\,\,$ 1 $\,\,$ & $\,\,$1.2919$\,\,$ & $\,\,$1.6169$\,\,$ & $\,\,$\color{red} 7.0010\color{black} $\,\,$ \\
$\,\,$0.7741$\,\,$ & $\,\,$ 1 $\,\,$ & $\,\,$1.2516$\,\,$ & $\,\,$\color{red} 5.4194\color{black}   $\,\,$ \\
$\,\,$0.6185$\,\,$ & $\,\,$0.7990$\,\,$ & $\,\,$ 1 $\,\,$ & $\,\,$\color{red} 4.3299\color{black}  $\,\,$ \\
$\,\,$\color{red} 0.1428\color{black} $\,\,$ & $\,\,$\color{red} 0.1845\color{black} $\,\,$ & $\,\,$\color{red} 0.2310\color{black} $\,\,$ & $\,\,$ 1  $\,\,$ \\
\end{pmatrix},
\end{equation*}

\begin{equation*}
\mathbf{w}^{\prime} =
\begin{pmatrix}
0.394413\\
0.305308\\
0.243934\\
0.056345
\end{pmatrix} =
0.999992\cdot
\begin{pmatrix}
0.394416\\
0.305311\\
0.243936\\
\color{gr} 0.056345\color{black}
\end{pmatrix},
\end{equation*}
\begin{equation*}
\left[ \frac{{w}^{\prime}_i}{{w}^{\prime}_j} \right] =
\begin{pmatrix}
$\,\,$ 1 $\,\,$ & $\,\,$1.2919$\,\,$ & $\,\,$1.6169$\,\,$ & $\,\,$\color{gr} \color{blue} 7\color{black} $\,\,$ \\
$\,\,$0.7741$\,\,$ & $\,\,$ 1 $\,\,$ & $\,\,$1.2516$\,\,$ & $\,\,$\color{gr} 5.4186\color{black}   $\,\,$ \\
$\,\,$0.6185$\,\,$ & $\,\,$0.7990$\,\,$ & $\,\,$ 1 $\,\,$ & $\,\,$\color{gr} 4.3293\color{black}  $\,\,$ \\
$\,\,$\color{gr} \color{blue}  1/7\color{black} $\,\,$ & $\,\,$\color{gr} 0.1846\color{black} $\,\,$ & $\,\,$\color{gr} 0.2310\color{black} $\,\,$ & $\,\,$ 1  $\,\,$ \\
\end{pmatrix},
\end{equation*}
\end{example}
\newpage
\begin{example}
\begin{equation*}
\mathbf{A} =
\begin{pmatrix}
$\,\,$ 1 $\,\,$ & $\,\,$2$\,\,$ & $\,\,$1$\,\,$ & $\,\,$7 $\,\,$ \\
$\,\,$ 1/2$\,\,$ & $\,\,$ 1 $\,\,$ & $\,\,$2$\,\,$ & $\,\,$6 $\,\,$ \\
$\,\,$ 1 $\,\,$ & $\,\,$ 1/2$\,\,$ & $\,\,$ 1 $\,\,$ & $\,\,$5 $\,\,$ \\
$\,\,$ 1/7$\,\,$ & $\,\,$ 1/6$\,\,$ & $\,\,$ 1/5$\,\,$ & $\,\,$ 1  $\,\,$ \\
\end{pmatrix},
\qquad
\lambda_{\max} =
4.1667,
\qquad
CR = 0.0629
\end{equation*}

\begin{equation*}
\mathbf{w}^{EM} =
\begin{pmatrix}
0.386811\\
0.313301\\
0.250150\\
\color{red} 0.049738\color{black}
\end{pmatrix}\end{equation*}
\begin{equation*}
\left[ \frac{{w}^{EM}_i}{{w}^{EM}_j} \right] =
\begin{pmatrix}
$\,\,$ 1 $\,\,$ & $\,\,$1.2346$\,\,$ & $\,\,$1.5463$\,\,$ & $\,\,$\color{red} 7.7770\color{black} $\,\,$ \\
$\,\,$0.8100$\,\,$ & $\,\,$ 1 $\,\,$ & $\,\,$1.2525$\,\,$ & $\,\,$\color{red} 6.2990\color{black}   $\,\,$ \\
$\,\,$0.6467$\,\,$ & $\,\,$0.7984$\,\,$ & $\,\,$ 1 $\,\,$ & $\,\,$\color{red} 5.0294\color{black}  $\,\,$ \\
$\,\,$\color{red} 0.1286\color{black} $\,\,$ & $\,\,$\color{red} 0.1588\color{black} $\,\,$ & $\,\,$\color{red} 0.1988\color{black} $\,\,$ & $\,\,$ 1  $\,\,$ \\
\end{pmatrix},
\end{equation*}

\begin{equation*}
\mathbf{w}^{\prime} =
\begin{pmatrix}
0.386698\\
0.313210\\
0.250077\\
0.050015
\end{pmatrix} =
0.999708\cdot
\begin{pmatrix}
0.386811\\
0.313301\\
0.250150\\
\color{gr} 0.050030\color{black}
\end{pmatrix},
\end{equation*}
\begin{equation*}
\left[ \frac{{w}^{\prime}_i}{{w}^{\prime}_j} \right] =
\begin{pmatrix}
$\,\,$ 1 $\,\,$ & $\,\,$1.2346$\,\,$ & $\,\,$1.5463$\,\,$ & $\,\,$\color{gr} 7.7316\color{black} $\,\,$ \\
$\,\,$0.8100$\,\,$ & $\,\,$ 1 $\,\,$ & $\,\,$1.2525$\,\,$ & $\,\,$\color{gr} 6.2623\color{black}   $\,\,$ \\
$\,\,$0.6467$\,\,$ & $\,\,$0.7984$\,\,$ & $\,\,$ 1 $\,\,$ & $\,\,$\color{gr} \color{blue} 5\color{black}  $\,\,$ \\
$\,\,$\color{gr} 0.1293\color{black} $\,\,$ & $\,\,$\color{gr} 0.1597\color{black} $\,\,$ & $\,\,$\color{gr} \color{blue}  1/5\color{black} $\,\,$ & $\,\,$ 1  $\,\,$ \\
\end{pmatrix},
\end{equation*}
\end{example}
\newpage
\begin{example}
\begin{equation*}
\mathbf{A} =
\begin{pmatrix}
$\,\,$ 1 $\,\,$ & $\,\,$2$\,\,$ & $\,\,$1$\,\,$ & $\,\,$8 $\,\,$ \\
$\,\,$ 1/2$\,\,$ & $\,\,$ 1 $\,\,$ & $\,\,$2$\,\,$ & $\,\,$6 $\,\,$ \\
$\,\,$ 1 $\,\,$ & $\,\,$ 1/2$\,\,$ & $\,\,$ 1 $\,\,$ & $\,\,$5 $\,\,$ \\
$\,\,$ 1/8$\,\,$ & $\,\,$ 1/6$\,\,$ & $\,\,$ 1/5$\,\,$ & $\,\,$ 1  $\,\,$ \\
\end{pmatrix},
\qquad
\lambda_{\max} =
4.1655,
\qquad
CR = 0.0624
\end{equation*}

\begin{equation*}
\mathbf{w}^{EM} =
\begin{pmatrix}
0.394314\\
0.309517\\
0.248596\\
\color{red} 0.047573\color{black}
\end{pmatrix}\end{equation*}
\begin{equation*}
\left[ \frac{{w}^{EM}_i}{{w}^{EM}_j} \right] =
\begin{pmatrix}
$\,\,$ 1 $\,\,$ & $\,\,$1.2740$\,\,$ & $\,\,$1.5862$\,\,$ & $\,\,$\color{red} 8.2886\color{black} $\,\,$ \\
$\,\,$0.7850$\,\,$ & $\,\,$ 1 $\,\,$ & $\,\,$1.2451$\,\,$ & $\,\,$\color{red} 6.5061\color{black}   $\,\,$ \\
$\,\,$0.6305$\,\,$ & $\,\,$0.8032$\,\,$ & $\,\,$ 1 $\,\,$ & $\,\,$\color{red} 5.2255\color{black}  $\,\,$ \\
$\,\,$\color{red} 0.1206\color{black} $\,\,$ & $\,\,$\color{red} 0.1537\color{black} $\,\,$ & $\,\,$\color{red} 0.1914\color{black} $\,\,$ & $\,\,$ 1  $\,\,$ \\
\end{pmatrix},
\end{equation*}

\begin{equation*}
\mathbf{w}^{\prime} =
\begin{pmatrix}
0.393639\\
0.308987\\
0.248170\\
0.049205
\end{pmatrix} =
0.998287\cdot
\begin{pmatrix}
0.394314\\
0.309517\\
0.248596\\
\color{gr} 0.049289\color{black}
\end{pmatrix},
\end{equation*}
\begin{equation*}
\left[ \frac{{w}^{\prime}_i}{{w}^{\prime}_j} \right] =
\begin{pmatrix}
$\,\,$ 1 $\,\,$ & $\,\,$1.2740$\,\,$ & $\,\,$1.5862$\,\,$ & $\,\,$\color{gr} \color{blue} 8\color{black} $\,\,$ \\
$\,\,$0.7850$\,\,$ & $\,\,$ 1 $\,\,$ & $\,\,$1.2451$\,\,$ & $\,\,$\color{gr} 6.2796\color{black}   $\,\,$ \\
$\,\,$0.6305$\,\,$ & $\,\,$0.8032$\,\,$ & $\,\,$ 1 $\,\,$ & $\,\,$\color{gr} 5.0436\color{black}  $\,\,$ \\
$\,\,$\color{gr} \color{blue}  1/8\color{black} $\,\,$ & $\,\,$\color{gr} 0.1592\color{black} $\,\,$ & $\,\,$\color{gr} 0.1983\color{black} $\,\,$ & $\,\,$ 1  $\,\,$ \\
\end{pmatrix},
\end{equation*}
\end{example}
\newpage
\begin{example}
\begin{equation*}
\mathbf{A} =
\begin{pmatrix}
$\,\,$ 1 $\,\,$ & $\,\,$2$\,\,$ & $\,\,$1$\,\,$ & $\,\,$8 $\,\,$ \\
$\,\,$ 1/2$\,\,$ & $\,\,$ 1 $\,\,$ & $\,\,$2$\,\,$ & $\,\,$7 $\,\,$ \\
$\,\,$ 1 $\,\,$ & $\,\,$ 1/2$\,\,$ & $\,\,$ 1 $\,\,$ & $\,\,$5 $\,\,$ \\
$\,\,$ 1/8$\,\,$ & $\,\,$ 1/7$\,\,$ & $\,\,$ 1/5$\,\,$ & $\,\,$ 1  $\,\,$ \\
\end{pmatrix},
\qquad
\lambda_{\max} =
4.1665,
\qquad
CR = 0.0628
\end{equation*}

\begin{equation*}
\mathbf{w}^{EM} =
\begin{pmatrix}
0.392200\\
0.317071\\
0.245439\\
\color{red} 0.045289\color{black}
\end{pmatrix}\end{equation*}
\begin{equation*}
\left[ \frac{{w}^{EM}_i}{{w}^{EM}_j} \right] =
\begin{pmatrix}
$\,\,$ 1 $\,\,$ & $\,\,$1.2369$\,\,$ & $\,\,$1.5979$\,\,$ & $\,\,$\color{red} 8.6598\color{black} $\,\,$ \\
$\,\,$0.8084$\,\,$ & $\,\,$ 1 $\,\,$ & $\,\,$1.2919$\,\,$ & $\,\,$\color{red} 7.0010\color{black}   $\,\,$ \\
$\,\,$0.6258$\,\,$ & $\,\,$0.7741$\,\,$ & $\,\,$ 1 $\,\,$ & $\,\,$\color{red} 5.4194\color{black}  $\,\,$ \\
$\,\,$\color{red} 0.1155\color{black} $\,\,$ & $\,\,$\color{red} 0.1428\color{black} $\,\,$ & $\,\,$\color{red} 0.1845\color{black} $\,\,$ & $\,\,$ 1  $\,\,$ \\
\end{pmatrix},
\end{equation*}

\begin{equation*}
\mathbf{w}^{\prime} =
\begin{pmatrix}
0.392197\\
0.317069\\
0.245438\\
0.045296
\end{pmatrix} =
0.999994\cdot
\begin{pmatrix}
0.392200\\
0.317071\\
0.245439\\
\color{gr} 0.045296\color{black}
\end{pmatrix},
\end{equation*}
\begin{equation*}
\left[ \frac{{w}^{\prime}_i}{{w}^{\prime}_j} \right] =
\begin{pmatrix}
$\,\,$ 1 $\,\,$ & $\,\,$1.2369$\,\,$ & $\,\,$1.5979$\,\,$ & $\,\,$\color{gr} 8.6586\color{black} $\,\,$ \\
$\,\,$0.8084$\,\,$ & $\,\,$ 1 $\,\,$ & $\,\,$1.2919$\,\,$ & $\,\,$\color{gr} \color{blue} 7\color{black}   $\,\,$ \\
$\,\,$0.6258$\,\,$ & $\,\,$0.7741$\,\,$ & $\,\,$ 1 $\,\,$ & $\,\,$\color{gr} 5.4186\color{black}  $\,\,$ \\
$\,\,$\color{gr} 0.1155\color{black} $\,\,$ & $\,\,$\color{gr} \color{blue}  1/7\color{black} $\,\,$ & $\,\,$\color{gr} 0.1846\color{black} $\,\,$ & $\,\,$ 1  $\,\,$ \\
\end{pmatrix},
\end{equation*}
\end{example}
\newpage
\begin{example}
\begin{equation*}
\mathbf{A} =
\begin{pmatrix}
$\,\,$ 1 $\,\,$ & $\,\,$2$\,\,$ & $\,\,$1$\,\,$ & $\,\,$9 $\,\,$ \\
$\,\,$ 1/2$\,\,$ & $\,\,$ 1 $\,\,$ & $\,\,$2$\,\,$ & $\,\,$7 $\,\,$ \\
$\,\,$ 1 $\,\,$ & $\,\,$ 1/2$\,\,$ & $\,\,$ 1 $\,\,$ & $\,\,$5 $\,\,$ \\
$\,\,$ 1/9$\,\,$ & $\,\,$ 1/7$\,\,$ & $\,\,$ 1/5$\,\,$ & $\,\,$ 1  $\,\,$ \\
\end{pmatrix},
\qquad
\lambda_{\max} =
4.1669,
\qquad
CR = 0.0629
\end{equation*}

\begin{equation*}
\mathbf{w}^{EM} =
\begin{pmatrix}
0.398823\\
0.313442\\
0.244183\\
\color{red} 0.043553\color{black}
\end{pmatrix}\end{equation*}
\begin{equation*}
\left[ \frac{{w}^{EM}_i}{{w}^{EM}_j} \right] =
\begin{pmatrix}
$\,\,$ 1 $\,\,$ & $\,\,$1.2724$\,\,$ & $\,\,$1.6333$\,\,$ & $\,\,$\color{red} 9.1573\color{black} $\,\,$ \\
$\,\,$0.7859$\,\,$ & $\,\,$ 1 $\,\,$ & $\,\,$1.2836$\,\,$ & $\,\,$\color{red} 7.1969\color{black}   $\,\,$ \\
$\,\,$0.6123$\,\,$ & $\,\,$0.7790$\,\,$ & $\,\,$ 1 $\,\,$ & $\,\,$\color{red} 5.6066\color{black}  $\,\,$ \\
$\,\,$\color{red} 0.1092\color{black} $\,\,$ & $\,\,$\color{red} 0.1389\color{black} $\,\,$ & $\,\,$\color{red} 0.1784\color{black} $\,\,$ & $\,\,$ 1  $\,\,$ \\
\end{pmatrix},
\end{equation*}

\begin{equation*}
\mathbf{w}^{\prime} =
\begin{pmatrix}
0.398520\\
0.313203\\
0.243997\\
0.044280
\end{pmatrix} =
0.999240\cdot
\begin{pmatrix}
0.398823\\
0.313442\\
0.244183\\
\color{gr} 0.044314\color{black}
\end{pmatrix},
\end{equation*}
\begin{equation*}
\left[ \frac{{w}^{\prime}_i}{{w}^{\prime}_j} \right] =
\begin{pmatrix}
$\,\,$ 1 $\,\,$ & $\,\,$1.2724$\,\,$ & $\,\,$1.6333$\,\,$ & $\,\,$\color{gr} \color{blue} 9\color{black} $\,\,$ \\
$\,\,$0.7859$\,\,$ & $\,\,$ 1 $\,\,$ & $\,\,$1.2836$\,\,$ & $\,\,$\color{gr} 7.0733\color{black}   $\,\,$ \\
$\,\,$0.6123$\,\,$ & $\,\,$0.7790$\,\,$ & $\,\,$ 1 $\,\,$ & $\,\,$\color{gr} 5.5103\color{black}  $\,\,$ \\
$\,\,$\color{gr} \color{blue}  1/9\color{black} $\,\,$ & $\,\,$\color{gr} 0.1414\color{black} $\,\,$ & $\,\,$\color{gr} 0.1815\color{black} $\,\,$ & $\,\,$ 1  $\,\,$ \\
\end{pmatrix},
\end{equation*}
\end{example}
\newpage
\begin{example}
\begin{equation*}
\mathbf{A} =
\begin{pmatrix}
$\,\,$ 1 $\,\,$ & $\,\,$2$\,\,$ & $\,\,$1$\,\,$ & $\,\,$9 $\,\,$ \\
$\,\,$ 1/2$\,\,$ & $\,\,$ 1 $\,\,$ & $\,\,$2$\,\,$ & $\,\,$7 $\,\,$ \\
$\,\,$ 1 $\,\,$ & $\,\,$ 1/2$\,\,$ & $\,\,$ 1 $\,\,$ & $\,\,$6 $\,\,$ \\
$\,\,$ 1/9$\,\,$ & $\,\,$ 1/7$\,\,$ & $\,\,$ 1/6$\,\,$ & $\,\,$ 1  $\,\,$ \\
\end{pmatrix},
\qquad
\lambda_{\max} =
4.1658,
\qquad
CR = 0.0625
\end{equation*}

\begin{equation*}
\mathbf{w}^{EM} =
\begin{pmatrix}
0.394221\\
0.312577\\
0.251994\\
\color{red} 0.041208\color{black}
\end{pmatrix}\end{equation*}
\begin{equation*}
\left[ \frac{{w}^{EM}_i}{{w}^{EM}_j} \right] =
\begin{pmatrix}
$\,\,$ 1 $\,\,$ & $\,\,$1.2612$\,\,$ & $\,\,$1.5644$\,\,$ & $\,\,$\color{red} 9.5666\color{black} $\,\,$ \\
$\,\,$0.7929$\,\,$ & $\,\,$ 1 $\,\,$ & $\,\,$1.2404$\,\,$ & $\,\,$\color{red} 7.5854\color{black}   $\,\,$ \\
$\,\,$0.6392$\,\,$ & $\,\,$0.8062$\,\,$ & $\,\,$ 1 $\,\,$ & $\,\,$\color{red} 6.1152\color{black}  $\,\,$ \\
$\,\,$\color{red} 0.1045\color{black} $\,\,$ & $\,\,$\color{red} 0.1318\color{black} $\,\,$ & $\,\,$\color{red} 0.1635\color{black} $\,\,$ & $\,\,$ 1  $\,\,$ \\
\end{pmatrix},
\end{equation*}

\begin{equation*}
\mathbf{w}^{\prime} =
\begin{pmatrix}
0.393910\\
0.312330\\
0.251794\\
0.041966
\end{pmatrix} =
0.999210\cdot
\begin{pmatrix}
0.394221\\
0.312577\\
0.251994\\
\color{gr} 0.041999\color{black}
\end{pmatrix},
\end{equation*}
\begin{equation*}
\left[ \frac{{w}^{\prime}_i}{{w}^{\prime}_j} \right] =
\begin{pmatrix}
$\,\,$ 1 $\,\,$ & $\,\,$1.2612$\,\,$ & $\,\,$1.5644$\,\,$ & $\,\,$\color{gr} 9.3865\color{black} $\,\,$ \\
$\,\,$0.7929$\,\,$ & $\,\,$ 1 $\,\,$ & $\,\,$1.2404$\,\,$ & $\,\,$\color{gr} 7.4425\color{black}   $\,\,$ \\
$\,\,$0.6392$\,\,$ & $\,\,$0.8062$\,\,$ & $\,\,$ 1 $\,\,$ & $\,\,$\color{gr} \color{blue} 6\color{black}  $\,\,$ \\
$\,\,$\color{gr} 0.1065\color{black} $\,\,$ & $\,\,$\color{gr} 0.1344\color{black} $\,\,$ & $\,\,$\color{gr} \color{blue}  1/6\color{black} $\,\,$ & $\,\,$ 1  $\,\,$ \\
\end{pmatrix},
\end{equation*}
\end{example}
\newpage
\begin{example}
\begin{equation*}
\mathbf{A} =
\begin{pmatrix}
$\,\,$ 1 $\,\,$ & $\,\,$2$\,\,$ & $\,\,$1$\,\,$ & $\,\,$9 $\,\,$ \\
$\,\,$ 1/2$\,\,$ & $\,\,$ 1 $\,\,$ & $\,\,$2$\,\,$ & $\,\,$8 $\,\,$ \\
$\,\,$ 1 $\,\,$ & $\,\,$ 1/2$\,\,$ & $\,\,$ 1 $\,\,$ & $\,\,$6 $\,\,$ \\
$\,\,$ 1/9$\,\,$ & $\,\,$ 1/8$\,\,$ & $\,\,$ 1/6$\,\,$ & $\,\,$ 1  $\,\,$ \\
\end{pmatrix},
\qquad
\lambda_{\max} =
4.1664,
\qquad
CR = 0.0627
\end{equation*}

\begin{equation*}
\mathbf{w}^{EM} =
\begin{pmatrix}
0.392386\\
0.319037\\
0.249102\\
\color{red} 0.039475\color{black}
\end{pmatrix}\end{equation*}
\begin{equation*}
\left[ \frac{{w}^{EM}_i}{{w}^{EM}_j} \right] =
\begin{pmatrix}
$\,\,$ 1 $\,\,$ & $\,\,$1.2299$\,\,$ & $\,\,$1.5752$\,\,$ & $\,\,$\color{red} 9.9400\color{black} $\,\,$ \\
$\,\,$0.8131$\,\,$ & $\,\,$ 1 $\,\,$ & $\,\,$1.2807$\,\,$ & $\,\,$\color{red} 8.0819\color{black}   $\,\,$ \\
$\,\,$0.6348$\,\,$ & $\,\,$0.7808$\,\,$ & $\,\,$ 1 $\,\,$ & $\,\,$\color{red} 6.3103\color{black}  $\,\,$ \\
$\,\,$\color{red} 0.1006\color{black} $\,\,$ & $\,\,$\color{red} 0.1237\color{black} $\,\,$ & $\,\,$\color{red} 0.1585\color{black} $\,\,$ & $\,\,$ 1  $\,\,$ \\
\end{pmatrix},
\end{equation*}

\begin{equation*}
\mathbf{w}^{\prime} =
\begin{pmatrix}
0.392228\\
0.318908\\
0.249001\\
0.039863
\end{pmatrix} =
0.999596\cdot
\begin{pmatrix}
0.392386\\
0.319037\\
0.249102\\
\color{gr} 0.039880\color{black}
\end{pmatrix},
\end{equation*}
\begin{equation*}
\left[ \frac{{w}^{\prime}_i}{{w}^{\prime}_j} \right] =
\begin{pmatrix}
$\,\,$ 1 $\,\,$ & $\,\,$1.2299$\,\,$ & $\,\,$1.5752$\,\,$ & $\,\,$\color{gr} 9.8393\color{black} $\,\,$ \\
$\,\,$0.8131$\,\,$ & $\,\,$ 1 $\,\,$ & $\,\,$1.2807$\,\,$ & $\,\,$\color{gr} \color{blue} 8\color{black}   $\,\,$ \\
$\,\,$0.6348$\,\,$ & $\,\,$0.7808$\,\,$ & $\,\,$ 1 $\,\,$ & $\,\,$\color{gr} 6.2463\color{black}  $\,\,$ \\
$\,\,$\color{gr} 0.1016\color{black} $\,\,$ & $\,\,$\color{gr} \color{blue}  1/8\color{black} $\,\,$ & $\,\,$\color{gr} 0.1601\color{black} $\,\,$ & $\,\,$ 1  $\,\,$ \\
\end{pmatrix},
\end{equation*}
\end{example}
\newpage
\begin{example}
\begin{equation*}
\mathbf{A} =
\begin{pmatrix}
$\,\,$ 1 $\,\,$ & $\,\,$2$\,\,$ & $\,\,$3$\,\,$ & $\,\,$2 $\,\,$ \\
$\,\,$ 1/2$\,\,$ & $\,\,$ 1 $\,\,$ & $\,\,$2$\,\,$ & $\,\,$3 $\,\,$ \\
$\,\,$ 1/3$\,\,$ & $\,\,$ 1/2$\,\,$ & $\,\,$ 1 $\,\,$ & $\,\,$1 $\,\,$ \\
$\,\,$ 1/2$\,\,$ & $\,\,$ 1/3$\,\,$ & $\,\,$ 1 $\,\,$ & $\,\,$ 1  $\,\,$ \\
\end{pmatrix},
\qquad
\lambda_{\max} =
4.1031,
\qquad
CR = 0.0389
\end{equation*}

\begin{equation*}
\mathbf{w}^{EM} =
\begin{pmatrix}
0.419234\\
0.296979\\
\color{red} 0.139411\color{black} \\
0.144376
\end{pmatrix}\end{equation*}
\begin{equation*}
\left[ \frac{{w}^{EM}_i}{{w}^{EM}_j} \right] =
\begin{pmatrix}
$\,\,$ 1 $\,\,$ & $\,\,$1.4117$\,\,$ & $\,\,$\color{red} 3.0072\color{black} $\,\,$ & $\,\,$2.9038$\,\,$ \\
$\,\,$0.7084$\,\,$ & $\,\,$ 1 $\,\,$ & $\,\,$\color{red} 2.1302\color{black} $\,\,$ & $\,\,$2.0570  $\,\,$ \\
$\,\,$\color{red} 0.3325\color{black} $\,\,$ & $\,\,$\color{red} 0.4694\color{black} $\,\,$ & $\,\,$ 1 $\,\,$ & $\,\,$\color{red} 0.9656\color{black}  $\,\,$ \\
$\,\,$0.3444$\,\,$ & $\,\,$0.4862$\,\,$ & $\,\,$\color{red} 1.0356\color{black} $\,\,$ & $\,\,$ 1  $\,\,$ \\
\end{pmatrix},
\end{equation*}

\begin{equation*}
\mathbf{w}^{\prime} =
\begin{pmatrix}
0.419094\\
0.296880\\
0.139698\\
0.144328
\end{pmatrix} =
0.999666\cdot
\begin{pmatrix}
0.419234\\
0.296979\\
\color{gr} 0.139745\color{black} \\
0.144376
\end{pmatrix},
\end{equation*}
\begin{equation*}
\left[ \frac{{w}^{\prime}_i}{{w}^{\prime}_j} \right] =
\begin{pmatrix}
$\,\,$ 1 $\,\,$ & $\,\,$1.4117$\,\,$ & $\,\,$\color{gr} \color{blue} 3\color{black} $\,\,$ & $\,\,$2.9038$\,\,$ \\
$\,\,$0.7084$\,\,$ & $\,\,$ 1 $\,\,$ & $\,\,$\color{gr} 2.1252\color{black} $\,\,$ & $\,\,$2.0570  $\,\,$ \\
$\,\,$\color{gr} \color{blue}  1/3\color{black} $\,\,$ & $\,\,$\color{gr} 0.4706\color{black} $\,\,$ & $\,\,$ 1 $\,\,$ & $\,\,$\color{gr} 0.9679\color{black}  $\,\,$ \\
$\,\,$0.3444$\,\,$ & $\,\,$0.4862$\,\,$ & $\,\,$\color{gr} 1.0331\color{black} $\,\,$ & $\,\,$ 1  $\,\,$ \\
\end{pmatrix},
\end{equation*}
\end{example}
\newpage
\begin{example}
\begin{equation*}
\mathbf{A} =
\begin{pmatrix}
$\,\,$ 1 $\,\,$ & $\,\,$2$\,\,$ & $\,\,$3$\,\,$ & $\,\,$2 $\,\,$ \\
$\,\,$ 1/2$\,\,$ & $\,\,$ 1 $\,\,$ & $\,\,$3$\,\,$ & $\,\,$5 $\,\,$ \\
$\,\,$ 1/3$\,\,$ & $\,\,$ 1/3$\,\,$ & $\,\,$ 1 $\,\,$ & $\,\,$1 $\,\,$ \\
$\,\,$ 1/2$\,\,$ & $\,\,$ 1/5$\,\,$ & $\,\,$ 1 $\,\,$ & $\,\,$ 1  $\,\,$ \\
\end{pmatrix},
\qquad
\lambda_{\max} =
4.2277,
\qquad
CR = 0.0859
\end{equation*}

\begin{equation*}
\mathbf{w}^{EM} =
\begin{pmatrix}
0.404811\\
0.358042\\
\color{red} 0.116240\color{black} \\
0.120907
\end{pmatrix}\end{equation*}
\begin{equation*}
\left[ \frac{{w}^{EM}_i}{{w}^{EM}_j} \right] =
\begin{pmatrix}
$\,\,$ 1 $\,\,$ & $\,\,$1.1306$\,\,$ & $\,\,$\color{red} 3.4825\color{black} $\,\,$ & $\,\,$3.3481$\,\,$ \\
$\,\,$0.8845$\,\,$ & $\,\,$ 1 $\,\,$ & $\,\,$\color{red} 3.0802\color{black} $\,\,$ & $\,\,$2.9613  $\,\,$ \\
$\,\,$\color{red} 0.2871\color{black} $\,\,$ & $\,\,$\color{red} 0.3247\color{black} $\,\,$ & $\,\,$ 1 $\,\,$ & $\,\,$\color{red} 0.9614\color{black}  $\,\,$ \\
$\,\,$0.2987$\,\,$ & $\,\,$0.3377$\,\,$ & $\,\,$\color{red} 1.0401\color{black} $\,\,$ & $\,\,$ 1  $\,\,$ \\
\end{pmatrix},
\end{equation*}

\begin{equation*}
\mathbf{w}^{\prime} =
\begin{pmatrix}
0.403557\\
0.356933\\
0.118978\\
0.120532
\end{pmatrix} =
0.996902\cdot
\begin{pmatrix}
0.404811\\
0.358042\\
\color{gr} 0.119347\color{black} \\
0.120907
\end{pmatrix},
\end{equation*}
\begin{equation*}
\left[ \frac{{w}^{\prime}_i}{{w}^{\prime}_j} \right] =
\begin{pmatrix}
$\,\,$ 1 $\,\,$ & $\,\,$1.1306$\,\,$ & $\,\,$\color{gr} 3.3919\color{black} $\,\,$ & $\,\,$3.3481$\,\,$ \\
$\,\,$0.8845$\,\,$ & $\,\,$ 1 $\,\,$ & $\,\,$\color{gr} \color{blue} 3\color{black} $\,\,$ & $\,\,$2.9613  $\,\,$ \\
$\,\,$\color{gr} 0.2948\color{black} $\,\,$ & $\,\,$\color{gr} \color{blue}  1/3\color{black} $\,\,$ & $\,\,$ 1 $\,\,$ & $\,\,$\color{gr} 0.9871\color{black}  $\,\,$ \\
$\,\,$0.2987$\,\,$ & $\,\,$0.3377$\,\,$ & $\,\,$\color{gr} 1.0131\color{black} $\,\,$ & $\,\,$ 1  $\,\,$ \\
\end{pmatrix},
\end{equation*}
\end{example}
\newpage
\begin{example}
\begin{equation*}
\mathbf{A} =
\begin{pmatrix}
$\,\,$ 1 $\,\,$ & $\,\,$2$\,\,$ & $\,\,$3$\,\,$ & $\,\,$6 $\,\,$ \\
$\,\,$ 1/2$\,\,$ & $\,\,$ 1 $\,\,$ & $\,\,$2$\,\,$ & $\,\,$2 $\,\,$ \\
$\,\,$ 1/3$\,\,$ & $\,\,$ 1/2$\,\,$ & $\,\,$ 1 $\,\,$ & $\,\,$3 $\,\,$ \\
$\,\,$ 1/6$\,\,$ & $\,\,$ 1/2$\,\,$ & $\,\,$ 1/3$\,\,$ & $\,\,$ 1  $\,\,$ \\
\end{pmatrix},
\qquad
\lambda_{\max} =
4.1031,
\qquad
CR = 0.0389
\end{equation*}

\begin{equation*}
\mathbf{w}^{EM} =
\begin{pmatrix}
\color{red} 0.492892\color{black} \\
0.247037\\
0.174997\\
0.085075
\end{pmatrix}\end{equation*}
\begin{equation*}
\left[ \frac{{w}^{EM}_i}{{w}^{EM}_j} \right] =
\begin{pmatrix}
$\,\,$ 1 $\,\,$ & $\,\,$\color{red} 1.9952\color{black} $\,\,$ & $\,\,$\color{red} 2.8166\color{black} $\,\,$ & $\,\,$\color{red} 5.7936\color{black} $\,\,$ \\
$\,\,$\color{red} 0.5012\color{black} $\,\,$ & $\,\,$ 1 $\,\,$ & $\,\,$1.4117$\,\,$ & $\,\,$2.9038  $\,\,$ \\
$\,\,$\color{red} 0.3550\color{black} $\,\,$ & $\,\,$0.7084$\,\,$ & $\,\,$ 1 $\,\,$ & $\,\,$2.0570 $\,\,$ \\
$\,\,$\color{red} 0.1726\color{black} $\,\,$ & $\,\,$0.3444$\,\,$ & $\,\,$0.4862$\,\,$ & $\,\,$ 1  $\,\,$ \\
\end{pmatrix},
\end{equation*}

\begin{equation*}
\mathbf{w}^{\prime} =
\begin{pmatrix}
0.493490\\
0.246745\\
0.174790\\
0.084974
\end{pmatrix} =
0.998820\cdot
\begin{pmatrix}
\color{gr} 0.494073\color{black} \\
0.247037\\
0.174997\\
0.085075
\end{pmatrix},
\end{equation*}
\begin{equation*}
\left[ \frac{{w}^{\prime}_i}{{w}^{\prime}_j} \right] =
\begin{pmatrix}
$\,\,$ 1 $\,\,$ & $\,\,$\color{gr} \color{blue} 2\color{black} $\,\,$ & $\,\,$\color{gr} 2.8233\color{black} $\,\,$ & $\,\,$\color{gr} 5.8075\color{black} $\,\,$ \\
$\,\,$\color{gr} \color{blue}  1/2\color{black} $\,\,$ & $\,\,$ 1 $\,\,$ & $\,\,$1.4117$\,\,$ & $\,\,$2.9038  $\,\,$ \\
$\,\,$\color{gr} 0.3542\color{black} $\,\,$ & $\,\,$0.7084$\,\,$ & $\,\,$ 1 $\,\,$ & $\,\,$2.0570 $\,\,$ \\
$\,\,$\color{gr} 0.1722\color{black} $\,\,$ & $\,\,$0.3444$\,\,$ & $\,\,$0.4862$\,\,$ & $\,\,$ 1  $\,\,$ \\
\end{pmatrix},
\end{equation*}
\end{example}
\newpage
\begin{example}
\begin{equation*}
\mathbf{A} =
\begin{pmatrix}
$\,\,$ 1 $\,\,$ & $\,\,$2$\,\,$ & $\,\,$3$\,\,$ & $\,\,$8 $\,\,$ \\
$\,\,$ 1/2$\,\,$ & $\,\,$ 1 $\,\,$ & $\,\,$1$\,\,$ & $\,\,$6 $\,\,$ \\
$\,\,$ 1/3$\,\,$ & $\,\,$ 1 $\,\,$ & $\,\,$ 1 $\,\,$ & $\,\,$2 $\,\,$ \\
$\,\,$ 1/8$\,\,$ & $\,\,$ 1/6$\,\,$ & $\,\,$ 1/2$\,\,$ & $\,\,$ 1  $\,\,$ \\
\end{pmatrix},
\qquad
\lambda_{\max} =
4.1031,
\qquad
CR = 0.0389
\end{equation*}

\begin{equation*}
\mathbf{w}^{EM} =
\begin{pmatrix}
\color{red} 0.500463\color{black} \\
0.259145\\
0.177685\\
0.062708
\end{pmatrix}\end{equation*}
\begin{equation*}
\left[ \frac{{w}^{EM}_i}{{w}^{EM}_j} \right] =
\begin{pmatrix}
$\,\,$ 1 $\,\,$ & $\,\,$\color{red} 1.9312\color{black} $\,\,$ & $\,\,$\color{red} 2.8166\color{black} $\,\,$ & $\,\,$\color{red} 7.9809\color{black} $\,\,$ \\
$\,\,$\color{red} 0.5178\color{black} $\,\,$ & $\,\,$ 1 $\,\,$ & $\,\,$1.4585$\,\,$ & $\,\,$4.1326  $\,\,$ \\
$\,\,$\color{red} 0.3550\color{black} $\,\,$ & $\,\,$0.6857$\,\,$ & $\,\,$ 1 $\,\,$ & $\,\,$2.8335 $\,\,$ \\
$\,\,$\color{red} 0.1253\color{black} $\,\,$ & $\,\,$0.2420$\,\,$ & $\,\,$0.3529$\,\,$ & $\,\,$ 1  $\,\,$ \\
\end{pmatrix},
\end{equation*}

\begin{equation*}
\mathbf{w}^{\prime} =
\begin{pmatrix}
0.501061\\
0.258834\\
0.177472\\
0.062633
\end{pmatrix} =
0.998802\cdot
\begin{pmatrix}
\color{gr} 0.501662\color{black} \\
0.259145\\
0.177685\\
0.062708
\end{pmatrix},
\end{equation*}
\begin{equation*}
\left[ \frac{{w}^{\prime}_i}{{w}^{\prime}_j} \right] =
\begin{pmatrix}
$\,\,$ 1 $\,\,$ & $\,\,$\color{gr} 1.9358\color{black} $\,\,$ & $\,\,$\color{gr} 2.8233\color{black} $\,\,$ & $\,\,$\color{gr} \color{blue} 8\color{black} $\,\,$ \\
$\,\,$\color{gr} 0.5166\color{black} $\,\,$ & $\,\,$ 1 $\,\,$ & $\,\,$1.4585$\,\,$ & $\,\,$4.1326  $\,\,$ \\
$\,\,$\color{gr} 0.3542\color{black} $\,\,$ & $\,\,$0.6857$\,\,$ & $\,\,$ 1 $\,\,$ & $\,\,$2.8335 $\,\,$ \\
$\,\,$\color{gr} \color{blue}  1/8\color{black} $\,\,$ & $\,\,$0.2420$\,\,$ & $\,\,$0.3529$\,\,$ & $\,\,$ 1  $\,\,$ \\
\end{pmatrix},
\end{equation*}
\end{example}
\newpage
\begin{example}
\begin{equation*}
\mathbf{A} =
\begin{pmatrix}
$\,\,$ 1 $\,\,$ & $\,\,$2$\,\,$ & $\,\,$3$\,\,$ & $\,\,$9 $\,\,$ \\
$\,\,$ 1/2$\,\,$ & $\,\,$ 1 $\,\,$ & $\,\,$1$\,\,$ & $\,\,$6 $\,\,$ \\
$\,\,$ 1/3$\,\,$ & $\,\,$ 1 $\,\,$ & $\,\,$ 1 $\,\,$ & $\,\,$2 $\,\,$ \\
$\,\,$ 1/9$\,\,$ & $\,\,$ 1/6$\,\,$ & $\,\,$ 1/2$\,\,$ & $\,\,$ 1  $\,\,$ \\
\end{pmatrix},
\qquad
\lambda_{\max} =
4.1031,
\qquad
CR = 0.0389
\end{equation*}

\begin{equation*}
\mathbf{w}^{EM} =
\begin{pmatrix}
\color{red} 0.508970\color{black} \\
0.255095\\
0.175700\\
0.060235
\end{pmatrix}\end{equation*}
\begin{equation*}
\left[ \frac{{w}^{EM}_i}{{w}^{EM}_j} \right] =
\begin{pmatrix}
$\,\,$ 1 $\,\,$ & $\,\,$\color{red} 1.9952\color{black} $\,\,$ & $\,\,$\color{red} 2.8968\color{black} $\,\,$ & $\,\,$\color{red} 8.4497\color{black} $\,\,$ \\
$\,\,$\color{red} 0.5012\color{black} $\,\,$ & $\,\,$ 1 $\,\,$ & $\,\,$1.4519$\,\,$ & $\,\,$4.2350  $\,\,$ \\
$\,\,$\color{red} 0.3452\color{black} $\,\,$ & $\,\,$0.6888$\,\,$ & $\,\,$ 1 $\,\,$ & $\,\,$2.9169 $\,\,$ \\
$\,\,$\color{red} 0.1183\color{black} $\,\,$ & $\,\,$0.2361$\,\,$ & $\,\,$0.3428$\,\,$ & $\,\,$ 1  $\,\,$ \\
\end{pmatrix},
\end{equation*}

\begin{equation*}
\mathbf{w}^{\prime} =
\begin{pmatrix}
0.509568\\
0.254784\\
0.175486\\
0.060162
\end{pmatrix} =
0.998782\cdot
\begin{pmatrix}
\color{gr} 0.510190\color{black} \\
0.255095\\
0.175700\\
0.060235
\end{pmatrix},
\end{equation*}
\begin{equation*}
\left[ \frac{{w}^{\prime}_i}{{w}^{\prime}_j} \right] =
\begin{pmatrix}
$\,\,$ 1 $\,\,$ & $\,\,$\color{gr} \color{blue} 2\color{black} $\,\,$ & $\,\,$\color{gr} 2.9038\color{black} $\,\,$ & $\,\,$\color{gr} 8.4700\color{black} $\,\,$ \\
$\,\,$\color{gr} \color{blue}  1/2\color{black} $\,\,$ & $\,\,$ 1 $\,\,$ & $\,\,$1.4519$\,\,$ & $\,\,$4.2350  $\,\,$ \\
$\,\,$\color{gr} 0.3444\color{black} $\,\,$ & $\,\,$0.6888$\,\,$ & $\,\,$ 1 $\,\,$ & $\,\,$2.9169 $\,\,$ \\
$\,\,$\color{gr} 0.1181\color{black} $\,\,$ & $\,\,$0.2361$\,\,$ & $\,\,$0.3428$\,\,$ & $\,\,$ 1  $\,\,$ \\
\end{pmatrix},
\end{equation*}
\end{example}
\newpage
\begin{example}
\begin{equation*}
\mathbf{A} =
\begin{pmatrix}
$\,\,$ 1 $\,\,$ & $\,\,$2$\,\,$ & $\,\,$3$\,\,$ & $\,\,$9 $\,\,$ \\
$\,\,$ 1/2$\,\,$ & $\,\,$ 1 $\,\,$ & $\,\,$1$\,\,$ & $\,\,$7 $\,\,$ \\
$\,\,$ 1/3$\,\,$ & $\,\,$ 1 $\,\,$ & $\,\,$ 1 $\,\,$ & $\,\,$2 $\,\,$ \\
$\,\,$ 1/9$\,\,$ & $\,\,$ 1/7$\,\,$ & $\,\,$ 1/2$\,\,$ & $\,\,$ 1  $\,\,$ \\
\end{pmatrix},
\qquad
\lambda_{\max} =
4.1342,
\qquad
CR = 0.0506
\end{equation*}

\begin{equation*}
\mathbf{w}^{EM} =
\begin{pmatrix}
\color{red} 0.502382\color{black} \\
0.264986\\
0.174850\\
0.057782
\end{pmatrix}\end{equation*}
\begin{equation*}
\left[ \frac{{w}^{EM}_i}{{w}^{EM}_j} \right] =
\begin{pmatrix}
$\,\,$ 1 $\,\,$ & $\,\,$\color{red} 1.8959\color{black} $\,\,$ & $\,\,$\color{red} 2.8732\color{black} $\,\,$ & $\,\,$\color{red} 8.6944\color{black} $\,\,$ \\
$\,\,$\color{red} 0.5275\color{black} $\,\,$ & $\,\,$ 1 $\,\,$ & $\,\,$1.5155$\,\,$ & $\,\,$4.5859  $\,\,$ \\
$\,\,$\color{red} 0.3480\color{black} $\,\,$ & $\,\,$0.6598$\,\,$ & $\,\,$ 1 $\,\,$ & $\,\,$3.0260 $\,\,$ \\
$\,\,$\color{red} 0.1150\color{black} $\,\,$ & $\,\,$0.2181$\,\,$ & $\,\,$0.3305$\,\,$ & $\,\,$ 1  $\,\,$ \\
\end{pmatrix},
\end{equation*}

\begin{equation*}
\mathbf{w}^{\prime} =
\begin{pmatrix}
0.511016\\
0.260388\\
0.171816\\
0.056780
\end{pmatrix} =
0.982649\cdot
\begin{pmatrix}
\color{gr} 0.520040\color{black} \\
0.264986\\
0.174850\\
0.057782
\end{pmatrix},
\end{equation*}
\begin{equation*}
\left[ \frac{{w}^{\prime}_i}{{w}^{\prime}_j} \right] =
\begin{pmatrix}
$\,\,$ 1 $\,\,$ & $\,\,$\color{gr} 1.9625\color{black} $\,\,$ & $\,\,$\color{gr} 2.9742\color{black} $\,\,$ & $\,\,$\color{gr} \color{blue} 9\color{black} $\,\,$ \\
$\,\,$\color{gr} 0.5095\color{black} $\,\,$ & $\,\,$ 1 $\,\,$ & $\,\,$1.5155$\,\,$ & $\,\,$4.5859  $\,\,$ \\
$\,\,$\color{gr} 0.3362\color{black} $\,\,$ & $\,\,$0.6598$\,\,$ & $\,\,$ 1 $\,\,$ & $\,\,$3.0260 $\,\,$ \\
$\,\,$\color{gr} \color{blue}  1/9\color{black} $\,\,$ & $\,\,$0.2181$\,\,$ & $\,\,$0.3305$\,\,$ & $\,\,$ 1  $\,\,$ \\
\end{pmatrix},
\end{equation*}
\end{example}
\newpage
\begin{example}
\begin{equation*}
\mathbf{A} =
\begin{pmatrix}
$\,\,$ 1 $\,\,$ & $\,\,$2$\,\,$ & $\,\,$3$\,\,$ & $\,\,$9 $\,\,$ \\
$\,\,$ 1/2$\,\,$ & $\,\,$ 1 $\,\,$ & $\,\,$1$\,\,$ & $\,\,$8 $\,\,$ \\
$\,\,$ 1/3$\,\,$ & $\,\,$ 1 $\,\,$ & $\,\,$ 1 $\,\,$ & $\,\,$2 $\,\,$ \\
$\,\,$ 1/9$\,\,$ & $\,\,$ 1/8$\,\,$ & $\,\,$ 1/2$\,\,$ & $\,\,$ 1  $\,\,$ \\
\end{pmatrix},
\qquad
\lambda_{\max} =
4.1664,
\qquad
CR = 0.0627
\end{equation*}

\begin{equation*}
\mathbf{w}^{EM} =
\begin{pmatrix}
\color{red} 0.496267\color{black} \\
0.274050\\
0.173978\\
0.055705
\end{pmatrix}\end{equation*}
\begin{equation*}
\left[ \frac{{w}^{EM}_i}{{w}^{EM}_j} \right] =
\begin{pmatrix}
$\,\,$ 1 $\,\,$ & $\,\,$\color{red} 1.8109\color{black} $\,\,$ & $\,\,$\color{red} 2.8525\color{black} $\,\,$ & $\,\,$\color{red} 8.9088\color{black} $\,\,$ \\
$\,\,$\color{red} 0.5522\color{black} $\,\,$ & $\,\,$ 1 $\,\,$ & $\,\,$1.5752$\,\,$ & $\,\,$4.9196  $\,\,$ \\
$\,\,$\color{red} 0.3506\color{black} $\,\,$ & $\,\,$0.6348$\,\,$ & $\,\,$ 1 $\,\,$ & $\,\,$3.1232 $\,\,$ \\
$\,\,$\color{red} 0.1122\color{black} $\,\,$ & $\,\,$0.2033$\,\,$ & $\,\,$0.3202$\,\,$ & $\,\,$ 1  $\,\,$ \\
\end{pmatrix},
\end{equation*}

\begin{equation*}
\mathbf{w}^{\prime} =
\begin{pmatrix}
0.498814\\
0.272665\\
0.173098\\
0.055424
\end{pmatrix} =
0.994944\cdot
\begin{pmatrix}
\color{gr} 0.501348\color{black} \\
0.274050\\
0.173978\\
0.055705
\end{pmatrix},
\end{equation*}
\begin{equation*}
\left[ \frac{{w}^{\prime}_i}{{w}^{\prime}_j} \right] =
\begin{pmatrix}
$\,\,$ 1 $\,\,$ & $\,\,$\color{gr} 1.8294\color{black} $\,\,$ & $\,\,$\color{gr} 2.8817\color{black} $\,\,$ & $\,\,$\color{gr} \color{blue} 9\color{black} $\,\,$ \\
$\,\,$\color{gr} 0.5466\color{black} $\,\,$ & $\,\,$ 1 $\,\,$ & $\,\,$1.5752$\,\,$ & $\,\,$4.9196  $\,\,$ \\
$\,\,$\color{gr} 0.3470\color{black} $\,\,$ & $\,\,$0.6348$\,\,$ & $\,\,$ 1 $\,\,$ & $\,\,$3.1232 $\,\,$ \\
$\,\,$\color{gr} \color{blue}  1/9\color{black} $\,\,$ & $\,\,$0.2033$\,\,$ & $\,\,$0.3202$\,\,$ & $\,\,$ 1  $\,\,$ \\
\end{pmatrix},
\end{equation*}
\end{example}
\newpage
\begin{example}
\begin{equation*}
\mathbf{A} =
\begin{pmatrix}
$\,\,$ 1 $\,\,$ & $\,\,$2$\,\,$ & $\,\,$4$\,\,$ & $\,\,$6 $\,\,$ \\
$\,\,$ 1/2$\,\,$ & $\,\,$ 1 $\,\,$ & $\,\,$3$\,\,$ & $\,\,$2 $\,\,$ \\
$\,\,$ 1/4$\,\,$ & $\,\,$ 1/3$\,\,$ & $\,\,$ 1 $\,\,$ & $\,\,$2 $\,\,$ \\
$\,\,$ 1/6$\,\,$ & $\,\,$ 1/2$\,\,$ & $\,\,$ 1/2$\,\,$ & $\,\,$ 1  $\,\,$ \\
\end{pmatrix},
\qquad
\lambda_{\max} =
4.1031,
\qquad
CR = 0.0389
\end{equation*}

\begin{equation*}
\mathbf{w}^{EM} =
\begin{pmatrix}
\color{red} 0.513893\color{black} \\
0.266099\\
0.128781\\
0.091227
\end{pmatrix}\end{equation*}
\begin{equation*}
\left[ \frac{{w}^{EM}_i}{{w}^{EM}_j} \right] =
\begin{pmatrix}
$\,\,$ 1 $\,\,$ & $\,\,$\color{red} 1.9312\color{black} $\,\,$ & $\,\,$\color{red} 3.9904\color{black} $\,\,$ & $\,\,$\color{red} 5.6331\color{black} $\,\,$ \\
$\,\,$\color{red} 0.5178\color{black} $\,\,$ & $\,\,$ 1 $\,\,$ & $\,\,$2.0663$\,\,$ & $\,\,$2.9169  $\,\,$ \\
$\,\,$\color{red} 0.2506\color{black} $\,\,$ & $\,\,$0.4840$\,\,$ & $\,\,$ 1 $\,\,$ & $\,\,$1.4117 $\,\,$ \\
$\,\,$\color{red} 0.1775\color{black} $\,\,$ & $\,\,$0.3428$\,\,$ & $\,\,$0.7084$\,\,$ & $\,\,$ 1  $\,\,$ \\
\end{pmatrix},
\end{equation*}

\begin{equation*}
\mathbf{w}^{\prime} =
\begin{pmatrix}
0.514491\\
0.265772\\
0.128623\\
0.091114
\end{pmatrix} =
0.998770\cdot
\begin{pmatrix}
\color{gr} 0.515125\color{black} \\
0.266099\\
0.128781\\
0.091227
\end{pmatrix},
\end{equation*}
\begin{equation*}
\left[ \frac{{w}^{\prime}_i}{{w}^{\prime}_j} \right] =
\begin{pmatrix}
$\,\,$ 1 $\,\,$ & $\,\,$\color{gr} 1.9358\color{black} $\,\,$ & $\,\,$\color{gr} \color{blue} 4\color{black} $\,\,$ & $\,\,$\color{gr} 5.6467\color{black} $\,\,$ \\
$\,\,$\color{gr} 0.5166\color{black} $\,\,$ & $\,\,$ 1 $\,\,$ & $\,\,$2.0663$\,\,$ & $\,\,$2.9169  $\,\,$ \\
$\,\,$\color{gr} \color{blue}  1/4\color{black} $\,\,$ & $\,\,$0.4840$\,\,$ & $\,\,$ 1 $\,\,$ & $\,\,$1.4117 $\,\,$ \\
$\,\,$\color{gr} 0.1771\color{black} $\,\,$ & $\,\,$0.3428$\,\,$ & $\,\,$0.7084$\,\,$ & $\,\,$ 1  $\,\,$ \\
\end{pmatrix},
\end{equation*}
\end{example}
\newpage
\begin{example}
\begin{equation*}
\mathbf{A} =
\begin{pmatrix}
$\,\,$ 1 $\,\,$ & $\,\,$2$\,\,$ & $\,\,$4$\,\,$ & $\,\,$6 $\,\,$ \\
$\,\,$ 1/2$\,\,$ & $\,\,$ 1 $\,\,$ & $\,\,$7$\,\,$ & $\,\,$4 $\,\,$ \\
$\,\,$ 1/4$\,\,$ & $\,\,$ 1/7$\,\,$ & $\,\,$ 1 $\,\,$ & $\,\,$1 $\,\,$ \\
$\,\,$ 1/6$\,\,$ & $\,\,$ 1/4$\,\,$ & $\,\,$ 1 $\,\,$ & $\,\,$ 1  $\,\,$ \\
\end{pmatrix},
\qquad
\lambda_{\max} =
4.1365,
\qquad
CR = 0.0515
\end{equation*}

\begin{equation*}
\mathbf{w}^{EM} =
\begin{pmatrix}
0.482649\\
0.357489\\
0.080163\\
\color{red} 0.079699\color{black}
\end{pmatrix}\end{equation*}
\begin{equation*}
\left[ \frac{{w}^{EM}_i}{{w}^{EM}_j} \right] =
\begin{pmatrix}
$\,\,$ 1 $\,\,$ & $\,\,$1.3501$\,\,$ & $\,\,$6.0208$\,\,$ & $\,\,$\color{red} 6.0559\color{black} $\,\,$ \\
$\,\,$0.7407$\,\,$ & $\,\,$ 1 $\,\,$ & $\,\,$4.4595$\,\,$ & $\,\,$\color{red} 4.4855\color{black}   $\,\,$ \\
$\,\,$0.1661$\,\,$ & $\,\,$0.2242$\,\,$ & $\,\,$ 1 $\,\,$ & $\,\,$\color{red} 1.0058\color{black}  $\,\,$ \\
$\,\,$\color{red} 0.1651\color{black} $\,\,$ & $\,\,$\color{red} 0.2229\color{black} $\,\,$ & $\,\,$\color{red} 0.9942\color{black} $\,\,$ & $\,\,$ 1  $\,\,$ \\
\end{pmatrix},
\end{equation*}

\begin{equation*}
\mathbf{w}^{\prime} =
\begin{pmatrix}
0.482425\\
0.357323\\
0.080126\\
0.080126
\end{pmatrix} =
0.999536\cdot
\begin{pmatrix}
0.482649\\
0.357489\\
0.080163\\
\color{gr} 0.080163\color{black}
\end{pmatrix},
\end{equation*}
\begin{equation*}
\left[ \frac{{w}^{\prime}_i}{{w}^{\prime}_j} \right] =
\begin{pmatrix}
$\,\,$ 1 $\,\,$ & $\,\,$1.3501$\,\,$ & $\,\,$6.0208$\,\,$ & $\,\,$\color{gr} 6.0208\color{black} $\,\,$ \\
$\,\,$0.7407$\,\,$ & $\,\,$ 1 $\,\,$ & $\,\,$4.4595$\,\,$ & $\,\,$\color{gr} 4.4595\color{black}   $\,\,$ \\
$\,\,$0.1661$\,\,$ & $\,\,$0.2242$\,\,$ & $\,\,$ 1 $\,\,$ & $\,\,$\color{gr} \color{blue} 1\color{black}  $\,\,$ \\
$\,\,$\color{gr} 0.1661\color{black} $\,\,$ & $\,\,$\color{gr} 0.2242\color{black} $\,\,$ & $\,\,$\color{gr} \color{blue} 1\color{black} $\,\,$ & $\,\,$ 1  $\,\,$ \\
\end{pmatrix},
\end{equation*}
\end{example}
\newpage
\begin{example}
\begin{equation*}
\mathbf{A} =
\begin{pmatrix}
$\,\,$ 1 $\,\,$ & $\,\,$2$\,\,$ & $\,\,$4$\,\,$ & $\,\,$7 $\,\,$ \\
$\,\,$ 1/2$\,\,$ & $\,\,$ 1 $\,\,$ & $\,\,$3$\,\,$ & $\,\,$2 $\,\,$ \\
$\,\,$ 1/4$\,\,$ & $\,\,$ 1/3$\,\,$ & $\,\,$ 1 $\,\,$ & $\,\,$3 $\,\,$ \\
$\,\,$ 1/7$\,\,$ & $\,\,$ 1/2$\,\,$ & $\,\,$ 1/3$\,\,$ & $\,\,$ 1  $\,\,$ \\
\end{pmatrix},
\qquad
\lambda_{\max} =
4.1964,
\qquad
CR = 0.0741
\end{equation*}

\begin{equation*}
\mathbf{w}^{EM} =
\begin{pmatrix}
\color{red} 0.515592\color{black} \\
0.263375\\
0.141984\\
0.079049
\end{pmatrix}\end{equation*}
\begin{equation*}
\left[ \frac{{w}^{EM}_i}{{w}^{EM}_j} \right] =
\begin{pmatrix}
$\,\,$ 1 $\,\,$ & $\,\,$\color{red} 1.9576\color{black} $\,\,$ & $\,\,$\color{red} 3.6313\color{black} $\,\,$ & $\,\,$\color{red} 6.5224\color{black} $\,\,$ \\
$\,\,$\color{red} 0.5108\color{black} $\,\,$ & $\,\,$ 1 $\,\,$ & $\,\,$1.8550$\,\,$ & $\,\,$3.3318  $\,\,$ \\
$\,\,$\color{red} 0.2754\color{black} $\,\,$ & $\,\,$0.5391$\,\,$ & $\,\,$ 1 $\,\,$ & $\,\,$1.7962 $\,\,$ \\
$\,\,$\color{red} 0.1533\color{black} $\,\,$ & $\,\,$0.3001$\,\,$ & $\,\,$0.5567$\,\,$ & $\,\,$ 1  $\,\,$ \\
\end{pmatrix},
\end{equation*}

\begin{equation*}
\mathbf{w}^{\prime} =
\begin{pmatrix}
0.520937\\
0.260469\\
0.140418\\
0.078177
\end{pmatrix} =
0.988966\cdot
\begin{pmatrix}
\color{gr} 0.526749\color{black} \\
0.263375\\
0.141984\\
0.079049
\end{pmatrix},
\end{equation*}
\begin{equation*}
\left[ \frac{{w}^{\prime}_i}{{w}^{\prime}_j} \right] =
\begin{pmatrix}
$\,\,$ 1 $\,\,$ & $\,\,$\color{gr} \color{blue} 2\color{black} $\,\,$ & $\,\,$\color{gr} 3.7099\color{black} $\,\,$ & $\,\,$\color{gr} 6.6636\color{black} $\,\,$ \\
$\,\,$\color{gr} \color{blue}  1/2\color{black} $\,\,$ & $\,\,$ 1 $\,\,$ & $\,\,$1.8550$\,\,$ & $\,\,$3.3318  $\,\,$ \\
$\,\,$\color{gr} 0.2695\color{black} $\,\,$ & $\,\,$0.5391$\,\,$ & $\,\,$ 1 $\,\,$ & $\,\,$1.7962 $\,\,$ \\
$\,\,$\color{gr} 0.1501\color{black} $\,\,$ & $\,\,$0.3001$\,\,$ & $\,\,$0.5567$\,\,$ & $\,\,$ 1  $\,\,$ \\
\end{pmatrix},
\end{equation*}
\end{example}
\newpage
\begin{example}
\begin{equation*}
\mathbf{A} =
\begin{pmatrix}
$\,\,$ 1 $\,\,$ & $\,\,$2$\,\,$ & $\,\,$5$\,\,$ & $\,\,$3 $\,\,$ \\
$\,\,$ 1/2$\,\,$ & $\,\,$ 1 $\,\,$ & $\,\,$4$\,\,$ & $\,\,$6 $\,\,$ \\
$\,\,$ 1/5$\,\,$ & $\,\,$ 1/4$\,\,$ & $\,\,$ 1 $\,\,$ & $\,\,$1 $\,\,$ \\
$\,\,$ 1/3$\,\,$ & $\,\,$ 1/6$\,\,$ & $\,\,$ 1 $\,\,$ & $\,\,$ 1  $\,\,$ \\
\end{pmatrix},
\qquad
\lambda_{\max} =
4.1655,
\qquad
CR = 0.0624
\end{equation*}

\begin{equation*}
\mathbf{w}^{EM} =
\begin{pmatrix}
0.456212\\
0.361814\\
\color{red} 0.087304\color{black} \\
0.094669
\end{pmatrix}\end{equation*}
\begin{equation*}
\left[ \frac{{w}^{EM}_i}{{w}^{EM}_j} \right] =
\begin{pmatrix}
$\,\,$ 1 $\,\,$ & $\,\,$1.2609$\,\,$ & $\,\,$\color{red} 5.2255\color{black} $\,\,$ & $\,\,$4.8190$\,\,$ \\
$\,\,$0.7931$\,\,$ & $\,\,$ 1 $\,\,$ & $\,\,$\color{red} 4.1443\color{black} $\,\,$ & $\,\,$3.8219  $\,\,$ \\
$\,\,$\color{red} 0.1914\color{black} $\,\,$ & $\,\,$\color{red} 0.2413\color{black} $\,\,$ & $\,\,$ 1 $\,\,$ & $\,\,$\color{red} 0.9222\color{black}  $\,\,$ \\
$\,\,$0.2075$\,\,$ & $\,\,$0.2617$\,\,$ & $\,\,$\color{red} 1.0844\color{black} $\,\,$ & $\,\,$ 1  $\,\,$ \\
\end{pmatrix},
\end{equation*}

\begin{equation*}
\mathbf{w}^{\prime} =
\begin{pmatrix}
0.454780\\
0.360679\\
0.090170\\
0.094372
\end{pmatrix} =
0.996861\cdot
\begin{pmatrix}
0.456212\\
0.361814\\
\color{gr} 0.090454\color{black} \\
0.094669
\end{pmatrix},
\end{equation*}
\begin{equation*}
\left[ \frac{{w}^{\prime}_i}{{w}^{\prime}_j} \right] =
\begin{pmatrix}
$\,\,$ 1 $\,\,$ & $\,\,$1.2609$\,\,$ & $\,\,$\color{gr} 5.0436\color{black} $\,\,$ & $\,\,$4.8190$\,\,$ \\
$\,\,$0.7931$\,\,$ & $\,\,$ 1 $\,\,$ & $\,\,$\color{gr} \color{blue} 4\color{black} $\,\,$ & $\,\,$3.8219  $\,\,$ \\
$\,\,$\color{gr} 0.1983\color{black} $\,\,$ & $\,\,$\color{gr} \color{blue}  1/4\color{black} $\,\,$ & $\,\,$ 1 $\,\,$ & $\,\,$\color{gr} 0.9555\color{black}  $\,\,$ \\
$\,\,$0.2075$\,\,$ & $\,\,$0.2617$\,\,$ & $\,\,$\color{gr} 1.0466\color{black} $\,\,$ & $\,\,$ 1  $\,\,$ \\
\end{pmatrix},
\end{equation*}
\end{example}
\newpage
\begin{example}
\begin{equation*}
\mathbf{A} =
\begin{pmatrix}
$\,\,$ 1 $\,\,$ & $\,\,$2$\,\,$ & $\,\,$5$\,\,$ & $\,\,$3 $\,\,$ \\
$\,\,$ 1/2$\,\,$ & $\,\,$ 1 $\,\,$ & $\,\,$4$\,\,$ & $\,\,$7 $\,\,$ \\
$\,\,$ 1/5$\,\,$ & $\,\,$ 1/4$\,\,$ & $\,\,$ 1 $\,\,$ & $\,\,$1 $\,\,$ \\
$\,\,$ 1/3$\,\,$ & $\,\,$ 1/7$\,\,$ & $\,\,$ 1 $\,\,$ & $\,\,$ 1  $\,\,$ \\
\end{pmatrix},
\qquad
\lambda_{\max} =
4.2057,
\qquad
CR = 0.0776
\end{equation*}

\begin{equation*}
\mathbf{w}^{EM} =
\begin{pmatrix}
0.450725\\
0.373725\\
\color{red} 0.085391\color{black} \\
0.090159
\end{pmatrix}\end{equation*}
\begin{equation*}
\left[ \frac{{w}^{EM}_i}{{w}^{EM}_j} \right] =
\begin{pmatrix}
$\,\,$ 1 $\,\,$ & $\,\,$1.2060$\,\,$ & $\,\,$\color{red} 5.2784\color{black} $\,\,$ & $\,\,$4.9992$\,\,$ \\
$\,\,$0.8292$\,\,$ & $\,\,$ 1 $\,\,$ & $\,\,$\color{red} 4.3766\color{black} $\,\,$ & $\,\,$4.1452  $\,\,$ \\
$\,\,$\color{red} 0.1895\color{black} $\,\,$ & $\,\,$\color{red} 0.2285\color{black} $\,\,$ & $\,\,$ 1 $\,\,$ & $\,\,$\color{red} 0.9471\color{black}  $\,\,$ \\
$\,\,$0.2000$\,\,$ & $\,\,$0.2412$\,\,$ & $\,\,$\color{red} 1.0558\color{black} $\,\,$ & $\,\,$ 1  $\,\,$ \\
\end{pmatrix},
\end{equation*}

\begin{equation*}
\mathbf{w}^{\prime} =
\begin{pmatrix}
0.448593\\
0.371956\\
0.089719\\
0.089733
\end{pmatrix} =
0.995268\cdot
\begin{pmatrix}
0.450725\\
0.373725\\
\color{gr} 0.090145\color{black} \\
0.090159
\end{pmatrix},
\end{equation*}
\begin{equation*}
\left[ \frac{{w}^{\prime}_i}{{w}^{\prime}_j} \right] =
\begin{pmatrix}
$\,\,$ 1 $\,\,$ & $\,\,$1.2060$\,\,$ & $\,\,$\color{gr} \color{blue} 5\color{black} $\,\,$ & $\,\,$4.9992$\,\,$ \\
$\,\,$0.8292$\,\,$ & $\,\,$ 1 $\,\,$ & $\,\,$\color{gr} 4.1458\color{black} $\,\,$ & $\,\,$4.1452  $\,\,$ \\
$\,\,$\color{gr} \color{blue}  1/5\color{black} $\,\,$ & $\,\,$\color{gr} 0.2412\color{black} $\,\,$ & $\,\,$ 1 $\,\,$ & $\,\,$\color{gr} 0.9998\color{black}  $\,\,$ \\
$\,\,$0.2000$\,\,$ & $\,\,$0.2412$\,\,$ & $\,\,$\color{gr} 1.0002\color{black} $\,\,$ & $\,\,$ 1  $\,\,$ \\
\end{pmatrix},
\end{equation*}
\end{example}
\newpage
\begin{example}
\begin{equation*}
\mathbf{A} =
\begin{pmatrix}
$\,\,$ 1 $\,\,$ & $\,\,$2$\,\,$ & $\,\,$5$\,\,$ & $\,\,$3 $\,\,$ \\
$\,\,$ 1/2$\,\,$ & $\,\,$ 1 $\,\,$ & $\,\,$4$\,\,$ & $\,\,$8 $\,\,$ \\
$\,\,$ 1/5$\,\,$ & $\,\,$ 1/4$\,\,$ & $\,\,$ 1 $\,\,$ & $\,\,$1 $\,\,$ \\
$\,\,$ 1/3$\,\,$ & $\,\,$ 1/8$\,\,$ & $\,\,$ 1 $\,\,$ & $\,\,$ 1  $\,\,$ \\
\end{pmatrix},
\qquad
\lambda_{\max} =
4.2460,
\qquad
CR = 0.0928
\end{equation*}

\begin{equation*}
\mathbf{w}^{EM} =
\begin{pmatrix}
0.445536\\
0.384478\\
\color{red} 0.083657\color{black} \\
0.086329
\end{pmatrix}\end{equation*}
\begin{equation*}
\left[ \frac{{w}^{EM}_i}{{w}^{EM}_j} \right] =
\begin{pmatrix}
$\,\,$ 1 $\,\,$ & $\,\,$1.1588$\,\,$ & $\,\,$\color{red} 5.3257\color{black} $\,\,$ & $\,\,$5.1609$\,\,$ \\
$\,\,$0.8630$\,\,$ & $\,\,$ 1 $\,\,$ & $\,\,$\color{red} 4.5959\color{black} $\,\,$ & $\,\,$4.4536  $\,\,$ \\
$\,\,$\color{red} 0.1878\color{black} $\,\,$ & $\,\,$\color{red} 0.2176\color{black} $\,\,$ & $\,\,$ 1 $\,\,$ & $\,\,$\color{red} 0.9690\color{black}  $\,\,$ \\
$\,\,$0.1938$\,\,$ & $\,\,$0.2245$\,\,$ & $\,\,$\color{red} 1.0319\color{black} $\,\,$ & $\,\,$ 1  $\,\,$ \\
\end{pmatrix},
\end{equation*}

\begin{equation*}
\mathbf{w}^{\prime} =
\begin{pmatrix}
0.444349\\
0.383453\\
0.086099\\
0.086099
\end{pmatrix} =
0.997335\cdot
\begin{pmatrix}
0.445536\\
0.384478\\
\color{gr} 0.086329\color{black} \\
0.086329
\end{pmatrix},
\end{equation*}
\begin{equation*}
\left[ \frac{{w}^{\prime}_i}{{w}^{\prime}_j} \right] =
\begin{pmatrix}
$\,\,$ 1 $\,\,$ & $\,\,$1.1588$\,\,$ & $\,\,$\color{gr} 5.1609\color{black} $\,\,$ & $\,\,$5.1609$\,\,$ \\
$\,\,$0.8630$\,\,$ & $\,\,$ 1 $\,\,$ & $\,\,$\color{gr} 4.4536\color{black} $\,\,$ & $\,\,$4.4536  $\,\,$ \\
$\,\,$\color{gr} 0.1938\color{black} $\,\,$ & $\,\,$\color{gr} 0.2245\color{black} $\,\,$ & $\,\,$ 1 $\,\,$ & $\,\,$\color{gr} \color{blue} 1\color{black}  $\,\,$ \\
$\,\,$0.1938$\,\,$ & $\,\,$0.2245$\,\,$ & $\,\,$\color{gr} \color{blue} 1\color{black} $\,\,$ & $\,\,$ 1  $\,\,$ \\
\end{pmatrix},
\end{equation*}
\end{example}
\newpage
\begin{example}
\begin{equation*}
\mathbf{A} =
\begin{pmatrix}
$\,\,$ 1 $\,\,$ & $\,\,$2$\,\,$ & $\,\,$5$\,\,$ & $\,\,$3 $\,\,$ \\
$\,\,$ 1/2$\,\,$ & $\,\,$ 1 $\,\,$ & $\,\,$5$\,\,$ & $\,\,$8 $\,\,$ \\
$\,\,$ 1/5$\,\,$ & $\,\,$ 1/5$\,\,$ & $\,\,$ 1 $\,\,$ & $\,\,$1 $\,\,$ \\
$\,\,$ 1/3$\,\,$ & $\,\,$ 1/8$\,\,$ & $\,\,$ 1 $\,\,$ & $\,\,$ 1  $\,\,$ \\
\end{pmatrix},
\qquad
\lambda_{\max} =
4.2460,
\qquad
CR = 0.0928
\end{equation*}

\begin{equation*}
\mathbf{w}^{EM} =
\begin{pmatrix}
0.441767\\
0.396028\\
\color{red} 0.077664\color{black} \\
0.084541
\end{pmatrix}\end{equation*}
\begin{equation*}
\left[ \frac{{w}^{EM}_i}{{w}^{EM}_j} \right] =
\begin{pmatrix}
$\,\,$ 1 $\,\,$ & $\,\,$1.1155$\,\,$ & $\,\,$\color{red} 5.6882\color{black} $\,\,$ & $\,\,$5.2255$\,\,$ \\
$\,\,$0.8965$\,\,$ & $\,\,$ 1 $\,\,$ & $\,\,$\color{red} 5.0993\color{black} $\,\,$ & $\,\,$4.6845  $\,\,$ \\
$\,\,$\color{red} 0.1758\color{black} $\,\,$ & $\,\,$\color{red} 0.1961\color{black} $\,\,$ & $\,\,$ 1 $\,\,$ & $\,\,$\color{red} 0.9187\color{black}  $\,\,$ \\
$\,\,$0.1914$\,\,$ & $\,\,$0.2135$\,\,$ & $\,\,$\color{red} 1.0885\color{black} $\,\,$ & $\,\,$ 1  $\,\,$ \\
\end{pmatrix},
\end{equation*}

\begin{equation*}
\mathbf{w}^{\prime} =
\begin{pmatrix}
0.441087\\
0.395419\\
0.079084\\
0.084410
\end{pmatrix} =
0.998460\cdot
\begin{pmatrix}
0.441767\\
0.396028\\
\color{gr} 0.079206\color{black} \\
0.084541
\end{pmatrix},
\end{equation*}
\begin{equation*}
\left[ \frac{{w}^{\prime}_i}{{w}^{\prime}_j} \right] =
\begin{pmatrix}
$\,\,$ 1 $\,\,$ & $\,\,$1.1155$\,\,$ & $\,\,$\color{gr} 5.5775\color{black} $\,\,$ & $\,\,$5.2255$\,\,$ \\
$\,\,$0.8965$\,\,$ & $\,\,$ 1 $\,\,$ & $\,\,$\color{gr} \color{blue} 5\color{black} $\,\,$ & $\,\,$4.6845  $\,\,$ \\
$\,\,$\color{gr} 0.1793\color{black} $\,\,$ & $\,\,$\color{gr} \color{blue}  1/5\color{black} $\,\,$ & $\,\,$ 1 $\,\,$ & $\,\,$\color{gr} 0.9369\color{black}  $\,\,$ \\
$\,\,$0.1914$\,\,$ & $\,\,$0.2135$\,\,$ & $\,\,$\color{gr} 1.0674\color{black} $\,\,$ & $\,\,$ 1  $\,\,$ \\
\end{pmatrix},
\end{equation*}
\end{example}
\newpage
\begin{example}
\begin{equation*}
\mathbf{A} =
\begin{pmatrix}
$\,\,$ 1 $\,\,$ & $\,\,$2$\,\,$ & $\,\,$5$\,\,$ & $\,\,$6 $\,\,$ \\
$\,\,$ 1/2$\,\,$ & $\,\,$ 1 $\,\,$ & $\,\,$4$\,\,$ & $\,\,$2 $\,\,$ \\
$\,\,$ 1/5$\,\,$ & $\,\,$ 1/4$\,\,$ & $\,\,$ 1 $\,\,$ & $\,\,$2 $\,\,$ \\
$\,\,$ 1/6$\,\,$ & $\,\,$ 1/2$\,\,$ & $\,\,$ 1/2$\,\,$ & $\,\,$ 1  $\,\,$ \\
\end{pmatrix},
\qquad
\lambda_{\max} =
4.1655,
\qquad
CR = 0.0624
\end{equation*}

\begin{equation*}
\mathbf{w}^{EM} =
\begin{pmatrix}
\color{red} 0.520722\color{black} \\
0.279025\\
0.111200\\
0.089053
\end{pmatrix}\end{equation*}
\begin{equation*}
\left[ \frac{{w}^{EM}_i}{{w}^{EM}_j} \right] =
\begin{pmatrix}
$\,\,$ 1 $\,\,$ & $\,\,$\color{red} 1.8662\color{black} $\,\,$ & $\,\,$\color{red} 4.6828\color{black} $\,\,$ & $\,\,$\color{red} 5.8473\color{black} $\,\,$ \\
$\,\,$\color{red} 0.5358\color{black} $\,\,$ & $\,\,$ 1 $\,\,$ & $\,\,$2.5092$\,\,$ & $\,\,$3.1333  $\,\,$ \\
$\,\,$\color{red} 0.2135\color{black} $\,\,$ & $\,\,$0.3985$\,\,$ & $\,\,$ 1 $\,\,$ & $\,\,$1.2487 $\,\,$ \\
$\,\,$\color{red} 0.1710\color{black} $\,\,$ & $\,\,$0.3192$\,\,$ & $\,\,$0.8008$\,\,$ & $\,\,$ 1  $\,\,$ \\
\end{pmatrix},
\end{equation*}

\begin{equation*}
\mathbf{w}^{\prime} =
\begin{pmatrix}
0.527150\\
0.275283\\
0.109708\\
0.087858
\end{pmatrix} =
0.986588\cdot
\begin{pmatrix}
\color{gr} 0.534316\color{black} \\
0.279025\\
0.111200\\
0.089053
\end{pmatrix},
\end{equation*}
\begin{equation*}
\left[ \frac{{w}^{\prime}_i}{{w}^{\prime}_j} \right] =
\begin{pmatrix}
$\,\,$ 1 $\,\,$ & $\,\,$\color{gr} 1.9149\color{black} $\,\,$ & $\,\,$\color{gr} 4.8050\color{black} $\,\,$ & $\,\,$\color{gr} \color{blue} 6\color{black} $\,\,$ \\
$\,\,$\color{gr} 0.5222\color{black} $\,\,$ & $\,\,$ 1 $\,\,$ & $\,\,$2.5092$\,\,$ & $\,\,$3.1333  $\,\,$ \\
$\,\,$\color{gr} 0.2081\color{black} $\,\,$ & $\,\,$0.3985$\,\,$ & $\,\,$ 1 $\,\,$ & $\,\,$1.2487 $\,\,$ \\
$\,\,$\color{gr} \color{blue}  1/6\color{black} $\,\,$ & $\,\,$0.3192$\,\,$ & $\,\,$0.8008$\,\,$ & $\,\,$ 1  $\,\,$ \\
\end{pmatrix},
\end{equation*}
\end{example}
\newpage
\begin{example}
\begin{equation*}
\mathbf{A} =
\begin{pmatrix}
$\,\,$ 1 $\,\,$ & $\,\,$2$\,\,$ & $\,\,$5$\,\,$ & $\,\,$7 $\,\,$ \\
$\,\,$ 1/2$\,\,$ & $\,\,$ 1 $\,\,$ & $\,\,$4$\,\,$ & $\,\,$2 $\,\,$ \\
$\,\,$ 1/5$\,\,$ & $\,\,$ 1/4$\,\,$ & $\,\,$ 1 $\,\,$ & $\,\,$2 $\,\,$ \\
$\,\,$ 1/7$\,\,$ & $\,\,$ 1/2$\,\,$ & $\,\,$ 1/2$\,\,$ & $\,\,$ 1  $\,\,$ \\
\end{pmatrix},
\qquad
\lambda_{\max} =
4.1665,
\qquad
CR = 0.0628
\end{equation*}

\begin{equation*}
\mathbf{w}^{EM} =
\begin{pmatrix}
\color{red} 0.532045\color{black} \\
0.274716\\
0.108694\\
0.084545
\end{pmatrix}\end{equation*}
\begin{equation*}
\left[ \frac{{w}^{EM}_i}{{w}^{EM}_j} \right] =
\begin{pmatrix}
$\,\,$ 1 $\,\,$ & $\,\,$\color{red} 1.9367\color{black} $\,\,$ & $\,\,$\color{red} 4.8949\color{black} $\,\,$ & $\,\,$\color{red} 6.2930\color{black} $\,\,$ \\
$\,\,$\color{red} 0.5163\color{black} $\,\,$ & $\,\,$ 1 $\,\,$ & $\,\,$2.5274$\,\,$ & $\,\,$3.2493  $\,\,$ \\
$\,\,$\color{red} 0.2043\color{black} $\,\,$ & $\,\,$0.3957$\,\,$ & $\,\,$ 1 $\,\,$ & $\,\,$1.2856 $\,\,$ \\
$\,\,$\color{red} 0.1589\color{black} $\,\,$ & $\,\,$0.3078$\,\,$ & $\,\,$0.7778$\,\,$ & $\,\,$ 1  $\,\,$ \\
\end{pmatrix},
\end{equation*}

\begin{equation*}
\mathbf{w}^{\prime} =
\begin{pmatrix}
0.537330\\
0.271613\\
0.107466\\
0.083590
\end{pmatrix} =
0.988705\cdot
\begin{pmatrix}
\color{gr} 0.543469\color{black} \\
0.274716\\
0.108694\\
0.084545
\end{pmatrix},
\end{equation*}
\begin{equation*}
\left[ \frac{{w}^{\prime}_i}{{w}^{\prime}_j} \right] =
\begin{pmatrix}
$\,\,$ 1 $\,\,$ & $\,\,$\color{gr} 1.9783\color{black} $\,\,$ & $\,\,$\color{gr} \color{blue} 5\color{black} $\,\,$ & $\,\,$\color{gr} 6.4282\color{black} $\,\,$ \\
$\,\,$\color{gr} 0.5055\color{black} $\,\,$ & $\,\,$ 1 $\,\,$ & $\,\,$2.5274$\,\,$ & $\,\,$3.2493  $\,\,$ \\
$\,\,$\color{gr} \color{blue}  1/5\color{black} $\,\,$ & $\,\,$0.3957$\,\,$ & $\,\,$ 1 $\,\,$ & $\,\,$1.2856 $\,\,$ \\
$\,\,$\color{gr} 0.1556\color{black} $\,\,$ & $\,\,$0.3078$\,\,$ & $\,\,$0.7778$\,\,$ & $\,\,$ 1  $\,\,$ \\
\end{pmatrix},
\end{equation*}
\end{example}
\newpage
\begin{example}
\begin{equation*}
\mathbf{A} =
\begin{pmatrix}
$\,\,$ 1 $\,\,$ & $\,\,$2$\,\,$ & $\,\,$5$\,\,$ & $\,\,$7 $\,\,$ \\
$\,\,$ 1/2$\,\,$ & $\,\,$ 1 $\,\,$ & $\,\,$8$\,\,$ & $\,\,$5 $\,\,$ \\
$\,\,$ 1/5$\,\,$ & $\,\,$ 1/8$\,\,$ & $\,\,$ 1 $\,\,$ & $\,\,$1 $\,\,$ \\
$\,\,$ 1/7$\,\,$ & $\,\,$ 1/5$\,\,$ & $\,\,$ 1 $\,\,$ & $\,\,$ 1  $\,\,$ \\
\end{pmatrix},
\qquad
\lambda_{\max} =
4.1159,
\qquad
CR = 0.0437
\end{equation*}

\begin{equation*}
\mathbf{w}^{EM} =
\begin{pmatrix}
0.497843\\
0.365367\\
0.068522\\
\color{red} 0.068268\color{black}
\end{pmatrix}\end{equation*}
\begin{equation*}
\left[ \frac{{w}^{EM}_i}{{w}^{EM}_j} \right] =
\begin{pmatrix}
$\,\,$ 1 $\,\,$ & $\,\,$1.3626$\,\,$ & $\,\,$7.2654$\,\,$ & $\,\,$\color{red} 7.2925\color{black} $\,\,$ \\
$\,\,$0.7339$\,\,$ & $\,\,$ 1 $\,\,$ & $\,\,$5.3321$\,\,$ & $\,\,$\color{red} 5.3519\color{black}   $\,\,$ \\
$\,\,$0.1376$\,\,$ & $\,\,$0.1875$\,\,$ & $\,\,$ 1 $\,\,$ & $\,\,$\color{red} 1.0037\color{black}  $\,\,$ \\
$\,\,$\color{red} 0.1371\color{black} $\,\,$ & $\,\,$\color{red} 0.1868\color{black} $\,\,$ & $\,\,$\color{red} 0.9963\color{black} $\,\,$ & $\,\,$ 1  $\,\,$ \\
\end{pmatrix},
\end{equation*}

\begin{equation*}
\mathbf{w}^{\prime} =
\begin{pmatrix}
0.497716\\
0.365274\\
0.068505\\
0.068505
\end{pmatrix} =
0.999746\cdot
\begin{pmatrix}
0.497843\\
0.365367\\
0.068522\\
\color{gr} 0.068522\color{black}
\end{pmatrix},
\end{equation*}
\begin{equation*}
\left[ \frac{{w}^{\prime}_i}{{w}^{\prime}_j} \right] =
\begin{pmatrix}
$\,\,$ 1 $\,\,$ & $\,\,$1.3626$\,\,$ & $\,\,$7.2654$\,\,$ & $\,\,$\color{gr} 7.2654\color{black} $\,\,$ \\
$\,\,$0.7339$\,\,$ & $\,\,$ 1 $\,\,$ & $\,\,$5.3321$\,\,$ & $\,\,$\color{gr} 5.3321\color{black}   $\,\,$ \\
$\,\,$0.1376$\,\,$ & $\,\,$0.1875$\,\,$ & $\,\,$ 1 $\,\,$ & $\,\,$\color{gr} \color{blue} 1\color{black}  $\,\,$ \\
$\,\,$\color{gr} 0.1376\color{black} $\,\,$ & $\,\,$\color{gr} 0.1875\color{black} $\,\,$ & $\,\,$\color{gr} \color{blue} 1\color{black} $\,\,$ & $\,\,$ 1  $\,\,$ \\
\end{pmatrix},
\end{equation*}
\end{example}
\newpage
\begin{example}
\begin{equation*}
\mathbf{A} =
\begin{pmatrix}
$\,\,$ 1 $\,\,$ & $\,\,$2$\,\,$ & $\,\,$6$\,\,$ & $\,\,$3 $\,\,$ \\
$\,\,$ 1/2$\,\,$ & $\,\,$ 1 $\,\,$ & $\,\,$5$\,\,$ & $\,\,$7 $\,\,$ \\
$\,\,$ 1/6$\,\,$ & $\,\,$ 1/5$\,\,$ & $\,\,$ 1 $\,\,$ & $\,\,$1 $\,\,$ \\
$\,\,$ 1/3$\,\,$ & $\,\,$ 1/7$\,\,$ & $\,\,$ 1 $\,\,$ & $\,\,$ 1  $\,\,$ \\
\end{pmatrix},
\qquad
\lambda_{\max} =
4.2095,
\qquad
CR = 0.0790
\end{equation*}

\begin{equation*}
\mathbf{w}^{EM} =
\begin{pmatrix}
0.458175\\
0.379286\\
\color{red} 0.074774\color{black} \\
0.087765
\end{pmatrix}\end{equation*}
\begin{equation*}
\left[ \frac{{w}^{EM}_i}{{w}^{EM}_j} \right] =
\begin{pmatrix}
$\,\,$ 1 $\,\,$ & $\,\,$1.2080$\,\,$ & $\,\,$\color{red} 6.1275\color{black} $\,\,$ & $\,\,$5.2205$\,\,$ \\
$\,\,$0.8278$\,\,$ & $\,\,$ 1 $\,\,$ & $\,\,$\color{red} 5.0725\color{black} $\,\,$ & $\,\,$4.3216  $\,\,$ \\
$\,\,$\color{red} 0.1632\color{black} $\,\,$ & $\,\,$\color{red} 0.1971\color{black} $\,\,$ & $\,\,$ 1 $\,\,$ & $\,\,$\color{red} 0.8520\color{black}  $\,\,$ \\
$\,\,$0.1916$\,\,$ & $\,\,$0.2314$\,\,$ & $\,\,$\color{red} 1.1737\color{black} $\,\,$ & $\,\,$ 1  $\,\,$ \\
\end{pmatrix},
\end{equation*}

\begin{equation*}
\mathbf{w}^{\prime} =
\begin{pmatrix}
0.457679\\
0.378875\\
0.075775\\
0.087670
\end{pmatrix} =
0.998918\cdot
\begin{pmatrix}
0.458175\\
0.379286\\
\color{gr} 0.075857\color{black} \\
0.087765
\end{pmatrix},
\end{equation*}
\begin{equation*}
\left[ \frac{{w}^{\prime}_i}{{w}^{\prime}_j} \right] =
\begin{pmatrix}
$\,\,$ 1 $\,\,$ & $\,\,$1.2080$\,\,$ & $\,\,$\color{gr} 6.0400\color{black} $\,\,$ & $\,\,$5.2205$\,\,$ \\
$\,\,$0.8278$\,\,$ & $\,\,$ 1 $\,\,$ & $\,\,$\color{gr} \color{blue} 5\color{black} $\,\,$ & $\,\,$4.3216  $\,\,$ \\
$\,\,$\color{gr} 0.1656\color{black} $\,\,$ & $\,\,$\color{gr} \color{blue}  1/5\color{black} $\,\,$ & $\,\,$ 1 $\,\,$ & $\,\,$\color{gr} 0.8643\color{black}  $\,\,$ \\
$\,\,$0.1916$\,\,$ & $\,\,$0.2314$\,\,$ & $\,\,$\color{gr} 1.1570\color{black} $\,\,$ & $\,\,$ 1  $\,\,$ \\
\end{pmatrix},
\end{equation*}
\end{example}
\newpage
\begin{example}
\begin{equation*}
\mathbf{A} =
\begin{pmatrix}
$\,\,$ 1 $\,\,$ & $\,\,$2$\,\,$ & $\,\,$6$\,\,$ & $\,\,$3 $\,\,$ \\
$\,\,$ 1/2$\,\,$ & $\,\,$ 1 $\,\,$ & $\,\,$5$\,\,$ & $\,\,$8 $\,\,$ \\
$\,\,$ 1/6$\,\,$ & $\,\,$ 1/5$\,\,$ & $\,\,$ 1 $\,\,$ & $\,\,$1 $\,\,$ \\
$\,\,$ 1/3$\,\,$ & $\,\,$ 1/8$\,\,$ & $\,\,$ 1 $\,\,$ & $\,\,$ 1  $\,\,$ \\
\end{pmatrix},
\qquad
\lambda_{\max} =
4.2460,
\qquad
CR = 0.0928
\end{equation*}

\begin{equation*}
\mathbf{w}^{EM} =
\begin{pmatrix}
0.453058\\
0.389697\\
\color{red} 0.073173\color{black} \\
0.084073
\end{pmatrix}\end{equation*}
\begin{equation*}
\left[ \frac{{w}^{EM}_i}{{w}^{EM}_j} \right] =
\begin{pmatrix}
$\,\,$ 1 $\,\,$ & $\,\,$1.1626$\,\,$ & $\,\,$\color{red} 6.1916\color{black} $\,\,$ & $\,\,$5.3889$\,\,$ \\
$\,\,$0.8601$\,\,$ & $\,\,$ 1 $\,\,$ & $\,\,$\color{red} 5.3257\color{black} $\,\,$ & $\,\,$4.6352  $\,\,$ \\
$\,\,$\color{red} 0.1615\color{black} $\,\,$ & $\,\,$\color{red} 0.1878\color{black} $\,\,$ & $\,\,$ 1 $\,\,$ & $\,\,$\color{red} 0.8703\color{black}  $\,\,$ \\
$\,\,$0.1856$\,\,$ & $\,\,$0.2157$\,\,$ & $\,\,$\color{red} 1.1490\color{black} $\,\,$ & $\,\,$ 1  $\,\,$ \\
\end{pmatrix},
\end{equation*}

\begin{equation*}
\mathbf{w}^{\prime} =
\begin{pmatrix}
0.452002\\
0.388788\\
0.075334\\
0.083877
\end{pmatrix} =
0.997668\cdot
\begin{pmatrix}
0.453058\\
0.389697\\
\color{gr} 0.075510\color{black} \\
0.084073
\end{pmatrix},
\end{equation*}
\begin{equation*}
\left[ \frac{{w}^{\prime}_i}{{w}^{\prime}_j} \right] =
\begin{pmatrix}
$\,\,$ 1 $\,\,$ & $\,\,$1.1626$\,\,$ & $\,\,$\color{gr} \color{blue} 6\color{black} $\,\,$ & $\,\,$5.3889$\,\,$ \\
$\,\,$0.8601$\,\,$ & $\,\,$ 1 $\,\,$ & $\,\,$\color{gr} 5.1609\color{black} $\,\,$ & $\,\,$4.6352  $\,\,$ \\
$\,\,$\color{gr} \color{blue}  1/6\color{black} $\,\,$ & $\,\,$\color{gr} 0.1938\color{black} $\,\,$ & $\,\,$ 1 $\,\,$ & $\,\,$\color{gr} 0.8981\color{black}  $\,\,$ \\
$\,\,$0.1856$\,\,$ & $\,\,$0.2157$\,\,$ & $\,\,$\color{gr} 1.1134\color{black} $\,\,$ & $\,\,$ 1  $\,\,$ \\
\end{pmatrix},
\end{equation*}
\end{example}
\newpage
\begin{example}
\begin{equation*}
\mathbf{A} =
\begin{pmatrix}
$\,\,$ 1 $\,\,$ & $\,\,$2$\,\,$ & $\,\,$6$\,\,$ & $\,\,$4 $\,\,$ \\
$\,\,$ 1/2$\,\,$ & $\,\,$ 1 $\,\,$ & $\,\,$2$\,\,$ & $\,\,$3 $\,\,$ \\
$\,\,$ 1/6$\,\,$ & $\,\,$ 1/2$\,\,$ & $\,\,$ 1 $\,\,$ & $\,\,$2 $\,\,$ \\
$\,\,$ 1/4$\,\,$ & $\,\,$ 1/3$\,\,$ & $\,\,$ 1/2$\,\,$ & $\,\,$ 1  $\,\,$ \\
\end{pmatrix},
\qquad
\lambda_{\max} =
4.1031,
\qquad
CR = 0.0389
\end{equation*}

\begin{equation*}
\mathbf{w}^{EM} =
\begin{pmatrix}
0.527371\\
\color{red} 0.254616\color{black} \\
0.127613\\
0.090399
\end{pmatrix}\end{equation*}
\begin{equation*}
\left[ \frac{{w}^{EM}_i}{{w}^{EM}_j} \right] =
\begin{pmatrix}
$\,\,$ 1 $\,\,$ & $\,\,$\color{red} 2.0712\color{black} $\,\,$ & $\,\,$4.1326$\,\,$ & $\,\,$5.8338$\,\,$ \\
$\,\,$\color{red} 0.4828\color{black} $\,\,$ & $\,\,$ 1 $\,\,$ & $\,\,$\color{red} 1.9952\color{black} $\,\,$ & $\,\,$\color{red} 2.8166\color{black}   $\,\,$ \\
$\,\,$0.2420$\,\,$ & $\,\,$\color{red} 0.5012\color{black} $\,\,$ & $\,\,$ 1 $\,\,$ & $\,\,$1.4117 $\,\,$ \\
$\,\,$0.1714$\,\,$ & $\,\,$\color{red} 0.3550\color{black} $\,\,$ & $\,\,$0.7084$\,\,$ & $\,\,$ 1  $\,\,$ \\
\end{pmatrix},
\end{equation*}

\begin{equation*}
\mathbf{w}^{\prime} =
\begin{pmatrix}
0.527050\\
0.255071\\
0.127535\\
0.090344
\end{pmatrix} =
0.999390\cdot
\begin{pmatrix}
0.527371\\
\color{gr} 0.255226\color{black} \\
0.127613\\
0.090399
\end{pmatrix},
\end{equation*}
\begin{equation*}
\left[ \frac{{w}^{\prime}_i}{{w}^{\prime}_j} \right] =
\begin{pmatrix}
$\,\,$ 1 $\,\,$ & $\,\,$\color{gr} 2.0663\color{black} $\,\,$ & $\,\,$4.1326$\,\,$ & $\,\,$5.8338$\,\,$ \\
$\,\,$\color{gr} 0.4840\color{black} $\,\,$ & $\,\,$ 1 $\,\,$ & $\,\,$\color{gr} \color{blue} 2\color{black} $\,\,$ & $\,\,$\color{gr} 2.8233\color{black}   $\,\,$ \\
$\,\,$0.2420$\,\,$ & $\,\,$\color{gr} \color{blue}  1/2\color{black} $\,\,$ & $\,\,$ 1 $\,\,$ & $\,\,$1.4117 $\,\,$ \\
$\,\,$0.1714$\,\,$ & $\,\,$\color{gr} 0.3542\color{black} $\,\,$ & $\,\,$0.7084$\,\,$ & $\,\,$ 1  $\,\,$ \\
\end{pmatrix},
\end{equation*}
\end{example}
\newpage
\begin{example}
\begin{equation*}
\mathbf{A} =
\begin{pmatrix}
$\,\,$ 1 $\,\,$ & $\,\,$2$\,\,$ & $\,\,$6$\,\,$ & $\,\,$4 $\,\,$ \\
$\,\,$ 1/2$\,\,$ & $\,\,$ 1 $\,\,$ & $\,\,$4$\,\,$ & $\,\,$6 $\,\,$ \\
$\,\,$ 1/6$\,\,$ & $\,\,$ 1/4$\,\,$ & $\,\,$ 1 $\,\,$ & $\,\,$1 $\,\,$ \\
$\,\,$ 1/4$\,\,$ & $\,\,$ 1/6$\,\,$ & $\,\,$ 1 $\,\,$ & $\,\,$ 1  $\,\,$ \\
\end{pmatrix},
\qquad
\lambda_{\max} =
4.1031,
\qquad
CR = 0.0389
\end{equation*}

\begin{equation*}
\mathbf{w}^{EM} =
\begin{pmatrix}
0.488557\\
0.346086\\
\color{red} 0.081231\color{black} \\
0.084125
\end{pmatrix}\end{equation*}
\begin{equation*}
\left[ \frac{{w}^{EM}_i}{{w}^{EM}_j} \right] =
\begin{pmatrix}
$\,\,$ 1 $\,\,$ & $\,\,$1.4117$\,\,$ & $\,\,$\color{red} 6.0144\color{black} $\,\,$ & $\,\,$5.8075$\,\,$ \\
$\,\,$0.7084$\,\,$ & $\,\,$ 1 $\,\,$ & $\,\,$\color{red} 4.2605\color{black} $\,\,$ & $\,\,$4.1140  $\,\,$ \\
$\,\,$\color{red} 0.1663\color{black} $\,\,$ & $\,\,$\color{red} 0.2347\color{black} $\,\,$ & $\,\,$ 1 $\,\,$ & $\,\,$\color{red} 0.9656\color{black}  $\,\,$ \\
$\,\,$0.1722$\,\,$ & $\,\,$0.2431$\,\,$ & $\,\,$\color{red} 1.0356\color{black} $\,\,$ & $\,\,$ 1  $\,\,$ \\
\end{pmatrix},
\end{equation*}

\begin{equation*}
\mathbf{w}^{\prime} =
\begin{pmatrix}
0.488462\\
0.346019\\
0.081410\\
0.084109
\end{pmatrix} =
0.999805\cdot
\begin{pmatrix}
0.488557\\
0.346086\\
\color{gr} 0.081426\color{black} \\
0.084125
\end{pmatrix},
\end{equation*}
\begin{equation*}
\left[ \frac{{w}^{\prime}_i}{{w}^{\prime}_j} \right] =
\begin{pmatrix}
$\,\,$ 1 $\,\,$ & $\,\,$1.4117$\,\,$ & $\,\,$\color{gr} \color{blue} 6\color{black} $\,\,$ & $\,\,$5.8075$\,\,$ \\
$\,\,$0.7084$\,\,$ & $\,\,$ 1 $\,\,$ & $\,\,$\color{gr} 4.2503\color{black} $\,\,$ & $\,\,$4.1140  $\,\,$ \\
$\,\,$\color{gr} \color{blue}  1/6\color{black} $\,\,$ & $\,\,$\color{gr} 0.2353\color{black} $\,\,$ & $\,\,$ 1 $\,\,$ & $\,\,$\color{gr} 0.9679\color{black}  $\,\,$ \\
$\,\,$0.1722$\,\,$ & $\,\,$0.2431$\,\,$ & $\,\,$\color{gr} 1.0331\color{black} $\,\,$ & $\,\,$ 1  $\,\,$ \\
\end{pmatrix},
\end{equation*}
\end{example}
\newpage
\begin{example}
\begin{equation*}
\mathbf{A} =
\begin{pmatrix}
$\,\,$ 1 $\,\,$ & $\,\,$2$\,\,$ & $\,\,$6$\,\,$ & $\,\,$4 $\,\,$ \\
$\,\,$ 1/2$\,\,$ & $\,\,$ 1 $\,\,$ & $\,\,$5$\,\,$ & $\,\,$8 $\,\,$ \\
$\,\,$ 1/6$\,\,$ & $\,\,$ 1/5$\,\,$ & $\,\,$ 1 $\,\,$ & $\,\,$1 $\,\,$ \\
$\,\,$ 1/4$\,\,$ & $\,\,$ 1/8$\,\,$ & $\,\,$ 1 $\,\,$ & $\,\,$ 1  $\,\,$ \\
\end{pmatrix},
\qquad
\lambda_{\max} =
4.1655,
\qquad
CR = 0.0624
\end{equation*}

\begin{equation*}
\mathbf{w}^{EM} =
\begin{pmatrix}
0.472563\\
0.379550\\
\color{red} 0.072634\color{black} \\
0.075254
\end{pmatrix}\end{equation*}
\begin{equation*}
\left[ \frac{{w}^{EM}_i}{{w}^{EM}_j} \right] =
\begin{pmatrix}
$\,\,$ 1 $\,\,$ & $\,\,$1.2451$\,\,$ & $\,\,$\color{red} 6.5061\color{black} $\,\,$ & $\,\,$6.2796$\,\,$ \\
$\,\,$0.8032$\,\,$ & $\,\,$ 1 $\,\,$ & $\,\,$\color{red} 5.2255\color{black} $\,\,$ & $\,\,$5.0436  $\,\,$ \\
$\,\,$\color{red} 0.1537\color{black} $\,\,$ & $\,\,$\color{red} 0.1914\color{black} $\,\,$ & $\,\,$ 1 $\,\,$ & $\,\,$\color{red} 0.9652\color{black}  $\,\,$ \\
$\,\,$0.1592$\,\,$ & $\,\,$0.1983$\,\,$ & $\,\,$\color{red} 1.0361\color{black} $\,\,$ & $\,\,$ 1  $\,\,$ \\
\end{pmatrix},
\end{equation*}

\begin{equation*}
\mathbf{w}^{\prime} =
\begin{pmatrix}
0.471328\\
0.378558\\
0.075057\\
0.075057
\end{pmatrix} =
0.997387\cdot
\begin{pmatrix}
0.472563\\
0.379550\\
\color{gr} 0.075254\color{black} \\
0.075254
\end{pmatrix},
\end{equation*}
\begin{equation*}
\left[ \frac{{w}^{\prime}_i}{{w}^{\prime}_j} \right] =
\begin{pmatrix}
$\,\,$ 1 $\,\,$ & $\,\,$1.2451$\,\,$ & $\,\,$\color{gr} 6.2796\color{black} $\,\,$ & $\,\,$6.2796$\,\,$ \\
$\,\,$0.8032$\,\,$ & $\,\,$ 1 $\,\,$ & $\,\,$\color{gr} 5.0436\color{black} $\,\,$ & $\,\,$5.0436  $\,\,$ \\
$\,\,$\color{gr} 0.1592\color{black} $\,\,$ & $\,\,$\color{gr} 0.1983\color{black} $\,\,$ & $\,\,$ 1 $\,\,$ & $\,\,$\color{gr} \color{blue} 1\color{black}  $\,\,$ \\
$\,\,$0.1592$\,\,$ & $\,\,$0.1983$\,\,$ & $\,\,$\color{gr} \color{blue} 1\color{black} $\,\,$ & $\,\,$ 1  $\,\,$ \\
\end{pmatrix},
\end{equation*}
\end{example}
\newpage
\begin{example}
\begin{equation*}
\mathbf{A} =
\begin{pmatrix}
$\,\,$ 1 $\,\,$ & $\,\,$2$\,\,$ & $\,\,$6$\,\,$ & $\,\,$4 $\,\,$ \\
$\,\,$ 1/2$\,\,$ & $\,\,$ 1 $\,\,$ & $\,\,$5$\,\,$ & $\,\,$9 $\,\,$ \\
$\,\,$ 1/6$\,\,$ & $\,\,$ 1/5$\,\,$ & $\,\,$ 1 $\,\,$ & $\,\,$1 $\,\,$ \\
$\,\,$ 1/4$\,\,$ & $\,\,$ 1/9$\,\,$ & $\,\,$ 1 $\,\,$ & $\,\,$ 1  $\,\,$ \\
\end{pmatrix},
\qquad
\lambda_{\max} =
4.1966,
\qquad
CR = 0.0741
\end{equation*}

\begin{equation*}
\mathbf{w}^{EM} =
\begin{pmatrix}
0.467653\\
0.388598\\
\color{red} 0.071347\color{black} \\
0.072402
\end{pmatrix}\end{equation*}
\begin{equation*}
\left[ \frac{{w}^{EM}_i}{{w}^{EM}_j} \right] =
\begin{pmatrix}
$\,\,$ 1 $\,\,$ & $\,\,$1.2034$\,\,$ & $\,\,$\color{red} 6.5547\color{black} $\,\,$ & $\,\,$6.4591$\,\,$ \\
$\,\,$0.8310$\,\,$ & $\,\,$ 1 $\,\,$ & $\,\,$\color{red} 5.4466\color{black} $\,\,$ & $\,\,$5.3672  $\,\,$ \\
$\,\,$\color{red} 0.1526\color{black} $\,\,$ & $\,\,$\color{red} 0.1836\color{black} $\,\,$ & $\,\,$ 1 $\,\,$ & $\,\,$\color{red} 0.9854\color{black}  $\,\,$ \\
$\,\,$0.1548$\,\,$ & $\,\,$0.1863$\,\,$ & $\,\,$\color{red} 1.0148\color{black} $\,\,$ & $\,\,$ 1  $\,\,$ \\
\end{pmatrix},
\end{equation*}

\begin{equation*}
\mathbf{w}^{\prime} =
\begin{pmatrix}
0.467160\\
0.388188\\
0.072326\\
0.072326
\end{pmatrix} =
0.998946\cdot
\begin{pmatrix}
0.467653\\
0.388598\\
\color{gr} 0.072402\color{black} \\
0.072402
\end{pmatrix},
\end{equation*}
\begin{equation*}
\left[ \frac{{w}^{\prime}_i}{{w}^{\prime}_j} \right] =
\begin{pmatrix}
$\,\,$ 1 $\,\,$ & $\,\,$1.2034$\,\,$ & $\,\,$\color{gr} 6.4591\color{black} $\,\,$ & $\,\,$6.4591$\,\,$ \\
$\,\,$0.8310$\,\,$ & $\,\,$ 1 $\,\,$ & $\,\,$\color{gr} 5.3672\color{black} $\,\,$ & $\,\,$5.3672  $\,\,$ \\
$\,\,$\color{gr} 0.1548\color{black} $\,\,$ & $\,\,$\color{gr} 0.1863\color{black} $\,\,$ & $\,\,$ 1 $\,\,$ & $\,\,$\color{gr} \color{blue} 1\color{black}  $\,\,$ \\
$\,\,$0.1548$\,\,$ & $\,\,$0.1863$\,\,$ & $\,\,$\color{gr} \color{blue} 1\color{black} $\,\,$ & $\,\,$ 1  $\,\,$ \\
\end{pmatrix},
\end{equation*}
\end{example}
\newpage
\begin{example}
\begin{equation*}
\mathbf{A} =
\begin{pmatrix}
$\,\,$ 1 $\,\,$ & $\,\,$2$\,\,$ & $\,\,$6$\,\,$ & $\,\,$5 $\,\,$ \\
$\,\,$ 1/2$\,\,$ & $\,\,$ 1 $\,\,$ & $\,\,$2$\,\,$ & $\,\,$4 $\,\,$ \\
$\,\,$ 1/6$\,\,$ & $\,\,$ 1/2$\,\,$ & $\,\,$ 1 $\,\,$ & $\,\,$3 $\,\,$ \\
$\,\,$ 1/5$\,\,$ & $\,\,$ 1/4$\,\,$ & $\,\,$ 1/3$\,\,$ & $\,\,$ 1  $\,\,$ \\
\end{pmatrix},
\qquad
\lambda_{\max} =
4.1406,
\qquad
CR = 0.0530
\end{equation*}

\begin{equation*}
\mathbf{w}^{EM} =
\begin{pmatrix}
0.535144\\
\color{red} 0.259825\color{black} \\
0.135850\\
0.069181
\end{pmatrix}\end{equation*}
\begin{equation*}
\left[ \frac{{w}^{EM}_i}{{w}^{EM}_j} \right] =
\begin{pmatrix}
$\,\,$ 1 $\,\,$ & $\,\,$\color{red} 2.0596\color{black} $\,\,$ & $\,\,$3.9392$\,\,$ & $\,\,$7.7354$\,\,$ \\
$\,\,$\color{red} 0.4855\color{black} $\,\,$ & $\,\,$ 1 $\,\,$ & $\,\,$\color{red} 1.9126\color{black} $\,\,$ & $\,\,$\color{red} 3.7557\color{black}   $\,\,$ \\
$\,\,$0.2539$\,\,$ & $\,\,$\color{red} 0.5229\color{black} $\,\,$ & $\,\,$ 1 $\,\,$ & $\,\,$1.9637 $\,\,$ \\
$\,\,$0.1293$\,\,$ & $\,\,$\color{red} 0.2663\color{black} $\,\,$ & $\,\,$0.5092$\,\,$ & $\,\,$ 1  $\,\,$ \\
\end{pmatrix},
\end{equation*}

\begin{equation*}
\mathbf{w}^{\prime} =
\begin{pmatrix}
0.531030\\
0.265515\\
0.134806\\
0.068649
\end{pmatrix} =
0.992312\cdot
\begin{pmatrix}
0.535144\\
\color{gr} 0.267572\color{black} \\
0.135850\\
0.069181
\end{pmatrix},
\end{equation*}
\begin{equation*}
\left[ \frac{{w}^{\prime}_i}{{w}^{\prime}_j} \right] =
\begin{pmatrix}
$\,\,$ 1 $\,\,$ & $\,\,$\color{gr} \color{blue} 2\color{black} $\,\,$ & $\,\,$3.9392$\,\,$ & $\,\,$7.7354$\,\,$ \\
$\,\,$\color{gr} \color{blue}  1/2\color{black} $\,\,$ & $\,\,$ 1 $\,\,$ & $\,\,$\color{gr} 1.9696\color{black} $\,\,$ & $\,\,$\color{gr} 3.8677\color{black}   $\,\,$ \\
$\,\,$0.2539$\,\,$ & $\,\,$\color{gr} 0.5077\color{black} $\,\,$ & $\,\,$ 1 $\,\,$ & $\,\,$1.9637 $\,\,$ \\
$\,\,$0.1293$\,\,$ & $\,\,$\color{gr} 0.2586\color{black} $\,\,$ & $\,\,$0.5092$\,\,$ & $\,\,$ 1  $\,\,$ \\
\end{pmatrix},
\end{equation*}
\end{example}
\newpage
\begin{example}
\begin{equation*}
\mathbf{A} =
\begin{pmatrix}
$\,\,$ 1 $\,\,$ & $\,\,$2$\,\,$ & $\,\,$6$\,\,$ & $\,\,$5 $\,\,$ \\
$\,\,$ 1/2$\,\,$ & $\,\,$ 1 $\,\,$ & $\,\,$2$\,\,$ & $\,\,$4 $\,\,$ \\
$\,\,$ 1/6$\,\,$ & $\,\,$ 1/2$\,\,$ & $\,\,$ 1 $\,\,$ & $\,\,$4 $\,\,$ \\
$\,\,$ 1/5$\,\,$ & $\,\,$ 1/4$\,\,$ & $\,\,$ 1/4$\,\,$ & $\,\,$ 1  $\,\,$ \\
\end{pmatrix},
\qquad
\lambda_{\max} =
4.2162,
\qquad
CR = 0.0815
\end{equation*}

\begin{equation*}
\mathbf{w}^{EM} =
\begin{pmatrix}
0.533486\\
\color{red} 0.254711\color{black} \\
0.147374\\
0.064429
\end{pmatrix}\end{equation*}
\begin{equation*}
\left[ \frac{{w}^{EM}_i}{{w}^{EM}_j} \right] =
\begin{pmatrix}
$\,\,$ 1 $\,\,$ & $\,\,$\color{red} 2.0945\color{black} $\,\,$ & $\,\,$3.6200$\,\,$ & $\,\,$8.2802$\,\,$ \\
$\,\,$\color{red} 0.4774\color{black} $\,\,$ & $\,\,$ 1 $\,\,$ & $\,\,$\color{red} 1.7283\color{black} $\,\,$ & $\,\,$\color{red} 3.9534\color{black}   $\,\,$ \\
$\,\,$0.2762$\,\,$ & $\,\,$\color{red} 0.5786\color{black} $\,\,$ & $\,\,$ 1 $\,\,$ & $\,\,$2.2874 $\,\,$ \\
$\,\,$0.1208$\,\,$ & $\,\,$\color{red} 0.2529\color{black} $\,\,$ & $\,\,$0.4372$\,\,$ & $\,\,$ 1  $\,\,$ \\
\end{pmatrix},
\end{equation*}

\begin{equation*}
\mathbf{w}^{\prime} =
\begin{pmatrix}
0.531888\\
0.256944\\
0.146932\\
0.064236
\end{pmatrix} =
0.997003\cdot
\begin{pmatrix}
0.533486\\
\color{gr} 0.257717\color{black} \\
0.147374\\
0.064429
\end{pmatrix},
\end{equation*}
\begin{equation*}
\left[ \frac{{w}^{\prime}_i}{{w}^{\prime}_j} \right] =
\begin{pmatrix}
$\,\,$ 1 $\,\,$ & $\,\,$\color{gr} 2.0700\color{black} $\,\,$ & $\,\,$3.6200$\,\,$ & $\,\,$8.2802$\,\,$ \\
$\,\,$\color{gr} 0.4831\color{black} $\,\,$ & $\,\,$ 1 $\,\,$ & $\,\,$\color{gr} 1.7487\color{black} $\,\,$ & $\,\,$\color{gr} \color{blue} 4\color{black}   $\,\,$ \\
$\,\,$0.2762$\,\,$ & $\,\,$\color{gr} 0.5718\color{black} $\,\,$ & $\,\,$ 1 $\,\,$ & $\,\,$2.2874 $\,\,$ \\
$\,\,$0.1208$\,\,$ & $\,\,$\color{gr} \color{blue}  1/4\color{black} $\,\,$ & $\,\,$0.4372$\,\,$ & $\,\,$ 1  $\,\,$ \\
\end{pmatrix},
\end{equation*}
\end{example}
\newpage
\begin{example}
\begin{equation*}
\mathbf{A} =
\begin{pmatrix}
$\,\,$ 1 $\,\,$ & $\,\,$2$\,\,$ & $\,\,$6$\,\,$ & $\,\,$6 $\,\,$ \\
$\,\,$ 1/2$\,\,$ & $\,\,$ 1 $\,\,$ & $\,\,$2$\,\,$ & $\,\,$4 $\,\,$ \\
$\,\,$ 1/6$\,\,$ & $\,\,$ 1/2$\,\,$ & $\,\,$ 1 $\,\,$ & $\,\,$3 $\,\,$ \\
$\,\,$ 1/6$\,\,$ & $\,\,$ 1/4$\,\,$ & $\,\,$ 1/3$\,\,$ & $\,\,$ 1  $\,\,$ \\
\end{pmatrix},
\qquad
\lambda_{\max} =
4.1031,
\qquad
CR = 0.0389
\end{equation*}

\begin{equation*}
\mathbf{w}^{EM} =
\begin{pmatrix}
0.546405\\
\color{red} 0.256499\color{black} \\
0.132818\\
0.064278
\end{pmatrix}\end{equation*}
\begin{equation*}
\left[ \frac{{w}^{EM}_i}{{w}^{EM}_j} \right] =
\begin{pmatrix}
$\,\,$ 1 $\,\,$ & $\,\,$\color{red} 2.1302\color{black} $\,\,$ & $\,\,$4.1140$\,\,$ & $\,\,$8.5006$\,\,$ \\
$\,\,$\color{red} 0.4694\color{black} $\,\,$ & $\,\,$ 1 $\,\,$ & $\,\,$\color{red} 1.9312\color{black} $\,\,$ & $\,\,$\color{red} 3.9904\color{black}   $\,\,$ \\
$\,\,$0.2431$\,\,$ & $\,\,$\color{red} 0.5178\color{black} $\,\,$ & $\,\,$ 1 $\,\,$ & $\,\,$2.0663 $\,\,$ \\
$\,\,$0.1176$\,\,$ & $\,\,$\color{red} 0.2506\color{black} $\,\,$ & $\,\,$0.4840$\,\,$ & $\,\,$ 1  $\,\,$ \\
\end{pmatrix},
\end{equation*}

\begin{equation*}
\mathbf{w}^{\prime} =
\begin{pmatrix}
0.546070\\
0.256955\\
0.132736\\
0.064239
\end{pmatrix} =
0.999386\cdot
\begin{pmatrix}
0.546405\\
\color{gr} 0.257113\color{black} \\
0.132818\\
0.064278
\end{pmatrix},
\end{equation*}
\begin{equation*}
\left[ \frac{{w}^{\prime}_i}{{w}^{\prime}_j} \right] =
\begin{pmatrix}
$\,\,$ 1 $\,\,$ & $\,\,$\color{gr} 2.1252\color{black} $\,\,$ & $\,\,$4.1140$\,\,$ & $\,\,$8.5006$\,\,$ \\
$\,\,$\color{gr} 0.4706\color{black} $\,\,$ & $\,\,$ 1 $\,\,$ & $\,\,$\color{gr} 1.9358\color{black} $\,\,$ & $\,\,$\color{gr} \color{blue} 4\color{black}   $\,\,$ \\
$\,\,$0.2431$\,\,$ & $\,\,$\color{gr} 0.5166\color{black} $\,\,$ & $\,\,$ 1 $\,\,$ & $\,\,$2.0663 $\,\,$ \\
$\,\,$0.1176$\,\,$ & $\,\,$\color{gr} \color{blue}  1/4\color{black} $\,\,$ & $\,\,$0.4840$\,\,$ & $\,\,$ 1  $\,\,$ \\
\end{pmatrix},
\end{equation*}
\end{example}
\newpage
\begin{example}
\begin{equation*}
\mathbf{A} =
\begin{pmatrix}
$\,\,$ 1 $\,\,$ & $\,\,$2$\,\,$ & $\,\,$6$\,\,$ & $\,\,$6 $\,\,$ \\
$\,\,$ 1/2$\,\,$ & $\,\,$ 1 $\,\,$ & $\,\,$2$\,\,$ & $\,\,$5 $\,\,$ \\
$\,\,$ 1/6$\,\,$ & $\,\,$ 1/2$\,\,$ & $\,\,$ 1 $\,\,$ & $\,\,$4 $\,\,$ \\
$\,\,$ 1/6$\,\,$ & $\,\,$ 1/5$\,\,$ & $\,\,$ 1/4$\,\,$ & $\,\,$ 1  $\,\,$ \\
\end{pmatrix},
\qquad
\lambda_{\max} =
4.1655,
\qquad
CR = 0.0624
\end{equation*}

\begin{equation*}
\mathbf{w}^{EM} =
\begin{pmatrix}
0.539828\\
\color{red} 0.263047\color{black} \\
0.140952\\
0.056173
\end{pmatrix}\end{equation*}
\begin{equation*}
\left[ \frac{{w}^{EM}_i}{{w}^{EM}_j} \right] =
\begin{pmatrix}
$\,\,$ 1 $\,\,$ & $\,\,$\color{red} 2.0522\color{black} $\,\,$ & $\,\,$3.8299$\,\,$ & $\,\,$9.6100$\,\,$ \\
$\,\,$\color{red} 0.4873\color{black} $\,\,$ & $\,\,$ 1 $\,\,$ & $\,\,$\color{red} 1.8662\color{black} $\,\,$ & $\,\,$\color{red} 4.6828\color{black}   $\,\,$ \\
$\,\,$0.2611$\,\,$ & $\,\,$\color{red} 0.5358\color{black} $\,\,$ & $\,\,$ 1 $\,\,$ & $\,\,$2.5092 $\,\,$ \\
$\,\,$0.1041$\,\,$ & $\,\,$\color{red} 0.2135\color{black} $\,\,$ & $\,\,$0.3985$\,\,$ & $\,\,$ 1  $\,\,$ \\
\end{pmatrix},
\end{equation*}

\begin{equation*}
\mathbf{w}^{\prime} =
\begin{pmatrix}
0.536146\\
0.268073\\
0.139991\\
0.055790
\end{pmatrix} =
0.993180\cdot
\begin{pmatrix}
0.539828\\
\color{gr} 0.269914\color{black} \\
0.140952\\
0.056173
\end{pmatrix},
\end{equation*}
\begin{equation*}
\left[ \frac{{w}^{\prime}_i}{{w}^{\prime}_j} \right] =
\begin{pmatrix}
$\,\,$ 1 $\,\,$ & $\,\,$\color{gr} \color{blue} 2\color{black} $\,\,$ & $\,\,$3.8299$\,\,$ & $\,\,$9.6100$\,\,$ \\
$\,\,$\color{gr} \color{blue}  1/2\color{black} $\,\,$ & $\,\,$ 1 $\,\,$ & $\,\,$\color{gr} 1.9149\color{black} $\,\,$ & $\,\,$\color{gr} 4.8050\color{black}   $\,\,$ \\
$\,\,$0.2611$\,\,$ & $\,\,$\color{gr} 0.5222\color{black} $\,\,$ & $\,\,$ 1 $\,\,$ & $\,\,$2.5092 $\,\,$ \\
$\,\,$0.1041$\,\,$ & $\,\,$\color{gr} 0.2081\color{black} $\,\,$ & $\,\,$0.3985$\,\,$ & $\,\,$ 1  $\,\,$ \\
\end{pmatrix},
\end{equation*}
\end{example}
\newpage
\begin{example}
\begin{equation*}
\mathbf{A} =
\begin{pmatrix}
$\,\,$ 1 $\,\,$ & $\,\,$2$\,\,$ & $\,\,$6$\,\,$ & $\,\,$6 $\,\,$ \\
$\,\,$ 1/2$\,\,$ & $\,\,$ 1 $\,\,$ & $\,\,$2$\,\,$ & $\,\,$5 $\,\,$ \\
$\,\,$ 1/6$\,\,$ & $\,\,$ 1/2$\,\,$ & $\,\,$ 1 $\,\,$ & $\,\,$5 $\,\,$ \\
$\,\,$ 1/6$\,\,$ & $\,\,$ 1/5$\,\,$ & $\,\,$ 1/5$\,\,$ & $\,\,$ 1  $\,\,$ \\
\end{pmatrix},
\qquad
\lambda_{\max} =
4.2277,
\qquad
CR = 0.0859
\end{equation*}

\begin{equation*}
\mathbf{w}^{EM} =
\begin{pmatrix}
0.538091\\
\color{red} 0.258661\color{black} \\
0.150133\\
0.053115
\end{pmatrix}\end{equation*}
\begin{equation*}
\left[ \frac{{w}^{EM}_i}{{w}^{EM}_j} \right] =
\begin{pmatrix}
$\,\,$ 1 $\,\,$ & $\,\,$\color{red} 2.0803\color{black} $\,\,$ & $\,\,$3.5841$\,\,$ & $\,\,$10.1307$\,\,$ \\
$\,\,$\color{red} 0.4807\color{black} $\,\,$ & $\,\,$ 1 $\,\,$ & $\,\,$\color{red} 1.7229\color{black} $\,\,$ & $\,\,$\color{red} 4.8698\color{black}   $\,\,$ \\
$\,\,$0.2790$\,\,$ & $\,\,$\color{red} 0.5804\color{black} $\,\,$ & $\,\,$ 1 $\,\,$ & $\,\,$2.8266 $\,\,$ \\
$\,\,$0.0987$\,\,$ & $\,\,$\color{red} 0.2053\color{black} $\,\,$ & $\,\,$0.3538$\,\,$ & $\,\,$ 1  $\,\,$ \\
\end{pmatrix},
\end{equation*}

\begin{equation*}
\mathbf{w}^{\prime} =
\begin{pmatrix}
0.534396\\
0.263752\\
0.149102\\
0.052750
\end{pmatrix} =
0.993133\cdot
\begin{pmatrix}
0.538091\\
\color{gr} 0.265575\color{black} \\
0.150133\\
0.053115
\end{pmatrix},
\end{equation*}
\begin{equation*}
\left[ \frac{{w}^{\prime}_i}{{w}^{\prime}_j} \right] =
\begin{pmatrix}
$\,\,$ 1 $\,\,$ & $\,\,$\color{gr} 2.0261\color{black} $\,\,$ & $\,\,$3.5841$\,\,$ & $\,\,$10.1307$\,\,$ \\
$\,\,$\color{gr} 0.4936\color{black} $\,\,$ & $\,\,$ 1 $\,\,$ & $\,\,$\color{gr} 1.7689\color{black} $\,\,$ & $\,\,$\color{gr} \color{blue} 5\color{black}   $\,\,$ \\
$\,\,$0.2790$\,\,$ & $\,\,$\color{gr} 0.5653\color{black} $\,\,$ & $\,\,$ 1 $\,\,$ & $\,\,$2.8266 $\,\,$ \\
$\,\,$0.0987$\,\,$ & $\,\,$\color{gr} \color{blue}  1/5\color{black} $\,\,$ & $\,\,$0.3538$\,\,$ & $\,\,$ 1  $\,\,$ \\
\end{pmatrix},
\end{equation*}
\end{example}
\newpage
\begin{example}
\begin{equation*}
\mathbf{A} =
\begin{pmatrix}
$\,\,$ 1 $\,\,$ & $\,\,$2$\,\,$ & $\,\,$6$\,\,$ & $\,\,$7 $\,\,$ \\
$\,\,$ 1/2$\,\,$ & $\,\,$ 1 $\,\,$ & $\,\,$2$\,\,$ & $\,\,$5 $\,\,$ \\
$\,\,$ 1/6$\,\,$ & $\,\,$ 1/2$\,\,$ & $\,\,$ 1 $\,\,$ & $\,\,$4 $\,\,$ \\
$\,\,$ 1/7$\,\,$ & $\,\,$ 1/5$\,\,$ & $\,\,$ 1/4$\,\,$ & $\,\,$ 1  $\,\,$ \\
\end{pmatrix},
\qquad
\lambda_{\max} =
4.1301,
\qquad
CR = 0.0490
\end{equation*}

\begin{equation*}
\mathbf{w}^{EM} =
\begin{pmatrix}
0.548958\\
\color{red} 0.260173\color{black} \\
0.138155\\
0.052713
\end{pmatrix}\end{equation*}
\begin{equation*}
\left[ \frac{{w}^{EM}_i}{{w}^{EM}_j} \right] =
\begin{pmatrix}
$\,\,$ 1 $\,\,$ & $\,\,$\color{red} 2.1100\color{black} $\,\,$ & $\,\,$3.9735$\,\,$ & $\,\,$10.4140$\,\,$ \\
$\,\,$\color{red} 0.4739\color{black} $\,\,$ & $\,\,$ 1 $\,\,$ & $\,\,$\color{red} 1.8832\color{black} $\,\,$ & $\,\,$\color{red} 4.9356\color{black}   $\,\,$ \\
$\,\,$0.2517$\,\,$ & $\,\,$\color{red} 0.5310\color{black} $\,\,$ & $\,\,$ 1 $\,\,$ & $\,\,$2.6209 $\,\,$ \\
$\,\,$0.0960$\,\,$ & $\,\,$\color{red} 0.2026\color{black} $\,\,$ & $\,\,$0.3816$\,\,$ & $\,\,$ 1  $\,\,$ \\
\end{pmatrix},
\end{equation*}

\begin{equation*}
\mathbf{w}^{\prime} =
\begin{pmatrix}
0.547101\\
0.262676\\
0.137688\\
0.052535
\end{pmatrix} =
0.996617\cdot
\begin{pmatrix}
0.548958\\
\color{gr} 0.263567\color{black} \\
0.138155\\
0.052713
\end{pmatrix},
\end{equation*}
\begin{equation*}
\left[ \frac{{w}^{\prime}_i}{{w}^{\prime}_j} \right] =
\begin{pmatrix}
$\,\,$ 1 $\,\,$ & $\,\,$\color{gr} 2.0828\color{black} $\,\,$ & $\,\,$3.9735$\,\,$ & $\,\,$10.4140$\,\,$ \\
$\,\,$\color{gr} 0.4801\color{black} $\,\,$ & $\,\,$ 1 $\,\,$ & $\,\,$\color{gr} 1.9078\color{black} $\,\,$ & $\,\,$\color{gr} \color{blue} 5\color{black}   $\,\,$ \\
$\,\,$0.2517$\,\,$ & $\,\,$\color{gr} 0.5242\color{black} $\,\,$ & $\,\,$ 1 $\,\,$ & $\,\,$2.6209 $\,\,$ \\
$\,\,$0.0960$\,\,$ & $\,\,$\color{gr} \color{blue}  1/5\color{black} $\,\,$ & $\,\,$0.3816$\,\,$ & $\,\,$ 1  $\,\,$ \\
\end{pmatrix},
\end{equation*}
\end{example}
\newpage
\begin{example}
\begin{equation*}
\mathbf{A} =
\begin{pmatrix}
$\,\,$ 1 $\,\,$ & $\,\,$2$\,\,$ & $\,\,$6$\,\,$ & $\,\,$7 $\,\,$ \\
$\,\,$ 1/2$\,\,$ & $\,\,$ 1 $\,\,$ & $\,\,$5$\,\,$ & $\,\,$2 $\,\,$ \\
$\,\,$ 1/6$\,\,$ & $\,\,$ 1/5$\,\,$ & $\,\,$ 1 $\,\,$ & $\,\,$2 $\,\,$ \\
$\,\,$ 1/7$\,\,$ & $\,\,$ 1/2$\,\,$ & $\,\,$ 1/2$\,\,$ & $\,\,$ 1  $\,\,$ \\
\end{pmatrix},
\qquad
\lambda_{\max} =
4.2251,
\qquad
CR = 0.0849
\end{equation*}

\begin{equation*}
\mathbf{w}^{EM} =
\begin{pmatrix}
\color{red} 0.536063\color{black} \\
0.284402\\
0.096706\\
0.082830
\end{pmatrix}\end{equation*}
\begin{equation*}
\left[ \frac{{w}^{EM}_i}{{w}^{EM}_j} \right] =
\begin{pmatrix}
$\,\,$ 1 $\,\,$ & $\,\,$\color{red} 1.8849\color{black} $\,\,$ & $\,\,$\color{red} 5.5432\color{black} $\,\,$ & $\,\,$\color{red} 6.4718\color{black} $\,\,$ \\
$\,\,$\color{red} 0.5305\color{black} $\,\,$ & $\,\,$ 1 $\,\,$ & $\,\,$2.9409$\,\,$ & $\,\,$3.4336  $\,\,$ \\
$\,\,$\color{red} 0.1804\color{black} $\,\,$ & $\,\,$0.3400$\,\,$ & $\,\,$ 1 $\,\,$ & $\,\,$1.1675 $\,\,$ \\
$\,\,$\color{red} 0.1545\color{black} $\,\,$ & $\,\,$0.2912$\,\,$ & $\,\,$0.8565$\,\,$ & $\,\,$ 1  $\,\,$ \\
\end{pmatrix},
\end{equation*}

\begin{equation*}
\mathbf{w}^{\prime} =
\begin{pmatrix}
0.550771\\
0.275385\\
0.093640\\
0.080204
\end{pmatrix} =
0.968297\cdot
\begin{pmatrix}
\color{gr} 0.568803\color{black} \\
0.284402\\
0.096706\\
0.082830
\end{pmatrix},
\end{equation*}
\begin{equation*}
\left[ \frac{{w}^{\prime}_i}{{w}^{\prime}_j} \right] =
\begin{pmatrix}
$\,\,$ 1 $\,\,$ & $\,\,$\color{gr} \color{blue} 2\color{black} $\,\,$ & $\,\,$\color{gr} 5.8818\color{black} $\,\,$ & $\,\,$\color{gr} 6.8671\color{black} $\,\,$ \\
$\,\,$\color{gr} \color{blue}  1/2\color{black} $\,\,$ & $\,\,$ 1 $\,\,$ & $\,\,$2.9409$\,\,$ & $\,\,$3.4336  $\,\,$ \\
$\,\,$\color{gr} 0.1700\color{black} $\,\,$ & $\,\,$0.3400$\,\,$ & $\,\,$ 1 $\,\,$ & $\,\,$1.1675 $\,\,$ \\
$\,\,$\color{gr} 0.1456\color{black} $\,\,$ & $\,\,$0.2912$\,\,$ & $\,\,$0.8565$\,\,$ & $\,\,$ 1  $\,\,$ \\
\end{pmatrix},
\end{equation*}
\end{example}
\newpage
\begin{example}
\begin{equation*}
\mathbf{A} =
\begin{pmatrix}
$\,\,$ 1 $\,\,$ & $\,\,$2$\,\,$ & $\,\,$6$\,\,$ & $\,\,$8 $\,\,$ \\
$\,\,$ 1/2$\,\,$ & $\,\,$ 1 $\,\,$ & $\,\,$1$\,\,$ & $\,\,$3 $\,\,$ \\
$\,\,$ 1/6$\,\,$ & $\,\,$ 1 $\,\,$ & $\,\,$ 1 $\,\,$ & $\,\,$2 $\,\,$ \\
$\,\,$ 1/8$\,\,$ & $\,\,$ 1/3$\,\,$ & $\,\,$ 1/2$\,\,$ & $\,\,$ 1  $\,\,$ \\
\end{pmatrix},
\qquad
\lambda_{\max} =
4.1031,
\qquad
CR = 0.0389
\end{equation*}

\begin{equation*}
\mathbf{w}^{EM} =
\begin{pmatrix}
0.583653\\
0.205980\\
0.141871\\
\color{red} 0.068496\color{black}
\end{pmatrix}\end{equation*}
\begin{equation*}
\left[ \frac{{w}^{EM}_i}{{w}^{EM}_j} \right] =
\begin{pmatrix}
$\,\,$ 1 $\,\,$ & $\,\,$2.8335$\,\,$ & $\,\,$4.1140$\,\,$ & $\,\,$\color{red} 8.5210\color{black} $\,\,$ \\
$\,\,$0.3529$\,\,$ & $\,\,$ 1 $\,\,$ & $\,\,$1.4519$\,\,$ & $\,\,$\color{red} 3.0072\color{black}   $\,\,$ \\
$\,\,$0.2431$\,\,$ & $\,\,$0.6888$\,\,$ & $\,\,$ 1 $\,\,$ & $\,\,$\color{red} 2.0712\color{black}  $\,\,$ \\
$\,\,$\color{red} 0.1174\color{black} $\,\,$ & $\,\,$\color{red} 0.3325\color{black} $\,\,$ & $\,\,$\color{red} 0.4828\color{black} $\,\,$ & $\,\,$ 1  $\,\,$ \\
\end{pmatrix},
\end{equation*}

\begin{equation*}
\mathbf{w}^{\prime} =
\begin{pmatrix}
0.583557\\
0.205946\\
0.141848\\
0.068649
\end{pmatrix} =
0.999836\cdot
\begin{pmatrix}
0.583653\\
0.205980\\
0.141871\\
\color{gr} 0.068660\color{black}
\end{pmatrix},
\end{equation*}
\begin{equation*}
\left[ \frac{{w}^{\prime}_i}{{w}^{\prime}_j} \right] =
\begin{pmatrix}
$\,\,$ 1 $\,\,$ & $\,\,$2.8335$\,\,$ & $\,\,$4.1140$\,\,$ & $\,\,$\color{gr} 8.5006\color{black} $\,\,$ \\
$\,\,$0.3529$\,\,$ & $\,\,$ 1 $\,\,$ & $\,\,$1.4519$\,\,$ & $\,\,$\color{gr} \color{blue} 3\color{black}   $\,\,$ \\
$\,\,$0.2431$\,\,$ & $\,\,$0.6888$\,\,$ & $\,\,$ 1 $\,\,$ & $\,\,$\color{gr} 2.0663\color{black}  $\,\,$ \\
$\,\,$\color{gr} 0.1176\color{black} $\,\,$ & $\,\,$\color{gr} \color{blue}  1/3\color{black} $\,\,$ & $\,\,$\color{gr} 0.4840\color{black} $\,\,$ & $\,\,$ 1  $\,\,$ \\
\end{pmatrix},
\end{equation*}
\end{example}
\newpage
\begin{example}
\begin{equation*}
\mathbf{A} =
\begin{pmatrix}
$\,\,$ 1 $\,\,$ & $\,\,$2$\,\,$ & $\,\,$6$\,\,$ & $\,\,$8 $\,\,$ \\
$\,\,$ 1/2$\,\,$ & $\,\,$ 1 $\,\,$ & $\,\,$5$\,\,$ & $\,\,$2 $\,\,$ \\
$\,\,$ 1/6$\,\,$ & $\,\,$ 1/5$\,\,$ & $\,\,$ 1 $\,\,$ & $\,\,$2 $\,\,$ \\
$\,\,$ 1/8$\,\,$ & $\,\,$ 1/2$\,\,$ & $\,\,$ 1/2$\,\,$ & $\,\,$ 1  $\,\,$ \\
\end{pmatrix},
\qquad
\lambda_{\max} =
4.2277,
\qquad
CR = 0.0859
\end{equation*}

\begin{equation*}
\mathbf{w}^{EM} =
\begin{pmatrix}
\color{red} 0.545903\color{black} \\
0.280247\\
0.094636\\
0.079214
\end{pmatrix}\end{equation*}
\begin{equation*}
\left[ \frac{{w}^{EM}_i}{{w}^{EM}_j} \right] =
\begin{pmatrix}
$\,\,$ 1 $\,\,$ & $\,\,$\color{red} 1.9479\color{black} $\,\,$ & $\,\,$\color{red} 5.7684\color{black} $\,\,$ & $\,\,$\color{red} 6.8915\color{black} $\,\,$ \\
$\,\,$\color{red} 0.5134\color{black} $\,\,$ & $\,\,$ 1 $\,\,$ & $\,\,$2.9613$\,\,$ & $\,\,$3.5379  $\,\,$ \\
$\,\,$\color{red} 0.1734\color{black} $\,\,$ & $\,\,$0.3377$\,\,$ & $\,\,$ 1 $\,\,$ & $\,\,$1.1947 $\,\,$ \\
$\,\,$\color{red} 0.1451\color{black} $\,\,$ & $\,\,$0.2827$\,\,$ & $\,\,$0.8370$\,\,$ & $\,\,$ 1  $\,\,$ \\
\end{pmatrix},
\end{equation*}

\begin{equation*}
\mathbf{w}^{\prime} =
\begin{pmatrix}
0.552433\\
0.276217\\
0.093275\\
0.078074
\end{pmatrix} =
0.985618\cdot
\begin{pmatrix}
\color{gr} 0.560495\color{black} \\
0.280247\\
0.094636\\
0.079214
\end{pmatrix},
\end{equation*}
\begin{equation*}
\left[ \frac{{w}^{\prime}_i}{{w}^{\prime}_j} \right] =
\begin{pmatrix}
$\,\,$ 1 $\,\,$ & $\,\,$\color{gr} \color{blue} 2\color{black} $\,\,$ & $\,\,$\color{gr} 5.9226\color{black} $\,\,$ & $\,\,$\color{gr} 7.0757\color{black} $\,\,$ \\
$\,\,$\color{gr} \color{blue}  1/2\color{black} $\,\,$ & $\,\,$ 1 $\,\,$ & $\,\,$2.9613$\,\,$ & $\,\,$3.5379  $\,\,$ \\
$\,\,$\color{gr} 0.1688\color{black} $\,\,$ & $\,\,$0.3377$\,\,$ & $\,\,$ 1 $\,\,$ & $\,\,$1.1947 $\,\,$ \\
$\,\,$\color{gr} 0.1413\color{black} $\,\,$ & $\,\,$0.2827$\,\,$ & $\,\,$0.8370$\,\,$ & $\,\,$ 1  $\,\,$ \\
\end{pmatrix},
\end{equation*}
\end{example}
\newpage
\begin{example}
\begin{equation*}
\mathbf{A} =
\begin{pmatrix}
$\,\,$ 1 $\,\,$ & $\,\,$2$\,\,$ & $\,\,$6$\,\,$ & $\,\,$8 $\,\,$ \\
$\,\,$ 1/2$\,\,$ & $\,\,$ 1 $\,\,$ & $\,\,$7$\,\,$ & $\,\,$5 $\,\,$ \\
$\,\,$ 1/6$\,\,$ & $\,\,$ 1/7$\,\,$ & $\,\,$ 1 $\,\,$ & $\,\,$1 $\,\,$ \\
$\,\,$ 1/8$\,\,$ & $\,\,$ 1/5$\,\,$ & $\,\,$ 1 $\,\,$ & $\,\,$ 1  $\,\,$ \\
\end{pmatrix},
\qquad
\lambda_{\max} =
4.0609,
\qquad
CR = 0.0230
\end{equation*}

\begin{equation*}
\mathbf{w}^{EM} =
\begin{pmatrix}
0.524945\\
0.343602\\
0.066002\\
\color{red} 0.065451\color{black}
\end{pmatrix}\end{equation*}
\begin{equation*}
\left[ \frac{{w}^{EM}_i}{{w}^{EM}_j} \right] =
\begin{pmatrix}
$\,\,$ 1 $\,\,$ & $\,\,$1.5278$\,\,$ & $\,\,$7.9535$\,\,$ & $\,\,$\color{red} 8.0205\color{black} $\,\,$ \\
$\,\,$0.6545$\,\,$ & $\,\,$ 1 $\,\,$ & $\,\,$5.2059$\,\,$ & $\,\,$\color{red} 5.2498\color{black}   $\,\,$ \\
$\,\,$0.1257$\,\,$ & $\,\,$0.1921$\,\,$ & $\,\,$ 1 $\,\,$ & $\,\,$\color{red} 1.0084\color{black}  $\,\,$ \\
$\,\,$\color{red} 0.1247\color{black} $\,\,$ & $\,\,$\color{red} 0.1905\color{black} $\,\,$ & $\,\,$\color{red} 0.9916\color{black} $\,\,$ & $\,\,$ 1  $\,\,$ \\
\end{pmatrix},
\end{equation*}

\begin{equation*}
\mathbf{w}^{\prime} =
\begin{pmatrix}
0.524858\\
0.343544\\
0.065991\\
0.065607
\end{pmatrix} =
0.999833\cdot
\begin{pmatrix}
0.524945\\
0.343602\\
0.066002\\
\color{gr} 0.065618\color{black}
\end{pmatrix},
\end{equation*}
\begin{equation*}
\left[ \frac{{w}^{\prime}_i}{{w}^{\prime}_j} \right] =
\begin{pmatrix}
$\,\,$ 1 $\,\,$ & $\,\,$1.5278$\,\,$ & $\,\,$7.9535$\,\,$ & $\,\,$\color{gr} \color{blue} 8\color{black} $\,\,$ \\
$\,\,$0.6545$\,\,$ & $\,\,$ 1 $\,\,$ & $\,\,$5.2059$\,\,$ & $\,\,$\color{gr} 5.2364\color{black}   $\,\,$ \\
$\,\,$0.1257$\,\,$ & $\,\,$0.1921$\,\,$ & $\,\,$ 1 $\,\,$ & $\,\,$\color{gr} 1.0058\color{black}  $\,\,$ \\
$\,\,$\color{gr} \color{blue}  1/8\color{black} $\,\,$ & $\,\,$\color{gr} 0.1910\color{black} $\,\,$ & $\,\,$\color{gr} 0.9942\color{black} $\,\,$ & $\,\,$ 1  $\,\,$ \\
\end{pmatrix},
\end{equation*}
\end{example}
\newpage
\begin{example}
\begin{equation*}
\mathbf{A} =
\begin{pmatrix}
$\,\,$ 1 $\,\,$ & $\,\,$2$\,\,$ & $\,\,$6$\,\,$ & $\,\,$8 $\,\,$ \\
$\,\,$ 1/2$\,\,$ & $\,\,$ 1 $\,\,$ & $\,\,$9$\,\,$ & $\,\,$6 $\,\,$ \\
$\,\,$ 1/6$\,\,$ & $\,\,$ 1/9$\,\,$ & $\,\,$ 1 $\,\,$ & $\,\,$1 $\,\,$ \\
$\,\,$ 1/8$\,\,$ & $\,\,$ 1/6$\,\,$ & $\,\,$ 1 $\,\,$ & $\,\,$ 1  $\,\,$ \\
\end{pmatrix},
\qquad
\lambda_{\max} =
4.1031,
\qquad
CR = 0.0389
\end{equation*}

\begin{equation*}
\mathbf{w}^{EM} =
\begin{pmatrix}
0.509108\\
0.371254\\
0.059891\\
\color{red} 0.059747\color{black}
\end{pmatrix}\end{equation*}
\begin{equation*}
\left[ \frac{{w}^{EM}_i}{{w}^{EM}_j} \right] =
\begin{pmatrix}
$\,\,$ 1 $\,\,$ & $\,\,$1.3713$\,\,$ & $\,\,$8.5006$\,\,$ & $\,\,$\color{red} 8.5210\color{black} $\,\,$ \\
$\,\,$0.7292$\,\,$ & $\,\,$ 1 $\,\,$ & $\,\,$6.1989$\,\,$ & $\,\,$\color{red} 6.2137\color{black}   $\,\,$ \\
$\,\,$0.1176$\,\,$ & $\,\,$0.1613$\,\,$ & $\,\,$ 1 $\,\,$ & $\,\,$\color{red} 1.0024\color{black}  $\,\,$ \\
$\,\,$\color{red} 0.1174\color{black} $\,\,$ & $\,\,$\color{red} 0.1609\color{black} $\,\,$ & $\,\,$\color{red} 0.9976\color{black} $\,\,$ & $\,\,$ 1  $\,\,$ \\
\end{pmatrix},
\end{equation*}

\begin{equation*}
\mathbf{w}^{\prime} =
\begin{pmatrix}
0.509035\\
0.371201\\
0.059882\\
0.059882
\end{pmatrix} =
0.999857\cdot
\begin{pmatrix}
0.509108\\
0.371254\\
0.059891\\
\color{gr} 0.059891\color{black}
\end{pmatrix},
\end{equation*}
\begin{equation*}
\left[ \frac{{w}^{\prime}_i}{{w}^{\prime}_j} \right] =
\begin{pmatrix}
$\,\,$ 1 $\,\,$ & $\,\,$1.3713$\,\,$ & $\,\,$8.5006$\,\,$ & $\,\,$\color{gr} 8.5006\color{black} $\,\,$ \\
$\,\,$0.7292$\,\,$ & $\,\,$ 1 $\,\,$ & $\,\,$6.1989$\,\,$ & $\,\,$\color{gr} 6.1989\color{black}   $\,\,$ \\
$\,\,$0.1176$\,\,$ & $\,\,$0.1613$\,\,$ & $\,\,$ 1 $\,\,$ & $\,\,$\color{gr} \color{blue} 1\color{black}  $\,\,$ \\
$\,\,$\color{gr} 0.1176\color{black} $\,\,$ & $\,\,$\color{gr} 0.1613\color{black} $\,\,$ & $\,\,$\color{gr} \color{blue} 1\color{black} $\,\,$ & $\,\,$ 1  $\,\,$ \\
\end{pmatrix},
\end{equation*}
\end{example}
\newpage
\begin{example}
\begin{equation*}
\mathbf{A} =
\begin{pmatrix}
$\,\,$ 1 $\,\,$ & $\,\,$2$\,\,$ & $\,\,$6$\,\,$ & $\,\,$9 $\,\,$ \\
$\,\,$ 1/2$\,\,$ & $\,\,$ 1 $\,\,$ & $\,\,$1$\,\,$ & $\,\,$3 $\,\,$ \\
$\,\,$ 1/6$\,\,$ & $\,\,$ 1 $\,\,$ & $\,\,$ 1 $\,\,$ & $\,\,$2 $\,\,$ \\
$\,\,$ 1/9$\,\,$ & $\,\,$ 1/3$\,\,$ & $\,\,$ 1/2$\,\,$ & $\,\,$ 1  $\,\,$ \\
\end{pmatrix},
\qquad
\lambda_{\max} =
4.1031,
\qquad
CR = 0.0389
\end{equation*}

\begin{equation*}
\mathbf{w}^{EM} =
\begin{pmatrix}
0.591242\\
0.203613\\
0.139609\\
\color{red} 0.065536\color{black}
\end{pmatrix}\end{equation*}
\begin{equation*}
\left[ \frac{{w}^{EM}_i}{{w}^{EM}_j} \right] =
\begin{pmatrix}
$\,\,$ 1 $\,\,$ & $\,\,$2.9038$\,\,$ & $\,\,$4.2350$\,\,$ & $\,\,$\color{red} 9.0216\color{black} $\,\,$ \\
$\,\,$0.3444$\,\,$ & $\,\,$ 1 $\,\,$ & $\,\,$1.4585$\,\,$ & $\,\,$\color{red} 3.1069\color{black}   $\,\,$ \\
$\,\,$0.2361$\,\,$ & $\,\,$0.6857$\,\,$ & $\,\,$ 1 $\,\,$ & $\,\,$\color{red} 2.1302\color{black}  $\,\,$ \\
$\,\,$\color{red} 0.1108\color{black} $\,\,$ & $\,\,$\color{red} 0.3219\color{black} $\,\,$ & $\,\,$\color{red} 0.4694\color{black} $\,\,$ & $\,\,$ 1  $\,\,$ \\
\end{pmatrix},
\end{equation*}

\begin{equation*}
\mathbf{w}^{\prime} =
\begin{pmatrix}
0.591149\\
0.203581\\
0.139587\\
0.065683
\end{pmatrix} =
0.999843\cdot
\begin{pmatrix}
0.591242\\
0.203613\\
0.139609\\
\color{gr} 0.065694\color{black}
\end{pmatrix},
\end{equation*}
\begin{equation*}
\left[ \frac{{w}^{\prime}_i}{{w}^{\prime}_j} \right] =
\begin{pmatrix}
$\,\,$ 1 $\,\,$ & $\,\,$2.9038$\,\,$ & $\,\,$4.2350$\,\,$ & $\,\,$\color{gr} \color{blue} 9\color{black} $\,\,$ \\
$\,\,$0.3444$\,\,$ & $\,\,$ 1 $\,\,$ & $\,\,$1.4585$\,\,$ & $\,\,$\color{gr} 3.0994\color{black}   $\,\,$ \\
$\,\,$0.2361$\,\,$ & $\,\,$0.6857$\,\,$ & $\,\,$ 1 $\,\,$ & $\,\,$\color{gr} 2.1252\color{black}  $\,\,$ \\
$\,\,$\color{gr} \color{blue}  1/9\color{black} $\,\,$ & $\,\,$\color{gr} 0.3226\color{black} $\,\,$ & $\,\,$\color{gr} 0.4706\color{black} $\,\,$ & $\,\,$ 1  $\,\,$ \\
\end{pmatrix},
\end{equation*}
\end{example}
\newpage
\begin{example}
\begin{equation*}
\mathbf{A} =
\begin{pmatrix}
$\,\,$ 1 $\,\,$ & $\,\,$2$\,\,$ & $\,\,$7$\,\,$ & $\,\,$4 $\,\,$ \\
$\,\,$ 1/2$\,\,$ & $\,\,$ 1 $\,\,$ & $\,\,$2$\,\,$ & $\,\,$4 $\,\,$ \\
$\,\,$ 1/7$\,\,$ & $\,\,$ 1/2$\,\,$ & $\,\,$ 1 $\,\,$ & $\,\,$3 $\,\,$ \\
$\,\,$ 1/4$\,\,$ & $\,\,$ 1/4$\,\,$ & $\,\,$ 1/3$\,\,$ & $\,\,$ 1  $\,\,$ \\
\end{pmatrix},
\qquad
\lambda_{\max} =
4.2421,
\qquad
CR = 0.0913
\end{equation*}

\begin{equation*}
\mathbf{w}^{EM} =
\begin{pmatrix}
0.536298\\
\color{red} 0.256585\color{black} \\
0.132368\\
0.074749
\end{pmatrix}\end{equation*}
\begin{equation*}
\left[ \frac{{w}^{EM}_i}{{w}^{EM}_j} \right] =
\begin{pmatrix}
$\,\,$ 1 $\,\,$ & $\,\,$\color{red} 2.0901\color{black} $\,\,$ & $\,\,$4.0516$\,\,$ & $\,\,$7.1747$\,\,$ \\
$\,\,$\color{red} 0.4784\color{black} $\,\,$ & $\,\,$ 1 $\,\,$ & $\,\,$\color{red} 1.9384\color{black} $\,\,$ & $\,\,$\color{red} 3.4326\color{black}   $\,\,$ \\
$\,\,$0.2468$\,\,$ & $\,\,$\color{red} 0.5159\color{black} $\,\,$ & $\,\,$ 1 $\,\,$ & $\,\,$1.7708 $\,\,$ \\
$\,\,$0.1394$\,\,$ & $\,\,$\color{red} 0.2913\color{black} $\,\,$ & $\,\,$0.5647$\,\,$ & $\,\,$ 1  $\,\,$ \\
\end{pmatrix},
\end{equation*}

\begin{equation*}
\mathbf{w}^{\prime} =
\begin{pmatrix}
0.531962\\
0.262596\\
0.131298\\
0.074144
\end{pmatrix} =
0.991915\cdot
\begin{pmatrix}
0.536298\\
\color{gr} 0.264736\color{black} \\
0.132368\\
0.074749
\end{pmatrix},
\end{equation*}
\begin{equation*}
\left[ \frac{{w}^{\prime}_i}{{w}^{\prime}_j} \right] =
\begin{pmatrix}
$\,\,$ 1 $\,\,$ & $\,\,$\color{gr} 2.0258\color{black} $\,\,$ & $\,\,$4.0516$\,\,$ & $\,\,$7.1747$\,\,$ \\
$\,\,$\color{gr} 0.4936\color{black} $\,\,$ & $\,\,$ 1 $\,\,$ & $\,\,$\color{gr} \color{blue} 2\color{black} $\,\,$ & $\,\,$\color{gr} 3.5417\color{black}   $\,\,$ \\
$\,\,$0.2468$\,\,$ & $\,\,$\color{gr} \color{blue}  1/2\color{black} $\,\,$ & $\,\,$ 1 $\,\,$ & $\,\,$1.7708 $\,\,$ \\
$\,\,$0.1394$\,\,$ & $\,\,$\color{gr} 0.2824\color{black} $\,\,$ & $\,\,$0.5647$\,\,$ & $\,\,$ 1  $\,\,$ \\
\end{pmatrix},
\end{equation*}
\end{example}
\newpage
\begin{example}
\begin{equation*}
\mathbf{A} =
\begin{pmatrix}
$\,\,$ 1 $\,\,$ & $\,\,$2$\,\,$ & $\,\,$7$\,\,$ & $\,\,$4 $\,\,$ \\
$\,\,$ 1/2$\,\,$ & $\,\,$ 1 $\,\,$ & $\,\,$5$\,\,$ & $\,\,$8 $\,\,$ \\
$\,\,$ 1/7$\,\,$ & $\,\,$ 1/5$\,\,$ & $\,\,$ 1 $\,\,$ & $\,\,$1 $\,\,$ \\
$\,\,$ 1/4$\,\,$ & $\,\,$ 1/8$\,\,$ & $\,\,$ 1 $\,\,$ & $\,\,$ 1  $\,\,$ \\
\end{pmatrix},
\qquad
\lambda_{\max} =
4.1665,
\qquad
CR = 0.0628
\end{equation*}

\begin{equation*}
\mathbf{w}^{EM} =
\begin{pmatrix}
0.482733\\
0.373675\\
\color{red} 0.068952\color{black} \\
0.074639
\end{pmatrix}\end{equation*}
\begin{equation*}
\left[ \frac{{w}^{EM}_i}{{w}^{EM}_j} \right] =
\begin{pmatrix}
$\,\,$ 1 $\,\,$ & $\,\,$1.2919$\,\,$ & $\,\,$\color{red} 7.0010\color{black} $\,\,$ & $\,\,$6.4676$\,\,$ \\
$\,\,$0.7741$\,\,$ & $\,\,$ 1 $\,\,$ & $\,\,$\color{red} 5.4194\color{black} $\,\,$ & $\,\,$5.0064  $\,\,$ \\
$\,\,$\color{red} 0.1428\color{black} $\,\,$ & $\,\,$\color{red} 0.1845\color{black} $\,\,$ & $\,\,$ 1 $\,\,$ & $\,\,$\color{red} 0.9238\color{black}  $\,\,$ \\
$\,\,$0.1546$\,\,$ & $\,\,$0.1997$\,\,$ & $\,\,$\color{red} 1.0825\color{black} $\,\,$ & $\,\,$ 1  $\,\,$ \\
\end{pmatrix},
\end{equation*}

\begin{equation*}
\mathbf{w}^{\prime} =
\begin{pmatrix}
0.482729\\
0.373672\\
0.068961\\
0.074639
\end{pmatrix} =
0.999990\cdot
\begin{pmatrix}
0.482733\\
0.373675\\
\color{gr} 0.068962\color{black} \\
0.074639
\end{pmatrix},
\end{equation*}
\begin{equation*}
\left[ \frac{{w}^{\prime}_i}{{w}^{\prime}_j} \right] =
\begin{pmatrix}
$\,\,$ 1 $\,\,$ & $\,\,$1.2919$\,\,$ & $\,\,$\color{gr} \color{blue} 7\color{black} $\,\,$ & $\,\,$6.4676$\,\,$ \\
$\,\,$0.7741$\,\,$ & $\,\,$ 1 $\,\,$ & $\,\,$\color{gr} 5.4186\color{black} $\,\,$ & $\,\,$5.0064  $\,\,$ \\
$\,\,$\color{gr} \color{blue}  1/7\color{black} $\,\,$ & $\,\,$\color{gr} 0.1846\color{black} $\,\,$ & $\,\,$ 1 $\,\,$ & $\,\,$\color{gr} 0.9239\color{black}  $\,\,$ \\
$\,\,$0.1546$\,\,$ & $\,\,$0.1997$\,\,$ & $\,\,$\color{gr} 1.0823\color{black} $\,\,$ & $\,\,$ 1  $\,\,$ \\
\end{pmatrix},
\end{equation*}
\end{example}
\newpage
\begin{example}
\begin{equation*}
\mathbf{A} =
\begin{pmatrix}
$\,\,$ 1 $\,\,$ & $\,\,$2$\,\,$ & $\,\,$7$\,\,$ & $\,\,$4 $\,\,$ \\
$\,\,$ 1/2$\,\,$ & $\,\,$ 1 $\,\,$ & $\,\,$5$\,\,$ & $\,\,$9 $\,\,$ \\
$\,\,$ 1/7$\,\,$ & $\,\,$ 1/5$\,\,$ & $\,\,$ 1 $\,\,$ & $\,\,$1 $\,\,$ \\
$\,\,$ 1/4$\,\,$ & $\,\,$ 1/9$\,\,$ & $\,\,$ 1 $\,\,$ & $\,\,$ 1  $\,\,$ \\
\end{pmatrix},
\qquad
\lambda_{\max} =
4.1975,
\qquad
CR = 0.0745
\end{equation*}

\begin{equation*}
\mathbf{w}^{EM} =
\begin{pmatrix}
0.477613\\
0.382807\\
\color{red} 0.067748\color{black} \\
0.071832
\end{pmatrix}\end{equation*}
\begin{equation*}
\left[ \frac{{w}^{EM}_i}{{w}^{EM}_j} \right] =
\begin{pmatrix}
$\,\,$ 1 $\,\,$ & $\,\,$1.2477$\,\,$ & $\,\,$\color{red} 7.0499\color{black} $\,\,$ & $\,\,$6.6490$\,\,$ \\
$\,\,$0.8015$\,\,$ & $\,\,$ 1 $\,\,$ & $\,\,$\color{red} 5.6505\color{black} $\,\,$ & $\,\,$5.3292  $\,\,$ \\
$\,\,$\color{red} 0.1418\color{black} $\,\,$ & $\,\,$\color{red} 0.1770\color{black} $\,\,$ & $\,\,$ 1 $\,\,$ & $\,\,$\color{red} 0.9431\color{black}  $\,\,$ \\
$\,\,$0.1504$\,\,$ & $\,\,$0.1876$\,\,$ & $\,\,$\color{red} 1.0603\color{black} $\,\,$ & $\,\,$ 1  $\,\,$ \\
\end{pmatrix},
\end{equation*}

\begin{equation*}
\mathbf{w}^{\prime} =
\begin{pmatrix}
0.477382\\
0.382623\\
0.068197\\
0.071798
\end{pmatrix} =
0.999517\cdot
\begin{pmatrix}
0.477613\\
0.382807\\
\color{gr} 0.068230\color{black} \\
0.071832
\end{pmatrix},
\end{equation*}
\begin{equation*}
\left[ \frac{{w}^{\prime}_i}{{w}^{\prime}_j} \right] =
\begin{pmatrix}
$\,\,$ 1 $\,\,$ & $\,\,$1.2477$\,\,$ & $\,\,$\color{gr} \color{blue} 7\color{black} $\,\,$ & $\,\,$6.6490$\,\,$ \\
$\,\,$0.8015$\,\,$ & $\,\,$ 1 $\,\,$ & $\,\,$\color{gr} 5.6105\color{black} $\,\,$ & $\,\,$5.3292  $\,\,$ \\
$\,\,$\color{gr} \color{blue}  1/7\color{black} $\,\,$ & $\,\,$\color{gr} 0.1782\color{black} $\,\,$ & $\,\,$ 1 $\,\,$ & $\,\,$\color{gr} 0.9499\color{black}  $\,\,$ \\
$\,\,$0.1504$\,\,$ & $\,\,$0.1876$\,\,$ & $\,\,$\color{gr} 1.0528\color{black} $\,\,$ & $\,\,$ 1  $\,\,$ \\
\end{pmatrix},
\end{equation*}
\end{example}
\newpage
\begin{example}
\begin{equation*}
\mathbf{A} =
\begin{pmatrix}
$\,\,$ 1 $\,\,$ & $\,\,$2$\,\,$ & $\,\,$7$\,\,$ & $\,\,$4 $\,\,$ \\
$\,\,$ 1/2$\,\,$ & $\,\,$ 1 $\,\,$ & $\,\,$6$\,\,$ & $\,\,$9 $\,\,$ \\
$\,\,$ 1/7$\,\,$ & $\,\,$ 1/6$\,\,$ & $\,\,$ 1 $\,\,$ & $\,\,$1 $\,\,$ \\
$\,\,$ 1/4$\,\,$ & $\,\,$ 1/9$\,\,$ & $\,\,$ 1 $\,\,$ & $\,\,$ 1  $\,\,$ \\
\end{pmatrix},
\qquad
\lambda_{\max} =
4.1964,
\qquad
CR = 0.0741
\end{equation*}

\begin{equation*}
\mathbf{w}^{EM} =
\begin{pmatrix}
0.473343\\
0.392370\\
\color{red} 0.063698\color{black} \\
0.070589
\end{pmatrix}\end{equation*}
\begin{equation*}
\left[ \frac{{w}^{EM}_i}{{w}^{EM}_j} \right] =
\begin{pmatrix}
$\,\,$ 1 $\,\,$ & $\,\,$1.2064$\,\,$ & $\,\,$\color{red} 7.4310\color{black} $\,\,$ & $\,\,$6.7056$\,\,$ \\
$\,\,$0.8289$\,\,$ & $\,\,$ 1 $\,\,$ & $\,\,$\color{red} 6.1598\color{black} $\,\,$ & $\,\,$5.5585  $\,\,$ \\
$\,\,$\color{red} 0.1346\color{black} $\,\,$ & $\,\,$\color{red} 0.1623\color{black} $\,\,$ & $\,\,$ 1 $\,\,$ & $\,\,$\color{red} 0.9024\color{black}  $\,\,$ \\
$\,\,$0.1491$\,\,$ & $\,\,$0.1799$\,\,$ & $\,\,$\color{red} 1.1082\color{black} $\,\,$ & $\,\,$ 1  $\,\,$ \\
\end{pmatrix},
\end{equation*}

\begin{equation*}
\mathbf{w}^{\prime} =
\begin{pmatrix}
0.472541\\
0.391705\\
0.065284\\
0.070470
\end{pmatrix} =
0.998306\cdot
\begin{pmatrix}
0.473343\\
0.392370\\
\color{gr} 0.065395\color{black} \\
0.070589
\end{pmatrix},
\end{equation*}
\begin{equation*}
\left[ \frac{{w}^{\prime}_i}{{w}^{\prime}_j} \right] =
\begin{pmatrix}
$\,\,$ 1 $\,\,$ & $\,\,$1.2064$\,\,$ & $\,\,$\color{gr} 7.2382\color{black} $\,\,$ & $\,\,$6.7056$\,\,$ \\
$\,\,$0.8289$\,\,$ & $\,\,$ 1 $\,\,$ & $\,\,$\color{gr} \color{blue} 6\color{black} $\,\,$ & $\,\,$5.5585  $\,\,$ \\
$\,\,$\color{gr} 0.1382\color{black} $\,\,$ & $\,\,$\color{gr} \color{blue}  1/6\color{black} $\,\,$ & $\,\,$ 1 $\,\,$ & $\,\,$\color{gr} 0.9264\color{black}  $\,\,$ \\
$\,\,$0.1491$\,\,$ & $\,\,$0.1799$\,\,$ & $\,\,$\color{gr} 1.0794\color{black} $\,\,$ & $\,\,$ 1  $\,\,$ \\
\end{pmatrix},
\end{equation*}
\end{example}
\newpage
\begin{example}
\begin{equation*}
\mathbf{A} =
\begin{pmatrix}
$\,\,$ 1 $\,\,$ & $\,\,$2$\,\,$ & $\,\,$7$\,\,$ & $\,\,$5 $\,\,$ \\
$\,\,$ 1/2$\,\,$ & $\,\,$ 1 $\,\,$ & $\,\,$2$\,\,$ & $\,\,$4 $\,\,$ \\
$\,\,$ 1/7$\,\,$ & $\,\,$ 1/2$\,\,$ & $\,\,$ 1 $\,\,$ & $\,\,$3 $\,\,$ \\
$\,\,$ 1/5$\,\,$ & $\,\,$ 1/4$\,\,$ & $\,\,$ 1/3$\,\,$ & $\,\,$ 1  $\,\,$ \\
\end{pmatrix},
\qquad
\lambda_{\max} =
4.1782,
\qquad
CR = 0.0672
\end{equation*}

\begin{equation*}
\mathbf{w}^{EM} =
\begin{pmatrix}
0.549998\\
\color{red} 0.253192\color{black} \\
0.128775\\
0.068034
\end{pmatrix}\end{equation*}
\begin{equation*}
\left[ \frac{{w}^{EM}_i}{{w}^{EM}_j} \right] =
\begin{pmatrix}
$\,\,$ 1 $\,\,$ & $\,\,$\color{red} 2.1723\color{black} $\,\,$ & $\,\,$4.2710$\,\,$ & $\,\,$8.0842$\,\,$ \\
$\,\,$\color{red} 0.4604\color{black} $\,\,$ & $\,\,$ 1 $\,\,$ & $\,\,$\color{red} 1.9662\color{black} $\,\,$ & $\,\,$\color{red} 3.7216\color{black}   $\,\,$ \\
$\,\,$0.2341$\,\,$ & $\,\,$\color{red} 0.5086\color{black} $\,\,$ & $\,\,$ 1 $\,\,$ & $\,\,$1.8928 $\,\,$ \\
$\,\,$0.1237$\,\,$ & $\,\,$\color{red} 0.2687\color{black} $\,\,$ & $\,\,$0.5283$\,\,$ & $\,\,$ 1  $\,\,$ \\
\end{pmatrix},
\end{equation*}

\begin{equation*}
\mathbf{w}^{\prime} =
\begin{pmatrix}
0.547611\\
0.256433\\
0.128217\\
0.067739
\end{pmatrix} =
0.995660\cdot
\begin{pmatrix}
0.549998\\
\color{gr} 0.257551\color{black} \\
0.128775\\
0.068034
\end{pmatrix},
\end{equation*}
\begin{equation*}
\left[ \frac{{w}^{\prime}_i}{{w}^{\prime}_j} \right] =
\begin{pmatrix}
$\,\,$ 1 $\,\,$ & $\,\,$\color{gr} 2.1355\color{black} $\,\,$ & $\,\,$4.2710$\,\,$ & $\,\,$8.0842$\,\,$ \\
$\,\,$\color{gr} 0.4683\color{black} $\,\,$ & $\,\,$ 1 $\,\,$ & $\,\,$\color{gr} \color{blue} 2\color{black} $\,\,$ & $\,\,$\color{gr} 3.7856\color{black}   $\,\,$ \\
$\,\,$0.2341$\,\,$ & $\,\,$\color{gr} \color{blue}  1/2\color{black} $\,\,$ & $\,\,$ 1 $\,\,$ & $\,\,$1.8928 $\,\,$ \\
$\,\,$0.1237$\,\,$ & $\,\,$\color{gr} 0.2642\color{black} $\,\,$ & $\,\,$0.5283$\,\,$ & $\,\,$ 1  $\,\,$ \\
\end{pmatrix},
\end{equation*}
\end{example}
\newpage
\begin{example}
\begin{equation*}
\mathbf{A} =
\begin{pmatrix}
$\,\,$ 1 $\,\,$ & $\,\,$2$\,\,$ & $\,\,$7$\,\,$ & $\,\,$5 $\,\,$ \\
$\,\,$ 1/2$\,\,$ & $\,\,$ 1 $\,\,$ & $\,\,$2$\,\,$ & $\,\,$4 $\,\,$ \\
$\,\,$ 1/7$\,\,$ & $\,\,$ 1/2$\,\,$ & $\,\,$ 1 $\,\,$ & $\,\,$4 $\,\,$ \\
$\,\,$ 1/5$\,\,$ & $\,\,$ 1/4$\,\,$ & $\,\,$ 1/4$\,\,$ & $\,\,$ 1  $\,\,$ \\
\end{pmatrix},
\qquad
\lambda_{\max} =
4.2610,
\qquad
CR = 0.0984
\end{equation*}

\begin{equation*}
\mathbf{w}^{EM} =
\begin{pmatrix}
0.549151\\
\color{red} 0.247681\color{black} \\
0.139783\\
0.063385
\end{pmatrix}\end{equation*}
\begin{equation*}
\left[ \frac{{w}^{EM}_i}{{w}^{EM}_j} \right] =
\begin{pmatrix}
$\,\,$ 1 $\,\,$ & $\,\,$\color{red} 2.2172\color{black} $\,\,$ & $\,\,$3.9286$\,\,$ & $\,\,$8.6638$\,\,$ \\
$\,\,$\color{red} 0.4510\color{black} $\,\,$ & $\,\,$ 1 $\,\,$ & $\,\,$\color{red} 1.7719\color{black} $\,\,$ & $\,\,$\color{red} 3.9076\color{black}   $\,\,$ \\
$\,\,$0.2545$\,\,$ & $\,\,$\color{red} 0.5644\color{black} $\,\,$ & $\,\,$ 1 $\,\,$ & $\,\,$2.2053 $\,\,$ \\
$\,\,$0.1154$\,\,$ & $\,\,$\color{red} 0.2559\color{black} $\,\,$ & $\,\,$0.4535$\,\,$ & $\,\,$ 1  $\,\,$ \\
\end{pmatrix},
\end{equation*}

\begin{equation*}
\mathbf{w}^{\prime} =
\begin{pmatrix}
0.545953\\
0.252062\\
0.138969\\
0.063016
\end{pmatrix} =
0.994176\cdot
\begin{pmatrix}
0.549151\\
\color{gr} 0.253539\color{black} \\
0.139783\\
0.063385
\end{pmatrix},
\end{equation*}
\begin{equation*}
\left[ \frac{{w}^{\prime}_i}{{w}^{\prime}_j} \right] =
\begin{pmatrix}
$\,\,$ 1 $\,\,$ & $\,\,$\color{gr} 2.1659\color{black} $\,\,$ & $\,\,$3.9286$\,\,$ & $\,\,$8.6638$\,\,$ \\
$\,\,$\color{gr} 0.4617\color{black} $\,\,$ & $\,\,$ 1 $\,\,$ & $\,\,$\color{gr} 1.8138\color{black} $\,\,$ & $\,\,$\color{gr} \color{blue} 4\color{black}   $\,\,$ \\
$\,\,$0.2545$\,\,$ & $\,\,$\color{gr} 0.5513\color{black} $\,\,$ & $\,\,$ 1 $\,\,$ & $\,\,$2.2053 $\,\,$ \\
$\,\,$0.1154$\,\,$ & $\,\,$\color{gr} \color{blue}  1/4\color{black} $\,\,$ & $\,\,$0.4535$\,\,$ & $\,\,$ 1  $\,\,$ \\
\end{pmatrix},
\end{equation*}
\end{example}
\newpage
\begin{example}
\begin{equation*}
\mathbf{A} =
\begin{pmatrix}
$\,\,$ 1 $\,\,$ & $\,\,$2$\,\,$ & $\,\,$7$\,\,$ & $\,\,$5 $\,\,$ \\
$\,\,$ 1/2$\,\,$ & $\,\,$ 1 $\,\,$ & $\,\,$2$\,\,$ & $\,\,$5 $\,\,$ \\
$\,\,$ 1/7$\,\,$ & $\,\,$ 1/2$\,\,$ & $\,\,$ 1 $\,\,$ & $\,\,$4 $\,\,$ \\
$\,\,$ 1/5$\,\,$ & $\,\,$ 1/5$\,\,$ & $\,\,$ 1/4$\,\,$ & $\,\,$ 1  $\,\,$ \\
\end{pmatrix},
\qquad
\lambda_{\max} =
4.2610,
\qquad
CR = 0.0984
\end{equation*}

\begin{equation*}
\mathbf{w}^{EM} =
\begin{pmatrix}
0.544321\\
\color{red} 0.259042\color{black} \\
0.136873\\
0.059765
\end{pmatrix}\end{equation*}
\begin{equation*}
\left[ \frac{{w}^{EM}_i}{{w}^{EM}_j} \right] =
\begin{pmatrix}
$\,\,$ 1 $\,\,$ & $\,\,$\color{red} 2.1013\color{black} $\,\,$ & $\,\,$3.9768$\,\,$ & $\,\,$9.1078$\,\,$ \\
$\,\,$\color{red} 0.4759\color{black} $\,\,$ & $\,\,$ 1 $\,\,$ & $\,\,$\color{red} 1.8926\color{black} $\,\,$ & $\,\,$\color{red} 4.3344\color{black}   $\,\,$ \\
$\,\,$0.2515$\,\,$ & $\,\,$\color{red} 0.5284\color{black} $\,\,$ & $\,\,$ 1 $\,\,$ & $\,\,$2.2902 $\,\,$ \\
$\,\,$0.1098$\,\,$ & $\,\,$\color{red} 0.2307\color{black} $\,\,$ & $\,\,$0.4366$\,\,$ & $\,\,$ 1  $\,\,$ \\
\end{pmatrix},
\end{equation*}

\begin{equation*}
\mathbf{w}^{\prime} =
\begin{pmatrix}
0.537272\\
0.268636\\
0.135101\\
0.058991
\end{pmatrix} =
0.987051\cdot
\begin{pmatrix}
0.544321\\
\color{gr} 0.272160\color{black} \\
0.136873\\
0.059765
\end{pmatrix},
\end{equation*}
\begin{equation*}
\left[ \frac{{w}^{\prime}_i}{{w}^{\prime}_j} \right] =
\begin{pmatrix}
$\,\,$ 1 $\,\,$ & $\,\,$\color{gr} \color{blue} 2\color{black} $\,\,$ & $\,\,$3.9768$\,\,$ & $\,\,$9.1078$\,\,$ \\
$\,\,$\color{gr} \color{blue}  1/2\color{black} $\,\,$ & $\,\,$ 1 $\,\,$ & $\,\,$\color{gr} 1.9884\color{black} $\,\,$ & $\,\,$\color{gr} 4.5539\color{black}   $\,\,$ \\
$\,\,$0.2515$\,\,$ & $\,\,$\color{gr} 0.5029\color{black} $\,\,$ & $\,\,$ 1 $\,\,$ & $\,\,$2.2902 $\,\,$ \\
$\,\,$0.1098$\,\,$ & $\,\,$\color{gr} 0.2196\color{black} $\,\,$ & $\,\,$0.4366$\,\,$ & $\,\,$ 1  $\,\,$ \\
\end{pmatrix},
\end{equation*}
\end{example}
\newpage
\begin{example}
\begin{equation*}
\mathbf{A} =
\begin{pmatrix}
$\,\,$ 1 $\,\,$ & $\,\,$2$\,\,$ & $\,\,$7$\,\,$ & $\,\,$5 $\,\,$ \\
$\,\,$ 1/2$\,\,$ & $\,\,$ 1 $\,\,$ & $\,\,$5$\,\,$ & $\,\,$7 $\,\,$ \\
$\,\,$ 1/7$\,\,$ & $\,\,$ 1/5$\,\,$ & $\,\,$ 1 $\,\,$ & $\,\,$1 $\,\,$ \\
$\,\,$ 1/5$\,\,$ & $\,\,$ 1/7$\,\,$ & $\,\,$ 1 $\,\,$ & $\,\,$ 1  $\,\,$ \\
\end{pmatrix},
\qquad
\lambda_{\max} =
4.0899,
\qquad
CR = 0.0339
\end{equation*}

\begin{equation*}
\mathbf{w}^{EM} =
\begin{pmatrix}
0.503288\\
0.355778\\
\color{red} 0.069437\color{black} \\
0.071498
\end{pmatrix}\end{equation*}
\begin{equation*}
\left[ \frac{{w}^{EM}_i}{{w}^{EM}_j} \right] =
\begin{pmatrix}
$\,\,$ 1 $\,\,$ & $\,\,$1.4146$\,\,$ & $\,\,$\color{red} 7.2481\color{black} $\,\,$ & $\,\,$7.0392$\,\,$ \\
$\,\,$0.7069$\,\,$ & $\,\,$ 1 $\,\,$ & $\,\,$\color{red} 5.1238\color{black} $\,\,$ & $\,\,$4.9761  $\,\,$ \\
$\,\,$\color{red} 0.1380\color{black} $\,\,$ & $\,\,$\color{red} 0.1952\color{black} $\,\,$ & $\,\,$ 1 $\,\,$ & $\,\,$\color{red} 0.9712\color{black}  $\,\,$ \\
$\,\,$0.1421$\,\,$ & $\,\,$0.2010$\,\,$ & $\,\,$\color{red} 1.0297\color{black} $\,\,$ & $\,\,$ 1  $\,\,$ \\
\end{pmatrix},
\end{equation*}

\begin{equation*}
\mathbf{w}^{\prime} =
\begin{pmatrix}
0.502424\\
0.355167\\
0.071033\\
0.071375
\end{pmatrix} =
0.998284\cdot
\begin{pmatrix}
0.503288\\
0.355778\\
\color{gr} 0.071156\color{black} \\
0.071498
\end{pmatrix},
\end{equation*}
\begin{equation*}
\left[ \frac{{w}^{\prime}_i}{{w}^{\prime}_j} \right] =
\begin{pmatrix}
$\,\,$ 1 $\,\,$ & $\,\,$1.4146$\,\,$ & $\,\,$\color{gr} 7.0731\color{black} $\,\,$ & $\,\,$7.0392$\,\,$ \\
$\,\,$0.7069$\,\,$ & $\,\,$ 1 $\,\,$ & $\,\,$\color{gr} \color{blue} 5\color{black} $\,\,$ & $\,\,$4.9761  $\,\,$ \\
$\,\,$\color{gr} 0.1414\color{black} $\,\,$ & $\,\,$\color{gr} \color{blue}  1/5\color{black} $\,\,$ & $\,\,$ 1 $\,\,$ & $\,\,$\color{gr} 0.9952\color{black}  $\,\,$ \\
$\,\,$0.1421$\,\,$ & $\,\,$0.2010$\,\,$ & $\,\,$\color{gr} 1.0048\color{black} $\,\,$ & $\,\,$ 1  $\,\,$ \\
\end{pmatrix},
\end{equation*}
\end{example}
\newpage
\begin{example}
\begin{equation*}
\mathbf{A} =
\begin{pmatrix}
$\,\,$ 1 $\,\,$ & $\,\,$2$\,\,$ & $\,\,$7$\,\,$ & $\,\,$5 $\,\,$ \\
$\,\,$ 1/2$\,\,$ & $\,\,$ 1 $\,\,$ & $\,\,$6$\,\,$ & $\,\,$9 $\,\,$ \\
$\,\,$ 1/7$\,\,$ & $\,\,$ 1/6$\,\,$ & $\,\,$ 1 $\,\,$ & $\,\,$1 $\,\,$ \\
$\,\,$ 1/5$\,\,$ & $\,\,$ 1/9$\,\,$ & $\,\,$ 1 $\,\,$ & $\,\,$ 1  $\,\,$ \\
\end{pmatrix},
\qquad
\lambda_{\max} =
4.1417,
\qquad
CR = 0.0534
\end{equation*}

\begin{equation*}
\mathbf{w}^{EM} =
\begin{pmatrix}
0.488194\\
0.383868\\
\color{red} 0.063176\color{black} \\
0.064762
\end{pmatrix}\end{equation*}
\begin{equation*}
\left[ \frac{{w}^{EM}_i}{{w}^{EM}_j} \right] =
\begin{pmatrix}
$\,\,$ 1 $\,\,$ & $\,\,$1.2718$\,\,$ & $\,\,$\color{red} 7.7275\color{black} $\,\,$ & $\,\,$7.5382$\,\,$ \\
$\,\,$0.7863$\,\,$ & $\,\,$ 1 $\,\,$ & $\,\,$\color{red} 6.0762\color{black} $\,\,$ & $\,\,$5.9273  $\,\,$ \\
$\,\,$\color{red} 0.1294\color{black} $\,\,$ & $\,\,$\color{red} 0.1646\color{black} $\,\,$ & $\,\,$ 1 $\,\,$ & $\,\,$\color{red} 0.9755\color{black}  $\,\,$ \\
$\,\,$0.1327$\,\,$ & $\,\,$0.1687$\,\,$ & $\,\,$\color{red} 1.0251\color{black} $\,\,$ & $\,\,$ 1  $\,\,$ \\
\end{pmatrix},
\end{equation*}

\begin{equation*}
\mathbf{w}^{\prime} =
\begin{pmatrix}
0.487803\\
0.383560\\
0.063927\\
0.064710
\end{pmatrix} =
0.999199\cdot
\begin{pmatrix}
0.488194\\
0.383868\\
\color{gr} 0.063978\color{black} \\
0.064762
\end{pmatrix},
\end{equation*}
\begin{equation*}
\left[ \frac{{w}^{\prime}_i}{{w}^{\prime}_j} \right] =
\begin{pmatrix}
$\,\,$ 1 $\,\,$ & $\,\,$1.2718$\,\,$ & $\,\,$\color{gr} 7.6307\color{black} $\,\,$ & $\,\,$7.5382$\,\,$ \\
$\,\,$0.7863$\,\,$ & $\,\,$ 1 $\,\,$ & $\,\,$\color{gr} \color{blue} 6\color{black} $\,\,$ & $\,\,$5.9273  $\,\,$ \\
$\,\,$\color{gr} 0.1311\color{black} $\,\,$ & $\,\,$\color{gr} \color{blue}  1/6\color{black} $\,\,$ & $\,\,$ 1 $\,\,$ & $\,\,$\color{gr} 0.9879\color{black}  $\,\,$ \\
$\,\,$0.1327$\,\,$ & $\,\,$0.1687$\,\,$ & $\,\,$\color{gr} 1.0123\color{black} $\,\,$ & $\,\,$ 1  $\,\,$ \\
\end{pmatrix},
\end{equation*}
\end{example}
\newpage
\begin{example}
\begin{equation*}
\mathbf{A} =
\begin{pmatrix}
$\,\,$ 1 $\,\,$ & $\,\,$2$\,\,$ & $\,\,$7$\,\,$ & $\,\,$6 $\,\,$ \\
$\,\,$ 1/2$\,\,$ & $\,\,$ 1 $\,\,$ & $\,\,$2$\,\,$ & $\,\,$4 $\,\,$ \\
$\,\,$ 1/7$\,\,$ & $\,\,$ 1/2$\,\,$ & $\,\,$ 1 $\,\,$ & $\,\,$3 $\,\,$ \\
$\,\,$ 1/6$\,\,$ & $\,\,$ 1/4$\,\,$ & $\,\,$ 1/3$\,\,$ & $\,\,$ 1  $\,\,$ \\
\end{pmatrix},
\qquad
\lambda_{\max} =
4.1365,
\qquad
CR = 0.0515
\end{equation*}

\begin{equation*}
\mathbf{w}^{EM} =
\begin{pmatrix}
0.560968\\
\color{red} 0.250127\color{black} \\
0.125791\\
0.063114
\end{pmatrix}\end{equation*}
\begin{equation*}
\left[ \frac{{w}^{EM}_i}{{w}^{EM}_j} \right] =
\begin{pmatrix}
$\,\,$ 1 $\,\,$ & $\,\,$\color{red} 2.2427\color{black} $\,\,$ & $\,\,$4.4595$\,\,$ & $\,\,$8.8882$\,\,$ \\
$\,\,$\color{red} 0.4459\color{black} $\,\,$ & $\,\,$ 1 $\,\,$ & $\,\,$\color{red} 1.9884\color{black} $\,\,$ & $\,\,$\color{red} 3.9631\color{black}   $\,\,$ \\
$\,\,$0.2242$\,\,$ & $\,\,$\color{red} 0.5029\color{black} $\,\,$ & $\,\,$ 1 $\,\,$ & $\,\,$1.9931 $\,\,$ \\
$\,\,$0.1125$\,\,$ & $\,\,$\color{red} 0.2523\color{black} $\,\,$ & $\,\,$0.5017$\,\,$ & $\,\,$ 1  $\,\,$ \\
\end{pmatrix},
\end{equation*}

\begin{equation*}
\mathbf{w}^{\prime} =
\begin{pmatrix}
0.560153\\
0.251217\\
0.125608\\
0.063022
\end{pmatrix} =
0.998547\cdot
\begin{pmatrix}
0.560968\\
\color{gr} 0.251582\color{black} \\
0.125791\\
0.063114
\end{pmatrix},
\end{equation*}
\begin{equation*}
\left[ \frac{{w}^{\prime}_i}{{w}^{\prime}_j} \right] =
\begin{pmatrix}
$\,\,$ 1 $\,\,$ & $\,\,$\color{gr} 2.2298\color{black} $\,\,$ & $\,\,$4.4595$\,\,$ & $\,\,$8.8882$\,\,$ \\
$\,\,$\color{gr} 0.4485\color{black} $\,\,$ & $\,\,$ 1 $\,\,$ & $\,\,$\color{gr} \color{blue} 2\color{black} $\,\,$ & $\,\,$\color{gr} 3.9862\color{black}   $\,\,$ \\
$\,\,$0.2242$\,\,$ & $\,\,$\color{gr} \color{blue}  1/2\color{black} $\,\,$ & $\,\,$ 1 $\,\,$ & $\,\,$1.9931 $\,\,$ \\
$\,\,$0.1125$\,\,$ & $\,\,$\color{gr} 0.2509\color{black} $\,\,$ & $\,\,$0.5017$\,\,$ & $\,\,$ 1  $\,\,$ \\
\end{pmatrix},
\end{equation*}
\end{example}
\newpage
\begin{example}
\begin{equation*}
\mathbf{A} =
\begin{pmatrix}
$\,\,$ 1 $\,\,$ & $\,\,$2$\,\,$ & $\,\,$7$\,\,$ & $\,\,$6 $\,\,$ \\
$\,\,$ 1/2$\,\,$ & $\,\,$ 1 $\,\,$ & $\,\,$2$\,\,$ & $\,\,$5 $\,\,$ \\
$\,\,$ 1/7$\,\,$ & $\,\,$ 1/2$\,\,$ & $\,\,$ 1 $\,\,$ & $\,\,$4 $\,\,$ \\
$\,\,$ 1/6$\,\,$ & $\,\,$ 1/5$\,\,$ & $\,\,$ 1/4$\,\,$ & $\,\,$ 1  $\,\,$ \\
\end{pmatrix},
\qquad
\lambda_{\max} =
4.2057,
\qquad
CR = 0.0776
\end{equation*}

\begin{equation*}
\mathbf{w}^{EM} =
\begin{pmatrix}
0.555001\\
\color{red} 0.256117\color{black} \\
0.133627\\
0.055255
\end{pmatrix}\end{equation*}
\begin{equation*}
\left[ \frac{{w}^{EM}_i}{{w}^{EM}_j} \right] =
\begin{pmatrix}
$\,\,$ 1 $\,\,$ & $\,\,$\color{red} 2.1670\color{black} $\,\,$ & $\,\,$4.1534$\,\,$ & $\,\,$10.0443$\,\,$ \\
$\,\,$\color{red} 0.4615\color{black} $\,\,$ & $\,\,$ 1 $\,\,$ & $\,\,$\color{red} 1.9167\color{black} $\,\,$ & $\,\,$\color{red} 4.6352\color{black}   $\,\,$ \\
$\,\,$0.2408$\,\,$ & $\,\,$\color{red} 0.5217\color{black} $\,\,$ & $\,\,$ 1 $\,\,$ & $\,\,$2.4184 $\,\,$ \\
$\,\,$0.0996$\,\,$ & $\,\,$\color{red} 0.2157\color{black} $\,\,$ & $\,\,$0.4135$\,\,$ & $\,\,$ 1  $\,\,$ \\
\end{pmatrix},
\end{equation*}

\begin{equation*}
\mathbf{w}^{\prime} =
\begin{pmatrix}
0.548888\\
0.264310\\
0.132155\\
0.054647
\end{pmatrix} =
0.988986\cdot
\begin{pmatrix}
0.555001\\
\color{gr} 0.267254\color{black} \\
0.133627\\
0.055255
\end{pmatrix},
\end{equation*}
\begin{equation*}
\left[ \frac{{w}^{\prime}_i}{{w}^{\prime}_j} \right] =
\begin{pmatrix}
$\,\,$ 1 $\,\,$ & $\,\,$\color{gr} 2.0767\color{black} $\,\,$ & $\,\,$4.1534$\,\,$ & $\,\,$10.0443$\,\,$ \\
$\,\,$\color{gr} 0.4815\color{black} $\,\,$ & $\,\,$ 1 $\,\,$ & $\,\,$\color{gr} \color{blue} 2\color{black} $\,\,$ & $\,\,$\color{gr} 4.8367\color{black}   $\,\,$ \\
$\,\,$0.2408$\,\,$ & $\,\,$\color{gr} \color{blue}  1/2\color{black} $\,\,$ & $\,\,$ 1 $\,\,$ & $\,\,$2.4184 $\,\,$ \\
$\,\,$0.0996$\,\,$ & $\,\,$\color{gr} 0.2068\color{black} $\,\,$ & $\,\,$0.4135$\,\,$ & $\,\,$ 1  $\,\,$ \\
\end{pmatrix},
\end{equation*}
\end{example}
\newpage
\begin{example}
\begin{equation*}
\mathbf{A} =
\begin{pmatrix}
$\,\,$ 1 $\,\,$ & $\,\,$2$\,\,$ & $\,\,$7$\,\,$ & $\,\,$7 $\,\,$ \\
$\,\,$ 1/2$\,\,$ & $\,\,$ 1 $\,\,$ & $\,\,$2$\,\,$ & $\,\,$5 $\,\,$ \\
$\,\,$ 1/7$\,\,$ & $\,\,$ 1/2$\,\,$ & $\,\,$ 1 $\,\,$ & $\,\,$4 $\,\,$ \\
$\,\,$ 1/7$\,\,$ & $\,\,$ 1/5$\,\,$ & $\,\,$ 1/4$\,\,$ & $\,\,$ 1  $\,\,$ \\
\end{pmatrix},
\qquad
\lambda_{\max} =
4.1665,
\qquad
CR = 0.0628
\end{equation*}

\begin{equation*}
\mathbf{w}^{EM} =
\begin{pmatrix}
0.563879\\
\color{red} 0.253465\color{black} \\
0.130874\\
0.051781
\end{pmatrix}\end{equation*}
\begin{equation*}
\left[ \frac{{w}^{EM}_i}{{w}^{EM}_j} \right] =
\begin{pmatrix}
$\,\,$ 1 $\,\,$ & $\,\,$\color{red} 2.2247\color{black} $\,\,$ & $\,\,$4.3086$\,\,$ & $\,\,$10.8896$\,\,$ \\
$\,\,$\color{red} 0.4495\color{black} $\,\,$ & $\,\,$ 1 $\,\,$ & $\,\,$\color{red} 1.9367\color{black} $\,\,$ & $\,\,$\color{red} 4.8949\color{black}   $\,\,$ \\
$\,\,$0.2321$\,\,$ & $\,\,$\color{red} 0.5163\color{black} $\,\,$ & $\,\,$ 1 $\,\,$ & $\,\,$2.5274 $\,\,$ \\
$\,\,$0.0918$\,\,$ & $\,\,$\color{red} 0.2043\color{black} $\,\,$ & $\,\,$0.3957$\,\,$ & $\,\,$ 1  $\,\,$ \\
\end{pmatrix},
\end{equation*}

\begin{equation*}
\mathbf{w}^{\prime} =
\begin{pmatrix}
0.560827\\
0.257506\\
0.130166\\
0.051501
\end{pmatrix} =
0.994587\cdot
\begin{pmatrix}
0.563879\\
\color{gr} 0.258907\color{black} \\
0.130874\\
0.051781
\end{pmatrix},
\end{equation*}
\begin{equation*}
\left[ \frac{{w}^{\prime}_i}{{w}^{\prime}_j} \right] =
\begin{pmatrix}
$\,\,$ 1 $\,\,$ & $\,\,$\color{gr} 2.1779\color{black} $\,\,$ & $\,\,$4.3086$\,\,$ & $\,\,$10.8896$\,\,$ \\
$\,\,$\color{gr} 0.4592\color{black} $\,\,$ & $\,\,$ 1 $\,\,$ & $\,\,$\color{gr} 1.9783\color{black} $\,\,$ & $\,\,$\color{gr} \color{blue} 5\color{black}   $\,\,$ \\
$\,\,$0.2321$\,\,$ & $\,\,$\color{gr} 0.5055\color{black} $\,\,$ & $\,\,$ 1 $\,\,$ & $\,\,$2.5274 $\,\,$ \\
$\,\,$0.0918$\,\,$ & $\,\,$\color{gr} \color{blue}  1/5\color{black} $\,\,$ & $\,\,$0.3957$\,\,$ & $\,\,$ 1  $\,\,$ \\
\end{pmatrix},
\end{equation*}
\end{example}
\newpage
\begin{example}
\begin{equation*}
\mathbf{A} =
\begin{pmatrix}
$\,\,$ 1 $\,\,$ & $\,\,$2$\,\,$ & $\,\,$7$\,\,$ & $\,\,$7 $\,\,$ \\
$\,\,$ 1/2$\,\,$ & $\,\,$ 1 $\,\,$ & $\,\,$2$\,\,$ & $\,\,$6 $\,\,$ \\
$\,\,$ 1/7$\,\,$ & $\,\,$ 1/2$\,\,$ & $\,\,$ 1 $\,\,$ & $\,\,$5 $\,\,$ \\
$\,\,$ 1/7$\,\,$ & $\,\,$ 1/6$\,\,$ & $\,\,$ 1/5$\,\,$ & $\,\,$ 1  $\,\,$ \\
\end{pmatrix},
\qquad
\lambda_{\max} =
4.2251,
\qquad
CR = 0.0849
\end{equation*}

\begin{equation*}
\mathbf{w}^{EM} =
\begin{pmatrix}
0.558372\\
\color{red} 0.258121\color{black} \\
0.136943\\
0.046565
\end{pmatrix}\end{equation*}
\begin{equation*}
\left[ \frac{{w}^{EM}_i}{{w}^{EM}_j} \right] =
\begin{pmatrix}
$\,\,$ 1 $\,\,$ & $\,\,$\color{red} 2.1632\color{black} $\,\,$ & $\,\,$4.0774$\,\,$ & $\,\,$11.9912$\,\,$ \\
$\,\,$\color{red} 0.4623\color{black} $\,\,$ & $\,\,$ 1 $\,\,$ & $\,\,$\color{red} 1.8849\color{black} $\,\,$ & $\,\,$\color{red} 5.5432\color{black}   $\,\,$ \\
$\,\,$0.2453$\,\,$ & $\,\,$\color{red} 0.5305\color{black} $\,\,$ & $\,\,$ 1 $\,\,$ & $\,\,$2.9409 $\,\,$ \\
$\,\,$0.0834$\,\,$ & $\,\,$\color{red} 0.1804\color{black} $\,\,$ & $\,\,$0.3400$\,\,$ & $\,\,$ 1  $\,\,$ \\
\end{pmatrix},
\end{equation*}

\begin{equation*}
\mathbf{w}^{\prime} =
\begin{pmatrix}
0.549705\\
0.269635\\
0.134817\\
0.045842
\end{pmatrix} =
0.984480\cdot
\begin{pmatrix}
0.558372\\
\color{gr} 0.273886\color{black} \\
0.136943\\
0.046565
\end{pmatrix},
\end{equation*}
\begin{equation*}
\left[ \frac{{w}^{\prime}_i}{{w}^{\prime}_j} \right] =
\begin{pmatrix}
$\,\,$ 1 $\,\,$ & $\,\,$\color{gr} 2.0387\color{black} $\,\,$ & $\,\,$4.0774$\,\,$ & $\,\,$11.9912$\,\,$ \\
$\,\,$\color{gr} 0.4905\color{black} $\,\,$ & $\,\,$ 1 $\,\,$ & $\,\,$\color{gr} \color{blue} 2\color{black} $\,\,$ & $\,\,$\color{gr} 5.8818\color{black}   $\,\,$ \\
$\,\,$0.2453$\,\,$ & $\,\,$\color{gr} \color{blue}  1/2\color{black} $\,\,$ & $\,\,$ 1 $\,\,$ & $\,\,$2.9409 $\,\,$ \\
$\,\,$0.0834$\,\,$ & $\,\,$\color{gr} 0.1700\color{black} $\,\,$ & $\,\,$0.3400$\,\,$ & $\,\,$ 1  $\,\,$ \\
\end{pmatrix},
\end{equation*}
\end{example}
\newpage
\begin{example}
\begin{equation*}
\mathbf{A} =
\begin{pmatrix}
$\,\,$ 1 $\,\,$ & $\,\,$2$\,\,$ & $\,\,$7$\,\,$ & $\,\,$7 $\,\,$ \\
$\,\,$ 1/2$\,\,$ & $\,\,$ 1 $\,\,$ & $\,\,$5$\,\,$ & $\,\,$2 $\,\,$ \\
$\,\,$ 1/7$\,\,$ & $\,\,$ 1/5$\,\,$ & $\,\,$ 1 $\,\,$ & $\,\,$2 $\,\,$ \\
$\,\,$ 1/7$\,\,$ & $\,\,$ 1/2$\,\,$ & $\,\,$ 1/2$\,\,$ & $\,\,$ 1  $\,\,$ \\
\end{pmatrix},
\qquad
\lambda_{\max} =
4.2287,
\qquad
CR = 0.0862
\end{equation*}

\begin{equation*}
\mathbf{w}^{EM} =
\begin{pmatrix}
\color{red} 0.548483\color{black} \\
0.277948\\
0.092008\\
0.081561
\end{pmatrix}\end{equation*}
\begin{equation*}
\left[ \frac{{w}^{EM}_i}{{w}^{EM}_j} \right] =
\begin{pmatrix}
$\,\,$ 1 $\,\,$ & $\,\,$\color{red} 1.9733\color{black} $\,\,$ & $\,\,$\color{red} 5.9612\color{black} $\,\,$ & $\,\,$\color{red} 6.7249\color{black} $\,\,$ \\
$\,\,$\color{red} 0.5068\color{black} $\,\,$ & $\,\,$ 1 $\,\,$ & $\,\,$3.0209$\,\,$ & $\,\,$3.4079  $\,\,$ \\
$\,\,$\color{red} 0.1678\color{black} $\,\,$ & $\,\,$0.3310$\,\,$ & $\,\,$ 1 $\,\,$ & $\,\,$1.1281 $\,\,$ \\
$\,\,$\color{red} 0.1487\color{black} $\,\,$ & $\,\,$0.2934$\,\,$ & $\,\,$0.8864$\,\,$ & $\,\,$ 1  $\,\,$ \\
\end{pmatrix},
\end{equation*}

\begin{equation*}
\mathbf{w}^{\prime} =
\begin{pmatrix}
0.551805\\
0.275903\\
0.091331\\
0.080960
\end{pmatrix} =
0.992642\cdot
\begin{pmatrix}
\color{gr} 0.555896\color{black} \\
0.277948\\
0.092008\\
0.081561
\end{pmatrix},
\end{equation*}
\begin{equation*}
\left[ \frac{{w}^{\prime}_i}{{w}^{\prime}_j} \right] =
\begin{pmatrix}
$\,\,$ 1 $\,\,$ & $\,\,$\color{gr} \color{blue} 2\color{black} $\,\,$ & $\,\,$\color{gr} 6.0418\color{black} $\,\,$ & $\,\,$\color{gr} 6.8157\color{black} $\,\,$ \\
$\,\,$\color{gr} \color{blue}  1/2\color{black} $\,\,$ & $\,\,$ 1 $\,\,$ & $\,\,$3.0209$\,\,$ & $\,\,$3.4079  $\,\,$ \\
$\,\,$\color{gr} 0.1655\color{black} $\,\,$ & $\,\,$0.3310$\,\,$ & $\,\,$ 1 $\,\,$ & $\,\,$1.1281 $\,\,$ \\
$\,\,$\color{gr} 0.1467\color{black} $\,\,$ & $\,\,$0.2934$\,\,$ & $\,\,$0.8864$\,\,$ & $\,\,$ 1  $\,\,$ \\
\end{pmatrix},
\end{equation*}
\end{example}
\newpage
\begin{example}
\begin{equation*}
\mathbf{A} =
\begin{pmatrix}
$\,\,$ 1 $\,\,$ & $\,\,$2$\,\,$ & $\,\,$7$\,\,$ & $\,\,$8 $\,\,$ \\
$\,\,$ 1/2$\,\,$ & $\,\,$ 1 $\,\,$ & $\,\,$1$\,\,$ & $\,\,$3 $\,\,$ \\
$\,\,$ 1/7$\,\,$ & $\,\,$ 1 $\,\,$ & $\,\,$ 1 $\,\,$ & $\,\,$2 $\,\,$ \\
$\,\,$ 1/8$\,\,$ & $\,\,$ 1/3$\,\,$ & $\,\,$ 1/2$\,\,$ & $\,\,$ 1  $\,\,$ \\
\end{pmatrix},
\qquad
\lambda_{\max} =
4.1365,
\qquad
CR = 0.0515
\end{equation*}

\begin{equation*}
\mathbf{w}^{EM} =
\begin{pmatrix}
0.597644\\
0.201721\\
0.134015\\
\color{red} 0.066620\color{black}
\end{pmatrix}\end{equation*}
\begin{equation*}
\left[ \frac{{w}^{EM}_i}{{w}^{EM}_j} \right] =
\begin{pmatrix}
$\,\,$ 1 $\,\,$ & $\,\,$2.9627$\,\,$ & $\,\,$4.4595$\,\,$ & $\,\,$\color{red} 8.9709\color{black} $\,\,$ \\
$\,\,$0.3375$\,\,$ & $\,\,$ 1 $\,\,$ & $\,\,$1.5052$\,\,$ & $\,\,$\color{red} 3.0279\color{black}   $\,\,$ \\
$\,\,$0.2242$\,\,$ & $\,\,$0.6644$\,\,$ & $\,\,$ 1 $\,\,$ & $\,\,$\color{red} 2.0116\color{black}  $\,\,$ \\
$\,\,$\color{red} 0.1115\color{black} $\,\,$ & $\,\,$\color{red} 0.3303\color{black} $\,\,$ & $\,\,$\color{red} 0.4971\color{black} $\,\,$ & $\,\,$ 1  $\,\,$ \\
\end{pmatrix},
\end{equation*}

\begin{equation*}
\mathbf{w}^{\prime} =
\begin{pmatrix}
0.597412\\
0.201643\\
0.133963\\
0.066982
\end{pmatrix} =
0.999612\cdot
\begin{pmatrix}
0.597644\\
0.201721\\
0.134015\\
\color{gr} 0.067008\color{black}
\end{pmatrix},
\end{equation*}
\begin{equation*}
\left[ \frac{{w}^{\prime}_i}{{w}^{\prime}_j} \right] =
\begin{pmatrix}
$\,\,$ 1 $\,\,$ & $\,\,$2.9627$\,\,$ & $\,\,$4.4595$\,\,$ & $\,\,$\color{gr} 8.9190\color{black} $\,\,$ \\
$\,\,$0.3375$\,\,$ & $\,\,$ 1 $\,\,$ & $\,\,$1.5052$\,\,$ & $\,\,$\color{gr} 3.0104\color{black}   $\,\,$ \\
$\,\,$0.2242$\,\,$ & $\,\,$0.6644$\,\,$ & $\,\,$ 1 $\,\,$ & $\,\,$\color{gr} \color{blue} 2\color{black}  $\,\,$ \\
$\,\,$\color{gr} 0.1121\color{black} $\,\,$ & $\,\,$\color{gr} 0.3322\color{black} $\,\,$ & $\,\,$\color{gr} \color{blue}  1/2\color{black} $\,\,$ & $\,\,$ 1  $\,\,$ \\
\end{pmatrix},
\end{equation*}
\end{example}
\newpage
\begin{example}
\begin{equation*}
\mathbf{A} =
\begin{pmatrix}
$\,\,$ 1 $\,\,$ & $\,\,$2$\,\,$ & $\,\,$7$\,\,$ & $\,\,$8 $\,\,$ \\
$\,\,$ 1/2$\,\,$ & $\,\,$ 1 $\,\,$ & $\,\,$2$\,\,$ & $\,\,$6 $\,\,$ \\
$\,\,$ 1/7$\,\,$ & $\,\,$ 1/2$\,\,$ & $\,\,$ 1 $\,\,$ & $\,\,$5 $\,\,$ \\
$\,\,$ 1/8$\,\,$ & $\,\,$ 1/6$\,\,$ & $\,\,$ 1/5$\,\,$ & $\,\,$ 1  $\,\,$ \\
\end{pmatrix},
\qquad
\lambda_{\max} =
4.1888,
\qquad
CR = 0.0712
\end{equation*}

\begin{equation*}
\mathbf{w}^{EM} =
\begin{pmatrix}
0.565827\\
\color{red} 0.255778\color{black} \\
0.134415\\
0.043980
\end{pmatrix}\end{equation*}
\begin{equation*}
\left[ \frac{{w}^{EM}_i}{{w}^{EM}_j} \right] =
\begin{pmatrix}
$\,\,$ 1 $\,\,$ & $\,\,$\color{red} 2.2122\color{black} $\,\,$ & $\,\,$4.2096$\,\,$ & $\,\,$12.8657$\,\,$ \\
$\,\,$\color{red} 0.4520\color{black} $\,\,$ & $\,\,$ 1 $\,\,$ & $\,\,$\color{red} 1.9029\color{black} $\,\,$ & $\,\,$\color{red} 5.8158\color{black}   $\,\,$ \\
$\,\,$0.2376$\,\,$ & $\,\,$\color{red} 0.5255\color{black} $\,\,$ & $\,\,$ 1 $\,\,$ & $\,\,$3.0563 $\,\,$ \\
$\,\,$0.0777$\,\,$ & $\,\,$\color{red} 0.1719\color{black} $\,\,$ & $\,\,$0.3272$\,\,$ & $\,\,$ 1  $\,\,$ \\
\end{pmatrix},
\end{equation*}

\begin{equation*}
\mathbf{w}^{\prime} =
\begin{pmatrix}
0.561281\\
0.261757\\
0.133335\\
0.043626
\end{pmatrix} =
0.991966\cdot
\begin{pmatrix}
0.565827\\
\color{gr} 0.263877\color{black} \\
0.134415\\
0.043980
\end{pmatrix},
\end{equation*}
\begin{equation*}
\left[ \frac{{w}^{\prime}_i}{{w}^{\prime}_j} \right] =
\begin{pmatrix}
$\,\,$ 1 $\,\,$ & $\,\,$\color{gr} 2.1443\color{black} $\,\,$ & $\,\,$4.2096$\,\,$ & $\,\,$12.8657$\,\,$ \\
$\,\,$\color{gr} 0.4664\color{black} $\,\,$ & $\,\,$ 1 $\,\,$ & $\,\,$\color{gr} 1.9632\color{black} $\,\,$ & $\,\,$\color{gr} \color{blue} 6\color{black}   $\,\,$ \\
$\,\,$0.2376$\,\,$ & $\,\,$\color{gr} 0.5094\color{black} $\,\,$ & $\,\,$ 1 $\,\,$ & $\,\,$3.0563 $\,\,$ \\
$\,\,$0.0777$\,\,$ & $\,\,$\color{gr} \color{blue}  1/6\color{black} $\,\,$ & $\,\,$0.3272$\,\,$ & $\,\,$ 1  $\,\,$ \\
\end{pmatrix},
\end{equation*}
\end{example}
\newpage
\begin{example}
\begin{equation*}
\mathbf{A} =
\begin{pmatrix}
$\,\,$ 1 $\,\,$ & $\,\,$2$\,\,$ & $\,\,$7$\,\,$ & $\,\,$9 $\,\,$ \\
$\,\,$ 1/2$\,\,$ & $\,\,$ 1 $\,\,$ & $\,\,$1$\,\,$ & $\,\,$3 $\,\,$ \\
$\,\,$ 1/7$\,\,$ & $\,\,$ 1 $\,\,$ & $\,\,$ 1 $\,\,$ & $\,\,$2 $\,\,$ \\
$\,\,$ 1/9$\,\,$ & $\,\,$ 1/3$\,\,$ & $\,\,$ 1/2$\,\,$ & $\,\,$ 1  $\,\,$ \\
\end{pmatrix},
\qquad
\lambda_{\max} =
4.1342,
\qquad
CR = 0.0506
\end{equation*}

\begin{equation*}
\mathbf{w}^{EM} =
\begin{pmatrix}
0.604849\\
0.199553\\
0.131892\\
\color{red} 0.063707\color{black}
\end{pmatrix}\end{equation*}
\begin{equation*}
\left[ \frac{{w}^{EM}_i}{{w}^{EM}_j} \right] =
\begin{pmatrix}
$\,\,$ 1 $\,\,$ & $\,\,$3.0310$\,\,$ & $\,\,$4.5859$\,\,$ & $\,\,$\color{red} 9.4943\color{black} $\,\,$ \\
$\,\,$0.3299$\,\,$ & $\,\,$ 1 $\,\,$ & $\,\,$1.5130$\,\,$ & $\,\,$\color{red} 3.1324\color{black}   $\,\,$ \\
$\,\,$0.2181$\,\,$ & $\,\,$0.6609$\,\,$ & $\,\,$ 1 $\,\,$ & $\,\,$\color{red} 2.0703\color{black}  $\,\,$ \\
$\,\,$\color{red} 0.1053\color{black} $\,\,$ & $\,\,$\color{red} 0.3192\color{black} $\,\,$ & $\,\,$\color{red} 0.4830\color{black} $\,\,$ & $\,\,$ 1  $\,\,$ \\
\end{pmatrix},
\end{equation*}

\begin{equation*}
\mathbf{w}^{\prime} =
\begin{pmatrix}
0.603497\\
0.199107\\
0.131597\\
0.065799
\end{pmatrix} =
0.997766\cdot
\begin{pmatrix}
0.604849\\
0.199553\\
0.131892\\
\color{gr} 0.065946\color{black}
\end{pmatrix},
\end{equation*}
\begin{equation*}
\left[ \frac{{w}^{\prime}_i}{{w}^{\prime}_j} \right] =
\begin{pmatrix}
$\,\,$ 1 $\,\,$ & $\,\,$3.0310$\,\,$ & $\,\,$4.5859$\,\,$ & $\,\,$\color{gr} 9.1719\color{black} $\,\,$ \\
$\,\,$0.3299$\,\,$ & $\,\,$ 1 $\,\,$ & $\,\,$1.5130$\,\,$ & $\,\,$\color{gr} 3.0260\color{black}   $\,\,$ \\
$\,\,$0.2181$\,\,$ & $\,\,$0.6609$\,\,$ & $\,\,$ 1 $\,\,$ & $\,\,$\color{gr} \color{blue} 2\color{black}  $\,\,$ \\
$\,\,$\color{gr} 0.1090\color{black} $\,\,$ & $\,\,$\color{gr} 0.3305\color{black} $\,\,$ & $\,\,$\color{gr} \color{blue}  1/2\color{black} $\,\,$ & $\,\,$ 1  $\,\,$ \\
\end{pmatrix},
\end{equation*}
\end{example}
\newpage
\begin{example}
\begin{equation*}
\mathbf{A} =
\begin{pmatrix}
$\,\,$ 1 $\,\,$ & $\,\,$2$\,\,$ & $\,\,$7$\,\,$ & $\,\,$9 $\,\,$ \\
$\,\,$ 1/2$\,\,$ & $\,\,$ 1 $\,\,$ & $\,\,$8$\,\,$ & $\,\,$6 $\,\,$ \\
$\,\,$ 1/7$\,\,$ & $\,\,$ 1/8$\,\,$ & $\,\,$ 1 $\,\,$ & $\,\,$1 $\,\,$ \\
$\,\,$ 1/9$\,\,$ & $\,\,$ 1/6$\,\,$ & $\,\,$ 1 $\,\,$ & $\,\,$ 1  $\,\,$ \\
\end{pmatrix},
\qquad
\lambda_{\max} =
4.0576,
\qquad
CR = 0.0217
\end{equation*}

\begin{equation*}
\mathbf{w}^{EM} =
\begin{pmatrix}
0.532478\\
0.351916\\
0.058079\\
\color{red} 0.057527\color{black}
\end{pmatrix}\end{equation*}
\begin{equation*}
\left[ \frac{{w}^{EM}_i}{{w}^{EM}_j} \right] =
\begin{pmatrix}
$\,\,$ 1 $\,\,$ & $\,\,$1.5131$\,\,$ & $\,\,$9.1682$\,\,$ & $\,\,$\color{red} 9.2562\color{black} $\,\,$ \\
$\,\,$0.6609$\,\,$ & $\,\,$ 1 $\,\,$ & $\,\,$6.0593$\,\,$ & $\,\,$\color{red} 6.1174\color{black}   $\,\,$ \\
$\,\,$0.1091$\,\,$ & $\,\,$0.1650$\,\,$ & $\,\,$ 1 $\,\,$ & $\,\,$\color{red} 1.0096\color{black}  $\,\,$ \\
$\,\,$\color{red} 0.1080\color{black} $\,\,$ & $\,\,$\color{red} 0.1635\color{black} $\,\,$ & $\,\,$\color{red} 0.9905\color{black} $\,\,$ & $\,\,$ 1  $\,\,$ \\
\end{pmatrix},
\end{equation*}

\begin{equation*}
\mathbf{w}^{\prime} =
\begin{pmatrix}
0.532184\\
0.351722\\
0.058047\\
0.058047
\end{pmatrix} =
0.999448\cdot
\begin{pmatrix}
0.532478\\
0.351916\\
0.058079\\
\color{gr} 0.058079\color{black}
\end{pmatrix},
\end{equation*}
\begin{equation*}
\left[ \frac{{w}^{\prime}_i}{{w}^{\prime}_j} \right] =
\begin{pmatrix}
$\,\,$ 1 $\,\,$ & $\,\,$1.5131$\,\,$ & $\,\,$9.1682$\,\,$ & $\,\,$\color{gr} 9.1682\color{black} $\,\,$ \\
$\,\,$0.6609$\,\,$ & $\,\,$ 1 $\,\,$ & $\,\,$6.0593$\,\,$ & $\,\,$\color{gr} 6.0593\color{black}   $\,\,$ \\
$\,\,$0.1091$\,\,$ & $\,\,$0.1650$\,\,$ & $\,\,$ 1 $\,\,$ & $\,\,$\color{gr} \color{blue} 1\color{black}  $\,\,$ \\
$\,\,$\color{gr} 0.1091\color{black} $\,\,$ & $\,\,$\color{gr} 0.1650\color{black} $\,\,$ & $\,\,$\color{gr} \color{blue} 1\color{black} $\,\,$ & $\,\,$ 1  $\,\,$ \\
\end{pmatrix},
\end{equation*}
\end{example}
\newpage
\begin{example}
\begin{equation*}
\mathbf{A} =
\begin{pmatrix}
$\,\,$ 1 $\,\,$ & $\,\,$2$\,\,$ & $\,\,$8$\,\,$ & $\,\,$5 $\,\,$ \\
$\,\,$ 1/2$\,\,$ & $\,\,$ 1 $\,\,$ & $\,\,$6$\,\,$ & $\,\,$8 $\,\,$ \\
$\,\,$ 1/8$\,\,$ & $\,\,$ 1/6$\,\,$ & $\,\,$ 1 $\,\,$ & $\,\,$1 $\,\,$ \\
$\,\,$ 1/5$\,\,$ & $\,\,$ 1/8$\,\,$ & $\,\,$ 1 $\,\,$ & $\,\,$ 1  $\,\,$ \\
\end{pmatrix},
\qquad
\lambda_{\max} =
4.1163,
\qquad
CR = 0.0439
\end{equation*}

\begin{equation*}
\mathbf{w}^{EM} =
\begin{pmatrix}
0.501969\\
0.369957\\
\color{red} 0.061336\color{black} \\
0.066737
\end{pmatrix}\end{equation*}
\begin{equation*}
\left[ \frac{{w}^{EM}_i}{{w}^{EM}_j} \right] =
\begin{pmatrix}
$\,\,$ 1 $\,\,$ & $\,\,$1.3568$\,\,$ & $\,\,$\color{red} 8.1839\color{black} $\,\,$ & $\,\,$7.5215$\,\,$ \\
$\,\,$0.7370$\,\,$ & $\,\,$ 1 $\,\,$ & $\,\,$\color{red} 6.0316\color{black} $\,\,$ & $\,\,$5.5435  $\,\,$ \\
$\,\,$\color{red} 0.1222\color{black} $\,\,$ & $\,\,$\color{red} 0.1658\color{black} $\,\,$ & $\,\,$ 1 $\,\,$ & $\,\,$\color{red} 0.9191\color{black}  $\,\,$ \\
$\,\,$0.1330$\,\,$ & $\,\,$0.1804$\,\,$ & $\,\,$\color{red} 1.0881\color{black} $\,\,$ & $\,\,$ 1  $\,\,$ \\
\end{pmatrix},
\end{equation*}

\begin{equation*}
\mathbf{w}^{\prime} =
\begin{pmatrix}
0.501807\\
0.369838\\
0.061640\\
0.066716
\end{pmatrix} =
0.999677\cdot
\begin{pmatrix}
0.501969\\
0.369957\\
\color{gr} 0.061660\color{black} \\
0.066737
\end{pmatrix},
\end{equation*}
\begin{equation*}
\left[ \frac{{w}^{\prime}_i}{{w}^{\prime}_j} \right] =
\begin{pmatrix}
$\,\,$ 1 $\,\,$ & $\,\,$1.3568$\,\,$ & $\,\,$\color{gr} 8.1410\color{black} $\,\,$ & $\,\,$7.5215$\,\,$ \\
$\,\,$0.7370$\,\,$ & $\,\,$ 1 $\,\,$ & $\,\,$\color{gr} \color{blue} 6\color{black} $\,\,$ & $\,\,$5.5435  $\,\,$ \\
$\,\,$\color{gr} 0.1228\color{black} $\,\,$ & $\,\,$\color{gr} \color{blue}  1/6\color{black} $\,\,$ & $\,\,$ 1 $\,\,$ & $\,\,$\color{gr} 0.9239\color{black}  $\,\,$ \\
$\,\,$0.1330$\,\,$ & $\,\,$0.1804$\,\,$ & $\,\,$\color{gr} 1.0824\color{black} $\,\,$ & $\,\,$ 1  $\,\,$ \\
\end{pmatrix},
\end{equation*}
\end{example}
\newpage
\begin{example}
\begin{equation*}
\mathbf{A} =
\begin{pmatrix}
$\,\,$ 1 $\,\,$ & $\,\,$2$\,\,$ & $\,\,$8$\,\,$ & $\,\,$5 $\,\,$ \\
$\,\,$ 1/2$\,\,$ & $\,\,$ 1 $\,\,$ & $\,\,$6$\,\,$ & $\,\,$9 $\,\,$ \\
$\,\,$ 1/8$\,\,$ & $\,\,$ 1/6$\,\,$ & $\,\,$ 1 $\,\,$ & $\,\,$1 $\,\,$ \\
$\,\,$ 1/5$\,\,$ & $\,\,$ 1/9$\,\,$ & $\,\,$ 1 $\,\,$ & $\,\,$ 1  $\,\,$ \\
\end{pmatrix},
\qquad
\lambda_{\max} =
4.1406,
\qquad
CR = 0.0530
\end{equation*}

\begin{equation*}
\mathbf{w}^{EM} =
\begin{pmatrix}
0.496962\\
0.378472\\
\color{red} 0.060322\color{black} \\
0.064245
\end{pmatrix}\end{equation*}
\begin{equation*}
\left[ \frac{{w}^{EM}_i}{{w}^{EM}_j} \right] =
\begin{pmatrix}
$\,\,$ 1 $\,\,$ & $\,\,$1.3131$\,\,$ & $\,\,$\color{red} 8.2385\color{black} $\,\,$ & $\,\,$7.7354$\,\,$ \\
$\,\,$0.7616$\,\,$ & $\,\,$ 1 $\,\,$ & $\,\,$\color{red} 6.2742\color{black} $\,\,$ & $\,\,$5.8911  $\,\,$ \\
$\,\,$\color{red} 0.1214\color{black} $\,\,$ & $\,\,$\color{red} 0.1594\color{black} $\,\,$ & $\,\,$ 1 $\,\,$ & $\,\,$\color{red} 0.9389\color{black}  $\,\,$ \\
$\,\,$0.1293$\,\,$ & $\,\,$0.1697$\,\,$ & $\,\,$\color{red} 1.0650\color{black} $\,\,$ & $\,\,$ 1  $\,\,$ \\
\end{pmatrix},
\end{equation*}

\begin{equation*}
\mathbf{w}^{\prime} =
\begin{pmatrix}
0.496069\\
0.377792\\
0.062009\\
0.064130
\end{pmatrix} =
0.998205\cdot
\begin{pmatrix}
0.496962\\
0.378472\\
\color{gr} 0.062120\color{black} \\
0.064245
\end{pmatrix},
\end{equation*}
\begin{equation*}
\left[ \frac{{w}^{\prime}_i}{{w}^{\prime}_j} \right] =
\begin{pmatrix}
$\,\,$ 1 $\,\,$ & $\,\,$1.3131$\,\,$ & $\,\,$\color{gr} \color{blue} 8\color{black} $\,\,$ & $\,\,$7.7354$\,\,$ \\
$\,\,$0.7616$\,\,$ & $\,\,$ 1 $\,\,$ & $\,\,$\color{gr} 6.0926\color{black} $\,\,$ & $\,\,$5.8911  $\,\,$ \\
$\,\,$\color{gr} \color{blue}  1/8\color{black} $\,\,$ & $\,\,$\color{gr} 0.1641\color{black} $\,\,$ & $\,\,$ 1 $\,\,$ & $\,\,$\color{gr} 0.9669\color{black}  $\,\,$ \\
$\,\,$0.1293$\,\,$ & $\,\,$0.1697$\,\,$ & $\,\,$\color{gr} 1.0342\color{black} $\,\,$ & $\,\,$ 1  $\,\,$ \\
\end{pmatrix},
\end{equation*}
\end{example}
\newpage
\begin{example}
\begin{equation*}
\mathbf{A} =
\begin{pmatrix}
$\,\,$ 1 $\,\,$ & $\,\,$2$\,\,$ & $\,\,$8$\,\,$ & $\,\,$6 $\,\,$ \\
$\,\,$ 1/2$\,\,$ & $\,\,$ 1 $\,\,$ & $\,\,$2$\,\,$ & $\,\,$5 $\,\,$ \\
$\,\,$ 1/8$\,\,$ & $\,\,$ 1/2$\,\,$ & $\,\,$ 1 $\,\,$ & $\,\,$4 $\,\,$ \\
$\,\,$ 1/6$\,\,$ & $\,\,$ 1/5$\,\,$ & $\,\,$ 1/4$\,\,$ & $\,\,$ 1  $\,\,$ \\
\end{pmatrix},
\qquad
\lambda_{\max} =
4.2460,
\qquad
CR = 0.0928
\end{equation*}

\begin{equation*}
\mathbf{w}^{EM} =
\begin{pmatrix}
0.568406\\
\color{red} 0.249817\color{black} \\
0.127389\\
0.054388
\end{pmatrix}\end{equation*}
\begin{equation*}
\left[ \frac{{w}^{EM}_i}{{w}^{EM}_j} \right] =
\begin{pmatrix}
$\,\,$ 1 $\,\,$ & $\,\,$\color{red} 2.2753\color{black} $\,\,$ & $\,\,$4.4620$\,\,$ & $\,\,$10.4510$\,\,$ \\
$\,\,$\color{red} 0.4395\color{black} $\,\,$ & $\,\,$ 1 $\,\,$ & $\,\,$\color{red} 1.9611\color{black} $\,\,$ & $\,\,$\color{red} 4.5933\color{black}   $\,\,$ \\
$\,\,$0.2241$\,\,$ & $\,\,$\color{red} 0.5099\color{black} $\,\,$ & $\,\,$ 1 $\,\,$ & $\,\,$2.3422 $\,\,$ \\
$\,\,$0.0957$\,\,$ & $\,\,$\color{red} 0.2177\color{black} $\,\,$ & $\,\,$0.4269$\,\,$ & $\,\,$ 1  $\,\,$ \\
\end{pmatrix},
\end{equation*}

\begin{equation*}
\mathbf{w}^{\prime} =
\begin{pmatrix}
0.565600\\
0.253520\\
0.126760\\
0.054119
\end{pmatrix} =
0.995064\cdot
\begin{pmatrix}
0.568406\\
\color{gr} 0.254778\color{black} \\
0.127389\\
0.054388
\end{pmatrix},
\end{equation*}
\begin{equation*}
\left[ \frac{{w}^{\prime}_i}{{w}^{\prime}_j} \right] =
\begin{pmatrix}
$\,\,$ 1 $\,\,$ & $\,\,$\color{gr} 2.2310\color{black} $\,\,$ & $\,\,$4.4620$\,\,$ & $\,\,$10.4510$\,\,$ \\
$\,\,$\color{gr} 0.4482\color{black} $\,\,$ & $\,\,$ 1 $\,\,$ & $\,\,$\color{gr} \color{blue} 2\color{black} $\,\,$ & $\,\,$\color{gr} 4.6845\color{black}   $\,\,$ \\
$\,\,$0.2241$\,\,$ & $\,\,$\color{gr} \color{blue}  1/2\color{black} $\,\,$ & $\,\,$ 1 $\,\,$ & $\,\,$2.3422 $\,\,$ \\
$\,\,$0.0957$\,\,$ & $\,\,$\color{gr} 0.2135\color{black} $\,\,$ & $\,\,$0.4269$\,\,$ & $\,\,$ 1  $\,\,$ \\
\end{pmatrix},
\end{equation*}
\end{example}
\newpage
\begin{example}
\begin{equation*}
\mathbf{A} =
\begin{pmatrix}
$\,\,$ 1 $\,\,$ & $\,\,$2$\,\,$ & $\,\,$8$\,\,$ & $\,\,$6 $\,\,$ \\
$\,\,$ 1/2$\,\,$ & $\,\,$ 1 $\,\,$ & $\,\,$6$\,\,$ & $\,\,$2 $\,\,$ \\
$\,\,$ 1/8$\,\,$ & $\,\,$ 1/6$\,\,$ & $\,\,$ 1 $\,\,$ & $\,\,$1 $\,\,$ \\
$\,\,$ 1/6$\,\,$ & $\,\,$ 1/2$\,\,$ & $\,\,$ 1 $\,\,$ & $\,\,$ 1  $\,\,$ \\
\end{pmatrix},
\qquad
\lambda_{\max} =
4.1031,
\qquad
CR = 0.0389
\end{equation*}

\begin{equation*}
\mathbf{w}^{EM} =
\begin{pmatrix}
\color{red} 0.549260\color{black} \\
0.284413\\
0.068822\\
0.097505
\end{pmatrix}\end{equation*}
\begin{equation*}
\left[ \frac{{w}^{EM}_i}{{w}^{EM}_j} \right] =
\begin{pmatrix}
$\,\,$ 1 $\,\,$ & $\,\,$\color{red} 1.9312\color{black} $\,\,$ & $\,\,$\color{red} 7.9809\color{black} $\,\,$ & $\,\,$\color{red} 5.6331\color{black} $\,\,$ \\
$\,\,$\color{red} 0.5178\color{black} $\,\,$ & $\,\,$ 1 $\,\,$ & $\,\,$4.1326$\,\,$ & $\,\,$2.9169  $\,\,$ \\
$\,\,$\color{red} 0.1253\color{black} $\,\,$ & $\,\,$0.2420$\,\,$ & $\,\,$ 1 $\,\,$ & $\,\,$0.7058 $\,\,$ \\
$\,\,$\color{red} 0.1775\color{black} $\,\,$ & $\,\,$0.3428$\,\,$ & $\,\,$1.4168$\,\,$ & $\,\,$ 1  $\,\,$ \\
\end{pmatrix},
\end{equation*}

\begin{equation*}
\mathbf{w}^{\prime} =
\begin{pmatrix}
0.549853\\
0.284039\\
0.068732\\
0.097377
\end{pmatrix} =
0.998685\cdot
\begin{pmatrix}
\color{gr} 0.550577\color{black} \\
0.284413\\
0.068822\\
0.097505
\end{pmatrix},
\end{equation*}
\begin{equation*}
\left[ \frac{{w}^{\prime}_i}{{w}^{\prime}_j} \right] =
\begin{pmatrix}
$\,\,$ 1 $\,\,$ & $\,\,$\color{gr} 1.9358\color{black} $\,\,$ & $\,\,$\color{gr} \color{blue} 8\color{black} $\,\,$ & $\,\,$\color{gr} 5.6467\color{black} $\,\,$ \\
$\,\,$\color{gr} 0.5166\color{black} $\,\,$ & $\,\,$ 1 $\,\,$ & $\,\,$4.1326$\,\,$ & $\,\,$2.9169  $\,\,$ \\
$\,\,$\color{gr} \color{blue}  1/8\color{black} $\,\,$ & $\,\,$0.2420$\,\,$ & $\,\,$ 1 $\,\,$ & $\,\,$0.7058 $\,\,$ \\
$\,\,$\color{gr} 0.1771\color{black} $\,\,$ & $\,\,$0.3428$\,\,$ & $\,\,$1.4168$\,\,$ & $\,\,$ 1  $\,\,$ \\
\end{pmatrix},
\end{equation*}
\end{example}
\newpage
\begin{example}
\begin{equation*}
\mathbf{A} =
\begin{pmatrix}
$\,\,$ 1 $\,\,$ & $\,\,$2$\,\,$ & $\,\,$8$\,\,$ & $\,\,$7 $\,\,$ \\
$\,\,$ 1/2$\,\,$ & $\,\,$ 1 $\,\,$ & $\,\,$2$\,\,$ & $\,\,$5 $\,\,$ \\
$\,\,$ 1/8$\,\,$ & $\,\,$ 1/2$\,\,$ & $\,\,$ 1 $\,\,$ & $\,\,$4 $\,\,$ \\
$\,\,$ 1/7$\,\,$ & $\,\,$ 1/5$\,\,$ & $\,\,$ 1/4$\,\,$ & $\,\,$ 1  $\,\,$ \\
\end{pmatrix},
\qquad
\lambda_{\max} =
4.2035,
\qquad
CR = 0.0767
\end{equation*}

\begin{equation*}
\mathbf{w}^{EM} =
\begin{pmatrix}
0.577044\\
\color{red} 0.247362\color{black} \\
0.124687\\
0.050906
\end{pmatrix}\end{equation*}
\begin{equation*}
\left[ \frac{{w}^{EM}_i}{{w}^{EM}_j} \right] =
\begin{pmatrix}
$\,\,$ 1 $\,\,$ & $\,\,$\color{red} 2.3328\color{black} $\,\,$ & $\,\,$4.6279$\,\,$ & $\,\,$11.3354$\,\,$ \\
$\,\,$\color{red} 0.4287\color{black} $\,\,$ & $\,\,$ 1 $\,\,$ & $\,\,$\color{red} 1.9839\color{black} $\,\,$ & $\,\,$\color{red} 4.8592\color{black}   $\,\,$ \\
$\,\,$0.2161$\,\,$ & $\,\,$\color{red} 0.5041\color{black} $\,\,$ & $\,\,$ 1 $\,\,$ & $\,\,$2.4493 $\,\,$ \\
$\,\,$0.0882$\,\,$ & $\,\,$\color{red} 0.2058\color{black} $\,\,$ & $\,\,$0.4083$\,\,$ & $\,\,$ 1  $\,\,$ \\
\end{pmatrix},
\end{equation*}

\begin{equation*}
\mathbf{w}^{\prime} =
\begin{pmatrix}
0.575885\\
0.248874\\
0.124437\\
0.050804
\end{pmatrix} =
0.997992\cdot
\begin{pmatrix}
0.577044\\
\color{gr} 0.249375\color{black} \\
0.124687\\
0.050906
\end{pmatrix},
\end{equation*}
\begin{equation*}
\left[ \frac{{w}^{\prime}_i}{{w}^{\prime}_j} \right] =
\begin{pmatrix}
$\,\,$ 1 $\,\,$ & $\,\,$\color{gr} 2.3140\color{black} $\,\,$ & $\,\,$4.6279$\,\,$ & $\,\,$11.3354$\,\,$ \\
$\,\,$\color{gr} 0.4322\color{black} $\,\,$ & $\,\,$ 1 $\,\,$ & $\,\,$\color{gr} \color{blue} 2\color{black} $\,\,$ & $\,\,$\color{gr} 4.8987\color{black}   $\,\,$ \\
$\,\,$0.2161$\,\,$ & $\,\,$\color{gr} \color{blue}  1/2\color{black} $\,\,$ & $\,\,$ 1 $\,\,$ & $\,\,$2.4493 $\,\,$ \\
$\,\,$0.0882$\,\,$ & $\,\,$\color{gr} 0.2041\color{black} $\,\,$ & $\,\,$0.4083$\,\,$ & $\,\,$ 1  $\,\,$ \\
\end{pmatrix},
\end{equation*}
\end{example}
\newpage
\begin{example}
\begin{equation*}
\mathbf{A} =
\begin{pmatrix}
$\,\,$ 1 $\,\,$ & $\,\,$2$\,\,$ & $\,\,$8$\,\,$ & $\,\,$8 $\,\,$ \\
$\,\,$ 1/2$\,\,$ & $\,\,$ 1 $\,\,$ & $\,\,$2$\,\,$ & $\,\,$6 $\,\,$ \\
$\,\,$ 1/8$\,\,$ & $\,\,$ 1/2$\,\,$ & $\,\,$ 1 $\,\,$ & $\,\,$5 $\,\,$ \\
$\,\,$ 1/8$\,\,$ & $\,\,$ 1/6$\,\,$ & $\,\,$ 1/5$\,\,$ & $\,\,$ 1  $\,\,$ \\
\end{pmatrix},
\qquad
\lambda_{\max} =
4.2277,
\qquad
CR = 0.0859
\end{equation*}

\begin{equation*}
\mathbf{w}^{EM} =
\begin{pmatrix}
0.579205\\
\color{red} 0.249475\color{black} \\
0.128072\\
0.043248
\end{pmatrix}\end{equation*}
\begin{equation*}
\left[ \frac{{w}^{EM}_i}{{w}^{EM}_j} \right] =
\begin{pmatrix}
$\,\,$ 1 $\,\,$ & $\,\,$\color{red} 2.3217\color{black} $\,\,$ & $\,\,$4.5225$\,\,$ & $\,\,$13.3925$\,\,$ \\
$\,\,$\color{red} 0.4307\color{black} $\,\,$ & $\,\,$ 1 $\,\,$ & $\,\,$\color{red} 1.9479\color{black} $\,\,$ & $\,\,$\color{red} 5.7684\color{black}   $\,\,$ \\
$\,\,$0.2211$\,\,$ & $\,\,$\color{red} 0.5134\color{black} $\,\,$ & $\,\,$ 1 $\,\,$ & $\,\,$2.9613 $\,\,$ \\
$\,\,$0.0747$\,\,$ & $\,\,$\color{red} 0.1734\color{black} $\,\,$ & $\,\,$0.3377$\,\,$ & $\,\,$ 1  $\,\,$ \\
\end{pmatrix},
\end{equation*}

\begin{equation*}
\mathbf{w}^{\prime} =
\begin{pmatrix}
0.575368\\
0.254447\\
0.127223\\
0.042962
\end{pmatrix} =
0.993376\cdot
\begin{pmatrix}
0.579205\\
\color{gr} 0.256144\color{black} \\
0.128072\\
0.043248
\end{pmatrix},
\end{equation*}
\begin{equation*}
\left[ \frac{{w}^{\prime}_i}{{w}^{\prime}_j} \right] =
\begin{pmatrix}
$\,\,$ 1 $\,\,$ & $\,\,$\color{gr} 2.2612\color{black} $\,\,$ & $\,\,$4.5225$\,\,$ & $\,\,$13.3925$\,\,$ \\
$\,\,$\color{gr} 0.4422\color{black} $\,\,$ & $\,\,$ 1 $\,\,$ & $\,\,$\color{gr} \color{blue} 2\color{black} $\,\,$ & $\,\,$\color{gr} 5.9226\color{black}   $\,\,$ \\
$\,\,$0.2211$\,\,$ & $\,\,$\color{gr} \color{blue}  1/2\color{black} $\,\,$ & $\,\,$ 1 $\,\,$ & $\,\,$2.9613 $\,\,$ \\
$\,\,$0.0747$\,\,$ & $\,\,$\color{gr} 0.1688\color{black} $\,\,$ & $\,\,$0.3377$\,\,$ & $\,\,$ 1  $\,\,$ \\
\end{pmatrix},
\end{equation*}
\end{example}
\newpage
\begin{example}
\begin{equation*}
\mathbf{A} =
\begin{pmatrix}
$\,\,$ 1 $\,\,$ & $\,\,$2$\,\,$ & $\,\,$8$\,\,$ & $\,\,$9 $\,\,$ \\
$\,\,$ 1/2$\,\,$ & $\,\,$ 1 $\,\,$ & $\,\,$1$\,\,$ & $\,\,$3 $\,\,$ \\
$\,\,$ 1/8$\,\,$ & $\,\,$ 1 $\,\,$ & $\,\,$ 1 $\,\,$ & $\,\,$2 $\,\,$ \\
$\,\,$ 1/9$\,\,$ & $\,\,$ 1/3$\,\,$ & $\,\,$ 1/2$\,\,$ & $\,\,$ 1  $\,\,$ \\
\end{pmatrix},
\qquad
\lambda_{\max} =
4.1664,
\qquad
CR = 0.0627
\end{equation*}

\begin{equation*}
\mathbf{w}^{EM} =
\begin{pmatrix}
0.616793\\
0.195782\\
0.125374\\
\color{red} 0.062051\color{black}
\end{pmatrix}\end{equation*}
\begin{equation*}
\left[ \frac{{w}^{EM}_i}{{w}^{EM}_j} \right] =
\begin{pmatrix}
$\,\,$ 1 $\,\,$ & $\,\,$3.1504$\,\,$ & $\,\,$4.9196$\,\,$ & $\,\,$\color{red} 9.9400\color{black} $\,\,$ \\
$\,\,$0.3174$\,\,$ & $\,\,$ 1 $\,\,$ & $\,\,$1.5616$\,\,$ & $\,\,$\color{red} 3.1552\color{black}   $\,\,$ \\
$\,\,$0.2033$\,\,$ & $\,\,$0.6404$\,\,$ & $\,\,$ 1 $\,\,$ & $\,\,$\color{red} 2.0205\color{black}  $\,\,$ \\
$\,\,$\color{red} 0.1006\color{black} $\,\,$ & $\,\,$\color{red} 0.3169\color{black} $\,\,$ & $\,\,$\color{red} 0.4949\color{black} $\,\,$ & $\,\,$ 1  $\,\,$ \\
\end{pmatrix},
\end{equation*}

\begin{equation*}
\mathbf{w}^{\prime} =
\begin{pmatrix}
0.616401\\
0.195658\\
0.125294\\
0.062647
\end{pmatrix} =
0.999365\cdot
\begin{pmatrix}
0.616793\\
0.195782\\
0.125374\\
\color{gr} 0.062687\color{black}
\end{pmatrix},
\end{equation*}
\begin{equation*}
\left[ \frac{{w}^{\prime}_i}{{w}^{\prime}_j} \right] =
\begin{pmatrix}
$\,\,$ 1 $\,\,$ & $\,\,$3.1504$\,\,$ & $\,\,$4.9196$\,\,$ & $\,\,$\color{gr} 9.8393\color{black} $\,\,$ \\
$\,\,$0.3174$\,\,$ & $\,\,$ 1 $\,\,$ & $\,\,$1.5616$\,\,$ & $\,\,$\color{gr} 3.1232\color{black}   $\,\,$ \\
$\,\,$0.2033$\,\,$ & $\,\,$0.6404$\,\,$ & $\,\,$ 1 $\,\,$ & $\,\,$\color{gr} \color{blue} 2\color{black}  $\,\,$ \\
$\,\,$\color{gr} 0.1016\color{black} $\,\,$ & $\,\,$\color{gr} 0.3202\color{black} $\,\,$ & $\,\,$\color{gr} \color{blue}  1/2\color{black} $\,\,$ & $\,\,$ 1  $\,\,$ \\
\end{pmatrix},
\end{equation*}
\end{example}
\newpage
\begin{example}
\begin{equation*}
\mathbf{A} =
\begin{pmatrix}
$\,\,$ 1 $\,\,$ & $\,\,$2$\,\,$ & $\,\,$8$\,\,$ & $\,\,$9 $\,\,$ \\
$\,\,$ 1/2$\,\,$ & $\,\,$ 1 $\,\,$ & $\,\,$2$\,\,$ & $\,\,$7 $\,\,$ \\
$\,\,$ 1/8$\,\,$ & $\,\,$ 1/2$\,\,$ & $\,\,$ 1 $\,\,$ & $\,\,$6 $\,\,$ \\
$\,\,$ 1/9$\,\,$ & $\,\,$ 1/7$\,\,$ & $\,\,$ 1/6$\,\,$ & $\,\,$ 1  $\,\,$ \\
\end{pmatrix},
\qquad
\lambda_{\max} =
4.2463,
\qquad
CR = 0.0929
\end{equation*}

\begin{equation*}
\mathbf{w}^{EM} =
\begin{pmatrix}
0.580762\\
\color{red} 0.251034\color{black} \\
0.130575\\
0.037629
\end{pmatrix}\end{equation*}
\begin{equation*}
\left[ \frac{{w}^{EM}_i}{{w}^{EM}_j} \right] =
\begin{pmatrix}
$\,\,$ 1 $\,\,$ & $\,\,$\color{red} 2.3135\color{black} $\,\,$ & $\,\,$4.4477$\,\,$ & $\,\,$15.4340$\,\,$ \\
$\,\,$\color{red} 0.4322\color{black} $\,\,$ & $\,\,$ 1 $\,\,$ & $\,\,$\color{red} 1.9225\color{black} $\,\,$ & $\,\,$\color{red} 6.6714\color{black}   $\,\,$ \\
$\,\,$0.2248$\,\,$ & $\,\,$\color{red} 0.5201\color{black} $\,\,$ & $\,\,$ 1 $\,\,$ & $\,\,$3.4701 $\,\,$ \\
$\,\,$0.0648$\,\,$ & $\,\,$\color{red} 0.1499\color{black} $\,\,$ & $\,\,$0.2882$\,\,$ & $\,\,$ 1  $\,\,$ \\
\end{pmatrix},
\end{equation*}

\begin{equation*}
\mathbf{w}^{\prime} =
\begin{pmatrix}
0.574946\\
0.258534\\
0.129267\\
0.037252
\end{pmatrix} =
0.989986\cdot
\begin{pmatrix}
0.580762\\
\color{gr} 0.261150\color{black} \\
0.130575\\
0.037629
\end{pmatrix},
\end{equation*}
\begin{equation*}
\left[ \frac{{w}^{\prime}_i}{{w}^{\prime}_j} \right] =
\begin{pmatrix}
$\,\,$ 1 $\,\,$ & $\,\,$\color{gr} 2.2239\color{black} $\,\,$ & $\,\,$4.4477$\,\,$ & $\,\,$15.4340$\,\,$ \\
$\,\,$\color{gr} 0.4497\color{black} $\,\,$ & $\,\,$ 1 $\,\,$ & $\,\,$\color{gr} \color{blue} 2\color{black} $\,\,$ & $\,\,$\color{gr} 6.9402\color{black}   $\,\,$ \\
$\,\,$0.2248$\,\,$ & $\,\,$\color{gr} \color{blue}  1/2\color{black} $\,\,$ & $\,\,$ 1 $\,\,$ & $\,\,$3.4701 $\,\,$ \\
$\,\,$0.0648$\,\,$ & $\,\,$\color{gr} 0.1441\color{black} $\,\,$ & $\,\,$0.2882$\,\,$ & $\,\,$ 1  $\,\,$ \\
\end{pmatrix},
\end{equation*}
\end{example}
\newpage
\begin{example}
\begin{equation*}
\mathbf{A} =
\begin{pmatrix}
$\,\,$ 1 $\,\,$ & $\,\,$2$\,\,$ & $\,\,$9$\,\,$ & $\,\,$3 $\,\,$ \\
$\,\,$ 1/2$\,\,$ & $\,\,$ 1 $\,\,$ & $\,\,$3$\,\,$ & $\,\,$2 $\,\,$ \\
$\,\,$ 1/9$\,\,$ & $\,\,$ 1/3$\,\,$ & $\,\,$ 1 $\,\,$ & $\,\,$1 $\,\,$ \\
$\,\,$ 1/3$\,\,$ & $\,\,$ 1/2$\,\,$ & $\,\,$ 1 $\,\,$ & $\,\,$ 1  $\,\,$ \\
\end{pmatrix},
\qquad
\lambda_{\max} =
4.1031,
\qquad
CR = 0.0389
\end{equation*}

\begin{equation*}
\mathbf{w}^{EM} =
\begin{pmatrix}
0.535689\\
\color{red} 0.251468\color{black} \\
0.086808\\
0.126035
\end{pmatrix}\end{equation*}
\begin{equation*}
\left[ \frac{{w}^{EM}_i}{{w}^{EM}_j} \right] =
\begin{pmatrix}
$\,\,$ 1 $\,\,$ & $\,\,$\color{red} 2.1302\color{black} $\,\,$ & $\,\,$6.1709$\,\,$ & $\,\,$4.2503$\,\,$ \\
$\,\,$\color{red} 0.4694\color{black} $\,\,$ & $\,\,$ 1 $\,\,$ & $\,\,$\color{red} 2.8968\color{black} $\,\,$ & $\,\,$\color{red} 1.9952\color{black}   $\,\,$ \\
$\,\,$0.1621$\,\,$ & $\,\,$\color{red} 0.3452\color{black} $\,\,$ & $\,\,$ 1 $\,\,$ & $\,\,$0.6888 $\,\,$ \\
$\,\,$0.2353$\,\,$ & $\,\,$\color{red} 0.5012\color{black} $\,\,$ & $\,\,$1.4519$\,\,$ & $\,\,$ 1  $\,\,$ \\
\end{pmatrix},
\end{equation*}

\begin{equation*}
\mathbf{w}^{\prime} =
\begin{pmatrix}
0.535366\\
0.251919\\
0.086756\\
0.125959
\end{pmatrix} =
0.999398\cdot
\begin{pmatrix}
0.535689\\
\color{gr} 0.252070\color{black} \\
0.086808\\
0.126035
\end{pmatrix},
\end{equation*}
\begin{equation*}
\left[ \frac{{w}^{\prime}_i}{{w}^{\prime}_j} \right] =
\begin{pmatrix}
$\,\,$ 1 $\,\,$ & $\,\,$\color{gr} 2.1252\color{black} $\,\,$ & $\,\,$6.1709$\,\,$ & $\,\,$4.2503$\,\,$ \\
$\,\,$\color{gr} 0.4706\color{black} $\,\,$ & $\,\,$ 1 $\,\,$ & $\,\,$\color{gr} 2.9038\color{black} $\,\,$ & $\,\,$\color{gr} \color{blue} 2\color{black}   $\,\,$ \\
$\,\,$0.1621$\,\,$ & $\,\,$\color{gr} 0.3444\color{black} $\,\,$ & $\,\,$ 1 $\,\,$ & $\,\,$0.6888 $\,\,$ \\
$\,\,$0.2353$\,\,$ & $\,\,$\color{gr} \color{blue}  1/2\color{black} $\,\,$ & $\,\,$1.4519$\,\,$ & $\,\,$ 1  $\,\,$ \\
\end{pmatrix},
\end{equation*}
\end{example}
\newpage
\begin{example}
\begin{equation*}
\mathbf{A} =
\begin{pmatrix}
$\,\,$ 1 $\,\,$ & $\,\,$2$\,\,$ & $\,\,$9$\,\,$ & $\,\,$5 $\,\,$ \\
$\,\,$ 1/2$\,\,$ & $\,\,$ 1 $\,\,$ & $\,\,$3$\,\,$ & $\,\,$4 $\,\,$ \\
$\,\,$ 1/9$\,\,$ & $\,\,$ 1/3$\,\,$ & $\,\,$ 1 $\,\,$ & $\,\,$2 $\,\,$ \\
$\,\,$ 1/5$\,\,$ & $\,\,$ 1/4$\,\,$ & $\,\,$ 1/2$\,\,$ & $\,\,$ 1  $\,\,$ \\
\end{pmatrix},
\qquad
\lambda_{\max} =
4.1406,
\qquad
CR = 0.0530
\end{equation*}

\begin{equation*}
\mathbf{w}^{EM} =
\begin{pmatrix}
0.560527\\
\color{red} 0.272148\color{black} \\
0.094862\\
0.072463
\end{pmatrix}\end{equation*}
\begin{equation*}
\left[ \frac{{w}^{EM}_i}{{w}^{EM}_j} \right] =
\begin{pmatrix}
$\,\,$ 1 $\,\,$ & $\,\,$\color{red} 2.0596\color{black} $\,\,$ & $\,\,$5.9088$\,\,$ & $\,\,$7.7354$\,\,$ \\
$\,\,$\color{red} 0.4855\color{black} $\,\,$ & $\,\,$ 1 $\,\,$ & $\,\,$\color{red} 2.8689\color{black} $\,\,$ & $\,\,$\color{red} 3.7557\color{black}   $\,\,$ \\
$\,\,$0.1692$\,\,$ & $\,\,$\color{red} 0.3486\color{black} $\,\,$ & $\,\,$ 1 $\,\,$ & $\,\,$1.3091 $\,\,$ \\
$\,\,$0.1293$\,\,$ & $\,\,$\color{red} 0.2663\color{black} $\,\,$ & $\,\,$0.7639$\,\,$ & $\,\,$ 1  $\,\,$ \\
\end{pmatrix},
\end{equation*}

\begin{equation*}
\mathbf{w}^{\prime} =
\begin{pmatrix}
0.556015\\
0.278007\\
0.094099\\
0.071879
\end{pmatrix} =
0.991951\cdot
\begin{pmatrix}
0.560527\\
\color{gr} 0.280263\color{black} \\
0.094862\\
0.072463
\end{pmatrix},
\end{equation*}
\begin{equation*}
\left[ \frac{{w}^{\prime}_i}{{w}^{\prime}_j} \right] =
\begin{pmatrix}
$\,\,$ 1 $\,\,$ & $\,\,$\color{gr} \color{blue} 2\color{black} $\,\,$ & $\,\,$5.9088$\,\,$ & $\,\,$7.7354$\,\,$ \\
$\,\,$\color{gr} \color{blue}  1/2\color{black} $\,\,$ & $\,\,$ 1 $\,\,$ & $\,\,$\color{gr} 2.9544\color{black} $\,\,$ & $\,\,$\color{gr} 3.8677\color{black}   $\,\,$ \\
$\,\,$0.1692$\,\,$ & $\,\,$\color{gr} 0.3385\color{black} $\,\,$ & $\,\,$ 1 $\,\,$ & $\,\,$1.3091 $\,\,$ \\
$\,\,$0.1293$\,\,$ & $\,\,$\color{gr} 0.2586\color{black} $\,\,$ & $\,\,$0.7639$\,\,$ & $\,\,$ 1  $\,\,$ \\
\end{pmatrix},
\end{equation*}
\end{example}
\newpage
\begin{example}
\begin{equation*}
\mathbf{A} =
\begin{pmatrix}
$\,\,$ 1 $\,\,$ & $\,\,$2$\,\,$ & $\,\,$9$\,\,$ & $\,\,$5 $\,\,$ \\
$\,\,$ 1/2$\,\,$ & $\,\,$ 1 $\,\,$ & $\,\,$3$\,\,$ & $\,\,$5 $\,\,$ \\
$\,\,$ 1/9$\,\,$ & $\,\,$ 1/3$\,\,$ & $\,\,$ 1 $\,\,$ & $\,\,$3 $\,\,$ \\
$\,\,$ 1/5$\,\,$ & $\,\,$ 1/5$\,\,$ & $\,\,$ 1/3$\,\,$ & $\,\,$ 1  $\,\,$ \\
\end{pmatrix},
\qquad
\lambda_{\max} =
4.2507,
\qquad
CR = 0.0946
\end{equation*}

\begin{equation*}
\mathbf{w}^{EM} =
\begin{pmatrix}
0.555864\\
\color{red} 0.277462\color{black} \\
0.104671\\
0.062003
\end{pmatrix}\end{equation*}
\begin{equation*}
\left[ \frac{{w}^{EM}_i}{{w}^{EM}_j} \right] =
\begin{pmatrix}
$\,\,$ 1 $\,\,$ & $\,\,$\color{red} 2.0034\color{black} $\,\,$ & $\,\,$5.3106$\,\,$ & $\,\,$8.9651$\,\,$ \\
$\,\,$\color{red} 0.4992\color{black} $\,\,$ & $\,\,$ 1 $\,\,$ & $\,\,$\color{red} 2.6508\color{black} $\,\,$ & $\,\,$\color{red} 4.4750\color{black}   $\,\,$ \\
$\,\,$0.1883$\,\,$ & $\,\,$\color{red} 0.3772\color{black} $\,\,$ & $\,\,$ 1 $\,\,$ & $\,\,$1.6882 $\,\,$ \\
$\,\,$0.1115$\,\,$ & $\,\,$\color{red} 0.2235\color{black} $\,\,$ & $\,\,$0.5924$\,\,$ & $\,\,$ 1  $\,\,$ \\
\end{pmatrix},
\end{equation*}

\begin{equation*}
\mathbf{w}^{\prime} =
\begin{pmatrix}
0.555603\\
0.277802\\
0.104622\\
0.061974
\end{pmatrix} =
0.999530\cdot
\begin{pmatrix}
0.555864\\
\color{gr} 0.277932\color{black} \\
0.104671\\
0.062003
\end{pmatrix},
\end{equation*}
\begin{equation*}
\left[ \frac{{w}^{\prime}_i}{{w}^{\prime}_j} \right] =
\begin{pmatrix}
$\,\,$ 1 $\,\,$ & $\,\,$\color{gr} \color{blue} 2\color{black} $\,\,$ & $\,\,$5.3106$\,\,$ & $\,\,$8.9651$\,\,$ \\
$\,\,$\color{gr} \color{blue}  1/2\color{black} $\,\,$ & $\,\,$ 1 $\,\,$ & $\,\,$\color{gr} 2.6553\color{black} $\,\,$ & $\,\,$\color{gr} 4.4826\color{black}   $\,\,$ \\
$\,\,$0.1883$\,\,$ & $\,\,$\color{gr} 0.3766\color{black} $\,\,$ & $\,\,$ 1 $\,\,$ & $\,\,$1.6882 $\,\,$ \\
$\,\,$0.1115$\,\,$ & $\,\,$\color{gr} 0.2231\color{black} $\,\,$ & $\,\,$0.5924$\,\,$ & $\,\,$ 1  $\,\,$ \\
\end{pmatrix},
\end{equation*}
\end{example}
\newpage
\begin{example}
\begin{equation*}
\mathbf{A} =
\begin{pmatrix}
$\,\,$ 1 $\,\,$ & $\,\,$2$\,\,$ & $\,\,$9$\,\,$ & $\,\,$6 $\,\,$ \\
$\,\,$ 1/2$\,\,$ & $\,\,$ 1 $\,\,$ & $\,\,$3$\,\,$ & $\,\,$4 $\,\,$ \\
$\,\,$ 1/9$\,\,$ & $\,\,$ 1/3$\,\,$ & $\,\,$ 1 $\,\,$ & $\,\,$2 $\,\,$ \\
$\,\,$ 1/6$\,\,$ & $\,\,$ 1/4$\,\,$ & $\,\,$ 1/2$\,\,$ & $\,\,$ 1  $\,\,$ \\
\end{pmatrix},
\qquad
\lambda_{\max} =
4.1031,
\qquad
CR = 0.0389
\end{equation*}

\begin{equation*}
\mathbf{w}^{EM} =
\begin{pmatrix}
0.571717\\
\color{red} 0.268380\color{black} \\
0.092647\\
0.067256
\end{pmatrix}\end{equation*}
\begin{equation*}
\left[ \frac{{w}^{EM}_i}{{w}^{EM}_j} \right] =
\begin{pmatrix}
$\,\,$ 1 $\,\,$ & $\,\,$\color{red} 2.1302\color{black} $\,\,$ & $\,\,$6.1709$\,\,$ & $\,\,$8.5006$\,\,$ \\
$\,\,$\color{red} 0.4694\color{black} $\,\,$ & $\,\,$ 1 $\,\,$ & $\,\,$\color{red} 2.8968\color{black} $\,\,$ & $\,\,$\color{red} 3.9904\color{black}   $\,\,$ \\
$\,\,$0.1621$\,\,$ & $\,\,$\color{red} 0.3452\color{black} $\,\,$ & $\,\,$ 1 $\,\,$ & $\,\,$1.3775 $\,\,$ \\
$\,\,$0.1176$\,\,$ & $\,\,$\color{red} 0.2506\color{black} $\,\,$ & $\,\,$0.7259$\,\,$ & $\,\,$ 1  $\,\,$ \\
\end{pmatrix},
\end{equation*}

\begin{equation*}
\mathbf{w}^{\prime} =
\begin{pmatrix}
0.571349\\
0.268851\\
0.092587\\
0.067213
\end{pmatrix} =
0.999357\cdot
\begin{pmatrix}
0.571717\\
\color{gr} 0.269024\color{black} \\
0.092647\\
0.067256
\end{pmatrix},
\end{equation*}
\begin{equation*}
\left[ \frac{{w}^{\prime}_i}{{w}^{\prime}_j} \right] =
\begin{pmatrix}
$\,\,$ 1 $\,\,$ & $\,\,$\color{gr} 2.1252\color{black} $\,\,$ & $\,\,$6.1709$\,\,$ & $\,\,$8.5006$\,\,$ \\
$\,\,$\color{gr} 0.4706\color{black} $\,\,$ & $\,\,$ 1 $\,\,$ & $\,\,$\color{gr} 2.9038\color{black} $\,\,$ & $\,\,$\color{gr} \color{blue} 4\color{black}   $\,\,$ \\
$\,\,$0.1621$\,\,$ & $\,\,$\color{gr} 0.3444\color{black} $\,\,$ & $\,\,$ 1 $\,\,$ & $\,\,$1.3775 $\,\,$ \\
$\,\,$0.1176$\,\,$ & $\,\,$\color{gr} \color{blue}  1/4\color{black} $\,\,$ & $\,\,$0.7259$\,\,$ & $\,\,$ 1  $\,\,$ \\
\end{pmatrix},
\end{equation*}
\end{example}
\newpage
\begin{example}
\begin{equation*}
\mathbf{A} =
\begin{pmatrix}
$\,\,$ 1 $\,\,$ & $\,\,$2$\,\,$ & $\,\,$9$\,\,$ & $\,\,$6 $\,\,$ \\
$\,\,$ 1/2$\,\,$ & $\,\,$ 1 $\,\,$ & $\,\,$3$\,\,$ & $\,\,$5 $\,\,$ \\
$\,\,$ 1/9$\,\,$ & $\,\,$ 1/3$\,\,$ & $\,\,$ 1 $\,\,$ & $\,\,$3 $\,\,$ \\
$\,\,$ 1/6$\,\,$ & $\,\,$ 1/5$\,\,$ & $\,\,$ 1/3$\,\,$ & $\,\,$ 1  $\,\,$ \\
\end{pmatrix},
\qquad
\lambda_{\max} =
4.1966,
\qquad
CR = 0.0741
\end{equation*}

\begin{equation*}
\mathbf{w}^{EM} =
\begin{pmatrix}
0.566503\\
\color{red} 0.274088\color{black} \\
0.102078\\
0.057331
\end{pmatrix}\end{equation*}
\begin{equation*}
\left[ \frac{{w}^{EM}_i}{{w}^{EM}_j} \right] =
\begin{pmatrix}
$\,\,$ 1 $\,\,$ & $\,\,$\color{red} 2.0669\color{black} $\,\,$ & $\,\,$5.5497$\,\,$ & $\,\,$9.8813$\,\,$ \\
$\,\,$\color{red} 0.4838\color{black} $\,\,$ & $\,\,$ 1 $\,\,$ & $\,\,$\color{red} 2.6851\color{black} $\,\,$ & $\,\,$\color{red} 4.7808\color{black}   $\,\,$ \\
$\,\,$0.1802$\,\,$ & $\,\,$\color{red} 0.3724\color{black} $\,\,$ & $\,\,$ 1 $\,\,$ & $\,\,$1.7805 $\,\,$ \\
$\,\,$0.1012$\,\,$ & $\,\,$\color{red} 0.2092\color{black} $\,\,$ & $\,\,$0.5616$\,\,$ & $\,\,$ 1  $\,\,$ \\
\end{pmatrix},
\end{equation*}

\begin{equation*}
\mathbf{w}^{\prime} =
\begin{pmatrix}
0.561359\\
0.280680\\
0.101151\\
0.056810
\end{pmatrix} =
0.990920\cdot
\begin{pmatrix}
0.566503\\
\color{gr} 0.283252\color{black} \\
0.102078\\
0.057331
\end{pmatrix},
\end{equation*}
\begin{equation*}
\left[ \frac{{w}^{\prime}_i}{{w}^{\prime}_j} \right] =
\begin{pmatrix}
$\,\,$ 1 $\,\,$ & $\,\,$\color{gr} \color{blue} 2\color{black} $\,\,$ & $\,\,$5.5497$\,\,$ & $\,\,$9.8813$\,\,$ \\
$\,\,$\color{gr} \color{blue}  1/2\color{black} $\,\,$ & $\,\,$ 1 $\,\,$ & $\,\,$\color{gr} 2.7749\color{black} $\,\,$ & $\,\,$\color{gr} 4.9407\color{black}   $\,\,$ \\
$\,\,$0.1802$\,\,$ & $\,\,$\color{gr} 0.3604\color{black} $\,\,$ & $\,\,$ 1 $\,\,$ & $\,\,$1.7805 $\,\,$ \\
$\,\,$0.1012$\,\,$ & $\,\,$\color{gr} 0.2024\color{black} $\,\,$ & $\,\,$0.5616$\,\,$ & $\,\,$ 1  $\,\,$ \\
\end{pmatrix},
\end{equation*}
\end{example}
\newpage
\begin{example}
\begin{equation*}
\mathbf{A} =
\begin{pmatrix}
$\,\,$ 1 $\,\,$ & $\,\,$2$\,\,$ & $\,\,$9$\,\,$ & $\,\,$6 $\,\,$ \\
$\,\,$ 1/2$\,\,$ & $\,\,$ 1 $\,\,$ & $\,\,$6$\,\,$ & $\,\,$2 $\,\,$ \\
$\,\,$ 1/9$\,\,$ & $\,\,$ 1/6$\,\,$ & $\,\,$ 1 $\,\,$ & $\,\,$1 $\,\,$ \\
$\,\,$ 1/6$\,\,$ & $\,\,$ 1/2$\,\,$ & $\,\,$ 1 $\,\,$ & $\,\,$ 1  $\,\,$ \\
\end{pmatrix},
\qquad
\lambda_{\max} =
4.1031,
\qquad
CR = 0.0389
\end{equation*}

\begin{equation*}
\mathbf{w}^{EM} =
\begin{pmatrix}
\color{red} 0.557989\color{black} \\
0.279663\\
0.066036\\
0.096311
\end{pmatrix}\end{equation*}
\begin{equation*}
\left[ \frac{{w}^{EM}_i}{{w}^{EM}_j} \right] =
\begin{pmatrix}
$\,\,$ 1 $\,\,$ & $\,\,$\color{red} 1.9952\color{black} $\,\,$ & $\,\,$\color{red} 8.4497\color{black} $\,\,$ & $\,\,$\color{red} 5.7936\color{black} $\,\,$ \\
$\,\,$\color{red} 0.5012\color{black} $\,\,$ & $\,\,$ 1 $\,\,$ & $\,\,$4.2350$\,\,$ & $\,\,$2.9038  $\,\,$ \\
$\,\,$\color{red} 0.1183\color{black} $\,\,$ & $\,\,$0.2361$\,\,$ & $\,\,$ 1 $\,\,$ & $\,\,$0.6857 $\,\,$ \\
$\,\,$\color{red} 0.1726\color{black} $\,\,$ & $\,\,$0.3444$\,\,$ & $\,\,$1.4585$\,\,$ & $\,\,$ 1  $\,\,$ \\
\end{pmatrix},
\end{equation*}

\begin{equation*}
\mathbf{w}^{\prime} =
\begin{pmatrix}
0.558580\\
0.279290\\
0.065948\\
0.096182
\end{pmatrix} =
0.998664\cdot
\begin{pmatrix}
\color{gr} 0.559327\color{black} \\
0.279663\\
0.066036\\
0.096311
\end{pmatrix},
\end{equation*}
\begin{equation*}
\left[ \frac{{w}^{\prime}_i}{{w}^{\prime}_j} \right] =
\begin{pmatrix}
$\,\,$ 1 $\,\,$ & $\,\,$\color{gr} \color{blue} 2\color{black} $\,\,$ & $\,\,$\color{gr} 8.4700\color{black} $\,\,$ & $\,\,$\color{gr} 5.8075\color{black} $\,\,$ \\
$\,\,$\color{gr} \color{blue}  1/2\color{black} $\,\,$ & $\,\,$ 1 $\,\,$ & $\,\,$4.2350$\,\,$ & $\,\,$2.9038  $\,\,$ \\
$\,\,$\color{gr} 0.1181\color{black} $\,\,$ & $\,\,$0.2361$\,\,$ & $\,\,$ 1 $\,\,$ & $\,\,$0.6857 $\,\,$ \\
$\,\,$\color{gr} 0.1722\color{black} $\,\,$ & $\,\,$0.3444$\,\,$ & $\,\,$1.4585$\,\,$ & $\,\,$ 1  $\,\,$ \\
\end{pmatrix},
\end{equation*}
\end{example}
\newpage
\begin{example}
\begin{equation*}
\mathbf{A} =
\begin{pmatrix}
$\,\,$ 1 $\,\,$ & $\,\,$2$\,\,$ & $\,\,$9$\,\,$ & $\,\,$6 $\,\,$ \\
$\,\,$ 1/2$\,\,$ & $\,\,$ 1 $\,\,$ & $\,\,$6$\,\,$ & $\,\,$9 $\,\,$ \\
$\,\,$ 1/9$\,\,$ & $\,\,$ 1/6$\,\,$ & $\,\,$ 1 $\,\,$ & $\,\,$1 $\,\,$ \\
$\,\,$ 1/6$\,\,$ & $\,\,$ 1/9$\,\,$ & $\,\,$ 1 $\,\,$ & $\,\,$ 1  $\,\,$ \\
\end{pmatrix},
\qquad
\lambda_{\max} =
4.1031,
\qquad
CR = 0.0389
\end{equation*}

\begin{equation*}
\mathbf{w}^{EM} =
\begin{pmatrix}
0.517057\\
0.366275\\
\color{red} 0.057313\color{black} \\
0.059355
\end{pmatrix}\end{equation*}
\begin{equation*}
\left[ \frac{{w}^{EM}_i}{{w}^{EM}_j} \right] =
\begin{pmatrix}
$\,\,$ 1 $\,\,$ & $\,\,$1.4117$\,\,$ & $\,\,$\color{red} 9.0216\color{black} $\,\,$ & $\,\,$8.7113$\,\,$ \\
$\,\,$0.7084$\,\,$ & $\,\,$ 1 $\,\,$ & $\,\,$\color{red} 6.3907\color{black} $\,\,$ & $\,\,$6.1709  $\,\,$ \\
$\,\,$\color{red} 0.1108\color{black} $\,\,$ & $\,\,$\color{red} 0.1565\color{black} $\,\,$ & $\,\,$ 1 $\,\,$ & $\,\,$\color{red} 0.9656\color{black}  $\,\,$ \\
$\,\,$0.1148$\,\,$ & $\,\,$0.1621$\,\,$ & $\,\,$\color{red} 1.0356\color{black} $\,\,$ & $\,\,$ 1  $\,\,$ \\
\end{pmatrix},
\end{equation*}

\begin{equation*}
\mathbf{w}^{\prime} =
\begin{pmatrix}
0.516986\\
0.366225\\
0.057443\\
0.059347
\end{pmatrix} =
0.999863\cdot
\begin{pmatrix}
0.517057\\
0.366275\\
\color{gr} 0.057451\color{black} \\
0.059355
\end{pmatrix},
\end{equation*}
\begin{equation*}
\left[ \frac{{w}^{\prime}_i}{{w}^{\prime}_j} \right] =
\begin{pmatrix}
$\,\,$ 1 $\,\,$ & $\,\,$1.4117$\,\,$ & $\,\,$\color{gr} \color{blue} 9\color{black} $\,\,$ & $\,\,$8.7113$\,\,$ \\
$\,\,$0.7084$\,\,$ & $\,\,$ 1 $\,\,$ & $\,\,$\color{gr} 6.3755\color{black} $\,\,$ & $\,\,$6.1709  $\,\,$ \\
$\,\,$\color{gr} \color{blue}  1/9\color{black} $\,\,$ & $\,\,$\color{gr} 0.1569\color{black} $\,\,$ & $\,\,$ 1 $\,\,$ & $\,\,$\color{gr} 0.9679\color{black}  $\,\,$ \\
$\,\,$0.1148$\,\,$ & $\,\,$0.1621$\,\,$ & $\,\,$\color{gr} 1.0331\color{black} $\,\,$ & $\,\,$ 1  $\,\,$ \\
\end{pmatrix},
\end{equation*}
\end{example}
\newpage
\begin{example}
\begin{equation*}
\mathbf{A} =
\begin{pmatrix}
$\,\,$ 1 $\,\,$ & $\,\,$2$\,\,$ & $\,\,$9$\,\,$ & $\,\,$6 $\,\,$ \\
$\,\,$ 1/2$\,\,$ & $\,\,$ 1 $\,\,$ & $\,\,$7$\,\,$ & $\,\,$2 $\,\,$ \\
$\,\,$ 1/9$\,\,$ & $\,\,$ 1/7$\,\,$ & $\,\,$ 1 $\,\,$ & $\,\,$1 $\,\,$ \\
$\,\,$ 1/6$\,\,$ & $\,\,$ 1/2$\,\,$ & $\,\,$ 1 $\,\,$ & $\,\,$ 1  $\,\,$ \\
\end{pmatrix},
\qquad
\lambda_{\max} =
4.1342,
\qquad
CR = 0.0506
\end{equation*}

\begin{equation*}
\mathbf{w}^{EM} =
\begin{pmatrix}
\color{red} 0.550511\color{black} \\
0.290372\\
0.063318\\
0.095800
\end{pmatrix}\end{equation*}
\begin{equation*}
\left[ \frac{{w}^{EM}_i}{{w}^{EM}_j} \right] =
\begin{pmatrix}
$\,\,$ 1 $\,\,$ & $\,\,$\color{red} 1.8959\color{black} $\,\,$ & $\,\,$\color{red} 8.6944\color{black} $\,\,$ & $\,\,$\color{red} 5.7465\color{black} $\,\,$ \\
$\,\,$\color{red} 0.5275\color{black} $\,\,$ & $\,\,$ 1 $\,\,$ & $\,\,$4.5859$\,\,$ & $\,\,$3.0310  $\,\,$ \\
$\,\,$\color{red} 0.1150\color{black} $\,\,$ & $\,\,$0.2181$\,\,$ & $\,\,$ 1 $\,\,$ & $\,\,$0.6609 $\,\,$ \\
$\,\,$\color{red} 0.1740\color{black} $\,\,$ & $\,\,$0.3299$\,\,$ & $\,\,$1.5130$\,\,$ & $\,\,$ 1  $\,\,$ \\
\end{pmatrix},
\end{equation*}

\begin{equation*}
\mathbf{w}^{\prime} =
\begin{pmatrix}
0.559043\\
0.284860\\
0.062116\\
0.093982
\end{pmatrix} =
0.981018\cdot
\begin{pmatrix}
\color{gr} 0.569859\color{black} \\
0.290372\\
0.063318\\
0.095800
\end{pmatrix},
\end{equation*}
\begin{equation*}
\left[ \frac{{w}^{\prime}_i}{{w}^{\prime}_j} \right] =
\begin{pmatrix}
$\,\,$ 1 $\,\,$ & $\,\,$\color{gr} 1.9625\color{black} $\,\,$ & $\,\,$\color{gr} \color{blue} 9\color{black} $\,\,$ & $\,\,$\color{gr} 5.9484\color{black} $\,\,$ \\
$\,\,$\color{gr} 0.5095\color{black} $\,\,$ & $\,\,$ 1 $\,\,$ & $\,\,$4.5859$\,\,$ & $\,\,$3.0310  $\,\,$ \\
$\,\,$\color{gr} \color{blue}  1/9\color{black} $\,\,$ & $\,\,$0.2181$\,\,$ & $\,\,$ 1 $\,\,$ & $\,\,$0.6609 $\,\,$ \\
$\,\,$\color{gr} 0.1681\color{black} $\,\,$ & $\,\,$0.3299$\,\,$ & $\,\,$1.5130$\,\,$ & $\,\,$ 1  $\,\,$ \\
\end{pmatrix},
\end{equation*}
\end{example}
\newpage
\begin{example}
\begin{equation*}
\mathbf{A} =
\begin{pmatrix}
$\,\,$ 1 $\,\,$ & $\,\,$2$\,\,$ & $\,\,$9$\,\,$ & $\,\,$6 $\,\,$ \\
$\,\,$ 1/2$\,\,$ & $\,\,$ 1 $\,\,$ & $\,\,$8$\,\,$ & $\,\,$2 $\,\,$ \\
$\,\,$ 1/9$\,\,$ & $\,\,$ 1/8$\,\,$ & $\,\,$ 1 $\,\,$ & $\,\,$1 $\,\,$ \\
$\,\,$ 1/6$\,\,$ & $\,\,$ 1/2$\,\,$ & $\,\,$ 1 $\,\,$ & $\,\,$ 1  $\,\,$ \\
\end{pmatrix},
\qquad
\lambda_{\max} =
4.1664,
\qquad
CR = 0.0627
\end{equation*}

\begin{equation*}
\mathbf{w}^{EM} =
\begin{pmatrix}
\color{red} 0.543550\color{black} \\
0.300161\\
0.061013\\
0.095277
\end{pmatrix}\end{equation*}
\begin{equation*}
\left[ \frac{{w}^{EM}_i}{{w}^{EM}_j} \right] =
\begin{pmatrix}
$\,\,$ 1 $\,\,$ & $\,\,$\color{red} 1.8109\color{black} $\,\,$ & $\,\,$\color{red} 8.9088\color{black} $\,\,$ & $\,\,$\color{red} 5.7050\color{black} $\,\,$ \\
$\,\,$\color{red} 0.5522\color{black} $\,\,$ & $\,\,$ 1 $\,\,$ & $\,\,$4.9196$\,\,$ & $\,\,$3.1504  $\,\,$ \\
$\,\,$\color{red} 0.1122\color{black} $\,\,$ & $\,\,$0.2033$\,\,$ & $\,\,$ 1 $\,\,$ & $\,\,$0.6404 $\,\,$ \\
$\,\,$\color{red} 0.1753\color{black} $\,\,$ & $\,\,$0.3174$\,\,$ & $\,\,$1.5616$\,\,$ & $\,\,$ 1  $\,\,$ \\
\end{pmatrix},
\end{equation*}

\begin{equation*}
\mathbf{w}^{\prime} =
\begin{pmatrix}
0.546076\\
0.298499\\
0.060675\\
0.094749
\end{pmatrix} =
0.994465\cdot
\begin{pmatrix}
\color{gr} 0.549115\color{black} \\
0.300161\\
0.061013\\
0.095277
\end{pmatrix},
\end{equation*}
\begin{equation*}
\left[ \frac{{w}^{\prime}_i}{{w}^{\prime}_j} \right] =
\begin{pmatrix}
$\,\,$ 1 $\,\,$ & $\,\,$\color{gr} 1.8294\color{black} $\,\,$ & $\,\,$\color{gr} \color{blue} 9\color{black} $\,\,$ & $\,\,$\color{gr} 5.7634\color{black} $\,\,$ \\
$\,\,$\color{gr} 0.5466\color{black} $\,\,$ & $\,\,$ 1 $\,\,$ & $\,\,$4.9196$\,\,$ & $\,\,$3.1504  $\,\,$ \\
$\,\,$\color{gr} \color{blue}  1/9\color{black} $\,\,$ & $\,\,$0.2033$\,\,$ & $\,\,$ 1 $\,\,$ & $\,\,$0.6404 $\,\,$ \\
$\,\,$\color{gr} 0.1735\color{black} $\,\,$ & $\,\,$0.3174$\,\,$ & $\,\,$1.5616$\,\,$ & $\,\,$ 1  $\,\,$ \\
\end{pmatrix},
\end{equation*}
\end{example}
\newpage
\begin{example}
\begin{equation*}
\mathbf{A} =
\begin{pmatrix}
$\,\,$ 1 $\,\,$ & $\,\,$2$\,\,$ & $\,\,$9$\,\,$ & $\,\,$7 $\,\,$ \\
$\,\,$ 1/2$\,\,$ & $\,\,$ 1 $\,\,$ & $\,\,$3$\,\,$ & $\,\,$6 $\,\,$ \\
$\,\,$ 1/9$\,\,$ & $\,\,$ 1/3$\,\,$ & $\,\,$ 1 $\,\,$ & $\,\,$3 $\,\,$ \\
$\,\,$ 1/7$\,\,$ & $\,\,$ 1/6$\,\,$ & $\,\,$ 1/3$\,\,$ & $\,\,$ 1  $\,\,$ \\
\end{pmatrix},
\qquad
\lambda_{\max} =
4.1571,
\qquad
CR = 0.0593
\end{equation*}

\begin{equation*}
\mathbf{w}^{EM} =
\begin{pmatrix}
0.570430\\
\color{red} 0.280465\color{black} \\
0.098128\\
0.050978
\end{pmatrix}\end{equation*}
\begin{equation*}
\left[ \frac{{w}^{EM}_i}{{w}^{EM}_j} \right] =
\begin{pmatrix}
$\,\,$ 1 $\,\,$ & $\,\,$\color{red} 2.0339\color{black} $\,\,$ & $\,\,$5.8131$\,\,$ & $\,\,$11.1898$\,\,$ \\
$\,\,$\color{red} 0.4917\color{black} $\,\,$ & $\,\,$ 1 $\,\,$ & $\,\,$\color{red} 2.8582\color{black} $\,\,$ & $\,\,$\color{red} 5.5017\color{black}   $\,\,$ \\
$\,\,$0.1720$\,\,$ & $\,\,$\color{red} 0.3499\color{black} $\,\,$ & $\,\,$ 1 $\,\,$ & $\,\,$1.9249 $\,\,$ \\
$\,\,$0.0894$\,\,$ & $\,\,$\color{red} 0.1818\color{black} $\,\,$ & $\,\,$0.5195$\,\,$ & $\,\,$ 1  $\,\,$ \\
\end{pmatrix},
\end{equation*}

\begin{equation*}
\mathbf{w}^{\prime} =
\begin{pmatrix}
0.567733\\
0.283866\\
0.097664\\
0.050737
\end{pmatrix} =
0.995272\cdot
\begin{pmatrix}
0.570430\\
\color{gr} 0.285215\color{black} \\
0.098128\\
0.050978
\end{pmatrix},
\end{equation*}
\begin{equation*}
\left[ \frac{{w}^{\prime}_i}{{w}^{\prime}_j} \right] =
\begin{pmatrix}
$\,\,$ 1 $\,\,$ & $\,\,$\color{gr} \color{blue} 2\color{black} $\,\,$ & $\,\,$5.8131$\,\,$ & $\,\,$11.1898$\,\,$ \\
$\,\,$\color{gr} \color{blue}  1/2\color{black} $\,\,$ & $\,\,$ 1 $\,\,$ & $\,\,$\color{gr} 2.9066\color{black} $\,\,$ & $\,\,$\color{gr} 5.5949\color{black}   $\,\,$ \\
$\,\,$0.1720$\,\,$ & $\,\,$\color{gr} 0.3440\color{black} $\,\,$ & $\,\,$ 1 $\,\,$ & $\,\,$1.9249 $\,\,$ \\
$\,\,$0.0894$\,\,$ & $\,\,$\color{gr} 0.1787\color{black} $\,\,$ & $\,\,$0.5195$\,\,$ & $\,\,$ 1  $\,\,$ \\
\end{pmatrix},
\end{equation*}
\end{example}
\newpage
\begin{example}
\begin{equation*}
\mathbf{A} =
\begin{pmatrix}
$\,\,$ 1 $\,\,$ & $\,\,$2$\,\,$ & $\,\,$9$\,\,$ & $\,\,$7 $\,\,$ \\
$\,\,$ 1/2$\,\,$ & $\,\,$ 1 $\,\,$ & $\,\,$3$\,\,$ & $\,\,$6 $\,\,$ \\
$\,\,$ 1/9$\,\,$ & $\,\,$ 1/3$\,\,$ & $\,\,$ 1 $\,\,$ & $\,\,$4 $\,\,$ \\
$\,\,$ 1/7$\,\,$ & $\,\,$ 1/6$\,\,$ & $\,\,$ 1/4$\,\,$ & $\,\,$ 1  $\,\,$ \\
\end{pmatrix},
\qquad
\lambda_{\max} =
4.2359,
\qquad
CR = 0.0890
\end{equation*}

\begin{equation*}
\mathbf{w}^{EM} =
\begin{pmatrix}
0.570212\\
\color{red} 0.275386\color{black} \\
0.106795\\
0.047607
\end{pmatrix}\end{equation*}
\begin{equation*}
\left[ \frac{{w}^{EM}_i}{{w}^{EM}_j} \right] =
\begin{pmatrix}
$\,\,$ 1 $\,\,$ & $\,\,$\color{red} 2.0706\color{black} $\,\,$ & $\,\,$5.3393$\,\,$ & $\,\,$11.9774$\,\,$ \\
$\,\,$\color{red} 0.4830\color{black} $\,\,$ & $\,\,$ 1 $\,\,$ & $\,\,$\color{red} 2.5786\color{black} $\,\,$ & $\,\,$\color{red} 5.7845\color{black}   $\,\,$ \\
$\,\,$0.1873$\,\,$ & $\,\,$\color{red} 0.3878\color{black} $\,\,$ & $\,\,$ 1 $\,\,$ & $\,\,$2.2432 $\,\,$ \\
$\,\,$0.0835$\,\,$ & $\,\,$\color{red} 0.1729\color{black} $\,\,$ & $\,\,$0.4458$\,\,$ & $\,\,$ 1  $\,\,$ \\
\end{pmatrix},
\end{equation*}

\begin{equation*}
\mathbf{w}^{\prime} =
\begin{pmatrix}
0.564723\\
0.282361\\
0.105767\\
0.047149
\end{pmatrix} =
0.990374\cdot
\begin{pmatrix}
0.570212\\
\color{gr} 0.285106\color{black} \\
0.106795\\
0.047607
\end{pmatrix},
\end{equation*}
\begin{equation*}
\left[ \frac{{w}^{\prime}_i}{{w}^{\prime}_j} \right] =
\begin{pmatrix}
$\,\,$ 1 $\,\,$ & $\,\,$\color{gr} \color{blue} 2\color{black} $\,\,$ & $\,\,$5.3393$\,\,$ & $\,\,$11.9774$\,\,$ \\
$\,\,$\color{gr} \color{blue}  1/2\color{black} $\,\,$ & $\,\,$ 1 $\,\,$ & $\,\,$\color{gr} 2.6697\color{black} $\,\,$ & $\,\,$\color{gr} 5.9887\color{black}   $\,\,$ \\
$\,\,$0.1873$\,\,$ & $\,\,$\color{gr} 0.3746\color{black} $\,\,$ & $\,\,$ 1 $\,\,$ & $\,\,$2.2432 $\,\,$ \\
$\,\,$0.0835$\,\,$ & $\,\,$\color{gr} 0.1670\color{black} $\,\,$ & $\,\,$0.4458$\,\,$ & $\,\,$ 1  $\,\,$ \\
\end{pmatrix},
\end{equation*}
\end{example}
\newpage
\begin{example}   % Example 1.116
\begin{equation*}
\mathbf{A} =
\begin{pmatrix}
$\,\,$ 1 $\,\,$ & $\,\,$2$\,\,$ & $\,\,$9$\,\,$ & $\,\,$8 $\,\,$ \\
$\,\,$ 1/2$\,\,$ & $\,\,$ 1 $\,\,$ & $\,\,$3$\,\,$ & $\,\,$6 $\,\,$ \\
$\,\,$ 1/9$\,\,$ & $\,\,$ 1/3$\,\,$ & $\,\,$ 1 $\,\,$ & $\,\,$3 $\,\,$ \\
$\,\,$ 1/8$\,\,$ & $\,\,$ 1/6$\,\,$ & $\,\,$ 1/3$\,\,$ & $\,\,$ 1  $\,\,$ \\
\end{pmatrix},
\qquad
\lambda_{\max} =
4.1263,
\qquad
CR = 0.0476
\end{equation*}

\begin{equation*}
\mathbf{w}^{EM} =
\begin{pmatrix}
0.578100\\
\color{red} 0.277375\color{black} \\
0.096350\\
0.048175
\end{pmatrix}\end{equation*}
\begin{equation*}
\left[ \frac{{w}^{EM}_i}{{w}^{EM}_j} \right] =
\begin{pmatrix}
$\,\,$ 1 $\,\,$ & $\,\,$\color{red} 2.0842\color{black} $\,\,$ & $\,\,$6$\,\,$ & $\,\,$12$\,\,$ \\
$\,\,$\color{red} 0.4798\color{black} $\,\,$ & $\,\,$ 1 $\,\,$ & $\,\,$\color{red} 2.8788\color{black} $\,\,$ & $\,\,$\color{red} 5.7577\color{black}   $\,\,$ \\
$\,\,$1/6$\,\,$ & $\,\,$\color{red} 0.3474\color{black} $\,\,$ & $\,\,$ 1 $\,\,$ & $\,\,$2 $\,\,$ \\
$\,\,$1/12$\,\,$ & $\,\,$\color{red} 0.1737\color{black} $\,\,$ & $\,\,$1/2$\,\,$ & $\,\,$ 1  $\,\,$ \\
\end{pmatrix},
\end{equation*}

\begin{equation*}
\mathbf{w}^{\prime} =
\begin{pmatrix}
0.571429\\
0.285714\\
0.095238\\
0.047619
\end{pmatrix} =
0.988460\cdot
\begin{pmatrix}
0.578100\\
\color{gr} 0.289050\color{black} \\
0.096350\\
0.048175
\end{pmatrix},
\end{equation*}
\begin{equation*}
\left[ \frac{{w}^{\prime}_i}{{w}^{\prime}_j} \right] =
\begin{pmatrix}
$\,\,$ 1 $\,\,$ & $\,\,$\color{blue} 2\color{black} $\,\,$ & $\,\,$6$\,\,$ & $\,\,$12$\,\,$ \\
$\,\,$\color{blue} 1/2\color{black} $\,\,$ & $\,\,$ 1 $\,\,$ & $\,\,$\color{blue} 3\color{black} $\,\,$ & $\,\,$\color{gr} \color{blue} 6\color{black}   $\,\,$ \\
$\,\,$1/6$\,\,$ & $\,\,$\color{blue} 1/3\color{black} $\,\,$ & $\,\,$ 1 $\,\,$ & $\,\,$2 $\,\,$ \\
$\,\,$1/12$\,\,$ & $\,\,$\color{gr} \color{blue}  1/6\color{black} $\,\,$ & $\,\,$1/2$\,\,$ & $\,\,$ 1  $\,\,$ \\
\end{pmatrix},
\end{equation*}
\end{example}
\newpage
\begin{example}
\begin{equation*}
\mathbf{A} =
\begin{pmatrix}
$\,\,$ 1 $\,\,$ & $\,\,$2$\,\,$ & $\,\,$9$\,\,$ & $\,\,$8 $\,\,$ \\
$\,\,$ 1/2$\,\,$ & $\,\,$ 1 $\,\,$ & $\,\,$3$\,\,$ & $\,\,$7 $\,\,$ \\
$\,\,$ 1/9$\,\,$ & $\,\,$ 1/3$\,\,$ & $\,\,$ 1 $\,\,$ & $\,\,$4 $\,\,$ \\
$\,\,$ 1/8$\,\,$ & $\,\,$ 1/7$\,\,$ & $\,\,$ 1/4$\,\,$ & $\,\,$ 1  $\,\,$ \\
\end{pmatrix},
\qquad
\lambda_{\max} =
4.1964,
\qquad
CR = 0.0741
\end{equation*}

\begin{equation*}
\mathbf{w}^{EM} =
\begin{pmatrix}
0.573348\\
\color{red} 0.280601\color{black} \\
0.103030\\
0.043021
\end{pmatrix}\end{equation*}
\begin{equation*}
\left[ \frac{{w}^{EM}_i}{{w}^{EM}_j} \right] =
\begin{pmatrix}
$\,\,$ 1 $\,\,$ & $\,\,$\color{red} 2.0433\color{black} $\,\,$ & $\,\,$5.5649$\,\,$ & $\,\,$13.3271$\,\,$ \\
$\,\,$\color{red} 0.4894\color{black} $\,\,$ & $\,\,$ 1 $\,\,$ & $\,\,$\color{red} 2.7235\color{black} $\,\,$ & $\,\,$\color{red} 6.5224\color{black}   $\,\,$ \\
$\,\,$0.1797$\,\,$ & $\,\,$\color{red} 0.3672\color{black} $\,\,$ & $\,\,$ 1 $\,\,$ & $\,\,$2.3949 $\,\,$ \\
$\,\,$0.0750$\,\,$ & $\,\,$\color{red} 0.1533\color{black} $\,\,$ & $\,\,$0.4176$\,\,$ & $\,\,$ 1  $\,\,$ \\
\end{pmatrix},
\end{equation*}

\begin{equation*}
\mathbf{w}^{\prime} =
\begin{pmatrix}
0.569887\\
0.284944\\
0.102408\\
0.042761
\end{pmatrix} =
0.993964\cdot
\begin{pmatrix}
0.573348\\
\color{gr} 0.286674\color{black} \\
0.103030\\
0.043021
\end{pmatrix},
\end{equation*}
\begin{equation*}
\left[ \frac{{w}^{\prime}_i}{{w}^{\prime}_j} \right] =
\begin{pmatrix}
$\,\,$ 1 $\,\,$ & $\,\,$\color{gr} \color{blue} 2\color{black} $\,\,$ & $\,\,$5.5649$\,\,$ & $\,\,$13.3271$\,\,$ \\
$\,\,$\color{gr} \color{blue}  1/2\color{black} $\,\,$ & $\,\,$ 1 $\,\,$ & $\,\,$\color{gr} 2.7824\color{black} $\,\,$ & $\,\,$\color{gr} 6.6636\color{black}   $\,\,$ \\
$\,\,$0.1797$\,\,$ & $\,\,$\color{gr} 0.3594\color{black} $\,\,$ & $\,\,$ 1 $\,\,$ & $\,\,$2.3949 $\,\,$ \\
$\,\,$0.0750$\,\,$ & $\,\,$\color{gr} 0.1501\color{black} $\,\,$ & $\,\,$0.4176$\,\,$ & $\,\,$ 1  $\,\,$ \\
\end{pmatrix},
\end{equation*}
\end{example}
\newpage
\begin{example}
\begin{equation*}
\mathbf{A} =
\begin{pmatrix}
$\,\,$ 1 $\,\,$ & $\,\,$2$\,\,$ & $\,\,$9$\,\,$ & $\,\,$8 $\,\,$ \\
$\,\,$ 1/2$\,\,$ & $\,\,$ 1 $\,\,$ & $\,\,$3$\,\,$ & $\,\,$7 $\,\,$ \\
$\,\,$ 1/9$\,\,$ & $\,\,$ 1/3$\,\,$ & $\,\,$ 1 $\,\,$ & $\,\,$5 $\,\,$ \\
$\,\,$ 1/8$\,\,$ & $\,\,$ 1/7$\,\,$ & $\,\,$ 1/5$\,\,$ & $\,\,$ 1  $\,\,$ \\
\end{pmatrix},
\qquad
\lambda_{\max} =
4.2649,
\qquad
CR = 0.0999
\end{equation*}

\begin{equation*}
\mathbf{w}^{EM} =
\begin{pmatrix}
0.572771\\
\color{red} 0.276325\color{black} \\
0.110137\\
0.040767
\end{pmatrix}\end{equation*}
\begin{equation*}
\left[ \frac{{w}^{EM}_i}{{w}^{EM}_j} \right] =
\begin{pmatrix}
$\,\,$ 1 $\,\,$ & $\,\,$\color{red} 2.0728\color{black} $\,\,$ & $\,\,$5.2005$\,\,$ & $\,\,$14.0499$\,\,$ \\
$\,\,$\color{red} 0.4824\color{black} $\,\,$ & $\,\,$ 1 $\,\,$ & $\,\,$\color{red} 2.5089\color{black} $\,\,$ & $\,\,$\color{red} 6.7782\color{black}   $\,\,$ \\
$\,\,$0.1923$\,\,$ & $\,\,$\color{red} 0.3986\color{black} $\,\,$ & $\,\,$ 1 $\,\,$ & $\,\,$2.7016 $\,\,$ \\
$\,\,$0.0712$\,\,$ & $\,\,$\color{red} 0.1475\color{black} $\,\,$ & $\,\,$0.3701$\,\,$ & $\,\,$ 1  $\,\,$ \\
\end{pmatrix},
\end{equation*}

\begin{equation*}
\mathbf{w}^{\prime} =
\begin{pmatrix}
0.567637\\
0.282811\\
0.109150\\
0.040402
\end{pmatrix} =
0.991038\cdot
\begin{pmatrix}
0.572771\\
\color{gr} 0.285369\color{black} \\
0.110137\\
0.040767
\end{pmatrix},
\end{equation*}
\begin{equation*}
\left[ \frac{{w}^{\prime}_i}{{w}^{\prime}_j} \right] =
\begin{pmatrix}
$\,\,$ 1 $\,\,$ & $\,\,$\color{gr} 2.0071\color{black} $\,\,$ & $\,\,$5.2005$\,\,$ & $\,\,$14.0499$\,\,$ \\
$\,\,$\color{gr} 0.4982\color{black} $\,\,$ & $\,\,$ 1 $\,\,$ & $\,\,$\color{gr} 2.5910\color{black} $\,\,$ & $\,\,$\color{gr} \color{blue} 7\color{black}   $\,\,$ \\
$\,\,$0.1923$\,\,$ & $\,\,$\color{gr} 0.3859\color{black} $\,\,$ & $\,\,$ 1 $\,\,$ & $\,\,$2.7016 $\,\,$ \\
$\,\,$0.0712$\,\,$ & $\,\,$\color{gr} \color{blue}  1/7\color{black} $\,\,$ & $\,\,$0.3701$\,\,$ & $\,\,$ 1  $\,\,$ \\
\end{pmatrix},
\end{equation*}
\end{example}
\newpage
\begin{example}
\begin{equation*}
\mathbf{A} =
\begin{pmatrix}
$\,\,$ 1 $\,\,$ & $\,\,$2$\,\,$ & $\,\,$9$\,\,$ & $\,\,$8 $\,\,$ \\
$\,\,$ 1/2$\,\,$ & $\,\,$ 1 $\,\,$ & $\,\,$3$\,\,$ & $\,\,$8 $\,\,$ \\
$\,\,$ 1/9$\,\,$ & $\,\,$ 1/3$\,\,$ & $\,\,$ 1 $\,\,$ & $\,\,$5 $\,\,$ \\
$\,\,$ 1/8$\,\,$ & $\,\,$ 1/8$\,\,$ & $\,\,$ 1/5$\,\,$ & $\,\,$ 1  $\,\,$ \\
\end{pmatrix},
\qquad
\lambda_{\max} =
4.2637,
\qquad
CR = 0.0994
\end{equation*}

\begin{equation*}
\mathbf{w}^{EM} =
\begin{pmatrix}
0.569021\\
\color{red} 0.283203\color{black} \\
0.108488\\
0.039289
\end{pmatrix}\end{equation*}
\begin{equation*}
\left[ \frac{{w}^{EM}_i}{{w}^{EM}_j} \right] =
\begin{pmatrix}
$\,\,$ 1 $\,\,$ & $\,\,$\color{red} 2.0092\color{black} $\,\,$ & $\,\,$5.2450$\,\,$ & $\,\,$14.4831$\,\,$ \\
$\,\,$\color{red} 0.4977\color{black} $\,\,$ & $\,\,$ 1 $\,\,$ & $\,\,$\color{red} 2.6105\color{black} $\,\,$ & $\,\,$\color{red} 7.2083\color{black}   $\,\,$ \\
$\,\,$0.1907$\,\,$ & $\,\,$\color{red} 0.3831\color{black} $\,\,$ & $\,\,$ 1 $\,\,$ & $\,\,$2.7613 $\,\,$ \\
$\,\,$0.0690$\,\,$ & $\,\,$\color{red} 0.1387\color{black} $\,\,$ & $\,\,$0.3621$\,\,$ & $\,\,$ 1  $\,\,$ \\
\end{pmatrix},
\end{equation*}

\begin{equation*}
\mathbf{w}^{\prime} =
\begin{pmatrix}
0.568278\\
0.284139\\
0.108346\\
0.039237
\end{pmatrix} =
0.998694\cdot
\begin{pmatrix}
0.569021\\
\color{gr} 0.284511\color{black} \\
0.108488\\
0.039289
\end{pmatrix},
\end{equation*}
\begin{equation*}
\left[ \frac{{w}^{\prime}_i}{{w}^{\prime}_j} \right] =
\begin{pmatrix}
$\,\,$ 1 $\,\,$ & $\,\,$\color{gr} \color{blue} 2\color{black} $\,\,$ & $\,\,$5.2450$\,\,$ & $\,\,$14.4831$\,\,$ \\
$\,\,$\color{gr} \color{blue}  1/2\color{black} $\,\,$ & $\,\,$ 1 $\,\,$ & $\,\,$\color{gr} 2.6225\color{black} $\,\,$ & $\,\,$\color{gr} 7.2415\color{black}   $\,\,$ \\
$\,\,$0.1907$\,\,$ & $\,\,$\color{gr} 0.3813\color{black} $\,\,$ & $\,\,$ 1 $\,\,$ & $\,\,$2.7613 $\,\,$ \\
$\,\,$0.0690$\,\,$ & $\,\,$\color{gr} 0.1381\color{black} $\,\,$ & $\,\,$0.3621$\,\,$ & $\,\,$ 1  $\,\,$ \\
\end{pmatrix},
\end{equation*}
\end{example}
\newpage
\begin{example}
\begin{equation*}
\mathbf{A} =
\begin{pmatrix}
$\,\,$ 1 $\,\,$ & $\,\,$2$\,\,$ & $\,\,$9$\,\,$ & $\,\,$9 $\,\,$ \\
$\,\,$ 1/2$\,\,$ & $\,\,$ 1 $\,\,$ & $\,\,$8$\,\,$ & $\,\,$3 $\,\,$ \\
$\,\,$ 1/9$\,\,$ & $\,\,$ 1/8$\,\,$ & $\,\,$ 1 $\,\,$ & $\,\,$2 $\,\,$ \\
$\,\,$ 1/9$\,\,$ & $\,\,$ 1/3$\,\,$ & $\,\,$ 1/2$\,\,$ & $\,\,$ 1  $\,\,$ \\
\end{pmatrix},
\qquad
\lambda_{\max} =
4.2469,
\qquad
CR = 0.0931
\end{equation*}

\begin{equation*}
\mathbf{w}^{EM} =
\begin{pmatrix}
\color{red} 0.555911\color{black} \\
0.313115\\
0.069156\\
0.061818
\end{pmatrix}\end{equation*}
\begin{equation*}
\left[ \frac{{w}^{EM}_i}{{w}^{EM}_j} \right] =
\begin{pmatrix}
$\,\,$ 1 $\,\,$ & $\,\,$\color{red} 1.7754\color{black} $\,\,$ & $\,\,$\color{red} 8.0385\color{black} $\,\,$ & $\,\,$\color{red} 8.9927\color{black} $\,\,$ \\
$\,\,$\color{red} 0.5632\color{black} $\,\,$ & $\,\,$ 1 $\,\,$ & $\,\,$4.5277$\,\,$ & $\,\,$5.0651  $\,\,$ \\
$\,\,$\color{red} 0.1244\color{black} $\,\,$ & $\,\,$0.2209$\,\,$ & $\,\,$ 1 $\,\,$ & $\,\,$1.1187 $\,\,$ \\
$\,\,$\color{red} 0.1112\color{black} $\,\,$ & $\,\,$0.1974$\,\,$ & $\,\,$0.8939$\,\,$ & $\,\,$ 1  $\,\,$ \\
\end{pmatrix},
\end{equation*}

\begin{equation*}
\mathbf{w}^{\prime} =
\begin{pmatrix}
0.556111\\
0.312974\\
0.069125\\
0.061790
\end{pmatrix} =
0.999550\cdot
\begin{pmatrix}
\color{gr} 0.556362\color{black} \\
0.313115\\
0.069156\\
0.061818
\end{pmatrix},
\end{equation*}
\begin{equation*}
\left[ \frac{{w}^{\prime}_i}{{w}^{\prime}_j} \right] =
\begin{pmatrix}
$\,\,$ 1 $\,\,$ & $\,\,$\color{gr} 1.7769\color{black} $\,\,$ & $\,\,$\color{gr} 8.0450\color{black} $\,\,$ & $\,\,$\color{gr} \color{blue} 9\color{black} $\,\,$ \\
$\,\,$\color{gr} 0.5628\color{black} $\,\,$ & $\,\,$ 1 $\,\,$ & $\,\,$4.5277$\,\,$ & $\,\,$5.0651  $\,\,$ \\
$\,\,$\color{gr} 0.1243\color{black} $\,\,$ & $\,\,$0.2209$\,\,$ & $\,\,$ 1 $\,\,$ & $\,\,$1.1187 $\,\,$ \\
$\,\,$\color{gr} \color{blue}  1/9\color{black} $\,\,$ & $\,\,$0.1974$\,\,$ & $\,\,$0.8939$\,\,$ & $\,\,$ 1  $\,\,$ \\
\end{pmatrix},
\end{equation*}
\end{example}
\newpage
\begin{example}
\begin{equation*}
\mathbf{A} =
\begin{pmatrix}
$\,\,$ 1 $\,\,$ & $\,\,$3$\,\,$ & $\,\,$2$\,\,$ & $\,\,$3 $\,\,$ \\
$\,\,$ 1/3$\,\,$ & $\,\,$ 1 $\,\,$ & $\,\,$1$\,\,$ & $\,\,$3 $\,\,$ \\
$\,\,$ 1/2$\,\,$ & $\,\,$ 1 $\,\,$ & $\,\,$ 1 $\,\,$ & $\,\,$2 $\,\,$ \\
$\,\,$ 1/3$\,\,$ & $\,\,$ 1/3$\,\,$ & $\,\,$ 1/2$\,\,$ & $\,\,$ 1  $\,\,$ \\
\end{pmatrix},
\qquad
\lambda_{\max} =
4.1031,
\qquad
CR = 0.0389
\end{equation*}

\begin{equation*}
\mathbf{w}^{EM} =
\begin{pmatrix}
0.456442\\
0.221900\\
\color{red} 0.214267\color{black} \\
0.107390
\end{pmatrix}\end{equation*}
\begin{equation*}
\left[ \frac{{w}^{EM}_i}{{w}^{EM}_j} \right] =
\begin{pmatrix}
$\,\,$ 1 $\,\,$ & $\,\,$2.0570$\,\,$ & $\,\,$\color{red} 2.1302\color{black} $\,\,$ & $\,\,$4.2503$\,\,$ \\
$\,\,$0.4862$\,\,$ & $\,\,$ 1 $\,\,$ & $\,\,$\color{red} 1.0356\color{black} $\,\,$ & $\,\,$2.0663  $\,\,$ \\
$\,\,$\color{red} 0.4694\color{black} $\,\,$ & $\,\,$\color{red} 0.9656\color{black} $\,\,$ & $\,\,$ 1 $\,\,$ & $\,\,$\color{red} 1.9952\color{black}  $\,\,$ \\
$\,\,$0.2353$\,\,$ & $\,\,$0.4840$\,\,$ & $\,\,$\color{red} 0.5012\color{black} $\,\,$ & $\,\,$ 1  $\,\,$ \\
\end{pmatrix},
\end{equation*}

\begin{equation*}
\mathbf{w}^{\prime} =
\begin{pmatrix}
0.456208\\
0.221786\\
0.214671\\
0.107335
\end{pmatrix} =
0.999487\cdot
\begin{pmatrix}
0.456442\\
0.221900\\
\color{gr} 0.214781\color{black} \\
0.107390
\end{pmatrix},
\end{equation*}
\begin{equation*}
\left[ \frac{{w}^{\prime}_i}{{w}^{\prime}_j} \right] =
\begin{pmatrix}
$\,\,$ 1 $\,\,$ & $\,\,$2.0570$\,\,$ & $\,\,$\color{gr} 2.1252\color{black} $\,\,$ & $\,\,$4.2503$\,\,$ \\
$\,\,$0.4862$\,\,$ & $\,\,$ 1 $\,\,$ & $\,\,$\color{gr} 1.0331\color{black} $\,\,$ & $\,\,$2.0663  $\,\,$ \\
$\,\,$\color{gr} 0.4706\color{black} $\,\,$ & $\,\,$\color{gr} 0.9679\color{black} $\,\,$ & $\,\,$ 1 $\,\,$ & $\,\,$\color{gr} \color{blue} 2\color{black}  $\,\,$ \\
$\,\,$0.2353$\,\,$ & $\,\,$0.4840$\,\,$ & $\,\,$\color{gr} \color{blue}  1/2\color{black} $\,\,$ & $\,\,$ 1  $\,\,$ \\
\end{pmatrix},
\end{equation*}
\end{example}
\newpage
\begin{example}
\begin{equation*}
\mathbf{A} =
\begin{pmatrix}
$\,\,$ 1 $\,\,$ & $\,\,$3$\,\,$ & $\,\,$2$\,\,$ & $\,\,$4 $\,\,$ \\
$\,\,$ 1/3$\,\,$ & $\,\,$ 1 $\,\,$ & $\,\,$1$\,\,$ & $\,\,$4 $\,\,$ \\
$\,\,$ 1/2$\,\,$ & $\,\,$ 1 $\,\,$ & $\,\,$ 1 $\,\,$ & $\,\,$3 $\,\,$ \\
$\,\,$ 1/4$\,\,$ & $\,\,$ 1/4$\,\,$ & $\,\,$ 1/3$\,\,$ & $\,\,$ 1  $\,\,$ \\
\end{pmatrix},
\qquad
\lambda_{\max} =
4.1031,
\qquad
CR = 0.0389
\end{equation*}

\begin{equation*}
\mathbf{w}^{EM} =
\begin{pmatrix}
0.467688\\
0.226342\\
\color{red} 0.225801\color{black} \\
0.080169
\end{pmatrix}\end{equation*}
\begin{equation*}
\left[ \frac{{w}^{EM}_i}{{w}^{EM}_j} \right] =
\begin{pmatrix}
$\,\,$ 1 $\,\,$ & $\,\,$2.0663$\,\,$ & $\,\,$\color{red} 2.0712\color{black} $\,\,$ & $\,\,$5.8338$\,\,$ \\
$\,\,$0.4840$\,\,$ & $\,\,$ 1 $\,\,$ & $\,\,$\color{red} 1.0024\color{black} $\,\,$ & $\,\,$2.8233  $\,\,$ \\
$\,\,$\color{red} 0.4828\color{black} $\,\,$ & $\,\,$\color{red} 0.9976\color{black} $\,\,$ & $\,\,$ 1 $\,\,$ & $\,\,$\color{red} 2.8166\color{black}  $\,\,$ \\
$\,\,$0.1714$\,\,$ & $\,\,$0.3542$\,\,$ & $\,\,$\color{red} 0.3550\color{black} $\,\,$ & $\,\,$ 1  $\,\,$ \\
\end{pmatrix},
\end{equation*}

\begin{equation*}
\mathbf{w}^{\prime} =
\begin{pmatrix}
0.467435\\
0.226220\\
0.226220\\
0.080125
\end{pmatrix} =
0.999459\cdot
\begin{pmatrix}
0.467688\\
0.226342\\
\color{gr} 0.226342\color{black} \\
0.080169
\end{pmatrix},
\end{equation*}
\begin{equation*}
\left[ \frac{{w}^{\prime}_i}{{w}^{\prime}_j} \right] =
\begin{pmatrix}
$\,\,$ 1 $\,\,$ & $\,\,$2.0663$\,\,$ & $\,\,$\color{gr} 2.0663\color{black} $\,\,$ & $\,\,$5.8338$\,\,$ \\
$\,\,$0.4840$\,\,$ & $\,\,$ 1 $\,\,$ & $\,\,$\color{gr} \color{blue} 1\color{black} $\,\,$ & $\,\,$2.8233  $\,\,$ \\
$\,\,$\color{gr} 0.4840\color{black} $\,\,$ & $\,\,$\color{gr} \color{blue} 1\color{black} $\,\,$ & $\,\,$ 1 $\,\,$ & $\,\,$\color{gr} 2.8233\color{black}  $\,\,$ \\
$\,\,$0.1714$\,\,$ & $\,\,$0.3542$\,\,$ & $\,\,$\color{gr} 0.3542\color{black} $\,\,$ & $\,\,$ 1  $\,\,$ \\
\end{pmatrix},
\end{equation*}
\end{example}
\newpage
\begin{example}
\begin{equation*}
\mathbf{A} =
\begin{pmatrix}
$\,\,$ 1 $\,\,$ & $\,\,$3$\,\,$ & $\,\,$2$\,\,$ & $\,\,$4 $\,\,$ \\
$\,\,$ 1/3$\,\,$ & $\,\,$ 1 $\,\,$ & $\,\,$1$\,\,$ & $\,\,$5 $\,\,$ \\
$\,\,$ 1/2$\,\,$ & $\,\,$ 1 $\,\,$ & $\,\,$ 1 $\,\,$ & $\,\,$3 $\,\,$ \\
$\,\,$ 1/4$\,\,$ & $\,\,$ 1/5$\,\,$ & $\,\,$ 1/3$\,\,$ & $\,\,$ 1  $\,\,$ \\
\end{pmatrix},
\qquad
\lambda_{\max} =
4.1502,
\qquad
CR = 0.0566
\end{equation*}

\begin{equation*}
\mathbf{w}^{EM} =
\begin{pmatrix}
0.464056\\
0.239113\\
\color{red} 0.221397\color{black} \\
0.075435
\end{pmatrix}\end{equation*}
\begin{equation*}
\left[ \frac{{w}^{EM}_i}{{w}^{EM}_j} \right] =
\begin{pmatrix}
$\,\,$ 1 $\,\,$ & $\,\,$1.9407$\,\,$ & $\,\,$\color{red} 2.0960\color{black} $\,\,$ & $\,\,$6.1517$\,\,$ \\
$\,\,$0.5153$\,\,$ & $\,\,$ 1 $\,\,$ & $\,\,$\color{red} 1.0800\color{black} $\,\,$ & $\,\,$3.1698  $\,\,$ \\
$\,\,$\color{red} 0.4771\color{black} $\,\,$ & $\,\,$\color{red} 0.9259\color{black} $\,\,$ & $\,\,$ 1 $\,\,$ & $\,\,$\color{red} 2.9349\color{black}  $\,\,$ \\
$\,\,$0.1626$\,\,$ & $\,\,$0.3155$\,\,$ & $\,\,$\color{red} 0.3407\color{black} $\,\,$ & $\,\,$ 1  $\,\,$ \\
\end{pmatrix},
\end{equation*}

\begin{equation*}
\mathbf{w}^{\prime} =
\begin{pmatrix}
0.461789\\
0.237945\\
0.225199\\
0.075066
\end{pmatrix} =
0.995116\cdot
\begin{pmatrix}
0.464056\\
0.239113\\
\color{gr} 0.226304\color{black} \\
0.075435
\end{pmatrix},
\end{equation*}
\begin{equation*}
\left[ \frac{{w}^{\prime}_i}{{w}^{\prime}_j} \right] =
\begin{pmatrix}
$\,\,$ 1 $\,\,$ & $\,\,$1.9407$\,\,$ & $\,\,$\color{gr} 2.0506\color{black} $\,\,$ & $\,\,$6.1517$\,\,$ \\
$\,\,$0.5153$\,\,$ & $\,\,$ 1 $\,\,$ & $\,\,$\color{gr} 1.0566\color{black} $\,\,$ & $\,\,$3.1698  $\,\,$ \\
$\,\,$\color{gr} 0.4877\color{black} $\,\,$ & $\,\,$\color{gr} 0.9464\color{black} $\,\,$ & $\,\,$ 1 $\,\,$ & $\,\,$\color{gr} \color{blue} 3\color{black}  $\,\,$ \\
$\,\,$0.1626$\,\,$ & $\,\,$0.3155$\,\,$ & $\,\,$\color{gr} \color{blue}  1/3\color{black} $\,\,$ & $\,\,$ 1  $\,\,$ \\
\end{pmatrix},
\end{equation*}
\end{example}
\newpage
\begin{example}
\begin{equation*}
\mathbf{A} =
\begin{pmatrix}
$\,\,$ 1 $\,\,$ & $\,\,$3$\,\,$ & $\,\,$2$\,\,$ & $\,\,$5 $\,\,$ \\
$\,\,$ 1/3$\,\,$ & $\,\,$ 1 $\,\,$ & $\,\,$1$\,\,$ & $\,\,$6 $\,\,$ \\
$\,\,$ 1/2$\,\,$ & $\,\,$ 1 $\,\,$ & $\,\,$ 1 $\,\,$ & $\,\,$4 $\,\,$ \\
$\,\,$ 1/5$\,\,$ & $\,\,$ 1/6$\,\,$ & $\,\,$ 1/4$\,\,$ & $\,\,$ 1  $\,\,$ \\
\end{pmatrix},
\qquad
\lambda_{\max} =
4.1406,
\qquad
CR = 0.0530
\end{equation*}

\begin{equation*}
\mathbf{w}^{EM} =
\begin{pmatrix}
0.471140\\
0.239204\\
\color{red} 0.228749\color{black} \\
0.060907
\end{pmatrix}\end{equation*}
\begin{equation*}
\left[ \frac{{w}^{EM}_i}{{w}^{EM}_j} \right] =
\begin{pmatrix}
$\,\,$ 1 $\,\,$ & $\,\,$1.9696$\,\,$ & $\,\,$\color{red} 2.0596\color{black} $\,\,$ & $\,\,$7.7354$\,\,$ \\
$\,\,$0.5077$\,\,$ & $\,\,$ 1 $\,\,$ & $\,\,$\color{red} 1.0457\color{black} $\,\,$ & $\,\,$3.9274  $\,\,$ \\
$\,\,$\color{red} 0.4855\color{black} $\,\,$ & $\,\,$\color{red} 0.9563\color{black} $\,\,$ & $\,\,$ 1 $\,\,$ & $\,\,$\color{red} 3.7557\color{black}  $\,\,$ \\
$\,\,$0.1293$\,\,$ & $\,\,$0.2546$\,\,$ & $\,\,$\color{red} 0.2663\color{black} $\,\,$ & $\,\,$ 1  $\,\,$ \\
\end{pmatrix},
\end{equation*}

\begin{equation*}
\mathbf{w}^{\prime} =
\begin{pmatrix}
0.467948\\
0.237584\\
0.233974\\
0.060494
\end{pmatrix} =
0.993225\cdot
\begin{pmatrix}
0.471140\\
0.239204\\
\color{gr} 0.235570\color{black} \\
0.060907
\end{pmatrix},
\end{equation*}
\begin{equation*}
\left[ \frac{{w}^{\prime}_i}{{w}^{\prime}_j} \right] =
\begin{pmatrix}
$\,\,$ 1 $\,\,$ & $\,\,$1.9696$\,\,$ & $\,\,$\color{gr} \color{blue} 2\color{black} $\,\,$ & $\,\,$7.7354$\,\,$ \\
$\,\,$0.5077$\,\,$ & $\,\,$ 1 $\,\,$ & $\,\,$\color{gr} 1.0154\color{black} $\,\,$ & $\,\,$3.9274  $\,\,$ \\
$\,\,$\color{gr} \color{blue}  1/2\color{black} $\,\,$ & $\,\,$\color{gr} 0.9848\color{black} $\,\,$ & $\,\,$ 1 $\,\,$ & $\,\,$\color{gr} 3.8677\color{black}  $\,\,$ \\
$\,\,$0.1293$\,\,$ & $\,\,$0.2546$\,\,$ & $\,\,$\color{gr} 0.2586\color{black} $\,\,$ & $\,\,$ 1  $\,\,$ \\
\end{pmatrix},
\end{equation*}
\end{example}
\newpage
\begin{example}
\begin{equation*}
\mathbf{A} =
\begin{pmatrix}
$\,\,$ 1 $\,\,$ & $\,\,$3$\,\,$ & $\,\,$2$\,\,$ & $\,\,$5 $\,\,$ \\
$\,\,$ 1/3$\,\,$ & $\,\,$ 1 $\,\,$ & $\,\,$1$\,\,$ & $\,\,$7 $\,\,$ \\
$\,\,$ 1/2$\,\,$ & $\,\,$ 1 $\,\,$ & $\,\,$ 1 $\,\,$ & $\,\,$4 $\,\,$ \\
$\,\,$ 1/5$\,\,$ & $\,\,$ 1/7$\,\,$ & $\,\,$ 1/4$\,\,$ & $\,\,$ 1  $\,\,$ \\
\end{pmatrix},
\qquad
\lambda_{\max} =
4.1782,
\qquad
CR = 0.0672
\end{equation*}

\begin{equation*}
\mathbf{w}^{EM} =
\begin{pmatrix}
0.468012\\
0.248435\\
\color{red} 0.225218\color{black} \\
0.058335
\end{pmatrix}\end{equation*}
\begin{equation*}
\left[ \frac{{w}^{EM}_i}{{w}^{EM}_j} \right] =
\begin{pmatrix}
$\,\,$ 1 $\,\,$ & $\,\,$1.8838$\,\,$ & $\,\,$\color{red} 2.0780\color{black} $\,\,$ & $\,\,$8.0228$\,\,$ \\
$\,\,$0.5308$\,\,$ & $\,\,$ 1 $\,\,$ & $\,\,$\color{red} 1.1031\color{black} $\,\,$ & $\,\,$4.2588  $\,\,$ \\
$\,\,$\color{red} 0.4812\color{black} $\,\,$ & $\,\,$\color{red} 0.9065\color{black} $\,\,$ & $\,\,$ 1 $\,\,$ & $\,\,$\color{red} 3.8608\color{black}  $\,\,$ \\
$\,\,$0.1246$\,\,$ & $\,\,$0.2348$\,\,$ & $\,\,$\color{red} 0.2590\color{black} $\,\,$ & $\,\,$ 1  $\,\,$ \\
\end{pmatrix},
\end{equation*}

\begin{equation*}
\mathbf{w}^{\prime} =
\begin{pmatrix}
0.464241\\
0.246434\\
0.231460\\
0.057865
\end{pmatrix} =
0.991944\cdot
\begin{pmatrix}
0.468012\\
0.248435\\
\color{gr} 0.233340\color{black} \\
0.058335
\end{pmatrix},
\end{equation*}
\begin{equation*}
\left[ \frac{{w}^{\prime}_i}{{w}^{\prime}_j} \right] =
\begin{pmatrix}
$\,\,$ 1 $\,\,$ & $\,\,$1.8838$\,\,$ & $\,\,$\color{gr} 2.0057\color{black} $\,\,$ & $\,\,$8.0228$\,\,$ \\
$\,\,$0.5308$\,\,$ & $\,\,$ 1 $\,\,$ & $\,\,$\color{gr} 1.0647\color{black} $\,\,$ & $\,\,$4.2588  $\,\,$ \\
$\,\,$\color{gr} 0.4986\color{black} $\,\,$ & $\,\,$\color{gr} 0.9392\color{black} $\,\,$ & $\,\,$ 1 $\,\,$ & $\,\,$\color{gr} \color{blue} 4\color{black}  $\,\,$ \\
$\,\,$0.1246$\,\,$ & $\,\,$0.2348$\,\,$ & $\,\,$\color{gr} \color{blue}  1/4\color{black} $\,\,$ & $\,\,$ 1  $\,\,$ \\
\end{pmatrix},
\end{equation*}
\end{example}
\newpage
\begin{example}
\begin{equation*}
\mathbf{A} =
\begin{pmatrix}
$\,\,$ 1 $\,\,$ & $\,\,$3$\,\,$ & $\,\,$2$\,\,$ & $\,\,$5 $\,\,$ \\
$\,\,$ 1/3$\,\,$ & $\,\,$ 1 $\,\,$ & $\,\,$1$\,\,$ & $\,\,$8 $\,\,$ \\
$\,\,$ 1/2$\,\,$ & $\,\,$ 1 $\,\,$ & $\,\,$ 1 $\,\,$ & $\,\,$4 $\,\,$ \\
$\,\,$ 1/5$\,\,$ & $\,\,$ 1/8$\,\,$ & $\,\,$ 1/4$\,\,$ & $\,\,$ 1  $\,\,$ \\
\end{pmatrix},
\qquad
\lambda_{\max} =
4.2162,
\qquad
CR = 0.0815
\end{equation*}

\begin{equation*}
\mathbf{w}^{EM} =
\begin{pmatrix}
0.464963\\
0.256889\\
\color{red} 0.221995\color{black} \\
0.056154
\end{pmatrix}\end{equation*}
\begin{equation*}
\left[ \frac{{w}^{EM}_i}{{w}^{EM}_j} \right] =
\begin{pmatrix}
$\,\,$ 1 $\,\,$ & $\,\,$1.8100$\,\,$ & $\,\,$\color{red} 2.0945\color{black} $\,\,$ & $\,\,$8.2802$\,\,$ \\
$\,\,$0.5525$\,\,$ & $\,\,$ 1 $\,\,$ & $\,\,$\color{red} 1.1572\color{black} $\,\,$ & $\,\,$4.5748  $\,\,$ \\
$\,\,$\color{red} 0.4774\color{black} $\,\,$ & $\,\,$\color{red} 0.8642\color{black} $\,\,$ & $\,\,$ 1 $\,\,$ & $\,\,$\color{red} 3.9534\color{black}  $\,\,$ \\
$\,\,$0.1208$\,\,$ & $\,\,$0.2186$\,\,$ & $\,\,$\color{red} 0.2529\color{black} $\,\,$ & $\,\,$ 1  $\,\,$ \\
\end{pmatrix},
\end{equation*}

\begin{equation*}
\mathbf{w}^{\prime} =
\begin{pmatrix}
0.463748\\
0.256218\\
0.224027\\
0.056007
\end{pmatrix} =
0.997387\cdot
\begin{pmatrix}
0.464963\\
0.256889\\
\color{gr} 0.224614\color{black} \\
0.056154
\end{pmatrix},
\end{equation*}
\begin{equation*}
\left[ \frac{{w}^{\prime}_i}{{w}^{\prime}_j} \right] =
\begin{pmatrix}
$\,\,$ 1 $\,\,$ & $\,\,$1.8100$\,\,$ & $\,\,$\color{gr} 2.0700\color{black} $\,\,$ & $\,\,$8.2802$\,\,$ \\
$\,\,$0.5525$\,\,$ & $\,\,$ 1 $\,\,$ & $\,\,$\color{gr} 1.1437\color{black} $\,\,$ & $\,\,$4.5748  $\,\,$ \\
$\,\,$\color{gr} 0.4831\color{black} $\,\,$ & $\,\,$\color{gr} 0.8744\color{black} $\,\,$ & $\,\,$ 1 $\,\,$ & $\,\,$\color{gr} \color{blue} 4\color{black}  $\,\,$ \\
$\,\,$0.1208$\,\,$ & $\,\,$0.2186$\,\,$ & $\,\,$\color{gr} \color{blue}  1/4\color{black} $\,\,$ & $\,\,$ 1  $\,\,$ \\
\end{pmatrix},
\end{equation*}
\end{example}
\newpage
\begin{example}
\begin{equation*}
\mathbf{A} =
\begin{pmatrix}
$\,\,$ 1 $\,\,$ & $\,\,$3$\,\,$ & $\,\,$2$\,\,$ & $\,\,$5 $\,\,$ \\
$\,\,$ 1/3$\,\,$ & $\,\,$ 1 $\,\,$ & $\,\,$1$\,\,$ & $\,\,$9 $\,\,$ \\
$\,\,$ 1/2$\,\,$ & $\,\,$ 1 $\,\,$ & $\,\,$ 1 $\,\,$ & $\,\,$5 $\,\,$ \\
$\,\,$ 1/5$\,\,$ & $\,\,$ 1/9$\,\,$ & $\,\,$ 1/5$\,\,$ & $\,\,$ 1  $\,\,$ \\
\end{pmatrix},
\qquad
\lambda_{\max} =
4.2507,
\qquad
CR = 0.0946
\end{equation*}

\begin{equation*}
\mathbf{w}^{EM} =
\begin{pmatrix}
0.459642\\
0.259656\\
\color{red} 0.229432\color{black} \\
0.051270
\end{pmatrix}\end{equation*}
\begin{equation*}
\left[ \frac{{w}^{EM}_i}{{w}^{EM}_j} \right] =
\begin{pmatrix}
$\,\,$ 1 $\,\,$ & $\,\,$1.7702$\,\,$ & $\,\,$\color{red} 2.0034\color{black} $\,\,$ & $\,\,$8.9651$\,\,$ \\
$\,\,$0.5649$\,\,$ & $\,\,$ 1 $\,\,$ & $\,\,$\color{red} 1.1317\color{black} $\,\,$ & $\,\,$5.0645  $\,\,$ \\
$\,\,$\color{red} 0.4992\color{black} $\,\,$ & $\,\,$\color{red} 0.8836\color{black} $\,\,$ & $\,\,$ 1 $\,\,$ & $\,\,$\color{red} 4.4750\color{black}  $\,\,$ \\
$\,\,$0.1115$\,\,$ & $\,\,$0.1975$\,\,$ & $\,\,$\color{red} 0.2235\color{black} $\,\,$ & $\,\,$ 1  $\,\,$ \\
\end{pmatrix},
\end{equation*}

\begin{equation*}
\mathbf{w}^{\prime} =
\begin{pmatrix}
0.459464\\
0.259555\\
0.229732\\
0.051250
\end{pmatrix} =
0.999611\cdot
\begin{pmatrix}
0.459642\\
0.259656\\
\color{gr} 0.229821\color{black} \\
0.051270
\end{pmatrix},
\end{equation*}
\begin{equation*}
\left[ \frac{{w}^{\prime}_i}{{w}^{\prime}_j} \right] =
\begin{pmatrix}
$\,\,$ 1 $\,\,$ & $\,\,$1.7702$\,\,$ & $\,\,$\color{gr} \color{blue} 2\color{black} $\,\,$ & $\,\,$8.9651$\,\,$ \\
$\,\,$0.5649$\,\,$ & $\,\,$ 1 $\,\,$ & $\,\,$\color{gr} 1.1298\color{black} $\,\,$ & $\,\,$5.0645  $\,\,$ \\
$\,\,$\color{gr} \color{blue}  1/2\color{black} $\,\,$ & $\,\,$\color{gr} 0.8851\color{black} $\,\,$ & $\,\,$ 1 $\,\,$ & $\,\,$\color{gr} 4.4826\color{black}  $\,\,$ \\
$\,\,$0.1115$\,\,$ & $\,\,$0.1975$\,\,$ & $\,\,$\color{gr} 0.2231\color{black} $\,\,$ & $\,\,$ 1  $\,\,$ \\
\end{pmatrix},
\end{equation*}
\end{example}
\newpage
\begin{example}
\begin{equation*}
\mathbf{A} =
\begin{pmatrix}
$\,\,$ 1 $\,\,$ & $\,\,$3$\,\,$ & $\,\,$2$\,\,$ & $\,\,$6 $\,\,$ \\
$\,\,$ 1/3$\,\,$ & $\,\,$ 1 $\,\,$ & $\,\,$1$\,\,$ & $\,\,$6 $\,\,$ \\
$\,\,$ 1/2$\,\,$ & $\,\,$ 1 $\,\,$ & $\,\,$ 1 $\,\,$ & $\,\,$4 $\,\,$ \\
$\,\,$ 1/6$\,\,$ & $\,\,$ 1/6$\,\,$ & $\,\,$ 1/4$\,\,$ & $\,\,$ 1  $\,\,$ \\
\end{pmatrix},
\qquad
\lambda_{\max} =
4.1031,
\qquad
CR = 0.0389
\end{equation*}

\begin{equation*}
\mathbf{w}^{EM} =
\begin{pmatrix}
0.482342\\
0.234491\\
\color{red} 0.226425\color{black} \\
0.056742
\end{pmatrix}\end{equation*}
\begin{equation*}
\left[ \frac{{w}^{EM}_i}{{w}^{EM}_j} \right] =
\begin{pmatrix}
$\,\,$ 1 $\,\,$ & $\,\,$2.0570$\,\,$ & $\,\,$\color{red} 2.1302\color{black} $\,\,$ & $\,\,$8.5006$\,\,$ \\
$\,\,$0.4862$\,\,$ & $\,\,$ 1 $\,\,$ & $\,\,$\color{red} 1.0356\color{black} $\,\,$ & $\,\,$4.1326  $\,\,$ \\
$\,\,$\color{red} 0.4694\color{black} $\,\,$ & $\,\,$\color{red} 0.9656\color{black} $\,\,$ & $\,\,$ 1 $\,\,$ & $\,\,$\color{red} 3.9904\color{black}  $\,\,$ \\
$\,\,$0.1176$\,\,$ & $\,\,$0.2420$\,\,$ & $\,\,$\color{red} 0.2506\color{black} $\,\,$ & $\,\,$ 1  $\,\,$ \\
\end{pmatrix},
\end{equation*}

\begin{equation*}
\mathbf{w}^{\prime} =
\begin{pmatrix}
0.482080\\
0.234364\\
0.226845\\
0.056711
\end{pmatrix} =
0.999458\cdot
\begin{pmatrix}
0.482342\\
0.234491\\
\color{gr} 0.226968\color{black} \\
0.056742
\end{pmatrix},
\end{equation*}
\begin{equation*}
\left[ \frac{{w}^{\prime}_i}{{w}^{\prime}_j} \right] =
\begin{pmatrix}
$\,\,$ 1 $\,\,$ & $\,\,$2.0570$\,\,$ & $\,\,$\color{gr} 2.1252\color{black} $\,\,$ & $\,\,$8.5006$\,\,$ \\
$\,\,$0.4862$\,\,$ & $\,\,$ 1 $\,\,$ & $\,\,$\color{gr} 1.0331\color{black} $\,\,$ & $\,\,$4.1326  $\,\,$ \\
$\,\,$\color{gr} 0.4706\color{black} $\,\,$ & $\,\,$\color{gr} 0.9679\color{black} $\,\,$ & $\,\,$ 1 $\,\,$ & $\,\,$\color{gr} \color{blue} 4\color{black}  $\,\,$ \\
$\,\,$0.1176$\,\,$ & $\,\,$0.2420$\,\,$ & $\,\,$\color{gr} \color{blue}  1/4\color{black} $\,\,$ & $\,\,$ 1  $\,\,$ \\
\end{pmatrix},
\end{equation*}
\end{example}
\newpage
\begin{example}
\begin{equation*}
\mathbf{A} =
\begin{pmatrix}
$\,\,$ 1 $\,\,$ & $\,\,$3$\,\,$ & $\,\,$2$\,\,$ & $\,\,$6 $\,\,$ \\
$\,\,$ 1/3$\,\,$ & $\,\,$ 1 $\,\,$ & $\,\,$1$\,\,$ & $\,\,$7 $\,\,$ \\
$\,\,$ 1/2$\,\,$ & $\,\,$ 1 $\,\,$ & $\,\,$ 1 $\,\,$ & $\,\,$5 $\,\,$ \\
$\,\,$ 1/6$\,\,$ & $\,\,$ 1/7$\,\,$ & $\,\,$ 1/5$\,\,$ & $\,\,$ 1  $\,\,$ \\
\end{pmatrix},
\qquad
\lambda_{\max} =
4.1351,
\qquad
CR = 0.0509
\end{equation*}

\begin{equation*}
\mathbf{w}^{EM} =
\begin{pmatrix}
0.475884\\
0.239272\\
\color{red} 0.233732\color{black} \\
0.051112
\end{pmatrix}\end{equation*}
\begin{equation*}
\left[ \frac{{w}^{EM}_i}{{w}^{EM}_j} \right] =
\begin{pmatrix}
$\,\,$ 1 $\,\,$ & $\,\,$1.9889$\,\,$ & $\,\,$\color{red} 2.0360\color{black} $\,\,$ & $\,\,$9.3106$\,\,$ \\
$\,\,$0.5028$\,\,$ & $\,\,$ 1 $\,\,$ & $\,\,$\color{red} 1.0237\color{black} $\,\,$ & $\,\,$4.6813  $\,\,$ \\
$\,\,$\color{red} 0.4912\color{black} $\,\,$ & $\,\,$\color{red} 0.9768\color{black} $\,\,$ & $\,\,$ 1 $\,\,$ & $\,\,$\color{red} 4.5729\color{black}  $\,\,$ \\
$\,\,$0.1074$\,\,$ & $\,\,$0.2136$\,\,$ & $\,\,$\color{red} 0.2187\color{black} $\,\,$ & $\,\,$ 1  $\,\,$ \\
\end{pmatrix},
\end{equation*}

\begin{equation*}
\mathbf{w}^{\prime} =
\begin{pmatrix}
0.473889\\
0.238269\\
0.236945\\
0.050898
\end{pmatrix} =
0.995807\cdot
\begin{pmatrix}
0.475884\\
0.239272\\
\color{gr} 0.237942\color{black} \\
0.051112
\end{pmatrix},
\end{equation*}
\begin{equation*}
\left[ \frac{{w}^{\prime}_i}{{w}^{\prime}_j} \right] =
\begin{pmatrix}
$\,\,$ 1 $\,\,$ & $\,\,$1.9889$\,\,$ & $\,\,$\color{gr} \color{blue} 2\color{black} $\,\,$ & $\,\,$9.3106$\,\,$ \\
$\,\,$0.5028$\,\,$ & $\,\,$ 1 $\,\,$ & $\,\,$\color{gr} 1.0056\color{black} $\,\,$ & $\,\,$4.6813  $\,\,$ \\
$\,\,$\color{gr} \color{blue}  1/2\color{black} $\,\,$ & $\,\,$\color{gr} 0.9944\color{black} $\,\,$ & $\,\,$ 1 $\,\,$ & $\,\,$\color{gr} 4.6553\color{black}  $\,\,$ \\
$\,\,$0.1074$\,\,$ & $\,\,$0.2136$\,\,$ & $\,\,$\color{gr} 0.2148\color{black} $\,\,$ & $\,\,$ 1  $\,\,$ \\
\end{pmatrix},
\end{equation*}
\end{example}
\newpage
\begin{example}
\begin{equation*}
\mathbf{A} =
\begin{pmatrix}
$\,\,$ 1 $\,\,$ & $\,\,$3$\,\,$ & $\,\,$2$\,\,$ & $\,\,$6 $\,\,$ \\
$\,\,$ 1/3$\,\,$ & $\,\,$ 1 $\,\,$ & $\,\,$1$\,\,$ & $\,\,$8 $\,\,$ \\
$\,\,$ 1/2$\,\,$ & $\,\,$ 1 $\,\,$ & $\,\,$ 1 $\,\,$ & $\,\,$5 $\,\,$ \\
$\,\,$ 1/6$\,\,$ & $\,\,$ 1/8$\,\,$ & $\,\,$ 1/5$\,\,$ & $\,\,$ 1  $\,\,$ \\
\end{pmatrix},
\qquad
\lambda_{\max} =
4.1655,
\qquad
CR = 0.0624
\end{equation*}

\begin{equation*}
\mathbf{w}^{EM} =
\begin{pmatrix}
0.473138\\
0.247078\\
\color{red} 0.230550\color{black} \\
0.049234
\end{pmatrix}\end{equation*}
\begin{equation*}
\left[ \frac{{w}^{EM}_i}{{w}^{EM}_j} \right] =
\begin{pmatrix}
$\,\,$ 1 $\,\,$ & $\,\,$1.9149$\,\,$ & $\,\,$\color{red} 2.0522\color{black} $\,\,$ & $\,\,$9.6100$\,\,$ \\
$\,\,$0.5222$\,\,$ & $\,\,$ 1 $\,\,$ & $\,\,$\color{red} 1.0717\color{black} $\,\,$ & $\,\,$5.0185  $\,\,$ \\
$\,\,$\color{red} 0.4873\color{black} $\,\,$ & $\,\,$\color{red} 0.9331\color{black} $\,\,$ & $\,\,$ 1 $\,\,$ & $\,\,$\color{red} 4.6828\color{black}  $\,\,$ \\
$\,\,$0.1041$\,\,$ & $\,\,$0.1993$\,\,$ & $\,\,$\color{red} 0.2135\color{black} $\,\,$ & $\,\,$ 1  $\,\,$ \\
\end{pmatrix},
\end{equation*}

\begin{equation*}
\mathbf{w}^{\prime} =
\begin{pmatrix}
0.470308\\
0.245599\\
0.235154\\
0.048939
\end{pmatrix} =
0.994017\cdot
\begin{pmatrix}
0.473138\\
0.247078\\
\color{gr} 0.236569\color{black} \\
0.049234
\end{pmatrix},
\end{equation*}
\begin{equation*}
\left[ \frac{{w}^{\prime}_i}{{w}^{\prime}_j} \right] =
\begin{pmatrix}
$\,\,$ 1 $\,\,$ & $\,\,$1.9149$\,\,$ & $\,\,$\color{gr} \color{blue} 2\color{black} $\,\,$ & $\,\,$9.6100$\,\,$ \\
$\,\,$0.5222$\,\,$ & $\,\,$ 1 $\,\,$ & $\,\,$\color{gr} 1.0444\color{black} $\,\,$ & $\,\,$5.0185  $\,\,$ \\
$\,\,$\color{gr} \color{blue}  1/2\color{black} $\,\,$ & $\,\,$\color{gr} 0.9575\color{black} $\,\,$ & $\,\,$ 1 $\,\,$ & $\,\,$\color{gr} 4.8050\color{black}  $\,\,$ \\
$\,\,$0.1041$\,\,$ & $\,\,$0.1993$\,\,$ & $\,\,$\color{gr} 0.2081\color{black} $\,\,$ & $\,\,$ 1  $\,\,$ \\
\end{pmatrix},
\end{equation*}
\end{example}
\newpage
\begin{example}
\begin{equation*}
\mathbf{A} =
\begin{pmatrix}
$\,\,$ 1 $\,\,$ & $\,\,$3$\,\,$ & $\,\,$2$\,\,$ & $\,\,$6 $\,\,$ \\
$\,\,$ 1/3$\,\,$ & $\,\,$ 1 $\,\,$ & $\,\,$1$\,\,$ & $\,\,$9 $\,\,$ \\
$\,\,$ 1/2$\,\,$ & $\,\,$ 1 $\,\,$ & $\,\,$ 1 $\,\,$ & $\,\,$5 $\,\,$ \\
$\,\,$ 1/6$\,\,$ & $\,\,$ 1/9$\,\,$ & $\,\,$ 1/5$\,\,$ & $\,\,$ 1  $\,\,$ \\
\end{pmatrix},
\qquad
\lambda_{\max} =
4.1966,
\qquad
CR = 0.0741
\end{equation*}

\begin{equation*}
\mathbf{w}^{EM} =
\begin{pmatrix}
0.470457\\
0.254315\\
\color{red} 0.227618\color{black} \\
0.047611
\end{pmatrix}\end{equation*}
\begin{equation*}
\left[ \frac{{w}^{EM}_i}{{w}^{EM}_j} \right] =
\begin{pmatrix}
$\,\,$ 1 $\,\,$ & $\,\,$1.8499$\,\,$ & $\,\,$\color{red} 2.0669\color{black} $\,\,$ & $\,\,$9.8813$\,\,$ \\
$\,\,$0.5406$\,\,$ & $\,\,$ 1 $\,\,$ & $\,\,$\color{red} 1.1173\color{black} $\,\,$ & $\,\,$5.3416  $\,\,$ \\
$\,\,$\color{red} 0.4838\color{black} $\,\,$ & $\,\,$\color{red} 0.8950\color{black} $\,\,$ & $\,\,$ 1 $\,\,$ & $\,\,$\color{red} 4.7808\color{black}  $\,\,$ \\
$\,\,$0.1012$\,\,$ & $\,\,$0.1872$\,\,$ & $\,\,$\color{red} 0.2092\color{black} $\,\,$ & $\,\,$ 1  $\,\,$ \\
\end{pmatrix},
\end{equation*}

\begin{equation*}
\mathbf{w}^{\prime} =
\begin{pmatrix}
0.466903\\
0.252394\\
0.233452\\
0.047251
\end{pmatrix} =
0.992447\cdot
\begin{pmatrix}
0.470457\\
0.254315\\
\color{gr} 0.235228\color{black} \\
0.047611
\end{pmatrix},
\end{equation*}
\begin{equation*}
\left[ \frac{{w}^{\prime}_i}{{w}^{\prime}_j} \right] =
\begin{pmatrix}
$\,\,$ 1 $\,\,$ & $\,\,$1.8499$\,\,$ & $\,\,$\color{gr} \color{blue} 2\color{black} $\,\,$ & $\,\,$9.8813$\,\,$ \\
$\,\,$0.5406$\,\,$ & $\,\,$ 1 $\,\,$ & $\,\,$\color{gr} 1.0811\color{black} $\,\,$ & $\,\,$5.3416  $\,\,$ \\
$\,\,$\color{gr} \color{blue}  1/2\color{black} $\,\,$ & $\,\,$\color{gr} 0.9250\color{black} $\,\,$ & $\,\,$ 1 $\,\,$ & $\,\,$\color{gr} 4.9407\color{black}  $\,\,$ \\
$\,\,$0.1012$\,\,$ & $\,\,$0.1872$\,\,$ & $\,\,$\color{gr} 0.2024\color{black} $\,\,$ & $\,\,$ 1  $\,\,$ \\
\end{pmatrix},
\end{equation*}
\end{example}
\newpage
\begin{example}
\begin{equation*}
\mathbf{A} =
\begin{pmatrix}
$\,\,$ 1 $\,\,$ & $\,\,$3$\,\,$ & $\,\,$2$\,\,$ & $\,\,$6 $\,\,$ \\
$\,\,$ 1/3$\,\,$ & $\,\,$ 1 $\,\,$ & $\,\,$2$\,\,$ & $\,\,$3 $\,\,$ \\
$\,\,$ 1/2$\,\,$ & $\,\,$ 1/2$\,\,$ & $\,\,$ 1 $\,\,$ & $\,\,$2 $\,\,$ \\
$\,\,$ 1/6$\,\,$ & $\,\,$ 1/3$\,\,$ & $\,\,$ 1/2$\,\,$ & $\,\,$ 1  $\,\,$ \\
\end{pmatrix},
\qquad
\lambda_{\max} =
4.1031,
\qquad
CR = 0.0389
\end{equation*}

\begin{equation*}
\mathbf{w}^{EM} =
\begin{pmatrix}
0.503088\\
0.243474\\
0.172473\\
\color{red} 0.080964\color{black}
\end{pmatrix}\end{equation*}
\begin{equation*}
\left[ \frac{{w}^{EM}_i}{{w}^{EM}_j} \right] =
\begin{pmatrix}
$\,\,$ 1 $\,\,$ & $\,\,$2.0663$\,\,$ & $\,\,$2.9169$\,\,$ & $\,\,$\color{red} 6.2137\color{black} $\,\,$ \\
$\,\,$0.4840$\,\,$ & $\,\,$ 1 $\,\,$ & $\,\,$1.4117$\,\,$ & $\,\,$\color{red} 3.0072\color{black}   $\,\,$ \\
$\,\,$0.3428$\,\,$ & $\,\,$0.7084$\,\,$ & $\,\,$ 1 $\,\,$ & $\,\,$\color{red} 2.1302\color{black}  $\,\,$ \\
$\,\,$\color{red} 0.1609\color{black} $\,\,$ & $\,\,$\color{red} 0.3325\color{black} $\,\,$ & $\,\,$\color{red} 0.4694\color{black} $\,\,$ & $\,\,$ 1  $\,\,$ \\
\end{pmatrix},
\end{equation*}

\begin{equation*}
\mathbf{w}^{\prime} =
\begin{pmatrix}
0.502991\\
0.243427\\
0.172440\\
0.081142
\end{pmatrix} =
0.999806\cdot
\begin{pmatrix}
0.503088\\
0.243474\\
0.172473\\
\color{gr} 0.081158\color{black}
\end{pmatrix},
\end{equation*}
\begin{equation*}
\left[ \frac{{w}^{\prime}_i}{{w}^{\prime}_j} \right] =
\begin{pmatrix}
$\,\,$ 1 $\,\,$ & $\,\,$2.0663$\,\,$ & $\,\,$2.9169$\,\,$ & $\,\,$\color{gr} 6.1989\color{black} $\,\,$ \\
$\,\,$0.4840$\,\,$ & $\,\,$ 1 $\,\,$ & $\,\,$1.4117$\,\,$ & $\,\,$\color{gr} \color{blue} 3\color{black}   $\,\,$ \\
$\,\,$0.3428$\,\,$ & $\,\,$0.7084$\,\,$ & $\,\,$ 1 $\,\,$ & $\,\,$\color{gr} 2.1252\color{black}  $\,\,$ \\
$\,\,$\color{gr} 0.1613\color{black} $\,\,$ & $\,\,$\color{gr} \color{blue}  1/3\color{black} $\,\,$ & $\,\,$\color{gr} 0.4706\color{black} $\,\,$ & $\,\,$ 1  $\,\,$ \\
\end{pmatrix},
\end{equation*}
\end{example}
\newpage
\begin{example}
\begin{equation*}
\mathbf{A} =
\begin{pmatrix}
$\,\,$ 1 $\,\,$ & $\,\,$3$\,\,$ & $\,\,$2$\,\,$ & $\,\,$7 $\,\,$ \\
$\,\,$ 1/3$\,\,$ & $\,\,$ 1 $\,\,$ & $\,\,$1$\,\,$ & $\,\,$7 $\,\,$ \\
$\,\,$ 1/2$\,\,$ & $\,\,$ 1 $\,\,$ & $\,\,$ 1 $\,\,$ & $\,\,$5 $\,\,$ \\
$\,\,$ 1/7$\,\,$ & $\,\,$ 1/7$\,\,$ & $\,\,$ 1/5$\,\,$ & $\,\,$ 1  $\,\,$ \\
\end{pmatrix},
\qquad
\lambda_{\max} =
4.1027,
\qquad
CR = 0.0387
\end{equation*}

\begin{equation*}
\mathbf{w}^{EM} =
\begin{pmatrix}
0.485172\\
0.235235\\
\color{red} 0.231501\color{black} \\
0.048092
\end{pmatrix}\end{equation*}
\begin{equation*}
\left[ \frac{{w}^{EM}_i}{{w}^{EM}_j} \right] =
\begin{pmatrix}
$\,\,$ 1 $\,\,$ & $\,\,$2.0625$\,\,$ & $\,\,$\color{red} 2.0958\color{black} $\,\,$ & $\,\,$10.0884$\,\,$ \\
$\,\,$0.4848$\,\,$ & $\,\,$ 1 $\,\,$ & $\,\,$\color{red} 1.0161\color{black} $\,\,$ & $\,\,$4.8914  $\,\,$ \\
$\,\,$\color{red} 0.4772\color{black} $\,\,$ & $\,\,$\color{red} 0.9841\color{black} $\,\,$ & $\,\,$ 1 $\,\,$ & $\,\,$\color{red} 4.8137\color{black}  $\,\,$ \\
$\,\,$0.0991$\,\,$ & $\,\,$0.2044$\,\,$ & $\,\,$\color{red} 0.2077\color{black} $\,\,$ & $\,\,$ 1  $\,\,$ \\
\end{pmatrix},
\end{equation*}

\begin{equation*}
\mathbf{w}^{\prime} =
\begin{pmatrix}
0.483367\\
0.234360\\
0.234360\\
0.047913
\end{pmatrix} =
0.996279\cdot
\begin{pmatrix}
0.485172\\
0.235235\\
\color{gr} 0.235235\color{black} \\
0.048092
\end{pmatrix},
\end{equation*}
\begin{equation*}
\left[ \frac{{w}^{\prime}_i}{{w}^{\prime}_j} \right] =
\begin{pmatrix}
$\,\,$ 1 $\,\,$ & $\,\,$2.0625$\,\,$ & $\,\,$\color{gr} 2.0625\color{black} $\,\,$ & $\,\,$10.0884$\,\,$ \\
$\,\,$0.4848$\,\,$ & $\,\,$ 1 $\,\,$ & $\,\,$\color{gr} \color{blue} 1\color{black} $\,\,$ & $\,\,$4.8914  $\,\,$ \\
$\,\,$\color{gr} 0.4848\color{black} $\,\,$ & $\,\,$\color{gr} \color{blue} 1\color{black} $\,\,$ & $\,\,$ 1 $\,\,$ & $\,\,$\color{gr} 4.8914\color{black}  $\,\,$ \\
$\,\,$0.0991$\,\,$ & $\,\,$0.2044$\,\,$ & $\,\,$\color{gr} 0.2044\color{black} $\,\,$ & $\,\,$ 1  $\,\,$ \\
\end{pmatrix},
\end{equation*}
\end{example}
\newpage
\begin{example}
\begin{equation*}
\mathbf{A} =
\begin{pmatrix}
$\,\,$ 1 $\,\,$ & $\,\,$3$\,\,$ & $\,\,$2$\,\,$ & $\,\,$7 $\,\,$ \\
$\,\,$ 1/3$\,\,$ & $\,\,$ 1 $\,\,$ & $\,\,$1$\,\,$ & $\,\,$8 $\,\,$ \\
$\,\,$ 1/2$\,\,$ & $\,\,$ 1 $\,\,$ & $\,\,$ 1 $\,\,$ & $\,\,$5 $\,\,$ \\
$\,\,$ 1/7$\,\,$ & $\,\,$ 1/8$\,\,$ & $\,\,$ 1/5$\,\,$ & $\,\,$ 1  $\,\,$ \\
\end{pmatrix},
\qquad
\lambda_{\max} =
4.1301,
\qquad
CR = 0.0490
\end{equation*}

\begin{equation*}
\mathbf{w}^{EM} =
\begin{pmatrix}
0.482323\\
0.242770\\
\color{red} 0.228592\color{black} \\
0.046315
\end{pmatrix}\end{equation*}
\begin{equation*}
\left[ \frac{{w}^{EM}_i}{{w}^{EM}_j} \right] =
\begin{pmatrix}
$\,\,$ 1 $\,\,$ & $\,\,$1.9867$\,\,$ & $\,\,$\color{red} 2.1100\color{black} $\,\,$ & $\,\,$10.4140$\,\,$ \\
$\,\,$0.5033$\,\,$ & $\,\,$ 1 $\,\,$ & $\,\,$\color{red} 1.0620\color{black} $\,\,$ & $\,\,$5.2417  $\,\,$ \\
$\,\,$\color{red} 0.4739\color{black} $\,\,$ & $\,\,$\color{red} 0.9416\color{black} $\,\,$ & $\,\,$ 1 $\,\,$ & $\,\,$\color{red} 4.9356\color{black}  $\,\,$ \\
$\,\,$0.0960$\,\,$ & $\,\,$0.1908$\,\,$ & $\,\,$\color{red} 0.2026\color{black} $\,\,$ & $\,\,$ 1  $\,\,$ \\
\end{pmatrix},
\end{equation*}

\begin{equation*}
\mathbf{w}^{\prime} =
\begin{pmatrix}
0.480889\\
0.242048\\
0.230885\\
0.046177
\end{pmatrix} =
0.997027\cdot
\begin{pmatrix}
0.482323\\
0.242770\\
\color{gr} 0.231574\color{black} \\
0.046315
\end{pmatrix},
\end{equation*}
\begin{equation*}
\left[ \frac{{w}^{\prime}_i}{{w}^{\prime}_j} \right] =
\begin{pmatrix}
$\,\,$ 1 $\,\,$ & $\,\,$1.9867$\,\,$ & $\,\,$\color{gr} 2.0828\color{black} $\,\,$ & $\,\,$10.4140$\,\,$ \\
$\,\,$0.5033$\,\,$ & $\,\,$ 1 $\,\,$ & $\,\,$\color{gr} 1.0483\color{black} $\,\,$ & $\,\,$5.2417  $\,\,$ \\
$\,\,$\color{gr} 0.4801\color{black} $\,\,$ & $\,\,$\color{gr} 0.9539\color{black} $\,\,$ & $\,\,$ 1 $\,\,$ & $\,\,$\color{gr} \color{blue} 5\color{black}  $\,\,$ \\
$\,\,$0.0960$\,\,$ & $\,\,$0.1908$\,\,$ & $\,\,$\color{gr} \color{blue}  1/5\color{black} $\,\,$ & $\,\,$ 1  $\,\,$ \\
\end{pmatrix},
\end{equation*}
\end{example}
\newpage
\begin{example}
\begin{equation*}
\mathbf{A} =
\begin{pmatrix}
$\,\,$ 1 $\,\,$ & $\,\,$3$\,\,$ & $\,\,$2$\,\,$ & $\,\,$7 $\,\,$ \\
$\,\,$ 1/3$\,\,$ & $\,\,$ 1 $\,\,$ & $\,\,$1$\,\,$ & $\,\,$8 $\,\,$ \\
$\,\,$ 1/2$\,\,$ & $\,\,$ 1 $\,\,$ & $\,\,$ 1 $\,\,$ & $\,\,$6 $\,\,$ \\
$\,\,$ 1/7$\,\,$ & $\,\,$ 1/8$\,\,$ & $\,\,$ 1/6$\,\,$ & $\,\,$ 1  $\,\,$ \\
\end{pmatrix},
\qquad
\lambda_{\max} =
4.1317,
\qquad
CR = 0.0496
\end{equation*}

\begin{equation*}
\mathbf{w}^{EM} =
\begin{pmatrix}
0.479292\\
0.239324\\
\color{red} 0.237336\color{black} \\
0.044048
\end{pmatrix}\end{equation*}
\begin{equation*}
\left[ \frac{{w}^{EM}_i}{{w}^{EM}_j} \right] =
\begin{pmatrix}
$\,\,$ 1 $\,\,$ & $\,\,$2.0027$\,\,$ & $\,\,$\color{red} 2.0195\color{black} $\,\,$ & $\,\,$10.8812$\,\,$ \\
$\,\,$0.4993$\,\,$ & $\,\,$ 1 $\,\,$ & $\,\,$\color{red} 1.0084\color{black} $\,\,$ & $\,\,$5.4333  $\,\,$ \\
$\,\,$\color{red} 0.4952\color{black} $\,\,$ & $\,\,$\color{red} 0.9917\color{black} $\,\,$ & $\,\,$ 1 $\,\,$ & $\,\,$\color{red} 5.3882\color{black}  $\,\,$ \\
$\,\,$0.0919$\,\,$ & $\,\,$0.1840$\,\,$ & $\,\,$\color{red} 0.1856\color{black} $\,\,$ & $\,\,$ 1  $\,\,$ \\
\end{pmatrix},
\end{equation*}

\begin{equation*}
\mathbf{w}^{\prime} =
\begin{pmatrix}
0.478341\\
0.238849\\
0.238849\\
0.043960
\end{pmatrix} =
0.998016\cdot
\begin{pmatrix}
0.479292\\
0.239324\\
\color{gr} 0.239324\color{black} \\
0.044048
\end{pmatrix},
\end{equation*}
\begin{equation*}
\left[ \frac{{w}^{\prime}_i}{{w}^{\prime}_j} \right] =
\begin{pmatrix}
$\,\,$ 1 $\,\,$ & $\,\,$2.0027$\,\,$ & $\,\,$\color{gr} 2.0027\color{black} $\,\,$ & $\,\,$10.8812$\,\,$ \\
$\,\,$0.4993$\,\,$ & $\,\,$ 1 $\,\,$ & $\,\,$\color{gr} \color{blue} 1\color{black} $\,\,$ & $\,\,$5.4333  $\,\,$ \\
$\,\,$\color{gr} 0.4993\color{black} $\,\,$ & $\,\,$\color{gr} \color{blue} 1\color{black} $\,\,$ & $\,\,$ 1 $\,\,$ & $\,\,$\color{gr} 5.4333\color{black}  $\,\,$ \\
$\,\,$0.0919$\,\,$ & $\,\,$0.1840$\,\,$ & $\,\,$\color{gr} 0.1840\color{black} $\,\,$ & $\,\,$ 1  $\,\,$ \\
\end{pmatrix},
\end{equation*}
\end{example}
\newpage
\begin{example}
\begin{equation*}
\mathbf{A} =
\begin{pmatrix}
$\,\,$ 1 $\,\,$ & $\,\,$3$\,\,$ & $\,\,$2$\,\,$ & $\,\,$7 $\,\,$ \\
$\,\,$ 1/3$\,\,$ & $\,\,$ 1 $\,\,$ & $\,\,$1$\,\,$ & $\,\,$9 $\,\,$ \\
$\,\,$ 1/2$\,\,$ & $\,\,$ 1 $\,\,$ & $\,\,$ 1 $\,\,$ & $\,\,$6 $\,\,$ \\
$\,\,$ 1/7$\,\,$ & $\,\,$ 1/9$\,\,$ & $\,\,$ 1/6$\,\,$ & $\,\,$ 1  $\,\,$ \\
\end{pmatrix},
\qquad
\lambda_{\max} =
4.1571,
\qquad
CR = 0.0593
\end{equation*}

\begin{equation*}
\mathbf{w}^{EM} =
\begin{pmatrix}
0.476846\\
0.246087\\
\color{red} 0.234452\color{black} \\
0.042614
\end{pmatrix}\end{equation*}
\begin{equation*}
\left[ \frac{{w}^{EM}_i}{{w}^{EM}_j} \right] =
\begin{pmatrix}
$\,\,$ 1 $\,\,$ & $\,\,$1.9377$\,\,$ & $\,\,$\color{red} 2.0339\color{black} $\,\,$ & $\,\,$11.1898$\,\,$ \\
$\,\,$0.5161$\,\,$ & $\,\,$ 1 $\,\,$ & $\,\,$\color{red} 1.0496\color{black} $\,\,$ & $\,\,$5.7747  $\,\,$ \\
$\,\,$\color{red} 0.4917\color{black} $\,\,$ & $\,\,$\color{red} 0.9527\color{black} $\,\,$ & $\,\,$ 1 $\,\,$ & $\,\,$\color{red} 5.5017\color{black}  $\,\,$ \\
$\,\,$0.0894$\,\,$ & $\,\,$0.1732$\,\,$ & $\,\,$\color{red} 0.1818\color{black} $\,\,$ & $\,\,$ 1  $\,\,$ \\
\end{pmatrix},
\end{equation*}

\begin{equation*}
\mathbf{w}^{\prime} =
\begin{pmatrix}
0.474960\\
0.245114\\
0.237480\\
0.042446
\end{pmatrix} =
0.996045\cdot
\begin{pmatrix}
0.476846\\
0.246087\\
\color{gr} 0.238423\color{black} \\
0.042614
\end{pmatrix},
\end{equation*}
\begin{equation*}
\left[ \frac{{w}^{\prime}_i}{{w}^{\prime}_j} \right] =
\begin{pmatrix}
$\,\,$ 1 $\,\,$ & $\,\,$1.9377$\,\,$ & $\,\,$\color{gr} \color{blue} 2\color{black} $\,\,$ & $\,\,$11.1898$\,\,$ \\
$\,\,$0.5161$\,\,$ & $\,\,$ 1 $\,\,$ & $\,\,$\color{gr} 1.0321\color{black} $\,\,$ & $\,\,$5.7747  $\,\,$ \\
$\,\,$\color{gr} \color{blue}  1/2\color{black} $\,\,$ & $\,\,$\color{gr} 0.9689\color{black} $\,\,$ & $\,\,$ 1 $\,\,$ & $\,\,$\color{gr} 5.5949\color{black}  $\,\,$ \\
$\,\,$0.0894$\,\,$ & $\,\,$0.1732$\,\,$ & $\,\,$\color{gr} 0.1787\color{black} $\,\,$ & $\,\,$ 1  $\,\,$ \\
\end{pmatrix},
\end{equation*}
\end{example}
\newpage
\begin{example}
\begin{equation*}
\mathbf{A} =
\begin{pmatrix}
$\,\,$ 1 $\,\,$ & $\,\,$3$\,\,$ & $\,\,$2$\,\,$ & $\,\,$7 $\,\,$ \\
$\,\,$ 1/3$\,\,$ & $\,\,$ 1 $\,\,$ & $\,\,$3$\,\,$ & $\,\,$4 $\,\,$ \\
$\,\,$ 1/2$\,\,$ & $\,\,$ 1/3$\,\,$ & $\,\,$ 1 $\,\,$ & $\,\,$2 $\,\,$ \\
$\,\,$ 1/7$\,\,$ & $\,\,$ 1/4$\,\,$ & $\,\,$ 1/2$\,\,$ & $\,\,$ 1  $\,\,$ \\
\end{pmatrix},
\qquad
\lambda_{\max} =
4.1964,
\qquad
CR = 0.0741
\end{equation*}

\begin{equation*}
\mathbf{w}^{EM} =
\begin{pmatrix}
0.503663\\
0.278336\\
0.150222\\
\color{red} 0.067779\color{black}
\end{pmatrix}\end{equation*}
\begin{equation*}
\left[ \frac{{w}^{EM}_i}{{w}^{EM}_j} \right] =
\begin{pmatrix}
$\,\,$ 1 $\,\,$ & $\,\,$1.8096$\,\,$ & $\,\,$3.3528$\,\,$ & $\,\,$\color{red} 7.4310\color{black} $\,\,$ \\
$\,\,$0.5526$\,\,$ & $\,\,$ 1 $\,\,$ & $\,\,$1.8528$\,\,$ & $\,\,$\color{red} 4.1065\color{black}   $\,\,$ \\
$\,\,$0.2983$\,\,$ & $\,\,$0.5397$\,\,$ & $\,\,$ 1 $\,\,$ & $\,\,$\color{red} 2.2164\color{black}  $\,\,$ \\
$\,\,$\color{red} 0.1346\color{black} $\,\,$ & $\,\,$\color{red} 0.2435\color{black} $\,\,$ & $\,\,$\color{red} 0.4512\color{black} $\,\,$ & $\,\,$ 1  $\,\,$ \\
\end{pmatrix},
\end{equation*}

\begin{equation*}
\mathbf{w}^{\prime} =
\begin{pmatrix}
0.502756\\
0.277834\\
0.149951\\
0.069459
\end{pmatrix} =
0.998198\cdot
\begin{pmatrix}
0.503663\\
0.278336\\
0.150222\\
\color{gr} 0.069584\color{black}
\end{pmatrix},
\end{equation*}
\begin{equation*}
\left[ \frac{{w}^{\prime}_i}{{w}^{\prime}_j} \right] =
\begin{pmatrix}
$\,\,$ 1 $\,\,$ & $\,\,$1.8096$\,\,$ & $\,\,$3.3528$\,\,$ & $\,\,$\color{gr} 7.2382\color{black} $\,\,$ \\
$\,\,$0.5526$\,\,$ & $\,\,$ 1 $\,\,$ & $\,\,$1.8528$\,\,$ & $\,\,$\color{gr} \color{blue} 4\color{black}   $\,\,$ \\
$\,\,$0.2983$\,\,$ & $\,\,$0.5397$\,\,$ & $\,\,$ 1 $\,\,$ & $\,\,$\color{gr} 2.1589\color{black}  $\,\,$ \\
$\,\,$\color{gr} 0.1382\color{black} $\,\,$ & $\,\,$\color{gr} \color{blue}  1/4\color{black} $\,\,$ & $\,\,$\color{gr} 0.4632\color{black} $\,\,$ & $\,\,$ 1  $\,\,$ \\
\end{pmatrix},
\end{equation*}
\end{example}
\newpage
\begin{example}
\begin{equation*}
\mathbf{A} =
\begin{pmatrix}
$\,\,$ 1 $\,\,$ & $\,\,$3$\,\,$ & $\,\,$2$\,\,$ & $\,\,$8 $\,\,$ \\
$\,\,$ 1/3$\,\,$ & $\,\,$ 1 $\,\,$ & $\,\,$1$\,\,$ & $\,\,$8 $\,\,$ \\
$\,\,$ 1/2$\,\,$ & $\,\,$ 1 $\,\,$ & $\,\,$ 1 $\,\,$ & $\,\,$6 $\,\,$ \\
$\,\,$ 1/8$\,\,$ & $\,\,$ 1/8$\,\,$ & $\,\,$ 1/6$\,\,$ & $\,\,$ 1  $\,\,$ \\
\end{pmatrix},
\qquad
\lambda_{\max} =
4.1031,
\qquad
CR = 0.0389
\end{equation*}

\begin{equation*}
\mathbf{w}^{EM} =
\begin{pmatrix}
0.487218\\
0.235794\\
\color{red} 0.235230\color{black} \\
0.041758
\end{pmatrix}\end{equation*}
\begin{equation*}
\left[ \frac{{w}^{EM}_i}{{w}^{EM}_j} \right] =
\begin{pmatrix}
$\,\,$ 1 $\,\,$ & $\,\,$2.0663$\,\,$ & $\,\,$\color{red} 2.0712\color{black} $\,\,$ & $\,\,$11.6676$\,\,$ \\
$\,\,$0.4840$\,\,$ & $\,\,$ 1 $\,\,$ & $\,\,$\color{red} 1.0024\color{black} $\,\,$ & $\,\,$5.6467  $\,\,$ \\
$\,\,$\color{red} 0.4828\color{black} $\,\,$ & $\,\,$\color{red} 0.9976\color{black} $\,\,$ & $\,\,$ 1 $\,\,$ & $\,\,$\color{red} 5.6331\color{black}  $\,\,$ \\
$\,\,$0.0857$\,\,$ & $\,\,$0.1771$\,\,$ & $\,\,$\color{red} 0.1775\color{black} $\,\,$ & $\,\,$ 1  $\,\,$ \\
\end{pmatrix},
\end{equation*}

\begin{equation*}
\mathbf{w}^{\prime} =
\begin{pmatrix}
0.486944\\
0.235661\\
0.235661\\
0.041735
\end{pmatrix} =
0.999437\cdot
\begin{pmatrix}
0.487218\\
0.235794\\
\color{gr} 0.235794\color{black} \\
0.041758
\end{pmatrix},
\end{equation*}
\begin{equation*}
\left[ \frac{{w}^{\prime}_i}{{w}^{\prime}_j} \right] =
\begin{pmatrix}
$\,\,$ 1 $\,\,$ & $\,\,$2.0663$\,\,$ & $\,\,$\color{gr} 2.0663\color{black} $\,\,$ & $\,\,$11.6676$\,\,$ \\
$\,\,$0.4840$\,\,$ & $\,\,$ 1 $\,\,$ & $\,\,$\color{gr} \color{blue} 1\color{black} $\,\,$ & $\,\,$5.6467  $\,\,$ \\
$\,\,$\color{gr} 0.4840\color{black} $\,\,$ & $\,\,$\color{gr} \color{blue} 1\color{black} $\,\,$ & $\,\,$ 1 $\,\,$ & $\,\,$\color{gr} 5.6467\color{black}  $\,\,$ \\
$\,\,$0.0857$\,\,$ & $\,\,$0.1771$\,\,$ & $\,\,$\color{gr} 0.1771\color{black} $\,\,$ & $\,\,$ 1  $\,\,$ \\
\end{pmatrix},
\end{equation*}
\end{example}
\newpage
\begin{example}   % Example 1.139
\begin{equation*}
\mathbf{A} =
\begin{pmatrix}
$\,\,$ 1 $\,\,$ & $\,\,$3$\,\,$ & $\,\,$2$\,\,$ & $\,\,$8 $\,\,$ \\
$\,\,$ 1/3$\,\,$ & $\,\,$ 1 $\,\,$ & $\,\,$1$\,\,$ & $\,\,$9 $\,\,$ \\
$\,\,$ 1/2$\,\,$ & $\,\,$ 1 $\,\,$ & $\,\,$ 1 $\,\,$ & $\,\,$6 $\,\,$ \\
$\,\,$ 1/8$\,\,$ & $\,\,$ 1/9$\,\,$ & $\,\,$ 1/6$\,\,$ & $\,\,$ 1  $\,\,$ \\
\end{pmatrix},
\qquad
\lambda_{\max} =
4.1263,
\qquad
CR = 0.0476
\end{equation*}

\begin{equation*}
\mathbf{w}^{EM} =
\begin{pmatrix}
0.484699\\
0.242349\\
\color{red} 0.232560\color{black} \\
0.040392
\end{pmatrix}\end{equation*}
\begin{equation*}
\left[ \frac{{w}^{EM}_i}{{w}^{EM}_j} \right] =
\begin{pmatrix}
$\,\,$ 1 $\,\,$ & $\,\,$2$\,\,$ & $\,\,$\color{red} 2.0842\color{black} $\,\,$ & $\,\,$12$\,\,$ \\
$\,\,$1/2$\,\,$ & $\,\,$ 1 $\,\,$ & $\,\,$\color{red} 1.0421\color{black} $\,\,$ & $\,\,$6  $\,\,$ \\
$\,\,$\color{red} 0.4798\color{black} $\,\,$ & $\,\,$\color{red} 0.9596\color{black} $\,\,$ & $\,\,$ 1 $\,\,$ & $\,\,$\color{red} 5.7577\color{black}  $\,\,$ \\
$\,\,$1/12$\,\,$ & $\,\,$1/6$\,\,$ & $\,\,$\color{red} 0.1737\color{black} $\,\,$ & $\,\,$ 1  $\,\,$ \\
\end{pmatrix},
\end{equation*}

\begin{equation*}
\mathbf{w}^{\prime} =
\begin{pmatrix}
0.48\\
0.24\\
0.24\\
0.04
\end{pmatrix} =
0.990306\cdot
\begin{pmatrix}
0.484699\\
0.242349\\
\color{gr} 0.242349\color{black} \\
0.040392
\end{pmatrix},
\end{equation*}
\begin{equation*}
\left[ \frac{{w}^{\prime}_i}{{w}^{\prime}_j} \right] =
\begin{pmatrix}
$\,\,$ 1 $\,\,$ & $\,\,$2$\,\,$ & $\,\,$\color{blue} 2\color{black} $\,\,$ & $\,\,$12$\,\,$ \\
$\,\,$1/2$\,\,$ & $\,\,$ 1 $\,\,$ & $\,\,$\color{blue} 1\color{black} $\,\,$ & $\,\,$6  $\,\,$ \\
$\,\,$\color{blue} 1/2\color{black} $\,\,$ & $\,\,$\color{blue} 1\color{black} $\,\,$ & $\,\,$ 1 $\,\,$ & $\,\,$\color{gr} \color{blue} 6\color{black}  $\,\,$ \\
$\,\,$1/12$\,\,$ & $\,\,$1/6$\,\,$ & $\,\,$\color{gr} \color{blue}  1/6\color{black} $\,\,$ & $\,\,$ 1  $\,\,$ \\
\end{pmatrix},
\end{equation*}
\end{example}
\newpage
\begin{example}
\begin{equation*}
\mathbf{A} =
\begin{pmatrix}
$\,\,$ 1 $\,\,$ & $\,\,$3$\,\,$ & $\,\,$2$\,\,$ & $\,\,$8 $\,\,$ \\
$\,\,$ 1/3$\,\,$ & $\,\,$ 1 $\,\,$ & $\,\,$2$\,\,$ & $\,\,$4 $\,\,$ \\
$\,\,$ 1/2$\,\,$ & $\,\,$ 1/2$\,\,$ & $\,\,$ 1 $\,\,$ & $\,\,$3 $\,\,$ \\
$\,\,$ 1/8$\,\,$ & $\,\,$ 1/4$\,\,$ & $\,\,$ 1/3$\,\,$ & $\,\,$ 1  $\,\,$ \\
\end{pmatrix},
\qquad
\lambda_{\max} =
4.1031,
\qquad
CR = 0.0389
\end{equation*}

\begin{equation*}
\mathbf{w}^{EM} =
\begin{pmatrix}
0.511137\\
0.248489\\
0.180388\\
\color{red} 0.059986\color{black}
\end{pmatrix}\end{equation*}
\begin{equation*}
\left[ \frac{{w}^{EM}_i}{{w}^{EM}_j} \right] =
\begin{pmatrix}
$\,\,$ 1 $\,\,$ & $\,\,$2.0570$\,\,$ & $\,\,$2.8335$\,\,$ & $\,\,$\color{red} 8.5210\color{black} $\,\,$ \\
$\,\,$0.4862$\,\,$ & $\,\,$ 1 $\,\,$ & $\,\,$1.3775$\,\,$ & $\,\,$\color{red} 4.1425\color{black}   $\,\,$ \\
$\,\,$0.3529$\,\,$ & $\,\,$0.7259$\,\,$ & $\,\,$ 1 $\,\,$ & $\,\,$\color{red} 3.0072\color{black}  $\,\,$ \\
$\,\,$\color{red} 0.1174\color{black} $\,\,$ & $\,\,$\color{red} 0.2414\color{black} $\,\,$ & $\,\,$\color{red} 0.3325\color{black} $\,\,$ & $\,\,$ 1  $\,\,$ \\
\end{pmatrix},
\end{equation*}

\begin{equation*}
\mathbf{w}^{\prime} =
\begin{pmatrix}
0.511063\\
0.248454\\
0.180362\\
0.060121
\end{pmatrix} =
0.999856\cdot
\begin{pmatrix}
0.511137\\
0.248489\\
0.180388\\
\color{gr} 0.060129\color{black}
\end{pmatrix},
\end{equation*}
\begin{equation*}
\left[ \frac{{w}^{\prime}_i}{{w}^{\prime}_j} \right] =
\begin{pmatrix}
$\,\,$ 1 $\,\,$ & $\,\,$2.0570$\,\,$ & $\,\,$2.8335$\,\,$ & $\,\,$\color{gr} 8.5006\color{black} $\,\,$ \\
$\,\,$0.4862$\,\,$ & $\,\,$ 1 $\,\,$ & $\,\,$1.3775$\,\,$ & $\,\,$\color{gr} 4.1326\color{black}   $\,\,$ \\
$\,\,$0.3529$\,\,$ & $\,\,$0.7259$\,\,$ & $\,\,$ 1 $\,\,$ & $\,\,$\color{gr} \color{blue} 3\color{black}  $\,\,$ \\
$\,\,$\color{gr} 0.1176\color{black} $\,\,$ & $\,\,$\color{gr} 0.2420\color{black} $\,\,$ & $\,\,$\color{gr} \color{blue}  1/3\color{black} $\,\,$ & $\,\,$ 1  $\,\,$ \\
\end{pmatrix},
\end{equation*}
\end{example}
\newpage
\begin{example}
\begin{equation*}
\mathbf{A} =
\begin{pmatrix}
$\,\,$ 1 $\,\,$ & $\,\,$3$\,\,$ & $\,\,$2$\,\,$ & $\,\,$9 $\,\,$ \\
$\,\,$ 1/3$\,\,$ & $\,\,$ 1 $\,\,$ & $\,\,$1$\,\,$ & $\,\,$9 $\,\,$ \\
$\,\,$ 1/2$\,\,$ & $\,\,$ 1 $\,\,$ & $\,\,$ 1 $\,\,$ & $\,\,$6 $\,\,$ \\
$\,\,$ 1/9$\,\,$ & $\,\,$ 1/9$\,\,$ & $\,\,$ 1/6$\,\,$ & $\,\,$ 1  $\,\,$ \\
\end{pmatrix},
\qquad
\lambda_{\max} =
4.1031,
\qquad
CR = 0.0389
\end{equation*}

\begin{equation*}
\mathbf{w}^{EM} =
\begin{pmatrix}
0.491641\\
0.239011\\
\color{red} 0.230790\color{black} \\
0.038557
\end{pmatrix}\end{equation*}
\begin{equation*}
\left[ \frac{{w}^{EM}_i}{{w}^{EM}_j} \right] =
\begin{pmatrix}
$\,\,$ 1 $\,\,$ & $\,\,$2.0570$\,\,$ & $\,\,$\color{red} 2.1302\color{black} $\,\,$ & $\,\,$12.7509$\,\,$ \\
$\,\,$0.4862$\,\,$ & $\,\,$ 1 $\,\,$ & $\,\,$\color{red} 1.0356\color{black} $\,\,$ & $\,\,$6.1989  $\,\,$ \\
$\,\,$\color{red} 0.4694\color{black} $\,\,$ & $\,\,$\color{red} 0.9656\color{black} $\,\,$ & $\,\,$ 1 $\,\,$ & $\,\,$\color{red} 5.9857\color{black}  $\,\,$ \\
$\,\,$0.0784$\,\,$ & $\,\,$0.1613$\,\,$ & $\,\,$\color{red} 0.1671\color{black} $\,\,$ & $\,\,$ 1  $\,\,$ \\
\end{pmatrix},
\end{equation*}

\begin{equation*}
\mathbf{w}^{\prime} =
\begin{pmatrix}
0.491369\\
0.238879\\
0.231216\\
0.038536
\end{pmatrix} =
0.999447\cdot
\begin{pmatrix}
0.491641\\
0.239011\\
\color{gr} 0.231344\color{black} \\
0.038557
\end{pmatrix},
\end{equation*}
\begin{equation*}
\left[ \frac{{w}^{\prime}_i}{{w}^{\prime}_j} \right] =
\begin{pmatrix}
$\,\,$ 1 $\,\,$ & $\,\,$2.0570$\,\,$ & $\,\,$\color{gr} 2.1252\color{black} $\,\,$ & $\,\,$12.7509$\,\,$ \\
$\,\,$0.4862$\,\,$ & $\,\,$ 1 $\,\,$ & $\,\,$\color{gr} 1.0331\color{black} $\,\,$ & $\,\,$6.1989  $\,\,$ \\
$\,\,$\color{gr} 0.4706\color{black} $\,\,$ & $\,\,$\color{gr} 0.9679\color{black} $\,\,$ & $\,\,$ 1 $\,\,$ & $\,\,$\color{gr} \color{blue} 6\color{black}  $\,\,$ \\
$\,\,$0.0784$\,\,$ & $\,\,$0.1613$\,\,$ & $\,\,$\color{gr} \color{blue}  1/6\color{black} $\,\,$ & $\,\,$ 1  $\,\,$ \\
\end{pmatrix},
\end{equation*}
\end{example}
\newpage
\begin{example}
\begin{equation*}
\mathbf{A} =
\begin{pmatrix}
$\,\,$ 1 $\,\,$ & $\,\,$3$\,\,$ & $\,\,$2$\,\,$ & $\,\,$9 $\,\,$ \\
$\,\,$ 1/3$\,\,$ & $\,\,$ 1 $\,\,$ & $\,\,$2$\,\,$ & $\,\,$4 $\,\,$ \\
$\,\,$ 1/2$\,\,$ & $\,\,$ 1/2$\,\,$ & $\,\,$ 1 $\,\,$ & $\,\,$3 $\,\,$ \\
$\,\,$ 1/9$\,\,$ & $\,\,$ 1/4$\,\,$ & $\,\,$ 1/3$\,\,$ & $\,\,$ 1  $\,\,$ \\
\end{pmatrix},
\qquad
\lambda_{\max} =
4.1031,
\qquad
CR = 0.0389
\end{equation*}

\begin{equation*}
\mathbf{w}^{EM} =
\begin{pmatrix}
0.518811\\
0.245012\\
0.178669\\
\color{red} 0.057508\color{black}
\end{pmatrix}\end{equation*}
\begin{equation*}
\left[ \frac{{w}^{EM}_i}{{w}^{EM}_j} \right] =
\begin{pmatrix}
$\,\,$ 1 $\,\,$ & $\,\,$2.1175$\,\,$ & $\,\,$2.9038$\,\,$ & $\,\,$\color{red} 9.0216\color{black} $\,\,$ \\
$\,\,$0.4723$\,\,$ & $\,\,$ 1 $\,\,$ & $\,\,$1.3713$\,\,$ & $\,\,$\color{red} 4.2605\color{black}   $\,\,$ \\
$\,\,$0.3444$\,\,$ & $\,\,$0.7292$\,\,$ & $\,\,$ 1 $\,\,$ & $\,\,$\color{red} 3.1069\color{black}  $\,\,$ \\
$\,\,$\color{red} 0.1108\color{black} $\,\,$ & $\,\,$\color{red} 0.2347\color{black} $\,\,$ & $\,\,$\color{red} 0.3219\color{black} $\,\,$ & $\,\,$ 1  $\,\,$ \\
\end{pmatrix},
\end{equation*}

\begin{equation*}
\mathbf{w}^{\prime} =
\begin{pmatrix}
0.518740\\
0.244978\\
0.178644\\
0.057638
\end{pmatrix} =
0.999862\cdot
\begin{pmatrix}
0.518811\\
0.245012\\
0.178669\\
\color{gr} 0.057646\color{black}
\end{pmatrix},
\end{equation*}
\begin{equation*}
\left[ \frac{{w}^{\prime}_i}{{w}^{\prime}_j} \right] =
\begin{pmatrix}
$\,\,$ 1 $\,\,$ & $\,\,$2.1175$\,\,$ & $\,\,$2.9038$\,\,$ & $\,\,$\color{gr} \color{blue} 9\color{black} $\,\,$ \\
$\,\,$0.4723$\,\,$ & $\,\,$ 1 $\,\,$ & $\,\,$1.3713$\,\,$ & $\,\,$\color{gr} 4.2503\color{black}   $\,\,$ \\
$\,\,$0.3444$\,\,$ & $\,\,$0.7292$\,\,$ & $\,\,$ 1 $\,\,$ & $\,\,$\color{gr} 3.0994\color{black}  $\,\,$ \\
$\,\,$\color{gr} \color{blue}  1/9\color{black} $\,\,$ & $\,\,$\color{gr} 0.2353\color{black} $\,\,$ & $\,\,$\color{gr} 0.3226\color{black} $\,\,$ & $\,\,$ 1  $\,\,$ \\
\end{pmatrix},
\end{equation*}
\end{example}
\newpage
\begin{example}
\begin{equation*}
\mathbf{A} =
\begin{pmatrix}
$\,\,$ 1 $\,\,$ & $\,\,$3$\,\,$ & $\,\,$2$\,\,$ & $\,\,$9 $\,\,$ \\
$\,\,$ 1/3$\,\,$ & $\,\,$ 1 $\,\,$ & $\,\,$3$\,\,$ & $\,\,$5 $\,\,$ \\
$\,\,$ 1/2$\,\,$ & $\,\,$ 1/3$\,\,$ & $\,\,$ 1 $\,\,$ & $\,\,$3 $\,\,$ \\
$\,\,$ 1/9$\,\,$ & $\,\,$ 1/5$\,\,$ & $\,\,$ 1/3$\,\,$ & $\,\,$ 1  $\,\,$ \\
\end{pmatrix},
\qquad
\lambda_{\max} =
4.1966,
\qquad
CR = 0.0741
\end{equation*}

\begin{equation*}
\mathbf{w}^{EM} =
\begin{pmatrix}
0.508824\\
0.281872\\
0.157552\\
\color{red} 0.051752\color{black}
\end{pmatrix}\end{equation*}
\begin{equation*}
\left[ \frac{{w}^{EM}_i}{{w}^{EM}_j} \right] =
\begin{pmatrix}
$\,\,$ 1 $\,\,$ & $\,\,$1.8052$\,\,$ & $\,\,$3.2296$\,\,$ & $\,\,$\color{red} 9.8320\color{black} $\,\,$ \\
$\,\,$0.5540$\,\,$ & $\,\,$ 1 $\,\,$ & $\,\,$1.7891$\,\,$ & $\,\,$\color{red} 5.4466\color{black}   $\,\,$ \\
$\,\,$0.3096$\,\,$ & $\,\,$0.5589$\,\,$ & $\,\,$ 1 $\,\,$ & $\,\,$\color{red} 3.0444\color{black}  $\,\,$ \\
$\,\,$\color{red} 0.1017\color{black} $\,\,$ & $\,\,$\color{red} 0.1836\color{black} $\,\,$ & $\,\,$\color{red} 0.3285\color{black} $\,\,$ & $\,\,$ 1  $\,\,$ \\
\end{pmatrix},
\end{equation*}

\begin{equation*}
\mathbf{w}^{\prime} =
\begin{pmatrix}
0.508435\\
0.281657\\
0.157431\\
0.052477
\end{pmatrix} =
0.999235\cdot
\begin{pmatrix}
0.508824\\
0.281872\\
0.157552\\
\color{gr} 0.052517\color{black}
\end{pmatrix},
\end{equation*}
\begin{equation*}
\left[ \frac{{w}^{\prime}_i}{{w}^{\prime}_j} \right] =
\begin{pmatrix}
$\,\,$ 1 $\,\,$ & $\,\,$1.8052$\,\,$ & $\,\,$3.2296$\,\,$ & $\,\,$\color{gr} 9.6887\color{black} $\,\,$ \\
$\,\,$0.5540$\,\,$ & $\,\,$ 1 $\,\,$ & $\,\,$1.7891$\,\,$ & $\,\,$\color{gr} 5.3672\color{black}   $\,\,$ \\
$\,\,$0.3096$\,\,$ & $\,\,$0.5589$\,\,$ & $\,\,$ 1 $\,\,$ & $\,\,$\color{gr} \color{blue} 3\color{black}  $\,\,$ \\
$\,\,$\color{gr} 0.1032\color{black} $\,\,$ & $\,\,$\color{gr} 0.1863\color{black} $\,\,$ & $\,\,$\color{gr} \color{blue}  1/3\color{black} $\,\,$ & $\,\,$ 1  $\,\,$ \\
\end{pmatrix},
\end{equation*}
\end{example}
\newpage
\begin{example}
\begin{equation*}
\mathbf{A} =
\begin{pmatrix}
$\,\,$ 1 $\,\,$ & $\,\,$3$\,\,$ & $\,\,$3$\,\,$ & $\,\,$4 $\,\,$ \\
$\,\,$ 1/3$\,\,$ & $\,\,$ 1 $\,\,$ & $\,\,$2$\,\,$ & $\,\,$7 $\,\,$ \\
$\,\,$ 1/3$\,\,$ & $\,\,$ 1/2$\,\,$ & $\,\,$ 1 $\,\,$ & $\,\,$2 $\,\,$ \\
$\,\,$ 1/4$\,\,$ & $\,\,$ 1/7$\,\,$ & $\,\,$ 1/2$\,\,$ & $\,\,$ 1  $\,\,$ \\
\end{pmatrix},
\qquad
\lambda_{\max} =
4.2421,
\qquad
CR = 0.0913
\end{equation*}

\begin{equation*}
\mathbf{w}^{EM} =
\begin{pmatrix}
0.492243\\
0.294308\\
\color{red} 0.140808\color{black} \\
0.072640
\end{pmatrix}\end{equation*}
\begin{equation*}
\left[ \frac{{w}^{EM}_i}{{w}^{EM}_j} \right] =
\begin{pmatrix}
$\,\,$ 1 $\,\,$ & $\,\,$1.6725$\,\,$ & $\,\,$\color{red} 3.4958\color{black} $\,\,$ & $\,\,$6.7764$\,\,$ \\
$\,\,$0.5979$\,\,$ & $\,\,$ 1 $\,\,$ & $\,\,$\color{red} 2.0901\color{black} $\,\,$ & $\,\,$4.0516  $\,\,$ \\
$\,\,$\color{red} 0.2861\color{black} $\,\,$ & $\,\,$\color{red} 0.4784\color{black} $\,\,$ & $\,\,$ 1 $\,\,$ & $\,\,$\color{red} 1.9384\color{black}  $\,\,$ \\
$\,\,$0.1476$\,\,$ & $\,\,$0.2468$\,\,$ & $\,\,$\color{red} 0.5159\color{black} $\,\,$ & $\,\,$ 1  $\,\,$ \\
\end{pmatrix},
\end{equation*}

\begin{equation*}
\mathbf{w}^{\prime} =
\begin{pmatrix}
0.490051\\
0.292997\\
0.144634\\
0.072317
\end{pmatrix} =
0.995547\cdot
\begin{pmatrix}
0.492243\\
0.294308\\
\color{gr} 0.145281\color{black} \\
0.072640
\end{pmatrix},
\end{equation*}
\begin{equation*}
\left[ \frac{{w}^{\prime}_i}{{w}^{\prime}_j} \right] =
\begin{pmatrix}
$\,\,$ 1 $\,\,$ & $\,\,$1.6725$\,\,$ & $\,\,$\color{gr} 3.3882\color{black} $\,\,$ & $\,\,$6.7764$\,\,$ \\
$\,\,$0.5979$\,\,$ & $\,\,$ 1 $\,\,$ & $\,\,$\color{gr} 2.0258\color{black} $\,\,$ & $\,\,$4.0516  $\,\,$ \\
$\,\,$\color{gr} 0.2951\color{black} $\,\,$ & $\,\,$\color{gr} 0.4936\color{black} $\,\,$ & $\,\,$ 1 $\,\,$ & $\,\,$\color{gr} \color{blue} 2\color{black}  $\,\,$ \\
$\,\,$0.1476$\,\,$ & $\,\,$0.2468$\,\,$ & $\,\,$\color{gr} \color{blue}  1/2\color{black} $\,\,$ & $\,\,$ 1  $\,\,$ \\
\end{pmatrix},
\end{equation*}
\end{example}
\newpage
\begin{example}
\begin{equation*}
\mathbf{A} =
\begin{pmatrix}
$\,\,$ 1 $\,\,$ & $\,\,$3$\,\,$ & $\,\,$3$\,\,$ & $\,\,$4 $\,\,$ \\
$\,\,$ 1/3$\,\,$ & $\,\,$ 1 $\,\,$ & $\,\,$3$\,\,$ & $\,\,$2 $\,\,$ \\
$\,\,$ 1/3$\,\,$ & $\,\,$ 1/3$\,\,$ & $\,\,$ 1 $\,\,$ & $\,\,$1 $\,\,$ \\
$\,\,$ 1/4$\,\,$ & $\,\,$ 1/2$\,\,$ & $\,\,$ 1 $\,\,$ & $\,\,$ 1  $\,\,$ \\
\end{pmatrix},
\qquad
\lambda_{\max} =
4.1031,
\qquad
CR = 0.0389
\end{equation*}

\begin{equation*}
\mathbf{w}^{EM} =
\begin{pmatrix}
0.511210\\
0.248525\\
0.120276\\
\color{red} 0.119988\color{black}
\end{pmatrix}\end{equation*}
\begin{equation*}
\left[ \frac{{w}^{EM}_i}{{w}^{EM}_j} \right] =
\begin{pmatrix}
$\,\,$ 1 $\,\,$ & $\,\,$2.0570$\,\,$ & $\,\,$4.2503$\,\,$ & $\,\,$\color{red} 4.2605\color{black} $\,\,$ \\
$\,\,$0.4862$\,\,$ & $\,\,$ 1 $\,\,$ & $\,\,$2.0663$\,\,$ & $\,\,$\color{red} 2.0712\color{black}   $\,\,$ \\
$\,\,$0.2353$\,\,$ & $\,\,$0.4840$\,\,$ & $\,\,$ 1 $\,\,$ & $\,\,$\color{red} 1.0024\color{black}  $\,\,$ \\
$\,\,$\color{red} 0.2347\color{black} $\,\,$ & $\,\,$\color{red} 0.4828\color{black} $\,\,$ & $\,\,$\color{red} 0.9976\color{black} $\,\,$ & $\,\,$ 1  $\,\,$ \\
\end{pmatrix},
\end{equation*}

\begin{equation*}
\mathbf{w}^{\prime} =
\begin{pmatrix}
0.511063\\
0.248454\\
0.120241\\
0.120241
\end{pmatrix} =
0.999712\cdot
\begin{pmatrix}
0.511210\\
0.248525\\
0.120276\\
\color{gr} 0.120276\color{black}
\end{pmatrix},
\end{equation*}
\begin{equation*}
\left[ \frac{{w}^{\prime}_i}{{w}^{\prime}_j} \right] =
\begin{pmatrix}
$\,\,$ 1 $\,\,$ & $\,\,$2.0570$\,\,$ & $\,\,$4.2503$\,\,$ & $\,\,$\color{gr} 4.2503\color{black} $\,\,$ \\
$\,\,$0.4862$\,\,$ & $\,\,$ 1 $\,\,$ & $\,\,$2.0663$\,\,$ & $\,\,$\color{gr} 2.0663\color{black}   $\,\,$ \\
$\,\,$0.2353$\,\,$ & $\,\,$0.4840$\,\,$ & $\,\,$ 1 $\,\,$ & $\,\,$\color{gr} \color{blue} 1\color{black}  $\,\,$ \\
$\,\,$\color{gr} 0.2353\color{black} $\,\,$ & $\,\,$\color{gr} 0.4840\color{black} $\,\,$ & $\,\,$\color{gr} \color{blue} 1\color{black} $\,\,$ & $\,\,$ 1  $\,\,$ \\
\end{pmatrix},
\end{equation*}
\end{example}
\newpage
\begin{example}
\begin{equation*}
\mathbf{A} =
\begin{pmatrix}
$\,\,$ 1 $\,\,$ & $\,\,$3$\,\,$ & $\,\,$3$\,\,$ & $\,\,$5 $\,\,$ \\
$\,\,$ 1/3$\,\,$ & $\,\,$ 1 $\,\,$ & $\,\,$2$\,\,$ & $\,\,$9 $\,\,$ \\
$\,\,$ 1/3$\,\,$ & $\,\,$ 1/2$\,\,$ & $\,\,$ 1 $\,\,$ & $\,\,$3 $\,\,$ \\
$\,\,$ 1/5$\,\,$ & $\,\,$ 1/9$\,\,$ & $\,\,$ 1/3$\,\,$ & $\,\,$ 1  $\,\,$ \\
\end{pmatrix},
\qquad
\lambda_{\max} =
4.2507,
\qquad
CR = 0.0946
\end{equation*}

\begin{equation*}
\mathbf{w}^{EM} =
\begin{pmatrix}
0.497871\\
0.297566\\
\color{red} 0.148531\color{black} \\
0.056032
\end{pmatrix}\end{equation*}
\begin{equation*}
\left[ \frac{{w}^{EM}_i}{{w}^{EM}_j} \right] =
\begin{pmatrix}
$\,\,$ 1 $\,\,$ & $\,\,$1.6731$\,\,$ & $\,\,$\color{red} 3.3520\color{black} $\,\,$ & $\,\,$8.8854$\,\,$ \\
$\,\,$0.5977$\,\,$ & $\,\,$ 1 $\,\,$ & $\,\,$\color{red} 2.0034\color{black} $\,\,$ & $\,\,$5.3106  $\,\,$ \\
$\,\,$\color{red} 0.2983\color{black} $\,\,$ & $\,\,$\color{red} 0.4992\color{black} $\,\,$ & $\,\,$ 1 $\,\,$ & $\,\,$\color{red} 2.6508\color{black}  $\,\,$ \\
$\,\,$0.1125$\,\,$ & $\,\,$0.1883$\,\,$ & $\,\,$\color{red} 0.3772\color{black} $\,\,$ & $\,\,$ 1  $\,\,$ \\
\end{pmatrix},
\end{equation*}

\begin{equation*}
\mathbf{w}^{\prime} =
\begin{pmatrix}
0.497745\\
0.297491\\
0.148745\\
0.056018
\end{pmatrix} =
0.999748\cdot
\begin{pmatrix}
0.497871\\
0.297566\\
\color{gr} 0.148783\color{black} \\
0.056032
\end{pmatrix},
\end{equation*}
\begin{equation*}
\left[ \frac{{w}^{\prime}_i}{{w}^{\prime}_j} \right] =
\begin{pmatrix}
$\,\,$ 1 $\,\,$ & $\,\,$1.6731$\,\,$ & $\,\,$\color{gr} 3.3463\color{black} $\,\,$ & $\,\,$8.8854$\,\,$ \\
$\,\,$0.5977$\,\,$ & $\,\,$ 1 $\,\,$ & $\,\,$\color{gr} \color{blue} 2\color{black} $\,\,$ & $\,\,$5.3106  $\,\,$ \\
$\,\,$\color{gr} 0.2988\color{black} $\,\,$ & $\,\,$\color{gr} \color{blue}  1/2\color{black} $\,\,$ & $\,\,$ 1 $\,\,$ & $\,\,$\color{gr} 2.6553\color{black}  $\,\,$ \\
$\,\,$0.1125$\,\,$ & $\,\,$0.1883$\,\,$ & $\,\,$\color{gr} 0.3766\color{black} $\,\,$ & $\,\,$ 1  $\,\,$ \\
\end{pmatrix},
\end{equation*}
\end{example}
\newpage
\begin{example}
\begin{equation*}
\mathbf{A} =
\begin{pmatrix}
$\,\,$ 1 $\,\,$ & $\,\,$3$\,\,$ & $\,\,$3$\,\,$ & $\,\,$8 $\,\,$ \\
$\,\,$ 1/3$\,\,$ & $\,\,$ 1 $\,\,$ & $\,\,$4$\,\,$ & $\,\,$5 $\,\,$ \\
$\,\,$ 1/3$\,\,$ & $\,\,$ 1/4$\,\,$ & $\,\,$ 1 $\,\,$ & $\,\,$2 $\,\,$ \\
$\,\,$ 1/8$\,\,$ & $\,\,$ 1/5$\,\,$ & $\,\,$ 1/2$\,\,$ & $\,\,$ 1  $\,\,$ \\
\end{pmatrix},
\qquad
\lambda_{\max} =
4.1689,
\qquad
CR = 0.0637
\end{equation*}

\begin{equation*}
\mathbf{w}^{EM} =
\begin{pmatrix}
0.533160\\
0.293334\\
0.115705\\
\color{red} 0.057801\color{black}
\end{pmatrix}\end{equation*}
\begin{equation*}
\left[ \frac{{w}^{EM}_i}{{w}^{EM}_j} \right] =
\begin{pmatrix}
$\,\,$ 1 $\,\,$ & $\,\,$1.8176$\,\,$ & $\,\,$4.6079$\,\,$ & $\,\,$\color{red} 9.2241\color{black} $\,\,$ \\
$\,\,$0.5502$\,\,$ & $\,\,$ 1 $\,\,$ & $\,\,$2.5352$\,\,$ & $\,\,$\color{red} 5.0749\color{black}   $\,\,$ \\
$\,\,$0.2170$\,\,$ & $\,\,$0.3944$\,\,$ & $\,\,$ 1 $\,\,$ & $\,\,$\color{red} 2.0018\color{black}  $\,\,$ \\
$\,\,$\color{red} 0.1084\color{black} $\,\,$ & $\,\,$\color{red} 0.1970\color{black} $\,\,$ & $\,\,$\color{red} 0.4996\color{black} $\,\,$ & $\,\,$ 1  $\,\,$ \\
\end{pmatrix},
\end{equation*}

\begin{equation*}
\mathbf{w}^{\prime} =
\begin{pmatrix}
0.533133\\
0.293319\\
0.115699\\
0.057849
\end{pmatrix} =
0.999948\cdot
\begin{pmatrix}
0.533160\\
0.293334\\
0.115705\\
\color{gr} 0.057852\color{black}
\end{pmatrix},
\end{equation*}
\begin{equation*}
\left[ \frac{{w}^{\prime}_i}{{w}^{\prime}_j} \right] =
\begin{pmatrix}
$\,\,$ 1 $\,\,$ & $\,\,$1.8176$\,\,$ & $\,\,$4.6079$\,\,$ & $\,\,$\color{gr} 9.2159\color{black} $\,\,$ \\
$\,\,$0.5502$\,\,$ & $\,\,$ 1 $\,\,$ & $\,\,$2.5352$\,\,$ & $\,\,$\color{gr} 5.0704\color{black}   $\,\,$ \\
$\,\,$0.2170$\,\,$ & $\,\,$0.3944$\,\,$ & $\,\,$ 1 $\,\,$ & $\,\,$\color{gr} \color{blue} 2\color{black}  $\,\,$ \\
$\,\,$\color{gr} 0.1085\color{black} $\,\,$ & $\,\,$\color{gr} 0.1972\color{black} $\,\,$ & $\,\,$\color{gr} \color{blue}  1/2\color{black} $\,\,$ & $\,\,$ 1  $\,\,$ \\
\end{pmatrix},
\end{equation*}
\end{example}
\newpage
\begin{example}
\begin{equation*}
\mathbf{A} =
\begin{pmatrix}
$\,\,$ 1 $\,\,$ & $\,\,$3$\,\,$ & $\,\,$3$\,\,$ & $\,\,$9 $\,\,$ \\
$\,\,$ 1/3$\,\,$ & $\,\,$ 1 $\,\,$ & $\,\,$2$\,\,$ & $\,\,$2 $\,\,$ \\
$\,\,$ 1/3$\,\,$ & $\,\,$ 1/2$\,\,$ & $\,\,$ 1 $\,\,$ & $\,\,$5 $\,\,$ \\
$\,\,$ 1/9$\,\,$ & $\,\,$ 1/2$\,\,$ & $\,\,$ 1/5$\,\,$ & $\,\,$ 1  $\,\,$ \\
\end{pmatrix},
\qquad
\lambda_{\max} =
4.2277,
\qquad
CR = 0.0859
\end{equation*}

\begin{equation*}
\mathbf{w}^{EM} =
\begin{pmatrix}
\color{red} 0.542075\color{black} \\
0.209755\\
0.185522\\
0.062649
\end{pmatrix}\end{equation*}
\begin{equation*}
\left[ \frac{{w}^{EM}_i}{{w}^{EM}_j} \right] =
\begin{pmatrix}
$\,\,$ 1 $\,\,$ & $\,\,$\color{red} 2.5843\color{black} $\,\,$ & $\,\,$\color{red} 2.9219\color{black} $\,\,$ & $\,\,$\color{red} 8.6526\color{black} $\,\,$ \\
$\,\,$\color{red} 0.3869\color{black} $\,\,$ & $\,\,$ 1 $\,\,$ & $\,\,$1.1306$\,\,$ & $\,\,$3.3481  $\,\,$ \\
$\,\,$\color{red} 0.3422\color{black} $\,\,$ & $\,\,$0.8845$\,\,$ & $\,\,$ 1 $\,\,$ & $\,\,$2.9613 $\,\,$ \\
$\,\,$\color{red} 0.1156\color{black} $\,\,$ & $\,\,$0.2987$\,\,$ & $\,\,$0.3377$\,\,$ & $\,\,$ 1  $\,\,$ \\
\end{pmatrix},
\end{equation*}

\begin{equation*}
\mathbf{w}^{\prime} =
\begin{pmatrix}
0.548615\\
0.206759\\
0.182872\\
0.061754
\end{pmatrix} =
0.985717\cdot
\begin{pmatrix}
\color{gr} 0.556565\color{black} \\
0.209755\\
0.185522\\
0.062649
\end{pmatrix},
\end{equation*}
\begin{equation*}
\left[ \frac{{w}^{\prime}_i}{{w}^{\prime}_j} \right] =
\begin{pmatrix}
$\,\,$ 1 $\,\,$ & $\,\,$\color{gr} 2.6534\color{black} $\,\,$ & $\,\,$\color{gr} \color{blue} 3\color{black} $\,\,$ & $\,\,$\color{gr} 8.8839\color{black} $\,\,$ \\
$\,\,$\color{gr} 0.3769\color{black} $\,\,$ & $\,\,$ 1 $\,\,$ & $\,\,$1.1306$\,\,$ & $\,\,$3.3481  $\,\,$ \\
$\,\,$\color{gr} \color{blue}  1/3\color{black} $\,\,$ & $\,\,$0.8845$\,\,$ & $\,\,$ 1 $\,\,$ & $\,\,$2.9613 $\,\,$ \\
$\,\,$\color{gr} 0.1126\color{black} $\,\,$ & $\,\,$0.2987$\,\,$ & $\,\,$0.3377$\,\,$ & $\,\,$ 1  $\,\,$ \\
\end{pmatrix},
\end{equation*}
\end{example}
\newpage
\begin{example}
\begin{equation*}
\mathbf{A} =
\begin{pmatrix}
$\,\,$ 1 $\,\,$ & $\,\,$3$\,\,$ & $\,\,$3$\,\,$ & $\,\,$9 $\,\,$ \\
$\,\,$ 1/3$\,\,$ & $\,\,$ 1 $\,\,$ & $\,\,$4$\,\,$ & $\,\,$5 $\,\,$ \\
$\,\,$ 1/3$\,\,$ & $\,\,$ 1/4$\,\,$ & $\,\,$ 1 $\,\,$ & $\,\,$2 $\,\,$ \\
$\,\,$ 1/9$\,\,$ & $\,\,$ 1/5$\,\,$ & $\,\,$ 1/2$\,\,$ & $\,\,$ 1  $\,\,$ \\
\end{pmatrix},
\qquad
\lambda_{\max} =
4.1655,
\qquad
CR = 0.0624
\end{equation*}

\begin{equation*}
\mathbf{w}^{EM} =
\begin{pmatrix}
0.540469\\
0.289393\\
0.114757\\
\color{red} 0.055381\color{black}
\end{pmatrix}\end{equation*}
\begin{equation*}
\left[ \frac{{w}^{EM}_i}{{w}^{EM}_j} \right] =
\begin{pmatrix}
$\,\,$ 1 $\,\,$ & $\,\,$1.8676$\,\,$ & $\,\,$4.7097$\,\,$ & $\,\,$\color{red} 9.7592\color{black} $\,\,$ \\
$\,\,$0.5354$\,\,$ & $\,\,$ 1 $\,\,$ & $\,\,$2.5218$\,\,$ & $\,\,$\color{red} 5.2255\color{black}   $\,\,$ \\
$\,\,$0.2123$\,\,$ & $\,\,$0.3965$\,\,$ & $\,\,$ 1 $\,\,$ & $\,\,$\color{red} 2.0721\color{black}  $\,\,$ \\
$\,\,$\color{red} 0.1025\color{black} $\,\,$ & $\,\,$\color{red} 0.1914\color{black} $\,\,$ & $\,\,$\color{red} 0.4826\color{black} $\,\,$ & $\,\,$ 1  $\,\,$ \\
\end{pmatrix},
\end{equation*}

\begin{equation*}
\mathbf{w}^{\prime} =
\begin{pmatrix}
0.539392\\
0.288817\\
0.114528\\
0.057264
\end{pmatrix} =
0.998006\cdot
\begin{pmatrix}
0.540469\\
0.289393\\
0.114757\\
\color{gr} 0.057378\color{black}
\end{pmatrix},
\end{equation*}
\begin{equation*}
\left[ \frac{{w}^{\prime}_i}{{w}^{\prime}_j} \right] =
\begin{pmatrix}
$\,\,$ 1 $\,\,$ & $\,\,$1.8676$\,\,$ & $\,\,$4.7097$\,\,$ & $\,\,$\color{gr} 9.4194\color{black} $\,\,$ \\
$\,\,$0.5354$\,\,$ & $\,\,$ 1 $\,\,$ & $\,\,$2.5218$\,\,$ & $\,\,$\color{gr} 5.0436\color{black}   $\,\,$ \\
$\,\,$0.2123$\,\,$ & $\,\,$0.3965$\,\,$ & $\,\,$ 1 $\,\,$ & $\,\,$\color{gr} \color{blue} 2\color{black}  $\,\,$ \\
$\,\,$\color{gr} 0.1062\color{black} $\,\,$ & $\,\,$\color{gr} 0.1983\color{black} $\,\,$ & $\,\,$\color{gr} \color{blue}  1/2\color{black} $\,\,$ & $\,\,$ 1  $\,\,$ \\
\end{pmatrix},
\end{equation*}
\end{example}
\newpage
\begin{example}
\begin{equation*}
\mathbf{A} =
\begin{pmatrix}
$\,\,$ 1 $\,\,$ & $\,\,$3$\,\,$ & $\,\,$3$\,\,$ & $\,\,$9 $\,\,$ \\
$\,\,$ 1/3$\,\,$ & $\,\,$ 1 $\,\,$ & $\,\,$5$\,\,$ & $\,\,$6 $\,\,$ \\
$\,\,$ 1/3$\,\,$ & $\,\,$ 1/5$\,\,$ & $\,\,$ 1 $\,\,$ & $\,\,$2 $\,\,$ \\
$\,\,$ 1/9$\,\,$ & $\,\,$ 1/6$\,\,$ & $\,\,$ 1/2$\,\,$ & $\,\,$ 1  $\,\,$ \\
\end{pmatrix},
\qquad
\lambda_{\max} =
4.2277,
\qquad
CR = 0.0859
\end{equation*}

\begin{equation*}
\mathbf{w}^{EM} =
\begin{pmatrix}
0.530651\\
0.312896\\
0.105661\\
\color{red} 0.050792\color{black}
\end{pmatrix}\end{equation*}
\begin{equation*}
\left[ \frac{{w}^{EM}_i}{{w}^{EM}_j} \right] =
\begin{pmatrix}
$\,\,$ 1 $\,\,$ & $\,\,$1.6959$\,\,$ & $\,\,$5.0222$\,\,$ & $\,\,$\color{red} 10.4476\color{black} $\,\,$ \\
$\,\,$0.5896$\,\,$ & $\,\,$ 1 $\,\,$ & $\,\,$2.9613$\,\,$ & $\,\,$\color{red} 6.1604\color{black}   $\,\,$ \\
$\,\,$0.1991$\,\,$ & $\,\,$0.3377$\,\,$ & $\,\,$ 1 $\,\,$ & $\,\,$\color{red} 2.0803\color{black}  $\,\,$ \\
$\,\,$\color{red} 0.0957\color{black} $\,\,$ & $\,\,$\color{red} 0.1623\color{black} $\,\,$ & $\,\,$\color{red} 0.4807\color{black} $\,\,$ & $\,\,$ 1  $\,\,$ \\
\end{pmatrix},
\end{equation*}

\begin{equation*}
\mathbf{w}^{\prime} =
\begin{pmatrix}
0.529932\\
0.312471\\
0.105518\\
0.052079
\end{pmatrix} =
0.998644\cdot
\begin{pmatrix}
0.530651\\
0.312896\\
0.105661\\
\color{gr} 0.052149\color{black}
\end{pmatrix},
\end{equation*}
\begin{equation*}
\left[ \frac{{w}^{\prime}_i}{{w}^{\prime}_j} \right] =
\begin{pmatrix}
$\,\,$ 1 $\,\,$ & $\,\,$1.6959$\,\,$ & $\,\,$5.0222$\,\,$ & $\,\,$\color{gr} 10.1756\color{black} $\,\,$ \\
$\,\,$0.5896$\,\,$ & $\,\,$ 1 $\,\,$ & $\,\,$2.9613$\,\,$ & $\,\,$\color{gr} \color{blue} 6\color{black}   $\,\,$ \\
$\,\,$0.1991$\,\,$ & $\,\,$0.3377$\,\,$ & $\,\,$ 1 $\,\,$ & $\,\,$\color{gr} 2.0261\color{black}  $\,\,$ \\
$\,\,$\color{gr} 0.0983\color{black} $\,\,$ & $\,\,$\color{gr} \color{blue}  1/6\color{black} $\,\,$ & $\,\,$\color{gr} 0.4936\color{black} $\,\,$ & $\,\,$ 1  $\,\,$ \\
\end{pmatrix},
\end{equation*}
\end{example}
\newpage
\begin{example}
\begin{equation*}
\mathbf{A} =
\begin{pmatrix}
$\,\,$ 1 $\,\,$ & $\,\,$3$\,\,$ & $\,\,$4$\,\,$ & $\,\,$5 $\,\,$ \\
$\,\,$ 1/3$\,\,$ & $\,\,$ 1 $\,\,$ & $\,\,$2$\,\,$ & $\,\,$6 $\,\,$ \\
$\,\,$ 1/4$\,\,$ & $\,\,$ 1/2$\,\,$ & $\,\,$ 1 $\,\,$ & $\,\,$2 $\,\,$ \\
$\,\,$ 1/5$\,\,$ & $\,\,$ 1/6$\,\,$ & $\,\,$ 1/2$\,\,$ & $\,\,$ 1  $\,\,$ \\
\end{pmatrix},
\qquad
\lambda_{\max} =
4.1406,
\qquad
CR = 0.0530
\end{equation*}

\begin{equation*}
\mathbf{w}^{EM} =
\begin{pmatrix}
0.531985\\
0.270096\\
\color{red} 0.129145\color{black} \\
0.068773
\end{pmatrix}\end{equation*}
\begin{equation*}
\left[ \frac{{w}^{EM}_i}{{w}^{EM}_j} \right] =
\begin{pmatrix}
$\,\,$ 1 $\,\,$ & $\,\,$1.9696$\,\,$ & $\,\,$\color{red} 4.1193\color{black} $\,\,$ & $\,\,$7.7354$\,\,$ \\
$\,\,$0.5077$\,\,$ & $\,\,$ 1 $\,\,$ & $\,\,$\color{red} 2.0914\color{black} $\,\,$ & $\,\,$3.9274  $\,\,$ \\
$\,\,$\color{red} 0.2428\color{black} $\,\,$ & $\,\,$\color{red} 0.4781\color{black} $\,\,$ & $\,\,$ 1 $\,\,$ & $\,\,$\color{red} 1.8779\color{black}  $\,\,$ \\
$\,\,$0.1293$\,\,$ & $\,\,$0.2546$\,\,$ & $\,\,$\color{red} 0.5325\color{black} $\,\,$ & $\,\,$ 1  $\,\,$ \\
\end{pmatrix},
\end{equation*}

\begin{equation*}
\mathbf{w}^{\prime} =
\begin{pmatrix}
0.529944\\
0.269060\\
0.132486\\
0.068509
\end{pmatrix} =
0.996164\cdot
\begin{pmatrix}
0.531985\\
0.270096\\
\color{gr} 0.132996\color{black} \\
0.068773
\end{pmatrix},
\end{equation*}
\begin{equation*}
\left[ \frac{{w}^{\prime}_i}{{w}^{\prime}_j} \right] =
\begin{pmatrix}
$\,\,$ 1 $\,\,$ & $\,\,$1.9696$\,\,$ & $\,\,$\color{gr} \color{blue} 4\color{black} $\,\,$ & $\,\,$7.7354$\,\,$ \\
$\,\,$0.5077$\,\,$ & $\,\,$ 1 $\,\,$ & $\,\,$\color{gr} 2.0309\color{black} $\,\,$ & $\,\,$3.9274  $\,\,$ \\
$\,\,$\color{gr} \color{blue}  1/4\color{black} $\,\,$ & $\,\,$\color{gr} 0.4924\color{black} $\,\,$ & $\,\,$ 1 $\,\,$ & $\,\,$\color{gr} 1.9338\color{black}  $\,\,$ \\
$\,\,$0.1293$\,\,$ & $\,\,$0.2546$\,\,$ & $\,\,$\color{gr} 0.5171\color{black} $\,\,$ & $\,\,$ 1  $\,\,$ \\
\end{pmatrix},
\end{equation*}
\end{example}
\newpage
\begin{example}
\begin{equation*}
\mathbf{A} =
\begin{pmatrix}
$\,\,$ 1 $\,\,$ & $\,\,$3$\,\,$ & $\,\,$4$\,\,$ & $\,\,$5 $\,\,$ \\
$\,\,$ 1/3$\,\,$ & $\,\,$ 1 $\,\,$ & $\,\,$2$\,\,$ & $\,\,$7 $\,\,$ \\
$\,\,$ 1/4$\,\,$ & $\,\,$ 1/2$\,\,$ & $\,\,$ 1 $\,\,$ & $\,\,$2 $\,\,$ \\
$\,\,$ 1/5$\,\,$ & $\,\,$ 1/7$\,\,$ & $\,\,$ 1/2$\,\,$ & $\,\,$ 1  $\,\,$ \\
\end{pmatrix},
\qquad
\lambda_{\max} =
4.1782,
\qquad
CR = 0.0672
\end{equation*}

\begin{equation*}
\mathbf{w}^{EM} =
\begin{pmatrix}
0.527402\\
0.279961\\
\color{red} 0.126899\color{black} \\
0.065738
\end{pmatrix}\end{equation*}
\begin{equation*}
\left[ \frac{{w}^{EM}_i}{{w}^{EM}_j} \right] =
\begin{pmatrix}
$\,\,$ 1 $\,\,$ & $\,\,$1.8838$\,\,$ & $\,\,$\color{red} 4.1561\color{black} $\,\,$ & $\,\,$8.0228$\,\,$ \\
$\,\,$0.5308$\,\,$ & $\,\,$ 1 $\,\,$ & $\,\,$\color{red} 2.2062\color{black} $\,\,$ & $\,\,$4.2588  $\,\,$ \\
$\,\,$\color{red} 0.2406\color{black} $\,\,$ & $\,\,$\color{red} 0.4533\color{black} $\,\,$ & $\,\,$ 1 $\,\,$ & $\,\,$\color{red} 1.9304\color{black}  $\,\,$ \\
$\,\,$0.1246$\,\,$ & $\,\,$0.2348$\,\,$ & $\,\,$\color{red} 0.5180\color{black} $\,\,$ & $\,\,$ 1  $\,\,$ \\
\end{pmatrix},
\end{equation*}

\begin{equation*}
\mathbf{w}^{\prime} =
\begin{pmatrix}
0.525000\\
0.278686\\
0.130876\\
0.065438
\end{pmatrix} =
0.995445\cdot
\begin{pmatrix}
0.527402\\
0.279961\\
\color{gr} 0.131475\color{black} \\
0.065738
\end{pmatrix},
\end{equation*}
\begin{equation*}
\left[ \frac{{w}^{\prime}_i}{{w}^{\prime}_j} \right] =
\begin{pmatrix}
$\,\,$ 1 $\,\,$ & $\,\,$1.8838$\,\,$ & $\,\,$\color{gr} 4.0114\color{black} $\,\,$ & $\,\,$8.0228$\,\,$ \\
$\,\,$0.5308$\,\,$ & $\,\,$ 1 $\,\,$ & $\,\,$\color{gr} 2.1294\color{black} $\,\,$ & $\,\,$4.2588  $\,\,$ \\
$\,\,$\color{gr} 0.2493\color{black} $\,\,$ & $\,\,$\color{gr} 0.4696\color{black} $\,\,$ & $\,\,$ 1 $\,\,$ & $\,\,$\color{gr} \color{blue} 2\color{black}  $\,\,$ \\
$\,\,$0.1246$\,\,$ & $\,\,$0.2348$\,\,$ & $\,\,$\color{gr} \color{blue}  1/2\color{black} $\,\,$ & $\,\,$ 1  $\,\,$ \\
\end{pmatrix},
\end{equation*}
\end{example}
\newpage
\begin{example}
\begin{equation*}
\mathbf{A} =
\begin{pmatrix}
$\,\,$ 1 $\,\,$ & $\,\,$3$\,\,$ & $\,\,$4$\,\,$ & $\,\,$5 $\,\,$ \\
$\,\,$ 1/3$\,\,$ & $\,\,$ 1 $\,\,$ & $\,\,$2$\,\,$ & $\,\,$8 $\,\,$ \\
$\,\,$ 1/4$\,\,$ & $\,\,$ 1/2$\,\,$ & $\,\,$ 1 $\,\,$ & $\,\,$2 $\,\,$ \\
$\,\,$ 1/5$\,\,$ & $\,\,$ 1/8$\,\,$ & $\,\,$ 1/2$\,\,$ & $\,\,$ 1  $\,\,$ \\
\end{pmatrix},
\qquad
\lambda_{\max} =
4.2162,
\qquad
CR = 0.0815
\end{equation*}

\begin{equation*}
\mathbf{w}^{EM} =
\begin{pmatrix}
0.523016\\
0.288963\\
\color{red} 0.124856\color{black} \\
0.063165
\end{pmatrix}\end{equation*}
\begin{equation*}
\left[ \frac{{w}^{EM}_i}{{w}^{EM}_j} \right] =
\begin{pmatrix}
$\,\,$ 1 $\,\,$ & $\,\,$1.8100$\,\,$ & $\,\,$\color{red} 4.1890\color{black} $\,\,$ & $\,\,$8.2802$\,\,$ \\
$\,\,$0.5525$\,\,$ & $\,\,$ 1 $\,\,$ & $\,\,$\color{red} 2.3144\color{black} $\,\,$ & $\,\,$4.5748  $\,\,$ \\
$\,\,$\color{red} 0.2387\color{black} $\,\,$ & $\,\,$\color{red} 0.4321\color{black} $\,\,$ & $\,\,$ 1 $\,\,$ & $\,\,$\color{red} 1.9767\color{black}  $\,\,$ \\
$\,\,$0.1208$\,\,$ & $\,\,$0.2186$\,\,$ & $\,\,$\color{red} 0.5059\color{black} $\,\,$ & $\,\,$ 1  $\,\,$ \\
\end{pmatrix},
\end{equation*}

\begin{equation*}
\mathbf{w}^{\prime} =
\begin{pmatrix}
0.522247\\
0.288538\\
0.126144\\
0.063072
\end{pmatrix} =
0.998529\cdot
\begin{pmatrix}
0.523016\\
0.288963\\
\color{gr} 0.126329\color{black} \\
0.063165
\end{pmatrix},
\end{equation*}
\begin{equation*}
\left[ \frac{{w}^{\prime}_i}{{w}^{\prime}_j} \right] =
\begin{pmatrix}
$\,\,$ 1 $\,\,$ & $\,\,$1.8100$\,\,$ & $\,\,$\color{gr} 4.1401\color{black} $\,\,$ & $\,\,$8.2802$\,\,$ \\
$\,\,$0.5525$\,\,$ & $\,\,$ 1 $\,\,$ & $\,\,$\color{gr} 2.2874\color{black} $\,\,$ & $\,\,$4.5748  $\,\,$ \\
$\,\,$\color{gr} 0.2415\color{black} $\,\,$ & $\,\,$\color{gr} 0.4372\color{black} $\,\,$ & $\,\,$ 1 $\,\,$ & $\,\,$\color{gr} \color{blue} 2\color{black}  $\,\,$ \\
$\,\,$0.1208$\,\,$ & $\,\,$0.2186$\,\,$ & $\,\,$\color{gr} \color{blue}  1/2\color{black} $\,\,$ & $\,\,$ 1  $\,\,$ \\
\end{pmatrix},
\end{equation*}
\end{example}
\newpage
\begin{example}
\begin{equation*}
\mathbf{A} =
\begin{pmatrix}
$\,\,$ 1 $\,\,$ & $\,\,$3$\,\,$ & $\,\,$4$\,\,$ & $\,\,$6 $\,\,$ \\
$\,\,$ 1/3$\,\,$ & $\,\,$ 1 $\,\,$ & $\,\,$2$\,\,$ & $\,\,$6 $\,\,$ \\
$\,\,$ 1/4$\,\,$ & $\,\,$ 1/2$\,\,$ & $\,\,$ 1 $\,\,$ & $\,\,$2 $\,\,$ \\
$\,\,$ 1/6$\,\,$ & $\,\,$ 1/6$\,\,$ & $\,\,$ 1/2$\,\,$ & $\,\,$ 1  $\,\,$ \\
\end{pmatrix},
\qquad
\lambda_{\max} =
4.1031,
\qquad
CR = 0.0389
\end{equation*}

\begin{equation*}
\mathbf{w}^{EM} =
\begin{pmatrix}
0.543921\\
0.264427\\
\color{red} 0.127666\color{black} \\
0.063986
\end{pmatrix}\end{equation*}
\begin{equation*}
\left[ \frac{{w}^{EM}_i}{{w}^{EM}_j} \right] =
\begin{pmatrix}
$\,\,$ 1 $\,\,$ & $\,\,$2.0570$\,\,$ & $\,\,$\color{red} 4.2605\color{black} $\,\,$ & $\,\,$8.5006$\,\,$ \\
$\,\,$0.4862$\,\,$ & $\,\,$ 1 $\,\,$ & $\,\,$\color{red} 2.0712\color{black} $\,\,$ & $\,\,$4.1326  $\,\,$ \\
$\,\,$\color{red} 0.2347\color{black} $\,\,$ & $\,\,$\color{red} 0.4828\color{black} $\,\,$ & $\,\,$ 1 $\,\,$ & $\,\,$\color{red} 1.9952\color{black}  $\,\,$ \\
$\,\,$0.1176$\,\,$ & $\,\,$0.2420$\,\,$ & $\,\,$\color{red} 0.5012\color{black} $\,\,$ & $\,\,$ 1  $\,\,$ \\
\end{pmatrix},
\end{equation*}

\begin{equation*}
\mathbf{w}^{\prime} =
\begin{pmatrix}
0.543754\\
0.264346\\
0.127933\\
0.063966
\end{pmatrix} =
0.999694\cdot
\begin{pmatrix}
0.543921\\
0.264427\\
\color{gr} 0.127972\color{black} \\
0.063986
\end{pmatrix},
\end{equation*}
\begin{equation*}
\left[ \frac{{w}^{\prime}_i}{{w}^{\prime}_j} \right] =
\begin{pmatrix}
$\,\,$ 1 $\,\,$ & $\,\,$2.0570$\,\,$ & $\,\,$\color{gr} 4.2503\color{black} $\,\,$ & $\,\,$8.5006$\,\,$ \\
$\,\,$0.4862$\,\,$ & $\,\,$ 1 $\,\,$ & $\,\,$\color{gr} 2.0663\color{black} $\,\,$ & $\,\,$4.1326  $\,\,$ \\
$\,\,$\color{gr} 0.2353\color{black} $\,\,$ & $\,\,$\color{gr} 0.4840\color{black} $\,\,$ & $\,\,$ 1 $\,\,$ & $\,\,$\color{gr} \color{blue} 2\color{black}  $\,\,$ \\
$\,\,$0.1176$\,\,$ & $\,\,$0.2420$\,\,$ & $\,\,$\color{gr} \color{blue}  1/2\color{black} $\,\,$ & $\,\,$ 1  $\,\,$ \\
\end{pmatrix},
\end{equation*}
\end{example}
\newpage
\begin{example}
\begin{equation*}
\mathbf{A} =
\begin{pmatrix}
$\,\,$ 1 $\,\,$ & $\,\,$3$\,\,$ & $\,\,$4$\,\,$ & $\,\,$6 $\,\,$ \\
$\,\,$ 1/3$\,\,$ & $\,\,$ 1 $\,\,$ & $\,\,$4$\,\,$ & $\,\,$3 $\,\,$ \\
$\,\,$ 1/4$\,\,$ & $\,\,$ 1/4$\,\,$ & $\,\,$ 1 $\,\,$ & $\,\,$1 $\,\,$ \\
$\,\,$ 1/6$\,\,$ & $\,\,$ 1/3$\,\,$ & $\,\,$ 1 $\,\,$ & $\,\,$ 1  $\,\,$ \\
\end{pmatrix},
\qquad
\lambda_{\max} =
4.1031,
\qquad
CR = 0.0389
\end{equation*}

\begin{equation*}
\mathbf{w}^{EM} =
\begin{pmatrix}
0.550567\\
0.266452\\
0.094375\\
\color{red} 0.088605\color{black}
\end{pmatrix}\end{equation*}
\begin{equation*}
\left[ \frac{{w}^{EM}_i}{{w}^{EM}_j} \right] =
\begin{pmatrix}
$\,\,$ 1 $\,\,$ & $\,\,$2.0663$\,\,$ & $\,\,$5.8338$\,\,$ & $\,\,$\color{red} 6.2137\color{black} $\,\,$ \\
$\,\,$0.4840$\,\,$ & $\,\,$ 1 $\,\,$ & $\,\,$2.8233$\,\,$ & $\,\,$\color{red} 3.0072\color{black}   $\,\,$ \\
$\,\,$0.1714$\,\,$ & $\,\,$0.3542$\,\,$ & $\,\,$ 1 $\,\,$ & $\,\,$\color{red} 1.0651\color{black}  $\,\,$ \\
$\,\,$\color{red} 0.1609\color{black} $\,\,$ & $\,\,$\color{red} 0.3325\color{black} $\,\,$ & $\,\,$\color{red} 0.9389\color{black} $\,\,$ & $\,\,$ 1  $\,\,$ \\
\end{pmatrix},
\end{equation*}

\begin{equation*}
\mathbf{w}^{\prime} =
\begin{pmatrix}
0.550450\\
0.266396\\
0.094355\\
0.088799
\end{pmatrix} =
0.999788\cdot
\begin{pmatrix}
0.550567\\
0.266452\\
0.094375\\
\color{gr} 0.088817\color{black}
\end{pmatrix},
\end{equation*}
\begin{equation*}
\left[ \frac{{w}^{\prime}_i}{{w}^{\prime}_j} \right] =
\begin{pmatrix}
$\,\,$ 1 $\,\,$ & $\,\,$2.0663$\,\,$ & $\,\,$5.8338$\,\,$ & $\,\,$\color{gr} 6.1989\color{black} $\,\,$ \\
$\,\,$0.4840$\,\,$ & $\,\,$ 1 $\,\,$ & $\,\,$2.8233$\,\,$ & $\,\,$\color{gr} \color{blue} 3\color{black}   $\,\,$ \\
$\,\,$0.1714$\,\,$ & $\,\,$0.3542$\,\,$ & $\,\,$ 1 $\,\,$ & $\,\,$\color{gr} 1.0626\color{black}  $\,\,$ \\
$\,\,$\color{gr} 0.1613\color{black} $\,\,$ & $\,\,$\color{gr} \color{blue}  1/3\color{black} $\,\,$ & $\,\,$\color{gr} 0.9411\color{black} $\,\,$ & $\,\,$ 1  $\,\,$ \\
\end{pmatrix},
\end{equation*}
\end{example}
\newpage
\begin{example}
\begin{equation*}
\mathbf{A} =
\begin{pmatrix}
$\,\,$ 1 $\,\,$ & $\,\,$3$\,\,$ & $\,\,$4$\,\,$ & $\,\,$6 $\,\,$ \\
$\,\,$ 1/3$\,\,$ & $\,\,$ 1 $\,\,$ & $\,\,$5$\,\,$ & $\,\,$3 $\,\,$ \\
$\,\,$ 1/4$\,\,$ & $\,\,$ 1/5$\,\,$ & $\,\,$ 1 $\,\,$ & $\,\,$1 $\,\,$ \\
$\,\,$ 1/6$\,\,$ & $\,\,$ 1/3$\,\,$ & $\,\,$ 1 $\,\,$ & $\,\,$ 1  $\,\,$ \\
\end{pmatrix},
\qquad
\lambda_{\max} =
4.1502,
\qquad
CR = 0.0566
\end{equation*}

\begin{equation*}
\mathbf{w}^{EM} =
\begin{pmatrix}
0.544409\\
0.280517\\
0.088497\\
\color{red} 0.086578\color{black}
\end{pmatrix}\end{equation*}
\begin{equation*}
\left[ \frac{{w}^{EM}_i}{{w}^{EM}_j} \right] =
\begin{pmatrix}
$\,\,$ 1 $\,\,$ & $\,\,$1.9407$\,\,$ & $\,\,$6.1517$\,\,$ & $\,\,$\color{red} 6.2881\color{black} $\,\,$ \\
$\,\,$0.5153$\,\,$ & $\,\,$ 1 $\,\,$ & $\,\,$3.1698$\,\,$ & $\,\,$\color{red} 3.2401\color{black}   $\,\,$ \\
$\,\,$0.1626$\,\,$ & $\,\,$0.3155$\,\,$ & $\,\,$ 1 $\,\,$ & $\,\,$\color{red} 1.0222\color{black}  $\,\,$ \\
$\,\,$\color{red} 0.1590\color{black} $\,\,$ & $\,\,$\color{red} 0.3086\color{black} $\,\,$ & $\,\,$\color{red} 0.9783\color{black} $\,\,$ & $\,\,$ 1  $\,\,$ \\
\end{pmatrix},
\end{equation*}

\begin{equation*}
\mathbf{w}^{\prime} =
\begin{pmatrix}
0.543366\\
0.279979\\
0.088327\\
0.088327
\end{pmatrix} =
0.998084\cdot
\begin{pmatrix}
0.544409\\
0.280517\\
0.088497\\
\color{gr} 0.088497\color{black}
\end{pmatrix},
\end{equation*}
\begin{equation*}
\left[ \frac{{w}^{\prime}_i}{{w}^{\prime}_j} \right] =
\begin{pmatrix}
$\,\,$ 1 $\,\,$ & $\,\,$1.9407$\,\,$ & $\,\,$6.1517$\,\,$ & $\,\,$\color{gr} 6.1517\color{black} $\,\,$ \\
$\,\,$0.5153$\,\,$ & $\,\,$ 1 $\,\,$ & $\,\,$3.1698$\,\,$ & $\,\,$\color{gr} 3.1698\color{black}   $\,\,$ \\
$\,\,$0.1626$\,\,$ & $\,\,$0.3155$\,\,$ & $\,\,$ 1 $\,\,$ & $\,\,$\color{gr} \color{blue} 1\color{black}  $\,\,$ \\
$\,\,$\color{gr} 0.1626\color{black} $\,\,$ & $\,\,$\color{gr} 0.3155\color{black} $\,\,$ & $\,\,$\color{gr} \color{blue} 1\color{black} $\,\,$ & $\,\,$ 1  $\,\,$ \\
\end{pmatrix},
\end{equation*}
\end{example}
\newpage
\begin{example}
\begin{equation*}
\mathbf{A} =
\begin{pmatrix}
$\,\,$ 1 $\,\,$ & $\,\,$3$\,\,$ & $\,\,$4$\,\,$ & $\,\,$6 $\,\,$ \\
$\,\,$ 1/3$\,\,$ & $\,\,$ 1 $\,\,$ & $\,\,$7$\,\,$ & $\,\,$4 $\,\,$ \\
$\,\,$ 1/4$\,\,$ & $\,\,$ 1/7$\,\,$ & $\,\,$ 1 $\,\,$ & $\,\,$1 $\,\,$ \\
$\,\,$ 1/6$\,\,$ & $\,\,$ 1/4$\,\,$ & $\,\,$ 1 $\,\,$ & $\,\,$ 1  $\,\,$ \\
\end{pmatrix},
\qquad
\lambda_{\max} =
4.2421,
\qquad
CR = 0.0913
\end{equation*}

\begin{equation*}
\mathbf{w}^{EM} =
\begin{pmatrix}
0.529524\\
0.316598\\
0.078142\\
\color{red} 0.075736\color{black}
\end{pmatrix}\end{equation*}
\begin{equation*}
\left[ \frac{{w}^{EM}_i}{{w}^{EM}_j} \right] =
\begin{pmatrix}
$\,\,$ 1 $\,\,$ & $\,\,$1.6725$\,\,$ & $\,\,$6.7764$\,\,$ & $\,\,$\color{red} 6.9917\color{black} $\,\,$ \\
$\,\,$0.5979$\,\,$ & $\,\,$ 1 $\,\,$ & $\,\,$4.0516$\,\,$ & $\,\,$\color{red} 4.1803\color{black}   $\,\,$ \\
$\,\,$0.1476$\,\,$ & $\,\,$0.2468$\,\,$ & $\,\,$ 1 $\,\,$ & $\,\,$\color{red} 1.0318\color{black}  $\,\,$ \\
$\,\,$\color{red} 0.1430\color{black} $\,\,$ & $\,\,$\color{red} 0.2392\color{black} $\,\,$ & $\,\,$\color{red} 0.9692\color{black} $\,\,$ & $\,\,$ 1  $\,\,$ \\
\end{pmatrix},
\end{equation*}

\begin{equation*}
\mathbf{w}^{\prime} =
\begin{pmatrix}
0.528253\\
0.315838\\
0.077954\\
0.077954
\end{pmatrix} =
0.997600\cdot
\begin{pmatrix}
0.529524\\
0.316598\\
0.078142\\
\color{gr} 0.078142\color{black}
\end{pmatrix},
\end{equation*}
\begin{equation*}
\left[ \frac{{w}^{\prime}_i}{{w}^{\prime}_j} \right] =
\begin{pmatrix}
$\,\,$ 1 $\,\,$ & $\,\,$1.6725$\,\,$ & $\,\,$6.7764$\,\,$ & $\,\,$\color{gr} 6.7764\color{black} $\,\,$ \\
$\,\,$0.5979$\,\,$ & $\,\,$ 1 $\,\,$ & $\,\,$4.0516$\,\,$ & $\,\,$\color{gr} 4.0516\color{black}   $\,\,$ \\
$\,\,$0.1476$\,\,$ & $\,\,$0.2468$\,\,$ & $\,\,$ 1 $\,\,$ & $\,\,$\color{gr} \color{blue} 1\color{black}  $\,\,$ \\
$\,\,$\color{gr} 0.1476\color{black} $\,\,$ & $\,\,$\color{gr} 0.2468\color{black} $\,\,$ & $\,\,$\color{gr} \color{blue} 1\color{black} $\,\,$ & $\,\,$ 1  $\,\,$ \\
\end{pmatrix},
\end{equation*}
\end{example}
\newpage
\begin{example}
\begin{equation*}
\mathbf{A} =
\begin{pmatrix}
$\,\,$ 1 $\,\,$ & $\,\,$3$\,\,$ & $\,\,$4$\,\,$ & $\,\,$7 $\,\,$ \\
$\,\,$ 1/3$\,\,$ & $\,\,$ 1 $\,\,$ & $\,\,$2$\,\,$ & $\,\,$8 $\,\,$ \\
$\,\,$ 1/4$\,\,$ & $\,\,$ 1/2$\,\,$ & $\,\,$ 1 $\,\,$ & $\,\,$3 $\,\,$ \\
$\,\,$ 1/7$\,\,$ & $\,\,$ 1/8$\,\,$ & $\,\,$ 1/3$\,\,$ & $\,\,$ 1  $\,\,$ \\
\end{pmatrix},
\qquad
\lambda_{\max} =
4.1317,
\qquad
CR = 0.0496
\end{equation*}

\begin{equation*}
\mathbf{w}^{EM} =
\begin{pmatrix}
0.543827\\
0.271548\\
\color{red} 0.134646\color{black} \\
0.049978
\end{pmatrix}\end{equation*}
\begin{equation*}
\left[ \frac{{w}^{EM}_i}{{w}^{EM}_j} \right] =
\begin{pmatrix}
$\,\,$ 1 $\,\,$ & $\,\,$2.0027$\,\,$ & $\,\,$\color{red} 4.0389\color{black} $\,\,$ & $\,\,$10.8812$\,\,$ \\
$\,\,$0.4993$\,\,$ & $\,\,$ 1 $\,\,$ & $\,\,$\color{red} 2.0168\color{black} $\,\,$ & $\,\,$5.4333  $\,\,$ \\
$\,\,$\color{red} 0.2476\color{black} $\,\,$ & $\,\,$\color{red} 0.4958\color{black} $\,\,$ & $\,\,$ 1 $\,\,$ & $\,\,$\color{red} 2.6941\color{black}  $\,\,$ \\
$\,\,$0.0919$\,\,$ & $\,\,$0.1840$\,\,$ & $\,\,$\color{red} 0.3712\color{black} $\,\,$ & $\,\,$ 1  $\,\,$ \\
\end{pmatrix},
\end{equation*}

\begin{equation*}
\mathbf{w}^{\prime} =
\begin{pmatrix}
0.543214\\
0.271242\\
0.135621\\
0.049922
\end{pmatrix} =
0.998874\cdot
\begin{pmatrix}
0.543827\\
0.271548\\
\color{gr} 0.135774\color{black} \\
0.049978
\end{pmatrix},
\end{equation*}
\begin{equation*}
\left[ \frac{{w}^{\prime}_i}{{w}^{\prime}_j} \right] =
\begin{pmatrix}
$\,\,$ 1 $\,\,$ & $\,\,$2.0027$\,\,$ & $\,\,$\color{gr} 4.0054\color{black} $\,\,$ & $\,\,$10.8812$\,\,$ \\
$\,\,$0.4993$\,\,$ & $\,\,$ 1 $\,\,$ & $\,\,$\color{gr} \color{blue} 2\color{black} $\,\,$ & $\,\,$5.4333  $\,\,$ \\
$\,\,$\color{gr} 0.2497\color{black} $\,\,$ & $\,\,$\color{gr} \color{blue}  1/2\color{black} $\,\,$ & $\,\,$ 1 $\,\,$ & $\,\,$\color{gr} 2.7167\color{black}  $\,\,$ \\
$\,\,$0.0919$\,\,$ & $\,\,$0.1840$\,\,$ & $\,\,$\color{gr} 0.3681\color{black} $\,\,$ & $\,\,$ 1  $\,\,$ \\
\end{pmatrix},
\end{equation*}
\end{example}
\newpage
\begin{example}
\begin{equation*}
\mathbf{A} =
\begin{pmatrix}
$\,\,$ 1 $\,\,$ & $\,\,$3$\,\,$ & $\,\,$4$\,\,$ & $\,\,$7 $\,\,$ \\
$\,\,$ 1/3$\,\,$ & $\,\,$ 1 $\,\,$ & $\,\,$2$\,\,$ & $\,\,$9 $\,\,$ \\
$\,\,$ 1/4$\,\,$ & $\,\,$ 1/2$\,\,$ & $\,\,$ 1 $\,\,$ & $\,\,$3 $\,\,$ \\
$\,\,$ 1/7$\,\,$ & $\,\,$ 1/9$\,\,$ & $\,\,$ 1/3$\,\,$ & $\,\,$ 1  $\,\,$ \\
\end{pmatrix},
\qquad
\lambda_{\max} =
4.1571,
\qquad
CR = 0.0593
\end{equation*}

\begin{equation*}
\mathbf{w}^{EM} =
\begin{pmatrix}
0.540168\\
0.278766\\
\color{red} 0.132793\color{black} \\
0.048273
\end{pmatrix}\end{equation*}
\begin{equation*}
\left[ \frac{{w}^{EM}_i}{{w}^{EM}_j} \right] =
\begin{pmatrix}
$\,\,$ 1 $\,\,$ & $\,\,$1.9377$\,\,$ & $\,\,$\color{red} 4.0677\color{black} $\,\,$ & $\,\,$11.1898$\,\,$ \\
$\,\,$0.5161$\,\,$ & $\,\,$ 1 $\,\,$ & $\,\,$\color{red} 2.0993\color{black} $\,\,$ & $\,\,$5.7747  $\,\,$ \\
$\,\,$\color{red} 0.2458\color{black} $\,\,$ & $\,\,$\color{red} 0.4764\color{black} $\,\,$ & $\,\,$ 1 $\,\,$ & $\,\,$\color{red} 2.7509\color{black}  $\,\,$ \\
$\,\,$0.0894$\,\,$ & $\,\,$0.1732$\,\,$ & $\,\,$\color{red} 0.3635\color{black} $\,\,$ & $\,\,$ 1  $\,\,$ \\
\end{pmatrix},
\end{equation*}

\begin{equation*}
\mathbf{w}^{\prime} =
\begin{pmatrix}
0.538956\\
0.278140\\
0.134739\\
0.048165
\end{pmatrix} =
0.997756\cdot
\begin{pmatrix}
0.540168\\
0.278766\\
\color{gr} 0.135042\color{black} \\
0.048273
\end{pmatrix},
\end{equation*}
\begin{equation*}
\left[ \frac{{w}^{\prime}_i}{{w}^{\prime}_j} \right] =
\begin{pmatrix}
$\,\,$ 1 $\,\,$ & $\,\,$1.9377$\,\,$ & $\,\,$\color{gr} \color{blue} 4\color{black} $\,\,$ & $\,\,$11.1898$\,\,$ \\
$\,\,$0.5161$\,\,$ & $\,\,$ 1 $\,\,$ & $\,\,$\color{gr} 2.0643\color{black} $\,\,$ & $\,\,$5.7747  $\,\,$ \\
$\,\,$\color{gr} \color{blue}  1/4\color{black} $\,\,$ & $\,\,$\color{gr} 0.4844\color{black} $\,\,$ & $\,\,$ 1 $\,\,$ & $\,\,$\color{gr} 2.7974\color{black}  $\,\,$ \\
$\,\,$0.0894$\,\,$ & $\,\,$0.1732$\,\,$ & $\,\,$\color{gr} 0.3575\color{black} $\,\,$ & $\,\,$ 1  $\,\,$ \\
\end{pmatrix},
\end{equation*}
\end{example}
\newpage
\begin{example}
\begin{equation*}
\mathbf{A} =
\begin{pmatrix}
$\,\,$ 1 $\,\,$ & $\,\,$3$\,\,$ & $\,\,$4$\,\,$ & $\,\,$8 $\,\,$ \\
$\,\,$ 1/3$\,\,$ & $\,\,$ 1 $\,\,$ & $\,\,$2$\,\,$ & $\,\,$2 $\,\,$ \\
$\,\,$ 1/4$\,\,$ & $\,\,$ 1/2$\,\,$ & $\,\,$ 1 $\,\,$ & $\,\,$3 $\,\,$ \\
$\,\,$ 1/8$\,\,$ & $\,\,$ 1/2$\,\,$ & $\,\,$ 1/3$\,\,$ & $\,\,$ 1  $\,\,$ \\
\end{pmatrix},
\qquad
\lambda_{\max} =
4.1031,
\qquad
CR = 0.0389
\end{equation*}

\begin{equation*}
\mathbf{w}^{EM} =
\begin{pmatrix}
\color{red} 0.574962\color{black} \\
0.204135\\
0.148861\\
0.072042
\end{pmatrix}\end{equation*}
\begin{equation*}
\left[ \frac{{w}^{EM}_i}{{w}^{EM}_j} \right] =
\begin{pmatrix}
$\,\,$ 1 $\,\,$ & $\,\,$\color{red} 2.8166\color{black} $\,\,$ & $\,\,$\color{red} 3.8624\color{black} $\,\,$ & $\,\,$\color{red} 7.9809\color{black} $\,\,$ \\
$\,\,$\color{red} 0.3550\color{black} $\,\,$ & $\,\,$ 1 $\,\,$ & $\,\,$1.3713$\,\,$ & $\,\,$2.8335  $\,\,$ \\
$\,\,$\color{red} 0.2589\color{black} $\,\,$ & $\,\,$0.7292$\,\,$ & $\,\,$ 1 $\,\,$ & $\,\,$2.0663 $\,\,$ \\
$\,\,$\color{red} 0.1253\color{black} $\,\,$ & $\,\,$0.3529$\,\,$ & $\,\,$0.4840$\,\,$ & $\,\,$ 1  $\,\,$ \\
\end{pmatrix},
\end{equation*}

\begin{equation*}
\mathbf{w}^{\prime} =
\begin{pmatrix}
0.575547\\
0.203854\\
0.148656\\
0.071943
\end{pmatrix} =
0.998624\cdot
\begin{pmatrix}
\color{gr} 0.576340\color{black} \\
0.204135\\
0.148861\\
0.072042
\end{pmatrix},
\end{equation*}
\begin{equation*}
\left[ \frac{{w}^{\prime}_i}{{w}^{\prime}_j} \right] =
\begin{pmatrix}
$\,\,$ 1 $\,\,$ & $\,\,$\color{gr} 2.8233\color{black} $\,\,$ & $\,\,$\color{gr} 3.8717\color{black} $\,\,$ & $\,\,$\color{gr} \color{blue} 8\color{black} $\,\,$ \\
$\,\,$\color{gr} 0.3542\color{black} $\,\,$ & $\,\,$ 1 $\,\,$ & $\,\,$1.3713$\,\,$ & $\,\,$2.8335  $\,\,$ \\
$\,\,$\color{gr} 0.2583\color{black} $\,\,$ & $\,\,$0.7292$\,\,$ & $\,\,$ 1 $\,\,$ & $\,\,$2.0663 $\,\,$ \\
$\,\,$\color{gr} \color{blue}  1/8\color{black} $\,\,$ & $\,\,$0.3529$\,\,$ & $\,\,$0.4840$\,\,$ & $\,\,$ 1  $\,\,$ \\
\end{pmatrix},
\end{equation*}
\end{example}
\newpage
\begin{example}
\begin{equation*}
\mathbf{A} =
\begin{pmatrix}
$\,\,$ 1 $\,\,$ & $\,\,$3$\,\,$ & $\,\,$4$\,\,$ & $\,\,$8 $\,\,$ \\
$\,\,$ 1/3$\,\,$ & $\,\,$ 1 $\,\,$ & $\,\,$2$\,\,$ & $\,\,$8 $\,\,$ \\
$\,\,$ 1/4$\,\,$ & $\,\,$ 1/2$\,\,$ & $\,\,$ 1 $\,\,$ & $\,\,$3 $\,\,$ \\
$\,\,$ 1/8$\,\,$ & $\,\,$ 1/8$\,\,$ & $\,\,$ 1/3$\,\,$ & $\,\,$ 1  $\,\,$ \\
\end{pmatrix},
\qquad
\lambda_{\max} =
4.1031,
\qquad
CR = 0.0389
\end{equation*}

\begin{equation*}
\mathbf{w}^{EM} =
\begin{pmatrix}
0.552160\\
0.267223\\
\color{red} 0.133292\color{black} \\
0.047324
\end{pmatrix}\end{equation*}
\begin{equation*}
\left[ \frac{{w}^{EM}_i}{{w}^{EM}_j} \right] =
\begin{pmatrix}
$\,\,$ 1 $\,\,$ & $\,\,$2.0663$\,\,$ & $\,\,$\color{red} 4.1425\color{black} $\,\,$ & $\,\,$11.6676$\,\,$ \\
$\,\,$0.4840$\,\,$ & $\,\,$ 1 $\,\,$ & $\,\,$\color{red} 2.0048\color{black} $\,\,$ & $\,\,$5.6467  $\,\,$ \\
$\,\,$\color{red} 0.2414\color{black} $\,\,$ & $\,\,$\color{red} 0.4988\color{black} $\,\,$ & $\,\,$ 1 $\,\,$ & $\,\,$\color{red} 2.8166\color{black}  $\,\,$ \\
$\,\,$0.0857$\,\,$ & $\,\,$0.1771$\,\,$ & $\,\,$\color{red} 0.3550\color{black} $\,\,$ & $\,\,$ 1  $\,\,$ \\
\end{pmatrix},
\end{equation*}

\begin{equation*}
\mathbf{w}^{\prime} =
\begin{pmatrix}
0.551984\\
0.267138\\
0.133569\\
0.047309
\end{pmatrix} =
0.999681\cdot
\begin{pmatrix}
0.552160\\
0.267223\\
\color{gr} 0.133612\color{black} \\
0.047324
\end{pmatrix},
\end{equation*}
\begin{equation*}
\left[ \frac{{w}^{\prime}_i}{{w}^{\prime}_j} \right] =
\begin{pmatrix}
$\,\,$ 1 $\,\,$ & $\,\,$2.0663$\,\,$ & $\,\,$\color{gr} 4.1326\color{black} $\,\,$ & $\,\,$11.6676$\,\,$ \\
$\,\,$0.4840$\,\,$ & $\,\,$ 1 $\,\,$ & $\,\,$\color{gr} \color{blue} 2\color{black} $\,\,$ & $\,\,$5.6467  $\,\,$ \\
$\,\,$\color{gr} 0.2420\color{black} $\,\,$ & $\,\,$\color{gr} \color{blue}  1/2\color{black} $\,\,$ & $\,\,$ 1 $\,\,$ & $\,\,$\color{gr} 2.8233\color{black}  $\,\,$ \\
$\,\,$0.0857$\,\,$ & $\,\,$0.1771$\,\,$ & $\,\,$\color{gr} 0.3542\color{black} $\,\,$ & $\,\,$ 1  $\,\,$ \\
\end{pmatrix},
\end{equation*}
\end{example}
\newpage
\begin{example}  % Example 1.162
\begin{equation*}
\mathbf{A} =
\begin{pmatrix}
$\,\,$ 1 $\,\,$ & $\,\,$3$\,\,$ & $\,\,$4$\,\,$ & $\,\,$8 $\,\,$ \\
$\,\,$ 1/3$\,\,$ & $\,\,$ 1 $\,\,$ & $\,\,$2$\,\,$ & $\,\,$9 $\,\,$ \\
$\,\,$ 1/4$\,\,$ & $\,\,$ 1/2$\,\,$ & $\,\,$ 1 $\,\,$ & $\,\,$3 $\,\,$ \\
$\,\,$ 1/8$\,\,$ & $\,\,$ 1/9$\,\,$ & $\,\,$ 1/3$\,\,$ & $\,\,$ 1  $\,\,$ \\
\end{pmatrix},
\qquad
\lambda_{\max} =
4.1263,
\qquad
CR = 0.0476
\end{equation*}

\begin{equation*}
\mathbf{w}^{EM} =
\begin{pmatrix}
0.548476\\
0.274238\\
\color{red} 0.131580\color{black} \\
0.045706
\end{pmatrix}\end{equation*}
\begin{equation*}
\left[ \frac{{w}^{EM}_i}{{w}^{EM}_j} \right] =
\begin{pmatrix}
$\,\,$ 1 $\,\,$ & $\,\,$2$\,\,$ & $\,\,$\color{red} 4.1684\color{black} $\,\,$ & $\,\,$12$\,\,$ \\
$\,\,$1/2$\,\,$ & $\,\,$ 1 $\,\,$ & $\,\,$\color{red} 2.0842\color{black} $\,\,$ & $\,\,$6  $\,\,$ \\
$\,\,$\color{red} 0.2399\color{black} $\,\,$ & $\,\,$\color{red} 0.4798\color{black} $\,\,$ & $\,\,$ 1 $\,\,$ & $\,\,$\color{red} 2.8788\color{black}  $\,\,$ \\
$\,\,$1/12$\,\,$ & $\,\,$1/6$\,\,$ & $\,\,$\color{red} 0.3474\color{black} $\,\,$ & $\,\,$ 1  $\,\,$ \\
\end{pmatrix},
\end{equation*}

\begin{equation*}
\mathbf{w}^{\prime} =
\begin{pmatrix}
0.545455\\
0.272727\\
0.136364\\
0.045455
\end{pmatrix} =
0.994492\cdot
\begin{pmatrix}
0.548476\\
0.274238\\
\color{gr} 0.137119\color{black} \\
0.045706
\end{pmatrix},
\end{equation*}
\begin{equation*}
\left[ \frac{{w}^{\prime}_i}{{w}^{\prime}_j} \right] =
\begin{pmatrix}
$\,\,$ 1 $\,\,$ & $\,\,$2$\,\,$ & $\,\,$\color{blue} 4\color{black} $\,\,$ & $\,\,$12$\,\,$ \\
$\,\,$1/2$\,\,$ & $\,\,$ 1 $\,\,$ & $\,\,$\color{blue} 2\color{black} $\,\,$ & $\,\,$6  $\,\,$ \\
$\,\,$\color{blue} 1/4\color{black} $\,\,$ & $\,\,$\color{blue} 1/2\color{black} $\,\,$ & $\,\,$ 1 $\,\,$ & $\,\,$\color{gr} \color{blue} 3\color{black}  $\,\,$ \\
$\,\,$1/12$\,\,$ & $\,\,$1/6$\,\,$ & $\,\,$\color{gr} \color{blue}  1/3\color{black} $\,\,$ & $\,\,$ 1  $\,\,$ \\
\end{pmatrix},
\end{equation*}
\end{example}
\newpage
\begin{example}
\begin{equation*}
\mathbf{A} =
\begin{pmatrix}
$\,\,$ 1 $\,\,$ & $\,\,$3$\,\,$ & $\,\,$4$\,\,$ & $\,\,$9 $\,\,$ \\
$\,\,$ 1/3$\,\,$ & $\,\,$ 1 $\,\,$ & $\,\,$2$\,\,$ & $\,\,$2 $\,\,$ \\
$\,\,$ 1/4$\,\,$ & $\,\,$ 1/2$\,\,$ & $\,\,$ 1 $\,\,$ & $\,\,$3 $\,\,$ \\
$\,\,$ 1/9$\,\,$ & $\,\,$ 1/2$\,\,$ & $\,\,$ 1/3$\,\,$ & $\,\,$ 1  $\,\,$ \\
\end{pmatrix},
\qquad
\lambda_{\max} =
4.1031,
\qquad
CR = 0.0389
\end{equation*}

\begin{equation*}
\mathbf{w}^{EM} =
\begin{pmatrix}
\color{red} 0.583378\color{black} \\
0.201386\\
0.146194\\
0.069041
\end{pmatrix}\end{equation*}
\begin{equation*}
\left[ \frac{{w}^{EM}_i}{{w}^{EM}_j} \right] =
\begin{pmatrix}
$\,\,$ 1 $\,\,$ & $\,\,$\color{red} 2.8968\color{black} $\,\,$ & $\,\,$\color{red} 3.9904\color{black} $\,\,$ & $\,\,$\color{red} 8.4497\color{black} $\,\,$ \\
$\,\,$\color{red} 0.3452\color{black} $\,\,$ & $\,\,$ 1 $\,\,$ & $\,\,$1.3775$\,\,$ & $\,\,$2.9169  $\,\,$ \\
$\,\,$\color{red} 0.2506\color{black} $\,\,$ & $\,\,$0.7259$\,\,$ & $\,\,$ 1 $\,\,$ & $\,\,$2.1175 $\,\,$ \\
$\,\,$\color{red} 0.1183\color{black} $\,\,$ & $\,\,$0.3428$\,\,$ & $\,\,$0.4723$\,\,$ & $\,\,$ 1  $\,\,$ \\
\end{pmatrix},
\end{equation*}

\begin{equation*}
\mathbf{w}^{\prime} =
\begin{pmatrix}
0.583960\\
0.201105\\
0.145990\\
0.068945
\end{pmatrix} =
0.998604\cdot
\begin{pmatrix}
\color{gr} 0.584777\color{black} \\
0.201386\\
0.146194\\
0.069041
\end{pmatrix},
\end{equation*}
\begin{equation*}
\left[ \frac{{w}^{\prime}_i}{{w}^{\prime}_j} \right] =
\begin{pmatrix}
$\,\,$ 1 $\,\,$ & $\,\,$\color{gr} 2.9038\color{black} $\,\,$ & $\,\,$\color{gr} \color{blue} 4\color{black} $\,\,$ & $\,\,$\color{gr} 8.4700\color{black} $\,\,$ \\
$\,\,$\color{gr} 0.3444\color{black} $\,\,$ & $\,\,$ 1 $\,\,$ & $\,\,$1.3775$\,\,$ & $\,\,$2.9169  $\,\,$ \\
$\,\,$\color{gr} \color{blue}  1/4\color{black} $\,\,$ & $\,\,$0.7259$\,\,$ & $\,\,$ 1 $\,\,$ & $\,\,$2.1175 $\,\,$ \\
$\,\,$\color{gr} 0.1181\color{black} $\,\,$ & $\,\,$0.3428$\,\,$ & $\,\,$0.4723$\,\,$ & $\,\,$ 1  $\,\,$ \\
\end{pmatrix},
\end{equation*}
\end{example}
\newpage
\begin{example}
\begin{equation*}
\mathbf{A} =
\begin{pmatrix}
$\,\,$ 1 $\,\,$ & $\,\,$3$\,\,$ & $\,\,$4$\,\,$ & $\,\,$9 $\,\,$ \\
$\,\,$ 1/3$\,\,$ & $\,\,$ 1 $\,\,$ & $\,\,$2$\,\,$ & $\,\,$2 $\,\,$ \\
$\,\,$ 1/4$\,\,$ & $\,\,$ 1/2$\,\,$ & $\,\,$ 1 $\,\,$ & $\,\,$4 $\,\,$ \\
$\,\,$ 1/9$\,\,$ & $\,\,$ 1/2$\,\,$ & $\,\,$ 1/4$\,\,$ & $\,\,$ 1  $\,\,$ \\
\end{pmatrix},
\qquad
\lambda_{\max} =
4.1664,
\qquad
CR = 0.0627
\end{equation*}

\begin{equation*}
\mathbf{w}^{EM} =
\begin{pmatrix}
\color{red} 0.575065\color{black} \\
0.201602\\
0.158782\\
0.064550
\end{pmatrix}\end{equation*}
\begin{equation*}
\left[ \frac{{w}^{EM}_i}{{w}^{EM}_j} \right] =
\begin{pmatrix}
$\,\,$ 1 $\,\,$ & $\,\,$\color{red} 2.8525\color{black} $\,\,$ & $\,\,$\color{red} 3.6217\color{black} $\,\,$ & $\,\,$\color{red} 8.9088\color{black} $\,\,$ \\
$\,\,$\color{red} 0.3506\color{black} $\,\,$ & $\,\,$ 1 $\,\,$ & $\,\,$1.2697$\,\,$ & $\,\,$3.1232  $\,\,$ \\
$\,\,$\color{red} 0.2761\color{black} $\,\,$ & $\,\,$0.7876$\,\,$ & $\,\,$ 1 $\,\,$ & $\,\,$2.4598 $\,\,$ \\
$\,\,$\color{red} 0.1122\color{black} $\,\,$ & $\,\,$0.3202$\,\,$ & $\,\,$0.4065$\,\,$ & $\,\,$ 1  $\,\,$ \\
\end{pmatrix},
\end{equation*}

\begin{equation*}
\mathbf{w}^{\prime} =
\begin{pmatrix}
0.577553\\
0.200422\\
0.157853\\
0.064173
\end{pmatrix} =
0.994146\cdot
\begin{pmatrix}
\color{gr} 0.580954\color{black} \\
0.201602\\
0.158782\\
0.064550
\end{pmatrix},
\end{equation*}
\begin{equation*}
\left[ \frac{{w}^{\prime}_i}{{w}^{\prime}_j} \right] =
\begin{pmatrix}
$\,\,$ 1 $\,\,$ & $\,\,$\color{gr} 2.8817\color{black} $\,\,$ & $\,\,$\color{gr} 3.6588\color{black} $\,\,$ & $\,\,$\color{gr} \color{blue} 9\color{black} $\,\,$ \\
$\,\,$\color{gr} 0.3470\color{black} $\,\,$ & $\,\,$ 1 $\,\,$ & $\,\,$1.2697$\,\,$ & $\,\,$3.1232  $\,\,$ \\
$\,\,$\color{gr} 0.2733\color{black} $\,\,$ & $\,\,$0.7876$\,\,$ & $\,\,$ 1 $\,\,$ & $\,\,$2.4598 $\,\,$ \\
$\,\,$\color{gr} \color{blue}  1/9\color{black} $\,\,$ & $\,\,$0.3202$\,\,$ & $\,\,$0.4065$\,\,$ & $\,\,$ 1  $\,\,$ \\
\end{pmatrix},
\end{equation*}
\end{example}
\newpage
\begin{example}
\begin{equation*}
\mathbf{A} =
\begin{pmatrix}
$\,\,$ 1 $\,\,$ & $\,\,$3$\,\,$ & $\,\,$4$\,\,$ & $\,\,$9 $\,\,$ \\
$\,\,$ 1/3$\,\,$ & $\,\,$ 1 $\,\,$ & $\,\,$2$\,\,$ & $\,\,$9 $\,\,$ \\
$\,\,$ 1/4$\,\,$ & $\,\,$ 1/2$\,\,$ & $\,\,$ 1 $\,\,$ & $\,\,$3 $\,\,$ \\
$\,\,$ 1/9$\,\,$ & $\,\,$ 1/9$\,\,$ & $\,\,$ 1/3$\,\,$ & $\,\,$ 1  $\,\,$ \\
\end{pmatrix},
\qquad
\lambda_{\max} =
4.1031,
\qquad
CR = 0.0389
\end{equation*}

\begin{equation*}
\mathbf{w}^{EM} =
\begin{pmatrix}
0.555775\\
0.270190\\
\color{red} 0.130448\color{black} \\
0.043587
\end{pmatrix}\end{equation*}
\begin{equation*}
\left[ \frac{{w}^{EM}_i}{{w}^{EM}_j} \right] =
\begin{pmatrix}
$\,\,$ 1 $\,\,$ & $\,\,$2.0570$\,\,$ & $\,\,$\color{red} 4.2605\color{black} $\,\,$ & $\,\,$12.7509$\,\,$ \\
$\,\,$0.4862$\,\,$ & $\,\,$ 1 $\,\,$ & $\,\,$\color{red} 2.0712\color{black} $\,\,$ & $\,\,$6.1989  $\,\,$ \\
$\,\,$\color{red} 0.2347\color{black} $\,\,$ & $\,\,$\color{red} 0.4828\color{black} $\,\,$ & $\,\,$ 1 $\,\,$ & $\,\,$\color{red} 2.9928\color{black}  $\,\,$ \\
$\,\,$0.0784$\,\,$ & $\,\,$0.1613$\,\,$ & $\,\,$\color{red} 0.3341\color{black} $\,\,$ & $\,\,$ 1  $\,\,$ \\
\end{pmatrix},
\end{equation*}

\begin{equation*}
\mathbf{w}^{\prime} =
\begin{pmatrix}
0.555601\\
0.270106\\
0.130720\\
0.043573
\end{pmatrix} =
0.999687\cdot
\begin{pmatrix}
0.555775\\
0.270190\\
\color{gr} 0.130761\color{black} \\
0.043587
\end{pmatrix},
\end{equation*}
\begin{equation*}
\left[ \frac{{w}^{\prime}_i}{{w}^{\prime}_j} \right] =
\begin{pmatrix}
$\,\,$ 1 $\,\,$ & $\,\,$2.0570$\,\,$ & $\,\,$\color{gr} 4.2503\color{black} $\,\,$ & $\,\,$12.7509$\,\,$ \\
$\,\,$0.4862$\,\,$ & $\,\,$ 1 $\,\,$ & $\,\,$\color{gr} 2.0663\color{black} $\,\,$ & $\,\,$6.1989  $\,\,$ \\
$\,\,$\color{gr} 0.2353\color{black} $\,\,$ & $\,\,$\color{gr} 0.4840\color{black} $\,\,$ & $\,\,$ 1 $\,\,$ & $\,\,$\color{gr} \color{blue} 3\color{black}  $\,\,$ \\
$\,\,$0.0784$\,\,$ & $\,\,$0.1613$\,\,$ & $\,\,$\color{gr} \color{blue}  1/3\color{black} $\,\,$ & $\,\,$ 1  $\,\,$ \\
\end{pmatrix},
\end{equation*}
\end{example}
\newpage
\begin{example}
\begin{equation*}
\mathbf{A} =
\begin{pmatrix}
$\,\,$ 1 $\,\,$ & $\,\,$3$\,\,$ & $\,\,$5$\,\,$ & $\,\,$3 $\,\,$ \\
$\,\,$ 1/3$\,\,$ & $\,\,$ 1 $\,\,$ & $\,\,$3$\,\,$ & $\,\,$5 $\,\,$ \\
$\,\,$ 1/5$\,\,$ & $\,\,$ 1/3$\,\,$ & $\,\,$ 1 $\,\,$ & $\,\,$1 $\,\,$ \\
$\,\,$ 1/3$\,\,$ & $\,\,$ 1/5$\,\,$ & $\,\,$ 1 $\,\,$ & $\,\,$ 1  $\,\,$ \\
\end{pmatrix},
\qquad
\lambda_{\max} =
4.2253,
\qquad
CR = 0.0849
\end{equation*}

\begin{equation*}
\mathbf{w}^{EM} =
\begin{pmatrix}
0.511864\\
0.294801\\
\color{red} 0.093243\color{black} \\
0.100092
\end{pmatrix}\end{equation*}
\begin{equation*}
\left[ \frac{{w}^{EM}_i}{{w}^{EM}_j} \right] =
\begin{pmatrix}
$\,\,$ 1 $\,\,$ & $\,\,$1.7363$\,\,$ & $\,\,$\color{red} 5.4896\color{black} $\,\,$ & $\,\,$5.1139$\,\,$ \\
$\,\,$0.5759$\,\,$ & $\,\,$ 1 $\,\,$ & $\,\,$\color{red} 3.1617\color{black} $\,\,$ & $\,\,$2.9453  $\,\,$ \\
$\,\,$\color{red} 0.1822\color{black} $\,\,$ & $\,\,$\color{red} 0.3163\color{black} $\,\,$ & $\,\,$ 1 $\,\,$ & $\,\,$\color{red} 0.9316\color{black}  $\,\,$ \\
$\,\,$0.1955$\,\,$ & $\,\,$0.3395$\,\,$ & $\,\,$\color{red} 1.0735\color{black} $\,\,$ & $\,\,$ 1  $\,\,$ \\
\end{pmatrix},
\end{equation*}

\begin{equation*}
\mathbf{w}^{\prime} =
\begin{pmatrix}
0.509305\\
0.293327\\
0.097776\\
0.099592
\end{pmatrix} =
0.995001\cdot
\begin{pmatrix}
0.511864\\
0.294801\\
\color{gr} 0.098267\color{black} \\
0.100092
\end{pmatrix},
\end{equation*}
\begin{equation*}
\left[ \frac{{w}^{\prime}_i}{{w}^{\prime}_j} \right] =
\begin{pmatrix}
$\,\,$ 1 $\,\,$ & $\,\,$1.7363$\,\,$ & $\,\,$\color{gr} 5.2089\color{black} $\,\,$ & $\,\,$5.1139$\,\,$ \\
$\,\,$0.5759$\,\,$ & $\,\,$ 1 $\,\,$ & $\,\,$\color{gr} \color{blue} 3\color{black} $\,\,$ & $\,\,$2.9453  $\,\,$ \\
$\,\,$\color{gr} 0.1920\color{black} $\,\,$ & $\,\,$\color{gr} \color{blue}  1/3\color{black} $\,\,$ & $\,\,$ 1 $\,\,$ & $\,\,$\color{gr} 0.9818\color{black}  $\,\,$ \\
$\,\,$0.1955$\,\,$ & $\,\,$0.3395$\,\,$ & $\,\,$\color{gr} 1.0186\color{black} $\,\,$ & $\,\,$ 1  $\,\,$ \\
\end{pmatrix},
\end{equation*}
\end{example}
\newpage
\begin{example}
\begin{equation*}
\mathbf{A} =
\begin{pmatrix}
$\,\,$ 1 $\,\,$ & $\,\,$3$\,\,$ & $\,\,$5$\,\,$ & $\,\,$7 $\,\,$ \\
$\,\,$ 1/3$\,\,$ & $\,\,$ 1 $\,\,$ & $\,\,$6$\,\,$ & $\,\,$4 $\,\,$ \\
$\,\,$ 1/5$\,\,$ & $\,\,$ 1/6$\,\,$ & $\,\,$ 1 $\,\,$ & $\,\,$1 $\,\,$ \\
$\,\,$ 1/7$\,\,$ & $\,\,$ 1/4$\,\,$ & $\,\,$ 1 $\,\,$ & $\,\,$ 1  $\,\,$ \\
\end{pmatrix},
\qquad
\lambda_{\max} =
4.1417,
\qquad
CR = 0.0534
\end{equation*}

\begin{equation*}
\mathbf{w}^{EM} =
\begin{pmatrix}
0.559827\\
0.293462\\
0.074265\\
\color{red} 0.072446\color{black}
\end{pmatrix}\end{equation*}
\begin{equation*}
\left[ \frac{{w}^{EM}_i}{{w}^{EM}_j} \right] =
\begin{pmatrix}
$\,\,$ 1 $\,\,$ & $\,\,$1.9077$\,\,$ & $\,\,$7.5382$\,\,$ & $\,\,$\color{red} 7.7275\color{black} $\,\,$ \\
$\,\,$0.5242$\,\,$ & $\,\,$ 1 $\,\,$ & $\,\,$3.9516$\,\,$ & $\,\,$\color{red} 4.0508\color{black}   $\,\,$ \\
$\,\,$0.1327$\,\,$ & $\,\,$0.2531$\,\,$ & $\,\,$ 1 $\,\,$ & $\,\,$\color{red} 1.0251\color{black}  $\,\,$ \\
$\,\,$\color{red} 0.1294\color{black} $\,\,$ & $\,\,$\color{red} 0.2469\color{black} $\,\,$ & $\,\,$\color{red} 0.9755\color{black} $\,\,$ & $\,\,$ 1  $\,\,$ \\
\end{pmatrix},
\end{equation*}

\begin{equation*}
\mathbf{w}^{\prime} =
\begin{pmatrix}
0.559313\\
0.293192\\
0.074197\\
0.073298
\end{pmatrix} =
0.999081\cdot
\begin{pmatrix}
0.559827\\
0.293462\\
0.074265\\
\color{gr} 0.073366\color{black}
\end{pmatrix},
\end{equation*}
\begin{equation*}
\left[ \frac{{w}^{\prime}_i}{{w}^{\prime}_j} \right] =
\begin{pmatrix}
$\,\,$ 1 $\,\,$ & $\,\,$1.9077$\,\,$ & $\,\,$7.5382$\,\,$ & $\,\,$\color{gr} 7.6307\color{black} $\,\,$ \\
$\,\,$0.5242$\,\,$ & $\,\,$ 1 $\,\,$ & $\,\,$3.9516$\,\,$ & $\,\,$\color{gr} \color{blue} 4\color{black}   $\,\,$ \\
$\,\,$0.1327$\,\,$ & $\,\,$0.2531$\,\,$ & $\,\,$ 1 $\,\,$ & $\,\,$\color{gr} 1.0123\color{black}  $\,\,$ \\
$\,\,$\color{gr} 0.1311\color{black} $\,\,$ & $\,\,$\color{gr} \color{blue}  1/4\color{black} $\,\,$ & $\,\,$\color{gr} 0.9879\color{black} $\,\,$ & $\,\,$ 1  $\,\,$ \\
\end{pmatrix},
\end{equation*}
\end{example}
\newpage
\begin{example}
\begin{equation*}
\mathbf{A} =
\begin{pmatrix}
$\,\,$ 1 $\,\,$ & $\,\,$3$\,\,$ & $\,\,$5$\,\,$ & $\,\,$7 $\,\,$ \\
$\,\,$ 1/3$\,\,$ & $\,\,$ 1 $\,\,$ & $\,\,$9$\,\,$ & $\,\,$5 $\,\,$ \\
$\,\,$ 1/5$\,\,$ & $\,\,$ 1/9$\,\,$ & $\,\,$ 1 $\,\,$ & $\,\,$1 $\,\,$ \\
$\,\,$ 1/7$\,\,$ & $\,\,$ 1/5$\,\,$ & $\,\,$ 1 $\,\,$ & $\,\,$ 1  $\,\,$ \\
\end{pmatrix},
\qquad
\lambda_{\max} =
4.2539,
\qquad
CR = 0.0957
\end{equation*}

\begin{equation*}
\mathbf{w}^{EM} =
\begin{pmatrix}
0.541040\\
0.330974\\
0.064169\\
\color{red} 0.063817\color{black}
\end{pmatrix}\end{equation*}
\begin{equation*}
\left[ \frac{{w}^{EM}_i}{{w}^{EM}_j} \right] =
\begin{pmatrix}
$\,\,$ 1 $\,\,$ & $\,\,$1.6347$\,\,$ & $\,\,$8.4315$\,\,$ & $\,\,$\color{red} 8.4779\color{black} $\,\,$ \\
$\,\,$0.6117$\,\,$ & $\,\,$ 1 $\,\,$ & $\,\,$5.1578$\,\,$ & $\,\,$\color{red} 5.1863\color{black}   $\,\,$ \\
$\,\,$0.1186$\,\,$ & $\,\,$0.1939$\,\,$ & $\,\,$ 1 $\,\,$ & $\,\,$\color{red} 1.0055\color{black}  $\,\,$ \\
$\,\,$\color{red} 0.1180\color{black} $\,\,$ & $\,\,$\color{red} 0.1928\color{black} $\,\,$ & $\,\,$\color{red} 0.9945\color{black} $\,\,$ & $\,\,$ 1  $\,\,$ \\
\end{pmatrix},
\end{equation*}

\begin{equation*}
\mathbf{w}^{\prime} =
\begin{pmatrix}
0.540849\\
0.330857\\
0.064147\\
0.064147
\end{pmatrix} =
0.999648\cdot
\begin{pmatrix}
0.541040\\
0.330974\\
0.064169\\
\color{gr} 0.064169\color{black}
\end{pmatrix},
\end{equation*}
\begin{equation*}
\left[ \frac{{w}^{\prime}_i}{{w}^{\prime}_j} \right] =
\begin{pmatrix}
$\,\,$ 1 $\,\,$ & $\,\,$1.6347$\,\,$ & $\,\,$8.4315$\,\,$ & $\,\,$\color{gr} 8.4315\color{black} $\,\,$ \\
$\,\,$0.6117$\,\,$ & $\,\,$ 1 $\,\,$ & $\,\,$5.1578$\,\,$ & $\,\,$\color{gr} 5.1578\color{black}   $\,\,$ \\
$\,\,$0.1186$\,\,$ & $\,\,$0.1939$\,\,$ & $\,\,$ 1 $\,\,$ & $\,\,$\color{gr} \color{blue} 1\color{black}  $\,\,$ \\
$\,\,$\color{gr} 0.1186\color{black} $\,\,$ & $\,\,$\color{gr} 0.1939\color{black} $\,\,$ & $\,\,$\color{gr} \color{blue} 1\color{black} $\,\,$ & $\,\,$ 1  $\,\,$ \\
\end{pmatrix},
\end{equation*}
\end{example}
\newpage
\begin{example}
\begin{equation*}
\mathbf{A} =
\begin{pmatrix}
$\,\,$ 1 $\,\,$ & $\,\,$3$\,\,$ & $\,\,$5$\,\,$ & $\,\,$8 $\,\,$ \\
$\,\,$ 1/3$\,\,$ & $\,\,$ 1 $\,\,$ & $\,\,$2$\,\,$ & $\,\,$5 $\,\,$ \\
$\,\,$ 1/5$\,\,$ & $\,\,$ 1/2$\,\,$ & $\,\,$ 1 $\,\,$ & $\,\,$2 $\,\,$ \\
$\,\,$ 1/8$\,\,$ & $\,\,$ 1/5$\,\,$ & $\,\,$ 1/2$\,\,$ & $\,\,$ 1  $\,\,$ \\
\end{pmatrix},
\qquad
\lambda_{\max} =
4.0332,
\qquad
CR = 0.0125
\end{equation*}

\begin{equation*}
\mathbf{w}^{EM} =
\begin{pmatrix}
0.585026\\
0.238888\\
\color{red} 0.116947\color{black} \\
0.059139
\end{pmatrix}\end{equation*}
\begin{equation*}
\left[ \frac{{w}^{EM}_i}{{w}^{EM}_j} \right] =
\begin{pmatrix}
$\,\,$ 1 $\,\,$ & $\,\,$2.4490$\,\,$ & $\,\,$\color{red} 5.0025\color{black} $\,\,$ & $\,\,$9.8925$\,\,$ \\
$\,\,$0.4083$\,\,$ & $\,\,$ 1 $\,\,$ & $\,\,$\color{red} 2.0427\color{black} $\,\,$ & $\,\,$4.0395  $\,\,$ \\
$\,\,$\color{red} 0.1999\color{black} $\,\,$ & $\,\,$\color{red} 0.4896\color{black} $\,\,$ & $\,\,$ 1 $\,\,$ & $\,\,$\color{red} 1.9775\color{black}  $\,\,$ \\
$\,\,$0.1011$\,\,$ & $\,\,$0.2476$\,\,$ & $\,\,$\color{red} 0.5057\color{black} $\,\,$ & $\,\,$ 1  $\,\,$ \\
\end{pmatrix},
\end{equation*}

\begin{equation*}
\mathbf{w}^{\prime} =
\begin{pmatrix}
0.584993\\
0.238874\\
0.116999\\
0.059135
\end{pmatrix} =
0.999942\cdot
\begin{pmatrix}
0.585026\\
0.238888\\
\color{gr} 0.117005\color{black} \\
0.059139
\end{pmatrix},
\end{equation*}
\begin{equation*}
\left[ \frac{{w}^{\prime}_i}{{w}^{\prime}_j} \right] =
\begin{pmatrix}
$\,\,$ 1 $\,\,$ & $\,\,$2.4490$\,\,$ & $\,\,$\color{gr} \color{blue} 5\color{black} $\,\,$ & $\,\,$9.8925$\,\,$ \\
$\,\,$0.4083$\,\,$ & $\,\,$ 1 $\,\,$ & $\,\,$\color{gr} 2.0417\color{black} $\,\,$ & $\,\,$4.0395  $\,\,$ \\
$\,\,$\color{gr} \color{blue}  1/5\color{black} $\,\,$ & $\,\,$\color{gr} 0.4898\color{black} $\,\,$ & $\,\,$ 1 $\,\,$ & $\,\,$\color{gr} 1.9785\color{black}  $\,\,$ \\
$\,\,$0.1011$\,\,$ & $\,\,$0.2476$\,\,$ & $\,\,$\color{gr} 0.5054\color{black} $\,\,$ & $\,\,$ 1  $\,\,$ \\
\end{pmatrix},
\end{equation*}
\end{example}
\newpage
\begin{example}
\begin{equation*}
\mathbf{A} =
\begin{pmatrix}
$\,\,$ 1 $\,\,$ & $\,\,$3$\,\,$ & $\,\,$5$\,\,$ & $\,\,$8 $\,\,$ \\
$\,\,$ 1/3$\,\,$ & $\,\,$ 1 $\,\,$ & $\,\,$6$\,\,$ & $\,\,$4 $\,\,$ \\
$\,\,$ 1/5$\,\,$ & $\,\,$ 1/6$\,\,$ & $\,\,$ 1 $\,\,$ & $\,\,$1 $\,\,$ \\
$\,\,$ 1/8$\,\,$ & $\,\,$ 1/4$\,\,$ & $\,\,$ 1 $\,\,$ & $\,\,$ 1  $\,\,$ \\
\end{pmatrix},
\qquad
\lambda_{\max} =
4.1406,
\qquad
CR = 0.0530
\end{equation*}

\begin{equation*}
\mathbf{w}^{EM} =
\begin{pmatrix}
0.568708\\
0.288741\\
0.073520\\
\color{red} 0.069030\color{black}
\end{pmatrix}\end{equation*}
\begin{equation*}
\left[ \frac{{w}^{EM}_i}{{w}^{EM}_j} \right] =
\begin{pmatrix}
$\,\,$ 1 $\,\,$ & $\,\,$1.9696$\,\,$ & $\,\,$7.7354$\,\,$ & $\,\,$\color{red} 8.2385\color{black} $\,\,$ \\
$\,\,$0.5077$\,\,$ & $\,\,$ 1 $\,\,$ & $\,\,$3.9274$\,\,$ & $\,\,$\color{red} 4.1828\color{black}   $\,\,$ \\
$\,\,$0.1293$\,\,$ & $\,\,$0.2546$\,\,$ & $\,\,$ 1 $\,\,$ & $\,\,$\color{red} 1.0650\color{black}  $\,\,$ \\
$\,\,$\color{red} 0.1214\color{black} $\,\,$ & $\,\,$\color{red} 0.2391\color{black} $\,\,$ & $\,\,$\color{red} 0.9389\color{black} $\,\,$ & $\,\,$ 1  $\,\,$ \\
\end{pmatrix},
\end{equation*}

\begin{equation*}
\mathbf{w}^{\prime} =
\begin{pmatrix}
0.567540\\
0.288148\\
0.073369\\
0.070943
\end{pmatrix} =
0.997946\cdot
\begin{pmatrix}
0.568708\\
0.288741\\
0.073520\\
\color{gr} 0.071089\color{black}
\end{pmatrix},
\end{equation*}
\begin{equation*}
\left[ \frac{{w}^{\prime}_i}{{w}^{\prime}_j} \right] =
\begin{pmatrix}
$\,\,$ 1 $\,\,$ & $\,\,$1.9696$\,\,$ & $\,\,$7.7354$\,\,$ & $\,\,$\color{gr} \color{blue} 8\color{black} $\,\,$ \\
$\,\,$0.5077$\,\,$ & $\,\,$ 1 $\,\,$ & $\,\,$3.9274$\,\,$ & $\,\,$\color{gr} 4.0617\color{black}   $\,\,$ \\
$\,\,$0.1293$\,\,$ & $\,\,$0.2546$\,\,$ & $\,\,$ 1 $\,\,$ & $\,\,$\color{gr} 1.0342\color{black}  $\,\,$ \\
$\,\,$\color{gr} \color{blue}  1/8\color{black} $\,\,$ & $\,\,$\color{gr} 0.2462\color{black} $\,\,$ & $\,\,$\color{gr} 0.9669\color{black} $\,\,$ & $\,\,$ 1  $\,\,$ \\
\end{pmatrix},
\end{equation*}
\end{example}
\newpage
\begin{example}
\begin{equation*}
\mathbf{A} =
\begin{pmatrix}
$\,\,$ 1 $\,\,$ & $\,\,$3$\,\,$ & $\,\,$5$\,\,$ & $\,\,$8 $\,\,$ \\
$\,\,$ 1/3$\,\,$ & $\,\,$ 1 $\,\,$ & $\,\,$7$\,\,$ & $\,\,$4 $\,\,$ \\
$\,\,$ 1/5$\,\,$ & $\,\,$ 1/7$\,\,$ & $\,\,$ 1 $\,\,$ & $\,\,$1 $\,\,$ \\
$\,\,$ 1/8$\,\,$ & $\,\,$ 1/4$\,\,$ & $\,\,$ 1 $\,\,$ & $\,\,$ 1  $\,\,$ \\
\end{pmatrix},
\qquad
\lambda_{\max} =
4.1782,
\qquad
CR = 0.0672
\end{equation*}

\begin{equation*}
\mathbf{w}^{EM} =
\begin{pmatrix}
0.563133\\
0.298928\\
0.070191\\
\color{red} 0.067748\color{black}
\end{pmatrix}\end{equation*}
\begin{equation*}
\left[ \frac{{w}^{EM}_i}{{w}^{EM}_j} \right] =
\begin{pmatrix}
$\,\,$ 1 $\,\,$ & $\,\,$1.8838$\,\,$ & $\,\,$8.0228$\,\,$ & $\,\,$\color{red} 8.3121\color{black} $\,\,$ \\
$\,\,$0.5308$\,\,$ & $\,\,$ 1 $\,\,$ & $\,\,$4.2588$\,\,$ & $\,\,$\color{red} 4.4123\color{black}   $\,\,$ \\
$\,\,$0.1246$\,\,$ & $\,\,$0.2348$\,\,$ & $\,\,$ 1 $\,\,$ & $\,\,$\color{red} 1.0361\color{black}  $\,\,$ \\
$\,\,$\color{red} 0.1203\color{black} $\,\,$ & $\,\,$\color{red} 0.2266\color{black} $\,\,$ & $\,\,$\color{red} 0.9652\color{black} $\,\,$ & $\,\,$ 1  $\,\,$ \\
\end{pmatrix},
\end{equation*}

\begin{equation*}
\mathbf{w}^{\prime} =
\begin{pmatrix}
0.561760\\
0.298200\\
0.070020\\
0.070020
\end{pmatrix} =
0.997563\cdot
\begin{pmatrix}
0.563133\\
0.298928\\
0.070191\\
\color{gr} 0.070191\color{black}
\end{pmatrix},
\end{equation*}
\begin{equation*}
\left[ \frac{{w}^{\prime}_i}{{w}^{\prime}_j} \right] =
\begin{pmatrix}
$\,\,$ 1 $\,\,$ & $\,\,$1.8838$\,\,$ & $\,\,$8.0228$\,\,$ & $\,\,$\color{gr} 8.0228\color{black} $\,\,$ \\
$\,\,$0.5308$\,\,$ & $\,\,$ 1 $\,\,$ & $\,\,$4.2588$\,\,$ & $\,\,$\color{gr} 4.2588\color{black}   $\,\,$ \\
$\,\,$0.1246$\,\,$ & $\,\,$0.2348$\,\,$ & $\,\,$ 1 $\,\,$ & $\,\,$\color{gr} \color{blue} 1\color{black}  $\,\,$ \\
$\,\,$\color{gr} 0.1246\color{black} $\,\,$ & $\,\,$\color{gr} 0.2348\color{black} $\,\,$ & $\,\,$\color{gr} \color{blue} 1\color{black} $\,\,$ & $\,\,$ 1  $\,\,$ \\
\end{pmatrix},
\end{equation*}
\end{example}
\newpage
\begin{example}
\begin{equation*}
\mathbf{A} =
\begin{pmatrix}
$\,\,$ 1 $\,\,$ & $\,\,$3$\,\,$ & $\,\,$5$\,\,$ & $\,\,$8 $\,\,$ \\
$\,\,$ 1/3$\,\,$ & $\,\,$ 1 $\,\,$ & $\,\,$8$\,\,$ & $\,\,$4 $\,\,$ \\
$\,\,$ 1/5$\,\,$ & $\,\,$ 1/8$\,\,$ & $\,\,$ 1 $\,\,$ & $\,\,$1 $\,\,$ \\
$\,\,$ 1/8$\,\,$ & $\,\,$ 1/4$\,\,$ & $\,\,$ 1 $\,\,$ & $\,\,$ 1  $\,\,$ \\
\end{pmatrix},
\qquad
\lambda_{\max} =
4.2162,
\qquad
CR = 0.0815
\end{equation*}

\begin{equation*}
\mathbf{w}^{EM} =
\begin{pmatrix}
0.557841\\
0.308203\\
0.067371\\
\color{red} 0.066585\color{black}
\end{pmatrix}\end{equation*}
\begin{equation*}
\left[ \frac{{w}^{EM}_i}{{w}^{EM}_j} \right] =
\begin{pmatrix}
$\,\,$ 1 $\,\,$ & $\,\,$1.8100$\,\,$ & $\,\,$8.2802$\,\,$ & $\,\,$\color{red} 8.3779\color{black} $\,\,$ \\
$\,\,$0.5525$\,\,$ & $\,\,$ 1 $\,\,$ & $\,\,$4.5748$\,\,$ & $\,\,$\color{red} 4.6287\color{black}   $\,\,$ \\
$\,\,$0.1208$\,\,$ & $\,\,$0.2186$\,\,$ & $\,\,$ 1 $\,\,$ & $\,\,$\color{red} 1.0118\color{black}  $\,\,$ \\
$\,\,$\color{red} 0.1194\color{black} $\,\,$ & $\,\,$\color{red} 0.2160\color{black} $\,\,$ & $\,\,$\color{red} 0.9883\color{black} $\,\,$ & $\,\,$ 1  $\,\,$ \\
\end{pmatrix},
\end{equation*}

\begin{equation*}
\mathbf{w}^{\prime} =
\begin{pmatrix}
0.557403\\
0.307961\\
0.067318\\
0.067318
\end{pmatrix} =
0.999215\cdot
\begin{pmatrix}
0.557841\\
0.308203\\
0.067371\\
\color{gr} 0.067371\color{black}
\end{pmatrix},
\end{equation*}
\begin{equation*}
\left[ \frac{{w}^{\prime}_i}{{w}^{\prime}_j} \right] =
\begin{pmatrix}
$\,\,$ 1 $\,\,$ & $\,\,$1.8100$\,\,$ & $\,\,$8.2802$\,\,$ & $\,\,$\color{gr} 8.2802\color{black} $\,\,$ \\
$\,\,$0.5525$\,\,$ & $\,\,$ 1 $\,\,$ & $\,\,$4.5748$\,\,$ & $\,\,$\color{gr} 4.5748\color{black}   $\,\,$ \\
$\,\,$0.1208$\,\,$ & $\,\,$0.2186$\,\,$ & $\,\,$ 1 $\,\,$ & $\,\,$\color{gr} \color{blue} 1\color{black}  $\,\,$ \\
$\,\,$\color{gr} 0.1208\color{black} $\,\,$ & $\,\,$\color{gr} 0.2186\color{black} $\,\,$ & $\,\,$\color{gr} \color{blue} 1\color{black} $\,\,$ & $\,\,$ 1  $\,\,$ \\
\end{pmatrix},
\end{equation*}
\end{example}
\newpage
\begin{example}
\begin{equation*}
\mathbf{A} =
\begin{pmatrix}
$\,\,$ 1 $\,\,$ & $\,\,$3$\,\,$ & $\,\,$5$\,\,$ & $\,\,$8 $\,\,$ \\
$\,\,$ 1/3$\,\,$ & $\,\,$ 1 $\,\,$ & $\,\,$9$\,\,$ & $\,\,$5 $\,\,$ \\
$\,\,$ 1/5$\,\,$ & $\,\,$ 1/9$\,\,$ & $\,\,$ 1 $\,\,$ & $\,\,$1 $\,\,$ \\
$\,\,$ 1/8$\,\,$ & $\,\,$ 1/5$\,\,$ & $\,\,$ 1 $\,\,$ & $\,\,$ 1  $\,\,$ \\
\end{pmatrix},
\qquad
\lambda_{\max} =
4.2489,
\qquad
CR = 0.0939
\end{equation*}

\begin{equation*}
\mathbf{w}^{EM} =
\begin{pmatrix}
0.549132\\
0.326362\\
0.063686\\
\color{red} 0.060820\color{black}
\end{pmatrix}\end{equation*}
\begin{equation*}
\left[ \frac{{w}^{EM}_i}{{w}^{EM}_j} \right] =
\begin{pmatrix}
$\,\,$ 1 $\,\,$ & $\,\,$1.6826$\,\,$ & $\,\,$8.6225$\,\,$ & $\,\,$\color{red} 9.0287\color{black} $\,\,$ \\
$\,\,$0.5943$\,\,$ & $\,\,$ 1 $\,\,$ & $\,\,$5.1246$\,\,$ & $\,\,$\color{red} 5.3660\color{black}   $\,\,$ \\
$\,\,$0.1160$\,\,$ & $\,\,$0.1951$\,\,$ & $\,\,$ 1 $\,\,$ & $\,\,$\color{red} 1.0471\color{black}  $\,\,$ \\
$\,\,$\color{red} 0.1108\color{black} $\,\,$ & $\,\,$\color{red} 0.1864\color{black} $\,\,$ & $\,\,$\color{red} 0.9550\color{black} $\,\,$ & $\,\,$ 1  $\,\,$ \\
\end{pmatrix},
\end{equation*}

\begin{equation*}
\mathbf{w}^{\prime} =
\begin{pmatrix}
0.547563\\
0.325429\\
0.063504\\
0.063504
\end{pmatrix} =
0.997143\cdot
\begin{pmatrix}
0.549132\\
0.326362\\
0.063686\\
\color{gr} 0.063686\color{black}
\end{pmatrix},
\end{equation*}
\begin{equation*}
\left[ \frac{{w}^{\prime}_i}{{w}^{\prime}_j} \right] =
\begin{pmatrix}
$\,\,$ 1 $\,\,$ & $\,\,$1.6826$\,\,$ & $\,\,$8.6225$\,\,$ & $\,\,$\color{gr} 8.6225\color{black} $\,\,$ \\
$\,\,$0.5943$\,\,$ & $\,\,$ 1 $\,\,$ & $\,\,$5.1246$\,\,$ & $\,\,$\color{gr} 5.1246\color{black}   $\,\,$ \\
$\,\,$0.1160$\,\,$ & $\,\,$0.1951$\,\,$ & $\,\,$ 1 $\,\,$ & $\,\,$\color{gr} \color{blue} 1\color{black}  $\,\,$ \\
$\,\,$\color{gr} 0.1160\color{black} $\,\,$ & $\,\,$\color{gr} 0.1951\color{black} $\,\,$ & $\,\,$\color{gr} \color{blue} 1\color{black} $\,\,$ & $\,\,$ 1  $\,\,$ \\
\end{pmatrix},
\end{equation*}
\end{example}
\newpage
\begin{example}
\begin{equation*}
\mathbf{A} =
\begin{pmatrix}
$\,\,$ 1 $\,\,$ & $\,\,$3$\,\,$ & $\,\,$5$\,\,$ & $\,\,$9 $\,\,$ \\
$\,\,$ 1/3$\,\,$ & $\,\,$ 1 $\,\,$ & $\,\,$3$\,\,$ & $\,\,$2 $\,\,$ \\
$\,\,$ 1/5$\,\,$ & $\,\,$ 1/3$\,\,$ & $\,\,$ 1 $\,\,$ & $\,\,$3 $\,\,$ \\
$\,\,$ 1/9$\,\,$ & $\,\,$ 1/2$\,\,$ & $\,\,$ 1/3$\,\,$ & $\,\,$ 1  $\,\,$ \\
\end{pmatrix},
\qquad
\lambda_{\max} =
4.1966,
\qquad
CR = 0.0741
\end{equation*}

\begin{equation*}
\mathbf{w}^{EM} =
\begin{pmatrix}
\color{red} 0.589476\color{black} \\
0.219538\\
0.123300\\
0.067687
\end{pmatrix}\end{equation*}
\begin{equation*}
\left[ \frac{{w}^{EM}_i}{{w}^{EM}_j} \right] =
\begin{pmatrix}
$\,\,$ 1 $\,\,$ & $\,\,$\color{red} 2.6851\color{black} $\,\,$ & $\,\,$\color{red} 4.7808\color{black} $\,\,$ & $\,\,$\color{red} 8.7088\color{black} $\,\,$ \\
$\,\,$\color{red} 0.3724\color{black} $\,\,$ & $\,\,$ 1 $\,\,$ & $\,\,$1.7805$\,\,$ & $\,\,$3.2434  $\,\,$ \\
$\,\,$\color{red} 0.2092\color{black} $\,\,$ & $\,\,$0.5616$\,\,$ & $\,\,$ 1 $\,\,$ & $\,\,$1.8216 $\,\,$ \\
$\,\,$\color{red} 0.1148\color{black} $\,\,$ & $\,\,$0.3083$\,\,$ & $\,\,$0.5490$\,\,$ & $\,\,$ 1  $\,\,$ \\
\end{pmatrix},
\end{equation*}

\begin{equation*}
\mathbf{w}^{\prime} =
\begin{pmatrix}
0.597410\\
0.215294\\
0.120917\\
0.066379
\end{pmatrix} =
0.980673\cdot
\begin{pmatrix}
\color{gr} 0.609184\color{black} \\
0.219538\\
0.123300\\
0.067687
\end{pmatrix},
\end{equation*}
\begin{equation*}
\left[ \frac{{w}^{\prime}_i}{{w}^{\prime}_j} \right] =
\begin{pmatrix}
$\,\,$ 1 $\,\,$ & $\,\,$\color{gr} 2.7749\color{black} $\,\,$ & $\,\,$\color{gr} 4.9407\color{black} $\,\,$ & $\,\,$\color{gr} \color{blue} 9\color{black} $\,\,$ \\
$\,\,$\color{gr} 0.3604\color{black} $\,\,$ & $\,\,$ 1 $\,\,$ & $\,\,$1.7805$\,\,$ & $\,\,$3.2434  $\,\,$ \\
$\,\,$\color{gr} 0.2024\color{black} $\,\,$ & $\,\,$0.5616$\,\,$ & $\,\,$ 1 $\,\,$ & $\,\,$1.8216 $\,\,$ \\
$\,\,$\color{gr} \color{blue}  1/9\color{black} $\,\,$ & $\,\,$0.3083$\,\,$ & $\,\,$0.5490$\,\,$ & $\,\,$ 1  $\,\,$ \\
\end{pmatrix},
\end{equation*}
\end{example}
\newpage
\begin{example}
\begin{equation*}
\mathbf{A} =
\begin{pmatrix}
$\,\,$ 1 $\,\,$ & $\,\,$3$\,\,$ & $\,\,$6$\,\,$ & $\,\,$8 $\,\,$ \\
$\,\,$ 1/3$\,\,$ & $\,\,$ 1 $\,\,$ & $\,\,$3$\,\,$ & $\,\,$2 $\,\,$ \\
$\,\,$ 1/6$\,\,$ & $\,\,$ 1/3$\,\,$ & $\,\,$ 1 $\,\,$ & $\,\,$2 $\,\,$ \\
$\,\,$ 1/8$\,\,$ & $\,\,$ 1/2$\,\,$ & $\,\,$ 1/2$\,\,$ & $\,\,$ 1  $\,\,$ \\
\end{pmatrix},
\qquad
\lambda_{\max} =
4.1031,
\qquad
CR = 0.0389
\end{equation*}

\begin{equation*}
\mathbf{w}^{EM} =
\begin{pmatrix}
\color{red} 0.604981\color{black} \\
0.214793\\
0.104422\\
0.075804
\end{pmatrix}\end{equation*}
\begin{equation*}
\left[ \frac{{w}^{EM}_i}{{w}^{EM}_j} \right] =
\begin{pmatrix}
$\,\,$ 1 $\,\,$ & $\,\,$\color{red} 2.8166\color{black} $\,\,$ & $\,\,$\color{red} 5.7936\color{black} $\,\,$ & $\,\,$\color{red} 7.9809\color{black} $\,\,$ \\
$\,\,$\color{red} 0.3550\color{black} $\,\,$ & $\,\,$ 1 $\,\,$ & $\,\,$2.0570$\,\,$ & $\,\,$2.8335  $\,\,$ \\
$\,\,$\color{red} 0.1726\color{black} $\,\,$ & $\,\,$0.4862$\,\,$ & $\,\,$ 1 $\,\,$ & $\,\,$1.3775 $\,\,$ \\
$\,\,$\color{red} 0.1253\color{black} $\,\,$ & $\,\,$0.3529$\,\,$ & $\,\,$0.7259$\,\,$ & $\,\,$ 1  $\,\,$ \\
\end{pmatrix},
\end{equation*}

\begin{equation*}
\mathbf{w}^{\prime} =
\begin{pmatrix}
0.605553\\
0.214482\\
0.104271\\
0.075694
\end{pmatrix} =
0.998552\cdot
\begin{pmatrix}
\color{gr} 0.606431\color{black} \\
0.214793\\
0.104422\\
0.075804
\end{pmatrix},
\end{equation*}
\begin{equation*}
\left[ \frac{{w}^{\prime}_i}{{w}^{\prime}_j} \right] =
\begin{pmatrix}
$\,\,$ 1 $\,\,$ & $\,\,$\color{gr} 2.8233\color{black} $\,\,$ & $\,\,$\color{gr} 5.8075\color{black} $\,\,$ & $\,\,$\color{gr} \color{blue} 8\color{black} $\,\,$ \\
$\,\,$\color{gr} 0.3542\color{black} $\,\,$ & $\,\,$ 1 $\,\,$ & $\,\,$2.0570$\,\,$ & $\,\,$2.8335  $\,\,$ \\
$\,\,$\color{gr} 0.1722\color{black} $\,\,$ & $\,\,$0.4862$\,\,$ & $\,\,$ 1 $\,\,$ & $\,\,$1.3775 $\,\,$ \\
$\,\,$\color{gr} \color{blue}  1/8\color{black} $\,\,$ & $\,\,$0.3529$\,\,$ & $\,\,$0.7259$\,\,$ & $\,\,$ 1  $\,\,$ \\
\end{pmatrix},
\end{equation*}
\end{example}
\newpage
\begin{example}
\begin{equation*}
\mathbf{A} =
\begin{pmatrix}
$\,\,$ 1 $\,\,$ & $\,\,$3$\,\,$ & $\,\,$6$\,\,$ & $\,\,$8 $\,\,$ \\
$\,\,$ 1/3$\,\,$ & $\,\,$ 1 $\,\,$ & $\,\,$6$\,\,$ & $\,\,$4 $\,\,$ \\
$\,\,$ 1/6$\,\,$ & $\,\,$ 1/6$\,\,$ & $\,\,$ 1 $\,\,$ & $\,\,$1 $\,\,$ \\
$\,\,$ 1/8$\,\,$ & $\,\,$ 1/4$\,\,$ & $\,\,$ 1 $\,\,$ & $\,\,$ 1  $\,\,$ \\
\end{pmatrix},
\qquad
\lambda_{\max} =
4.1031,
\qquad
CR = 0.0389
\end{equation*}

\begin{equation*}
\mathbf{w}^{EM} =
\begin{pmatrix}
0.581008\\
0.282457\\
0.068349\\
\color{red} 0.068186\color{black}
\end{pmatrix}\end{equation*}
\begin{equation*}
\left[ \frac{{w}^{EM}_i}{{w}^{EM}_j} \right] =
\begin{pmatrix}
$\,\,$ 1 $\,\,$ & $\,\,$2.0570$\,\,$ & $\,\,$8.5006$\,\,$ & $\,\,$\color{red} 8.5210\color{black} $\,\,$ \\
$\,\,$0.4862$\,\,$ & $\,\,$ 1 $\,\,$ & $\,\,$4.1326$\,\,$ & $\,\,$\color{red} 4.1425\color{black}   $\,\,$ \\
$\,\,$0.1176$\,\,$ & $\,\,$0.2420$\,\,$ & $\,\,$ 1 $\,\,$ & $\,\,$\color{red} 1.0024\color{black}  $\,\,$ \\
$\,\,$\color{red} 0.1174\color{black} $\,\,$ & $\,\,$\color{red} 0.2414\color{black} $\,\,$ & $\,\,$\color{red} 0.9976\color{black} $\,\,$ & $\,\,$ 1  $\,\,$ \\
\end{pmatrix},
\end{equation*}

\begin{equation*}
\mathbf{w}^{\prime} =
\begin{pmatrix}
0.580913\\
0.282411\\
0.068338\\
0.068338
\end{pmatrix} =
0.999837\cdot
\begin{pmatrix}
0.581008\\
0.282457\\
0.068349\\
\color{gr} 0.068349\color{black}
\end{pmatrix},
\end{equation*}
\begin{equation*}
\left[ \frac{{w}^{\prime}_i}{{w}^{\prime}_j} \right] =
\begin{pmatrix}
$\,\,$ 1 $\,\,$ & $\,\,$2.0570$\,\,$ & $\,\,$8.5006$\,\,$ & $\,\,$\color{gr} 8.5006\color{black} $\,\,$ \\
$\,\,$0.4862$\,\,$ & $\,\,$ 1 $\,\,$ & $\,\,$4.1326$\,\,$ & $\,\,$\color{gr} 4.1326\color{black}   $\,\,$ \\
$\,\,$0.1176$\,\,$ & $\,\,$0.2420$\,\,$ & $\,\,$ 1 $\,\,$ & $\,\,$\color{gr} \color{blue} 1\color{black}  $\,\,$ \\
$\,\,$\color{gr} 0.1176\color{black} $\,\,$ & $\,\,$\color{gr} 0.2420\color{black} $\,\,$ & $\,\,$\color{gr} \color{blue} 1\color{black} $\,\,$ & $\,\,$ 1  $\,\,$ \\
\end{pmatrix},
\end{equation*}
\end{example}
\newpage
\begin{example}
\begin{equation*}
\mathbf{A} =
\begin{pmatrix}
$\,\,$ 1 $\,\,$ & $\,\,$3$\,\,$ & $\,\,$6$\,\,$ & $\,\,$8 $\,\,$ \\
$\,\,$ 1/3$\,\,$ & $\,\,$ 1 $\,\,$ & $\,\,$8$\,\,$ & $\,\,$5 $\,\,$ \\
$\,\,$ 1/6$\,\,$ & $\,\,$ 1/8$\,\,$ & $\,\,$ 1 $\,\,$ & $\,\,$1 $\,\,$ \\
$\,\,$ 1/8$\,\,$ & $\,\,$ 1/5$\,\,$ & $\,\,$ 1 $\,\,$ & $\,\,$ 1  $\,\,$ \\
\end{pmatrix},
\qquad
\lambda_{\max} =
4.1689,
\qquad
CR = 0.0637
\end{equation*}

\begin{equation*}
\mathbf{w}^{EM} =
\begin{pmatrix}
0.565899\\
0.311346\\
0.061405\\
\color{red} 0.061350\color{black}
\end{pmatrix}\end{equation*}
\begin{equation*}
\left[ \frac{{w}^{EM}_i}{{w}^{EM}_j} \right] =
\begin{pmatrix}
$\,\,$ 1 $\,\,$ & $\,\,$1.8176$\,\,$ & $\,\,$9.2159$\,\,$ & $\,\,$\color{red} 9.2241\color{black} $\,\,$ \\
$\,\,$0.5502$\,\,$ & $\,\,$ 1 $\,\,$ & $\,\,$5.0704$\,\,$ & $\,\,$\color{red} 5.0749\color{black}   $\,\,$ \\
$\,\,$0.1085$\,\,$ & $\,\,$0.1972$\,\,$ & $\,\,$ 1 $\,\,$ & $\,\,$\color{red} 1.0009\color{black}  $\,\,$ \\
$\,\,$\color{red} 0.1084\color{black} $\,\,$ & $\,\,$\color{red} 0.1970\color{black} $\,\,$ & $\,\,$\color{red} 0.9991\color{black} $\,\,$ & $\,\,$ 1  $\,\,$ \\
\end{pmatrix},
\end{equation*}

\begin{equation*}
\mathbf{w}^{\prime} =
\begin{pmatrix}
0.565868\\
0.311329\\
0.061401\\
0.061401
\end{pmatrix} =
0.999945\cdot
\begin{pmatrix}
0.565899\\
0.311346\\
0.061405\\
\color{gr} 0.061405\color{black}
\end{pmatrix},
\end{equation*}
\begin{equation*}
\left[ \frac{{w}^{\prime}_i}{{w}^{\prime}_j} \right] =
\begin{pmatrix}
$\,\,$ 1 $\,\,$ & $\,\,$1.8176$\,\,$ & $\,\,$9.2159$\,\,$ & $\,\,$\color{gr} 9.2159\color{black} $\,\,$ \\
$\,\,$0.5502$\,\,$ & $\,\,$ 1 $\,\,$ & $\,\,$5.0704$\,\,$ & $\,\,$\color{gr} 5.0704\color{black}   $\,\,$ \\
$\,\,$0.1085$\,\,$ & $\,\,$0.1972$\,\,$ & $\,\,$ 1 $\,\,$ & $\,\,$\color{gr} \color{blue} 1\color{black}  $\,\,$ \\
$\,\,$\color{gr} 0.1085\color{black} $\,\,$ & $\,\,$\color{gr} 0.1972\color{black} $\,\,$ & $\,\,$\color{gr} \color{blue} 1\color{black} $\,\,$ & $\,\,$ 1  $\,\,$ \\
\end{pmatrix},
\end{equation*}
\end{example}
\newpage
\begin{example}
\begin{equation*}
\mathbf{A} =
\begin{pmatrix}
$\,\,$ 1 $\,\,$ & $\,\,$3$\,\,$ & $\,\,$6$\,\,$ & $\,\,$9 $\,\,$ \\
$\,\,$ 1/3$\,\,$ & $\,\,$ 1 $\,\,$ & $\,\,$3$\,\,$ & $\,\,$2 $\,\,$ \\
$\,\,$ 1/6$\,\,$ & $\,\,$ 1/3$\,\,$ & $\,\,$ 1 $\,\,$ & $\,\,$2 $\,\,$ \\
$\,\,$ 1/9$\,\,$ & $\,\,$ 1/2$\,\,$ & $\,\,$ 1/2$\,\,$ & $\,\,$ 1  $\,\,$ \\
\end{pmatrix},
\qquad
\lambda_{\max} =
4.1031,
\qquad
CR = 0.0389
\end{equation*}

\begin{equation*}
\mathbf{w}^{EM} =
\begin{pmatrix}
\color{red} 0.613264\color{black} \\
0.211703\\
0.102456\\
0.072578
\end{pmatrix}\end{equation*}
\begin{equation*}
\left[ \frac{{w}^{EM}_i}{{w}^{EM}_j} \right] =
\begin{pmatrix}
$\,\,$ 1 $\,\,$ & $\,\,$\color{red} 2.8968\color{black} $\,\,$ & $\,\,$\color{red} 5.9857\color{black} $\,\,$ & $\,\,$\color{red} 8.4497\color{black} $\,\,$ \\
$\,\,$\color{red} 0.3452\color{black} $\,\,$ & $\,\,$ 1 $\,\,$ & $\,\,$2.0663$\,\,$ & $\,\,$2.9169  $\,\,$ \\
$\,\,$\color{red} 0.1671\color{black} $\,\,$ & $\,\,$0.4840$\,\,$ & $\,\,$ 1 $\,\,$ & $\,\,$1.4117 $\,\,$ \\
$\,\,$\color{red} 0.1183\color{black} $\,\,$ & $\,\,$0.3428$\,\,$ & $\,\,$0.7084$\,\,$ & $\,\,$ 1  $\,\,$ \\
\end{pmatrix},
\end{equation*}

\begin{equation*}
\mathbf{w}^{\prime} =
\begin{pmatrix}
0.613831\\
0.211392\\
0.102305\\
0.072471
\end{pmatrix} =
0.998532\cdot
\begin{pmatrix}
\color{gr} 0.614733\color{black} \\
0.211703\\
0.102456\\
0.072578
\end{pmatrix},
\end{equation*}
\begin{equation*}
\left[ \frac{{w}^{\prime}_i}{{w}^{\prime}_j} \right] =
\begin{pmatrix}
$\,\,$ 1 $\,\,$ & $\,\,$\color{gr} 2.9038\color{black} $\,\,$ & $\,\,$\color{gr} \color{blue} 6\color{black} $\,\,$ & $\,\,$\color{gr} 8.4700\color{black} $\,\,$ \\
$\,\,$\color{gr} 0.3444\color{black} $\,\,$ & $\,\,$ 1 $\,\,$ & $\,\,$2.0663$\,\,$ & $\,\,$2.9169  $\,\,$ \\
$\,\,$\color{gr} \color{blue}  1/6\color{black} $\,\,$ & $\,\,$0.4840$\,\,$ & $\,\,$ 1 $\,\,$ & $\,\,$1.4117 $\,\,$ \\
$\,\,$\color{gr} 0.1181\color{black} $\,\,$ & $\,\,$0.3428$\,\,$ & $\,\,$0.7084$\,\,$ & $\,\,$ 1  $\,\,$ \\
\end{pmatrix},
\end{equation*}
\end{example}
\newpage
\begin{example}
\begin{equation*}
\mathbf{A} =
\begin{pmatrix}
$\,\,$ 1 $\,\,$ & $\,\,$3$\,\,$ & $\,\,$6$\,\,$ & $\,\,$9 $\,\,$ \\
$\,\,$ 1/3$\,\,$ & $\,\,$ 1 $\,\,$ & $\,\,$6$\,\,$ & $\,\,$4 $\,\,$ \\
$\,\,$ 1/6$\,\,$ & $\,\,$ 1/6$\,\,$ & $\,\,$ 1 $\,\,$ & $\,\,$1 $\,\,$ \\
$\,\,$ 1/9$\,\,$ & $\,\,$ 1/4$\,\,$ & $\,\,$ 1 $\,\,$ & $\,\,$ 1  $\,\,$ \\
\end{pmatrix},
\qquad
\lambda_{\max} =
4.1031,
\qquad
CR = 0.0389
\end{equation*}

\begin{equation*}
\mathbf{w}^{EM} =
\begin{pmatrix}
0.588964\\
0.278142\\
0.067609\\
\color{red} 0.065284\color{black}
\end{pmatrix}\end{equation*}
\begin{equation*}
\left[ \frac{{w}^{EM}_i}{{w}^{EM}_j} \right] =
\begin{pmatrix}
$\,\,$ 1 $\,\,$ & $\,\,$2.1175$\,\,$ & $\,\,$8.7113$\,\,$ & $\,\,$\color{red} 9.0216\color{black} $\,\,$ \\
$\,\,$0.4723$\,\,$ & $\,\,$ 1 $\,\,$ & $\,\,$4.1140$\,\,$ & $\,\,$\color{red} 4.2605\color{black}   $\,\,$ \\
$\,\,$0.1148$\,\,$ & $\,\,$0.2431$\,\,$ & $\,\,$ 1 $\,\,$ & $\,\,$\color{red} 1.0356\color{black}  $\,\,$ \\
$\,\,$\color{red} 0.1108\color{black} $\,\,$ & $\,\,$\color{red} 0.2347\color{black} $\,\,$ & $\,\,$\color{red} 0.9656\color{black} $\,\,$ & $\,\,$ 1  $\,\,$ \\
\end{pmatrix},
\end{equation*}

\begin{equation*}
\mathbf{w}^{\prime} =
\begin{pmatrix}
0.588872\\
0.278099\\
0.067599\\
0.065430
\end{pmatrix} =
0.999844\cdot
\begin{pmatrix}
0.588964\\
0.278142\\
0.067609\\
\color{gr} 0.065440\color{black}
\end{pmatrix},
\end{equation*}
\begin{equation*}
\left[ \frac{{w}^{\prime}_i}{{w}^{\prime}_j} \right] =
\begin{pmatrix}
$\,\,$ 1 $\,\,$ & $\,\,$2.1175$\,\,$ & $\,\,$8.7113$\,\,$ & $\,\,$\color{gr} \color{blue} 9\color{black} $\,\,$ \\
$\,\,$0.4723$\,\,$ & $\,\,$ 1 $\,\,$ & $\,\,$4.1140$\,\,$ & $\,\,$\color{gr} 4.2503\color{black}   $\,\,$ \\
$\,\,$0.1148$\,\,$ & $\,\,$0.2431$\,\,$ & $\,\,$ 1 $\,\,$ & $\,\,$\color{gr} 1.0331\color{black}  $\,\,$ \\
$\,\,$\color{gr} \color{blue}  1/9\color{black} $\,\,$ & $\,\,$\color{gr} 0.2353\color{black} $\,\,$ & $\,\,$\color{gr} 0.9679\color{black} $\,\,$ & $\,\,$ 1  $\,\,$ \\
\end{pmatrix},
\end{equation*}
\end{example}
\newpage
\begin{example}
\begin{equation*}
\mathbf{A} =
\begin{pmatrix}
$\,\,$ 1 $\,\,$ & $\,\,$3$\,\,$ & $\,\,$6$\,\,$ & $\,\,$9 $\,\,$ \\
$\,\,$ 1/3$\,\,$ & $\,\,$ 1 $\,\,$ & $\,\,$7$\,\,$ & $\,\,$4 $\,\,$ \\
$\,\,$ 1/6$\,\,$ & $\,\,$ 1/7$\,\,$ & $\,\,$ 1 $\,\,$ & $\,\,$1 $\,\,$ \\
$\,\,$ 1/9$\,\,$ & $\,\,$ 1/4$\,\,$ & $\,\,$ 1 $\,\,$ & $\,\,$ 1  $\,\,$ \\
\end{pmatrix},
\qquad
\lambda_{\max} =
4.1365,
\qquad
CR = 0.0515
\end{equation*}

\begin{equation*}
\mathbf{w}^{EM} =
\begin{pmatrix}
0.583226\\
0.287990\\
0.064579\\
\color{red} 0.064205\color{black}
\end{pmatrix}\end{equation*}
\begin{equation*}
\left[ \frac{{w}^{EM}_i}{{w}^{EM}_j} \right] =
\begin{pmatrix}
$\,\,$ 1 $\,\,$ & $\,\,$2.0252$\,\,$ & $\,\,$9.0313$\,\,$ & $\,\,$\color{red} 9.0838\color{black} $\,\,$ \\
$\,\,$0.4938$\,\,$ & $\,\,$ 1 $\,\,$ & $\,\,$4.4595$\,\,$ & $\,\,$\color{red} 4.4855\color{black}   $\,\,$ \\
$\,\,$0.1107$\,\,$ & $\,\,$0.2242$\,\,$ & $\,\,$ 1 $\,\,$ & $\,\,$\color{red} 1.0058\color{black}  $\,\,$ \\
$\,\,$\color{red} 0.1101\color{black} $\,\,$ & $\,\,$\color{red} 0.2229\color{black} $\,\,$ & $\,\,$\color{red} 0.9942\color{black} $\,\,$ & $\,\,$ 1  $\,\,$ \\
\end{pmatrix},
\end{equation*}

\begin{equation*}
\mathbf{w}^{\prime} =
\begin{pmatrix}
0.583008\\
0.287882\\
0.064555\\
0.064555
\end{pmatrix} =
0.999627\cdot
\begin{pmatrix}
0.583226\\
0.287990\\
0.064579\\
\color{gr} 0.064579\color{black}
\end{pmatrix},
\end{equation*}
\begin{equation*}
\left[ \frac{{w}^{\prime}_i}{{w}^{\prime}_j} \right] =
\begin{pmatrix}
$\,\,$ 1 $\,\,$ & $\,\,$2.0252$\,\,$ & $\,\,$9.0313$\,\,$ & $\,\,$\color{gr} 9.0313\color{black} $\,\,$ \\
$\,\,$0.4938$\,\,$ & $\,\,$ 1 $\,\,$ & $\,\,$4.4595$\,\,$ & $\,\,$\color{gr} 4.4595\color{black}   $\,\,$ \\
$\,\,$0.1107$\,\,$ & $\,\,$0.2242$\,\,$ & $\,\,$ 1 $\,\,$ & $\,\,$\color{gr} \color{blue} 1\color{black}  $\,\,$ \\
$\,\,$\color{gr} 0.1107\color{black} $\,\,$ & $\,\,$\color{gr} 0.2242\color{black} $\,\,$ & $\,\,$\color{gr} \color{blue} 1\color{black} $\,\,$ & $\,\,$ 1  $\,\,$ \\
\end{pmatrix},
\end{equation*}
\end{example}
\newpage
\begin{example}
\begin{equation*}
\mathbf{A} =
\begin{pmatrix}
$\,\,$ 1 $\,\,$ & $\,\,$3$\,\,$ & $\,\,$6$\,\,$ & $\,\,$9 $\,\,$ \\
$\,\,$ 1/3$\,\,$ & $\,\,$ 1 $\,\,$ & $\,\,$8$\,\,$ & $\,\,$5 $\,\,$ \\
$\,\,$ 1/6$\,\,$ & $\,\,$ 1/8$\,\,$ & $\,\,$ 1 $\,\,$ & $\,\,$1 $\,\,$ \\
$\,\,$ 1/9$\,\,$ & $\,\,$ 1/5$\,\,$ & $\,\,$ 1 $\,\,$ & $\,\,$ 1  $\,\,$ \\
\end{pmatrix},
\qquad
\lambda_{\max} =
4.1655,
\qquad
CR = 0.0624
\end{equation*}

\begin{equation*}
\mathbf{w}^{EM} =
\begin{pmatrix}
0.573368\\
0.307009\\
0.060871\\
\color{red} 0.058752\color{black}
\end{pmatrix}\end{equation*}
\begin{equation*}
\left[ \frac{{w}^{EM}_i}{{w}^{EM}_j} \right] =
\begin{pmatrix}
$\,\,$ 1 $\,\,$ & $\,\,$1.8676$\,\,$ & $\,\,$9.4194$\,\,$ & $\,\,$\color{red} 9.7592\color{black} $\,\,$ \\
$\,\,$0.5354$\,\,$ & $\,\,$ 1 $\,\,$ & $\,\,$5.0436$\,\,$ & $\,\,$\color{red} 5.2255\color{black}   $\,\,$ \\
$\,\,$0.1062$\,\,$ & $\,\,$0.1983$\,\,$ & $\,\,$ 1 $\,\,$ & $\,\,$\color{red} 1.0361\color{black}  $\,\,$ \\
$\,\,$\color{red} 0.1025\color{black} $\,\,$ & $\,\,$\color{red} 0.1914\color{black} $\,\,$ & $\,\,$\color{red} 0.9652\color{black} $\,\,$ & $\,\,$ 1  $\,\,$ \\
\end{pmatrix},
\end{equation*}

\begin{equation*}
\mathbf{w}^{\prime} =
\begin{pmatrix}
0.572156\\
0.306360\\
0.060742\\
0.060742
\end{pmatrix} =
0.997885\cdot
\begin{pmatrix}
0.573368\\
0.307009\\
0.060871\\
\color{gr} 0.060871\color{black}
\end{pmatrix},
\end{equation*}
\begin{equation*}
\left[ \frac{{w}^{\prime}_i}{{w}^{\prime}_j} \right] =
\begin{pmatrix}
$\,\,$ 1 $\,\,$ & $\,\,$1.8676$\,\,$ & $\,\,$9.4194$\,\,$ & $\,\,$\color{gr} 9.4194\color{black} $\,\,$ \\
$\,\,$0.5354$\,\,$ & $\,\,$ 1 $\,\,$ & $\,\,$5.0436$\,\,$ & $\,\,$\color{gr} 5.0436\color{black}   $\,\,$ \\
$\,\,$0.1062$\,\,$ & $\,\,$0.1983$\,\,$ & $\,\,$ 1 $\,\,$ & $\,\,$\color{gr} \color{blue} 1\color{black}  $\,\,$ \\
$\,\,$\color{gr} 0.1062\color{black} $\,\,$ & $\,\,$\color{gr} 0.1983\color{black} $\,\,$ & $\,\,$\color{gr} \color{blue} 1\color{black} $\,\,$ & $\,\,$ 1  $\,\,$ \\
\end{pmatrix},
\end{equation*}
\end{example}
\newpage
\begin{example}
\begin{equation*}
\mathbf{A} =
\begin{pmatrix}
$\,\,$ 1 $\,\,$ & $\,\,$3$\,\,$ & $\,\,$6$\,\,$ & $\,\,$9 $\,\,$ \\
$\,\,$ 1/3$\,\,$ & $\,\,$ 1 $\,\,$ & $\,\,$9$\,\,$ & $\,\,$5 $\,\,$ \\
$\,\,$ 1/6$\,\,$ & $\,\,$ 1/9$\,\,$ & $\,\,$ 1 $\,\,$ & $\,\,$1 $\,\,$ \\
$\,\,$ 1/9$\,\,$ & $\,\,$ 1/5$\,\,$ & $\,\,$ 1 $\,\,$ & $\,\,$ 1  $\,\,$ \\
\end{pmatrix},
\qquad
\lambda_{\max} =
4.1966,
\qquad
CR = 0.0741
\end{equation*}

\begin{equation*}
\mathbf{w}^{EM} =
\begin{pmatrix}
0.568540\\
0.314953\\
0.058681\\
\color{red} 0.057825\color{black}
\end{pmatrix}\end{equation*}
\begin{equation*}
\left[ \frac{{w}^{EM}_i}{{w}^{EM}_j} \right] =
\begin{pmatrix}
$\,\,$ 1 $\,\,$ & $\,\,$1.8052$\,\,$ & $\,\,$9.6887$\,\,$ & $\,\,$\color{red} 9.8320\color{black} $\,\,$ \\
$\,\,$0.5540$\,\,$ & $\,\,$ 1 $\,\,$ & $\,\,$5.3672$\,\,$ & $\,\,$\color{red} 5.4466\color{black}   $\,\,$ \\
$\,\,$0.1032$\,\,$ & $\,\,$0.1863$\,\,$ & $\,\,$ 1 $\,\,$ & $\,\,$\color{red} 1.0148\color{black}  $\,\,$ \\
$\,\,$\color{red} 0.1017\color{black} $\,\,$ & $\,\,$\color{red} 0.1836\color{black} $\,\,$ & $\,\,$\color{red} 0.9854\color{black} $\,\,$ & $\,\,$ 1  $\,\,$ \\
\end{pmatrix},
\end{equation*}

\begin{equation*}
\mathbf{w}^{\prime} =
\begin{pmatrix}
0.568054\\
0.314684\\
0.058631\\
0.058631
\end{pmatrix} =
0.999145\cdot
\begin{pmatrix}
0.568540\\
0.314953\\
0.058681\\
\color{gr} 0.058681\color{black}
\end{pmatrix},
\end{equation*}
\begin{equation*}
\left[ \frac{{w}^{\prime}_i}{{w}^{\prime}_j} \right] =
\begin{pmatrix}
$\,\,$ 1 $\,\,$ & $\,\,$1.8052$\,\,$ & $\,\,$9.6887$\,\,$ & $\,\,$\color{gr} 9.6887\color{black} $\,\,$ \\
$\,\,$0.5540$\,\,$ & $\,\,$ 1 $\,\,$ & $\,\,$5.3672$\,\,$ & $\,\,$\color{gr} 5.3672\color{black}   $\,\,$ \\
$\,\,$0.1032$\,\,$ & $\,\,$0.1863$\,\,$ & $\,\,$ 1 $\,\,$ & $\,\,$\color{gr} \color{blue} 1\color{black}  $\,\,$ \\
$\,\,$\color{gr} 0.1032\color{black} $\,\,$ & $\,\,$\color{gr} 0.1863\color{black} $\,\,$ & $\,\,$\color{gr} \color{blue} 1\color{black} $\,\,$ & $\,\,$ 1  $\,\,$ \\
\end{pmatrix},
\end{equation*}
\end{example}
\newpage
\begin{example}
\begin{equation*}
\mathbf{A} =
\begin{pmatrix}
$\,\,$ 1 $\,\,$ & $\,\,$3$\,\,$ & $\,\,$7$\,\,$ & $\,\,$4 $\,\,$ \\
$\,\,$ 1/3$\,\,$ & $\,\,$ 1 $\,\,$ & $\,\,$4$\,\,$ & $\,\,$6 $\,\,$ \\
$\,\,$ 1/7$\,\,$ & $\,\,$ 1/4$\,\,$ & $\,\,$ 1 $\,\,$ & $\,\,$1 $\,\,$ \\
$\,\,$ 1/4$\,\,$ & $\,\,$ 1/6$\,\,$ & $\,\,$ 1 $\,\,$ & $\,\,$ 1  $\,\,$ \\
\end{pmatrix},
\qquad
\lambda_{\max} =
4.1964,
\qquad
CR = 0.0741
\end{equation*}

\begin{equation*}
\mathbf{w}^{EM} =
\begin{pmatrix}
0.544566\\
0.300940\\
\color{red} 0.073283\color{black} \\
0.081211
\end{pmatrix}\end{equation*}
\begin{equation*}
\left[ \frac{{w}^{EM}_i}{{w}^{EM}_j} \right] =
\begin{pmatrix}
$\,\,$ 1 $\,\,$ & $\,\,$1.8096$\,\,$ & $\,\,$\color{red} 7.4310\color{black} $\,\,$ & $\,\,$6.7056$\,\,$ \\
$\,\,$0.5526$\,\,$ & $\,\,$ 1 $\,\,$ & $\,\,$\color{red} 4.1065\color{black} $\,\,$ & $\,\,$3.7057  $\,\,$ \\
$\,\,$\color{red} 0.1346\color{black} $\,\,$ & $\,\,$\color{red} 0.2435\color{black} $\,\,$ & $\,\,$ 1 $\,\,$ & $\,\,$\color{red} 0.9024\color{black}  $\,\,$ \\
$\,\,$0.1491$\,\,$ & $\,\,$0.2699$\,\,$ & $\,\,$\color{red} 1.1082\color{black} $\,\,$ & $\,\,$ 1  $\,\,$ \\
\end{pmatrix},
\end{equation*}

\begin{equation*}
\mathbf{w}^{\prime} =
\begin{pmatrix}
0.543505\\
0.300353\\
0.075088\\
0.081053
\end{pmatrix} =
0.998052\cdot
\begin{pmatrix}
0.544566\\
0.300940\\
\color{gr} 0.075235\color{black} \\
0.081211
\end{pmatrix},
\end{equation*}
\begin{equation*}
\left[ \frac{{w}^{\prime}_i}{{w}^{\prime}_j} \right] =
\begin{pmatrix}
$\,\,$ 1 $\,\,$ & $\,\,$1.8096$\,\,$ & $\,\,$\color{gr} 7.2382\color{black} $\,\,$ & $\,\,$6.7056$\,\,$ \\
$\,\,$0.5526$\,\,$ & $\,\,$ 1 $\,\,$ & $\,\,$\color{gr} \color{blue} 4\color{black} $\,\,$ & $\,\,$3.7057  $\,\,$ \\
$\,\,$\color{gr} 0.1382\color{black} $\,\,$ & $\,\,$\color{gr} \color{blue}  1/4\color{black} $\,\,$ & $\,\,$ 1 $\,\,$ & $\,\,$\color{gr} 0.9264\color{black}  $\,\,$ \\
$\,\,$0.1491$\,\,$ & $\,\,$0.2699$\,\,$ & $\,\,$\color{gr} 1.0794\color{black} $\,\,$ & $\,\,$ 1  $\,\,$ \\
\end{pmatrix},
\end{equation*}
\end{example}
\newpage
\begin{example}
\begin{equation*}
\mathbf{A} =
\begin{pmatrix}
$\,\,$ 1 $\,\,$ & $\,\,$3$\,\,$ & $\,\,$7$\,\,$ & $\,\,$4 $\,\,$ \\
$\,\,$ 1/3$\,\,$ & $\,\,$ 1 $\,\,$ & $\,\,$4$\,\,$ & $\,\,$7 $\,\,$ \\
$\,\,$ 1/7$\,\,$ & $\,\,$ 1/4$\,\,$ & $\,\,$ 1 $\,\,$ & $\,\,$1 $\,\,$ \\
$\,\,$ 1/4$\,\,$ & $\,\,$ 1/7$\,\,$ & $\,\,$ 1 $\,\,$ & $\,\,$ 1  $\,\,$ \\
\end{pmatrix},
\qquad
\lambda_{\max} =
4.2395,
\qquad
CR = 0.0903
\end{equation*}

\begin{equation*}
\mathbf{w}^{EM} =
\begin{pmatrix}
0.539224\\
0.311534\\
\color{red} 0.071744\color{black} \\
0.077499
\end{pmatrix}\end{equation*}
\begin{equation*}
\left[ \frac{{w}^{EM}_i}{{w}^{EM}_j} \right] =
\begin{pmatrix}
$\,\,$ 1 $\,\,$ & $\,\,$1.7309$\,\,$ & $\,\,$\color{red} 7.5159\color{black} $\,\,$ & $\,\,$6.9579$\,\,$ \\
$\,\,$0.5777$\,\,$ & $\,\,$ 1 $\,\,$ & $\,\,$\color{red} 4.3423\color{black} $\,\,$ & $\,\,$4.0199  $\,\,$ \\
$\,\,$\color{red} 0.1331\color{black} $\,\,$ & $\,\,$\color{red} 0.2303\color{black} $\,\,$ & $\,\,$ 1 $\,\,$ & $\,\,$\color{red} 0.9257\color{black}  $\,\,$ \\
$\,\,$0.1437$\,\,$ & $\,\,$0.2488$\,\,$ & $\,\,$\color{red} 1.0802\color{black} $\,\,$ & $\,\,$ 1  $\,\,$ \\
\end{pmatrix},
\end{equation*}

\begin{equation*}
\mathbf{w}^{\prime} =
\begin{pmatrix}
0.536387\\
0.309895\\
0.076627\\
0.077091
\end{pmatrix} =
0.994740\cdot
\begin{pmatrix}
0.539224\\
0.311534\\
\color{gr} 0.077032\color{black} \\
0.077499
\end{pmatrix},
\end{equation*}
\begin{equation*}
\left[ \frac{{w}^{\prime}_i}{{w}^{\prime}_j} \right] =
\begin{pmatrix}
$\,\,$ 1 $\,\,$ & $\,\,$1.7309$\,\,$ & $\,\,$\color{gr} \color{blue} 7\color{black} $\,\,$ & $\,\,$6.9579$\,\,$ \\
$\,\,$0.5777$\,\,$ & $\,\,$ 1 $\,\,$ & $\,\,$\color{gr} 4.0442\color{black} $\,\,$ & $\,\,$4.0199  $\,\,$ \\
$\,\,$\color{gr} \color{blue}  1/7\color{black} $\,\,$ & $\,\,$\color{gr} 0.2473\color{black} $\,\,$ & $\,\,$ 1 $\,\,$ & $\,\,$\color{gr} 0.9940\color{black}  $\,\,$ \\
$\,\,$0.1437$\,\,$ & $\,\,$0.2488$\,\,$ & $\,\,$\color{gr} 1.0061\color{black} $\,\,$ & $\,\,$ 1  $\,\,$ \\
\end{pmatrix},
\end{equation*}
\end{example}
\newpage
\begin{example}
\begin{equation*}
\mathbf{A} =
\begin{pmatrix}
$\,\,$ 1 $\,\,$ & $\,\,$3$\,\,$ & $\,\,$7$\,\,$ & $\,\,$9 $\,\,$ \\
$\,\,$ 1/3$\,\,$ & $\,\,$ 1 $\,\,$ & $\,\,$4$\,\,$ & $\,\,$2 $\,\,$ \\
$\,\,$ 1/7$\,\,$ & $\,\,$ 1/4$\,\,$ & $\,\,$ 1 $\,\,$ & $\,\,$2 $\,\,$ \\
$\,\,$ 1/9$\,\,$ & $\,\,$ 1/2$\,\,$ & $\,\,$ 1/2$\,\,$ & $\,\,$ 1  $\,\,$ \\
\end{pmatrix},
\qquad
\lambda_{\max} =
4.1658,
\qquad
CR = 0.0625
\end{equation*}

\begin{equation*}
\mathbf{w}^{EM} =
\begin{pmatrix}
\color{red} 0.614493\color{black} \\
0.223904\\
0.090396\\
0.071207
\end{pmatrix}\end{equation*}
\begin{equation*}
\left[ \frac{{w}^{EM}_i}{{w}^{EM}_j} \right] =
\begin{pmatrix}
$\,\,$ 1 $\,\,$ & $\,\,$\color{red} 2.7445\color{black} $\,\,$ & $\,\,$\color{red} 6.7978\color{black} $\,\,$ & $\,\,$\color{red} 8.6296\color{black} $\,\,$ \\
$\,\,$\color{red} 0.3644\color{black} $\,\,$ & $\,\,$ 1 $\,\,$ & $\,\,$2.4769$\,\,$ & $\,\,$3.1444  $\,\,$ \\
$\,\,$\color{red} 0.1471\color{black} $\,\,$ & $\,\,$0.4037$\,\,$ & $\,\,$ 1 $\,\,$ & $\,\,$1.2695 $\,\,$ \\
$\,\,$\color{red} 0.1159\color{black} $\,\,$ & $\,\,$0.3180$\,\,$ & $\,\,$0.7877$\,\,$ & $\,\,$ 1  $\,\,$ \\
\end{pmatrix},
\end{equation*}

\begin{equation*}
\mathbf{w}^{\prime} =
\begin{pmatrix}
0.621414\\
0.219884\\
0.088773\\
0.069929
\end{pmatrix} =
0.982047\cdot
\begin{pmatrix}
\color{gr} 0.632774\color{black} \\
0.223904\\
0.090396\\
0.071207
\end{pmatrix},
\end{equation*}
\begin{equation*}
\left[ \frac{{w}^{\prime}_i}{{w}^{\prime}_j} \right] =
\begin{pmatrix}
$\,\,$ 1 $\,\,$ & $\,\,$\color{gr} 2.8261\color{black} $\,\,$ & $\,\,$\color{gr} \color{blue} 7\color{black} $\,\,$ & $\,\,$\color{gr} 8.8864\color{black} $\,\,$ \\
$\,\,$\color{gr} 0.3538\color{black} $\,\,$ & $\,\,$ 1 $\,\,$ & $\,\,$2.4769$\,\,$ & $\,\,$3.1444  $\,\,$ \\
$\,\,$\color{gr} \color{blue}  1/7\color{black} $\,\,$ & $\,\,$0.4037$\,\,$ & $\,\,$ 1 $\,\,$ & $\,\,$1.2695 $\,\,$ \\
$\,\,$\color{gr} 0.1125\color{black} $\,\,$ & $\,\,$0.3180$\,\,$ & $\,\,$0.7877$\,\,$ & $\,\,$ 1  $\,\,$ \\
\end{pmatrix},
\end{equation*}
\end{example}
\newpage
\begin{example}
\begin{equation*}
\mathbf{A} =
\begin{pmatrix}
$\,\,$ 1 $\,\,$ & $\,\,$3$\,\,$ & $\,\,$8$\,\,$ & $\,\,$5 $\,\,$ \\
$\,\,$ 1/3$\,\,$ & $\,\,$ 1 $\,\,$ & $\,\,$5$\,\,$ & $\,\,$8 $\,\,$ \\
$\,\,$ 1/8$\,\,$ & $\,\,$ 1/5$\,\,$ & $\,\,$ 1 $\,\,$ & $\,\,$1 $\,\,$ \\
$\,\,$ 1/5$\,\,$ & $\,\,$ 1/8$\,\,$ & $\,\,$ 1 $\,\,$ & $\,\,$ 1  $\,\,$ \\
\end{pmatrix},
\qquad
\lambda_{\max} =
4.2144,
\qquad
CR = 0.0808
\end{equation*}

\begin{equation*}
\mathbf{w}^{EM} =
\begin{pmatrix}
0.553822\\
0.318194\\
\color{red} 0.061895\color{black} \\
0.066089
\end{pmatrix}\end{equation*}
\begin{equation*}
\left[ \frac{{w}^{EM}_i}{{w}^{EM}_j} \right] =
\begin{pmatrix}
$\,\,$ 1 $\,\,$ & $\,\,$1.7405$\,\,$ & $\,\,$\color{red} 8.9477\color{black} $\,\,$ & $\,\,$8.3800$\,\,$ \\
$\,\,$0.5745$\,\,$ & $\,\,$ 1 $\,\,$ & $\,\,$\color{red} 5.1408\color{black} $\,\,$ & $\,\,$4.8146  $\,\,$ \\
$\,\,$\color{red} 0.1118\color{black} $\,\,$ & $\,\,$\color{red} 0.1945\color{black} $\,\,$ & $\,\,$ 1 $\,\,$ & $\,\,$\color{red} 0.9366\color{black}  $\,\,$ \\
$\,\,$0.1193$\,\,$ & $\,\,$0.2077$\,\,$ & $\,\,$\color{red} 1.0677\color{black} $\,\,$ & $\,\,$ 1  $\,\,$ \\
\end{pmatrix},
\end{equation*}

\begin{equation*}
\mathbf{w}^{\prime} =
\begin{pmatrix}
0.552858\\
0.317640\\
0.063528\\
0.065974
\end{pmatrix} =
0.998260\cdot
\begin{pmatrix}
0.553822\\
0.318194\\
\color{gr} 0.063639\color{black} \\
0.066089
\end{pmatrix},
\end{equation*}
\begin{equation*}
\left[ \frac{{w}^{\prime}_i}{{w}^{\prime}_j} \right] =
\begin{pmatrix}
$\,\,$ 1 $\,\,$ & $\,\,$1.7405$\,\,$ & $\,\,$\color{gr} 8.7026\color{black} $\,\,$ & $\,\,$8.3800$\,\,$ \\
$\,\,$0.5745$\,\,$ & $\,\,$ 1 $\,\,$ & $\,\,$\color{gr} \color{blue} 5\color{black} $\,\,$ & $\,\,$4.8146  $\,\,$ \\
$\,\,$\color{gr} 0.1149\color{black} $\,\,$ & $\,\,$\color{gr} \color{blue}  1/5\color{black} $\,\,$ & $\,\,$ 1 $\,\,$ & $\,\,$\color{gr} 0.9629\color{black}  $\,\,$ \\
$\,\,$0.1193$\,\,$ & $\,\,$0.2077$\,\,$ & $\,\,$\color{gr} 1.0385\color{black} $\,\,$ & $\,\,$ 1  $\,\,$ \\
\end{pmatrix},
\end{equation*}
\end{example}
\newpage
\begin{example}
\begin{equation*}
\mathbf{A} =
\begin{pmatrix}
$\,\,$ 1 $\,\,$ & $\,\,$3$\,\,$ & $\,\,$8$\,\,$ & $\,\,$7 $\,\,$ \\
$\,\,$ 1/3$\,\,$ & $\,\,$ 1 $\,\,$ & $\,\,$2$\,\,$ & $\,\,$3 $\,\,$ \\
$\,\,$ 1/8$\,\,$ & $\,\,$ 1/2$\,\,$ & $\,\,$ 1 $\,\,$ & $\,\,$2 $\,\,$ \\
$\,\,$ 1/7$\,\,$ & $\,\,$ 1/3$\,\,$ & $\,\,$ 1/2$\,\,$ & $\,\,$ 1  $\,\,$ \\
\end{pmatrix},
\qquad
\lambda_{\max} =
4.0576,
\qquad
CR = 0.0217
\end{equation*}

\begin{equation*}
\mathbf{w}^{EM} =
\begin{pmatrix}
0.625623\\
\color{red} 0.202769\color{black} \\
0.103369\\
0.068239
\end{pmatrix}\end{equation*}
\begin{equation*}
\left[ \frac{{w}^{EM}_i}{{w}^{EM}_j} \right] =
\begin{pmatrix}
$\,\,$ 1 $\,\,$ & $\,\,$\color{red} 3.0854\color{black} $\,\,$ & $\,\,$6.0523$\,\,$ & $\,\,$9.1682$\,\,$ \\
$\,\,$\color{red} 0.3241\color{black} $\,\,$ & $\,\,$ 1 $\,\,$ & $\,\,$\color{red} 1.9616\color{black} $\,\,$ & $\,\,$\color{red} 2.9715\color{black}   $\,\,$ \\
$\,\,$0.1652$\,\,$ & $\,\,$\color{red} 0.5098\color{black} $\,\,$ & $\,\,$ 1 $\,\,$ & $\,\,$1.5148 $\,\,$ \\
$\,\,$0.1091$\,\,$ & $\,\,$\color{red} 0.3365\color{black} $\,\,$ & $\,\,$0.6601$\,\,$ & $\,\,$ 1  $\,\,$ \\
\end{pmatrix},
\end{equation*}

\begin{equation*}
\mathbf{w}^{\prime} =
\begin{pmatrix}
0.624408\\
0.204318\\
0.103168\\
0.068106
\end{pmatrix} =
0.998057\cdot
\begin{pmatrix}
0.625623\\
\color{gr} 0.204716\color{black} \\
0.103369\\
0.068239
\end{pmatrix},
\end{equation*}
\begin{equation*}
\left[ \frac{{w}^{\prime}_i}{{w}^{\prime}_j} \right] =
\begin{pmatrix}
$\,\,$ 1 $\,\,$ & $\,\,$\color{gr} 3.0561\color{black} $\,\,$ & $\,\,$6.0523$\,\,$ & $\,\,$9.1682$\,\,$ \\
$\,\,$\color{gr} 0.3272\color{black} $\,\,$ & $\,\,$ 1 $\,\,$ & $\,\,$\color{gr} 1.9804\color{black} $\,\,$ & $\,\,$\color{gr} \color{blue} 3\color{black}   $\,\,$ \\
$\,\,$0.1652$\,\,$ & $\,\,$\color{gr} 0.5049\color{black} $\,\,$ & $\,\,$ 1 $\,\,$ & $\,\,$1.5148 $\,\,$ \\
$\,\,$0.1091$\,\,$ & $\,\,$\color{gr} \color{blue}  1/3\color{black} $\,\,$ & $\,\,$0.6601$\,\,$ & $\,\,$ 1  $\,\,$ \\
\end{pmatrix},
\end{equation*}
\end{example}
\newpage
\begin{example}
\begin{equation*}
\mathbf{A} =
\begin{pmatrix}
$\,\,$ 1 $\,\,$ & $\,\,$3$\,\,$ & $\,\,$8$\,\,$ & $\,\,$8 $\,\,$ \\
$\,\,$ 1/3$\,\,$ & $\,\,$ 1 $\,\,$ & $\,\,$2$\,\,$ & $\,\,$4 $\,\,$ \\
$\,\,$ 1/8$\,\,$ & $\,\,$ 1/2$\,\,$ & $\,\,$ 1 $\,\,$ & $\,\,$3 $\,\,$ \\
$\,\,$ 1/8$\,\,$ & $\,\,$ 1/4$\,\,$ & $\,\,$ 1/3$\,\,$ & $\,\,$ 1  $\,\,$ \\
\end{pmatrix},
\qquad
\lambda_{\max} =
4.1031,
\qquad
CR = 0.0389
\end{equation*}

\begin{equation*}
\mathbf{w}^{EM} =
\begin{pmatrix}
0.626673\\
\color{red} 0.208392\color{black} \\
0.110981\\
0.053954
\end{pmatrix}\end{equation*}
\begin{equation*}
\left[ \frac{{w}^{EM}_i}{{w}^{EM}_j} \right] =
\begin{pmatrix}
$\,\,$ 1 $\,\,$ & $\,\,$\color{red} 3.0072\color{black} $\,\,$ & $\,\,$5.6467$\,\,$ & $\,\,$11.6150$\,\,$ \\
$\,\,$\color{red} 0.3325\color{black} $\,\,$ & $\,\,$ 1 $\,\,$ & $\,\,$\color{red} 1.8777\color{black} $\,\,$ & $\,\,$\color{red} 3.8624\color{black}   $\,\,$ \\
$\,\,$0.1771$\,\,$ & $\,\,$\color{red} 0.5326\color{black} $\,\,$ & $\,\,$ 1 $\,\,$ & $\,\,$2.0570 $\,\,$ \\
$\,\,$0.0861$\,\,$ & $\,\,$\color{red} 0.2589\color{black} $\,\,$ & $\,\,$0.4862$\,\,$ & $\,\,$ 1  $\,\,$ \\
\end{pmatrix},
\end{equation*}

\begin{equation*}
\mathbf{w}^{\prime} =
\begin{pmatrix}
0.626360\\
0.208787\\
0.110926\\
0.053927
\end{pmatrix} =
0.999501\cdot
\begin{pmatrix}
0.626673\\
\color{gr} 0.208891\color{black} \\
0.110981\\
0.053954
\end{pmatrix},
\end{equation*}
\begin{equation*}
\left[ \frac{{w}^{\prime}_i}{{w}^{\prime}_j} \right] =
\begin{pmatrix}
$\,\,$ 1 $\,\,$ & $\,\,$\color{gr} \color{blue} 3\color{black} $\,\,$ & $\,\,$5.6467$\,\,$ & $\,\,$11.6150$\,\,$ \\
$\,\,$\color{gr} \color{blue}  1/3\color{black} $\,\,$ & $\,\,$ 1 $\,\,$ & $\,\,$\color{gr} 1.8822\color{black} $\,\,$ & $\,\,$\color{gr} 3.8717\color{black}   $\,\,$ \\
$\,\,$0.1771$\,\,$ & $\,\,$\color{gr} 0.5313\color{black} $\,\,$ & $\,\,$ 1 $\,\,$ & $\,\,$2.0570 $\,\,$ \\
$\,\,$0.0861$\,\,$ & $\,\,$\color{gr} 0.2583\color{black} $\,\,$ & $\,\,$0.4862$\,\,$ & $\,\,$ 1  $\,\,$ \\
\end{pmatrix},
\end{equation*}
\end{example}
\newpage
\begin{example}
\begin{equation*}
\mathbf{A} =
\begin{pmatrix}
$\,\,$ 1 $\,\,$ & $\,\,$3$\,\,$ & $\,\,$8$\,\,$ & $\,\,$9 $\,\,$ \\
$\,\,$ 1/3$\,\,$ & $\,\,$ 1 $\,\,$ & $\,\,$4$\,\,$ & $\,\,$2 $\,\,$ \\
$\,\,$ 1/8$\,\,$ & $\,\,$ 1/4$\,\,$ & $\,\,$ 1 $\,\,$ & $\,\,$2 $\,\,$ \\
$\,\,$ 1/9$\,\,$ & $\,\,$ 1/2$\,\,$ & $\,\,$ 1/2$\,\,$ & $\,\,$ 1  $\,\,$ \\
\end{pmatrix},
\qquad
\lambda_{\max} =
4.1664,
\qquad
CR = 0.0627
\end{equation*}

\begin{equation*}
\mathbf{w}^{EM} =
\begin{pmatrix}
\color{red} 0.624657\color{black} \\
0.218988\\
0.086238\\
0.070117
\end{pmatrix}\end{equation*}
\begin{equation*}
\left[ \frac{{w}^{EM}_i}{{w}^{EM}_j} \right] =
\begin{pmatrix}
$\,\,$ 1 $\,\,$ & $\,\,$\color{red} 2.8525\color{black} $\,\,$ & $\,\,$\color{red} 7.2434\color{black} $\,\,$ & $\,\,$\color{red} 8.9088\color{black} $\,\,$ \\
$\,\,$\color{red} 0.3506\color{black} $\,\,$ & $\,\,$ 1 $\,\,$ & $\,\,$2.5394$\,\,$ & $\,\,$3.1232  $\,\,$ \\
$\,\,$\color{red} 0.1381\color{black} $\,\,$ & $\,\,$0.3938$\,\,$ & $\,\,$ 1 $\,\,$ & $\,\,$1.2299 $\,\,$ \\
$\,\,$\color{red} 0.1122\color{black} $\,\,$ & $\,\,$0.3202$\,\,$ & $\,\,$0.8131$\,\,$ & $\,\,$ 1  $\,\,$ \\
\end{pmatrix},
\end{equation*}

\begin{equation*}
\mathbf{w}^{\prime} =
\begin{pmatrix}
0.627043\\
0.217596\\
0.085690\\
0.069671
\end{pmatrix} =
0.993644\cdot
\begin{pmatrix}
\color{gr} 0.631054\color{black} \\
0.218988\\
0.086238\\
0.070117
\end{pmatrix},
\end{equation*}
\begin{equation*}
\left[ \frac{{w}^{\prime}_i}{{w}^{\prime}_j} \right] =
\begin{pmatrix}
$\,\,$ 1 $\,\,$ & $\,\,$\color{gr} 2.8817\color{black} $\,\,$ & $\,\,$\color{gr} 7.3176\color{black} $\,\,$ & $\,\,$\color{gr} \color{blue} 9\color{black} $\,\,$ \\
$\,\,$\color{gr} 0.3470\color{black} $\,\,$ & $\,\,$ 1 $\,\,$ & $\,\,$2.5394$\,\,$ & $\,\,$3.1232  $\,\,$ \\
$\,\,$\color{gr} 0.1367\color{black} $\,\,$ & $\,\,$0.3938$\,\,$ & $\,\,$ 1 $\,\,$ & $\,\,$1.2299 $\,\,$ \\
$\,\,$\color{gr} \color{blue}  1/9\color{black} $\,\,$ & $\,\,$0.3202$\,\,$ & $\,\,$0.8131$\,\,$ & $\,\,$ 1  $\,\,$ \\
\end{pmatrix},
\end{equation*}
\end{example}
\newpage
\begin{example}
\begin{equation*}
\mathbf{A} =
\begin{pmatrix}
$\,\,$ 1 $\,\,$ & $\,\,$3$\,\,$ & $\,\,$8$\,\,$ & $\,\,$9 $\,\,$ \\
$\,\,$ 1/3$\,\,$ & $\,\,$ 1 $\,\,$ & $\,\,$5$\,\,$ & $\,\,$2 $\,\,$ \\
$\,\,$ 1/8$\,\,$ & $\,\,$ 1/5$\,\,$ & $\,\,$ 1 $\,\,$ & $\,\,$2 $\,\,$ \\
$\,\,$ 1/9$\,\,$ & $\,\,$ 1/2$\,\,$ & $\,\,$ 1/2$\,\,$ & $\,\,$ 1  $\,\,$ \\
\end{pmatrix},
\qquad
\lambda_{\max} =
4.2267,
\qquad
CR = 0.0855
\end{equation*}

\begin{equation*}
\mathbf{w}^{EM} =
\begin{pmatrix}
\color{red} 0.614838\color{black} \\
0.233474\\
0.081681\\
0.070006
\end{pmatrix}\end{equation*}
\begin{equation*}
\left[ \frac{{w}^{EM}_i}{{w}^{EM}_j} \right] =
\begin{pmatrix}
$\,\,$ 1 $\,\,$ & $\,\,$\color{red} 2.6334\color{black} $\,\,$ & $\,\,$\color{red} 7.5273\color{black} $\,\,$ & $\,\,$\color{red} 8.7826\color{black} $\,\,$ \\
$\,\,$\color{red} 0.3797\color{black} $\,\,$ & $\,\,$ 1 $\,\,$ & $\,\,$2.8584$\,\,$ & $\,\,$3.3350  $\,\,$ \\
$\,\,$\color{red} 0.1328\color{black} $\,\,$ & $\,\,$0.3498$\,\,$ & $\,\,$ 1 $\,\,$ & $\,\,$1.1668 $\,\,$ \\
$\,\,$\color{red} 0.1139\color{black} $\,\,$ & $\,\,$0.2998$\,\,$ & $\,\,$0.8571$\,\,$ & $\,\,$ 1  $\,\,$ \\
\end{pmatrix},
\end{equation*}

\begin{equation*}
\mathbf{w}^{\prime} =
\begin{pmatrix}
0.620613\\
0.229974\\
0.080456\\
0.068957
\end{pmatrix} =
0.985009\cdot
\begin{pmatrix}
\color{gr} 0.630058\color{black} \\
0.233474\\
0.081681\\
0.070006
\end{pmatrix},
\end{equation*}
\begin{equation*}
\left[ \frac{{w}^{\prime}_i}{{w}^{\prime}_j} \right] =
\begin{pmatrix}
$\,\,$ 1 $\,\,$ & $\,\,$\color{gr} 2.6986\color{black} $\,\,$ & $\,\,$\color{gr} 7.7137\color{black} $\,\,$ & $\,\,$\color{gr} \color{blue} 9\color{black} $\,\,$ \\
$\,\,$\color{gr} 0.3706\color{black} $\,\,$ & $\,\,$ 1 $\,\,$ & $\,\,$2.8584$\,\,$ & $\,\,$3.3350  $\,\,$ \\
$\,\,$\color{gr} 0.1296\color{black} $\,\,$ & $\,\,$0.3498$\,\,$ & $\,\,$ 1 $\,\,$ & $\,\,$1.1668 $\,\,$ \\
$\,\,$\color{gr} \color{blue}  1/9\color{black} $\,\,$ & $\,\,$0.2998$\,\,$ & $\,\,$0.8571$\,\,$ & $\,\,$ 1  $\,\,$ \\
\end{pmatrix},
\end{equation*}
\end{example}
\newpage
\begin{example}
\begin{equation*}
\mathbf{A} =
\begin{pmatrix}
$\,\,$ 1 $\,\,$ & $\,\,$3$\,\,$ & $\,\,$9$\,\,$ & $\,\,$5 $\,\,$ \\
$\,\,$ 1/3$\,\,$ & $\,\,$ 1 $\,\,$ & $\,\,$2$\,\,$ & $\,\,$3 $\,\,$ \\
$\,\,$ 1/9$\,\,$ & $\,\,$ 1/2$\,\,$ & $\,\,$ 1 $\,\,$ & $\,\,$2 $\,\,$ \\
$\,\,$ 1/5$\,\,$ & $\,\,$ 1/3$\,\,$ & $\,\,$ 1/2$\,\,$ & $\,\,$ 1  $\,\,$ \\
\end{pmatrix},
\qquad
\lambda_{\max} =
4.1433,
\qquad
CR = 0.0540
\end{equation*}

\begin{equation*}
\mathbf{w}^{EM} =
\begin{pmatrix}
0.614534\\
\color{red} 0.204741\color{black} \\
0.103455\\
0.077269
\end{pmatrix}\end{equation*}
\begin{equation*}
\left[ \frac{{w}^{EM}_i}{{w}^{EM}_j} \right] =
\begin{pmatrix}
$\,\,$ 1 $\,\,$ & $\,\,$\color{red} 3.0015\color{black} $\,\,$ & $\,\,$5.9401$\,\,$ & $\,\,$7.9531$\,\,$ \\
$\,\,$\color{red} 0.3332\color{black} $\,\,$ & $\,\,$ 1 $\,\,$ & $\,\,$\color{red} 1.9790\color{black} $\,\,$ & $\,\,$\color{red} 2.6497\color{black}   $\,\,$ \\
$\,\,$0.1683$\,\,$ & $\,\,$\color{red} 0.5053\color{black} $\,\,$ & $\,\,$ 1 $\,\,$ & $\,\,$1.3389 $\,\,$ \\
$\,\,$0.1257$\,\,$ & $\,\,$\color{red} 0.3774\color{black} $\,\,$ & $\,\,$0.7469$\,\,$ & $\,\,$ 1  $\,\,$ \\
\end{pmatrix},
\end{equation*}

\begin{equation*}
\mathbf{w}^{\prime} =
\begin{pmatrix}
0.614471\\
0.204824\\
0.103444\\
0.077261
\end{pmatrix} =
0.999896\cdot
\begin{pmatrix}
0.614534\\
\color{gr} 0.204845\color{black} \\
0.103455\\
0.077269
\end{pmatrix},
\end{equation*}
\begin{equation*}
\left[ \frac{{w}^{\prime}_i}{{w}^{\prime}_j} \right] =
\begin{pmatrix}
$\,\,$ 1 $\,\,$ & $\,\,$\color{gr} \color{blue} 3\color{black} $\,\,$ & $\,\,$5.9401$\,\,$ & $\,\,$7.9531$\,\,$ \\
$\,\,$\color{gr} \color{blue}  1/3\color{black} $\,\,$ & $\,\,$ 1 $\,\,$ & $\,\,$\color{gr} 1.9800\color{black} $\,\,$ & $\,\,$\color{gr} 2.6510\color{black}   $\,\,$ \\
$\,\,$0.1683$\,\,$ & $\,\,$\color{gr} 0.5050\color{black} $\,\,$ & $\,\,$ 1 $\,\,$ & $\,\,$1.3389 $\,\,$ \\
$\,\,$0.1257$\,\,$ & $\,\,$\color{gr} 0.3772\color{black} $\,\,$ & $\,\,$0.7469$\,\,$ & $\,\,$ 1  $\,\,$ \\
\end{pmatrix},
\end{equation*}
\end{example}
\newpage
\begin{example}
\begin{equation*}
\mathbf{A} =
\begin{pmatrix}
$\,\,$ 1 $\,\,$ & $\,\,$3$\,\,$ & $\,\,$9$\,\,$ & $\,\,$5 $\,\,$ \\
$\,\,$ 1/3$\,\,$ & $\,\,$ 1 $\,\,$ & $\,\,$2$\,\,$ & $\,\,$3 $\,\,$ \\
$\,\,$ 1/9$\,\,$ & $\,\,$ 1/2$\,\,$ & $\,\,$ 1 $\,\,$ & $\,\,$3 $\,\,$ \\
$\,\,$ 1/5$\,\,$ & $\,\,$ 1/3$\,\,$ & $\,\,$ 1/3$\,\,$ & $\,\,$ 1  $\,\,$ \\
\end{pmatrix},
\qquad
\lambda_{\max} =
4.2507,
\qquad
CR = 0.0946
\end{equation*}

\begin{equation*}
\mathbf{w}^{EM} =
\begin{pmatrix}
0.614117\\
\color{red} 0.199330\color{black} \\
0.116396\\
0.070158
\end{pmatrix}\end{equation*}
\begin{equation*}
\left[ \frac{{w}^{EM}_i}{{w}^{EM}_j} \right] =
\begin{pmatrix}
$\,\,$ 1 $\,\,$ & $\,\,$\color{red} 3.0809\color{black} $\,\,$ & $\,\,$5.2761$\,\,$ & $\,\,$8.7534$\,\,$ \\
$\,\,$\color{red} 0.3246\color{black} $\,\,$ & $\,\,$ 1 $\,\,$ & $\,\,$\color{red} 1.7125\color{black} $\,\,$ & $\,\,$\color{red} 2.8412\color{black}   $\,\,$ \\
$\,\,$0.1895$\,\,$ & $\,\,$\color{red} 0.5839\color{black} $\,\,$ & $\,\,$ 1 $\,\,$ & $\,\,$1.6591 $\,\,$ \\
$\,\,$0.1142$\,\,$ & $\,\,$\color{red} 0.3520\color{black} $\,\,$ & $\,\,$0.6028$\,\,$ & $\,\,$ 1  $\,\,$ \\
\end{pmatrix},
\end{equation*}

\begin{equation*}
\mathbf{w}^{\prime} =
\begin{pmatrix}
0.610833\\
0.203611\\
0.115773\\
0.069783
\end{pmatrix} =
0.994653\cdot
\begin{pmatrix}
0.614117\\
\color{gr} 0.204706\color{black} \\
0.116396\\
0.070158
\end{pmatrix},
\end{equation*}
\begin{equation*}
\left[ \frac{{w}^{\prime}_i}{{w}^{\prime}_j} \right] =
\begin{pmatrix}
$\,\,$ 1 $\,\,$ & $\,\,$\color{gr} \color{blue} 3\color{black} $\,\,$ & $\,\,$5.2761$\,\,$ & $\,\,$8.7534$\,\,$ \\
$\,\,$\color{gr} \color{blue}  1/3\color{black} $\,\,$ & $\,\,$ 1 $\,\,$ & $\,\,$\color{gr} 1.7587\color{black} $\,\,$ & $\,\,$\color{gr} 2.9178\color{black}   $\,\,$ \\
$\,\,$0.1895$\,\,$ & $\,\,$\color{gr} 0.5686\color{black} $\,\,$ & $\,\,$ 1 $\,\,$ & $\,\,$1.6591 $\,\,$ \\
$\,\,$0.1142$\,\,$ & $\,\,$\color{gr} 0.3427\color{black} $\,\,$ & $\,\,$0.6028$\,\,$ & $\,\,$ 1  $\,\,$ \\
\end{pmatrix},
\end{equation*}
\end{example}
\newpage
\begin{example}
\begin{equation*}
\mathbf{A} =
\begin{pmatrix}
$\,\,$ 1 $\,\,$ & $\,\,$3$\,\,$ & $\,\,$9$\,\,$ & $\,\,$5 $\,\,$ \\
$\,\,$ 1/3$\,\,$ & $\,\,$ 1 $\,\,$ & $\,\,$5$\,\,$ & $\,\,$7 $\,\,$ \\
$\,\,$ 1/9$\,\,$ & $\,\,$ 1/5$\,\,$ & $\,\,$ 1 $\,\,$ & $\,\,$1 $\,\,$ \\
$\,\,$ 1/5$\,\,$ & $\,\,$ 1/7$\,\,$ & $\,\,$ 1 $\,\,$ & $\,\,$ 1  $\,\,$ \\
\end{pmatrix},
\qquad
\lambda_{\max} =
4.1795,
\qquad
CR = 0.0677
\end{equation*}

\begin{equation*}
\mathbf{w}^{EM} =
\begin{pmatrix}
0.566299\\
0.304902\\
\color{red} 0.060462\color{black} \\
0.068337
\end{pmatrix}\end{equation*}
\begin{equation*}
\left[ \frac{{w}^{EM}_i}{{w}^{EM}_j} \right] =
\begin{pmatrix}
$\,\,$ 1 $\,\,$ & $\,\,$1.8573$\,\,$ & $\,\,$\color{red} 9.3662\color{black} $\,\,$ & $\,\,$8.2868$\,\,$ \\
$\,\,$0.5384$\,\,$ & $\,\,$ 1 $\,\,$ & $\,\,$\color{red} 5.0429\color{black} $\,\,$ & $\,\,$4.4617  $\,\,$ \\
$\,\,$\color{red} 0.1068\color{black} $\,\,$ & $\,\,$\color{red} 0.1983\color{black} $\,\,$ & $\,\,$ 1 $\,\,$ & $\,\,$\color{red} 0.8848\color{black}  $\,\,$ \\
$\,\,$0.1207$\,\,$ & $\,\,$0.2241$\,\,$ & $\,\,$\color{red} 1.1303\color{black} $\,\,$ & $\,\,$ 1  $\,\,$ \\
\end{pmatrix},
\end{equation*}

\begin{equation*}
\mathbf{w}^{\prime} =
\begin{pmatrix}
0.566005\\
0.304744\\
0.060949\\
0.068302
\end{pmatrix} =
0.999482\cdot
\begin{pmatrix}
0.566299\\
0.304902\\
\color{gr} 0.060980\color{black} \\
0.068337
\end{pmatrix},
\end{equation*}
\begin{equation*}
\left[ \frac{{w}^{\prime}_i}{{w}^{\prime}_j} \right] =
\begin{pmatrix}
$\,\,$ 1 $\,\,$ & $\,\,$1.8573$\,\,$ & $\,\,$\color{gr} 9.2866\color{black} $\,\,$ & $\,\,$8.2868$\,\,$ \\
$\,\,$0.5384$\,\,$ & $\,\,$ 1 $\,\,$ & $\,\,$\color{gr} \color{blue} 5\color{black} $\,\,$ & $\,\,$4.4617  $\,\,$ \\
$\,\,$\color{gr} 0.1077\color{black} $\,\,$ & $\,\,$\color{gr} \color{blue}  1/5\color{black} $\,\,$ & $\,\,$ 1 $\,\,$ & $\,\,$\color{gr} 0.8923\color{black}  $\,\,$ \\
$\,\,$0.1207$\,\,$ & $\,\,$0.2241$\,\,$ & $\,\,$\color{gr} 1.1206\color{black} $\,\,$ & $\,\,$ 1  $\,\,$ \\
\end{pmatrix},
\end{equation*}
\end{example}
\newpage
\begin{example}
\begin{equation*}
\mathbf{A} =
\begin{pmatrix}
$\,\,$ 1 $\,\,$ & $\,\,$3$\,\,$ & $\,\,$9$\,\,$ & $\,\,$5 $\,\,$ \\
$\,\,$ 1/3$\,\,$ & $\,\,$ 1 $\,\,$ & $\,\,$5$\,\,$ & $\,\,$8 $\,\,$ \\
$\,\,$ 1/9$\,\,$ & $\,\,$ 1/5$\,\,$ & $\,\,$ 1 $\,\,$ & $\,\,$1 $\,\,$ \\
$\,\,$ 1/5$\,\,$ & $\,\,$ 1/8$\,\,$ & $\,\,$ 1 $\,\,$ & $\,\,$ 1  $\,\,$ \\
\end{pmatrix},
\qquad
\lambda_{\max} =
4.2138,
\qquad
CR = 0.0806
\end{equation*}

\begin{equation*}
\mathbf{w}^{EM} =
\begin{pmatrix}
0.561218\\
0.313837\\
\color{red} 0.059346\color{black} \\
0.065599
\end{pmatrix}\end{equation*}
\begin{equation*}
\left[ \frac{{w}^{EM}_i}{{w}^{EM}_j} \right] =
\begin{pmatrix}
$\,\,$ 1 $\,\,$ & $\,\,$1.7882$\,\,$ & $\,\,$\color{red} 9.4567\color{black} $\,\,$ & $\,\,$8.5553$\,\,$ \\
$\,\,$0.5592$\,\,$ & $\,\,$ 1 $\,\,$ & $\,\,$\color{red} 5.2882\color{black} $\,\,$ & $\,\,$4.7842  $\,\,$ \\
$\,\,$\color{red} 0.1057\color{black} $\,\,$ & $\,\,$\color{red} 0.1891\color{black} $\,\,$ & $\,\,$ 1 $\,\,$ & $\,\,$\color{red} 0.9047\color{black}  $\,\,$ \\
$\,\,$0.1169$\,\,$ & $\,\,$0.2090$\,\,$ & $\,\,$\color{red} 1.1054\color{black} $\,\,$ & $\,\,$ 1  $\,\,$ \\
\end{pmatrix},
\end{equation*}

\begin{equation*}
\mathbf{w}^{\prime} =
\begin{pmatrix}
0.559533\\
0.312894\\
0.062170\\
0.065402
\end{pmatrix} =
0.996998\cdot
\begin{pmatrix}
0.561218\\
0.313837\\
\color{gr} 0.062358\color{black} \\
0.065599
\end{pmatrix},
\end{equation*}
\begin{equation*}
\left[ \frac{{w}^{\prime}_i}{{w}^{\prime}_j} \right] =
\begin{pmatrix}
$\,\,$ 1 $\,\,$ & $\,\,$1.7882$\,\,$ & $\,\,$\color{gr} \color{blue} 9\color{black} $\,\,$ & $\,\,$8.5553$\,\,$ \\
$\,\,$0.5592$\,\,$ & $\,\,$ 1 $\,\,$ & $\,\,$\color{gr} 5.0329\color{black} $\,\,$ & $\,\,$4.7842  $\,\,$ \\
$\,\,$\color{gr} \color{blue}  1/9\color{black} $\,\,$ & $\,\,$\color{gr} 0.1987\color{black} $\,\,$ & $\,\,$ 1 $\,\,$ & $\,\,$\color{gr} 0.9506\color{black}  $\,\,$ \\
$\,\,$0.1169$\,\,$ & $\,\,$0.2090$\,\,$ & $\,\,$\color{gr} 1.0520\color{black} $\,\,$ & $\,\,$ 1  $\,\,$ \\
\end{pmatrix},
\end{equation*}
\end{example}
\newpage
\begin{example}
\begin{equation*}
\mathbf{A} =
\begin{pmatrix}
$\,\,$ 1 $\,\,$ & $\,\,$3$\,\,$ & $\,\,$9$\,\,$ & $\,\,$5 $\,\,$ \\
$\,\,$ 1/3$\,\,$ & $\,\,$ 1 $\,\,$ & $\,\,$5$\,\,$ & $\,\,$9 $\,\,$ \\
$\,\,$ 1/9$\,\,$ & $\,\,$ 1/5$\,\,$ & $\,\,$ 1 $\,\,$ & $\,\,$1 $\,\,$ \\
$\,\,$ 1/5$\,\,$ & $\,\,$ 1/9$\,\,$ & $\,\,$ 1 $\,\,$ & $\,\,$ 1  $\,\,$ \\
\end{pmatrix},
\qquad
\lambda_{\max} =
4.2483,
\qquad
CR = 0.0936
\end{equation*}

\begin{equation*}
\mathbf{w}^{EM} =
\begin{pmatrix}
0.556379\\
0.322064\\
\color{red} 0.058327\color{black} \\
0.063229
\end{pmatrix}\end{equation*}
\begin{equation*}
\left[ \frac{{w}^{EM}_i}{{w}^{EM}_j} \right] =
\begin{pmatrix}
$\,\,$ 1 $\,\,$ & $\,\,$1.7275$\,\,$ & $\,\,$\color{red} 9.5390\color{black} $\,\,$ & $\,\,$8.7994$\,\,$ \\
$\,\,$0.5789$\,\,$ & $\,\,$ 1 $\,\,$ & $\,\,$\color{red} 5.5217\color{black} $\,\,$ & $\,\,$5.0936  $\,\,$ \\
$\,\,$\color{red} 0.1048\color{black} $\,\,$ & $\,\,$\color{red} 0.1811\color{black} $\,\,$ & $\,\,$ 1 $\,\,$ & $\,\,$\color{red} 0.9225\color{black}  $\,\,$ \\
$\,\,$0.1136$\,\,$ & $\,\,$0.1963$\,\,$ & $\,\,$\color{red} 1.0841\color{black} $\,\,$ & $\,\,$ 1  $\,\,$ \\
\end{pmatrix},
\end{equation*}

\begin{equation*}
\mathbf{w}^{\prime} =
\begin{pmatrix}
0.554443\\
0.320943\\
0.061605\\
0.063009
\end{pmatrix} =
0.996519\cdot
\begin{pmatrix}
0.556379\\
0.322064\\
\color{gr} 0.061820\color{black} \\
0.063229
\end{pmatrix},
\end{equation*}
\begin{equation*}
\left[ \frac{{w}^{\prime}_i}{{w}^{\prime}_j} \right] =
\begin{pmatrix}
$\,\,$ 1 $\,\,$ & $\,\,$1.7275$\,\,$ & $\,\,$\color{gr} \color{blue} 9\color{black} $\,\,$ & $\,\,$8.7994$\,\,$ \\
$\,\,$0.5789$\,\,$ & $\,\,$ 1 $\,\,$ & $\,\,$\color{gr} 5.2097\color{black} $\,\,$ & $\,\,$5.0936  $\,\,$ \\
$\,\,$\color{gr} \color{blue}  1/9\color{black} $\,\,$ & $\,\,$\color{gr} 0.1919\color{black} $\,\,$ & $\,\,$ 1 $\,\,$ & $\,\,$\color{gr} 0.9777\color{black}  $\,\,$ \\
$\,\,$0.1136$\,\,$ & $\,\,$0.1963$\,\,$ & $\,\,$\color{gr} 1.0228\color{black} $\,\,$ & $\,\,$ 1  $\,\,$ \\
\end{pmatrix},
\end{equation*}
\end{example}
\newpage
\begin{example}
\begin{equation*}
\mathbf{A} =
\begin{pmatrix}
$\,\,$ 1 $\,\,$ & $\,\,$3$\,\,$ & $\,\,$9$\,\,$ & $\,\,$5 $\,\,$ \\
$\,\,$ 1/3$\,\,$ & $\,\,$ 1 $\,\,$ & $\,\,$6$\,\,$ & $\,\,$9 $\,\,$ \\
$\,\,$ 1/9$\,\,$ & $\,\,$ 1/6$\,\,$ & $\,\,$ 1 $\,\,$ & $\,\,$1 $\,\,$ \\
$\,\,$ 1/5$\,\,$ & $\,\,$ 1/9$\,\,$ & $\,\,$ 1 $\,\,$ & $\,\,$ 1  $\,\,$ \\
\end{pmatrix},
\qquad
\lambda_{\max} =
4.2507,
\qquad
CR = 0.0946
\end{equation*}

\begin{equation*}
\mathbf{w}^{EM} =
\begin{pmatrix}
0.552588\\
0.330269\\
\color{red} 0.054952\color{black} \\
0.062191
\end{pmatrix}\end{equation*}
\begin{equation*}
\left[ \frac{{w}^{EM}_i}{{w}^{EM}_j} \right] =
\begin{pmatrix}
$\,\,$ 1 $\,\,$ & $\,\,$1.6731$\,\,$ & $\,\,$\color{red} 10.0559\color{black} $\,\,$ & $\,\,$8.8854$\,\,$ \\
$\,\,$0.5977$\,\,$ & $\,\,$ 1 $\,\,$ & $\,\,$\color{red} 6.0102\color{black} $\,\,$ & $\,\,$5.3106  $\,\,$ \\
$\,\,$\color{red} 0.0994\color{black} $\,\,$ & $\,\,$\color{red} 0.1664\color{black} $\,\,$ & $\,\,$ 1 $\,\,$ & $\,\,$\color{red} 0.8836\color{black}  $\,\,$ \\
$\,\,$0.1125$\,\,$ & $\,\,$0.1883$\,\,$ & $\,\,$\color{red} 1.1317\color{black} $\,\,$ & $\,\,$ 1  $\,\,$ \\
\end{pmatrix},
\end{equation*}

\begin{equation*}
\mathbf{w}^{\prime} =
\begin{pmatrix}
0.552537\\
0.330238\\
0.055040\\
0.062185
\end{pmatrix} =
0.999907\cdot
\begin{pmatrix}
0.552588\\
0.330269\\
\color{gr} 0.055045\color{black} \\
0.062191
\end{pmatrix},
\end{equation*}
\begin{equation*}
\left[ \frac{{w}^{\prime}_i}{{w}^{\prime}_j} \right] =
\begin{pmatrix}
$\,\,$ 1 $\,\,$ & $\,\,$1.6731$\,\,$ & $\,\,$\color{gr} 10.0389\color{black} $\,\,$ & $\,\,$8.8854$\,\,$ \\
$\,\,$0.5977$\,\,$ & $\,\,$ 1 $\,\,$ & $\,\,$\color{gr} \color{blue} 6\color{black} $\,\,$ & $\,\,$5.3106  $\,\,$ \\
$\,\,$\color{gr} 0.0996\color{black} $\,\,$ & $\,\,$\color{gr} \color{blue}  1/6\color{black} $\,\,$ & $\,\,$ 1 $\,\,$ & $\,\,$\color{gr} 0.8851\color{black}  $\,\,$ \\
$\,\,$0.1125$\,\,$ & $\,\,$0.1883$\,\,$ & $\,\,$\color{gr} 1.1298\color{black} $\,\,$ & $\,\,$ 1  $\,\,$ \\
\end{pmatrix},
\end{equation*}
\end{example}
\newpage
\begin{example}
\begin{equation*}
\mathbf{A} =
\begin{pmatrix}
$\,\,$ 1 $\,\,$ & $\,\,$3$\,\,$ & $\,\,$9$\,\,$ & $\,\,$6 $\,\,$ \\
$\,\,$ 1/3$\,\,$ & $\,\,$ 1 $\,\,$ & $\,\,$2$\,\,$ & $\,\,$3 $\,\,$ \\
$\,\,$ 1/9$\,\,$ & $\,\,$ 1/2$\,\,$ & $\,\,$ 1 $\,\,$ & $\,\,$2 $\,\,$ \\
$\,\,$ 1/6$\,\,$ & $\,\,$ 1/3$\,\,$ & $\,\,$ 1/2$\,\,$ & $\,\,$ 1  $\,\,$ \\
\end{pmatrix},
\qquad
\lambda_{\max} =
4.1031,
\qquad
CR = 0.0389
\end{equation*}

\begin{equation*}
\mathbf{w}^{EM} =
\begin{pmatrix}
0.625992\\
\color{red} 0.201487\color{black} \\
0.100985\\
0.071536
\end{pmatrix}\end{equation*}
\begin{equation*}
\left[ \frac{{w}^{EM}_i}{{w}^{EM}_j} \right] =
\begin{pmatrix}
$\,\,$ 1 $\,\,$ & $\,\,$\color{red} 3.1069\color{black} $\,\,$ & $\,\,$6.1989$\,\,$ & $\,\,$8.7507$\,\,$ \\
$\,\,$\color{red} 0.3219\color{black} $\,\,$ & $\,\,$ 1 $\,\,$ & $\,\,$\color{red} 1.9952\color{black} $\,\,$ & $\,\,$\color{red} 2.8166\color{black}   $\,\,$ \\
$\,\,$0.1613$\,\,$ & $\,\,$\color{red} 0.5012\color{black} $\,\,$ & $\,\,$ 1 $\,\,$ & $\,\,$1.4117 $\,\,$ \\
$\,\,$0.1143$\,\,$ & $\,\,$\color{red} 0.3550\color{black} $\,\,$ & $\,\,$0.7084$\,\,$ & $\,\,$ 1  $\,\,$ \\
\end{pmatrix},
\end{equation*}

\begin{equation*}
\mathbf{w}^{\prime} =
\begin{pmatrix}
0.625690\\
0.201872\\
0.100936\\
0.071502
\end{pmatrix} =
0.999517\cdot
\begin{pmatrix}
0.625992\\
\color{gr} 0.201970\color{black} \\
0.100985\\
0.071536
\end{pmatrix},
\end{equation*}
\begin{equation*}
\left[ \frac{{w}^{\prime}_i}{{w}^{\prime}_j} \right] =
\begin{pmatrix}
$\,\,$ 1 $\,\,$ & $\,\,$\color{gr} 3.0994\color{black} $\,\,$ & $\,\,$6.1989$\,\,$ & $\,\,$8.7507$\,\,$ \\
$\,\,$\color{gr} 0.3226\color{black} $\,\,$ & $\,\,$ 1 $\,\,$ & $\,\,$\color{gr} \color{blue} 2\color{black} $\,\,$ & $\,\,$\color{gr} 2.8233\color{black}   $\,\,$ \\
$\,\,$0.1613$\,\,$ & $\,\,$\color{gr} \color{blue}  1/2\color{black} $\,\,$ & $\,\,$ 1 $\,\,$ & $\,\,$1.4117 $\,\,$ \\
$\,\,$0.1143$\,\,$ & $\,\,$\color{gr} 0.3542\color{black} $\,\,$ & $\,\,$0.7084$\,\,$ & $\,\,$ 1  $\,\,$ \\
\end{pmatrix},
\end{equation*}
\end{example}
\newpage
\begin{example}
\begin{equation*}
\mathbf{A} =
\begin{pmatrix}
$\,\,$ 1 $\,\,$ & $\,\,$3$\,\,$ & $\,\,$9$\,\,$ & $\,\,$7 $\,\,$ \\
$\,\,$ 1/3$\,\,$ & $\,\,$ 1 $\,\,$ & $\,\,$2$\,\,$ & $\,\,$4 $\,\,$ \\
$\,\,$ 1/9$\,\,$ & $\,\,$ 1/2$\,\,$ & $\,\,$ 1 $\,\,$ & $\,\,$3 $\,\,$ \\
$\,\,$ 1/7$\,\,$ & $\,\,$ 1/4$\,\,$ & $\,\,$ 1/3$\,\,$ & $\,\,$ 1  $\,\,$ \\
\end{pmatrix},
\qquad
\lambda_{\max} =
4.1571,
\qquad
CR = 0.0593
\end{equation*}

\begin{equation*}
\mathbf{w}^{EM} =
\begin{pmatrix}
0.629258\\
\color{red} 0.206259\color{black} \\
0.108248\\
0.056235
\end{pmatrix}\end{equation*}
\begin{equation*}
\left[ \frac{{w}^{EM}_i}{{w}^{EM}_j} \right] =
\begin{pmatrix}
$\,\,$ 1 $\,\,$ & $\,\,$\color{red} 3.0508\color{black} $\,\,$ & $\,\,$5.8131$\,\,$ & $\,\,$11.1898$\,\,$ \\
$\,\,$\color{red} 0.3278\color{black} $\,\,$ & $\,\,$ 1 $\,\,$ & $\,\,$\color{red} 1.9054\color{black} $\,\,$ & $\,\,$\color{red} 3.6678\color{black}   $\,\,$ \\
$\,\,$0.1720$\,\,$ & $\,\,$\color{red} 0.5248\color{black} $\,\,$ & $\,\,$ 1 $\,\,$ & $\,\,$1.9249 $\,\,$ \\
$\,\,$0.0894$\,\,$ & $\,\,$\color{red} 0.2726\color{black} $\,\,$ & $\,\,$0.5195$\,\,$ & $\,\,$ 1  $\,\,$ \\
\end{pmatrix},
\end{equation*}

\begin{equation*}
\mathbf{w}^{\prime} =
\begin{pmatrix}
0.627067\\
0.209022\\
0.107871\\
0.056039
\end{pmatrix} =
0.996519\cdot
\begin{pmatrix}
0.629258\\
\color{gr} 0.209753\color{black} \\
0.108248\\
0.056235
\end{pmatrix},
\end{equation*}
\begin{equation*}
\left[ \frac{{w}^{\prime}_i}{{w}^{\prime}_j} \right] =
\begin{pmatrix}
$\,\,$ 1 $\,\,$ & $\,\,$\color{gr} \color{blue} 3\color{black} $\,\,$ & $\,\,$5.8131$\,\,$ & $\,\,$11.1898$\,\,$ \\
$\,\,$\color{gr} \color{blue}  1/3\color{black} $\,\,$ & $\,\,$ 1 $\,\,$ & $\,\,$\color{gr} 1.9377\color{black} $\,\,$ & $\,\,$\color{gr} 3.7299\color{black}   $\,\,$ \\
$\,\,$0.1720$\,\,$ & $\,\,$\color{gr} 0.5161\color{black} $\,\,$ & $\,\,$ 1 $\,\,$ & $\,\,$1.9249 $\,\,$ \\
$\,\,$0.0894$\,\,$ & $\,\,$\color{gr} 0.2681\color{black} $\,\,$ & $\,\,$0.5195$\,\,$ & $\,\,$ 1  $\,\,$ \\
\end{pmatrix},
\end{equation*}
\end{example}
\newpage
\begin{example}
\begin{equation*}
\mathbf{A} =
\begin{pmatrix}
$\,\,$ 1 $\,\,$ & $\,\,$3$\,\,$ & $\,\,$9$\,\,$ & $\,\,$7 $\,\,$ \\
$\,\,$ 1/3$\,\,$ & $\,\,$ 1 $\,\,$ & $\,\,$2$\,\,$ & $\,\,$4 $\,\,$ \\
$\,\,$ 1/9$\,\,$ & $\,\,$ 1/2$\,\,$ & $\,\,$ 1 $\,\,$ & $\,\,$4 $\,\,$ \\
$\,\,$ 1/7$\,\,$ & $\,\,$ 1/4$\,\,$ & $\,\,$ 1/4$\,\,$ & $\,\,$ 1  $\,\,$ \\
\end{pmatrix},
\qquad
\lambda_{\max} =
4.2359,
\qquad
CR = 0.0890
\end{equation*}

\begin{equation*}
\mathbf{w}^{EM} =
\begin{pmatrix}
0.627845\\
\color{red} 0.202147\color{black} \\
0.117589\\
0.052419
\end{pmatrix}\end{equation*}
\begin{equation*}
\left[ \frac{{w}^{EM}_i}{{w}^{EM}_j} \right] =
\begin{pmatrix}
$\,\,$ 1 $\,\,$ & $\,\,$\color{red} 3.1059\color{black} $\,\,$ & $\,\,$5.3393$\,\,$ & $\,\,$11.9774$\,\,$ \\
$\,\,$\color{red} 0.3220\color{black} $\,\,$ & $\,\,$ 1 $\,\,$ & $\,\,$\color{red} 1.7191\color{black} $\,\,$ & $\,\,$\color{red} 3.8563\color{black}   $\,\,$ \\
$\,\,$0.1873$\,\,$ & $\,\,$\color{red} 0.5817\color{black} $\,\,$ & $\,\,$ 1 $\,\,$ & $\,\,$2.2432 $\,\,$ \\
$\,\,$0.0835$\,\,$ & $\,\,$\color{red} 0.2593\color{black} $\,\,$ & $\,\,$0.4458$\,\,$ & $\,\,$ 1  $\,\,$ \\
\end{pmatrix},
\end{equation*}

\begin{equation*}
\mathbf{w}^{\prime} =
\begin{pmatrix}
0.623397\\
0.207799\\
0.116756\\
0.052048
\end{pmatrix} =
0.992916\cdot
\begin{pmatrix}
0.627845\\
\color{gr} 0.209282\color{black} \\
0.117589\\
0.052419
\end{pmatrix},
\end{equation*}
\begin{equation*}
\left[ \frac{{w}^{\prime}_i}{{w}^{\prime}_j} \right] =
\begin{pmatrix}
$\,\,$ 1 $\,\,$ & $\,\,$\color{gr} \color{blue} 3\color{black} $\,\,$ & $\,\,$5.3393$\,\,$ & $\,\,$11.9774$\,\,$ \\
$\,\,$\color{gr} \color{blue}  1/3\color{black} $\,\,$ & $\,\,$ 1 $\,\,$ & $\,\,$\color{gr} 1.7798\color{black} $\,\,$ & $\,\,$\color{gr} 3.9925\color{black}   $\,\,$ \\
$\,\,$0.1873$\,\,$ & $\,\,$\color{gr} 0.5619\color{black} $\,\,$ & $\,\,$ 1 $\,\,$ & $\,\,$2.2432 $\,\,$ \\
$\,\,$0.0835$\,\,$ & $\,\,$\color{gr} 0.2505\color{black} $\,\,$ & $\,\,$0.4458$\,\,$ & $\,\,$ 1  $\,\,$ \\
\end{pmatrix},
\end{equation*}
\end{example}
\newpage
\begin{example}
\begin{equation*}
\mathbf{A} =
\begin{pmatrix}
$\,\,$ 1 $\,\,$ & $\,\,$3$\,\,$ & $\,\,$9$\,\,$ & $\,\,$8 $\,\,$ \\
$\,\,$ 1/3$\,\,$ & $\,\,$ 1 $\,\,$ & $\,\,$2$\,\,$ & $\,\,$4 $\,\,$ \\
$\,\,$ 1/9$\,\,$ & $\,\,$ 1/2$\,\,$ & $\,\,$ 1 $\,\,$ & $\,\,$3 $\,\,$ \\
$\,\,$ 1/8$\,\,$ & $\,\,$ 1/4$\,\,$ & $\,\,$ 1/3$\,\,$ & $\,\,$ 1  $\,\,$ \\
\end{pmatrix},
\qquad
\lambda_{\max} =
4.1263,
\qquad
CR = 0.0476
\end{equation*}

\begin{equation*}
\mathbf{w}^{EM} =
\begin{pmatrix}
0.636996\\
\color{red} 0.203755\color{black} \\
0.106166\\
0.053083
\end{pmatrix}\end{equation*}
\begin{equation*}
\left[ \frac{{w}^{EM}_i}{{w}^{EM}_j} \right] =
\begin{pmatrix}
$\,\,$ 1 $\,\,$ & $\,\,$\color{red} 3.1263\color{black} $\,\,$ & $\,\,$6$\,\,$ & $\,\,$12$\,\,$ \\
$\,\,$\color{red} 0.3199\color{black} $\,\,$ & $\,\,$ 1 $\,\,$ & $\,\,$\color{red} 1.9192\color{black} $\,\,$ & $\,\,$\color{red} 3.8384\color{black}   $\,\,$ \\
$\,\,$1/6$\,\,$ & $\,\,$\color{red} 0.5210\color{black} $\,\,$ & $\,\,$ 1 $\,\,$ & $\,\,$2 $\,\,$ \\
$\,\,$1/12$\,\,$ & $\,\,$\color{red} 0.2605\color{black} $\,\,$ & $\,\,$1/2$\,\,$ & $\,\,$ 1  $\,\,$ \\
\end{pmatrix},
\end{equation*}

\begin{equation*}
\mathbf{w}^{\prime} =
\begin{pmatrix}
0.631579\\
0.210526\\
0.105263\\
0.052632
\end{pmatrix} =
0.991497\cdot
\begin{pmatrix}
0.636996\\
\color{gr} 0.212332\color{black} \\
0.106166\\
0.053083
\end{pmatrix},
\end{equation*}
\begin{equation*}
\left[ \frac{{w}^{\prime}_i}{{w}^{\prime}_j} \right] =
\begin{pmatrix}
$\,\,$ 1 $\,\,$ & $\,\,$\color{blue} 3\color{black} $\,\,$ & $\,\,$6$\,\,$ & $\,\,$12$\,\,$ \\
$\,\,$\color{blue} 1/3\color{black} $\,\,$ & $\,\,$ 1 $\,\,$ & $\,\,$\color{blue} 2\color{black} $\,\,$ & $\,\,$\color{gr} \color{blue} 4\color{black}   $\,\,$ \\
$\,\,$1/6$\,\,$ & $\,\,$\color{blue} 1/2\color{black} $\,\,$ & $\,\,$ 1 $\,\,$ & $\,\,$2 $\,\,$ \\
$\,\,$1/12$\,\,$ & $\,\,$\color{gr} \color{blue}  1/4\color{black} $\,\,$ & $\,\,$1/2$\,\,$ & $\,\,$ 1  $\,\,$ \\
\end{pmatrix},
\end{equation*}
\end{example}
\newpage
\begin{example}
\begin{equation*}
\mathbf{A} =
\begin{pmatrix}
$\,\,$ 1 $\,\,$ & $\,\,$3$\,\,$ & $\,\,$9$\,\,$ & $\,\,$8 $\,\,$ \\
$\,\,$ 1/3$\,\,$ & $\,\,$ 1 $\,\,$ & $\,\,$2$\,\,$ & $\,\,$5 $\,\,$ \\
$\,\,$ 1/9$\,\,$ & $\,\,$ 1/2$\,\,$ & $\,\,$ 1 $\,\,$ & $\,\,$4 $\,\,$ \\
$\,\,$ 1/8$\,\,$ & $\,\,$ 1/5$\,\,$ & $\,\,$ 1/4$\,\,$ & $\,\,$ 1  $\,\,$ \\
\end{pmatrix},
\qquad
\lambda_{\max} =
4.1972,
\qquad
CR = 0.0744
\end{equation*}

\begin{equation*}
\mathbf{w}^{EM} =
\begin{pmatrix}
0.631073\\
\color{red} 0.209348\color{black} \\
0.112976\\
0.046603
\end{pmatrix}\end{equation*}
\begin{equation*}
\left[ \frac{{w}^{EM}_i}{{w}^{EM}_j} \right] =
\begin{pmatrix}
$\,\,$ 1 $\,\,$ & $\,\,$\color{red} 3.0145\color{black} $\,\,$ & $\,\,$5.5859$\,\,$ & $\,\,$13.5415$\,\,$ \\
$\,\,$\color{red} 0.3317\color{black} $\,\,$ & $\,\,$ 1 $\,\,$ & $\,\,$\color{red} 1.8530\color{black} $\,\,$ & $\,\,$\color{red} 4.4922\color{black}   $\,\,$ \\
$\,\,$0.1790$\,\,$ & $\,\,$\color{red} 0.5397\color{black} $\,\,$ & $\,\,$ 1 $\,\,$ & $\,\,$2.4242 $\,\,$ \\
$\,\,$0.0738$\,\,$ & $\,\,$\color{red} 0.2226\color{black} $\,\,$ & $\,\,$0.4125$\,\,$ & $\,\,$ 1  $\,\,$ \\
\end{pmatrix},
\end{equation*}

\begin{equation*}
\mathbf{w}^{\prime} =
\begin{pmatrix}
0.630436\\
0.210145\\
0.112862\\
0.046556
\end{pmatrix} =
0.998992\cdot
\begin{pmatrix}
0.631073\\
\color{gr} 0.210358\color{black} \\
0.112976\\
0.046603
\end{pmatrix},
\end{equation*}
\begin{equation*}
\left[ \frac{{w}^{\prime}_i}{{w}^{\prime}_j} \right] =
\begin{pmatrix}
$\,\,$ 1 $\,\,$ & $\,\,$\color{gr} \color{blue} 3\color{black} $\,\,$ & $\,\,$5.5859$\,\,$ & $\,\,$13.5415$\,\,$ \\
$\,\,$\color{gr} \color{blue}  1/3\color{black} $\,\,$ & $\,\,$ 1 $\,\,$ & $\,\,$\color{gr} 1.8620\color{black} $\,\,$ & $\,\,$\color{gr} 4.5138\color{black}   $\,\,$ \\
$\,\,$0.1790$\,\,$ & $\,\,$\color{gr} 0.5371\color{black} $\,\,$ & $\,\,$ 1 $\,\,$ & $\,\,$2.4242 $\,\,$ \\
$\,\,$0.0738$\,\,$ & $\,\,$\color{gr} 0.2215\color{black} $\,\,$ & $\,\,$0.4125$\,\,$ & $\,\,$ 1  $\,\,$ \\
\end{pmatrix},
\end{equation*}
\end{example}
\newpage
\begin{example}
\begin{equation*}
\mathbf{A} =
\begin{pmatrix}
$\,\,$ 1 $\,\,$ & $\,\,$3$\,\,$ & $\,\,$9$\,\,$ & $\,\,$8 $\,\,$ \\
$\,\,$ 1/3$\,\,$ & $\,\,$ 1 $\,\,$ & $\,\,$2$\,\,$ & $\,\,$5 $\,\,$ \\
$\,\,$ 1/9$\,\,$ & $\,\,$ 1/2$\,\,$ & $\,\,$ 1 $\,\,$ & $\,\,$5 $\,\,$ \\
$\,\,$ 1/8$\,\,$ & $\,\,$ 1/5$\,\,$ & $\,\,$ 1/5$\,\,$ & $\,\,$ 1  $\,\,$ \\
\end{pmatrix},
\qquad
\lambda_{\max} =
4.2637,
\qquad
CR = 0.0994
\end{equation*}

\begin{equation*}
\mathbf{w}^{EM} =
\begin{pmatrix}
0.629618\\
\color{red} 0.205743\color{black} \\
0.120530\\
0.044109
\end{pmatrix}\end{equation*}
\begin{equation*}
\left[ \frac{{w}^{EM}_i}{{w}^{EM}_j} \right] =
\begin{pmatrix}
$\,\,$ 1 $\,\,$ & $\,\,$\color{red} 3.0602\color{black} $\,\,$ & $\,\,$5.2237$\,\,$ & $\,\,$14.2742$\,\,$ \\
$\,\,$\color{red} 0.3268\color{black} $\,\,$ & $\,\,$ 1 $\,\,$ & $\,\,$\color{red} 1.7070\color{black} $\,\,$ & $\,\,$\color{red} 4.6644\color{black}   $\,\,$ \\
$\,\,$0.1914$\,\,$ & $\,\,$\color{red} 0.5858\color{black} $\,\,$ & $\,\,$ 1 $\,\,$ & $\,\,$2.7326 $\,\,$ \\
$\,\,$0.0701$\,\,$ & $\,\,$\color{red} 0.2144\color{black} $\,\,$ & $\,\,$0.3660$\,\,$ & $\,\,$ 1  $\,\,$ \\
\end{pmatrix},
\end{equation*}

\begin{equation*}
\mathbf{w}^{\prime} =
\begin{pmatrix}
0.627028\\
0.209009\\
0.120035\\
0.043927
\end{pmatrix} =
0.995887\cdot
\begin{pmatrix}
0.629618\\
\color{gr} 0.209873\color{black} \\
0.120530\\
0.044109
\end{pmatrix},
\end{equation*}
\begin{equation*}
\left[ \frac{{w}^{\prime}_i}{{w}^{\prime}_j} \right] =
\begin{pmatrix}
$\,\,$ 1 $\,\,$ & $\,\,$\color{gr} \color{blue} 3\color{black} $\,\,$ & $\,\,$5.2237$\,\,$ & $\,\,$14.2742$\,\,$ \\
$\,\,$\color{gr} \color{blue}  1/3\color{black} $\,\,$ & $\,\,$ 1 $\,\,$ & $\,\,$\color{gr} 1.7412\color{black} $\,\,$ & $\,\,$\color{gr} 4.7581\color{black}   $\,\,$ \\
$\,\,$0.1914$\,\,$ & $\,\,$\color{gr} 0.5743\color{black} $\,\,$ & $\,\,$ 1 $\,\,$ & $\,\,$2.7326 $\,\,$ \\
$\,\,$0.0701$\,\,$ & $\,\,$\color{gr} 0.2102\color{black} $\,\,$ & $\,\,$0.3660$\,\,$ & $\,\,$ 1  $\,\,$ \\
\end{pmatrix},
\end{equation*}
\end{example}
\newpage
\begin{example}
\begin{equation*}
\mathbf{A} =
\begin{pmatrix}
$\,\,$ 1 $\,\,$ & $\,\,$3$\,\,$ & $\,\,$9$\,\,$ & $\,\,$9 $\,\,$ \\
$\,\,$ 1/3$\,\,$ & $\,\,$ 1 $\,\,$ & $\,\,$2$\,\,$ & $\,\,$4 $\,\,$ \\
$\,\,$ 1/9$\,\,$ & $\,\,$ 1/2$\,\,$ & $\,\,$ 1 $\,\,$ & $\,\,$3 $\,\,$ \\
$\,\,$ 1/9$\,\,$ & $\,\,$ 1/4$\,\,$ & $\,\,$ 1/3$\,\,$ & $\,\,$ 1  $\,\,$ \\
\end{pmatrix},
\qquad
\lambda_{\max} =
4.1031,
\qquad
CR = 0.0389
\end{equation*}

\begin{equation*}
\mathbf{w}^{EM} =
\begin{pmatrix}
0.643737\\
\color{red} 0.201459\color{black} \\
0.104318\\
0.050486
\end{pmatrix}\end{equation*}
\begin{equation*}
\left[ \frac{{w}^{EM}_i}{{w}^{EM}_j} \right] =
\begin{pmatrix}
$\,\,$ 1 $\,\,$ & $\,\,$\color{red} 3.1954\color{black} $\,\,$ & $\,\,$6.1709$\,\,$ & $\,\,$12.7509$\,\,$ \\
$\,\,$\color{red} 0.3130\color{black} $\,\,$ & $\,\,$ 1 $\,\,$ & $\,\,$\color{red} 1.9312\color{black} $\,\,$ & $\,\,$\color{red} 3.9904\color{black}   $\,\,$ \\
$\,\,$0.1621$\,\,$ & $\,\,$\color{red} 0.5178\color{black} $\,\,$ & $\,\,$ 1 $\,\,$ & $\,\,$2.0663 $\,\,$ \\
$\,\,$0.0784$\,\,$ & $\,\,$\color{red} 0.2506\color{black} $\,\,$ & $\,\,$0.4840$\,\,$ & $\,\,$ 1  $\,\,$ \\
\end{pmatrix},
\end{equation*}

\begin{equation*}
\mathbf{w}^{\prime} =
\begin{pmatrix}
0.643427\\
0.201845\\
0.104267\\
0.050461
\end{pmatrix} =
0.999517\cdot
\begin{pmatrix}
0.643737\\
\color{gr} 0.201942\color{black} \\
0.104318\\
0.050486
\end{pmatrix},
\end{equation*}
\begin{equation*}
\left[ \frac{{w}^{\prime}_i}{{w}^{\prime}_j} \right] =
\begin{pmatrix}
$\,\,$ 1 $\,\,$ & $\,\,$\color{gr} 3.1877\color{black} $\,\,$ & $\,\,$6.1709$\,\,$ & $\,\,$12.7509$\,\,$ \\
$\,\,$\color{gr} 0.3137\color{black} $\,\,$ & $\,\,$ 1 $\,\,$ & $\,\,$\color{gr} 1.9358\color{black} $\,\,$ & $\,\,$\color{gr} \color{blue} 4\color{black}   $\,\,$ \\
$\,\,$0.1621$\,\,$ & $\,\,$\color{gr} 0.5166\color{black} $\,\,$ & $\,\,$ 1 $\,\,$ & $\,\,$2.0663 $\,\,$ \\
$\,\,$0.0784$\,\,$ & $\,\,$\color{gr} \color{blue}  1/4\color{black} $\,\,$ & $\,\,$0.4840$\,\,$ & $\,\,$ 1  $\,\,$ \\
\end{pmatrix},
\end{equation*}
\end{example}
\newpage
\begin{example}
\begin{equation*}
\mathbf{A} =
\begin{pmatrix}
$\,\,$ 1 $\,\,$ & $\,\,$3$\,\,$ & $\,\,$9$\,\,$ & $\,\,$9 $\,\,$ \\
$\,\,$ 1/3$\,\,$ & $\,\,$ 1 $\,\,$ & $\,\,$2$\,\,$ & $\,\,$5 $\,\,$ \\
$\,\,$ 1/9$\,\,$ & $\,\,$ 1/2$\,\,$ & $\,\,$ 1 $\,\,$ & $\,\,$4 $\,\,$ \\
$\,\,$ 1/9$\,\,$ & $\,\,$ 1/5$\,\,$ & $\,\,$ 1/4$\,\,$ & $\,\,$ 1  $\,\,$ \\
\end{pmatrix},
\qquad
\lambda_{\max} =
4.1655,
\qquad
CR = 0.0624
\end{equation*}

\begin{equation*}
\mathbf{w}^{EM} =
\begin{pmatrix}
0.637635\\
\color{red} 0.207138\color{black} \\
0.110993\\
0.044234
\end{pmatrix}\end{equation*}
\begin{equation*}
\left[ \frac{{w}^{EM}_i}{{w}^{EM}_j} \right] =
\begin{pmatrix}
$\,\,$ 1 $\,\,$ & $\,\,$\color{red} 3.0783\color{black} $\,\,$ & $\,\,$5.7448$\,\,$ & $\,\,$14.4151$\,\,$ \\
$\,\,$\color{red} 0.3249\color{black} $\,\,$ & $\,\,$ 1 $\,\,$ & $\,\,$\color{red} 1.8662\color{black} $\,\,$ & $\,\,$\color{red} 4.6828\color{black}   $\,\,$ \\
$\,\,$0.1741$\,\,$ & $\,\,$\color{red} 0.5358\color{black} $\,\,$ & $\,\,$ 1 $\,\,$ & $\,\,$2.5092 $\,\,$ \\
$\,\,$0.0694$\,\,$ & $\,\,$\color{red} 0.2135\color{black} $\,\,$ & $\,\,$0.3985$\,\,$ & $\,\,$ 1  $\,\,$ \\
\end{pmatrix},
\end{equation*}

\begin{equation*}
\mathbf{w}^{\prime} =
\begin{pmatrix}
0.634206\\
0.211402\\
0.110396\\
0.043996
\end{pmatrix} =
0.994622\cdot
\begin{pmatrix}
0.637635\\
\color{gr} 0.212545\color{black} \\
0.110993\\
0.044234
\end{pmatrix},
\end{equation*}
\begin{equation*}
\left[ \frac{{w}^{\prime}_i}{{w}^{\prime}_j} \right] =
\begin{pmatrix}
$\,\,$ 1 $\,\,$ & $\,\,$\color{gr} \color{blue} 3\color{black} $\,\,$ & $\,\,$5.7448$\,\,$ & $\,\,$14.4151$\,\,$ \\
$\,\,$\color{gr} \color{blue}  1/3\color{black} $\,\,$ & $\,\,$ 1 $\,\,$ & $\,\,$\color{gr} 1.9149\color{black} $\,\,$ & $\,\,$\color{gr} 4.8050\color{black}   $\,\,$ \\
$\,\,$0.1741$\,\,$ & $\,\,$\color{gr} 0.5222\color{black} $\,\,$ & $\,\,$ 1 $\,\,$ & $\,\,$2.5092 $\,\,$ \\
$\,\,$0.0694$\,\,$ & $\,\,$\color{gr} 0.2081\color{black} $\,\,$ & $\,\,$0.3985$\,\,$ & $\,\,$ 1  $\,\,$ \\
\end{pmatrix},
\end{equation*}
\end{example}
\newpage
\begin{example}
\begin{equation*}
\mathbf{A} =
\begin{pmatrix}
$\,\,$ 1 $\,\,$ & $\,\,$3$\,\,$ & $\,\,$9$\,\,$ & $\,\,$9 $\,\,$ \\
$\,\,$ 1/3$\,\,$ & $\,\,$ 1 $\,\,$ & $\,\,$2$\,\,$ & $\,\,$5 $\,\,$ \\
$\,\,$ 1/9$\,\,$ & $\,\,$ 1/2$\,\,$ & $\,\,$ 1 $\,\,$ & $\,\,$5 $\,\,$ \\
$\,\,$ 1/9$\,\,$ & $\,\,$ 1/5$\,\,$ & $\,\,$ 1/5$\,\,$ & $\,\,$ 1  $\,\,$ \\
\end{pmatrix},
\qquad
\lambda_{\max} =
4.2277,
\qquad
CR = 0.0859
\end{equation*}

\begin{equation*}
\mathbf{w}^{EM} =
\begin{pmatrix}
0.636019\\
\color{red} 0.203823\color{black} \\
0.118304\\
0.041854
\end{pmatrix}\end{equation*}
\begin{equation*}
\left[ \frac{{w}^{EM}_i}{{w}^{EM}_j} \right] =
\begin{pmatrix}
$\,\,$ 1 $\,\,$ & $\,\,$\color{red} 3.1204\color{black} $\,\,$ & $\,\,$5.3761$\,\,$ & $\,\,$15.1960$\,\,$ \\
$\,\,$\color{red} 0.3205\color{black} $\,\,$ & $\,\,$ 1 $\,\,$ & $\,\,$\color{red} 1.7229\color{black} $\,\,$ & $\,\,$\color{red} 4.8698\color{black}   $\,\,$ \\
$\,\,$0.1860$\,\,$ & $\,\,$\color{red} 0.5804\color{black} $\,\,$ & $\,\,$ 1 $\,\,$ & $\,\,$2.8266 $\,\,$ \\
$\,\,$0.0658$\,\,$ & $\,\,$\color{red} 0.2053\color{black} $\,\,$ & $\,\,$0.3538$\,\,$ & $\,\,$ 1  $\,\,$ \\
\end{pmatrix},
\end{equation*}

\begin{equation*}
\mathbf{w}^{\prime} =
\begin{pmatrix}
0.632572\\
0.208138\\
0.117663\\
0.041628
\end{pmatrix} =
0.994581\cdot
\begin{pmatrix}
0.636019\\
\color{gr} 0.209272\color{black} \\
0.118304\\
0.041854
\end{pmatrix},
\end{equation*}
\begin{equation*}
\left[ \frac{{w}^{\prime}_i}{{w}^{\prime}_j} \right] =
\begin{pmatrix}
$\,\,$ 1 $\,\,$ & $\,\,$\color{gr} 3.0392\color{black} $\,\,$ & $\,\,$5.3761$\,\,$ & $\,\,$15.1960$\,\,$ \\
$\,\,$\color{gr} 0.3290\color{black} $\,\,$ & $\,\,$ 1 $\,\,$ & $\,\,$\color{gr} 1.7689\color{black} $\,\,$ & $\,\,$\color{gr} \color{blue} 5\color{black}   $\,\,$ \\
$\,\,$0.1860$\,\,$ & $\,\,$\color{gr} 0.5653\color{black} $\,\,$ & $\,\,$ 1 $\,\,$ & $\,\,$2.8266 $\,\,$ \\
$\,\,$0.0658$\,\,$ & $\,\,$\color{gr} \color{blue}  1/5\color{black} $\,\,$ & $\,\,$0.3538$\,\,$ & $\,\,$ 1  $\,\,$ \\
\end{pmatrix},
\end{equation*}
\end{example}
\newpage
\begin{example}
\begin{equation*}
\mathbf{A} =
\begin{pmatrix}
$\,\,$ 1 $\,\,$ & $\,\,$4$\,\,$ & $\,\,$2$\,\,$ & $\,\,$4 $\,\,$ \\
$\,\,$ 1/4$\,\,$ & $\,\,$ 1 $\,\,$ & $\,\,$1$\,\,$ & $\,\,$5 $\,\,$ \\
$\,\,$ 1/2$\,\,$ & $\,\,$ 1 $\,\,$ & $\,\,$ 1 $\,\,$ & $\,\,$3 $\,\,$ \\
$\,\,$ 1/4$\,\,$ & $\,\,$ 1/5$\,\,$ & $\,\,$ 1/3$\,\,$ & $\,\,$ 1  $\,\,$ \\
\end{pmatrix},
\qquad
\lambda_{\max} =
4.2277,
\qquad
CR = 0.0859
\end{equation*}

\begin{equation*}
\mathbf{w}^{EM} =
\begin{pmatrix}
0.494489\\
0.218679\\
\color{red} 0.212986\color{black} \\
0.073846
\end{pmatrix}\end{equation*}
\begin{equation*}
\left[ \frac{{w}^{EM}_i}{{w}^{EM}_j} \right] =
\begin{pmatrix}
$\,\,$ 1 $\,\,$ & $\,\,$2.2612$\,\,$ & $\,\,$\color{red} 2.3217\color{black} $\,\,$ & $\,\,$6.6962$\,\,$ \\
$\,\,$0.4422$\,\,$ & $\,\,$ 1 $\,\,$ & $\,\,$\color{red} 1.0267\color{black} $\,\,$ & $\,\,$2.9613  $\,\,$ \\
$\,\,$\color{red} 0.4307\color{black} $\,\,$ & $\,\,$\color{red} 0.9740\color{black} $\,\,$ & $\,\,$ 1 $\,\,$ & $\,\,$\color{red} 2.8842\color{black}  $\,\,$ \\
$\,\,$0.1493$\,\,$ & $\,\,$0.3377$\,\,$ & $\,\,$\color{red} 0.3467\color{black} $\,\,$ & $\,\,$ 1  $\,\,$ \\
\end{pmatrix},
\end{equation*}

\begin{equation*}
\mathbf{w}^{\prime} =
\begin{pmatrix}
0.491689\\
0.217441\\
0.217441\\
0.073428
\end{pmatrix} =
0.994339\cdot
\begin{pmatrix}
0.494489\\
0.218679\\
\color{gr} 0.218679\color{black} \\
0.073846
\end{pmatrix},
\end{equation*}
\begin{equation*}
\left[ \frac{{w}^{\prime}_i}{{w}^{\prime}_j} \right] =
\begin{pmatrix}
$\,\,$ 1 $\,\,$ & $\,\,$2.2612$\,\,$ & $\,\,$\color{gr} 2.2612\color{black} $\,\,$ & $\,\,$6.6962$\,\,$ \\
$\,\,$0.4422$\,\,$ & $\,\,$ 1 $\,\,$ & $\,\,$\color{gr} \color{blue} 1\color{black} $\,\,$ & $\,\,$2.9613  $\,\,$ \\
$\,\,$\color{gr} 0.4422\color{black} $\,\,$ & $\,\,$\color{gr} \color{blue} 1\color{black} $\,\,$ & $\,\,$ 1 $\,\,$ & $\,\,$\color{gr} 2.9613\color{black}  $\,\,$ \\
$\,\,$0.1493$\,\,$ & $\,\,$0.3377$\,\,$ & $\,\,$\color{gr} 0.3377\color{black} $\,\,$ & $\,\,$ 1  $\,\,$ \\
\end{pmatrix},
\end{equation*}
\end{example}
\newpage
\begin{example}
\begin{equation*}
\mathbf{A} =
\begin{pmatrix}
$\,\,$ 1 $\,\,$ & $\,\,$4$\,\,$ & $\,\,$2$\,\,$ & $\,\,$5 $\,\,$ \\
$\,\,$ 1/4$\,\,$ & $\,\,$ 1 $\,\,$ & $\,\,$1$\,\,$ & $\,\,$7 $\,\,$ \\
$\,\,$ 1/2$\,\,$ & $\,\,$ 1 $\,\,$ & $\,\,$ 1 $\,\,$ & $\,\,$4 $\,\,$ \\
$\,\,$ 1/5$\,\,$ & $\,\,$ 1/7$\,\,$ & $\,\,$ 1/4$\,\,$ & $\,\,$ 1  $\,\,$ \\
\end{pmatrix},
\qquad
\lambda_{\max} =
4.2610,
\qquad
CR = 0.0984
\end{equation*}

\begin{equation*}
\mathbf{w}^{EM} =
\begin{pmatrix}
0.499168\\
0.227315\\
\color{red} 0.216358\color{black} \\
0.057160
\end{pmatrix}\end{equation*}
\begin{equation*}
\left[ \frac{{w}^{EM}_i}{{w}^{EM}_j} \right] =
\begin{pmatrix}
$\,\,$ 1 $\,\,$ & $\,\,$2.1959$\,\,$ & $\,\,$\color{red} 2.3071\color{black} $\,\,$ & $\,\,$8.7329$\,\,$ \\
$\,\,$0.4554$\,\,$ & $\,\,$ 1 $\,\,$ & $\,\,$\color{red} 1.0506\color{black} $\,\,$ & $\,\,$3.9768  $\,\,$ \\
$\,\,$\color{red} 0.4334\color{black} $\,\,$ & $\,\,$\color{red} 0.9518\color{black} $\,\,$ & $\,\,$ 1 $\,\,$ & $\,\,$\color{red} 3.7851\color{black}  $\,\,$ \\
$\,\,$0.1145$\,\,$ & $\,\,$0.2515$\,\,$ & $\,\,$\color{red} 0.2642\color{black} $\,\,$ & $\,\,$ 1  $\,\,$ \\
\end{pmatrix},
\end{equation*}

\begin{equation*}
\mathbf{w}^{\prime} =
\begin{pmatrix}
0.493758\\
0.224851\\
0.224851\\
0.056540
\end{pmatrix} =
0.989162\cdot
\begin{pmatrix}
0.499168\\
0.227315\\
\color{gr} 0.227315\color{black} \\
0.057160
\end{pmatrix},
\end{equation*}
\begin{equation*}
\left[ \frac{{w}^{\prime}_i}{{w}^{\prime}_j} \right] =
\begin{pmatrix}
$\,\,$ 1 $\,\,$ & $\,\,$2.1959$\,\,$ & $\,\,$\color{gr} 2.1959\color{black} $\,\,$ & $\,\,$8.7329$\,\,$ \\
$\,\,$0.4554$\,\,$ & $\,\,$ 1 $\,\,$ & $\,\,$\color{gr} \color{blue} 1\color{black} $\,\,$ & $\,\,$3.9768  $\,\,$ \\
$\,\,$\color{gr} 0.4554\color{black} $\,\,$ & $\,\,$\color{gr} \color{blue} 1\color{black} $\,\,$ & $\,\,$ 1 $\,\,$ & $\,\,$\color{gr} 3.9768\color{black}  $\,\,$ \\
$\,\,$0.1145$\,\,$ & $\,\,$0.2515$\,\,$ & $\,\,$\color{gr} 0.2515\color{black} $\,\,$ & $\,\,$ 1  $\,\,$ \\
\end{pmatrix},
\end{equation*}
\end{example}
\newpage
\begin{example}
\begin{equation*}
\mathbf{A} =
\begin{pmatrix}
$\,\,$ 1 $\,\,$ & $\,\,$4$\,\,$ & $\,\,$2$\,\,$ & $\,\,$6 $\,\,$ \\
$\,\,$ 1/4$\,\,$ & $\,\,$ 1 $\,\,$ & $\,\,$1$\,\,$ & $\,\,$8 $\,\,$ \\
$\,\,$ 1/2$\,\,$ & $\,\,$ 1 $\,\,$ & $\,\,$ 1 $\,\,$ & $\,\,$5 $\,\,$ \\
$\,\,$ 1/6$\,\,$ & $\,\,$ 1/8$\,\,$ & $\,\,$ 1/5$\,\,$ & $\,\,$ 1  $\,\,$ \\
\end{pmatrix},
\qquad
\lambda_{\max} =
4.2460,
\qquad
CR = 0.0928
\end{equation*}

\begin{equation*}
\mathbf{w}^{EM} =
\begin{pmatrix}
0.504179\\
0.225989\\
\color{red} 0.221589\color{black} \\
0.048242
\end{pmatrix}\end{equation*}
\begin{equation*}
\left[ \frac{{w}^{EM}_i}{{w}^{EM}_j} \right] =
\begin{pmatrix}
$\,\,$ 1 $\,\,$ & $\,\,$2.2310$\,\,$ & $\,\,$\color{red} 2.2753\color{black} $\,\,$ & $\,\,$10.4510$\,\,$ \\
$\,\,$0.4482$\,\,$ & $\,\,$ 1 $\,\,$ & $\,\,$\color{red} 1.0199\color{black} $\,\,$ & $\,\,$4.6845  $\,\,$ \\
$\,\,$\color{red} 0.4395\color{black} $\,\,$ & $\,\,$\color{red} 0.9805\color{black} $\,\,$ & $\,\,$ 1 $\,\,$ & $\,\,$\color{red} 4.5933\color{black}  $\,\,$ \\
$\,\,$0.0957$\,\,$ & $\,\,$0.2135$\,\,$ & $\,\,$\color{red} 0.2177\color{black} $\,\,$ & $\,\,$ 1  $\,\,$ \\
\end{pmatrix},
\end{equation*}

\begin{equation*}
\mathbf{w}^{\prime} =
\begin{pmatrix}
0.501971\\
0.224999\\
0.224999\\
0.048031
\end{pmatrix} =
0.995619\cdot
\begin{pmatrix}
0.504179\\
0.225989\\
\color{gr} 0.225989\color{black} \\
0.048242
\end{pmatrix},
\end{equation*}
\begin{equation*}
\left[ \frac{{w}^{\prime}_i}{{w}^{\prime}_j} \right] =
\begin{pmatrix}
$\,\,$ 1 $\,\,$ & $\,\,$2.2310$\,\,$ & $\,\,$\color{gr} 2.2310\color{black} $\,\,$ & $\,\,$10.4510$\,\,$ \\
$\,\,$0.4482$\,\,$ & $\,\,$ 1 $\,\,$ & $\,\,$\color{gr} \color{blue} 1\color{black} $\,\,$ & $\,\,$4.6845  $\,\,$ \\
$\,\,$\color{gr} 0.4482\color{black} $\,\,$ & $\,\,$\color{gr} \color{blue} 1\color{black} $\,\,$ & $\,\,$ 1 $\,\,$ & $\,\,$\color{gr} 4.6845\color{black}  $\,\,$ \\
$\,\,$0.0957$\,\,$ & $\,\,$0.2135$\,\,$ & $\,\,$\color{gr} 0.2135\color{black} $\,\,$ & $\,\,$ 1  $\,\,$ \\
\end{pmatrix},
\end{equation*}
\end{example}
\newpage
\begin{example}
\begin{equation*}
\mathbf{A} =
\begin{pmatrix}
$\,\,$ 1 $\,\,$ & $\,\,$4$\,\,$ & $\,\,$2$\,\,$ & $\,\,$7 $\,\,$ \\
$\,\,$ 1/4$\,\,$ & $\,\,$ 1 $\,\,$ & $\,\,$1$\,\,$ & $\,\,$8 $\,\,$ \\
$\,\,$ 1/2$\,\,$ & $\,\,$ 1 $\,\,$ & $\,\,$ 1 $\,\,$ & $\,\,$5 $\,\,$ \\
$\,\,$ 1/7$\,\,$ & $\,\,$ 1/8$\,\,$ & $\,\,$ 1/5$\,\,$ & $\,\,$ 1  $\,\,$ \\
\end{pmatrix},
\qquad
\lambda_{\max} =
4.2035,
\qquad
CR = 0.0767
\end{equation*}

\begin{equation*}
\mathbf{w}^{EM} =
\begin{pmatrix}
0.513071\\
0.221728\\
\color{red} 0.219939\color{black} \\
0.045263
\end{pmatrix}\end{equation*}
\begin{equation*}
\left[ \frac{{w}^{EM}_i}{{w}^{EM}_j} \right] =
\begin{pmatrix}
$\,\,$ 1 $\,\,$ & $\,\,$2.3140$\,\,$ & $\,\,$\color{red} 2.3328\color{black} $\,\,$ & $\,\,$11.3354$\,\,$ \\
$\,\,$0.4322$\,\,$ & $\,\,$ 1 $\,\,$ & $\,\,$\color{red} 1.0081\color{black} $\,\,$ & $\,\,$4.8987  $\,\,$ \\
$\,\,$\color{red} 0.4287\color{black} $\,\,$ & $\,\,$\color{red} 0.9919\color{black} $\,\,$ & $\,\,$ 1 $\,\,$ & $\,\,$\color{red} 4.8592\color{black}  $\,\,$ \\
$\,\,$0.0882$\,\,$ & $\,\,$0.2041$\,\,$ & $\,\,$\color{red} 0.2058\color{black} $\,\,$ & $\,\,$ 1  $\,\,$ \\
\end{pmatrix},
\end{equation*}

\begin{equation*}
\mathbf{w}^{\prime} =
\begin{pmatrix}
0.512154\\
0.221332\\
0.221332\\
0.045182
\end{pmatrix} =
0.998214\cdot
\begin{pmatrix}
0.513071\\
0.221728\\
\color{gr} 0.221728\color{black} \\
0.045263
\end{pmatrix},
\end{equation*}
\begin{equation*}
\left[ \frac{{w}^{\prime}_i}{{w}^{\prime}_j} \right] =
\begin{pmatrix}
$\,\,$ 1 $\,\,$ & $\,\,$2.3140$\,\,$ & $\,\,$\color{gr} 2.3140\color{black} $\,\,$ & $\,\,$11.3354$\,\,$ \\
$\,\,$0.4322$\,\,$ & $\,\,$ 1 $\,\,$ & $\,\,$\color{gr} \color{blue} 1\color{black} $\,\,$ & $\,\,$4.8987  $\,\,$ \\
$\,\,$\color{gr} 0.4322\color{black} $\,\,$ & $\,\,$\color{gr} \color{blue} 1\color{black} $\,\,$ & $\,\,$ 1 $\,\,$ & $\,\,$\color{gr} 4.8987\color{black}  $\,\,$ \\
$\,\,$0.0882$\,\,$ & $\,\,$0.2041$\,\,$ & $\,\,$\color{gr} 0.2041\color{black} $\,\,$ & $\,\,$ 1  $\,\,$ \\
\end{pmatrix},
\end{equation*}
\end{example}
\newpage
\begin{example}
\begin{equation*}
\mathbf{A} =
\begin{pmatrix}
$\,\,$ 1 $\,\,$ & $\,\,$4$\,\,$ & $\,\,$2$\,\,$ & $\,\,$7 $\,\,$ \\
$\,\,$ 1/4$\,\,$ & $\,\,$ 1 $\,\,$ & $\,\,$1$\,\,$ & $\,\,$9 $\,\,$ \\
$\,\,$ 1/2$\,\,$ & $\,\,$ 1 $\,\,$ & $\,\,$ 1 $\,\,$ & $\,\,$5 $\,\,$ \\
$\,\,$ 1/7$\,\,$ & $\,\,$ 1/9$\,\,$ & $\,\,$ 1/5$\,\,$ & $\,\,$ 1  $\,\,$ \\
\end{pmatrix},
\qquad
\lambda_{\max} =
4.2371,
\qquad
CR = 0.0894
\end{equation*}

\begin{equation*}
\mathbf{w}^{EM} =
\begin{pmatrix}
0.510875\\
0.228266\\
\color{red} 0.217067\color{black} \\
0.043792
\end{pmatrix}\end{equation*}
\begin{equation*}
\left[ \frac{{w}^{EM}_i}{{w}^{EM}_j} \right] =
\begin{pmatrix}
$\,\,$ 1 $\,\,$ & $\,\,$2.2381$\,\,$ & $\,\,$\color{red} 2.3535\color{black} $\,\,$ & $\,\,$11.6659$\,\,$ \\
$\,\,$0.4468$\,\,$ & $\,\,$ 1 $\,\,$ & $\,\,$\color{red} 1.0516\color{black} $\,\,$ & $\,\,$5.2125  $\,\,$ \\
$\,\,$\color{red} 0.4249\color{black} $\,\,$ & $\,\,$\color{red} 0.9509\color{black} $\,\,$ & $\,\,$ 1 $\,\,$ & $\,\,$\color{red} 4.9568\color{black}  $\,\,$ \\
$\,\,$0.0857$\,\,$ & $\,\,$0.1918$\,\,$ & $\,\,$\color{red} 0.2017\color{black} $\,\,$ & $\,\,$ 1  $\,\,$ \\
\end{pmatrix},
\end{equation*}

\begin{equation*}
\mathbf{w}^{\prime} =
\begin{pmatrix}
0.509909\\
0.227835\\
0.218547\\
0.043709
\end{pmatrix} =
0.998110\cdot
\begin{pmatrix}
0.510875\\
0.228266\\
\color{gr} 0.218960\color{black} \\
0.043792
\end{pmatrix},
\end{equation*}
\begin{equation*}
\left[ \frac{{w}^{\prime}_i}{{w}^{\prime}_j} \right] =
\begin{pmatrix}
$\,\,$ 1 $\,\,$ & $\,\,$2.2381$\,\,$ & $\,\,$\color{gr} 2.3332\color{black} $\,\,$ & $\,\,$11.6659$\,\,$ \\
$\,\,$0.4468$\,\,$ & $\,\,$ 1 $\,\,$ & $\,\,$\color{gr} 1.0425\color{black} $\,\,$ & $\,\,$5.2125  $\,\,$ \\
$\,\,$\color{gr} 0.4286\color{black} $\,\,$ & $\,\,$\color{gr} 0.9592\color{black} $\,\,$ & $\,\,$ 1 $\,\,$ & $\,\,$\color{gr} \color{blue} 5\color{black}  $\,\,$ \\
$\,\,$0.0857$\,\,$ & $\,\,$0.1918$\,\,$ & $\,\,$\color{gr} \color{blue}  1/5\color{black} $\,\,$ & $\,\,$ 1  $\,\,$ \\
\end{pmatrix},
\end{equation*}
\end{example}
\newpage
\begin{example}
\begin{equation*}
\mathbf{A} =
\begin{pmatrix}
$\,\,$ 1 $\,\,$ & $\,\,$4$\,\,$ & $\,\,$2$\,\,$ & $\,\,$9 $\,\,$ \\
$\,\,$ 1/4$\,\,$ & $\,\,$ 1 $\,\,$ & $\,\,$2$\,\,$ & $\,\,$4 $\,\,$ \\
$\,\,$ 1/2$\,\,$ & $\,\,$ 1/2$\,\,$ & $\,\,$ 1 $\,\,$ & $\,\,$3 $\,\,$ \\
$\,\,$ 1/9$\,\,$ & $\,\,$ 1/4$\,\,$ & $\,\,$ 1/3$\,\,$ & $\,\,$ 1  $\,\,$ \\
\end{pmatrix},
\qquad
\lambda_{\max} =
4.1664,
\qquad
CR = 0.0627
\end{equation*}

\begin{equation*}
\mathbf{w}^{EM} =
\begin{pmatrix}
0.548078\\
0.222813\\
0.173971\\
\color{red} 0.055139\color{black}
\end{pmatrix}\end{equation*}
\begin{equation*}
\left[ \frac{{w}^{EM}_i}{{w}^{EM}_j} \right] =
\begin{pmatrix}
$\,\,$ 1 $\,\,$ & $\,\,$2.4598$\,\,$ & $\,\,$3.1504$\,\,$ & $\,\,$\color{red} 9.9400\color{black} $\,\,$ \\
$\,\,$0.4065$\,\,$ & $\,\,$ 1 $\,\,$ & $\,\,$1.2807$\,\,$ & $\,\,$\color{red} 4.0410\color{black}   $\,\,$ \\
$\,\,$0.3174$\,\,$ & $\,\,$0.7808$\,\,$ & $\,\,$ 1 $\,\,$ & $\,\,$\color{red} 3.1552\color{black}  $\,\,$ \\
$\,\,$\color{red} 0.1006\color{black} $\,\,$ & $\,\,$\color{red} 0.2475\color{black} $\,\,$ & $\,\,$\color{red} 0.3169\color{black} $\,\,$ & $\,\,$ 1  $\,\,$ \\
\end{pmatrix},
\end{equation*}

\begin{equation*}
\mathbf{w}^{\prime} =
\begin{pmatrix}
0.547769\\
0.222687\\
0.173872\\
0.055672
\end{pmatrix} =
0.999436\cdot
\begin{pmatrix}
0.548078\\
0.222813\\
0.173971\\
\color{gr} 0.055703\color{black}
\end{pmatrix},
\end{equation*}
\begin{equation*}
\left[ \frac{{w}^{\prime}_i}{{w}^{\prime}_j} \right] =
\begin{pmatrix}
$\,\,$ 1 $\,\,$ & $\,\,$2.4598$\,\,$ & $\,\,$3.1504$\,\,$ & $\,\,$\color{gr} 9.8393\color{black} $\,\,$ \\
$\,\,$0.4065$\,\,$ & $\,\,$ 1 $\,\,$ & $\,\,$1.2807$\,\,$ & $\,\,$\color{gr} \color{blue} 4\color{black}   $\,\,$ \\
$\,\,$0.3174$\,\,$ & $\,\,$0.7808$\,\,$ & $\,\,$ 1 $\,\,$ & $\,\,$\color{gr} 3.1232\color{black}  $\,\,$ \\
$\,\,$\color{gr} 0.1016\color{black} $\,\,$ & $\,\,$\color{gr} \color{blue}  1/4\color{black} $\,\,$ & $\,\,$\color{gr} 0.3202\color{black} $\,\,$ & $\,\,$ 1  $\,\,$ \\
\end{pmatrix},
\end{equation*}
\end{example}
\newpage
\begin{example}
\begin{equation*}
\mathbf{A} =
\begin{pmatrix}
$\,\,$ 1 $\,\,$ & $\,\,$4$\,\,$ & $\,\,$3$\,\,$ & $\,\,$4 $\,\,$ \\
$\,\,$ 1/4$\,\,$ & $\,\,$ 1 $\,\,$ & $\,\,$1$\,\,$ & $\,\,$3 $\,\,$ \\
$\,\,$ 1/3$\,\,$ & $\,\,$ 1 $\,\,$ & $\,\,$ 1 $\,\,$ & $\,\,$2 $\,\,$ \\
$\,\,$ 1/4$\,\,$ & $\,\,$ 1/3$\,\,$ & $\,\,$ 1/2$\,\,$ & $\,\,$ 1  $\,\,$ \\
\end{pmatrix},
\qquad
\lambda_{\max} =
4.1031,
\qquad
CR = 0.0389
\end{equation*}

\begin{equation*}
\mathbf{w}^{EM} =
\begin{pmatrix}
0.537947\\
0.190537\\
\color{red} 0.178887\color{black} \\
0.092629
\end{pmatrix}\end{equation*}
\begin{equation*}
\left[ \frac{{w}^{EM}_i}{{w}^{EM}_j} \right] =
\begin{pmatrix}
$\,\,$ 1 $\,\,$ & $\,\,$2.8233$\,\,$ & $\,\,$\color{red} 3.0072\color{black} $\,\,$ & $\,\,$5.8075$\,\,$ \\
$\,\,$0.3542$\,\,$ & $\,\,$ 1 $\,\,$ & $\,\,$\color{red} 1.0651\color{black} $\,\,$ & $\,\,$2.0570  $\,\,$ \\
$\,\,$\color{red} 0.3325\color{black} $\,\,$ & $\,\,$\color{red} 0.9389\color{black} $\,\,$ & $\,\,$ 1 $\,\,$ & $\,\,$\color{red} 1.9312\color{black}  $\,\,$ \\
$\,\,$0.1722$\,\,$ & $\,\,$0.4862$\,\,$ & $\,\,$\color{red} 0.5178\color{black} $\,\,$ & $\,\,$ 1  $\,\,$ \\
\end{pmatrix},
\end{equation*}

\begin{equation*}
\mathbf{w}^{\prime} =
\begin{pmatrix}
0.537716\\
0.190455\\
0.179239\\
0.092590
\end{pmatrix} =
0.999571\cdot
\begin{pmatrix}
0.537947\\
0.190537\\
\color{gr} 0.179316\color{black} \\
0.092629
\end{pmatrix},
\end{equation*}
\begin{equation*}
\left[ \frac{{w}^{\prime}_i}{{w}^{\prime}_j} \right] =
\begin{pmatrix}
$\,\,$ 1 $\,\,$ & $\,\,$2.8233$\,\,$ & $\,\,$\color{gr} \color{blue} 3\color{black} $\,\,$ & $\,\,$5.8075$\,\,$ \\
$\,\,$0.3542$\,\,$ & $\,\,$ 1 $\,\,$ & $\,\,$\color{gr} 1.0626\color{black} $\,\,$ & $\,\,$2.0570  $\,\,$ \\
$\,\,$\color{gr} \color{blue}  1/3\color{black} $\,\,$ & $\,\,$\color{gr} 0.9411\color{black} $\,\,$ & $\,\,$ 1 $\,\,$ & $\,\,$\color{gr} 1.9358\color{black}  $\,\,$ \\
$\,\,$0.1722$\,\,$ & $\,\,$0.4862$\,\,$ & $\,\,$\color{gr} 0.5166\color{black} $\,\,$ & $\,\,$ 1  $\,\,$ \\
\end{pmatrix},
\end{equation*}
\end{example}
\newpage
\begin{example}
\begin{equation*}
\mathbf{A} =
\begin{pmatrix}
$\,\,$ 1 $\,\,$ & $\,\,$4$\,\,$ & $\,\,$3$\,\,$ & $\,\,$5 $\,\,$ \\
$\,\,$ 1/4$\,\,$ & $\,\,$ 1 $\,\,$ & $\,\,$3$\,\,$ & $\,\,$2 $\,\,$ \\
$\,\,$ 1/3$\,\,$ & $\,\,$ 1/3$\,\,$ & $\,\,$ 1 $\,\,$ & $\,\,$1 $\,\,$ \\
$\,\,$ 1/5$\,\,$ & $\,\,$ 1/2$\,\,$ & $\,\,$ 1 $\,\,$ & $\,\,$ 1  $\,\,$ \\
\end{pmatrix},
\qquad
\lambda_{\max} =
4.1655,
\qquad
CR = 0.0624
\end{equation*}

\begin{equation*}
\mathbf{w}^{EM} =
\begin{pmatrix}
0.556973\\
0.220863\\
0.115578\\
\color{red} 0.106587\color{black}
\end{pmatrix}\end{equation*}
\begin{equation*}
\left[ \frac{{w}^{EM}_i}{{w}^{EM}_j} \right] =
\begin{pmatrix}
$\,\,$ 1 $\,\,$ & $\,\,$2.5218$\,\,$ & $\,\,$4.8190$\,\,$ & $\,\,$\color{red} 5.2255\color{black} $\,\,$ \\
$\,\,$0.3965$\,\,$ & $\,\,$ 1 $\,\,$ & $\,\,$1.9109$\,\,$ & $\,\,$\color{red} 2.0721\color{black}   $\,\,$ \\
$\,\,$0.2075$\,\,$ & $\,\,$0.5233$\,\,$ & $\,\,$ 1 $\,\,$ & $\,\,$\color{red} 1.0844\color{black}  $\,\,$ \\
$\,\,$\color{red} 0.1914\color{black} $\,\,$ & $\,\,$\color{red} 0.4826\color{black} $\,\,$ & $\,\,$\color{red} 0.9222\color{black} $\,\,$ & $\,\,$ 1  $\,\,$ \\
\end{pmatrix},
\end{equation*}

\begin{equation*}
\mathbf{w}^{\prime} =
\begin{pmatrix}
0.554840\\
0.220017\\
0.115135\\
0.110008
\end{pmatrix} =
0.996170\cdot
\begin{pmatrix}
0.556973\\
0.220863\\
0.115578\\
\color{gr} 0.110431\color{black}
\end{pmatrix},
\end{equation*}
\begin{equation*}
\left[ \frac{{w}^{\prime}_i}{{w}^{\prime}_j} \right] =
\begin{pmatrix}
$\,\,$ 1 $\,\,$ & $\,\,$2.5218$\,\,$ & $\,\,$4.8190$\,\,$ & $\,\,$\color{gr} 5.0436\color{black} $\,\,$ \\
$\,\,$0.3965$\,\,$ & $\,\,$ 1 $\,\,$ & $\,\,$1.9109$\,\,$ & $\,\,$\color{gr} \color{blue} 2\color{black}   $\,\,$ \\
$\,\,$0.2075$\,\,$ & $\,\,$0.5233$\,\,$ & $\,\,$ 1 $\,\,$ & $\,\,$\color{gr} 1.0466\color{black}  $\,\,$ \\
$\,\,$\color{gr} 0.1983\color{black} $\,\,$ & $\,\,$\color{gr} \color{blue}  1/2\color{black} $\,\,$ & $\,\,$\color{gr} 0.9555\color{black} $\,\,$ & $\,\,$ 1  $\,\,$ \\
\end{pmatrix},
\end{equation*}
\end{example}
\newpage
\begin{example}
\begin{equation*}
\mathbf{A} =
\begin{pmatrix}
$\,\,$ 1 $\,\,$ & $\,\,$4$\,\,$ & $\,\,$3$\,\,$ & $\,\,$5 $\,\,$ \\
$\,\,$ 1/4$\,\,$ & $\,\,$ 1 $\,\,$ & $\,\,$4$\,\,$ & $\,\,$2 $\,\,$ \\
$\,\,$ 1/3$\,\,$ & $\,\,$ 1/4$\,\,$ & $\,\,$ 1 $\,\,$ & $\,\,$1 $\,\,$ \\
$\,\,$ 1/5$\,\,$ & $\,\,$ 1/2$\,\,$ & $\,\,$ 1 $\,\,$ & $\,\,$ 1  $\,\,$ \\
\end{pmatrix},
\qquad
\lambda_{\max} =
4.2460,
\qquad
CR = 0.0928
\end{equation*}

\begin{equation*}
\mathbf{w}^{EM} =
\begin{pmatrix}
0.551569\\
0.237990\\
0.106875\\
\color{red} 0.103567\color{black}
\end{pmatrix}\end{equation*}
\begin{equation*}
\left[ \frac{{w}^{EM}_i}{{w}^{EM}_j} \right] =
\begin{pmatrix}
$\,\,$ 1 $\,\,$ & $\,\,$2.3176$\,\,$ & $\,\,$5.1609$\,\,$ & $\,\,$\color{red} 5.3257\color{black} $\,\,$ \\
$\,\,$0.4315$\,\,$ & $\,\,$ 1 $\,\,$ & $\,\,$2.2268$\,\,$ & $\,\,$\color{red} 2.2979\color{black}   $\,\,$ \\
$\,\,$0.1938$\,\,$ & $\,\,$0.4491$\,\,$ & $\,\,$ 1 $\,\,$ & $\,\,$\color{red} 1.0319\color{black}  $\,\,$ \\
$\,\,$\color{red} 0.1878\color{black} $\,\,$ & $\,\,$\color{red} 0.4352\color{black} $\,\,$ & $\,\,$\color{red} 0.9690\color{black} $\,\,$ & $\,\,$ 1  $\,\,$ \\
\end{pmatrix},
\end{equation*}

\begin{equation*}
\mathbf{w}^{\prime} =
\begin{pmatrix}
0.549750\\
0.237205\\
0.106522\\
0.106522
\end{pmatrix} =
0.996703\cdot
\begin{pmatrix}
0.551569\\
0.237990\\
0.106875\\
\color{gr} 0.106875\color{black}
\end{pmatrix},
\end{equation*}
\begin{equation*}
\left[ \frac{{w}^{\prime}_i}{{w}^{\prime}_j} \right] =
\begin{pmatrix}
$\,\,$ 1 $\,\,$ & $\,\,$2.3176$\,\,$ & $\,\,$5.1609$\,\,$ & $\,\,$\color{gr} 5.1609\color{black} $\,\,$ \\
$\,\,$0.4315$\,\,$ & $\,\,$ 1 $\,\,$ & $\,\,$2.2268$\,\,$ & $\,\,$\color{gr} 2.2268\color{black}   $\,\,$ \\
$\,\,$0.1938$\,\,$ & $\,\,$0.4491$\,\,$ & $\,\,$ 1 $\,\,$ & $\,\,$\color{gr} \color{blue} 1\color{black}  $\,\,$ \\
$\,\,$\color{gr} 0.1938\color{black} $\,\,$ & $\,\,$\color{gr} 0.4491\color{black} $\,\,$ & $\,\,$\color{gr} \color{blue} 1\color{black} $\,\,$ & $\,\,$ 1  $\,\,$ \\
\end{pmatrix},
\end{equation*}
\end{example}
\newpage
\begin{example}
\begin{equation*}
\mathbf{A} =
\begin{pmatrix}
$\,\,$ 1 $\,\,$ & $\,\,$4$\,\,$ & $\,\,$3$\,\,$ & $\,\,$6 $\,\,$ \\
$\,\,$ 1/4$\,\,$ & $\,\,$ 1 $\,\,$ & $\,\,$1$\,\,$ & $\,\,$5 $\,\,$ \\
$\,\,$ 1/3$\,\,$ & $\,\,$ 1 $\,\,$ & $\,\,$ 1 $\,\,$ & $\,\,$3 $\,\,$ \\
$\,\,$ 1/6$\,\,$ & $\,\,$ 1/5$\,\,$ & $\,\,$ 1/3$\,\,$ & $\,\,$ 1  $\,\,$ \\
\end{pmatrix},
\qquad
\lambda_{\max} =
4.1252,
\qquad
CR = 0.0472
\end{equation*}

\begin{equation*}
\mathbf{w}^{EM} =
\begin{pmatrix}
0.553080\\
0.201918\\
\color{red} 0.183060\color{black} \\
0.061942
\end{pmatrix}\end{equation*}
\begin{equation*}
\left[ \frac{{w}^{EM}_i}{{w}^{EM}_j} \right] =
\begin{pmatrix}
$\,\,$ 1 $\,\,$ & $\,\,$2.7391$\,\,$ & $\,\,$\color{red} 3.0213\color{black} $\,\,$ & $\,\,$8.9290$\,\,$ \\
$\,\,$0.3651$\,\,$ & $\,\,$ 1 $\,\,$ & $\,\,$\color{red} 1.1030\color{black} $\,\,$ & $\,\,$3.2598  $\,\,$ \\
$\,\,$\color{red} 0.3310\color{black} $\,\,$ & $\,\,$\color{red} 0.9066\color{black} $\,\,$ & $\,\,$ 1 $\,\,$ & $\,\,$\color{red} 2.9553\color{black}  $\,\,$ \\
$\,\,$0.1120$\,\,$ & $\,\,$0.3068$\,\,$ & $\,\,$\color{red} 0.3384\color{black} $\,\,$ & $\,\,$ 1  $\,\,$ \\
\end{pmatrix},
\end{equation*}

\begin{equation*}
\mathbf{w}^{\prime} =
\begin{pmatrix}
0.552362\\
0.201656\\
0.184121\\
0.061862
\end{pmatrix} =
0.998701\cdot
\begin{pmatrix}
0.553080\\
0.201918\\
\color{gr} 0.184360\color{black} \\
0.061942
\end{pmatrix},
\end{equation*}
\begin{equation*}
\left[ \frac{{w}^{\prime}_i}{{w}^{\prime}_j} \right] =
\begin{pmatrix}
$\,\,$ 1 $\,\,$ & $\,\,$2.7391$\,\,$ & $\,\,$\color{gr} \color{blue} 3\color{black} $\,\,$ & $\,\,$8.9290$\,\,$ \\
$\,\,$0.3651$\,\,$ & $\,\,$ 1 $\,\,$ & $\,\,$\color{gr} 1.0952\color{black} $\,\,$ & $\,\,$3.2598  $\,\,$ \\
$\,\,$\color{gr} \color{blue}  1/3\color{black} $\,\,$ & $\,\,$\color{gr} 0.9130\color{black} $\,\,$ & $\,\,$ 1 $\,\,$ & $\,\,$\color{gr} 2.9763\color{black}  $\,\,$ \\
$\,\,$0.1120$\,\,$ & $\,\,$0.3068$\,\,$ & $\,\,$\color{gr} 0.3360\color{black} $\,\,$ & $\,\,$ 1  $\,\,$ \\
\end{pmatrix},
\end{equation*}
\end{example}
\newpage
\begin{example}
\begin{equation*}
\mathbf{A} =
\begin{pmatrix}
$\,\,$ 1 $\,\,$ & $\,\,$4$\,\,$ & $\,\,$3$\,\,$ & $\,\,$7 $\,\,$ \\
$\,\,$ 1/4$\,\,$ & $\,\,$ 1 $\,\,$ & $\,\,$1$\,\,$ & $\,\,$4 $\,\,$ \\
$\,\,$ 1/3$\,\,$ & $\,\,$ 1 $\,\,$ & $\,\,$ 1 $\,\,$ & $\,\,$3 $\,\,$ \\
$\,\,$ 1/7$\,\,$ & $\,\,$ 1/4$\,\,$ & $\,\,$ 1/3$\,\,$ & $\,\,$ 1  $\,\,$ \\
\end{pmatrix},
\qquad
\lambda_{\max} =
4.0576,
\qquad
CR = 0.0217
\end{equation*}

\begin{equation*}
\mathbf{w}^{EM} =
\begin{pmatrix}
0.567012\\
0.187370\\
\color{red} 0.183773\color{black} \\
0.061846
\end{pmatrix}\end{equation*}
\begin{equation*}
\left[ \frac{{w}^{EM}_i}{{w}^{EM}_j} \right] =
\begin{pmatrix}
$\,\,$ 1 $\,\,$ & $\,\,$3.0262$\,\,$ & $\,\,$\color{red} 3.0854\color{black} $\,\,$ & $\,\,$9.1682$\,\,$ \\
$\,\,$0.3305$\,\,$ & $\,\,$ 1 $\,\,$ & $\,\,$\color{red} 1.0196\color{black} $\,\,$ & $\,\,$3.0296  $\,\,$ \\
$\,\,$\color{red} 0.3241\color{black} $\,\,$ & $\,\,$\color{red} 0.9808\color{black} $\,\,$ & $\,\,$ 1 $\,\,$ & $\,\,$\color{red} 2.9715\color{black}  $\,\,$ \\
$\,\,$0.1091$\,\,$ & $\,\,$0.3301$\,\,$ & $\,\,$\color{red} 0.3365\color{black} $\,\,$ & $\,\,$ 1  $\,\,$ \\
\end{pmatrix},
\end{equation*}

\begin{equation*}
\mathbf{w}^{\prime} =
\begin{pmatrix}
0.566013\\
0.187040\\
0.185210\\
0.061737
\end{pmatrix} =
0.998239\cdot
\begin{pmatrix}
0.567012\\
0.187370\\
\color{gr} 0.185537\color{black} \\
0.061846
\end{pmatrix},
\end{equation*}
\begin{equation*}
\left[ \frac{{w}^{\prime}_i}{{w}^{\prime}_j} \right] =
\begin{pmatrix}
$\,\,$ 1 $\,\,$ & $\,\,$3.0262$\,\,$ & $\,\,$\color{gr} 3.0561\color{black} $\,\,$ & $\,\,$9.1682$\,\,$ \\
$\,\,$0.3305$\,\,$ & $\,\,$ 1 $\,\,$ & $\,\,$\color{gr} 1.0099\color{black} $\,\,$ & $\,\,$3.0296  $\,\,$ \\
$\,\,$\color{gr} 0.3272\color{black} $\,\,$ & $\,\,$\color{gr} 0.9902\color{black} $\,\,$ & $\,\,$ 1 $\,\,$ & $\,\,$\color{gr} \color{blue} 3\color{black}  $\,\,$ \\
$\,\,$0.1091$\,\,$ & $\,\,$0.3301$\,\,$ & $\,\,$\color{gr} \color{blue}  1/3\color{black} $\,\,$ & $\,\,$ 1  $\,\,$ \\
\end{pmatrix},
\end{equation*}
\end{example}
\newpage
\begin{example}
\begin{equation*}
\mathbf{A} =
\begin{pmatrix}
$\,\,$ 1 $\,\,$ & $\,\,$4$\,\,$ & $\,\,$3$\,\,$ & $\,\,$8 $\,\,$ \\
$\,\,$ 1/4$\,\,$ & $\,\,$ 1 $\,\,$ & $\,\,$1$\,\,$ & $\,\,$6 $\,\,$ \\
$\,\,$ 1/3$\,\,$ & $\,\,$ 1 $\,\,$ & $\,\,$ 1 $\,\,$ & $\,\,$4 $\,\,$ \\
$\,\,$ 1/8$\,\,$ & $\,\,$ 1/6$\,\,$ & $\,\,$ 1/4$\,\,$ & $\,\,$ 1  $\,\,$ \\
\end{pmatrix},
\qquad
\lambda_{\max} =
4.1031,
\qquad
CR = 0.0389
\end{equation*}

\begin{equation*}
\mathbf{w}^{EM} =
\begin{pmatrix}
0.564072\\
0.199790\\
\color{red} 0.187574\color{black} \\
0.048564
\end{pmatrix}\end{equation*}
\begin{equation*}
\left[ \frac{{w}^{EM}_i}{{w}^{EM}_j} \right] =
\begin{pmatrix}
$\,\,$ 1 $\,\,$ & $\,\,$2.8233$\,\,$ & $\,\,$\color{red} 3.0072\color{black} $\,\,$ & $\,\,$11.6150$\,\,$ \\
$\,\,$0.3542$\,\,$ & $\,\,$ 1 $\,\,$ & $\,\,$\color{red} 1.0651\color{black} $\,\,$ & $\,\,$4.1140  $\,\,$ \\
$\,\,$\color{red} 0.3325\color{black} $\,\,$ & $\,\,$\color{red} 0.9389\color{black} $\,\,$ & $\,\,$ 1 $\,\,$ & $\,\,$\color{red} 3.8624\color{black}  $\,\,$ \\
$\,\,$0.0861$\,\,$ & $\,\,$0.2431$\,\,$ & $\,\,$\color{red} 0.2589\color{black} $\,\,$ & $\,\,$ 1  $\,\,$ \\
\end{pmatrix},
\end{equation*}

\begin{equation*}
\mathbf{w}^{\prime} =
\begin{pmatrix}
0.563818\\
0.199700\\
0.187939\\
0.048542
\end{pmatrix} =
0.999551\cdot
\begin{pmatrix}
0.564072\\
0.199790\\
\color{gr} 0.188024\color{black} \\
0.048564
\end{pmatrix},
\end{equation*}
\begin{equation*}
\left[ \frac{{w}^{\prime}_i}{{w}^{\prime}_j} \right] =
\begin{pmatrix}
$\,\,$ 1 $\,\,$ & $\,\,$2.8233$\,\,$ & $\,\,$\color{gr} \color{blue} 3\color{black} $\,\,$ & $\,\,$11.6150$\,\,$ \\
$\,\,$0.3542$\,\,$ & $\,\,$ 1 $\,\,$ & $\,\,$\color{gr} 1.0626\color{black} $\,\,$ & $\,\,$4.1140  $\,\,$ \\
$\,\,$\color{gr} \color{blue}  1/3\color{black} $\,\,$ & $\,\,$\color{gr} 0.9411\color{black} $\,\,$ & $\,\,$ 1 $\,\,$ & $\,\,$\color{gr} 3.8717\color{black}  $\,\,$ \\
$\,\,$0.0861$\,\,$ & $\,\,$0.2431$\,\,$ & $\,\,$\color{gr} 0.2583\color{black} $\,\,$ & $\,\,$ 1  $\,\,$ \\
\end{pmatrix},
\end{equation*}
\end{example}
\newpage
\begin{example}
\begin{equation*}
\mathbf{A} =
\begin{pmatrix}
$\,\,$ 1 $\,\,$ & $\,\,$4$\,\,$ & $\,\,$3$\,\,$ & $\,\,$8 $\,\,$ \\
$\,\,$ 1/4$\,\,$ & $\,\,$ 1 $\,\,$ & $\,\,$1$\,\,$ & $\,\,$7 $\,\,$ \\
$\,\,$ 1/3$\,\,$ & $\,\,$ 1 $\,\,$ & $\,\,$ 1 $\,\,$ & $\,\,$4 $\,\,$ \\
$\,\,$ 1/8$\,\,$ & $\,\,$ 1/7$\,\,$ & $\,\,$ 1/4$\,\,$ & $\,\,$ 1  $\,\,$ \\
\end{pmatrix},
\qquad
\lambda_{\max} =
4.1365,
\qquad
CR = 0.0515
\end{equation*}

\begin{equation*}
\mathbf{w}^{EM} =
\begin{pmatrix}
0.560650\\
0.207632\\
\color{red} 0.185159\color{black} \\
0.046559
\end{pmatrix}\end{equation*}
\begin{equation*}
\left[ \frac{{w}^{EM}_i}{{w}^{EM}_j} \right] =
\begin{pmatrix}
$\,\,$ 1 $\,\,$ & $\,\,$2.7002$\,\,$ & $\,\,$\color{red} 3.0279\color{black} $\,\,$ & $\,\,$12.0417$\,\,$ \\
$\,\,$0.3703$\,\,$ & $\,\,$ 1 $\,\,$ & $\,\,$\color{red} 1.1214\color{black} $\,\,$ & $\,\,$4.4595  $\,\,$ \\
$\,\,$\color{red} 0.3303\color{black} $\,\,$ & $\,\,$\color{red} 0.8918\color{black} $\,\,$ & $\,\,$ 1 $\,\,$ & $\,\,$\color{red} 3.9769\color{black}  $\,\,$ \\
$\,\,$0.0830$\,\,$ & $\,\,$0.2242$\,\,$ & $\,\,$\color{red} 0.2515\color{black} $\,\,$ & $\,\,$ 1  $\,\,$ \\
\end{pmatrix},
\end{equation*}

\begin{equation*}
\mathbf{w}^{\prime} =
\begin{pmatrix}
0.560047\\
0.207408\\
0.186036\\
0.046509
\end{pmatrix} =
0.998924\cdot
\begin{pmatrix}
0.560650\\
0.207632\\
\color{gr} 0.186237\color{black} \\
0.046559
\end{pmatrix},
\end{equation*}
\begin{equation*}
\left[ \frac{{w}^{\prime}_i}{{w}^{\prime}_j} \right] =
\begin{pmatrix}
$\,\,$ 1 $\,\,$ & $\,\,$2.7002$\,\,$ & $\,\,$\color{gr} 3.0104\color{black} $\,\,$ & $\,\,$12.0417$\,\,$ \\
$\,\,$0.3703$\,\,$ & $\,\,$ 1 $\,\,$ & $\,\,$\color{gr} 1.1149\color{black} $\,\,$ & $\,\,$4.4595  $\,\,$ \\
$\,\,$\color{gr} 0.3322\color{black} $\,\,$ & $\,\,$\color{gr} 0.8970\color{black} $\,\,$ & $\,\,$ 1 $\,\,$ & $\,\,$\color{gr} \color{blue} 4\color{black}  $\,\,$ \\
$\,\,$0.0830$\,\,$ & $\,\,$0.2242$\,\,$ & $\,\,$\color{gr} \color{blue}  1/4\color{black} $\,\,$ & $\,\,$ 1  $\,\,$ \\
\end{pmatrix},
\end{equation*}
\end{example}
\newpage
\begin{example}
\begin{equation*}
\mathbf{A} =
\begin{pmatrix}
$\,\,$ 1 $\,\,$ & $\,\,$4$\,\,$ & $\,\,$3$\,\,$ & $\,\,$9 $\,\,$ \\
$\,\,$ 1/4$\,\,$ & $\,\,$ 1 $\,\,$ & $\,\,$1$\,\,$ & $\,\,$5 $\,\,$ \\
$\,\,$ 1/3$\,\,$ & $\,\,$ 1 $\,\,$ & $\,\,$ 1 $\,\,$ & $\,\,$4 $\,\,$ \\
$\,\,$ 1/9$\,\,$ & $\,\,$ 1/5$\,\,$ & $\,\,$ 1/4$\,\,$ & $\,\,$ 1  $\,\,$ \\
\end{pmatrix},
\qquad
\lambda_{\max} =
4.0539,
\qquad
CR = 0.0203
\end{equation*}

\begin{equation*}
\mathbf{w}^{EM} =
\begin{pmatrix}
0.574879\\
0.188326\\
\color{red} 0.188142\color{black} \\
0.048652
\end{pmatrix}\end{equation*}
\begin{equation*}
\left[ \frac{{w}^{EM}_i}{{w}^{EM}_j} \right] =
\begin{pmatrix}
$\,\,$ 1 $\,\,$ & $\,\,$3.0526$\,\,$ & $\,\,$\color{red} 3.0556\color{black} $\,\,$ & $\,\,$11.8162$\,\,$ \\
$\,\,$0.3276$\,\,$ & $\,\,$ 1 $\,\,$ & $\,\,$\color{red} 1.0010\color{black} $\,\,$ & $\,\,$3.8709  $\,\,$ \\
$\,\,$\color{red} 0.3273\color{black} $\,\,$ & $\,\,$\color{red} 0.9990\color{black} $\,\,$ & $\,\,$ 1 $\,\,$ & $\,\,$\color{red} 3.8671\color{black}  $\,\,$ \\
$\,\,$0.0846$\,\,$ & $\,\,$0.2583$\,\,$ & $\,\,$\color{red} 0.2586\color{black} $\,\,$ & $\,\,$ 1  $\,\,$ \\
\end{pmatrix},
\end{equation*}

\begin{equation*}
\mathbf{w}^{\prime} =
\begin{pmatrix}
0.574774\\
0.188292\\
0.188292\\
0.048643
\end{pmatrix} =
0.999816\cdot
\begin{pmatrix}
0.574879\\
0.188326\\
\color{gr} 0.188326\color{black} \\
0.048652
\end{pmatrix},
\end{equation*}
\begin{equation*}
\left[ \frac{{w}^{\prime}_i}{{w}^{\prime}_j} \right] =
\begin{pmatrix}
$\,\,$ 1 $\,\,$ & $\,\,$3.0526$\,\,$ & $\,\,$\color{gr} 3.0526\color{black} $\,\,$ & $\,\,$11.8162$\,\,$ \\
$\,\,$0.3276$\,\,$ & $\,\,$ 1 $\,\,$ & $\,\,$\color{gr} \color{blue} 1\color{black} $\,\,$ & $\,\,$3.8709  $\,\,$ \\
$\,\,$\color{gr} 0.3276\color{black} $\,\,$ & $\,\,$\color{gr} \color{blue} 1\color{black} $\,\,$ & $\,\,$ 1 $\,\,$ & $\,\,$\color{gr} 3.8709\color{black}  $\,\,$ \\
$\,\,$0.0846$\,\,$ & $\,\,$0.2583$\,\,$ & $\,\,$\color{gr} 0.2583\color{black} $\,\,$ & $\,\,$ 1  $\,\,$ \\
\end{pmatrix},
\end{equation*}
\end{example}
\newpage
\begin{example}
\begin{equation*}
\mathbf{A} =
\begin{pmatrix}
$\,\,$ 1 $\,\,$ & $\,\,$4$\,\,$ & $\,\,$3$\,\,$ & $\,\,$9 $\,\,$ \\
$\,\,$ 1/4$\,\,$ & $\,\,$ 1 $\,\,$ & $\,\,$4$\,\,$ & $\,\,$4 $\,\,$ \\
$\,\,$ 1/3$\,\,$ & $\,\,$ 1/4$\,\,$ & $\,\,$ 1 $\,\,$ & $\,\,$2 $\,\,$ \\
$\,\,$ 1/9$\,\,$ & $\,\,$ 1/4$\,\,$ & $\,\,$ 1/2$\,\,$ & $\,\,$ 1  $\,\,$ \\
\end{pmatrix},
\qquad
\lambda_{\max} =
4.2469,
\qquad
CR = 0.0931
\end{equation*}

\begin{equation*}
\mathbf{w}^{EM} =
\begin{pmatrix}
0.575443\\
0.254190\\
0.113609\\
\color{red} 0.056759\color{black}
\end{pmatrix}\end{equation*}
\begin{equation*}
\left[ \frac{{w}^{EM}_i}{{w}^{EM}_j} \right] =
\begin{pmatrix}
$\,\,$ 1 $\,\,$ & $\,\,$2.2638$\,\,$ & $\,\,$5.0651$\,\,$ & $\,\,$\color{red} 10.1384\color{black} $\,\,$ \\
$\,\,$0.4417$\,\,$ & $\,\,$ 1 $\,\,$ & $\,\,$2.2374$\,\,$ & $\,\,$\color{red} 4.4784\color{black}   $\,\,$ \\
$\,\,$0.1974$\,\,$ & $\,\,$0.4469$\,\,$ & $\,\,$ 1 $\,\,$ & $\,\,$\color{red} 2.0016\color{black}  $\,\,$ \\
$\,\,$\color{red} 0.0986\color{black} $\,\,$ & $\,\,$\color{red} 0.2233\color{black} $\,\,$ & $\,\,$\color{red} 0.4996\color{black} $\,\,$ & $\,\,$ 1  $\,\,$ \\
\end{pmatrix},
\end{equation*}

\begin{equation*}
\mathbf{w}^{\prime} =
\begin{pmatrix}
0.575416\\
0.254178\\
0.113604\\
0.056802
\end{pmatrix} =
0.999954\cdot
\begin{pmatrix}
0.575443\\
0.254190\\
0.113609\\
\color{gr} 0.056805\color{black}
\end{pmatrix},
\end{equation*}
\begin{equation*}
\left[ \frac{{w}^{\prime}_i}{{w}^{\prime}_j} \right] =
\begin{pmatrix}
$\,\,$ 1 $\,\,$ & $\,\,$2.2638$\,\,$ & $\,\,$5.0651$\,\,$ & $\,\,$\color{gr} 10.1302\color{black} $\,\,$ \\
$\,\,$0.4417$\,\,$ & $\,\,$ 1 $\,\,$ & $\,\,$2.2374$\,\,$ & $\,\,$\color{gr} 4.4748\color{black}   $\,\,$ \\
$\,\,$0.1974$\,\,$ & $\,\,$0.4469$\,\,$ & $\,\,$ 1 $\,\,$ & $\,\,$\color{gr} \color{blue} 2\color{black}  $\,\,$ \\
$\,\,$\color{gr} 0.0987\color{black} $\,\,$ & $\,\,$\color{gr} 0.2235\color{black} $\,\,$ & $\,\,$\color{gr} \color{blue}  1/2\color{black} $\,\,$ & $\,\,$ 1  $\,\,$ \\
\end{pmatrix},
\end{equation*}
\end{example}
\newpage
\begin{example}
\begin{equation*}
\mathbf{A} =
\begin{pmatrix}
$\,\,$ 1 $\,\,$ & $\,\,$4$\,\,$ & $\,\,$4$\,\,$ & $\,\,$5 $\,\,$ \\
$\,\,$ 1/4$\,\,$ & $\,\,$ 1 $\,\,$ & $\,\,$2$\,\,$ & $\,\,$7 $\,\,$ \\
$\,\,$ 1/4$\,\,$ & $\,\,$ 1/2$\,\,$ & $\,\,$ 1 $\,\,$ & $\,\,$2 $\,\,$ \\
$\,\,$ 1/5$\,\,$ & $\,\,$ 1/7$\,\,$ & $\,\,$ 1/2$\,\,$ & $\,\,$ 1  $\,\,$ \\
\end{pmatrix},
\qquad
\lambda_{\max} =
4.2610,
\qquad
CR = 0.0984
\end{equation*}

\begin{equation*}
\mathbf{w}^{EM} =
\begin{pmatrix}
0.559717\\
0.254888\\
\color{red} 0.121301\color{black} \\
0.064093
\end{pmatrix}\end{equation*}
\begin{equation*}
\left[ \frac{{w}^{EM}_i}{{w}^{EM}_j} \right] =
\begin{pmatrix}
$\,\,$ 1 $\,\,$ & $\,\,$2.1959$\,\,$ & $\,\,$\color{red} 4.6143\color{black} $\,\,$ & $\,\,$8.7329$\,\,$ \\
$\,\,$0.4554$\,\,$ & $\,\,$ 1 $\,\,$ & $\,\,$\color{red} 2.1013\color{black} $\,\,$ & $\,\,$3.9768  $\,\,$ \\
$\,\,$\color{red} 0.2167\color{black} $\,\,$ & $\,\,$\color{red} 0.4759\color{black} $\,\,$ & $\,\,$ 1 $\,\,$ & $\,\,$\color{red} 1.8926\color{black}  $\,\,$ \\
$\,\,$0.1145$\,\,$ & $\,\,$0.2515$\,\,$ & $\,\,$\color{red} 0.5284\color{black} $\,\,$ & $\,\,$ 1  $\,\,$ \\
\end{pmatrix},
\end{equation*}

\begin{equation*}
\mathbf{w}^{\prime} =
\begin{pmatrix}
0.556300\\
0.253332\\
0.126666\\
0.063702
\end{pmatrix} =
0.993894\cdot
\begin{pmatrix}
0.559717\\
0.254888\\
\color{gr} 0.127444\color{black} \\
0.064093
\end{pmatrix},
\end{equation*}
\begin{equation*}
\left[ \frac{{w}^{\prime}_i}{{w}^{\prime}_j} \right] =
\begin{pmatrix}
$\,\,$ 1 $\,\,$ & $\,\,$2.1959$\,\,$ & $\,\,$\color{gr} 4.3919\color{black} $\,\,$ & $\,\,$8.7329$\,\,$ \\
$\,\,$0.4554$\,\,$ & $\,\,$ 1 $\,\,$ & $\,\,$\color{gr} \color{blue} 2\color{black} $\,\,$ & $\,\,$3.9768  $\,\,$ \\
$\,\,$\color{gr} 0.2277\color{black} $\,\,$ & $\,\,$\color{gr} \color{blue}  1/2\color{black} $\,\,$ & $\,\,$ 1 $\,\,$ & $\,\,$\color{gr} 1.9884\color{black}  $\,\,$ \\
$\,\,$0.1145$\,\,$ & $\,\,$0.2515$\,\,$ & $\,\,$\color{gr} 0.5029\color{black} $\,\,$ & $\,\,$ 1  $\,\,$ \\
\end{pmatrix},
\end{equation*}
\end{example}
\newpage
\begin{example}
\begin{equation*}
\mathbf{A} =
\begin{pmatrix}
$\,\,$ 1 $\,\,$ & $\,\,$4$\,\,$ & $\,\,$4$\,\,$ & $\,\,$6 $\,\,$ \\
$\,\,$ 1/4$\,\,$ & $\,\,$ 1 $\,\,$ & $\,\,$3$\,\,$ & $\,\,$2 $\,\,$ \\
$\,\,$ 1/4$\,\,$ & $\,\,$ 1/3$\,\,$ & $\,\,$ 1 $\,\,$ & $\,\,$1 $\,\,$ \\
$\,\,$ 1/6$\,\,$ & $\,\,$ 1/2$\,\,$ & $\,\,$ 1 $\,\,$ & $\,\,$ 1  $\,\,$ \\
\end{pmatrix},
\qquad
\lambda_{\max} =
4.1031,
\qquad
CR = 0.0389
\end{equation*}

\begin{equation*}
\mathbf{w}^{EM} =
\begin{pmatrix}
0.590789\\
0.209253\\
0.101728\\
\color{red} 0.098229\color{black}
\end{pmatrix}\end{equation*}
\begin{equation*}
\left[ \frac{{w}^{EM}_i}{{w}^{EM}_j} \right] =
\begin{pmatrix}
$\,\,$ 1 $\,\,$ & $\,\,$2.8233$\,\,$ & $\,\,$5.8075$\,\,$ & $\,\,$\color{red} 6.0144\color{black} $\,\,$ \\
$\,\,$0.3542$\,\,$ & $\,\,$ 1 $\,\,$ & $\,\,$2.0570$\,\,$ & $\,\,$\color{red} 2.1302\color{black}   $\,\,$ \\
$\,\,$0.1722$\,\,$ & $\,\,$0.4862$\,\,$ & $\,\,$ 1 $\,\,$ & $\,\,$\color{red} 1.0356\color{black}  $\,\,$ \\
$\,\,$\color{red} 0.1663\color{black} $\,\,$ & $\,\,$\color{red} 0.4694\color{black} $\,\,$ & $\,\,$\color{red} 0.9656\color{black} $\,\,$ & $\,\,$ 1  $\,\,$ \\
\end{pmatrix},
\end{equation*}

\begin{equation*}
\mathbf{w}^{\prime} =
\begin{pmatrix}
0.590650\\
0.209204\\
0.101704\\
0.098442
\end{pmatrix} =
0.999765\cdot
\begin{pmatrix}
0.590789\\
0.209253\\
0.101728\\
\color{gr} 0.098465\color{black}
\end{pmatrix},
\end{equation*}
\begin{equation*}
\left[ \frac{{w}^{\prime}_i}{{w}^{\prime}_j} \right] =
\begin{pmatrix}
$\,\,$ 1 $\,\,$ & $\,\,$2.8233$\,\,$ & $\,\,$5.8075$\,\,$ & $\,\,$\color{gr} \color{blue} 6\color{black} $\,\,$ \\
$\,\,$0.3542$\,\,$ & $\,\,$ 1 $\,\,$ & $\,\,$2.0570$\,\,$ & $\,\,$\color{gr} 2.1252\color{black}   $\,\,$ \\
$\,\,$0.1722$\,\,$ & $\,\,$0.4862$\,\,$ & $\,\,$ 1 $\,\,$ & $\,\,$\color{gr} 1.0331\color{black}  $\,\,$ \\
$\,\,$\color{gr} \color{blue}  1/6\color{black} $\,\,$ & $\,\,$\color{gr} 0.4706\color{black} $\,\,$ & $\,\,$\color{gr} 0.9679\color{black} $\,\,$ & $\,\,$ 1  $\,\,$ \\
\end{pmatrix},
\end{equation*}
\end{example}
\newpage
\begin{example}
\begin{equation*}
\mathbf{A} =
\begin{pmatrix}
$\,\,$ 1 $\,\,$ & $\,\,$4$\,\,$ & $\,\,$4$\,\,$ & $\,\,$6 $\,\,$ \\
$\,\,$ 1/4$\,\,$ & $\,\,$ 1 $\,\,$ & $\,\,$5$\,\,$ & $\,\,$3 $\,\,$ \\
$\,\,$ 1/4$\,\,$ & $\,\,$ 1/5$\,\,$ & $\,\,$ 1 $\,\,$ & $\,\,$1 $\,\,$ \\
$\,\,$ 1/6$\,\,$ & $\,\,$ 1/3$\,\,$ & $\,\,$ 1 $\,\,$ & $\,\,$ 1  $\,\,$ \\
\end{pmatrix},
\qquad
\lambda_{\max} =
4.2277,
\qquad
CR = 0.0859
\end{equation*}

\begin{equation*}
\mathbf{w}^{EM} =
\begin{pmatrix}
0.576321\\
0.254868\\
0.086066\\
\color{red} 0.082744\color{black}
\end{pmatrix}\end{equation*}
\begin{equation*}
\left[ \frac{{w}^{EM}_i}{{w}^{EM}_j} \right] =
\begin{pmatrix}
$\,\,$ 1 $\,\,$ & $\,\,$2.2612$\,\,$ & $\,\,$6.6962$\,\,$ & $\,\,$\color{red} 6.9651\color{black} $\,\,$ \\
$\,\,$0.4422$\,\,$ & $\,\,$ 1 $\,\,$ & $\,\,$2.9613$\,\,$ & $\,\,$\color{red} 3.0802\color{black}   $\,\,$ \\
$\,\,$0.1493$\,\,$ & $\,\,$0.3377$\,\,$ & $\,\,$ 1 $\,\,$ & $\,\,$\color{red} 1.0401\color{black}  $\,\,$ \\
$\,\,$\color{red} 0.1436\color{black} $\,\,$ & $\,\,$\color{red} 0.3247\color{black} $\,\,$ & $\,\,$\color{red} 0.9614\color{black} $\,\,$ & $\,\,$ 1  $\,\,$ \\
\end{pmatrix},
\end{equation*}

\begin{equation*}
\mathbf{w}^{\prime} =
\begin{pmatrix}
0.575049\\
0.254306\\
0.085876\\
0.084769
\end{pmatrix} =
0.997793\cdot
\begin{pmatrix}
0.576321\\
0.254868\\
0.086066\\
\color{gr} 0.084956\color{black}
\end{pmatrix},
\end{equation*}
\begin{equation*}
\left[ \frac{{w}^{\prime}_i}{{w}^{\prime}_j} \right] =
\begin{pmatrix}
$\,\,$ 1 $\,\,$ & $\,\,$2.2612$\,\,$ & $\,\,$6.6962$\,\,$ & $\,\,$\color{gr} 6.7837\color{black} $\,\,$ \\
$\,\,$0.4422$\,\,$ & $\,\,$ 1 $\,\,$ & $\,\,$2.9613$\,\,$ & $\,\,$\color{gr} \color{blue} 3\color{black}   $\,\,$ \\
$\,\,$0.1493$\,\,$ & $\,\,$0.3377$\,\,$ & $\,\,$ 1 $\,\,$ & $\,\,$\color{gr} 1.0131\color{black}  $\,\,$ \\
$\,\,$\color{gr} 0.1474\color{black} $\,\,$ & $\,\,$\color{gr} \color{blue}  1/3\color{black} $\,\,$ & $\,\,$\color{gr} 0.9871\color{black} $\,\,$ & $\,\,$ 1  $\,\,$ \\
\end{pmatrix},
\end{equation*}
\end{example}
\newpage
\begin{example}
\begin{equation*}
\mathbf{A} =
\begin{pmatrix}
$\,\,$ 1 $\,\,$ & $\,\,$4$\,\,$ & $\,\,$4$\,\,$ & $\,\,$7 $\,\,$ \\
$\,\,$ 1/4$\,\,$ & $\,\,$ 1 $\,\,$ & $\,\,$5$\,\,$ & $\,\,$3 $\,\,$ \\
$\,\,$ 1/4$\,\,$ & $\,\,$ 1/5$\,\,$ & $\,\,$ 1 $\,\,$ & $\,\,$1 $\,\,$ \\
$\,\,$ 1/7$\,\,$ & $\,\,$ 1/3$\,\,$ & $\,\,$ 1 $\,\,$ & $\,\,$ 1  $\,\,$ \\
\end{pmatrix},
\qquad
\lambda_{\max} =
4.2251,
\qquad
CR = 0.0849
\end{equation*}

\begin{equation*}
\mathbf{w}^{EM} =
\begin{pmatrix}
0.586125\\
0.250377\\
0.085230\\
\color{red} 0.078268\color{black}
\end{pmatrix}\end{equation*}
\begin{equation*}
\left[ \frac{{w}^{EM}_i}{{w}^{EM}_j} \right] =
\begin{pmatrix}
$\,\,$ 1 $\,\,$ & $\,\,$2.3410$\,\,$ & $\,\,$6.8770$\,\,$ & $\,\,$\color{red} 7.4887\color{black} $\,\,$ \\
$\,\,$0.4272$\,\,$ & $\,\,$ 1 $\,\,$ & $\,\,$2.9377$\,\,$ & $\,\,$\color{red} 3.1990\color{black}   $\,\,$ \\
$\,\,$0.1454$\,\,$ & $\,\,$0.3404$\,\,$ & $\,\,$ 1 $\,\,$ & $\,\,$\color{red} 1.0890\color{black}  $\,\,$ \\
$\,\,$\color{red} 0.1335\color{black} $\,\,$ & $\,\,$\color{red} 0.3126\color{black} $\,\,$ & $\,\,$\color{red} 0.9183\color{black} $\,\,$ & $\,\,$ 1  $\,\,$ \\
\end{pmatrix},
\end{equation*}

\begin{equation*}
\mathbf{w}^{\prime} =
\begin{pmatrix}
0.583098\\
0.249084\\
0.084790\\
0.083028
\end{pmatrix} =
0.994836\cdot
\begin{pmatrix}
0.586125\\
0.250377\\
0.085230\\
\color{gr} 0.083459\color{black}
\end{pmatrix},
\end{equation*}
\begin{equation*}
\left[ \frac{{w}^{\prime}_i}{{w}^{\prime}_j} \right] =
\begin{pmatrix}
$\,\,$ 1 $\,\,$ & $\,\,$2.3410$\,\,$ & $\,\,$6.8770$\,\,$ & $\,\,$\color{gr} 7.0229\color{black} $\,\,$ \\
$\,\,$0.4272$\,\,$ & $\,\,$ 1 $\,\,$ & $\,\,$2.9377$\,\,$ & $\,\,$\color{gr} \color{blue} 3\color{black}   $\,\,$ \\
$\,\,$0.1454$\,\,$ & $\,\,$0.3404$\,\,$ & $\,\,$ 1 $\,\,$ & $\,\,$\color{gr} 1.0212\color{black}  $\,\,$ \\
$\,\,$\color{gr} 0.1424\color{black} $\,\,$ & $\,\,$\color{gr} \color{blue}  1/3\color{black} $\,\,$ & $\,\,$\color{gr} 0.9792\color{black} $\,\,$ & $\,\,$ 1  $\,\,$ \\
\end{pmatrix},
\end{equation*}
\end{example}
\newpage
\begin{example}
\begin{equation*}
\mathbf{A} =
\begin{pmatrix}
$\,\,$ 1 $\,\,$ & $\,\,$4$\,\,$ & $\,\,$5$\,\,$ & $\,\,$5 $\,\,$ \\
$\,\,$ 1/4$\,\,$ & $\,\,$ 1 $\,\,$ & $\,\,$2$\,\,$ & $\,\,$6 $\,\,$ \\
$\,\,$ 1/5$\,\,$ & $\,\,$ 1/2$\,\,$ & $\,\,$ 1 $\,\,$ & $\,\,$2 $\,\,$ \\
$\,\,$ 1/5$\,\,$ & $\,\,$ 1/6$\,\,$ & $\,\,$ 1/2$\,\,$ & $\,\,$ 1  $\,\,$ \\
\end{pmatrix},
\qquad
\lambda_{\max} =
4.2162,
\qquad
CR = 0.0815
\end{equation*}

\begin{equation*}
\mathbf{w}^{EM} =
\begin{pmatrix}
0.579359\\
0.239860\\
\color{red} 0.114520\color{black} \\
0.066261
\end{pmatrix}\end{equation*}
\begin{equation*}
\left[ \frac{{w}^{EM}_i}{{w}^{EM}_j} \right] =
\begin{pmatrix}
$\,\,$ 1 $\,\,$ & $\,\,$2.4154$\,\,$ & $\,\,$\color{red} 5.0590\color{black} $\,\,$ & $\,\,$8.7436$\,\,$ \\
$\,\,$0.4140$\,\,$ & $\,\,$ 1 $\,\,$ & $\,\,$\color{red} 2.0945\color{black} $\,\,$ & $\,\,$3.6200  $\,\,$ \\
$\,\,$\color{red} 0.1977\color{black} $\,\,$ & $\,\,$\color{red} 0.4774\color{black} $\,\,$ & $\,\,$ 1 $\,\,$ & $\,\,$\color{red} 1.7283\color{black}  $\,\,$ \\
$\,\,$0.1144$\,\,$ & $\,\,$0.2762$\,\,$ & $\,\,$\color{red} 0.5786\color{black} $\,\,$ & $\,\,$ 1  $\,\,$ \\
\end{pmatrix},
\end{equation*}

\begin{equation*}
\mathbf{w}^{\prime} =
\begin{pmatrix}
0.578577\\
0.239537\\
0.115715\\
0.066171
\end{pmatrix} =
0.998650\cdot
\begin{pmatrix}
0.579359\\
0.239860\\
\color{gr} 0.115872\color{black} \\
0.066261
\end{pmatrix},
\end{equation*}
\begin{equation*}
\left[ \frac{{w}^{\prime}_i}{{w}^{\prime}_j} \right] =
\begin{pmatrix}
$\,\,$ 1 $\,\,$ & $\,\,$2.4154$\,\,$ & $\,\,$\color{gr} \color{blue} 5\color{black} $\,\,$ & $\,\,$8.7436$\,\,$ \\
$\,\,$0.4140$\,\,$ & $\,\,$ 1 $\,\,$ & $\,\,$\color{gr} 2.0700\color{black} $\,\,$ & $\,\,$3.6200  $\,\,$ \\
$\,\,$\color{gr} \color{blue}  1/5\color{black} $\,\,$ & $\,\,$\color{gr} 0.4831\color{black} $\,\,$ & $\,\,$ 1 $\,\,$ & $\,\,$\color{gr} 1.7487\color{black}  $\,\,$ \\
$\,\,$0.1144$\,\,$ & $\,\,$0.2762$\,\,$ & $\,\,$\color{gr} 0.5718\color{black} $\,\,$ & $\,\,$ 1  $\,\,$ \\
\end{pmatrix},
\end{equation*}
\end{example}
\newpage
\begin{example}
\begin{equation*}
\mathbf{A} =
\begin{pmatrix}
$\,\,$ 1 $\,\,$ & $\,\,$4$\,\,$ & $\,\,$5$\,\,$ & $\,\,$5 $\,\,$ \\
$\,\,$ 1/4$\,\,$ & $\,\,$ 1 $\,\,$ & $\,\,$2$\,\,$ & $\,\,$7 $\,\,$ \\
$\,\,$ 1/5$\,\,$ & $\,\,$ 1/2$\,\,$ & $\,\,$ 1 $\,\,$ & $\,\,$2 $\,\,$ \\
$\,\,$ 1/5$\,\,$ & $\,\,$ 1/7$\,\,$ & $\,\,$ 1/2$\,\,$ & $\,\,$ 1  $\,\,$ \\
\end{pmatrix},
\qquad
\lambda_{\max} =
4.2610,
\qquad
CR = 0.0984
\end{equation*}

\begin{equation*}
\mathbf{w}^{EM} =
\begin{pmatrix}
0.575098\\
0.249126\\
\color{red} 0.112362\color{black} \\
0.063414
\end{pmatrix}\end{equation*}
\begin{equation*}
\left[ \frac{{w}^{EM}_i}{{w}^{EM}_j} \right] =
\begin{pmatrix}
$\,\,$ 1 $\,\,$ & $\,\,$2.3085$\,\,$ & $\,\,$\color{red} 5.1183\color{black} $\,\,$ & $\,\,$9.0690$\,\,$ \\
$\,\,$0.4332$\,\,$ & $\,\,$ 1 $\,\,$ & $\,\,$\color{red} 2.2172\color{black} $\,\,$ & $\,\,$3.9286  $\,\,$ \\
$\,\,$\color{red} 0.1954\color{black} $\,\,$ & $\,\,$\color{red} 0.4510\color{black} $\,\,$ & $\,\,$ 1 $\,\,$ & $\,\,$\color{red} 1.7719\color{black}  $\,\,$ \\
$\,\,$0.1103$\,\,$ & $\,\,$0.2545$\,\,$ & $\,\,$\color{red} 0.5644\color{black} $\,\,$ & $\,\,$ 1  $\,\,$ \\
\end{pmatrix},
\end{equation*}

\begin{equation*}
\mathbf{w}^{\prime} =
\begin{pmatrix}
0.573574\\
0.248466\\
0.114715\\
0.063245
\end{pmatrix} =
0.997349\cdot
\begin{pmatrix}
0.575098\\
0.249126\\
\color{gr} 0.115020\color{black} \\
0.063414
\end{pmatrix},
\end{equation*}
\begin{equation*}
\left[ \frac{{w}^{\prime}_i}{{w}^{\prime}_j} \right] =
\begin{pmatrix}
$\,\,$ 1 $\,\,$ & $\,\,$2.3085$\,\,$ & $\,\,$\color{gr} \color{blue} 5\color{black} $\,\,$ & $\,\,$9.0690$\,\,$ \\
$\,\,$0.4332$\,\,$ & $\,\,$ 1 $\,\,$ & $\,\,$\color{gr} 2.1659\color{black} $\,\,$ & $\,\,$3.9286  $\,\,$ \\
$\,\,$\color{gr} \color{blue}  1/5\color{black} $\,\,$ & $\,\,$\color{gr} 0.4617\color{black} $\,\,$ & $\,\,$ 1 $\,\,$ & $\,\,$\color{gr} 1.8138\color{black}  $\,\,$ \\
$\,\,$0.1103$\,\,$ & $\,\,$0.2545$\,\,$ & $\,\,$\color{gr} 0.5513\color{black} $\,\,$ & $\,\,$ 1  $\,\,$ \\
\end{pmatrix},
\end{equation*}
\end{example}
\newpage
\begin{example}
\begin{equation*}
\mathbf{A} =
\begin{pmatrix}
$\,\,$ 1 $\,\,$ & $\,\,$4$\,\,$ & $\,\,$5$\,\,$ & $\,\,$6 $\,\,$ \\
$\,\,$ 1/4$\,\,$ & $\,\,$ 1 $\,\,$ & $\,\,$2$\,\,$ & $\,\,$6 $\,\,$ \\
$\,\,$ 1/5$\,\,$ & $\,\,$ 1/2$\,\,$ & $\,\,$ 1 $\,\,$ & $\,\,$2 $\,\,$ \\
$\,\,$ 1/6$\,\,$ & $\,\,$ 1/6$\,\,$ & $\,\,$ 1/2$\,\,$ & $\,\,$ 1  $\,\,$ \\
\end{pmatrix},
\qquad
\lambda_{\max} =
4.1655,
\qquad
CR = 0.0624
\end{equation*}

\begin{equation*}
\mathbf{w}^{EM} =
\begin{pmatrix}
0.591134\\
0.234409\\
\color{red} 0.113124\color{black} \\
0.061333
\end{pmatrix}\end{equation*}
\begin{equation*}
\left[ \frac{{w}^{EM}_i}{{w}^{EM}_j} \right] =
\begin{pmatrix}
$\,\,$ 1 $\,\,$ & $\,\,$2.5218$\,\,$ & $\,\,$\color{red} 5.2255\color{black} $\,\,$ & $\,\,$9.6381$\,\,$ \\
$\,\,$0.3965$\,\,$ & $\,\,$ 1 $\,\,$ & $\,\,$\color{red} 2.0721\color{black} $\,\,$ & $\,\,$3.8219  $\,\,$ \\
$\,\,$\color{red} 0.1914\color{black} $\,\,$ & $\,\,$\color{red} 0.4826\color{black} $\,\,$ & $\,\,$ 1 $\,\,$ & $\,\,$\color{red} 1.8444\color{black}  $\,\,$ \\
$\,\,$0.1038$\,\,$ & $\,\,$0.2617$\,\,$ & $\,\,$\color{red} 0.5422\color{black} $\,\,$ & $\,\,$ 1  $\,\,$ \\
\end{pmatrix},
\end{equation*}

\begin{equation*}
\mathbf{w}^{\prime} =
\begin{pmatrix}
0.588731\\
0.233456\\
0.116728\\
0.061084
\end{pmatrix} =
0.995936\cdot
\begin{pmatrix}
0.591134\\
0.234409\\
\color{gr} 0.117205\color{black} \\
0.061333
\end{pmatrix},
\end{equation*}
\begin{equation*}
\left[ \frac{{w}^{\prime}_i}{{w}^{\prime}_j} \right] =
\begin{pmatrix}
$\,\,$ 1 $\,\,$ & $\,\,$2.5218$\,\,$ & $\,\,$\color{gr} 5.0436\color{black} $\,\,$ & $\,\,$9.6381$\,\,$ \\
$\,\,$0.3965$\,\,$ & $\,\,$ 1 $\,\,$ & $\,\,$\color{gr} \color{blue} 2\color{black} $\,\,$ & $\,\,$3.8219  $\,\,$ \\
$\,\,$\color{gr} 0.1983\color{black} $\,\,$ & $\,\,$\color{gr} \color{blue}  1/2\color{black} $\,\,$ & $\,\,$ 1 $\,\,$ & $\,\,$\color{gr} 1.9109\color{black}  $\,\,$ \\
$\,\,$0.1038$\,\,$ & $\,\,$0.2617$\,\,$ & $\,\,$\color{gr} 0.5233\color{black} $\,\,$ & $\,\,$ 1  $\,\,$ \\
\end{pmatrix},
\end{equation*}
\end{example}
\newpage
\begin{example}
\begin{equation*}
\mathbf{A} =
\begin{pmatrix}
$\,\,$ 1 $\,\,$ & $\,\,$4$\,\,$ & $\,\,$5$\,\,$ & $\,\,$6 $\,\,$ \\
$\,\,$ 1/4$\,\,$ & $\,\,$ 1 $\,\,$ & $\,\,$2$\,\,$ & $\,\,$7 $\,\,$ \\
$\,\,$ 1/5$\,\,$ & $\,\,$ 1/2$\,\,$ & $\,\,$ 1 $\,\,$ & $\,\,$2 $\,\,$ \\
$\,\,$ 1/6$\,\,$ & $\,\,$ 1/7$\,\,$ & $\,\,$ 1/2$\,\,$ & $\,\,$ 1  $\,\,$ \\
\end{pmatrix},
\qquad
\lambda_{\max} =
4.2057,
\qquad
CR = 0.0776
\end{equation*}

\begin{equation*}
\mathbf{w}^{EM} =
\begin{pmatrix}
0.586838\\
0.243292\\
\color{red} 0.111178\color{black} \\
0.058693
\end{pmatrix}\end{equation*}
\begin{equation*}
\left[ \frac{{w}^{EM}_i}{{w}^{EM}_j} \right] =
\begin{pmatrix}
$\,\,$ 1 $\,\,$ & $\,\,$2.4121$\,\,$ & $\,\,$\color{red} 5.2784\color{black} $\,\,$ & $\,\,$9.9984$\,\,$ \\
$\,\,$0.4146$\,\,$ & $\,\,$ 1 $\,\,$ & $\,\,$\color{red} 2.1883\color{black} $\,\,$ & $\,\,$4.1452  $\,\,$ \\
$\,\,$\color{red} 0.1895\color{black} $\,\,$ & $\,\,$\color{red} 0.4570\color{black} $\,\,$ & $\,\,$ 1 $\,\,$ & $\,\,$\color{red} 1.8942\color{black}  $\,\,$ \\
$\,\,$0.1000$\,\,$ & $\,\,$0.2412$\,\,$ & $\,\,$\color{red} 0.5279\color{black} $\,\,$ & $\,\,$ 1  $\,\,$ \\
\end{pmatrix},
\end{equation*}

\begin{equation*}
\mathbf{w}^{\prime} =
\begin{pmatrix}
0.583227\\
0.241795\\
0.116645\\
0.058332
\end{pmatrix} =
0.993848\cdot
\begin{pmatrix}
0.586838\\
0.243292\\
\color{gr} 0.117368\color{black} \\
0.058693
\end{pmatrix},
\end{equation*}
\begin{equation*}
\left[ \frac{{w}^{\prime}_i}{{w}^{\prime}_j} \right] =
\begin{pmatrix}
$\,\,$ 1 $\,\,$ & $\,\,$2.4121$\,\,$ & $\,\,$\color{gr} \color{blue} 5\color{black} $\,\,$ & $\,\,$9.9984$\,\,$ \\
$\,\,$0.4146$\,\,$ & $\,\,$ 1 $\,\,$ & $\,\,$\color{gr} 2.0729\color{black} $\,\,$ & $\,\,$4.1452  $\,\,$ \\
$\,\,$\color{gr} \color{blue}  1/5\color{black} $\,\,$ & $\,\,$\color{gr} 0.4824\color{black} $\,\,$ & $\,\,$ 1 $\,\,$ & $\,\,$\color{gr} 1.9997\color{black}  $\,\,$ \\
$\,\,$0.1000$\,\,$ & $\,\,$0.2412$\,\,$ & $\,\,$\color{gr} 0.5001\color{black} $\,\,$ & $\,\,$ 1  $\,\,$ \\
\end{pmatrix},
\end{equation*}
\end{example}
\newpage
\begin{example}
\begin{equation*}
\mathbf{A} =
\begin{pmatrix}
$\,\,$ 1 $\,\,$ & $\,\,$4$\,\,$ & $\,\,$5$\,\,$ & $\,\,$6 $\,\,$ \\
$\,\,$ 1/4$\,\,$ & $\,\,$ 1 $\,\,$ & $\,\,$2$\,\,$ & $\,\,$8 $\,\,$ \\
$\,\,$ 1/5$\,\,$ & $\,\,$ 1/2$\,\,$ & $\,\,$ 1 $\,\,$ & $\,\,$2 $\,\,$ \\
$\,\,$ 1/6$\,\,$ & $\,\,$ 1/8$\,\,$ & $\,\,$ 1/2$\,\,$ & $\,\,$ 1  $\,\,$ \\
\end{pmatrix},
\qquad
\lambda_{\max} =
4.2460,
\qquad
CR = 0.0928
\end{equation*}

\begin{equation*}
\mathbf{w}^{EM} =
\begin{pmatrix}
0.582707\\
0.251425\\
\color{red} 0.109414\color{black} \\
0.056454
\end{pmatrix}\end{equation*}
\begin{equation*}
\left[ \frac{{w}^{EM}_i}{{w}^{EM}_j} \right] =
\begin{pmatrix}
$\,\,$ 1 $\,\,$ & $\,\,$2.3176$\,\,$ & $\,\,$\color{red} 5.3257\color{black} $\,\,$ & $\,\,$10.3218$\,\,$ \\
$\,\,$0.4315$\,\,$ & $\,\,$ 1 $\,\,$ & $\,\,$\color{red} 2.2979\color{black} $\,\,$ & $\,\,$4.4536  $\,\,$ \\
$\,\,$\color{red} 0.1878\color{black} $\,\,$ & $\,\,$\color{red} 0.4352\color{black} $\,\,$ & $\,\,$ 1 $\,\,$ & $\,\,$\color{red} 1.9381\color{black}  $\,\,$ \\
$\,\,$0.0969$\,\,$ & $\,\,$0.2245$\,\,$ & $\,\,$\color{red} 0.5160\color{black} $\,\,$ & $\,\,$ 1  $\,\,$ \\
\end{pmatrix},
\end{equation*}

\begin{equation*}
\mathbf{w}^{\prime} =
\begin{pmatrix}
0.580678\\
0.250550\\
0.112515\\
0.056258
\end{pmatrix} =
0.996518\cdot
\begin{pmatrix}
0.582707\\
0.251425\\
\color{gr} 0.112908\color{black} \\
0.056454
\end{pmatrix},
\end{equation*}
\begin{equation*}
\left[ \frac{{w}^{\prime}_i}{{w}^{\prime}_j} \right] =
\begin{pmatrix}
$\,\,$ 1 $\,\,$ & $\,\,$2.3176$\,\,$ & $\,\,$\color{gr} 5.1609\color{black} $\,\,$ & $\,\,$10.3218$\,\,$ \\
$\,\,$0.4315$\,\,$ & $\,\,$ 1 $\,\,$ & $\,\,$\color{gr} 2.2268\color{black} $\,\,$ & $\,\,$4.4536  $\,\,$ \\
$\,\,$\color{gr} 0.1938\color{black} $\,\,$ & $\,\,$\color{gr} 0.4491\color{black} $\,\,$ & $\,\,$ 1 $\,\,$ & $\,\,$\color{gr} \color{blue} 2\color{black}  $\,\,$ \\
$\,\,$0.0969$\,\,$ & $\,\,$0.2245$\,\,$ & $\,\,$\color{gr} \color{blue}  1/2\color{black} $\,\,$ & $\,\,$ 1  $\,\,$ \\
\end{pmatrix},
\end{equation*}
\end{example}
\newpage
\begin{example}
\begin{equation*}
\mathbf{A} =
\begin{pmatrix}
$\,\,$ 1 $\,\,$ & $\,\,$4$\,\,$ & $\,\,$5$\,\,$ & $\,\,$7 $\,\,$ \\
$\,\,$ 1/4$\,\,$ & $\,\,$ 1 $\,\,$ & $\,\,$2$\,\,$ & $\,\,$6 $\,\,$ \\
$\,\,$ 1/5$\,\,$ & $\,\,$ 1/2$\,\,$ & $\,\,$ 1 $\,\,$ & $\,\,$2 $\,\,$ \\
$\,\,$ 1/7$\,\,$ & $\,\,$ 1/6$\,\,$ & $\,\,$ 1/2$\,\,$ & $\,\,$ 1  $\,\,$ \\
\end{pmatrix},
\qquad
\lambda_{\max} =
4.1301,
\qquad
CR = 0.0490
\end{equation*}

\begin{equation*}
\mathbf{w}^{EM} =
\begin{pmatrix}
0.600896\\
0.229730\\
\color{red} 0.111849\color{black} \\
0.057525
\end{pmatrix}\end{equation*}
\begin{equation*}
\left[ \frac{{w}^{EM}_i}{{w}^{EM}_j} \right] =
\begin{pmatrix}
$\,\,$ 1 $\,\,$ & $\,\,$2.6157$\,\,$ & $\,\,$\color{red} 5.3724\color{black} $\,\,$ & $\,\,$10.4459$\,\,$ \\
$\,\,$0.3823$\,\,$ & $\,\,$ 1 $\,\,$ & $\,\,$\color{red} 2.0539\color{black} $\,\,$ & $\,\,$3.9936  $\,\,$ \\
$\,\,$\color{red} 0.1861\color{black} $\,\,$ & $\,\,$\color{red} 0.4869\color{black} $\,\,$ & $\,\,$ 1 $\,\,$ & $\,\,$\color{red} 1.9444\color{black}  $\,\,$ \\
$\,\,$0.0957$\,\,$ & $\,\,$0.2504$\,\,$ & $\,\,$\color{red} 0.5143\color{black} $\,\,$ & $\,\,$ 1  $\,\,$ \\
\end{pmatrix},
\end{equation*}

\begin{equation*}
\mathbf{w}^{\prime} =
\begin{pmatrix}
0.599089\\
0.229040\\
0.114520\\
0.057352
\end{pmatrix} =
0.996993\cdot
\begin{pmatrix}
0.600896\\
0.229730\\
\color{gr} 0.114865\color{black} \\
0.057525
\end{pmatrix},
\end{equation*}
\begin{equation*}
\left[ \frac{{w}^{\prime}_i}{{w}^{\prime}_j} \right] =
\begin{pmatrix}
$\,\,$ 1 $\,\,$ & $\,\,$2.6157$\,\,$ & $\,\,$\color{gr} 5.2313\color{black} $\,\,$ & $\,\,$10.4459$\,\,$ \\
$\,\,$0.3823$\,\,$ & $\,\,$ 1 $\,\,$ & $\,\,$\color{gr} \color{blue} 2\color{black} $\,\,$ & $\,\,$3.9936  $\,\,$ \\
$\,\,$\color{gr} 0.1912\color{black} $\,\,$ & $\,\,$\color{gr} \color{blue}  1/2\color{black} $\,\,$ & $\,\,$ 1 $\,\,$ & $\,\,$\color{gr} 1.9968\color{black}  $\,\,$ \\
$\,\,$0.0957$\,\,$ & $\,\,$0.2504$\,\,$ & $\,\,$\color{gr} 0.5008\color{black} $\,\,$ & $\,\,$ 1  $\,\,$ \\
\end{pmatrix},
\end{equation*}
\end{example}
\newpage
\begin{example}
\begin{equation*}
\mathbf{A} =
\begin{pmatrix}
$\,\,$ 1 $\,\,$ & $\,\,$4$\,\,$ & $\,\,$5$\,\,$ & $\,\,$7 $\,\,$ \\
$\,\,$ 1/4$\,\,$ & $\,\,$ 1 $\,\,$ & $\,\,$2$\,\,$ & $\,\,$7 $\,\,$ \\
$\,\,$ 1/5$\,\,$ & $\,\,$ 1/2$\,\,$ & $\,\,$ 1 $\,\,$ & $\,\,$2 $\,\,$ \\
$\,\,$ 1/7$\,\,$ & $\,\,$ 1/7$\,\,$ & $\,\,$ 1/2$\,\,$ & $\,\,$ 1  $\,\,$ \\
\end{pmatrix},
\qquad
\lambda_{\max} =
4.1665,
\qquad
CR = 0.0628
\end{equation*}

\begin{equation*}
\mathbf{w}^{EM} =
\begin{pmatrix}
0.596557\\
0.238317\\
\color{red} 0.110079\color{black} \\
0.055047
\end{pmatrix}\end{equation*}
\begin{equation*}
\left[ \frac{{w}^{EM}_i}{{w}^{EM}_j} \right] =
\begin{pmatrix}
$\,\,$ 1 $\,\,$ & $\,\,$2.5032$\,\,$ & $\,\,$\color{red} 5.4194\color{black} $\,\,$ & $\,\,$10.8372$\,\,$ \\
$\,\,$0.3995$\,\,$ & $\,\,$ 1 $\,\,$ & $\,\,$\color{red} 2.1650\color{black} $\,\,$ & $\,\,$4.3293  $\,\,$ \\
$\,\,$\color{red} 0.1845\color{black} $\,\,$ & $\,\,$\color{red} 0.4619\color{black} $\,\,$ & $\,\,$ 1 $\,\,$ & $\,\,$\color{red} 1.9997\color{black}  $\,\,$ \\
$\,\,$0.0923$\,\,$ & $\,\,$0.2310$\,\,$ & $\,\,$\color{red} 0.5001\color{black} $\,\,$ & $\,\,$ 1  $\,\,$ \\
\end{pmatrix},
\end{equation*}

\begin{equation*}
\mathbf{w}^{\prime} =
\begin{pmatrix}
0.596547\\
0.238313\\
0.110093\\
0.055046
\end{pmatrix} =
0.999984\cdot
\begin{pmatrix}
0.596557\\
0.238317\\
\color{gr} 0.110095\color{black} \\
0.055047
\end{pmatrix},
\end{equation*}
\begin{equation*}
\left[ \frac{{w}^{\prime}_i}{{w}^{\prime}_j} \right] =
\begin{pmatrix}
$\,\,$ 1 $\,\,$ & $\,\,$2.5032$\,\,$ & $\,\,$\color{gr} 5.4186\color{black} $\,\,$ & $\,\,$10.8372$\,\,$ \\
$\,\,$0.3995$\,\,$ & $\,\,$ 1 $\,\,$ & $\,\,$\color{gr} 2.1647\color{black} $\,\,$ & $\,\,$4.3293  $\,\,$ \\
$\,\,$\color{gr} 0.1846\color{black} $\,\,$ & $\,\,$\color{gr} 0.4620\color{black} $\,\,$ & $\,\,$ 1 $\,\,$ & $\,\,$\color{gr} \color{blue} 2\color{black}  $\,\,$ \\
$\,\,$0.0923$\,\,$ & $\,\,$0.2310$\,\,$ & $\,\,$\color{gr} \color{blue}  1/2\color{black} $\,\,$ & $\,\,$ 1  $\,\,$ \\
\end{pmatrix},
\end{equation*}
\end{example}
\newpage
\begin{example}
\begin{equation*}
\mathbf{A} =
\begin{pmatrix}
$\,\,$ 1 $\,\,$ & $\,\,$4$\,\,$ & $\,\,$5$\,\,$ & $\,\,$7 $\,\,$ \\
$\,\,$ 1/4$\,\,$ & $\,\,$ 1 $\,\,$ & $\,\,$5$\,\,$ & $\,\,$3 $\,\,$ \\
$\,\,$ 1/5$\,\,$ & $\,\,$ 1/5$\,\,$ & $\,\,$ 1 $\,\,$ & $\,\,$1 $\,\,$ \\
$\,\,$ 1/7$\,\,$ & $\,\,$ 1/3$\,\,$ & $\,\,$ 1 $\,\,$ & $\,\,$ 1  $\,\,$ \\
\end{pmatrix},
\qquad
\lambda_{\max} =
4.1667,
\qquad
CR = 0.0629
\end{equation*}

\begin{equation*}
\mathbf{w}^{EM} =
\begin{pmatrix}
0.601358\\
0.243538\\
0.077779\\
\color{red} 0.077325\color{black}
\end{pmatrix}\end{equation*}
\begin{equation*}
\left[ \frac{{w}^{EM}_i}{{w}^{EM}_j} \right] =
\begin{pmatrix}
$\,\,$ 1 $\,\,$ & $\,\,$2.4693$\,\,$ & $\,\,$7.7316$\,\,$ & $\,\,$\color{red} 7.7770\color{black} $\,\,$ \\
$\,\,$0.4050$\,\,$ & $\,\,$ 1 $\,\,$ & $\,\,$3.1311$\,\,$ & $\,\,$\color{red} 3.1495\color{black}   $\,\,$ \\
$\,\,$0.1293$\,\,$ & $\,\,$0.3194$\,\,$ & $\,\,$ 1 $\,\,$ & $\,\,$\color{red} 1.0059\color{black}  $\,\,$ \\
$\,\,$\color{red} 0.1286\color{black} $\,\,$ & $\,\,$\color{red} 0.3175\color{black} $\,\,$ & $\,\,$\color{red} 0.9942\color{black} $\,\,$ & $\,\,$ 1  $\,\,$ \\
\end{pmatrix},
\end{equation*}

\begin{equation*}
\mathbf{w}^{\prime} =
\begin{pmatrix}
0.601085\\
0.243427\\
0.077744\\
0.077744
\end{pmatrix} =
0.999546\cdot
\begin{pmatrix}
0.601358\\
0.243538\\
0.077779\\
\color{gr} 0.077779\color{black}
\end{pmatrix},
\end{equation*}
\begin{equation*}
\left[ \frac{{w}^{\prime}_i}{{w}^{\prime}_j} \right] =
\begin{pmatrix}
$\,\,$ 1 $\,\,$ & $\,\,$2.4693$\,\,$ & $\,\,$7.7316$\,\,$ & $\,\,$\color{gr} 7.7316\color{black} $\,\,$ \\
$\,\,$0.4050$\,\,$ & $\,\,$ 1 $\,\,$ & $\,\,$3.1311$\,\,$ & $\,\,$\color{gr} 3.1311\color{black}   $\,\,$ \\
$\,\,$0.1293$\,\,$ & $\,\,$0.3194$\,\,$ & $\,\,$ 1 $\,\,$ & $\,\,$\color{gr} \color{blue} 1\color{black}  $\,\,$ \\
$\,\,$\color{gr} 0.1293\color{black} $\,\,$ & $\,\,$\color{gr} 0.3194\color{black} $\,\,$ & $\,\,$\color{gr} \color{blue} 1\color{black} $\,\,$ & $\,\,$ 1  $\,\,$ \\
\end{pmatrix},
\end{equation*}
\end{example}
\newpage
\begin{example}
\begin{equation*}
\mathbf{A} =
\begin{pmatrix}
$\,\,$ 1 $\,\,$ & $\,\,$4$\,\,$ & $\,\,$5$\,\,$ & $\,\,$8 $\,\,$ \\
$\,\,$ 1/4$\,\,$ & $\,\,$ 1 $\,\,$ & $\,\,$2$\,\,$ & $\,\,$8 $\,\,$ \\
$\,\,$ 1/5$\,\,$ & $\,\,$ 1/2$\,\,$ & $\,\,$ 1 $\,\,$ & $\,\,$3 $\,\,$ \\
$\,\,$ 1/8$\,\,$ & $\,\,$ 1/8$\,\,$ & $\,\,$ 1/3$\,\,$ & $\,\,$ 1  $\,\,$ \\
\end{pmatrix},
\qquad
\lambda_{\max} =
4.1689,
\qquad
CR = 0.0637
\end{equation*}

\begin{equation*}
\mathbf{w}^{EM} =
\begin{pmatrix}
0.599798\\
0.236588\\
\color{red} 0.118189\color{black} \\
0.045424
\end{pmatrix}\end{equation*}
\begin{equation*}
\left[ \frac{{w}^{EM}_i}{{w}^{EM}_j} \right] =
\begin{pmatrix}
$\,\,$ 1 $\,\,$ & $\,\,$2.5352$\,\,$ & $\,\,$\color{red} 5.0749\color{black} $\,\,$ & $\,\,$13.2043$\,\,$ \\
$\,\,$0.3944$\,\,$ & $\,\,$ 1 $\,\,$ & $\,\,$\color{red} 2.0018\color{black} $\,\,$ & $\,\,$5.2084  $\,\,$ \\
$\,\,$\color{red} 0.1970\color{black} $\,\,$ & $\,\,$\color{red} 0.4996\color{black} $\,\,$ & $\,\,$ 1 $\,\,$ & $\,\,$\color{red} 2.6019\color{black}  $\,\,$ \\
$\,\,$0.0757$\,\,$ & $\,\,$0.1920$\,\,$ & $\,\,$\color{red} 0.3843\color{black} $\,\,$ & $\,\,$ 1  $\,\,$ \\
\end{pmatrix},
\end{equation*}

\begin{equation*}
\mathbf{w}^{\prime} =
\begin{pmatrix}
0.599735\\
0.236564\\
0.118282\\
0.045420
\end{pmatrix} =
0.999895\cdot
\begin{pmatrix}
0.599798\\
0.236588\\
\color{gr} 0.118294\color{black} \\
0.045424
\end{pmatrix},
\end{equation*}
\begin{equation*}
\left[ \frac{{w}^{\prime}_i}{{w}^{\prime}_j} \right] =
\begin{pmatrix}
$\,\,$ 1 $\,\,$ & $\,\,$2.5352$\,\,$ & $\,\,$\color{gr} 5.0704\color{black} $\,\,$ & $\,\,$13.2043$\,\,$ \\
$\,\,$0.3944$\,\,$ & $\,\,$ 1 $\,\,$ & $\,\,$\color{gr} \color{blue} 2\color{black} $\,\,$ & $\,\,$5.2084  $\,\,$ \\
$\,\,$\color{gr} 0.1972\color{black} $\,\,$ & $\,\,$\color{gr} \color{blue}  1/2\color{black} $\,\,$ & $\,\,$ 1 $\,\,$ & $\,\,$\color{gr} 2.6042\color{black}  $\,\,$ \\
$\,\,$0.0757$\,\,$ & $\,\,$0.1920$\,\,$ & $\,\,$\color{gr} 0.3840\color{black} $\,\,$ & $\,\,$ 1  $\,\,$ \\
\end{pmatrix},
\end{equation*}
\end{example}
\newpage
\begin{example}
\begin{equation*}
\mathbf{A} =
\begin{pmatrix}
$\,\,$ 1 $\,\,$ & $\,\,$4$\,\,$ & $\,\,$5$\,\,$ & $\,\,$8 $\,\,$ \\
$\,\,$ 1/4$\,\,$ & $\,\,$ 1 $\,\,$ & $\,\,$2$\,\,$ & $\,\,$9 $\,\,$ \\
$\,\,$ 1/5$\,\,$ & $\,\,$ 1/2$\,\,$ & $\,\,$ 1 $\,\,$ & $\,\,$3 $\,\,$ \\
$\,\,$ 1/8$\,\,$ & $\,\,$ 1/9$\,\,$ & $\,\,$ 1/3$\,\,$ & $\,\,$ 1  $\,\,$ \\
\end{pmatrix},
\qquad
\lambda_{\max} =
4.1972,
\qquad
CR = 0.0744
\end{equation*}

\begin{equation*}
\mathbf{w}^{EM} =
\begin{pmatrix}
0.596374\\
0.243164\\
\color{red} 0.116543\color{black} \\
0.043918
\end{pmatrix}\end{equation*}
\begin{equation*}
\left[ \frac{{w}^{EM}_i}{{w}^{EM}_j} \right] =
\begin{pmatrix}
$\,\,$ 1 $\,\,$ & $\,\,$2.4526$\,\,$ & $\,\,$\color{red} 5.1172\color{black} $\,\,$ & $\,\,$13.5794$\,\,$ \\
$\,\,$0.4077$\,\,$ & $\,\,$ 1 $\,\,$ & $\,\,$\color{red} 2.0865\color{black} $\,\,$ & $\,\,$5.5368  $\,\,$ \\
$\,\,$\color{red} 0.1954\color{black} $\,\,$ & $\,\,$\color{red} 0.4793\color{black} $\,\,$ & $\,\,$ 1 $\,\,$ & $\,\,$\color{red} 2.6537\color{black}  $\,\,$ \\
$\,\,$0.0736$\,\,$ & $\,\,$0.1806$\,\,$ & $\,\,$\color{red} 0.3768\color{black} $\,\,$ & $\,\,$ 1  $\,\,$ \\
\end{pmatrix},
\end{equation*}

\begin{equation*}
\mathbf{w}^{\prime} =
\begin{pmatrix}
0.594750\\
0.242502\\
0.118950\\
0.043798
\end{pmatrix} =
0.997276\cdot
\begin{pmatrix}
0.596374\\
0.243164\\
\color{gr} 0.119275\color{black} \\
0.043918
\end{pmatrix},
\end{equation*}
\begin{equation*}
\left[ \frac{{w}^{\prime}_i}{{w}^{\prime}_j} \right] =
\begin{pmatrix}
$\,\,$ 1 $\,\,$ & $\,\,$2.4526$\,\,$ & $\,\,$\color{gr} \color{blue} 5\color{black} $\,\,$ & $\,\,$13.5794$\,\,$ \\
$\,\,$0.4077$\,\,$ & $\,\,$ 1 $\,\,$ & $\,\,$\color{gr} 2.0387\color{black} $\,\,$ & $\,\,$5.5368  $\,\,$ \\
$\,\,$\color{gr} \color{blue}  1/5\color{black} $\,\,$ & $\,\,$\color{gr} 0.4905\color{black} $\,\,$ & $\,\,$ 1 $\,\,$ & $\,\,$\color{gr} 2.7159\color{black}  $\,\,$ \\
$\,\,$0.0736$\,\,$ & $\,\,$0.1806$\,\,$ & $\,\,$\color{gr} 0.3682\color{black} $\,\,$ & $\,\,$ 1  $\,\,$ \\
\end{pmatrix},
\end{equation*}
\end{example}
\newpage
\begin{example}
\begin{equation*}
\mathbf{A} =
\begin{pmatrix}
$\,\,$ 1 $\,\,$ & $\,\,$4$\,\,$ & $\,\,$5$\,\,$ & $\,\,$8 $\,\,$ \\
$\,\,$ 1/4$\,\,$ & $\,\,$ 1 $\,\,$ & $\,\,$4$\,\,$ & $\,\,$3 $\,\,$ \\
$\,\,$ 1/5$\,\,$ & $\,\,$ 1/4$\,\,$ & $\,\,$ 1 $\,\,$ & $\,\,$1 $\,\,$ \\
$\,\,$ 1/8$\,\,$ & $\,\,$ 1/3$\,\,$ & $\,\,$ 1 $\,\,$ & $\,\,$ 1  $\,\,$ \\
\end{pmatrix},
\qquad
\lambda_{\max} =
4.1163,
\qquad
CR = 0.0439
\end{equation*}

\begin{equation*}
\mathbf{w}^{EM} =
\begin{pmatrix}
0.615897\\
0.226962\\
0.081884\\
\color{red} 0.075257\color{black}
\end{pmatrix}\end{equation*}
\begin{equation*}
\left[ \frac{{w}^{EM}_i}{{w}^{EM}_j} \right] =
\begin{pmatrix}
$\,\,$ 1 $\,\,$ & $\,\,$2.7137$\,\,$ & $\,\,$7.5215$\,\,$ & $\,\,$\color{red} 8.1839\color{black} $\,\,$ \\
$\,\,$0.3685$\,\,$ & $\,\,$ 1 $\,\,$ & $\,\,$2.7717$\,\,$ & $\,\,$\color{red} 3.0158\color{black}   $\,\,$ \\
$\,\,$0.1330$\,\,$ & $\,\,$0.3608$\,\,$ & $\,\,$ 1 $\,\,$ & $\,\,$\color{red} 1.0881\color{black}  $\,\,$ \\
$\,\,$\color{red} 0.1222\color{black} $\,\,$ & $\,\,$\color{red} 0.3316\color{black} $\,\,$ & $\,\,$\color{red} 0.9191\color{black} $\,\,$ & $\,\,$ 1  $\,\,$ \\
\end{pmatrix},
\end{equation*}

\begin{equation*}
\mathbf{w}^{\prime} =
\begin{pmatrix}
0.615653\\
0.226872\\
0.081852\\
0.075624
\end{pmatrix} =
0.999604\cdot
\begin{pmatrix}
0.615897\\
0.226962\\
0.081884\\
\color{gr} 0.075654\color{black}
\end{pmatrix},
\end{equation*}
\begin{equation*}
\left[ \frac{{w}^{\prime}_i}{{w}^{\prime}_j} \right] =
\begin{pmatrix}
$\,\,$ 1 $\,\,$ & $\,\,$2.7137$\,\,$ & $\,\,$7.5215$\,\,$ & $\,\,$\color{gr} 8.1410\color{black} $\,\,$ \\
$\,\,$0.3685$\,\,$ & $\,\,$ 1 $\,\,$ & $\,\,$2.7717$\,\,$ & $\,\,$\color{gr} \color{blue} 3\color{black}   $\,\,$ \\
$\,\,$0.1330$\,\,$ & $\,\,$0.3608$\,\,$ & $\,\,$ 1 $\,\,$ & $\,\,$\color{gr} 1.0824\color{black}  $\,\,$ \\
$\,\,$\color{gr} 0.1228\color{black} $\,\,$ & $\,\,$\color{gr} \color{blue}  1/3\color{black} $\,\,$ & $\,\,$\color{gr} 0.9239\color{black} $\,\,$ & $\,\,$ 1  $\,\,$ \\
\end{pmatrix},
\end{equation*}
\end{example}
\newpage
\begin{example}
\begin{equation*}
\mathbf{A} =
\begin{pmatrix}
$\,\,$ 1 $\,\,$ & $\,\,$4$\,\,$ & $\,\,$5$\,\,$ & $\,\,$8 $\,\,$ \\
$\,\,$ 1/4$\,\,$ & $\,\,$ 1 $\,\,$ & $\,\,$5$\,\,$ & $\,\,$3 $\,\,$ \\
$\,\,$ 1/5$\,\,$ & $\,\,$ 1/5$\,\,$ & $\,\,$ 1 $\,\,$ & $\,\,$1 $\,\,$ \\
$\,\,$ 1/8$\,\,$ & $\,\,$ 1/3$\,\,$ & $\,\,$ 1 $\,\,$ & $\,\,$ 1  $\,\,$ \\
\end{pmatrix},
\qquad
\lambda_{\max} =
4.1655,
\qquad
CR = 0.0624
\end{equation*}

\begin{equation*}
\mathbf{w}^{EM} =
\begin{pmatrix}
0.610049\\
0.239429\\
0.076921\\
\color{red} 0.073601\color{black}
\end{pmatrix}\end{equation*}
\begin{equation*}
\left[ \frac{{w}^{EM}_i}{{w}^{EM}_j} \right] =
\begin{pmatrix}
$\,\,$ 1 $\,\,$ & $\,\,$2.5479$\,\,$ & $\,\,$7.9308$\,\,$ & $\,\,$\color{red} 8.2886\color{black} $\,\,$ \\
$\,\,$0.3925$\,\,$ & $\,\,$ 1 $\,\,$ & $\,\,$3.1127$\,\,$ & $\,\,$\color{red} 3.2531\color{black}   $\,\,$ \\
$\,\,$0.1261$\,\,$ & $\,\,$0.3213$\,\,$ & $\,\,$ 1 $\,\,$ & $\,\,$\color{red} 1.0451\color{black}  $\,\,$ \\
$\,\,$\color{red} 0.1206\color{black} $\,\,$ & $\,\,$\color{red} 0.3074\color{black} $\,\,$ & $\,\,$\color{red} 0.9568\color{black} $\,\,$ & $\,\,$ 1  $\,\,$ \\
\end{pmatrix},
\end{equation*}

\begin{equation*}
\mathbf{w}^{\prime} =
\begin{pmatrix}
0.608433\\
0.238795\\
0.076717\\
0.076054
\end{pmatrix} =
0.997352\cdot
\begin{pmatrix}
0.610049\\
0.239429\\
0.076921\\
\color{gr} 0.076256\color{black}
\end{pmatrix},
\end{equation*}
\begin{equation*}
\left[ \frac{{w}^{\prime}_i}{{w}^{\prime}_j} \right] =
\begin{pmatrix}
$\,\,$ 1 $\,\,$ & $\,\,$2.5479$\,\,$ & $\,\,$7.9308$\,\,$ & $\,\,$\color{gr} \color{blue} 8\color{black} $\,\,$ \\
$\,\,$0.3925$\,\,$ & $\,\,$ 1 $\,\,$ & $\,\,$3.1127$\,\,$ & $\,\,$\color{gr} 3.1398\color{black}   $\,\,$ \\
$\,\,$0.1261$\,\,$ & $\,\,$0.3213$\,\,$ & $\,\,$ 1 $\,\,$ & $\,\,$\color{gr} 1.0087\color{black}  $\,\,$ \\
$\,\,$\color{gr} \color{blue}  1/8\color{black} $\,\,$ & $\,\,$\color{gr} 0.3185\color{black} $\,\,$ & $\,\,$\color{gr} 0.9914\color{black} $\,\,$ & $\,\,$ 1  $\,\,$ \\
\end{pmatrix},
\end{equation*}
\end{example}
\newpage
\begin{example}
\begin{equation*}
\mathbf{A} =
\begin{pmatrix}
$\,\,$ 1 $\,\,$ & $\,\,$4$\,\,$ & $\,\,$5$\,\,$ & $\,\,$8 $\,\,$ \\
$\,\,$ 1/4$\,\,$ & $\,\,$ 1 $\,\,$ & $\,\,$6$\,\,$ & $\,\,$3 $\,\,$ \\
$\,\,$ 1/5$\,\,$ & $\,\,$ 1/6$\,\,$ & $\,\,$ 1 $\,\,$ & $\,\,$1 $\,\,$ \\
$\,\,$ 1/8$\,\,$ & $\,\,$ 1/3$\,\,$ & $\,\,$ 1 $\,\,$ & $\,\,$ 1  $\,\,$ \\
\end{pmatrix},
\qquad
\lambda_{\max} =
4.2162,
\qquad
CR = 0.0815
\end{equation*}

\begin{equation*}
\mathbf{w}^{EM} =
\begin{pmatrix}
0.604412\\
0.250450\\
0.072995\\
\color{red} 0.072144\color{black}
\end{pmatrix}\end{equation*}
\begin{equation*}
\left[ \frac{{w}^{EM}_i}{{w}^{EM}_j} \right] =
\begin{pmatrix}
$\,\,$ 1 $\,\,$ & $\,\,$2.4133$\,\,$ & $\,\,$8.2802$\,\,$ & $\,\,$\color{red} 8.3779\color{black} $\,\,$ \\
$\,\,$0.4144$\,\,$ & $\,\,$ 1 $\,\,$ & $\,\,$3.4311$\,\,$ & $\,\,$\color{red} 3.4716\color{black}   $\,\,$ \\
$\,\,$0.1208$\,\,$ & $\,\,$0.2915$\,\,$ & $\,\,$ 1 $\,\,$ & $\,\,$\color{red} 1.0118\color{black}  $\,\,$ \\
$\,\,$\color{red} 0.1194\color{black} $\,\,$ & $\,\,$\color{red} 0.2881\color{black} $\,\,$ & $\,\,$\color{red} 0.9883\color{black} $\,\,$ & $\,\,$ 1  $\,\,$ \\
\end{pmatrix},
\end{equation*}

\begin{equation*}
\mathbf{w}^{\prime} =
\begin{pmatrix}
0.603898\\
0.250237\\
0.072933\\
0.072933
\end{pmatrix} =
0.999149\cdot
\begin{pmatrix}
0.604412\\
0.250450\\
0.072995\\
\color{gr} 0.072995\color{black}
\end{pmatrix},
\end{equation*}
\begin{equation*}
\left[ \frac{{w}^{\prime}_i}{{w}^{\prime}_j} \right] =
\begin{pmatrix}
$\,\,$ 1 $\,\,$ & $\,\,$2.4133$\,\,$ & $\,\,$8.2802$\,\,$ & $\,\,$\color{gr} 8.2802\color{black} $\,\,$ \\
$\,\,$0.4144$\,\,$ & $\,\,$ 1 $\,\,$ & $\,\,$3.4311$\,\,$ & $\,\,$\color{gr} 3.4311\color{black}   $\,\,$ \\
$\,\,$0.1208$\,\,$ & $\,\,$0.2915$\,\,$ & $\,\,$ 1 $\,\,$ & $\,\,$\color{gr} \color{blue} 1\color{black}  $\,\,$ \\
$\,\,$\color{gr} 0.1208\color{black} $\,\,$ & $\,\,$\color{gr} 0.2915\color{black} $\,\,$ & $\,\,$\color{gr} \color{blue} 1\color{black} $\,\,$ & $\,\,$ 1  $\,\,$ \\
\end{pmatrix},
\end{equation*}
\end{example}
\newpage
\begin{example}
\begin{equation*}
\mathbf{A} =
\begin{pmatrix}
$\,\,$ 1 $\,\,$ & $\,\,$4$\,\,$ & $\,\,$5$\,\,$ & $\,\,$8 $\,\,$ \\
$\,\,$ 1/4$\,\,$ & $\,\,$ 1 $\,\,$ & $\,\,$7$\,\,$ & $\,\,$4 $\,\,$ \\
$\,\,$ 1/5$\,\,$ & $\,\,$ 1/7$\,\,$ & $\,\,$ 1 $\,\,$ & $\,\,$1 $\,\,$ \\
$\,\,$ 1/8$\,\,$ & $\,\,$ 1/4$\,\,$ & $\,\,$ 1 $\,\,$ & $\,\,$ 1  $\,\,$ \\
\end{pmatrix},
\qquad
\lambda_{\max} =
4.2610,
\qquad
CR = 0.0984
\end{equation*}

\begin{equation*}
\mathbf{w}^{EM} =
\begin{pmatrix}
0.595856\\
0.271346\\
0.068232\\
\color{red} 0.064567\color{black}
\end{pmatrix}\end{equation*}
\begin{equation*}
\left[ \frac{{w}^{EM}_i}{{w}^{EM}_j} \right] =
\begin{pmatrix}
$\,\,$ 1 $\,\,$ & $\,\,$2.1959$\,\,$ & $\,\,$8.7329$\,\,$ & $\,\,$\color{red} 9.2286\color{black} $\,\,$ \\
$\,\,$0.4554$\,\,$ & $\,\,$ 1 $\,\,$ & $\,\,$3.9768$\,\,$ & $\,\,$\color{red} 4.2026\color{black}   $\,\,$ \\
$\,\,$0.1145$\,\,$ & $\,\,$0.2515$\,\,$ & $\,\,$ 1 $\,\,$ & $\,\,$\color{red} 1.0568\color{black}  $\,\,$ \\
$\,\,$\color{red} 0.1084\color{black} $\,\,$ & $\,\,$\color{red} 0.2379\color{black} $\,\,$ & $\,\,$\color{red} 0.9463\color{black} $\,\,$ & $\,\,$ 1  $\,\,$ \\
\end{pmatrix},
\end{equation*}

\begin{equation*}
\mathbf{w}^{\prime} =
\begin{pmatrix}
0.593914\\
0.270461\\
0.068009\\
0.067615
\end{pmatrix} =
0.996741\cdot
\begin{pmatrix}
0.595856\\
0.271346\\
0.068232\\
\color{gr} 0.067836\color{black}
\end{pmatrix},
\end{equation*}
\begin{equation*}
\left[ \frac{{w}^{\prime}_i}{{w}^{\prime}_j} \right] =
\begin{pmatrix}
$\,\,$ 1 $\,\,$ & $\,\,$2.1959$\,\,$ & $\,\,$8.7329$\,\,$ & $\,\,$\color{gr} 8.7837\color{black} $\,\,$ \\
$\,\,$0.4554$\,\,$ & $\,\,$ 1 $\,\,$ & $\,\,$3.9768$\,\,$ & $\,\,$\color{gr} \color{blue} 4\color{black}   $\,\,$ \\
$\,\,$0.1145$\,\,$ & $\,\,$0.2515$\,\,$ & $\,\,$ 1 $\,\,$ & $\,\,$\color{gr} 1.0058\color{black}  $\,\,$ \\
$\,\,$\color{gr} 0.1138\color{black} $\,\,$ & $\,\,$\color{gr} \color{blue}  1/4\color{black} $\,\,$ & $\,\,$\color{gr} 0.9942\color{black} $\,\,$ & $\,\,$ 1  $\,\,$ \\
\end{pmatrix},
\end{equation*}
\end{example}
\newpage
\begin{example}
\begin{equation*}
\mathbf{A} =
\begin{pmatrix}
$\,\,$ 1 $\,\,$ & $\,\,$4$\,\,$ & $\,\,$5$\,\,$ & $\,\,$9 $\,\,$ \\
$\,\,$ 1/4$\,\,$ & $\,\,$ 1 $\,\,$ & $\,\,$2$\,\,$ & $\,\,$9 $\,\,$ \\
$\,\,$ 1/5$\,\,$ & $\,\,$ 1/2$\,\,$ & $\,\,$ 1 $\,\,$ & $\,\,$3 $\,\,$ \\
$\,\,$ 1/9$\,\,$ & $\,\,$ 1/9$\,\,$ & $\,\,$ 1/3$\,\,$ & $\,\,$ 1  $\,\,$ \\
\end{pmatrix},
\qquad
\lambda_{\max} =
4.1655,
\qquad
CR = 0.0624
\end{equation*}

\begin{equation*}
\mathbf{w}^{EM} =
\begin{pmatrix}
0.603471\\
0.239301\\
\color{red} 0.115485\color{black} \\
0.041742
\end{pmatrix}\end{equation*}
\begin{equation*}
\left[ \frac{{w}^{EM}_i}{{w}^{EM}_j} \right] =
\begin{pmatrix}
$\,\,$ 1 $\,\,$ & $\,\,$2.5218$\,\,$ & $\,\,$\color{red} 5.2255\color{black} $\,\,$ & $\,\,$14.4571$\,\,$ \\
$\,\,$0.3965$\,\,$ & $\,\,$ 1 $\,\,$ & $\,\,$\color{red} 2.0721\color{black} $\,\,$ & $\,\,$5.7328  $\,\,$ \\
$\,\,$\color{red} 0.1914\color{black} $\,\,$ & $\,\,$\color{red} 0.4826\color{black} $\,\,$ & $\,\,$ 1 $\,\,$ & $\,\,$\color{red} 2.7666\color{black}  $\,\,$ \\
$\,\,$0.0692$\,\,$ & $\,\,$0.1744$\,\,$ & $\,\,$\color{red} 0.3615\color{black} $\,\,$ & $\,\,$ 1  $\,\,$ \\
\end{pmatrix},
\end{equation*}

\begin{equation*}
\mathbf{w}^{\prime} =
\begin{pmatrix}
0.600968\\
0.238309\\
0.119154\\
0.041569
\end{pmatrix} =
0.995851\cdot
\begin{pmatrix}
0.603471\\
0.239301\\
\color{gr} 0.119651\color{black} \\
0.041742
\end{pmatrix},
\end{equation*}
\begin{equation*}
\left[ \frac{{w}^{\prime}_i}{{w}^{\prime}_j} \right] =
\begin{pmatrix}
$\,\,$ 1 $\,\,$ & $\,\,$2.5218$\,\,$ & $\,\,$\color{gr} 5.0436\color{black} $\,\,$ & $\,\,$14.4571$\,\,$ \\
$\,\,$0.3965$\,\,$ & $\,\,$ 1 $\,\,$ & $\,\,$\color{gr} \color{blue} 2\color{black} $\,\,$ & $\,\,$5.7328  $\,\,$ \\
$\,\,$\color{gr} 0.1983\color{black} $\,\,$ & $\,\,$\color{gr} \color{blue}  1/2\color{black} $\,\,$ & $\,\,$ 1 $\,\,$ & $\,\,$\color{gr} 2.8664\color{black}  $\,\,$ \\
$\,\,$0.0692$\,\,$ & $\,\,$0.1744$\,\,$ & $\,\,$\color{gr} 0.3489\color{black} $\,\,$ & $\,\,$ 1  $\,\,$ \\
\end{pmatrix},
\end{equation*}
\end{example}
\newpage
\begin{example}
\begin{equation*}
\mathbf{A} =
\begin{pmatrix}
$\,\,$ 1 $\,\,$ & $\,\,$4$\,\,$ & $\,\,$5$\,\,$ & $\,\,$9 $\,\,$ \\
$\,\,$ 1/4$\,\,$ & $\,\,$ 1 $\,\,$ & $\,\,$6$\,\,$ & $\,\,$4 $\,\,$ \\
$\,\,$ 1/5$\,\,$ & $\,\,$ 1/6$\,\,$ & $\,\,$ 1 $\,\,$ & $\,\,$1 $\,\,$ \\
$\,\,$ 1/9$\,\,$ & $\,\,$ 1/4$\,\,$ & $\,\,$ 1 $\,\,$ & $\,\,$ 1  $\,\,$ \\
\end{pmatrix},
\qquad
\lambda_{\max} =
4.2146,
\qquad
CR = 0.0809
\end{equation*}

\begin{equation*}
\mathbf{w}^{EM} =
\begin{pmatrix}
0.607993\\
0.258046\\
0.070841\\
\color{red} 0.063120\color{black}
\end{pmatrix}\end{equation*}
\begin{equation*}
\left[ \frac{{w}^{EM}_i}{{w}^{EM}_j} \right] =
\begin{pmatrix}
$\,\,$ 1 $\,\,$ & $\,\,$2.3561$\,\,$ & $\,\,$8.5825$\,\,$ & $\,\,$\color{red} 9.6323\color{black} $\,\,$ \\
$\,\,$0.4244$\,\,$ & $\,\,$ 1 $\,\,$ & $\,\,$3.6426$\,\,$ & $\,\,$\color{red} 4.0882\color{black}   $\,\,$ \\
$\,\,$0.1165$\,\,$ & $\,\,$0.2745$\,\,$ & $\,\,$ 1 $\,\,$ & $\,\,$\color{red} 1.1223\color{black}  $\,\,$ \\
$\,\,$\color{red} 0.1038\color{black} $\,\,$ & $\,\,$\color{red} 0.2446\color{black} $\,\,$ & $\,\,$\color{red} 0.8910\color{black} $\,\,$ & $\,\,$ 1  $\,\,$ \\
\end{pmatrix},
\end{equation*}

\begin{equation*}
\mathbf{w}^{\prime} =
\begin{pmatrix}
0.607148\\
0.257688\\
0.070742\\
0.064422
\end{pmatrix} =
0.998610\cdot
\begin{pmatrix}
0.607993\\
0.258046\\
0.070841\\
\color{gr} 0.064512\color{black}
\end{pmatrix},
\end{equation*}
\begin{equation*}
\left[ \frac{{w}^{\prime}_i}{{w}^{\prime}_j} \right] =
\begin{pmatrix}
$\,\,$ 1 $\,\,$ & $\,\,$2.3561$\,\,$ & $\,\,$8.5825$\,\,$ & $\,\,$\color{gr} 9.4246\color{black} $\,\,$ \\
$\,\,$0.4244$\,\,$ & $\,\,$ 1 $\,\,$ & $\,\,$3.6426$\,\,$ & $\,\,$\color{gr} \color{blue} 4\color{black}   $\,\,$ \\
$\,\,$0.1165$\,\,$ & $\,\,$0.2745$\,\,$ & $\,\,$ 1 $\,\,$ & $\,\,$\color{gr} 1.0981\color{black}  $\,\,$ \\
$\,\,$\color{gr} 0.1061\color{black} $\,\,$ & $\,\,$\color{gr} \color{blue}  1/4\color{black} $\,\,$ & $\,\,$\color{gr} 0.9107\color{black} $\,\,$ & $\,\,$ 1  $\,\,$ \\
\end{pmatrix},
\end{equation*}
\end{example}
\newpage
\begin{example}
\begin{equation*}
\mathbf{A} =
\begin{pmatrix}
$\,\,$ 1 $\,\,$ & $\,\,$4$\,\,$ & $\,\,$5$\,\,$ & $\,\,$9 $\,\,$ \\
$\,\,$ 1/4$\,\,$ & $\,\,$ 1 $\,\,$ & $\,\,$7$\,\,$ & $\,\,$4 $\,\,$ \\
$\,\,$ 1/5$\,\,$ & $\,\,$ 1/7$\,\,$ & $\,\,$ 1 $\,\,$ & $\,\,$1 $\,\,$ \\
$\,\,$ 1/9$\,\,$ & $\,\,$ 1/4$\,\,$ & $\,\,$ 1 $\,\,$ & $\,\,$ 1  $\,\,$ \\
\end{pmatrix},
\qquad
\lambda_{\max} =
4.2594,
\qquad
CR = 0.0978
\end{equation*}

\begin{equation*}
\mathbf{w}^{EM} =
\begin{pmatrix}
0.602932\\
0.267529\\
0.067696\\
\color{red} 0.061843\color{black}
\end{pmatrix}\end{equation*}
\begin{equation*}
\left[ \frac{{w}^{EM}_i}{{w}^{EM}_j} \right] =
\begin{pmatrix}
$\,\,$ 1 $\,\,$ & $\,\,$2.2537$\,\,$ & $\,\,$8.9064$\,\,$ & $\,\,$\color{red} 9.7494\color{black} $\,\,$ \\
$\,\,$0.4437$\,\,$ & $\,\,$ 1 $\,\,$ & $\,\,$3.9519$\,\,$ & $\,\,$\color{red} 4.3259\color{black}   $\,\,$ \\
$\,\,$0.1123$\,\,$ & $\,\,$0.2530$\,\,$ & $\,\,$ 1 $\,\,$ & $\,\,$\color{red} 1.0946\color{black}  $\,\,$ \\
$\,\,$\color{red} 0.1026\color{black} $\,\,$ & $\,\,$\color{red} 0.2312\color{black} $\,\,$ & $\,\,$\color{red} 0.9135\color{black} $\,\,$ & $\,\,$ 1  $\,\,$ \\
\end{pmatrix},
\end{equation*}

\begin{equation*}
\mathbf{w}^{\prime} =
\begin{pmatrix}
0.599909\\
0.266188\\
0.067357\\
0.066547
\end{pmatrix} =
0.994986\cdot
\begin{pmatrix}
0.602932\\
0.267529\\
0.067696\\
\color{gr} 0.066882\color{black}
\end{pmatrix},
\end{equation*}
\begin{equation*}
\left[ \frac{{w}^{\prime}_i}{{w}^{\prime}_j} \right] =
\begin{pmatrix}
$\,\,$ 1 $\,\,$ & $\,\,$2.2537$\,\,$ & $\,\,$8.9064$\,\,$ & $\,\,$\color{gr} 9.0148\color{black} $\,\,$ \\
$\,\,$0.4437$\,\,$ & $\,\,$ 1 $\,\,$ & $\,\,$3.9519$\,\,$ & $\,\,$\color{gr} \color{blue} 4\color{black}   $\,\,$ \\
$\,\,$0.1123$\,\,$ & $\,\,$0.2530$\,\,$ & $\,\,$ 1 $\,\,$ & $\,\,$\color{gr} 1.0122\color{black}  $\,\,$ \\
$\,\,$\color{gr} 0.1109\color{black} $\,\,$ & $\,\,$\color{gr} \color{blue}  1/4\color{black} $\,\,$ & $\,\,$\color{gr} 0.9880\color{black} $\,\,$ & $\,\,$ 1  $\,\,$ \\
\end{pmatrix},
\end{equation*}
\end{example}
\newpage
\begin{example}
\begin{equation*}
\mathbf{A} =
\begin{pmatrix}
$\,\,$ 1 $\,\,$ & $\,\,$4$\,\,$ & $\,\,$6$\,\,$ & $\,\,$8 $\,\,$ \\
$\,\,$ 1/4$\,\,$ & $\,\,$ 1 $\,\,$ & $\,\,$2$\,\,$ & $\,\,$6 $\,\,$ \\
$\,\,$ 1/6$\,\,$ & $\,\,$ 1/2$\,\,$ & $\,\,$ 1 $\,\,$ & $\,\,$2 $\,\,$ \\
$\,\,$ 1/8$\,\,$ & $\,\,$ 1/6$\,\,$ & $\,\,$ 1/2$\,\,$ & $\,\,$ 1  $\,\,$ \\
\end{pmatrix},
\qquad
\lambda_{\max} =
4.1031,
\qquad
CR = 0.0389
\end{equation*}

\begin{equation*}
\mathbf{w}^{EM} =
\begin{pmatrix}
0.622450\\
0.220467\\
\color{red} 0.103494\color{black} \\
0.053590
\end{pmatrix}\end{equation*}
\begin{equation*}
\left[ \frac{{w}^{EM}_i}{{w}^{EM}_j} \right] =
\begin{pmatrix}
$\,\,$ 1 $\,\,$ & $\,\,$2.8233$\,\,$ & $\,\,$\color{red} 6.0144\color{black} $\,\,$ & $\,\,$11.6150$\,\,$ \\
$\,\,$0.3542$\,\,$ & $\,\,$ 1 $\,\,$ & $\,\,$\color{red} 2.1302\color{black} $\,\,$ & $\,\,$4.1140  $\,\,$ \\
$\,\,$\color{red} 0.1663\color{black} $\,\,$ & $\,\,$\color{red} 0.4694\color{black} $\,\,$ & $\,\,$ 1 $\,\,$ & $\,\,$\color{red} 1.9312\color{black}  $\,\,$ \\
$\,\,$0.0861$\,\,$ & $\,\,$0.2431$\,\,$ & $\,\,$\color{red} 0.5178\color{black} $\,\,$ & $\,\,$ 1  $\,\,$ \\
\end{pmatrix},
\end{equation*}

\begin{equation*}
\mathbf{w}^{\prime} =
\begin{pmatrix}
0.622295\\
0.220412\\
0.103716\\
0.053577
\end{pmatrix} =
0.999752\cdot
\begin{pmatrix}
0.622450\\
0.220467\\
\color{gr} 0.103742\color{black} \\
0.053590
\end{pmatrix},
\end{equation*}
\begin{equation*}
\left[ \frac{{w}^{\prime}_i}{{w}^{\prime}_j} \right] =
\begin{pmatrix}
$\,\,$ 1 $\,\,$ & $\,\,$2.8233$\,\,$ & $\,\,$\color{gr} \color{blue} 6\color{black} $\,\,$ & $\,\,$11.6150$\,\,$ \\
$\,\,$0.3542$\,\,$ & $\,\,$ 1 $\,\,$ & $\,\,$\color{gr} 2.1252\color{black} $\,\,$ & $\,\,$4.1140  $\,\,$ \\
$\,\,$\color{gr} \color{blue}  1/6\color{black} $\,\,$ & $\,\,$\color{gr} 0.4706\color{black} $\,\,$ & $\,\,$ 1 $\,\,$ & $\,\,$\color{gr} 1.9358\color{black}  $\,\,$ \\
$\,\,$0.0861$\,\,$ & $\,\,$0.2431$\,\,$ & $\,\,$\color{gr} 0.5166\color{black} $\,\,$ & $\,\,$ 1  $\,\,$ \\
\end{pmatrix},
\end{equation*}
\end{example}
\newpage
\begin{example}
\begin{equation*}
\mathbf{A} =
\begin{pmatrix}
$\,\,$ 1 $\,\,$ & $\,\,$4$\,\,$ & $\,\,$6$\,\,$ & $\,\,$8 $\,\,$ \\
$\,\,$ 1/4$\,\,$ & $\,\,$ 1 $\,\,$ & $\,\,$2$\,\,$ & $\,\,$7 $\,\,$ \\
$\,\,$ 1/6$\,\,$ & $\,\,$ 1/2$\,\,$ & $\,\,$ 1 $\,\,$ & $\,\,$2 $\,\,$ \\
$\,\,$ 1/8$\,\,$ & $\,\,$ 1/7$\,\,$ & $\,\,$ 1/2$\,\,$ & $\,\,$ 1  $\,\,$ \\
\end{pmatrix},
\qquad
\lambda_{\max} =
4.1365,
\qquad
CR = 0.0515
\end{equation*}

\begin{equation*}
\mathbf{w}^{EM} =
\begin{pmatrix}
0.617850\\
0.228815\\
\color{red} 0.102025\color{black} \\
0.051309
\end{pmatrix}\end{equation*}
\begin{equation*}
\left[ \frac{{w}^{EM}_i}{{w}^{EM}_j} \right] =
\begin{pmatrix}
$\,\,$ 1 $\,\,$ & $\,\,$2.7002$\,\,$ & $\,\,$\color{red} 6.0559\color{black} $\,\,$ & $\,\,$12.0417$\,\,$ \\
$\,\,$0.3703$\,\,$ & $\,\,$ 1 $\,\,$ & $\,\,$\color{red} 2.2427\color{black} $\,\,$ & $\,\,$4.4595  $\,\,$ \\
$\,\,$\color{red} 0.1651\color{black} $\,\,$ & $\,\,$\color{red} 0.4459\color{black} $\,\,$ & $\,\,$ 1 $\,\,$ & $\,\,$\color{red} 1.9884\color{black}  $\,\,$ \\
$\,\,$0.0830$\,\,$ & $\,\,$0.2242$\,\,$ & $\,\,$\color{red} 0.5029\color{black} $\,\,$ & $\,\,$ 1  $\,\,$ \\
\end{pmatrix},
\end{equation*}

\begin{equation*}
\mathbf{w}^{\prime} =
\begin{pmatrix}
0.617484\\
0.228679\\
0.102558\\
0.051279
\end{pmatrix} =
0.999407\cdot
\begin{pmatrix}
0.617850\\
0.228815\\
\color{gr} 0.102619\color{black} \\
0.051309
\end{pmatrix},
\end{equation*}
\begin{equation*}
\left[ \frac{{w}^{\prime}_i}{{w}^{\prime}_j} \right] =
\begin{pmatrix}
$\,\,$ 1 $\,\,$ & $\,\,$2.7002$\,\,$ & $\,\,$\color{gr} 6.0208\color{black} $\,\,$ & $\,\,$12.0417$\,\,$ \\
$\,\,$0.3703$\,\,$ & $\,\,$ 1 $\,\,$ & $\,\,$\color{gr} 2.2298\color{black} $\,\,$ & $\,\,$4.4595  $\,\,$ \\
$\,\,$\color{gr} 0.1661\color{black} $\,\,$ & $\,\,$\color{gr} 0.4485\color{black} $\,\,$ & $\,\,$ 1 $\,\,$ & $\,\,$\color{gr} \color{blue} 2\color{black}  $\,\,$ \\
$\,\,$0.0830$\,\,$ & $\,\,$0.2242$\,\,$ & $\,\,$\color{gr} \color{blue}  1/2\color{black} $\,\,$ & $\,\,$ 1  $\,\,$ \\
\end{pmatrix},
\end{equation*}
\end{example}
\newpage
\begin{example}
\begin{equation*}
\mathbf{A} =
\begin{pmatrix}
$\,\,$ 1 $\,\,$ & $\,\,$4$\,\,$ & $\,\,$6$\,\,$ & $\,\,$9 $\,\,$ \\
$\,\,$ 1/4$\,\,$ & $\,\,$ 1 $\,\,$ & $\,\,$2$\,\,$ & $\,\,$5 $\,\,$ \\
$\,\,$ 1/6$\,\,$ & $\,\,$ 1/2$\,\,$ & $\,\,$ 1 $\,\,$ & $\,\,$2 $\,\,$ \\
$\,\,$ 1/9$\,\,$ & $\,\,$ 1/5$\,\,$ & $\,\,$ 1/2$\,\,$ & $\,\,$ 1  $\,\,$ \\
\end{pmatrix},
\qquad
\lambda_{\max} =
4.0539,
\qquad
CR = 0.0203
\end{equation*}

\begin{equation*}
\mathbf{w}^{EM} =
\begin{pmatrix}
0.634574\\
0.207882\\
\color{red} 0.103840\color{black} \\
0.053704
\end{pmatrix}\end{equation*}
\begin{equation*}
\left[ \frac{{w}^{EM}_i}{{w}^{EM}_j} \right] =
\begin{pmatrix}
$\,\,$ 1 $\,\,$ & $\,\,$3.0526$\,\,$ & $\,\,$\color{red} 6.1111\color{black} $\,\,$ & $\,\,$11.8162$\,\,$ \\
$\,\,$0.3276$\,\,$ & $\,\,$ 1 $\,\,$ & $\,\,$\color{red} 2.0020\color{black} $\,\,$ & $\,\,$3.8709  $\,\,$ \\
$\,\,$\color{red} 0.1636\color{black} $\,\,$ & $\,\,$\color{red} 0.4995\color{black} $\,\,$ & $\,\,$ 1 $\,\,$ & $\,\,$\color{red} 1.9336\color{black}  $\,\,$ \\
$\,\,$0.0846$\,\,$ & $\,\,$0.2583$\,\,$ & $\,\,$\color{red} 0.5172\color{black} $\,\,$ & $\,\,$ 1  $\,\,$ \\
\end{pmatrix},
\end{equation*}

\begin{equation*}
\mathbf{w}^{\prime} =
\begin{pmatrix}
0.634510\\
0.207861\\
0.103930\\
0.053699
\end{pmatrix} =
0.999899\cdot
\begin{pmatrix}
0.634574\\
0.207882\\
\color{gr} 0.103941\color{black} \\
0.053704
\end{pmatrix},
\end{equation*}
\begin{equation*}
\left[ \frac{{w}^{\prime}_i}{{w}^{\prime}_j} \right] =
\begin{pmatrix}
$\,\,$ 1 $\,\,$ & $\,\,$3.0526$\,\,$ & $\,\,$\color{gr} 6.1051\color{black} $\,\,$ & $\,\,$11.8162$\,\,$ \\
$\,\,$0.3276$\,\,$ & $\,\,$ 1 $\,\,$ & $\,\,$\color{gr} \color{blue} 2\color{black} $\,\,$ & $\,\,$3.8709  $\,\,$ \\
$\,\,$\color{gr} 0.1638\color{black} $\,\,$ & $\,\,$\color{gr} \color{blue}  1/2\color{black} $\,\,$ & $\,\,$ 1 $\,\,$ & $\,\,$\color{gr} 1.9354\color{black}  $\,\,$ \\
$\,\,$0.0846$\,\,$ & $\,\,$0.2583$\,\,$ & $\,\,$\color{gr} 0.5167\color{black} $\,\,$ & $\,\,$ 1  $\,\,$ \\
\end{pmatrix},
\end{equation*}
\end{example}
\newpage
\begin{example}
\begin{equation*}
\mathbf{A} =
\begin{pmatrix}
$\,\,$ 1 $\,\,$ & $\,\,$4$\,\,$ & $\,\,$6$\,\,$ & $\,\,$9 $\,\,$ \\
$\,\,$ 1/4$\,\,$ & $\,\,$ 1 $\,\,$ & $\,\,$5$\,\,$ & $\,\,$3 $\,\,$ \\
$\,\,$ 1/6$\,\,$ & $\,\,$ 1/5$\,\,$ & $\,\,$ 1 $\,\,$ & $\,\,$1 $\,\,$ \\
$\,\,$ 1/9$\,\,$ & $\,\,$ 1/3$\,\,$ & $\,\,$ 1 $\,\,$ & $\,\,$ 1  $\,\,$ \\
\end{pmatrix},
\qquad
\lambda_{\max} =
4.1252,
\qquad
CR = 0.0472
\end{equation*}

\begin{equation*}
\mathbf{w}^{EM} =
\begin{pmatrix}
0.629960\\
0.229985\\
0.070552\\
\color{red} 0.069502\color{black}
\end{pmatrix}\end{equation*}
\begin{equation*}
\left[ \frac{{w}^{EM}_i}{{w}^{EM}_j} \right] =
\begin{pmatrix}
$\,\,$ 1 $\,\,$ & $\,\,$2.7391$\,\,$ & $\,\,$8.9290$\,\,$ & $\,\,$\color{red} 9.0639\color{black} $\,\,$ \\
$\,\,$0.3651$\,\,$ & $\,\,$ 1 $\,\,$ & $\,\,$3.2598$\,\,$ & $\,\,$\color{red} 3.3090\color{black}   $\,\,$ \\
$\,\,$0.1120$\,\,$ & $\,\,$0.3068$\,\,$ & $\,\,$ 1 $\,\,$ & $\,\,$\color{red} 1.0151\color{black}  $\,\,$ \\
$\,\,$\color{red} 0.1103\color{black} $\,\,$ & $\,\,$\color{red} 0.3022\color{black} $\,\,$ & $\,\,$\color{red} 0.9851\color{black} $\,\,$ & $\,\,$ 1  $\,\,$ \\
\end{pmatrix},
\end{equation*}

\begin{equation*}
\mathbf{w}^{\prime} =
\begin{pmatrix}
0.629650\\
0.229872\\
0.070518\\
0.069961
\end{pmatrix} =
0.999507\cdot
\begin{pmatrix}
0.629960\\
0.229985\\
0.070552\\
\color{gr} 0.069996\color{black}
\end{pmatrix},
\end{equation*}
\begin{equation*}
\left[ \frac{{w}^{\prime}_i}{{w}^{\prime}_j} \right] =
\begin{pmatrix}
$\,\,$ 1 $\,\,$ & $\,\,$2.7391$\,\,$ & $\,\,$8.9290$\,\,$ & $\,\,$\color{gr} \color{blue} 9\color{black} $\,\,$ \\
$\,\,$0.3651$\,\,$ & $\,\,$ 1 $\,\,$ & $\,\,$3.2598$\,\,$ & $\,\,$\color{gr} 3.2857\color{black}   $\,\,$ \\
$\,\,$0.1120$\,\,$ & $\,\,$0.3068$\,\,$ & $\,\,$ 1 $\,\,$ & $\,\,$\color{gr} 1.0080\color{black}  $\,\,$ \\
$\,\,$\color{gr} \color{blue}  1/9\color{black} $\,\,$ & $\,\,$\color{gr} 0.3043\color{black} $\,\,$ & $\,\,$\color{gr} 0.9921\color{black} $\,\,$ & $\,\,$ 1  $\,\,$ \\
\end{pmatrix},
\end{equation*}
\end{example}
\newpage
\begin{example}
\begin{equation*}
\mathbf{A} =
\begin{pmatrix}
$\,\,$ 1 $\,\,$ & $\,\,$4$\,\,$ & $\,\,$6$\,\,$ & $\,\,$9 $\,\,$ \\
$\,\,$ 1/4$\,\,$ & $\,\,$ 1 $\,\,$ & $\,\,$6$\,\,$ & $\,\,$4 $\,\,$ \\
$\,\,$ 1/6$\,\,$ & $\,\,$ 1/6$\,\,$ & $\,\,$ 1 $\,\,$ & $\,\,$1 $\,\,$ \\
$\,\,$ 1/9$\,\,$ & $\,\,$ 1/4$\,\,$ & $\,\,$ 1 $\,\,$ & $\,\,$ 1  $\,\,$ \\
\end{pmatrix},
\qquad
\lambda_{\max} =
4.1664,
\qquad
CR = 0.0627
\end{equation*}

\begin{equation*}
\mathbf{w}^{EM} =
\begin{pmatrix}
0.619984\\
0.252045\\
0.065598\\
\color{red} 0.062373\color{black}
\end{pmatrix}\end{equation*}
\begin{equation*}
\left[ \frac{{w}^{EM}_i}{{w}^{EM}_j} \right] =
\begin{pmatrix}
$\,\,$ 1 $\,\,$ & $\,\,$2.4598$\,\,$ & $\,\,$9.4512$\,\,$ & $\,\,$\color{red} 9.9400\color{black} $\,\,$ \\
$\,\,$0.4065$\,\,$ & $\,\,$ 1 $\,\,$ & $\,\,$3.8422$\,\,$ & $\,\,$\color{red} 4.0410\color{black}   $\,\,$ \\
$\,\,$0.1058$\,\,$ & $\,\,$0.2603$\,\,$ & $\,\,$ 1 $\,\,$ & $\,\,$\color{red} 1.0517\color{black}  $\,\,$ \\
$\,\,$\color{red} 0.1006\color{black} $\,\,$ & $\,\,$\color{red} 0.2475\color{black} $\,\,$ & $\,\,$\color{red} 0.9508\color{black} $\,\,$ & $\,\,$ 1  $\,\,$ \\
\end{pmatrix},
\end{equation*}

\begin{equation*}
\mathbf{w}^{\prime} =
\begin{pmatrix}
0.619589\\
0.251884\\
0.065556\\
0.062971
\end{pmatrix} =
0.999362\cdot
\begin{pmatrix}
0.619984\\
0.252045\\
0.065598\\
\color{gr} 0.063011\color{black}
\end{pmatrix},
\end{equation*}
\begin{equation*}
\left[ \frac{{w}^{\prime}_i}{{w}^{\prime}_j} \right] =
\begin{pmatrix}
$\,\,$ 1 $\,\,$ & $\,\,$2.4598$\,\,$ & $\,\,$9.4512$\,\,$ & $\,\,$\color{gr} 9.8393\color{black} $\,\,$ \\
$\,\,$0.4065$\,\,$ & $\,\,$ 1 $\,\,$ & $\,\,$3.8422$\,\,$ & $\,\,$\color{gr} \color{blue} 4\color{black}   $\,\,$ \\
$\,\,$0.1058$\,\,$ & $\,\,$0.2603$\,\,$ & $\,\,$ 1 $\,\,$ & $\,\,$\color{gr} 1.0411\color{black}  $\,\,$ \\
$\,\,$\color{gr} 0.1016\color{black} $\,\,$ & $\,\,$\color{gr} \color{blue}  1/4\color{black} $\,\,$ & $\,\,$\color{gr} 0.9606\color{black} $\,\,$ & $\,\,$ 1  $\,\,$ \\
\end{pmatrix},
\end{equation*}
\end{example}
\newpage
\begin{example}
\begin{equation*}
\mathbf{A} =
\begin{pmatrix}
$\,\,$ 1 $\,\,$ & $\,\,$4$\,\,$ & $\,\,$6$\,\,$ & $\,\,$9 $\,\,$ \\
$\,\,$ 1/4$\,\,$ & $\,\,$ 1 $\,\,$ & $\,\,$7$\,\,$ & $\,\,$4 $\,\,$ \\
$\,\,$ 1/6$\,\,$ & $\,\,$ 1/7$\,\,$ & $\,\,$ 1 $\,\,$ & $\,\,$1 $\,\,$ \\
$\,\,$ 1/9$\,\,$ & $\,\,$ 1/4$\,\,$ & $\,\,$ 1 $\,\,$ & $\,\,$ 1  $\,\,$ \\
\end{pmatrix},
\qquad
\lambda_{\max} =
4.2065,
\qquad
CR = 0.0779
\end{equation*}

\begin{equation*}
\mathbf{w}^{EM} =
\begin{pmatrix}
0.614924\\
0.261166\\
0.062690\\
\color{red} 0.061220\color{black}
\end{pmatrix}\end{equation*}
\begin{equation*}
\left[ \frac{{w}^{EM}_i}{{w}^{EM}_j} \right] =
\begin{pmatrix}
$\,\,$ 1 $\,\,$ & $\,\,$2.3545$\,\,$ & $\,\,$9.8090$\,\,$ & $\,\,$\color{red} 10.0444\color{black} $\,\,$ \\
$\,\,$0.4247$\,\,$ & $\,\,$ 1 $\,\,$ & $\,\,$4.1660$\,\,$ & $\,\,$\color{red} 4.2660\color{black}   $\,\,$ \\
$\,\,$0.1019$\,\,$ & $\,\,$0.2400$\,\,$ & $\,\,$ 1 $\,\,$ & $\,\,$\color{red} 1.0240\color{black}  $\,\,$ \\
$\,\,$\color{red} 0.0996\color{black} $\,\,$ & $\,\,$\color{red} 0.2344\color{black} $\,\,$ & $\,\,$\color{red} 0.9766\color{black} $\,\,$ & $\,\,$ 1  $\,\,$ \\
\end{pmatrix},
\end{equation*}

\begin{equation*}
\mathbf{w}^{\prime} =
\begin{pmatrix}
0.614022\\
0.260783\\
0.062598\\
0.062598
\end{pmatrix} =
0.998533\cdot
\begin{pmatrix}
0.614924\\
0.261166\\
0.062690\\
\color{gr} 0.062690\color{black}
\end{pmatrix},
\end{equation*}
\begin{equation*}
\left[ \frac{{w}^{\prime}_i}{{w}^{\prime}_j} \right] =
\begin{pmatrix}
$\,\,$ 1 $\,\,$ & $\,\,$2.3545$\,\,$ & $\,\,$9.8090$\,\,$ & $\,\,$\color{gr} 9.8090\color{black} $\,\,$ \\
$\,\,$0.4247$\,\,$ & $\,\,$ 1 $\,\,$ & $\,\,$4.1660$\,\,$ & $\,\,$\color{gr} 4.1660\color{black}   $\,\,$ \\
$\,\,$0.1019$\,\,$ & $\,\,$0.2400$\,\,$ & $\,\,$ 1 $\,\,$ & $\,\,$\color{gr} \color{blue} 1\color{black}  $\,\,$ \\
$\,\,$\color{gr} 0.1019\color{black} $\,\,$ & $\,\,$\color{gr} 0.2400\color{black} $\,\,$ & $\,\,$\color{gr} \color{blue} 1\color{black} $\,\,$ & $\,\,$ 1  $\,\,$ \\
\end{pmatrix},
\end{equation*}
\end{example}
\newpage
\begin{example}
\begin{equation*}
\mathbf{A} =
\begin{pmatrix}
$\,\,$ 1 $\,\,$ & $\,\,$4$\,\,$ & $\,\,$6$\,\,$ & $\,\,$9 $\,\,$ \\
$\,\,$ 1/4$\,\,$ & $\,\,$ 1 $\,\,$ & $\,\,$8$\,\,$ & $\,\,$4 $\,\,$ \\
$\,\,$ 1/6$\,\,$ & $\,\,$ 1/8$\,\,$ & $\,\,$ 1 $\,\,$ & $\,\,$1 $\,\,$ \\
$\,\,$ 1/9$\,\,$ & $\,\,$ 1/4$\,\,$ & $\,\,$ 1 $\,\,$ & $\,\,$ 1  $\,\,$ \\
\end{pmatrix},
\qquad
\lambda_{\max} =
4.2469,
\qquad
CR = 0.0931
\end{equation*}

\begin{equation*}
\mathbf{w}^{EM} =
\begin{pmatrix}
0.610099\\
0.269498\\
0.060226\\
\color{red} 0.060177\color{black}
\end{pmatrix}\end{equation*}
\begin{equation*}
\left[ \frac{{w}^{EM}_i}{{w}^{EM}_j} \right] =
\begin{pmatrix}
$\,\,$ 1 $\,\,$ & $\,\,$2.2638$\,\,$ & $\,\,$10.1302$\,\,$ & $\,\,$\color{red} 10.1384\color{black} $\,\,$ \\
$\,\,$0.4417$\,\,$ & $\,\,$ 1 $\,\,$ & $\,\,$4.4748$\,\,$ & $\,\,$\color{red} 4.4784\color{black}   $\,\,$ \\
$\,\,$0.0987$\,\,$ & $\,\,$0.2235$\,\,$ & $\,\,$ 1 $\,\,$ & $\,\,$\color{red} 1.0008\color{black}  $\,\,$ \\
$\,\,$\color{red} 0.0986\color{black} $\,\,$ & $\,\,$\color{red} 0.2233\color{black} $\,\,$ & $\,\,$\color{red} 0.9992\color{black} $\,\,$ & $\,\,$ 1  $\,\,$ \\
\end{pmatrix},
\end{equation*}

\begin{equation*}
\mathbf{w}^{\prime} =
\begin{pmatrix}
0.610069\\
0.269485\\
0.060223\\
0.060223
\end{pmatrix} =
0.999951\cdot
\begin{pmatrix}
0.610099\\
0.269498\\
0.060226\\
\color{gr} 0.060226\color{black}
\end{pmatrix},
\end{equation*}
\begin{equation*}
\left[ \frac{{w}^{\prime}_i}{{w}^{\prime}_j} \right] =
\begin{pmatrix}
$\,\,$ 1 $\,\,$ & $\,\,$2.2638$\,\,$ & $\,\,$10.1302$\,\,$ & $\,\,$\color{gr} 10.1302\color{black} $\,\,$ \\
$\,\,$0.4417$\,\,$ & $\,\,$ 1 $\,\,$ & $\,\,$4.4748$\,\,$ & $\,\,$\color{gr} 4.4748\color{black}   $\,\,$ \\
$\,\,$0.0987$\,\,$ & $\,\,$0.2235$\,\,$ & $\,\,$ 1 $\,\,$ & $\,\,$\color{gr} \color{blue} 1\color{black}  $\,\,$ \\
$\,\,$\color{gr} 0.0987\color{black} $\,\,$ & $\,\,$\color{gr} 0.2235\color{black} $\,\,$ & $\,\,$\color{gr} \color{blue} 1\color{black} $\,\,$ & $\,\,$ 1  $\,\,$ \\
\end{pmatrix},
\end{equation*}
\end{example}
\newpage
\begin{example}
\begin{equation*}
\mathbf{A} =
\begin{pmatrix}
$\,\,$ 1 $\,\,$ & $\,\,$4$\,\,$ & $\,\,$7$\,\,$ & $\,\,$9 $\,\,$ \\
$\,\,$ 1/4$\,\,$ & $\,\,$ 1 $\,\,$ & $\,\,$4$\,\,$ & $\,\,$3 $\,\,$ \\
$\,\,$ 1/7$\,\,$ & $\,\,$ 1/4$\,\,$ & $\,\,$ 1 $\,\,$ & $\,\,$1 $\,\,$ \\
$\,\,$ 1/9$\,\,$ & $\,\,$ 1/3$\,\,$ & $\,\,$ 1 $\,\,$ & $\,\,$ 1  $\,\,$ \\
\end{pmatrix},
\qquad
\lambda_{\max} =
4.0576,
\qquad
CR = 0.0217
\end{equation*}

\begin{equation*}
\mathbf{w}^{EM} =
\begin{pmatrix}
0.646179\\
0.213530\\
0.070481\\
\color{red} 0.069810\color{black}
\end{pmatrix}\end{equation*}
\begin{equation*}
\left[ \frac{{w}^{EM}_i}{{w}^{EM}_j} \right] =
\begin{pmatrix}
$\,\,$ 1 $\,\,$ & $\,\,$3.0262$\,\,$ & $\,\,$9.1682$\,\,$ & $\,\,$\color{red} 9.2562\color{black} $\,\,$ \\
$\,\,$0.3305$\,\,$ & $\,\,$ 1 $\,\,$ & $\,\,$3.0296$\,\,$ & $\,\,$\color{red} 3.0587\color{black}   $\,\,$ \\
$\,\,$0.1091$\,\,$ & $\,\,$0.3301$\,\,$ & $\,\,$ 1 $\,\,$ & $\,\,$\color{red} 1.0096\color{black}  $\,\,$ \\
$\,\,$\color{red} 0.1080\color{black} $\,\,$ & $\,\,$\color{red} 0.3269\color{black} $\,\,$ & $\,\,$\color{red} 0.9905\color{black} $\,\,$ & $\,\,$ 1  $\,\,$ \\
\end{pmatrix},
\end{equation*}

\begin{equation*}
\mathbf{w}^{\prime} =
\begin{pmatrix}
0.645746\\
0.213387\\
0.070433\\
0.070433
\end{pmatrix} =
0.999330\cdot
\begin{pmatrix}
0.646179\\
0.213530\\
0.070481\\
\color{gr} 0.070481\color{black}
\end{pmatrix},
\end{equation*}
\begin{equation*}
\left[ \frac{{w}^{\prime}_i}{{w}^{\prime}_j} \right] =
\begin{pmatrix}
$\,\,$ 1 $\,\,$ & $\,\,$3.0262$\,\,$ & $\,\,$9.1682$\,\,$ & $\,\,$\color{gr} 9.1682\color{black} $\,\,$ \\
$\,\,$0.3305$\,\,$ & $\,\,$ 1 $\,\,$ & $\,\,$3.0296$\,\,$ & $\,\,$\color{gr} 3.0296\color{black}   $\,\,$ \\
$\,\,$0.1091$\,\,$ & $\,\,$0.3301$\,\,$ & $\,\,$ 1 $\,\,$ & $\,\,$\color{gr} \color{blue} 1\color{black}  $\,\,$ \\
$\,\,$\color{gr} 0.1091\color{black} $\,\,$ & $\,\,$\color{gr} 0.3301\color{black} $\,\,$ & $\,\,$\color{gr} \color{blue} 1\color{black} $\,\,$ & $\,\,$ 1  $\,\,$ \\
\end{pmatrix},
\end{equation*}
\end{example}
\newpage
\begin{example}
\begin{equation*}
\mathbf{A} =
\begin{pmatrix}
$\,\,$ 1 $\,\,$ & $\,\,$5$\,\,$ & $\,\,$2$\,\,$ & $\,\,$7 $\,\,$ \\
$\,\,$ 1/5$\,\,$ & $\,\,$ 1 $\,\,$ & $\,\,$2$\,\,$ & $\,\,$2 $\,\,$ \\
$\,\,$ 1/2$\,\,$ & $\,\,$ 1/2$\,\,$ & $\,\,$ 1 $\,\,$ & $\,\,$2 $\,\,$ \\
$\,\,$ 1/7$\,\,$ & $\,\,$ 1/2$\,\,$ & $\,\,$ 1/2$\,\,$ & $\,\,$ 1  $\,\,$ \\
\end{pmatrix},
\qquad
\lambda_{\max} =
4.2287,
\qquad
CR = 0.0862
\end{equation*}

\begin{equation*}
\mathbf{w}^{EM} =
\begin{pmatrix}
0.566438\\
0.187506\\
0.166215\\
\color{red} 0.079841\color{black}
\end{pmatrix}\end{equation*}
\begin{equation*}
\left[ \frac{{w}^{EM}_i}{{w}^{EM}_j} \right] =
\begin{pmatrix}
$\,\,$ 1 $\,\,$ & $\,\,$3.0209$\,\,$ & $\,\,$3.4079$\,\,$ & $\,\,$\color{red} 7.0946\color{black} $\,\,$ \\
$\,\,$0.3310$\,\,$ & $\,\,$ 1 $\,\,$ & $\,\,$1.1281$\,\,$ & $\,\,$\color{red} 2.3485\color{black}   $\,\,$ \\
$\,\,$0.2934$\,\,$ & $\,\,$0.8864$\,\,$ & $\,\,$ 1 $\,\,$ & $\,\,$\color{red} 2.0818\color{black}  $\,\,$ \\
$\,\,$\color{red} 0.1410\color{black} $\,\,$ & $\,\,$\color{red} 0.4258\color{black} $\,\,$ & $\,\,$\color{red} 0.4803\color{black} $\,\,$ & $\,\,$ 1  $\,\,$ \\
\end{pmatrix},
\end{equation*}

\begin{equation*}
\mathbf{w}^{\prime} =
\begin{pmatrix}
0.565828\\
0.187304\\
0.166035\\
0.080833
\end{pmatrix} =
0.998922\cdot
\begin{pmatrix}
0.566438\\
0.187506\\
0.166215\\
\color{gr} 0.080920\color{black}
\end{pmatrix},
\end{equation*}
\begin{equation*}
\left[ \frac{{w}^{\prime}_i}{{w}^{\prime}_j} \right] =
\begin{pmatrix}
$\,\,$ 1 $\,\,$ & $\,\,$3.0209$\,\,$ & $\,\,$3.4079$\,\,$ & $\,\,$\color{gr} \color{blue} 7\color{black} $\,\,$ \\
$\,\,$0.3310$\,\,$ & $\,\,$ 1 $\,\,$ & $\,\,$1.1281$\,\,$ & $\,\,$\color{gr} 2.3172\color{black}   $\,\,$ \\
$\,\,$0.2934$\,\,$ & $\,\,$0.8864$\,\,$ & $\,\,$ 1 $\,\,$ & $\,\,$\color{gr} 2.0541\color{black}  $\,\,$ \\
$\,\,$\color{gr} \color{blue}  1/7\color{black} $\,\,$ & $\,\,$\color{gr} 0.4316\color{black} $\,\,$ & $\,\,$\color{gr} 0.4868\color{black} $\,\,$ & $\,\,$ 1  $\,\,$ \\
\end{pmatrix},
\end{equation*}
\end{example}
\newpage
\begin{example}
\begin{equation*}
\mathbf{A} =
\begin{pmatrix}
$\,\,$ 1 $\,\,$ & $\,\,$5$\,\,$ & $\,\,$3$\,\,$ & $\,\,$3 $\,\,$ \\
$\,\,$ 1/5$\,\,$ & $\,\,$ 1 $\,\,$ & $\,\,$1$\,\,$ & $\,\,$3 $\,\,$ \\
$\,\,$ 1/3$\,\,$ & $\,\,$ 1 $\,\,$ & $\,\,$ 1 $\,\,$ & $\,\,$2 $\,\,$ \\
$\,\,$ 1/3$\,\,$ & $\,\,$ 1/3$\,\,$ & $\,\,$ 1/2$\,\,$ & $\,\,$ 1  $\,\,$ \\
\end{pmatrix},
\qquad
\lambda_{\max} =
4.2277,
\qquad
CR = 0.0859
\end{equation*}

\begin{equation*}
\mathbf{w}^{EM} =
\begin{pmatrix}
0.540312\\
0.182458\\
\color{red} 0.175415\color{black} \\
0.101815
\end{pmatrix}\end{equation*}
\begin{equation*}
\left[ \frac{{w}^{EM}_i}{{w}^{EM}_j} \right] =
\begin{pmatrix}
$\,\,$ 1 $\,\,$ & $\,\,$2.9613$\,\,$ & $\,\,$\color{red} 3.0802\color{black} $\,\,$ & $\,\,$5.3068$\,\,$ \\
$\,\,$0.3377$\,\,$ & $\,\,$ 1 $\,\,$ & $\,\,$\color{red} 1.0401\color{black} $\,\,$ & $\,\,$1.7920  $\,\,$ \\
$\,\,$\color{red} 0.3247\color{black} $\,\,$ & $\,\,$\color{red} 0.9614\color{black} $\,\,$ & $\,\,$ 1 $\,\,$ & $\,\,$\color{red} 1.7229\color{black}  $\,\,$ \\
$\,\,$0.1884$\,\,$ & $\,\,$0.5580$\,\,$ & $\,\,$\color{red} 0.5804\color{black} $\,\,$ & $\,\,$ 1  $\,\,$ \\
\end{pmatrix},
\end{equation*}

\begin{equation*}
\mathbf{w}^{\prime} =
\begin{pmatrix}
0.537791\\
0.181606\\
0.179264\\
0.101340
\end{pmatrix} =
0.995333\cdot
\begin{pmatrix}
0.540312\\
0.182458\\
\color{gr} 0.180104\color{black} \\
0.101815
\end{pmatrix},
\end{equation*}
\begin{equation*}
\left[ \frac{{w}^{\prime}_i}{{w}^{\prime}_j} \right] =
\begin{pmatrix}
$\,\,$ 1 $\,\,$ & $\,\,$2.9613$\,\,$ & $\,\,$\color{gr} \color{blue} 3\color{black} $\,\,$ & $\,\,$5.3068$\,\,$ \\
$\,\,$0.3377$\,\,$ & $\,\,$ 1 $\,\,$ & $\,\,$\color{gr} 1.0131\color{black} $\,\,$ & $\,\,$1.7920  $\,\,$ \\
$\,\,$\color{gr} \color{blue}  1/3\color{black} $\,\,$ & $\,\,$\color{gr} 0.9871\color{black} $\,\,$ & $\,\,$ 1 $\,\,$ & $\,\,$\color{gr} 1.7689\color{black}  $\,\,$ \\
$\,\,$0.1884$\,\,$ & $\,\,$0.5580$\,\,$ & $\,\,$\color{gr} 0.5653\color{black} $\,\,$ & $\,\,$ 1  $\,\,$ \\
\end{pmatrix},
\end{equation*}
\end{example}
\newpage
\begin{example}
\begin{equation*}
\mathbf{A} =
\begin{pmatrix}
$\,\,$ 1 $\,\,$ & $\,\,$5$\,\,$ & $\,\,$3$\,\,$ & $\,\,$4 $\,\,$ \\
$\,\,$ 1/5$\,\,$ & $\,\,$ 1 $\,\,$ & $\,\,$1$\,\,$ & $\,\,$3 $\,\,$ \\
$\,\,$ 1/3$\,\,$ & $\,\,$ 1 $\,\,$ & $\,\,$ 1 $\,\,$ & $\,\,$2 $\,\,$ \\
$\,\,$ 1/4$\,\,$ & $\,\,$ 1/3$\,\,$ & $\,\,$ 1/2$\,\,$ & $\,\,$ 1  $\,\,$ \\
\end{pmatrix},
\qquad
\lambda_{\max} =
4.1502,
\qquad
CR = 0.0566
\end{equation*}

\begin{equation*}
\mathbf{w}^{EM} =
\begin{pmatrix}
0.559960\\
0.176655\\
\color{red} 0.172824\color{black} \\
0.090561
\end{pmatrix}\end{equation*}
\begin{equation*}
\left[ \frac{{w}^{EM}_i}{{w}^{EM}_j} \right] =
\begin{pmatrix}
$\,\,$ 1 $\,\,$ & $\,\,$3.1698$\,\,$ & $\,\,$\color{red} 3.2401\color{black} $\,\,$ & $\,\,$6.1832$\,\,$ \\
$\,\,$0.3155$\,\,$ & $\,\,$ 1 $\,\,$ & $\,\,$\color{red} 1.0222\color{black} $\,\,$ & $\,\,$1.9507  $\,\,$ \\
$\,\,$\color{red} 0.3086\color{black} $\,\,$ & $\,\,$\color{red} 0.9783\color{black} $\,\,$ & $\,\,$ 1 $\,\,$ & $\,\,$\color{red} 1.9084\color{black}  $\,\,$ \\
$\,\,$0.1617$\,\,$ & $\,\,$0.5126$\,\,$ & $\,\,$\color{red} 0.5240\color{black} $\,\,$ & $\,\,$ 1  $\,\,$ \\
\end{pmatrix},
\end{equation*}

\begin{equation*}
\mathbf{w}^{\prime} =
\begin{pmatrix}
0.557823\\
0.175981\\
0.175981\\
0.090216
\end{pmatrix} =
0.996184\cdot
\begin{pmatrix}
0.559960\\
0.176655\\
\color{gr} 0.176655\color{black} \\
0.090561
\end{pmatrix},
\end{equation*}
\begin{equation*}
\left[ \frac{{w}^{\prime}_i}{{w}^{\prime}_j} \right] =
\begin{pmatrix}
$\,\,$ 1 $\,\,$ & $\,\,$3.1698$\,\,$ & $\,\,$\color{gr} 3.1698\color{black} $\,\,$ & $\,\,$6.1832$\,\,$ \\
$\,\,$0.3155$\,\,$ & $\,\,$ 1 $\,\,$ & $\,\,$\color{gr} \color{blue} 1\color{black} $\,\,$ & $\,\,$1.9507  $\,\,$ \\
$\,\,$\color{gr} 0.3155\color{black} $\,\,$ & $\,\,$\color{gr} \color{blue} 1\color{black} $\,\,$ & $\,\,$ 1 $\,\,$ & $\,\,$\color{gr} 1.9507\color{black}  $\,\,$ \\
$\,\,$0.1617$\,\,$ & $\,\,$0.5126$\,\,$ & $\,\,$\color{gr} 0.5126\color{black} $\,\,$ & $\,\,$ 1  $\,\,$ \\
\end{pmatrix},
\end{equation*}
\end{example}
\newpage
\begin{example}
\begin{equation*}
\mathbf{A} =
\begin{pmatrix}
$\,\,$ 1 $\,\,$ & $\,\,$5$\,\,$ & $\,\,$3$\,\,$ & $\,\,$5 $\,\,$ \\
$\,\,$ 1/5$\,\,$ & $\,\,$ 1 $\,\,$ & $\,\,$1$\,\,$ & $\,\,$5 $\,\,$ \\
$\,\,$ 1/3$\,\,$ & $\,\,$ 1 $\,\,$ & $\,\,$ 1 $\,\,$ & $\,\,$3 $\,\,$ \\
$\,\,$ 1/5$\,\,$ & $\,\,$ 1/5$\,\,$ & $\,\,$ 1/3$\,\,$ & $\,\,$ 1  $\,\,$ \\
\end{pmatrix},
\qquad
\lambda_{\max} =
4.2253,
\qquad
CR = 0.0849
\end{equation*}

\begin{equation*}
\mathbf{w}^{EM} =
\begin{pmatrix}
0.564472\\
0.191652\\
\color{red} 0.178537\color{black} \\
0.065339
\end{pmatrix}\end{equation*}
\begin{equation*}
\left[ \frac{{w}^{EM}_i}{{w}^{EM}_j} \right] =
\begin{pmatrix}
$\,\,$ 1 $\,\,$ & $\,\,$2.9453$\,\,$ & $\,\,$\color{red} 3.1617\color{black} $\,\,$ & $\,\,$8.6391$\,\,$ \\
$\,\,$0.3395$\,\,$ & $\,\,$ 1 $\,\,$ & $\,\,$\color{red} 1.0735\color{black} $\,\,$ & $\,\,$2.9332  $\,\,$ \\
$\,\,$\color{red} 0.3163\color{black} $\,\,$ & $\,\,$\color{red} 0.9316\color{black} $\,\,$ & $\,\,$ 1 $\,\,$ & $\,\,$\color{red} 2.7324\color{black}  $\,\,$ \\
$\,\,$0.1158$\,\,$ & $\,\,$0.3409$\,\,$ & $\,\,$\color{red} 0.3660\color{black} $\,\,$ & $\,\,$ 1  $\,\,$ \\
\end{pmatrix},
\end{equation*}

\begin{equation*}
\mathbf{w}^{\prime} =
\begin{pmatrix}
0.559093\\
0.189826\\
0.186364\\
0.064717
\end{pmatrix} =
0.990471\cdot
\begin{pmatrix}
0.564472\\
0.191652\\
\color{gr} 0.188157\color{black} \\
0.065339
\end{pmatrix},
\end{equation*}
\begin{equation*}
\left[ \frac{{w}^{\prime}_i}{{w}^{\prime}_j} \right] =
\begin{pmatrix}
$\,\,$ 1 $\,\,$ & $\,\,$2.9453$\,\,$ & $\,\,$\color{gr} \color{blue} 3\color{black} $\,\,$ & $\,\,$8.6391$\,\,$ \\
$\,\,$0.3395$\,\,$ & $\,\,$ 1 $\,\,$ & $\,\,$\color{gr} 1.0186\color{black} $\,\,$ & $\,\,$2.9332  $\,\,$ \\
$\,\,$\color{gr} \color{blue}  1/3\color{black} $\,\,$ & $\,\,$\color{gr} 0.9818\color{black} $\,\,$ & $\,\,$ 1 $\,\,$ & $\,\,$\color{gr} 2.8797\color{black}  $\,\,$ \\
$\,\,$0.1158$\,\,$ & $\,\,$0.3409$\,\,$ & $\,\,$\color{gr} 0.3473\color{black} $\,\,$ & $\,\,$ 1  $\,\,$ \\
\end{pmatrix},
\end{equation*}
\end{example}
\newpage
\begin{example}
\begin{equation*}
\mathbf{A} =
\begin{pmatrix}
$\,\,$ 1 $\,\,$ & $\,\,$5$\,\,$ & $\,\,$3$\,\,$ & $\,\,$6 $\,\,$ \\
$\,\,$ 1/5$\,\,$ & $\,\,$ 1 $\,\,$ & $\,\,$1$\,\,$ & $\,\,$5 $\,\,$ \\
$\,\,$ 1/3$\,\,$ & $\,\,$ 1 $\,\,$ & $\,\,$ 1 $\,\,$ & $\,\,$3 $\,\,$ \\
$\,\,$ 1/6$\,\,$ & $\,\,$ 1/5$\,\,$ & $\,\,$ 1/3$\,\,$ & $\,\,$ 1  $\,\,$ \\
\end{pmatrix},
\qquad
\lambda_{\max} =
4.1758,
\qquad
CR = 0.0663
\end{equation*}

\begin{equation*}
\mathbf{w}^{EM} =
\begin{pmatrix}
0.575775\\
0.187150\\
\color{red} 0.176543\color{black} \\
0.060532
\end{pmatrix}\end{equation*}
\begin{equation*}
\left[ \frac{{w}^{EM}_i}{{w}^{EM}_j} \right] =
\begin{pmatrix}
$\,\,$ 1 $\,\,$ & $\,\,$3.0765$\,\,$ & $\,\,$\color{red} 3.2614\color{black} $\,\,$ & $\,\,$9.5119$\,\,$ \\
$\,\,$0.3250$\,\,$ & $\,\,$ 1 $\,\,$ & $\,\,$\color{red} 1.0601\color{black} $\,\,$ & $\,\,$3.0917  $\,\,$ \\
$\,\,$\color{red} 0.3066\color{black} $\,\,$ & $\,\,$\color{red} 0.9433\color{black} $\,\,$ & $\,\,$ 1 $\,\,$ & $\,\,$\color{red} 2.9165\color{black}  $\,\,$ \\
$\,\,$0.1051$\,\,$ & $\,\,$0.3234$\,\,$ & $\,\,$\color{red} 0.3429\color{black} $\,\,$ & $\,\,$ 1  $\,\,$ \\
\end{pmatrix},
\end{equation*}

\begin{equation*}
\mathbf{w}^{\prime} =
\begin{pmatrix}
0.572880\\
0.186209\\
0.180683\\
0.060228
\end{pmatrix} =
0.994972\cdot
\begin{pmatrix}
0.575775\\
0.187150\\
\color{gr} 0.181596\color{black} \\
0.060532
\end{pmatrix},
\end{equation*}
\begin{equation*}
\left[ \frac{{w}^{\prime}_i}{{w}^{\prime}_j} \right] =
\begin{pmatrix}
$\,\,$ 1 $\,\,$ & $\,\,$3.0765$\,\,$ & $\,\,$\color{gr} 3.1706\color{black} $\,\,$ & $\,\,$9.5119$\,\,$ \\
$\,\,$0.3250$\,\,$ & $\,\,$ 1 $\,\,$ & $\,\,$\color{gr} 1.0306\color{black} $\,\,$ & $\,\,$3.0917  $\,\,$ \\
$\,\,$\color{gr} 0.3154\color{black} $\,\,$ & $\,\,$\color{gr} 0.9703\color{black} $\,\,$ & $\,\,$ 1 $\,\,$ & $\,\,$\color{gr} \color{blue} 3\color{black}  $\,\,$ \\
$\,\,$0.1051$\,\,$ & $\,\,$0.3234$\,\,$ & $\,\,$\color{gr} \color{blue}  1/3\color{black} $\,\,$ & $\,\,$ 1  $\,\,$ \\
\end{pmatrix},
\end{equation*}
\end{example}
\newpage
\begin{example}
\begin{equation*}
\mathbf{A} =
\begin{pmatrix}
$\,\,$ 1 $\,\,$ & $\,\,$5$\,\,$ & $\,\,$3$\,\,$ & $\,\,$6 $\,\,$ \\
$\,\,$ 1/5$\,\,$ & $\,\,$ 1 $\,\,$ & $\,\,$1$\,\,$ & $\,\,$6 $\,\,$ \\
$\,\,$ 1/3$\,\,$ & $\,\,$ 1 $\,\,$ & $\,\,$ 1 $\,\,$ & $\,\,$4 $\,\,$ \\
$\,\,$ 1/6$\,\,$ & $\,\,$ 1/6$\,\,$ & $\,\,$ 1/4$\,\,$ & $\,\,$ 1  $\,\,$ \\
\end{pmatrix},
\qquad
\lambda_{\max} =
4.2277,
\qquad
CR = 0.0859
\end{equation*}

\begin{equation*}
\mathbf{w}^{EM} =
\begin{pmatrix}
0.569294\\
0.192244\\
\color{red} 0.184824\color{black} \\
0.053638
\end{pmatrix}\end{equation*}
\begin{equation*}
\left[ \frac{{w}^{EM}_i}{{w}^{EM}_j} \right] =
\begin{pmatrix}
$\,\,$ 1 $\,\,$ & $\,\,$2.9613$\,\,$ & $\,\,$\color{red} 3.0802\color{black} $\,\,$ & $\,\,$10.6136$\,\,$ \\
$\,\,$0.3377$\,\,$ & $\,\,$ 1 $\,\,$ & $\,\,$\color{red} 1.0401\color{black} $\,\,$ & $\,\,$3.5841  $\,\,$ \\
$\,\,$\color{red} 0.3247\color{black} $\,\,$ & $\,\,$\color{red} 0.9614\color{black} $\,\,$ & $\,\,$ 1 $\,\,$ & $\,\,$\color{red} 3.4458\color{black}  $\,\,$ \\
$\,\,$0.0942$\,\,$ & $\,\,$0.2790$\,\,$ & $\,\,$\color{red} 0.2902\color{black} $\,\,$ & $\,\,$ 1  $\,\,$ \\
\end{pmatrix},
\end{equation*}

\begin{equation*}
\mathbf{w}^{\prime} =
\begin{pmatrix}
0.566495\\
0.191299\\
0.188832\\
0.053374
\end{pmatrix} =
0.995084\cdot
\begin{pmatrix}
0.569294\\
0.192244\\
\color{gr} 0.189765\color{black} \\
0.053638
\end{pmatrix},
\end{equation*}
\begin{equation*}
\left[ \frac{{w}^{\prime}_i}{{w}^{\prime}_j} \right] =
\begin{pmatrix}
$\,\,$ 1 $\,\,$ & $\,\,$2.9613$\,\,$ & $\,\,$\color{gr} \color{blue} 3\color{black} $\,\,$ & $\,\,$10.6136$\,\,$ \\
$\,\,$0.3377$\,\,$ & $\,\,$ 1 $\,\,$ & $\,\,$\color{gr} 1.0131\color{black} $\,\,$ & $\,\,$3.5841  $\,\,$ \\
$\,\,$\color{gr} \color{blue}  1/3\color{black} $\,\,$ & $\,\,$\color{gr} 0.9871\color{black} $\,\,$ & $\,\,$ 1 $\,\,$ & $\,\,$\color{gr} 3.5379\color{black}  $\,\,$ \\
$\,\,$0.0942$\,\,$ & $\,\,$0.2790$\,\,$ & $\,\,$\color{gr} 0.2827\color{black} $\,\,$ & $\,\,$ 1  $\,\,$ \\
\end{pmatrix},
\end{equation*}
\end{example}
\newpage
\begin{example}
\begin{equation*}
\mathbf{A} =
\begin{pmatrix}
$\,\,$ 1 $\,\,$ & $\,\,$5$\,\,$ & $\,\,$3$\,\,$ & $\,\,$6 $\,\,$ \\
$\,\,$ 1/5$\,\,$ & $\,\,$ 1 $\,\,$ & $\,\,$3$\,\,$ & $\,\,$2 $\,\,$ \\
$\,\,$ 1/3$\,\,$ & $\,\,$ 1/3$\,\,$ & $\,\,$ 1 $\,\,$ & $\,\,$1 $\,\,$ \\
$\,\,$ 1/6$\,\,$ & $\,\,$ 1/2$\,\,$ & $\,\,$ 1 $\,\,$ & $\,\,$ 1  $\,\,$ \\
\end{pmatrix},
\qquad
\lambda_{\max} =
4.2277,
\qquad
CR = 0.0859
\end{equation*}

\begin{equation*}
\mathbf{w}^{EM} =
\begin{pmatrix}
0.592258\\
0.199999\\
0.111604\\
\color{red} 0.096140\color{black}
\end{pmatrix}\end{equation*}
\begin{equation*}
\left[ \frac{{w}^{EM}_i}{{w}^{EM}_j} \right] =
\begin{pmatrix}
$\,\,$ 1 $\,\,$ & $\,\,$2.9613$\,\,$ & $\,\,$5.3068$\,\,$ & $\,\,$\color{red} 6.1604\color{black} $\,\,$ \\
$\,\,$0.3377$\,\,$ & $\,\,$ 1 $\,\,$ & $\,\,$1.7920$\,\,$ & $\,\,$\color{red} 2.0803\color{black}   $\,\,$ \\
$\,\,$0.1884$\,\,$ & $\,\,$0.5580$\,\,$ & $\,\,$ 1 $\,\,$ & $\,\,$\color{red} 1.1608\color{black}  $\,\,$ \\
$\,\,$\color{red} 0.1623\color{black} $\,\,$ & $\,\,$\color{red} 0.4807\color{black} $\,\,$ & $\,\,$\color{red} 0.8614\color{black} $\,\,$ & $\,\,$ 1  $\,\,$ \\
\end{pmatrix},
\end{equation*}

\begin{equation*}
\mathbf{w}^{\prime} =
\begin{pmatrix}
0.590740\\
0.199486\\
0.111317\\
0.098457
\end{pmatrix} =
0.997437\cdot
\begin{pmatrix}
0.592258\\
0.199999\\
0.111604\\
\color{gr} 0.098710\color{black}
\end{pmatrix},
\end{equation*}
\begin{equation*}
\left[ \frac{{w}^{\prime}_i}{{w}^{\prime}_j} \right] =
\begin{pmatrix}
$\,\,$ 1 $\,\,$ & $\,\,$2.9613$\,\,$ & $\,\,$5.3068$\,\,$ & $\,\,$\color{gr} \color{blue} 6\color{black} $\,\,$ \\
$\,\,$0.3377$\,\,$ & $\,\,$ 1 $\,\,$ & $\,\,$1.7920$\,\,$ & $\,\,$\color{gr} 2.0261\color{black}   $\,\,$ \\
$\,\,$0.1884$\,\,$ & $\,\,$0.5580$\,\,$ & $\,\,$ 1 $\,\,$ & $\,\,$\color{gr} 1.1306\color{black}  $\,\,$ \\
$\,\,$\color{gr} \color{blue}  1/6\color{black} $\,\,$ & $\,\,$\color{gr} 0.4936\color{black} $\,\,$ & $\,\,$\color{gr} 0.8845\color{black} $\,\,$ & $\,\,$ 1  $\,\,$ \\
\end{pmatrix},
\end{equation*}
\end{example}
\newpage
\begin{example}
\begin{equation*}
\mathbf{A} =
\begin{pmatrix}
$\,\,$ 1 $\,\,$ & $\,\,$5$\,\,$ & $\,\,$3$\,\,$ & $\,\,$7 $\,\,$ \\
$\,\,$ 1/5$\,\,$ & $\,\,$ 1 $\,\,$ & $\,\,$1$\,\,$ & $\,\,$6 $\,\,$ \\
$\,\,$ 1/3$\,\,$ & $\,\,$ 1 $\,\,$ & $\,\,$ 1 $\,\,$ & $\,\,$4 $\,\,$ \\
$\,\,$ 1/7$\,\,$ & $\,\,$ 1/6$\,\,$ & $\,\,$ 1/4$\,\,$ & $\,\,$ 1  $\,\,$ \\
\end{pmatrix},
\qquad
\lambda_{\max} =
4.1832,
\qquad
CR = 0.0691
\end{equation*}

\begin{equation*}
\mathbf{w}^{EM} =
\begin{pmatrix}
0.578587\\
0.188389\\
\color{red} 0.182835\color{black} \\
0.050189
\end{pmatrix}\end{equation*}
\begin{equation*}
\left[ \frac{{w}^{EM}_i}{{w}^{EM}_j} \right] =
\begin{pmatrix}
$\,\,$ 1 $\,\,$ & $\,\,$3.0712$\,\,$ & $\,\,$\color{red} 3.1645\color{black} $\,\,$ & $\,\,$11.5282$\,\,$ \\
$\,\,$0.3256$\,\,$ & $\,\,$ 1 $\,\,$ & $\,\,$\color{red} 1.0304\color{black} $\,\,$ & $\,\,$3.7536  $\,\,$ \\
$\,\,$\color{red} 0.3160\color{black} $\,\,$ & $\,\,$\color{red} 0.9705\color{black} $\,\,$ & $\,\,$ 1 $\,\,$ & $\,\,$\color{red} 3.6430\color{black}  $\,\,$ \\
$\,\,$0.0867$\,\,$ & $\,\,$0.2664$\,\,$ & $\,\,$\color{red} 0.2745\color{black} $\,\,$ & $\,\,$ 1  $\,\,$ \\
\end{pmatrix},
\end{equation*}

\begin{equation*}
\mathbf{w}^{\prime} =
\begin{pmatrix}
0.575391\\
0.187349\\
0.187349\\
0.049912
\end{pmatrix} =
0.994477\cdot
\begin{pmatrix}
0.578587\\
0.188389\\
\color{gr} 0.188389\color{black} \\
0.050189
\end{pmatrix},
\end{equation*}
\begin{equation*}
\left[ \frac{{w}^{\prime}_i}{{w}^{\prime}_j} \right] =
\begin{pmatrix}
$\,\,$ 1 $\,\,$ & $\,\,$3.0712$\,\,$ & $\,\,$\color{gr} 3.0712\color{black} $\,\,$ & $\,\,$11.5282$\,\,$ \\
$\,\,$0.3256$\,\,$ & $\,\,$ 1 $\,\,$ & $\,\,$\color{gr} \color{blue} 1\color{black} $\,\,$ & $\,\,$3.7536  $\,\,$ \\
$\,\,$\color{gr} 0.3256\color{black} $\,\,$ & $\,\,$\color{gr} \color{blue} 1\color{black} $\,\,$ & $\,\,$ 1 $\,\,$ & $\,\,$\color{gr} 3.7536\color{black}  $\,\,$ \\
$\,\,$0.0867$\,\,$ & $\,\,$0.2664$\,\,$ & $\,\,$\color{gr} 0.2664\color{black} $\,\,$ & $\,\,$ 1  $\,\,$ \\
\end{pmatrix},
\end{equation*}
\end{example}
\newpage
\begin{example}
\begin{equation*}
\mathbf{A} =
\begin{pmatrix}
$\,\,$ 1 $\,\,$ & $\,\,$5$\,\,$ & $\,\,$3$\,\,$ & $\,\,$7 $\,\,$ \\
$\,\,$ 1/5$\,\,$ & $\,\,$ 1 $\,\,$ & $\,\,$1$\,\,$ & $\,\,$7 $\,\,$ \\
$\,\,$ 1/3$\,\,$ & $\,\,$ 1 $\,\,$ & $\,\,$ 1 $\,\,$ & $\,\,$4 $\,\,$ \\
$\,\,$ 1/7$\,\,$ & $\,\,$ 1/7$\,\,$ & $\,\,$ 1/4$\,\,$ & $\,\,$ 1  $\,\,$ \\
\end{pmatrix},
\qquad
\lambda_{\max} =
4.2251,
\qquad
CR = 0.0849
\end{equation*}

\begin{equation*}
\mathbf{w}^{EM} =
\begin{pmatrix}
0.575833\\
0.196018\\
\color{red} 0.180006\color{black} \\
0.048143
\end{pmatrix}\end{equation*}
\begin{equation*}
\left[ \frac{{w}^{EM}_i}{{w}^{EM}_j} \right] =
\begin{pmatrix}
$\,\,$ 1 $\,\,$ & $\,\,$2.9377$\,\,$ & $\,\,$\color{red} 3.1990\color{black} $\,\,$ & $\,\,$11.9608$\,\,$ \\
$\,\,$0.3404$\,\,$ & $\,\,$ 1 $\,\,$ & $\,\,$\color{red} 1.0890\color{black} $\,\,$ & $\,\,$4.0716  $\,\,$ \\
$\,\,$\color{red} 0.3126\color{black} $\,\,$ & $\,\,$\color{red} 0.9183\color{black} $\,\,$ & $\,\,$ 1 $\,\,$ & $\,\,$\color{red} 3.7390\color{black}  $\,\,$ \\
$\,\,$0.0836$\,\,$ & $\,\,$0.2456$\,\,$ & $\,\,$\color{red} 0.2675\color{black} $\,\,$ & $\,\,$ 1  $\,\,$ \\
\end{pmatrix},
\end{equation*}

\begin{equation*}
\mathbf{w}^{\prime} =
\begin{pmatrix}
0.569039\\
0.193705\\
0.189680\\
0.047575
\end{pmatrix} =
0.988203\cdot
\begin{pmatrix}
0.575833\\
0.196018\\
\color{gr} 0.191944\color{black} \\
0.048143
\end{pmatrix},
\end{equation*}
\begin{equation*}
\left[ \frac{{w}^{\prime}_i}{{w}^{\prime}_j} \right] =
\begin{pmatrix}
$\,\,$ 1 $\,\,$ & $\,\,$2.9377$\,\,$ & $\,\,$\color{gr} \color{blue} 3\color{black} $\,\,$ & $\,\,$11.9608$\,\,$ \\
$\,\,$0.3404$\,\,$ & $\,\,$ 1 $\,\,$ & $\,\,$\color{gr} 1.0212\color{black} $\,\,$ & $\,\,$4.0716  $\,\,$ \\
$\,\,$\color{gr} \color{blue}  1/3\color{black} $\,\,$ & $\,\,$\color{gr} 0.9792\color{black} $\,\,$ & $\,\,$ 1 $\,\,$ & $\,\,$\color{gr} 3.9869\color{black}  $\,\,$ \\
$\,\,$0.0836$\,\,$ & $\,\,$0.2456$\,\,$ & $\,\,$\color{gr} 0.2508\color{black} $\,\,$ & $\,\,$ 1  $\,\,$ \\
\end{pmatrix},
\end{equation*}
\end{example}
\newpage
\begin{example}
\begin{equation*}
\mathbf{A} =
\begin{pmatrix}
$\,\,$ 1 $\,\,$ & $\,\,$5$\,\,$ & $\,\,$3$\,\,$ & $\,\,$7 $\,\,$ \\
$\,\,$ 1/5$\,\,$ & $\,\,$ 1 $\,\,$ & $\,\,$1$\,\,$ & $\,\,$7 $\,\,$ \\
$\,\,$ 1/3$\,\,$ & $\,\,$ 1 $\,\,$ & $\,\,$ 1 $\,\,$ & $\,\,$5 $\,\,$ \\
$\,\,$ 1/7$\,\,$ & $\,\,$ 1/7$\,\,$ & $\,\,$ 1/5$\,\,$ & $\,\,$ 1  $\,\,$ \\
\end{pmatrix},
\qquad
\lambda_{\max} =
4.2309,
\qquad
CR = 0.0871
\end{equation*}

\begin{equation*}
\mathbf{w}^{EM} =
\begin{pmatrix}
0.572563\\
0.192688\\
\color{red} 0.189201\color{black} \\
0.045548
\end{pmatrix}\end{equation*}
\begin{equation*}
\left[ \frac{{w}^{EM}_i}{{w}^{EM}_j} \right] =
\begin{pmatrix}
$\,\,$ 1 $\,\,$ & $\,\,$2.9714$\,\,$ & $\,\,$\color{red} 3.0262\color{black} $\,\,$ & $\,\,$12.5704$\,\,$ \\
$\,\,$0.3365$\,\,$ & $\,\,$ 1 $\,\,$ & $\,\,$\color{red} 1.0184\color{black} $\,\,$ & $\,\,$4.2304  $\,\,$ \\
$\,\,$\color{red} 0.3304\color{black} $\,\,$ & $\,\,$\color{red} 0.9819\color{black} $\,\,$ & $\,\,$ 1 $\,\,$ & $\,\,$\color{red} 4.1538\color{black}  $\,\,$ \\
$\,\,$0.0796$\,\,$ & $\,\,$0.2364$\,\,$ & $\,\,$\color{red} 0.2407\color{black} $\,\,$ & $\,\,$ 1  $\,\,$ \\
\end{pmatrix},
\end{equation*}

\begin{equation*}
\mathbf{w}^{\prime} =
\begin{pmatrix}
0.571618\\
0.192370\\
0.190539\\
0.045473
\end{pmatrix} =
0.998349\cdot
\begin{pmatrix}
0.572563\\
0.192688\\
\color{gr} 0.190854\color{black} \\
0.045548
\end{pmatrix},
\end{equation*}
\begin{equation*}
\left[ \frac{{w}^{\prime}_i}{{w}^{\prime}_j} \right] =
\begin{pmatrix}
$\,\,$ 1 $\,\,$ & $\,\,$2.9714$\,\,$ & $\,\,$\color{gr} \color{blue} 3\color{black} $\,\,$ & $\,\,$12.5704$\,\,$ \\
$\,\,$0.3365$\,\,$ & $\,\,$ 1 $\,\,$ & $\,\,$\color{gr} 1.0096\color{black} $\,\,$ & $\,\,$4.2304  $\,\,$ \\
$\,\,$\color{gr} \color{blue}  1/3\color{black} $\,\,$ & $\,\,$\color{gr} 0.9905\color{black} $\,\,$ & $\,\,$ 1 $\,\,$ & $\,\,$\color{gr} 4.1901\color{black}  $\,\,$ \\
$\,\,$0.0796$\,\,$ & $\,\,$0.2364$\,\,$ & $\,\,$\color{gr} 0.2387\color{black} $\,\,$ & $\,\,$ 1  $\,\,$ \\
\end{pmatrix},
\end{equation*}
\end{example}
\newpage
\begin{example}
\begin{equation*}
\mathbf{A} =
\begin{pmatrix}
$\,\,$ 1 $\,\,$ & $\,\,$5$\,\,$ & $\,\,$3$\,\,$ & $\,\,$8 $\,\,$ \\
$\,\,$ 1/5$\,\,$ & $\,\,$ 1 $\,\,$ & $\,\,$1$\,\,$ & $\,\,$6 $\,\,$ \\
$\,\,$ 1/3$\,\,$ & $\,\,$ 1 $\,\,$ & $\,\,$ 1 $\,\,$ & $\,\,$4 $\,\,$ \\
$\,\,$ 1/8$\,\,$ & $\,\,$ 1/6$\,\,$ & $\,\,$ 1/4$\,\,$ & $\,\,$ 1  $\,\,$ \\
\end{pmatrix},
\qquad
\lambda_{\max} =
4.1502,
\qquad
CR = 0.0566
\end{equation*}

\begin{equation*}
\mathbf{w}^{EM} =
\begin{pmatrix}
0.586518\\
0.185033\\
\color{red} 0.181020\color{black} \\
0.047428
\end{pmatrix}\end{equation*}
\begin{equation*}
\left[ \frac{{w}^{EM}_i}{{w}^{EM}_j} \right] =
\begin{pmatrix}
$\,\,$ 1 $\,\,$ & $\,\,$3.1698$\,\,$ & $\,\,$\color{red} 3.2401\color{black} $\,\,$ & $\,\,$12.3664$\,\,$ \\
$\,\,$0.3155$\,\,$ & $\,\,$ 1 $\,\,$ & $\,\,$\color{red} 1.0222\color{black} $\,\,$ & $\,\,$3.9013  $\,\,$ \\
$\,\,$\color{red} 0.3086\color{black} $\,\,$ & $\,\,$\color{red} 0.9783\color{black} $\,\,$ & $\,\,$ 1 $\,\,$ & $\,\,$\color{red} 3.8167\color{black}  $\,\,$ \\
$\,\,$0.0809$\,\,$ & $\,\,$0.2563$\,\,$ & $\,\,$\color{red} 0.2620\color{black} $\,\,$ & $\,\,$ 1  $\,\,$ \\
\end{pmatrix},
\end{equation*}

\begin{equation*}
\mathbf{w}^{\prime} =
\begin{pmatrix}
0.584174\\
0.184294\\
0.184294\\
0.047239
\end{pmatrix} =
0.996003\cdot
\begin{pmatrix}
0.586518\\
0.185033\\
\color{gr} 0.185033\color{black} \\
0.047428
\end{pmatrix},
\end{equation*}
\begin{equation*}
\left[ \frac{{w}^{\prime}_i}{{w}^{\prime}_j} \right] =
\begin{pmatrix}
$\,\,$ 1 $\,\,$ & $\,\,$3.1698$\,\,$ & $\,\,$\color{gr} 3.1698\color{black} $\,\,$ & $\,\,$12.3664$\,\,$ \\
$\,\,$0.3155$\,\,$ & $\,\,$ 1 $\,\,$ & $\,\,$\color{gr} \color{blue} 1\color{black} $\,\,$ & $\,\,$3.9013  $\,\,$ \\
$\,\,$\color{gr} 0.3155\color{black} $\,\,$ & $\,\,$\color{gr} \color{blue} 1\color{black} $\,\,$ & $\,\,$ 1 $\,\,$ & $\,\,$\color{gr} 3.9013\color{black}  $\,\,$ \\
$\,\,$0.0809$\,\,$ & $\,\,$0.2563$\,\,$ & $\,\,$\color{gr} 0.2563\color{black} $\,\,$ & $\,\,$ 1  $\,\,$ \\
\end{pmatrix},
\end{equation*}
\end{example}
\newpage
\begin{example}
\begin{equation*}
\mathbf{A} =
\begin{pmatrix}
$\,\,$ 1 $\,\,$ & $\,\,$5$\,\,$ & $\,\,$3$\,\,$ & $\,\,$8 $\,\,$ \\
$\,\,$ 1/5$\,\,$ & $\,\,$ 1 $\,\,$ & $\,\,$1$\,\,$ & $\,\,$7 $\,\,$ \\
$\,\,$ 1/3$\,\,$ & $\,\,$ 1 $\,\,$ & $\,\,$ 1 $\,\,$ & $\,\,$4 $\,\,$ \\
$\,\,$ 1/8$\,\,$ & $\,\,$ 1/7$\,\,$ & $\,\,$ 1/4$\,\,$ & $\,\,$ 1  $\,\,$ \\
\end{pmatrix},
\qquad
\lambda_{\max} =
4.1888,
\qquad
CR = 0.0712
\end{equation*}

\begin{equation*}
\mathbf{w}^{EM} =
\begin{pmatrix}
0.583681\\
0.192415\\
\color{red} 0.178416\color{black} \\
0.045488
\end{pmatrix}\end{equation*}
\begin{equation*}
\left[ \frac{{w}^{EM}_i}{{w}^{EM}_j} \right] =
\begin{pmatrix}
$\,\,$ 1 $\,\,$ & $\,\,$3.0334$\,\,$ & $\,\,$\color{red} 3.2715\color{black} $\,\,$ & $\,\,$12.8315$\,\,$ \\
$\,\,$0.3297$\,\,$ & $\,\,$ 1 $\,\,$ & $\,\,$\color{red} 1.0785\color{black} $\,\,$ & $\,\,$4.2300  $\,\,$ \\
$\,\,$\color{red} 0.3057\color{black} $\,\,$ & $\,\,$\color{red} 0.9272\color{black} $\,\,$ & $\,\,$ 1 $\,\,$ & $\,\,$\color{red} 3.9222\color{black}  $\,\,$ \\
$\,\,$0.0779$\,\,$ & $\,\,$0.2364$\,\,$ & $\,\,$\color{red} 0.2550\color{black} $\,\,$ & $\,\,$ 1  $\,\,$ \\
\end{pmatrix},
\end{equation*}

\begin{equation*}
\mathbf{w}^{\prime} =
\begin{pmatrix}
0.581624\\
0.191737\\
0.181312\\
0.045328
\end{pmatrix} =
0.996475\cdot
\begin{pmatrix}
0.583681\\
0.192415\\
\color{gr} 0.181953\color{black} \\
0.045488
\end{pmatrix},
\end{equation*}
\begin{equation*}
\left[ \frac{{w}^{\prime}_i}{{w}^{\prime}_j} \right] =
\begin{pmatrix}
$\,\,$ 1 $\,\,$ & $\,\,$3.0334$\,\,$ & $\,\,$\color{gr} 3.2079\color{black} $\,\,$ & $\,\,$12.8315$\,\,$ \\
$\,\,$0.3297$\,\,$ & $\,\,$ 1 $\,\,$ & $\,\,$\color{gr} 1.0575\color{black} $\,\,$ & $\,\,$4.2300  $\,\,$ \\
$\,\,$\color{gr} 0.3117\color{black} $\,\,$ & $\,\,$\color{gr} 0.9456\color{black} $\,\,$ & $\,\,$ 1 $\,\,$ & $\,\,$\color{gr} \color{blue} 4\color{black}  $\,\,$ \\
$\,\,$0.0779$\,\,$ & $\,\,$0.2364$\,\,$ & $\,\,$\color{gr} \color{blue}  1/4\color{black} $\,\,$ & $\,\,$ 1  $\,\,$ \\
\end{pmatrix},
\end{equation*}
\end{example}
\newpage
\begin{example}
\begin{equation*}
\mathbf{A} =
\begin{pmatrix}
$\,\,$ 1 $\,\,$ & $\,\,$5$\,\,$ & $\,\,$3$\,\,$ & $\,\,$8 $\,\,$ \\
$\,\,$ 1/5$\,\,$ & $\,\,$ 1 $\,\,$ & $\,\,$1$\,\,$ & $\,\,$7 $\,\,$ \\
$\,\,$ 1/3$\,\,$ & $\,\,$ 1 $\,\,$ & $\,\,$ 1 $\,\,$ & $\,\,$5 $\,\,$ \\
$\,\,$ 1/8$\,\,$ & $\,\,$ 1/7$\,\,$ & $\,\,$ 1/5$\,\,$ & $\,\,$ 1  $\,\,$ \\
\end{pmatrix},
\qquad
\lambda_{\max} =
4.1907,
\qquad
CR = 0.0719
\end{equation*}

\begin{equation*}
\mathbf{w}^{EM} =
\begin{pmatrix}
0.580448\\
0.189315\\
\color{red} 0.187283\color{black} \\
0.042955
\end{pmatrix}\end{equation*}
\begin{equation*}
\left[ \frac{{w}^{EM}_i}{{w}^{EM}_j} \right] =
\begin{pmatrix}
$\,\,$ 1 $\,\,$ & $\,\,$3.0660$\,\,$ & $\,\,$\color{red} 3.0993\color{black} $\,\,$ & $\,\,$13.5130$\,\,$ \\
$\,\,$0.3262$\,\,$ & $\,\,$ 1 $\,\,$ & $\,\,$\color{red} 1.0109\color{black} $\,\,$ & $\,\,$4.4073  $\,\,$ \\
$\,\,$\color{red} 0.3227\color{black} $\,\,$ & $\,\,$\color{red} 0.9893\color{black} $\,\,$ & $\,\,$ 1 $\,\,$ & $\,\,$\color{red} 4.3600\color{black}  $\,\,$ \\
$\,\,$0.0740$\,\,$ & $\,\,$0.2269$\,\,$ & $\,\,$\color{red} 0.2294\color{black} $\,\,$ & $\,\,$ 1  $\,\,$ \\
\end{pmatrix},
\end{equation*}

\begin{equation*}
\mathbf{w}^{\prime} =
\begin{pmatrix}
0.579271\\
0.188931\\
0.188931\\
0.042868
\end{pmatrix} =
0.997972\cdot
\begin{pmatrix}
0.580448\\
0.189315\\
\color{gr} 0.189315\color{black} \\
0.042955
\end{pmatrix},
\end{equation*}
\begin{equation*}
\left[ \frac{{w}^{\prime}_i}{{w}^{\prime}_j} \right] =
\begin{pmatrix}
$\,\,$ 1 $\,\,$ & $\,\,$3.0660$\,\,$ & $\,\,$\color{gr} 3.0660\color{black} $\,\,$ & $\,\,$13.5130$\,\,$ \\
$\,\,$0.3262$\,\,$ & $\,\,$ 1 $\,\,$ & $\,\,$\color{gr} \color{blue} 1\color{black} $\,\,$ & $\,\,$4.4073  $\,\,$ \\
$\,\,$\color{gr} 0.3262\color{black} $\,\,$ & $\,\,$\color{gr} \color{blue} 1\color{black} $\,\,$ & $\,\,$ 1 $\,\,$ & $\,\,$\color{gr} 4.4073\color{black}  $\,\,$ \\
$\,\,$0.0740$\,\,$ & $\,\,$0.2269$\,\,$ & $\,\,$\color{gr} 0.2269\color{black} $\,\,$ & $\,\,$ 1  $\,\,$ \\
\end{pmatrix},
\end{equation*}
\end{example}
\newpage
\begin{example}
\begin{equation*}
\mathbf{A} =
\begin{pmatrix}
$\,\,$ 1 $\,\,$ & $\,\,$5$\,\,$ & $\,\,$3$\,\,$ & $\,\,$8 $\,\,$ \\
$\,\,$ 1/5$\,\,$ & $\,\,$ 1 $\,\,$ & $\,\,$1$\,\,$ & $\,\,$8 $\,\,$ \\
$\,\,$ 1/3$\,\,$ & $\,\,$ 1 $\,\,$ & $\,\,$ 1 $\,\,$ & $\,\,$5 $\,\,$ \\
$\,\,$ 1/8$\,\,$ & $\,\,$ 1/8$\,\,$ & $\,\,$ 1/5$\,\,$ & $\,\,$ 1  $\,\,$ \\
\end{pmatrix},
\qquad
\lambda_{\max} =
4.2259,
\qquad
CR = 0.0852
\end{equation*}

\begin{equation*}
\mathbf{w}^{EM} =
\begin{pmatrix}
0.578051\\
0.195845\\
\color{red} 0.184667\color{black} \\
0.041437
\end{pmatrix}\end{equation*}
\begin{equation*}
\left[ \frac{{w}^{EM}_i}{{w}^{EM}_j} \right] =
\begin{pmatrix}
$\,\,$ 1 $\,\,$ & $\,\,$2.9516$\,\,$ & $\,\,$\color{red} 3.1302\color{black} $\,\,$ & $\,\,$13.9501$\,\,$ \\
$\,\,$0.3388$\,\,$ & $\,\,$ 1 $\,\,$ & $\,\,$\color{red} 1.0605\color{black} $\,\,$ & $\,\,$4.7263  $\,\,$ \\
$\,\,$\color{red} 0.3195\color{black} $\,\,$ & $\,\,$\color{red} 0.9429\color{black} $\,\,$ & $\,\,$ 1 $\,\,$ & $\,\,$\color{red} 4.4566\color{black}  $\,\,$ \\
$\,\,$0.0717$\,\,$ & $\,\,$0.2116$\,\,$ & $\,\,$\color{red} 0.2244\color{black} $\,\,$ & $\,\,$ 1  $\,\,$ \\
\end{pmatrix},
\end{equation*}

\begin{equation*}
\mathbf{w}^{\prime} =
\begin{pmatrix}
0.573454\\
0.194288\\
0.191151\\
0.041107
\end{pmatrix} =
0.992047\cdot
\begin{pmatrix}
0.578051\\
0.195845\\
\color{gr} 0.192684\color{black} \\
0.041437
\end{pmatrix},
\end{equation*}
\begin{equation*}
\left[ \frac{{w}^{\prime}_i}{{w}^{\prime}_j} \right] =
\begin{pmatrix}
$\,\,$ 1 $\,\,$ & $\,\,$2.9516$\,\,$ & $\,\,$\color{gr} \color{blue} 3\color{black} $\,\,$ & $\,\,$13.9501$\,\,$ \\
$\,\,$0.3388$\,\,$ & $\,\,$ 1 $\,\,$ & $\,\,$\color{gr} 1.0164\color{black} $\,\,$ & $\,\,$4.7263  $\,\,$ \\
$\,\,$\color{gr} \color{blue}  1/3\color{black} $\,\,$ & $\,\,$\color{gr} 0.9839\color{black} $\,\,$ & $\,\,$ 1 $\,\,$ & $\,\,$\color{gr} 4.6500\color{black}  $\,\,$ \\
$\,\,$0.0717$\,\,$ & $\,\,$0.2116$\,\,$ & $\,\,$\color{gr} 0.2151\color{black} $\,\,$ & $\,\,$ 1  $\,\,$ \\
\end{pmatrix},
\end{equation*}
\end{example}
\newpage
\begin{example}
\begin{equation*}
\mathbf{A} =
\begin{pmatrix}
$\,\,$ 1 $\,\,$ & $\,\,$5$\,\,$ & $\,\,$3$\,\,$ & $\,\,$8 $\,\,$ \\
$\,\,$ 1/5$\,\,$ & $\,\,$ 1 $\,\,$ & $\,\,$1$\,\,$ & $\,\,$9 $\,\,$ \\
$\,\,$ 1/3$\,\,$ & $\,\,$ 1 $\,\,$ & $\,\,$ 1 $\,\,$ & $\,\,$5 $\,\,$ \\
$\,\,$ 1/8$\,\,$ & $\,\,$ 1/9$\,\,$ & $\,\,$ 1/5$\,\,$ & $\,\,$ 1  $\,\,$ \\
\end{pmatrix},
\qquad
\lambda_{\max} =
4.2612,
\qquad
CR = 0.0985
\end{equation*}

\begin{equation*}
\mathbf{w}^{EM} =
\begin{pmatrix}
0.575680\\
0.201924\\
\color{red} 0.182273\color{black} \\
0.040123
\end{pmatrix}\end{equation*}
\begin{equation*}
\left[ \frac{{w}^{EM}_i}{{w}^{EM}_j} \right] =
\begin{pmatrix}
$\,\,$ 1 $\,\,$ & $\,\,$2.8510$\,\,$ & $\,\,$\color{red} 3.1583\color{black} $\,\,$ & $\,\,$14.3478$\,\,$ \\
$\,\,$0.3508$\,\,$ & $\,\,$ 1 $\,\,$ & $\,\,$\color{red} 1.1078\color{black} $\,\,$ & $\,\,$5.0326  $\,\,$ \\
$\,\,$\color{red} 0.3166\color{black} $\,\,$ & $\,\,$\color{red} 0.9027\color{black} $\,\,$ & $\,\,$ 1 $\,\,$ & $\,\,$\color{red} 4.5428\color{black}  $\,\,$ \\
$\,\,$0.0697$\,\,$ & $\,\,$0.1987$\,\,$ & $\,\,$\color{red} 0.2201\color{black} $\,\,$ & $\,\,$ 1  $\,\,$ \\
\end{pmatrix},
\end{equation*}

\begin{equation*}
\mathbf{w}^{\prime} =
\begin{pmatrix}
0.570195\\
0.199999\\
0.190065\\
0.039741
\end{pmatrix} =
0.990471\cdot
\begin{pmatrix}
0.575680\\
0.201924\\
\color{gr} 0.191893\color{black} \\
0.040123
\end{pmatrix},
\end{equation*}
\begin{equation*}
\left[ \frac{{w}^{\prime}_i}{{w}^{\prime}_j} \right] =
\begin{pmatrix}
$\,\,$ 1 $\,\,$ & $\,\,$2.8510$\,\,$ & $\,\,$\color{gr} \color{blue} 3\color{black} $\,\,$ & $\,\,$14.3478$\,\,$ \\
$\,\,$0.3508$\,\,$ & $\,\,$ 1 $\,\,$ & $\,\,$\color{gr} 1.0523\color{black} $\,\,$ & $\,\,$5.0326  $\,\,$ \\
$\,\,$\color{gr} \color{blue}  1/3\color{black} $\,\,$ & $\,\,$\color{gr} 0.9503\color{black} $\,\,$ & $\,\,$ 1 $\,\,$ & $\,\,$\color{gr} 4.7826\color{black}  $\,\,$ \\
$\,\,$0.0697$\,\,$ & $\,\,$0.1987$\,\,$ & $\,\,$\color{gr} 0.2091\color{black} $\,\,$ & $\,\,$ 1  $\,\,$ \\
\end{pmatrix},
\end{equation*}
\end{example}
\newpage
\begin{example}
\begin{equation*}
\mathbf{A} =
\begin{pmatrix}
$\,\,$ 1 $\,\,$ & $\,\,$5$\,\,$ & $\,\,$3$\,\,$ & $\,\,$9 $\,\,$ \\
$\,\,$ 1/5$\,\,$ & $\,\,$ 1 $\,\,$ & $\,\,$1$\,\,$ & $\,\,$6 $\,\,$ \\
$\,\,$ 1/3$\,\,$ & $\,\,$ 1 $\,\,$ & $\,\,$ 1 $\,\,$ & $\,\,$4 $\,\,$ \\
$\,\,$ 1/9$\,\,$ & $\,\,$ 1/6$\,\,$ & $\,\,$ 1/4$\,\,$ & $\,\,$ 1  $\,\,$ \\
\end{pmatrix},
\qquad
\lambda_{\max} =
4.1252,
\qquad
CR = 0.0472
\end{equation*}

\begin{equation*}
\mathbf{w}^{EM} =
\begin{pmatrix}
0.593452\\
0.182052\\
\color{red} 0.179342\color{black} \\
0.045154
\end{pmatrix}\end{equation*}
\begin{equation*}
\left[ \frac{{w}^{EM}_i}{{w}^{EM}_j} \right] =
\begin{pmatrix}
$\,\,$ 1 $\,\,$ & $\,\,$3.2598$\,\,$ & $\,\,$\color{red} 3.3090\color{black} $\,\,$ & $\,\,$13.1428$\,\,$ \\
$\,\,$0.3068$\,\,$ & $\,\,$ 1 $\,\,$ & $\,\,$\color{red} 1.0151\color{black} $\,\,$ & $\,\,$4.0318  $\,\,$ \\
$\,\,$\color{red} 0.3022\color{black} $\,\,$ & $\,\,$\color{red} 0.9851\color{black} $\,\,$ & $\,\,$ 1 $\,\,$ & $\,\,$\color{red} 3.9718\color{black}  $\,\,$ \\
$\,\,$0.0761$\,\,$ & $\,\,$0.2480$\,\,$ & $\,\,$\color{red} 0.2518\color{black} $\,\,$ & $\,\,$ 1  $\,\,$ \\
\end{pmatrix},
\end{equation*}

\begin{equation*}
\mathbf{w}^{\prime} =
\begin{pmatrix}
0.592697\\
0.181821\\
0.180386\\
0.045097
\end{pmatrix} =
0.998728\cdot
\begin{pmatrix}
0.593452\\
0.182052\\
\color{gr} 0.180616\color{black} \\
0.045154
\end{pmatrix},
\end{equation*}
\begin{equation*}
\left[ \frac{{w}^{\prime}_i}{{w}^{\prime}_j} \right] =
\begin{pmatrix}
$\,\,$ 1 $\,\,$ & $\,\,$3.2598$\,\,$ & $\,\,$\color{gr} 3.2857\color{black} $\,\,$ & $\,\,$13.1428$\,\,$ \\
$\,\,$0.3068$\,\,$ & $\,\,$ 1 $\,\,$ & $\,\,$\color{gr} 1.0080\color{black} $\,\,$ & $\,\,$4.0318  $\,\,$ \\
$\,\,$\color{gr} 0.3043\color{black} $\,\,$ & $\,\,$\color{gr} 0.9921\color{black} $\,\,$ & $\,\,$ 1 $\,\,$ & $\,\,$\color{gr} \color{blue} 4\color{black}  $\,\,$ \\
$\,\,$0.0761$\,\,$ & $\,\,$0.2480$\,\,$ & $\,\,$\color{gr} \color{blue}  1/4\color{black} $\,\,$ & $\,\,$ 1  $\,\,$ \\
\end{pmatrix},
\end{equation*}
\end{example}
\newpage
\begin{example}
\begin{equation*}
\mathbf{A} =
\begin{pmatrix}
$\,\,$ 1 $\,\,$ & $\,\,$5$\,\,$ & $\,\,$3$\,\,$ & $\,\,$9 $\,\,$ \\
$\,\,$ 1/5$\,\,$ & $\,\,$ 1 $\,\,$ & $\,\,$1$\,\,$ & $\,\,$7 $\,\,$ \\
$\,\,$ 1/3$\,\,$ & $\,\,$ 1 $\,\,$ & $\,\,$ 1 $\,\,$ & $\,\,$5 $\,\,$ \\
$\,\,$ 1/9$\,\,$ & $\,\,$ 1/7$\,\,$ & $\,\,$ 1/5$\,\,$ & $\,\,$ 1  $\,\,$ \\
\end{pmatrix},
\qquad
\lambda_{\max} =
4.1596,
\qquad
CR = 0.0602
\end{equation*}

\begin{equation*}
\mathbf{w}^{EM} =
\begin{pmatrix}
0.587309\\
0.186335\\
\color{red} 0.185533\color{black} \\
0.040822
\end{pmatrix}\end{equation*}
\begin{equation*}
\left[ \frac{{w}^{EM}_i}{{w}^{EM}_j} \right] =
\begin{pmatrix}
$\,\,$ 1 $\,\,$ & $\,\,$3.1519$\,\,$ & $\,\,$\color{red} 3.1655\color{black} $\,\,$ & $\,\,$14.3870$\,\,$ \\
$\,\,$0.3173$\,\,$ & $\,\,$ 1 $\,\,$ & $\,\,$\color{red} 1.0043\color{black} $\,\,$ & $\,\,$4.5646  $\,\,$ \\
$\,\,$\color{red} 0.3159\color{black} $\,\,$ & $\,\,$\color{red} 0.9957\color{black} $\,\,$ & $\,\,$ 1 $\,\,$ & $\,\,$\color{red} 4.5449\color{black}  $\,\,$ \\
$\,\,$0.0695$\,\,$ & $\,\,$0.2191$\,\,$ & $\,\,$\color{red} 0.2200\color{black} $\,\,$ & $\,\,$ 1  $\,\,$ \\
\end{pmatrix},
\end{equation*}

\begin{equation*}
\mathbf{w}^{\prime} =
\begin{pmatrix}
0.586838\\
0.186186\\
0.186186\\
0.040789
\end{pmatrix} =
0.999199\cdot
\begin{pmatrix}
0.587309\\
0.186335\\
\color{gr} 0.186335\color{black} \\
0.040822
\end{pmatrix},
\end{equation*}
\begin{equation*}
\left[ \frac{{w}^{\prime}_i}{{w}^{\prime}_j} \right] =
\begin{pmatrix}
$\,\,$ 1 $\,\,$ & $\,\,$3.1519$\,\,$ & $\,\,$\color{gr} 3.1519\color{black} $\,\,$ & $\,\,$14.3870$\,\,$ \\
$\,\,$0.3173$\,\,$ & $\,\,$ 1 $\,\,$ & $\,\,$\color{gr} \color{blue} 1\color{black} $\,\,$ & $\,\,$4.5646  $\,\,$ \\
$\,\,$\color{gr} 0.3173\color{black} $\,\,$ & $\,\,$\color{gr} \color{blue} 1\color{black} $\,\,$ & $\,\,$ 1 $\,\,$ & $\,\,$\color{gr} 4.5646\color{black}  $\,\,$ \\
$\,\,$0.0695$\,\,$ & $\,\,$0.2191$\,\,$ & $\,\,$\color{gr} 0.2191\color{black} $\,\,$ & $\,\,$ 1  $\,\,$ \\
\end{pmatrix},
\end{equation*}
\end{example}
\newpage
\begin{example}
\begin{equation*}
\mathbf{A} =
\begin{pmatrix}
$\,\,$ 1 $\,\,$ & $\,\,$5$\,\,$ & $\,\,$3$\,\,$ & $\,\,$9 $\,\,$ \\
$\,\,$ 1/5$\,\,$ & $\,\,$ 1 $\,\,$ & $\,\,$1$\,\,$ & $\,\,$8 $\,\,$ \\
$\,\,$ 1/3$\,\,$ & $\,\,$ 1 $\,\,$ & $\,\,$ 1 $\,\,$ & $\,\,$5 $\,\,$ \\
$\,\,$ 1/9$\,\,$ & $\,\,$ 1/8$\,\,$ & $\,\,$ 1/5$\,\,$ & $\,\,$ 1  $\,\,$ \\
\end{pmatrix},
\qquad
\lambda_{\max} =
4.1922,
\qquad
CR = 0.0725
\end{equation*}

\begin{equation*}
\mathbf{w}^{EM} =
\begin{pmatrix}
0.584858\\
0.192671\\
\color{red} 0.183097\color{black} \\
0.039373
\end{pmatrix}\end{equation*}
\begin{equation*}
\left[ \frac{{w}^{EM}_i}{{w}^{EM}_j} \right] =
\begin{pmatrix}
$\,\,$ 1 $\,\,$ & $\,\,$3.0355$\,\,$ & $\,\,$\color{red} 3.1943\color{black} $\,\,$ & $\,\,$14.8543$\,\,$ \\
$\,\,$0.3294$\,\,$ & $\,\,$ 1 $\,\,$ & $\,\,$\color{red} 1.0523\color{black} $\,\,$ & $\,\,$4.8935  $\,\,$ \\
$\,\,$\color{red} 0.3131\color{black} $\,\,$ & $\,\,$\color{red} 0.9503\color{black} $\,\,$ & $\,\,$ 1 $\,\,$ & $\,\,$\color{red} 4.6503\color{black}  $\,\,$ \\
$\,\,$0.0673$\,\,$ & $\,\,$0.2044$\,\,$ & $\,\,$\color{red} 0.2150\color{black} $\,\,$ & $\,\,$ 1  $\,\,$ \\
\end{pmatrix},
\end{equation*}

\begin{equation*}
\mathbf{w}^{\prime} =
\begin{pmatrix}
0.579312\\
0.190844\\
0.190844\\
0.039000
\end{pmatrix} =
0.990517\cdot
\begin{pmatrix}
0.584858\\
0.192671\\
\color{gr} 0.192671\color{black} \\
0.039373
\end{pmatrix},
\end{equation*}
\begin{equation*}
\left[ \frac{{w}^{\prime}_i}{{w}^{\prime}_j} \right] =
\begin{pmatrix}
$\,\,$ 1 $\,\,$ & $\,\,$3.0355$\,\,$ & $\,\,$\color{gr} 3.0355\color{black} $\,\,$ & $\,\,$14.8543$\,\,$ \\
$\,\,$0.3294$\,\,$ & $\,\,$ 1 $\,\,$ & $\,\,$\color{gr} \color{blue} 1\color{black} $\,\,$ & $\,\,$4.8935  $\,\,$ \\
$\,\,$\color{gr} 0.3294\color{black} $\,\,$ & $\,\,$\color{gr} \color{blue} 1\color{black} $\,\,$ & $\,\,$ 1 $\,\,$ & $\,\,$\color{gr} 4.8935\color{black}  $\,\,$ \\
$\,\,$0.0673$\,\,$ & $\,\,$0.2044$\,\,$ & $\,\,$\color{gr} 0.2044\color{black} $\,\,$ & $\,\,$ 1  $\,\,$ \\
\end{pmatrix},
\end{equation*}
\end{example}
\newpage
\begin{example}
\begin{equation*}
\mathbf{A} =
\begin{pmatrix}
$\,\,$ 1 $\,\,$ & $\,\,$5$\,\,$ & $\,\,$3$\,\,$ & $\,\,$9 $\,\,$ \\
$\,\,$ 1/5$\,\,$ & $\,\,$ 1 $\,\,$ & $\,\,$1$\,\,$ & $\,\,$9 $\,\,$ \\
$\,\,$ 1/3$\,\,$ & $\,\,$ 1 $\,\,$ & $\,\,$ 1 $\,\,$ & $\,\,$5 $\,\,$ \\
$\,\,$ 1/9$\,\,$ & $\,\,$ 1/9$\,\,$ & $\,\,$ 1/5$\,\,$ & $\,\,$ 1  $\,\,$ \\
\end{pmatrix},
\qquad
\lambda_{\max} =
4.2253,
\qquad
CR = 0.0849
\end{equation*}

\begin{equation*}
\mathbf{w}^{EM} =
\begin{pmatrix}
0.582445\\
0.198571\\
\color{red} 0.180862\color{black} \\
0.038122
\end{pmatrix}\end{equation*}
\begin{equation*}
\left[ \frac{{w}^{EM}_i}{{w}^{EM}_j} \right] =
\begin{pmatrix}
$\,\,$ 1 $\,\,$ & $\,\,$2.9332$\,\,$ & $\,\,$\color{red} 3.2204\color{black} $\,\,$ & $\,\,$15.2786$\,\,$ \\
$\,\,$0.3409$\,\,$ & $\,\,$ 1 $\,\,$ & $\,\,$\color{red} 1.0979\color{black} $\,\,$ & $\,\,$5.2089  $\,\,$ \\
$\,\,$\color{red} 0.3105\color{black} $\,\,$ & $\,\,$\color{red} 0.9108\color{black} $\,\,$ & $\,\,$ 1 $\,\,$ & $\,\,$\color{red} 4.7444\color{black}  $\,\,$ \\
$\,\,$0.0655$\,\,$ & $\,\,$0.1920$\,\,$ & $\,\,$\color{red} 0.2108\color{black} $\,\,$ & $\,\,$ 1  $\,\,$ \\
\end{pmatrix},
\end{equation*}

\begin{equation*}
\mathbf{w}^{\prime} =
\begin{pmatrix}
0.576823\\
0.196655\\
0.188768\\
0.037754
\end{pmatrix} =
0.990348\cdot
\begin{pmatrix}
0.582445\\
0.198571\\
\color{gr} 0.190608\color{black} \\
0.038122
\end{pmatrix},
\end{equation*}
\begin{equation*}
\left[ \frac{{w}^{\prime}_i}{{w}^{\prime}_j} \right] =
\begin{pmatrix}
$\,\,$ 1 $\,\,$ & $\,\,$2.9332$\,\,$ & $\,\,$\color{gr} 3.0557\color{black} $\,\,$ & $\,\,$15.2786$\,\,$ \\
$\,\,$0.3409$\,\,$ & $\,\,$ 1 $\,\,$ & $\,\,$\color{gr} 1.0418\color{black} $\,\,$ & $\,\,$5.2089  $\,\,$ \\
$\,\,$\color{gr} 0.3273\color{black} $\,\,$ & $\,\,$\color{gr} 0.9599\color{black} $\,\,$ & $\,\,$ 1 $\,\,$ & $\,\,$\color{gr} \color{blue} 5\color{black}  $\,\,$ \\
$\,\,$0.0655$\,\,$ & $\,\,$0.1920$\,\,$ & $\,\,$\color{gr} \color{blue}  1/5\color{black} $\,\,$ & $\,\,$ 1  $\,\,$ \\
\end{pmatrix},
\end{equation*}
\end{example}
\newpage
\begin{example}
\begin{equation*}
\mathbf{A} =
\begin{pmatrix}
$\,\,$ 1 $\,\,$ & $\,\,$5$\,\,$ & $\,\,$3$\,\,$ & $\,\,$9 $\,\,$ \\
$\,\,$ 1/5$\,\,$ & $\,\,$ 1 $\,\,$ & $\,\,$1$\,\,$ & $\,\,$9 $\,\,$ \\
$\,\,$ 1/3$\,\,$ & $\,\,$ 1 $\,\,$ & $\,\,$ 1 $\,\,$ & $\,\,$6 $\,\,$ \\
$\,\,$ 1/9$\,\,$ & $\,\,$ 1/9$\,\,$ & $\,\,$ 1/6$\,\,$ & $\,\,$ 1  $\,\,$ \\
\end{pmatrix},
\qquad
\lambda_{\max} =
4.2277,
\qquad
CR = 0.0859
\end{equation*}

\begin{equation*}
\mathbf{w}^{EM} =
\begin{pmatrix}
0.579657\\
0.195744\\
\color{red} 0.188189\color{black} \\
0.036410
\end{pmatrix}\end{equation*}
\begin{equation*}
\left[ \frac{{w}^{EM}_i}{{w}^{EM}_j} \right] =
\begin{pmatrix}
$\,\,$ 1 $\,\,$ & $\,\,$2.9613$\,\,$ & $\,\,$\color{red} 3.0802\color{black} $\,\,$ & $\,\,$15.9204$\,\,$ \\
$\,\,$0.3377$\,\,$ & $\,\,$ 1 $\,\,$ & $\,\,$\color{red} 1.0401\color{black} $\,\,$ & $\,\,$5.3761  $\,\,$ \\
$\,\,$\color{red} 0.3247\color{black} $\,\,$ & $\,\,$\color{red} 0.9614\color{black} $\,\,$ & $\,\,$ 1 $\,\,$ & $\,\,$\color{red} 5.1686\color{black}  $\,\,$ \\
$\,\,$0.0628$\,\,$ & $\,\,$0.1860$\,\,$ & $\,\,$\color{red} 0.1935\color{black} $\,\,$ & $\,\,$ 1  $\,\,$ \\
\end{pmatrix},
\end{equation*}

\begin{equation*}
\mathbf{w}^{\prime} =
\begin{pmatrix}
0.576756\\
0.194764\\
0.192252\\
0.036227
\end{pmatrix} =
0.994995\cdot
\begin{pmatrix}
0.579657\\
0.195744\\
\color{gr} 0.193219\color{black} \\
0.036410
\end{pmatrix},
\end{equation*}
\begin{equation*}
\left[ \frac{{w}^{\prime}_i}{{w}^{\prime}_j} \right] =
\begin{pmatrix}
$\,\,$ 1 $\,\,$ & $\,\,$2.9613$\,\,$ & $\,\,$\color{gr} \color{blue} 3\color{black} $\,\,$ & $\,\,$15.9204$\,\,$ \\
$\,\,$0.3377$\,\,$ & $\,\,$ 1 $\,\,$ & $\,\,$\color{gr} 1.0131\color{black} $\,\,$ & $\,\,$5.3761  $\,\,$ \\
$\,\,$\color{gr} \color{blue}  1/3\color{black} $\,\,$ & $\,\,$\color{gr} 0.9871\color{black} $\,\,$ & $\,\,$ 1 $\,\,$ & $\,\,$\color{gr} 5.3068\color{black}  $\,\,$ \\
$\,\,$0.0628$\,\,$ & $\,\,$0.1860$\,\,$ & $\,\,$\color{gr} 0.1884\color{black} $\,\,$ & $\,\,$ 1  $\,\,$ \\
\end{pmatrix},
\end{equation*}
\end{example}
\newpage
\begin{example}
\begin{equation*}
\mathbf{A} =
\begin{pmatrix}
$\,\,$ 1 $\,\,$ & $\,\,$5$\,\,$ & $\,\,$4$\,\,$ & $\,\,$6 $\,\,$ \\
$\,\,$ 1/5$\,\,$ & $\,\,$ 1 $\,\,$ & $\,\,$3$\,\,$ & $\,\,$2 $\,\,$ \\
$\,\,$ 1/4$\,\,$ & $\,\,$ 1/3$\,\,$ & $\,\,$ 1 $\,\,$ & $\,\,$1 $\,\,$ \\
$\,\,$ 1/6$\,\,$ & $\,\,$ 1/2$\,\,$ & $\,\,$ 1 $\,\,$ & $\,\,$ 1  $\,\,$ \\
\end{pmatrix},
\qquad
\lambda_{\max} =
4.1502,
\qquad
CR = 0.0566
\end{equation*}

\begin{equation*}
\mathbf{w}^{EM} =
\begin{pmatrix}
0.612924\\
0.193364\\
0.099127\\
\color{red} 0.094585\color{black}
\end{pmatrix}\end{equation*}
\begin{equation*}
\left[ \frac{{w}^{EM}_i}{{w}^{EM}_j} \right] =
\begin{pmatrix}
$\,\,$ 1 $\,\,$ & $\,\,$3.1698$\,\,$ & $\,\,$6.1832$\,\,$ & $\,\,$\color{red} 6.4801\color{black} $\,\,$ \\
$\,\,$0.3155$\,\,$ & $\,\,$ 1 $\,\,$ & $\,\,$1.9507$\,\,$ & $\,\,$\color{red} 2.0443\color{black}   $\,\,$ \\
$\,\,$0.1617$\,\,$ & $\,\,$0.5126$\,\,$ & $\,\,$ 1 $\,\,$ & $\,\,$\color{red} 1.0480\color{black}  $\,\,$ \\
$\,\,$\color{red} 0.1543\color{black} $\,\,$ & $\,\,$\color{red} 0.4892\color{black} $\,\,$ & $\,\,$\color{red} 0.9542\color{black} $\,\,$ & $\,\,$ 1  $\,\,$ \\
\end{pmatrix},
\end{equation*}

\begin{equation*}
\mathbf{w}^{\prime} =
\begin{pmatrix}
0.611642\\
0.192959\\
0.098920\\
0.096480
\end{pmatrix} =
0.997908\cdot
\begin{pmatrix}
0.612924\\
0.193364\\
0.099127\\
\color{gr} 0.096682\color{black}
\end{pmatrix},
\end{equation*}
\begin{equation*}
\left[ \frac{{w}^{\prime}_i}{{w}^{\prime}_j} \right] =
\begin{pmatrix}
$\,\,$ 1 $\,\,$ & $\,\,$3.1698$\,\,$ & $\,\,$6.1832$\,\,$ & $\,\,$\color{gr} 6.3396\color{black} $\,\,$ \\
$\,\,$0.3155$\,\,$ & $\,\,$ 1 $\,\,$ & $\,\,$1.9507$\,\,$ & $\,\,$\color{gr} \color{blue} 2\color{black}   $\,\,$ \\
$\,\,$0.1617$\,\,$ & $\,\,$0.5126$\,\,$ & $\,\,$ 1 $\,\,$ & $\,\,$\color{gr} 1.0253\color{black}  $\,\,$ \\
$\,\,$\color{gr} 0.1577\color{black} $\,\,$ & $\,\,$\color{gr} \color{blue}  1/2\color{black} $\,\,$ & $\,\,$\color{gr} 0.9753\color{black} $\,\,$ & $\,\,$ 1  $\,\,$ \\
\end{pmatrix},
\end{equation*}
\end{example}
\newpage
\begin{example}
\begin{equation*}
\mathbf{A} =
\begin{pmatrix}
$\,\,$ 1 $\,\,$ & $\,\,$5$\,\,$ & $\,\,$4$\,\,$ & $\,\,$7 $\,\,$ \\
$\,\,$ 1/5$\,\,$ & $\,\,$ 1 $\,\,$ & $\,\,$4$\,\,$ & $\,\,$2 $\,\,$ \\
$\,\,$ 1/4$\,\,$ & $\,\,$ 1/4$\,\,$ & $\,\,$ 1 $\,\,$ & $\,\,$1 $\,\,$ \\
$\,\,$ 1/7$\,\,$ & $\,\,$ 1/2$\,\,$ & $\,\,$ 1 $\,\,$ & $\,\,$ 1  $\,\,$ \\
\end{pmatrix},
\qquad
\lambda_{\max} =
4.2287,
\qquad
CR = 0.0862
\end{equation*}

\begin{equation*}
\mathbf{w}^{EM} =
\begin{pmatrix}
0.617780\\
0.204502\\
0.090640\\
\color{red} 0.087078\color{black}
\end{pmatrix}\end{equation*}
\begin{equation*}
\left[ \frac{{w}^{EM}_i}{{w}^{EM}_j} \right] =
\begin{pmatrix}
$\,\,$ 1 $\,\,$ & $\,\,$3.0209$\,\,$ & $\,\,$6.8157$\,\,$ & $\,\,$\color{red} 7.0946\color{black} $\,\,$ \\
$\,\,$0.3310$\,\,$ & $\,\,$ 1 $\,\,$ & $\,\,$2.2562$\,\,$ & $\,\,$\color{red} 2.3485\color{black}   $\,\,$ \\
$\,\,$0.1467$\,\,$ & $\,\,$0.4432$\,\,$ & $\,\,$ 1 $\,\,$ & $\,\,$\color{red} 1.0409\color{black}  $\,\,$ \\
$\,\,$\color{red} 0.1410\color{black} $\,\,$ & $\,\,$\color{red} 0.4258\color{black} $\,\,$ & $\,\,$\color{red} 0.9607\color{black} $\,\,$ & $\,\,$ 1  $\,\,$ \\
\end{pmatrix},
\end{equation*}

\begin{equation*}
\mathbf{w}^{\prime} =
\begin{pmatrix}
0.617054\\
0.204262\\
0.090534\\
0.088151
\end{pmatrix} =
0.998825\cdot
\begin{pmatrix}
0.617780\\
0.204502\\
0.090640\\
\color{gr} 0.088254\color{black}
\end{pmatrix},
\end{equation*}
\begin{equation*}
\left[ \frac{{w}^{\prime}_i}{{w}^{\prime}_j} \right] =
\begin{pmatrix}
$\,\,$ 1 $\,\,$ & $\,\,$3.0209$\,\,$ & $\,\,$6.8157$\,\,$ & $\,\,$\color{gr} \color{blue} 7\color{black} $\,\,$ \\
$\,\,$0.3310$\,\,$ & $\,\,$ 1 $\,\,$ & $\,\,$2.2562$\,\,$ & $\,\,$\color{gr} 2.3172\color{black}   $\,\,$ \\
$\,\,$0.1467$\,\,$ & $\,\,$0.4432$\,\,$ & $\,\,$ 1 $\,\,$ & $\,\,$\color{gr} 1.0270\color{black}  $\,\,$ \\
$\,\,$\color{gr} \color{blue}  1/7\color{black} $\,\,$ & $\,\,$\color{gr} 0.4316\color{black} $\,\,$ & $\,\,$\color{gr} 0.9737\color{black} $\,\,$ & $\,\,$ 1  $\,\,$ \\
\end{pmatrix},
\end{equation*}
\end{example}
\newpage
\begin{example}
\begin{equation*}
\mathbf{A} =
\begin{pmatrix}
$\,\,$ 1 $\,\,$ & $\,\,$5$\,\,$ & $\,\,$4$\,\,$ & $\,\,$9 $\,\,$ \\
$\,\,$ 1/5$\,\,$ & $\,\,$ 1 $\,\,$ & $\,\,$1$\,\,$ & $\,\,$4 $\,\,$ \\
$\,\,$ 1/4$\,\,$ & $\,\,$ 1 $\,\,$ & $\,\,$ 1 $\,\,$ & $\,\,$3 $\,\,$ \\
$\,\,$ 1/9$\,\,$ & $\,\,$ 1/4$\,\,$ & $\,\,$ 1/3$\,\,$ & $\,\,$ 1  $\,\,$ \\
\end{pmatrix},
\qquad
\lambda_{\max} =
4.0539,
\qquad
CR = 0.0203
\end{equation*}

\begin{equation*}
\mathbf{w}^{EM} =
\begin{pmatrix}
0.627790\\
0.162183\\
\color{red} 0.156794\color{black} \\
0.053233
\end{pmatrix}\end{equation*}
\begin{equation*}
\left[ \frac{{w}^{EM}_i}{{w}^{EM}_j} \right] =
\begin{pmatrix}
$\,\,$ 1 $\,\,$ & $\,\,$3.8709$\,\,$ & $\,\,$\color{red} 4.0039\color{black} $\,\,$ & $\,\,$11.7933$\,\,$ \\
$\,\,$0.2583$\,\,$ & $\,\,$ 1 $\,\,$ & $\,\,$\color{red} 1.0344\color{black} $\,\,$ & $\,\,$3.0467  $\,\,$ \\
$\,\,$\color{red} 0.2498\color{black} $\,\,$ & $\,\,$\color{red} 0.9668\color{black} $\,\,$ & $\,\,$ 1 $\,\,$ & $\,\,$\color{red} 2.9455\color{black}  $\,\,$ \\
$\,\,$0.0848$\,\,$ & $\,\,$0.3282$\,\,$ & $\,\,$\color{red} 0.3395\color{black} $\,\,$ & $\,\,$ 1  $\,\,$ \\
\end{pmatrix},
\end{equation*}

\begin{equation*}
\mathbf{w}^{\prime} =
\begin{pmatrix}
0.627694\\
0.162158\\
0.156924\\
0.053224
\end{pmatrix} =
0.999847\cdot
\begin{pmatrix}
0.627790\\
0.162183\\
\color{gr} 0.156948\color{black} \\
0.053233
\end{pmatrix},
\end{equation*}
\begin{equation*}
\left[ \frac{{w}^{\prime}_i}{{w}^{\prime}_j} \right] =
\begin{pmatrix}
$\,\,$ 1 $\,\,$ & $\,\,$3.8709$\,\,$ & $\,\,$\color{gr} \color{blue} 4\color{black} $\,\,$ & $\,\,$11.7933$\,\,$ \\
$\,\,$0.2583$\,\,$ & $\,\,$ 1 $\,\,$ & $\,\,$\color{gr} 1.0334\color{black} $\,\,$ & $\,\,$3.0467  $\,\,$ \\
$\,\,$\color{gr} \color{blue}  1/4\color{black} $\,\,$ & $\,\,$\color{gr} 0.9677\color{black} $\,\,$ & $\,\,$ 1 $\,\,$ & $\,\,$\color{gr} 2.9483\color{black}  $\,\,$ \\
$\,\,$0.0848$\,\,$ & $\,\,$0.3282$\,\,$ & $\,\,$\color{gr} 0.3392\color{black} $\,\,$ & $\,\,$ 1  $\,\,$ \\
\end{pmatrix},
\end{equation*}
\end{example}
\newpage
\begin{example}
\begin{equation*}
\mathbf{A} =
\begin{pmatrix}
$\,\,$ 1 $\,\,$ & $\,\,$5$\,\,$ & $\,\,$5$\,\,$ & $\,\,$7 $\,\,$ \\
$\,\,$ 1/5$\,\,$ & $\,\,$ 1 $\,\,$ & $\,\,$2$\,\,$ & $\,\,$7 $\,\,$ \\
$\,\,$ 1/5$\,\,$ & $\,\,$ 1/2$\,\,$ & $\,\,$ 1 $\,\,$ & $\,\,$2 $\,\,$ \\
$\,\,$ 1/7$\,\,$ & $\,\,$ 1/7$\,\,$ & $\,\,$ 1/2$\,\,$ & $\,\,$ 1  $\,\,$ \\
\end{pmatrix},
\qquad
\lambda_{\max} =
4.2287,
\qquad
CR = 0.0862
\end{equation*}

\begin{equation*}
\mathbf{w}^{EM} =
\begin{pmatrix}
0.620651\\
0.220070\\
\color{red} 0.105710\color{black} \\
0.053569
\end{pmatrix}\end{equation*}
\begin{equation*}
\left[ \frac{{w}^{EM}_i}{{w}^{EM}_j} \right] =
\begin{pmatrix}
$\,\,$ 1 $\,\,$ & $\,\,$2.8202$\,\,$ & $\,\,$\color{red} 5.8713\color{black} $\,\,$ & $\,\,$11.5860$\,\,$ \\
$\,\,$0.3546$\,\,$ & $\,\,$ 1 $\,\,$ & $\,\,$\color{red} 2.0818\color{black} $\,\,$ & $\,\,$4.1081  $\,\,$ \\
$\,\,$\color{red} 0.1703\color{black} $\,\,$ & $\,\,$\color{red} 0.4803\color{black} $\,\,$ & $\,\,$ 1 $\,\,$ & $\,\,$\color{red} 1.9733\color{black}  $\,\,$ \\
$\,\,$0.0863$\,\,$ & $\,\,$0.2434$\,\,$ & $\,\,$\color{red} 0.5068\color{black} $\,\,$ & $\,\,$ 1  $\,\,$ \\
\end{pmatrix},
\end{equation*}

\begin{equation*}
\mathbf{w}^{\prime} =
\begin{pmatrix}
0.619766\\
0.219756\\
0.106986\\
0.053493
\end{pmatrix} =
0.998573\cdot
\begin{pmatrix}
0.620651\\
0.220070\\
\color{gr} 0.107138\color{black} \\
0.053569
\end{pmatrix},
\end{equation*}
\begin{equation*}
\left[ \frac{{w}^{\prime}_i}{{w}^{\prime}_j} \right] =
\begin{pmatrix}
$\,\,$ 1 $\,\,$ & $\,\,$2.8202$\,\,$ & $\,\,$\color{gr} 5.7930\color{black} $\,\,$ & $\,\,$11.5860$\,\,$ \\
$\,\,$0.3546$\,\,$ & $\,\,$ 1 $\,\,$ & $\,\,$\color{gr} 2.0541\color{black} $\,\,$ & $\,\,$4.1081  $\,\,$ \\
$\,\,$\color{gr} 0.1726\color{black} $\,\,$ & $\,\,$\color{gr} 0.4868\color{black} $\,\,$ & $\,\,$ 1 $\,\,$ & $\,\,$\color{gr} \color{blue} 2\color{black}  $\,\,$ \\
$\,\,$0.0863$\,\,$ & $\,\,$0.2434$\,\,$ & $\,\,$\color{gr} \color{blue}  1/2\color{black} $\,\,$ & $\,\,$ 1  $\,\,$ \\
\end{pmatrix},
\end{equation*}
\end{example}
\newpage
\begin{example}
\begin{equation*}
\mathbf{A} =
\begin{pmatrix}
$\,\,$ 1 $\,\,$ & $\,\,$5$\,\,$ & $\,\,$5$\,\,$ & $\,\,$7 $\,\,$ \\
$\,\,$ 1/5$\,\,$ & $\,\,$ 1 $\,\,$ & $\,\,$3$\,\,$ & $\,\,$2 $\,\,$ \\
$\,\,$ 1/5$\,\,$ & $\,\,$ 1/3$\,\,$ & $\,\,$ 1 $\,\,$ & $\,\,$1 $\,\,$ \\
$\,\,$ 1/7$\,\,$ & $\,\,$ 1/2$\,\,$ & $\,\,$ 1 $\,\,$ & $\,\,$ 1  $\,\,$ \\
\end{pmatrix},
\qquad
\lambda_{\max} =
4.1027,
\qquad
CR = 0.0387
\end{equation*}

\begin{equation*}
\mathbf{w}^{EM} =
\begin{pmatrix}
0.638762\\
0.184117\\
0.089269\\
\color{red} 0.087852\color{black}
\end{pmatrix}\end{equation*}
\begin{equation*}
\left[ \frac{{w}^{EM}_i}{{w}^{EM}_j} \right] =
\begin{pmatrix}
$\,\,$ 1 $\,\,$ & $\,\,$3.4693$\,\,$ & $\,\,$7.1555$\,\,$ & $\,\,$\color{red} 7.2709\color{black} $\,\,$ \\
$\,\,$0.2882$\,\,$ & $\,\,$ 1 $\,\,$ & $\,\,$2.0625$\,\,$ & $\,\,$\color{red} 2.0958\color{black}   $\,\,$ \\
$\,\,$0.1398$\,\,$ & $\,\,$0.4848$\,\,$ & $\,\,$ 1 $\,\,$ & $\,\,$\color{red} 1.0161\color{black}  $\,\,$ \\
$\,\,$\color{red} 0.1375\color{black} $\,\,$ & $\,\,$\color{red} 0.4772\color{black} $\,\,$ & $\,\,$\color{red} 0.9841\color{black} $\,\,$ & $\,\,$ 1  $\,\,$ \\
\end{pmatrix},
\end{equation*}

\begin{equation*}
\mathbf{w}^{\prime} =
\begin{pmatrix}
0.637858\\
0.183857\\
0.089143\\
0.089143
\end{pmatrix} =
0.998585\cdot
\begin{pmatrix}
0.638762\\
0.184117\\
0.089269\\
\color{gr} 0.089269\color{black}
\end{pmatrix},
\end{equation*}
\begin{equation*}
\left[ \frac{{w}^{\prime}_i}{{w}^{\prime}_j} \right] =
\begin{pmatrix}
$\,\,$ 1 $\,\,$ & $\,\,$3.4693$\,\,$ & $\,\,$7.1555$\,\,$ & $\,\,$\color{gr} 7.1555\color{black} $\,\,$ \\
$\,\,$0.2882$\,\,$ & $\,\,$ 1 $\,\,$ & $\,\,$2.0625$\,\,$ & $\,\,$\color{gr} 2.0625\color{black}   $\,\,$ \\
$\,\,$0.1398$\,\,$ & $\,\,$0.4848$\,\,$ & $\,\,$ 1 $\,\,$ & $\,\,$\color{gr} \color{blue} 1\color{black}  $\,\,$ \\
$\,\,$\color{gr} 0.1398\color{black} $\,\,$ & $\,\,$\color{gr} 0.4848\color{black} $\,\,$ & $\,\,$\color{gr} \color{blue} 1\color{black} $\,\,$ & $\,\,$ 1  $\,\,$ \\
\end{pmatrix},
\end{equation*}
\end{example}
\newpage
\begin{example}
\begin{equation*}
\mathbf{A} =
\begin{pmatrix}
$\,\,$ 1 $\,\,$ & $\,\,$5$\,\,$ & $\,\,$5$\,\,$ & $\,\,$7 $\,\,$ \\
$\,\,$ 1/5$\,\,$ & $\,\,$ 1 $\,\,$ & $\,\,$5$\,\,$ & $\,\,$3 $\,\,$ \\
$\,\,$ 1/5$\,\,$ & $\,\,$ 1/5$\,\,$ & $\,\,$ 1 $\,\,$ & $\,\,$1 $\,\,$ \\
$\,\,$ 1/7$\,\,$ & $\,\,$ 1/3$\,\,$ & $\,\,$ 1 $\,\,$ & $\,\,$ 1  $\,\,$ \\
\end{pmatrix},
\qquad
\lambda_{\max} =
4.2309,
\qquad
CR = 0.0871
\end{equation*}

\begin{equation*}
\mathbf{w}^{EM} =
\begin{pmatrix}
0.625507\\
0.224654\\
0.075604\\
\color{red} 0.074236\color{black}
\end{pmatrix}\end{equation*}
\begin{equation*}
\left[ \frac{{w}^{EM}_i}{{w}^{EM}_j} \right] =
\begin{pmatrix}
$\,\,$ 1 $\,\,$ & $\,\,$2.7843$\,\,$ & $\,\,$8.2735$\,\,$ & $\,\,$\color{red} 8.4260\color{black} $\,\,$ \\
$\,\,$0.3592$\,\,$ & $\,\,$ 1 $\,\,$ & $\,\,$2.9714$\,\,$ & $\,\,$\color{red} 3.0262\color{black}   $\,\,$ \\
$\,\,$0.1209$\,\,$ & $\,\,$0.3365$\,\,$ & $\,\,$ 1 $\,\,$ & $\,\,$\color{red} 1.0184\color{black}  $\,\,$ \\
$\,\,$\color{red} 0.1187\color{black} $\,\,$ & $\,\,$\color{red} 0.3304\color{black} $\,\,$ & $\,\,$\color{red} 0.9819\color{black} $\,\,$ & $\,\,$ 1  $\,\,$ \\
\end{pmatrix},
\end{equation*}

\begin{equation*}
\mathbf{w}^{\prime} =
\begin{pmatrix}
0.625101\\
0.224508\\
0.075555\\
0.074836
\end{pmatrix} =
0.999352\cdot
\begin{pmatrix}
0.625507\\
0.224654\\
0.075604\\
\color{gr} 0.074885\color{black}
\end{pmatrix},
\end{equation*}
\begin{equation*}
\left[ \frac{{w}^{\prime}_i}{{w}^{\prime}_j} \right] =
\begin{pmatrix}
$\,\,$ 1 $\,\,$ & $\,\,$2.7843$\,\,$ & $\,\,$8.2735$\,\,$ & $\,\,$\color{gr} 8.3530\color{black} $\,\,$ \\
$\,\,$0.3592$\,\,$ & $\,\,$ 1 $\,\,$ & $\,\,$2.9714$\,\,$ & $\,\,$\color{gr} \color{blue} 3\color{black}   $\,\,$ \\
$\,\,$0.1209$\,\,$ & $\,\,$0.3365$\,\,$ & $\,\,$ 1 $\,\,$ & $\,\,$\color{gr} 1.0096\color{black}  $\,\,$ \\
$\,\,$\color{gr} 0.1197\color{black} $\,\,$ & $\,\,$\color{gr} \color{blue}  1/3\color{black} $\,\,$ & $\,\,$\color{gr} 0.9905\color{black} $\,\,$ & $\,\,$ 1  $\,\,$ \\
\end{pmatrix},
\end{equation*}
\end{example}
\newpage
\begin{example}
\begin{equation*}
\mathbf{A} =
\begin{pmatrix}
$\,\,$ 1 $\,\,$ & $\,\,$5$\,\,$ & $\,\,$5$\,\,$ & $\,\,$8 $\,\,$ \\
$\,\,$ 1/5$\,\,$ & $\,\,$ 1 $\,\,$ & $\,\,$2$\,\,$ & $\,\,$9 $\,\,$ \\
$\,\,$ 1/5$\,\,$ & $\,\,$ 1/2$\,\,$ & $\,\,$ 1 $\,\,$ & $\,\,$3 $\,\,$ \\
$\,\,$ 1/8$\,\,$ & $\,\,$ 1/9$\,\,$ & $\,\,$ 1/3$\,\,$ & $\,\,$ 1  $\,\,$ \\
\end{pmatrix},
\qquad
\lambda_{\max} =
4.2637,
\qquad
CR = 0.0994
\end{equation*}

\begin{equation*}
\mathbf{w}^{EM} =
\begin{pmatrix}
0.620611\\
0.224709\\
\color{red} 0.111838\color{black} \\
0.042842
\end{pmatrix}\end{equation*}
\begin{equation*}
\left[ \frac{{w}^{EM}_i}{{w}^{EM}_j} \right] =
\begin{pmatrix}
$\,\,$ 1 $\,\,$ & $\,\,$2.7618$\,\,$ & $\,\,$\color{red} 5.5492\color{black} $\,\,$ & $\,\,$14.4859$\,\,$ \\
$\,\,$0.3621$\,\,$ & $\,\,$ 1 $\,\,$ & $\,\,$\color{red} 2.0092\color{black} $\,\,$ & $\,\,$5.2450  $\,\,$ \\
$\,\,$\color{red} 0.1802\color{black} $\,\,$ & $\,\,$\color{red} 0.4977\color{black} $\,\,$ & $\,\,$ 1 $\,\,$ & $\,\,$\color{red} 2.6105\color{black}  $\,\,$ \\
$\,\,$0.0690$\,\,$ & $\,\,$0.1907$\,\,$ & $\,\,$\color{red} 0.3831\color{black} $\,\,$ & $\,\,$ 1  $\,\,$ \\
\end{pmatrix},
\end{equation*}

\begin{equation*}
\mathbf{w}^{\prime} =
\begin{pmatrix}
0.620290\\
0.224593\\
0.112297\\
0.042820
\end{pmatrix} =
0.999484\cdot
\begin{pmatrix}
0.620611\\
0.224709\\
\color{gr} 0.112355\color{black} \\
0.042842
\end{pmatrix},
\end{equation*}
\begin{equation*}
\left[ \frac{{w}^{\prime}_i}{{w}^{\prime}_j} \right] =
\begin{pmatrix}
$\,\,$ 1 $\,\,$ & $\,\,$2.7618$\,\,$ & $\,\,$\color{gr} 5.5237\color{black} $\,\,$ & $\,\,$14.4859$\,\,$ \\
$\,\,$0.3621$\,\,$ & $\,\,$ 1 $\,\,$ & $\,\,$\color{gr} \color{blue} 2\color{black} $\,\,$ & $\,\,$5.2450  $\,\,$ \\
$\,\,$\color{gr} 0.1810\color{black} $\,\,$ & $\,\,$\color{gr} \color{blue}  1/2\color{black} $\,\,$ & $\,\,$ 1 $\,\,$ & $\,\,$\color{gr} 2.6225\color{black}  $\,\,$ \\
$\,\,$0.0690$\,\,$ & $\,\,$0.1907$\,\,$ & $\,\,$\color{gr} 0.3813\color{black} $\,\,$ & $\,\,$ 1  $\,\,$ \\
\end{pmatrix},
\end{equation*}
\end{example}
\newpage
\begin{example}
\begin{equation*}
\mathbf{A} =
\begin{pmatrix}
$\,\,$ 1 $\,\,$ & $\,\,$5$\,\,$ & $\,\,$5$\,\,$ & $\,\,$8 $\,\,$ \\
$\,\,$ 1/5$\,\,$ & $\,\,$ 1 $\,\,$ & $\,\,$5$\,\,$ & $\,\,$3 $\,\,$ \\
$\,\,$ 1/5$\,\,$ & $\,\,$ 1/5$\,\,$ & $\,\,$ 1 $\,\,$ & $\,\,$1 $\,\,$ \\
$\,\,$ 1/8$\,\,$ & $\,\,$ 1/3$\,\,$ & $\,\,$ 1 $\,\,$ & $\,\,$ 1  $\,\,$ \\
\end{pmatrix},
\qquad
\lambda_{\max} =
4.2259,
\qquad
CR = 0.0852
\end{equation*}

\begin{equation*}
\mathbf{w}^{EM} =
\begin{pmatrix}
0.633582\\
0.220964\\
0.074863\\
\color{red} 0.070590\color{black}
\end{pmatrix}\end{equation*}
\begin{equation*}
\left[ \frac{{w}^{EM}_i}{{w}^{EM}_j} \right] =
\begin{pmatrix}
$\,\,$ 1 $\,\,$ & $\,\,$2.8674$\,\,$ & $\,\,$8.4632$\,\,$ & $\,\,$\color{red} 8.9755\color{black} $\,\,$ \\
$\,\,$0.3488$\,\,$ & $\,\,$ 1 $\,\,$ & $\,\,$2.9516$\,\,$ & $\,\,$\color{red} 3.1302\color{black}   $\,\,$ \\
$\,\,$0.1182$\,\,$ & $\,\,$0.3388$\,\,$ & $\,\,$ 1 $\,\,$ & $\,\,$\color{red} 1.0605\color{black}  $\,\,$ \\
$\,\,$\color{red} 0.1114\color{black} $\,\,$ & $\,\,$\color{red} 0.3195\color{black} $\,\,$ & $\,\,$\color{red} 0.9429\color{black} $\,\,$ & $\,\,$ 1  $\,\,$ \\
\end{pmatrix},
\end{equation*}

\begin{equation*}
\mathbf{w}^{\prime} =
\begin{pmatrix}
0.631647\\
0.220289\\
0.074635\\
0.073430
\end{pmatrix} =
0.996945\cdot
\begin{pmatrix}
0.633582\\
0.220964\\
0.074863\\
\color{gr} 0.073655\color{black}
\end{pmatrix},
\end{equation*}
\begin{equation*}
\left[ \frac{{w}^{\prime}_i}{{w}^{\prime}_j} \right] =
\begin{pmatrix}
$\,\,$ 1 $\,\,$ & $\,\,$2.8674$\,\,$ & $\,\,$8.4632$\,\,$ & $\,\,$\color{gr} 8.6021\color{black} $\,\,$ \\
$\,\,$0.3488$\,\,$ & $\,\,$ 1 $\,\,$ & $\,\,$2.9516$\,\,$ & $\,\,$\color{gr} \color{blue} 3\color{black}   $\,\,$ \\
$\,\,$0.1182$\,\,$ & $\,\,$0.3388$\,\,$ & $\,\,$ 1 $\,\,$ & $\,\,$\color{gr} 1.0164\color{black}  $\,\,$ \\
$\,\,$\color{gr} 0.1163\color{black} $\,\,$ & $\,\,$\color{gr} \color{blue}  1/3\color{black} $\,\,$ & $\,\,$\color{gr} 0.9839\color{black} $\,\,$ & $\,\,$ 1  $\,\,$ \\
\end{pmatrix},
\end{equation*}
\end{example}
\newpage
\begin{example}
\begin{equation*}
\mathbf{A} =
\begin{pmatrix}
$\,\,$ 1 $\,\,$ & $\,\,$5$\,\,$ & $\,\,$5$\,\,$ & $\,\,$9 $\,\,$ \\
$\,\,$ 1/5$\,\,$ & $\,\,$ 1 $\,\,$ & $\,\,$5$\,\,$ & $\,\,$3 $\,\,$ \\
$\,\,$ 1/5$\,\,$ & $\,\,$ 1/5$\,\,$ & $\,\,$ 1 $\,\,$ & $\,\,$1 $\,\,$ \\
$\,\,$ 1/9$\,\,$ & $\,\,$ 1/3$\,\,$ & $\,\,$ 1 $\,\,$ & $\,\,$ 1  $\,\,$ \\
\end{pmatrix},
\qquad
\lambda_{\max} =
4.2253,
\qquad
CR = 0.0849
\end{equation*}

\begin{equation*}
\mathbf{w}^{EM} =
\begin{pmatrix}
0.640735\\
0.217545\\
0.074167\\
\color{red} 0.067553\color{black}
\end{pmatrix}\end{equation*}
\begin{equation*}
\left[ \frac{{w}^{EM}_i}{{w}^{EM}_j} \right] =
\begin{pmatrix}
$\,\,$ 1 $\,\,$ & $\,\,$2.9453$\,\,$ & $\,\,$8.6391$\,\,$ & $\,\,$\color{red} 9.4850\color{black} $\,\,$ \\
$\,\,$0.3395$\,\,$ & $\,\,$ 1 $\,\,$ & $\,\,$2.9332$\,\,$ & $\,\,$\color{red} 3.2204\color{black}   $\,\,$ \\
$\,\,$0.1158$\,\,$ & $\,\,$0.3409$\,\,$ & $\,\,$ 1 $\,\,$ & $\,\,$\color{red} 1.0979\color{black}  $\,\,$ \\
$\,\,$\color{red} 0.1054\color{black} $\,\,$ & $\,\,$\color{red} 0.3105\color{black} $\,\,$ & $\,\,$\color{red} 0.9108\color{black} $\,\,$ & $\,\,$ 1  $\,\,$ \\
\end{pmatrix},
\end{equation*}

\begin{equation*}
\mathbf{w}^{\prime} =
\begin{pmatrix}
0.638411\\
0.216756\\
0.073898\\
0.070935
\end{pmatrix} =
0.996373\cdot
\begin{pmatrix}
0.640735\\
0.217545\\
0.074167\\
\color{gr} 0.071193\color{black}
\end{pmatrix},
\end{equation*}
\begin{equation*}
\left[ \frac{{w}^{\prime}_i}{{w}^{\prime}_j} \right] =
\begin{pmatrix}
$\,\,$ 1 $\,\,$ & $\,\,$2.9453$\,\,$ & $\,\,$8.6391$\,\,$ & $\,\,$\color{gr} \color{blue} 9\color{black} $\,\,$ \\
$\,\,$0.3395$\,\,$ & $\,\,$ 1 $\,\,$ & $\,\,$2.9332$\,\,$ & $\,\,$\color{gr} 3.0557\color{black}   $\,\,$ \\
$\,\,$0.1158$\,\,$ & $\,\,$0.3409$\,\,$ & $\,\,$ 1 $\,\,$ & $\,\,$\color{gr} 1.0418\color{black}  $\,\,$ \\
$\,\,$\color{gr} \color{blue}  1/9\color{black} $\,\,$ & $\,\,$\color{gr} 0.3273\color{black} $\,\,$ & $\,\,$\color{gr} 0.9599\color{black} $\,\,$ & $\,\,$ 1  $\,\,$ \\
\end{pmatrix},
\end{equation*}
\end{example}
\newpage
\begin{example}
\begin{equation*}
\mathbf{A} =
\begin{pmatrix}
$\,\,$ 1 $\,\,$ & $\,\,$5$\,\,$ & $\,\,$6$\,\,$ & $\,\,$6 $\,\,$ \\
$\,\,$ 1/5$\,\,$ & $\,\,$ 1 $\,\,$ & $\,\,$2$\,\,$ & $\,\,$6 $\,\,$ \\
$\,\,$ 1/6$\,\,$ & $\,\,$ 1/2$\,\,$ & $\,\,$ 1 $\,\,$ & $\,\,$2 $\,\,$ \\
$\,\,$ 1/6$\,\,$ & $\,\,$ 1/6$\,\,$ & $\,\,$ 1/2$\,\,$ & $\,\,$ 1  $\,\,$ \\
\end{pmatrix},
\qquad
\lambda_{\max} =
4.2277,
\qquad
CR = 0.0859
\end{equation*}

\begin{equation*}
\mathbf{w}^{EM} =
\begin{pmatrix}
0.627260\\
0.211819\\
\color{red} 0.101822\color{black} \\
0.059100
\end{pmatrix}\end{equation*}
\begin{equation*}
\left[ \frac{{w}^{EM}_i}{{w}^{EM}_j} \right] =
\begin{pmatrix}
$\,\,$ 1 $\,\,$ & $\,\,$2.9613$\,\,$ & $\,\,$\color{red} 6.1604\color{black} $\,\,$ & $\,\,$10.6136$\,\,$ \\
$\,\,$0.3377$\,\,$ & $\,\,$ 1 $\,\,$ & $\,\,$\color{red} 2.0803\color{black} $\,\,$ & $\,\,$3.5841  $\,\,$ \\
$\,\,$\color{red} 0.1623\color{black} $\,\,$ & $\,\,$\color{red} 0.4807\color{black} $\,\,$ & $\,\,$ 1 $\,\,$ & $\,\,$\color{red} 1.7229\color{black}  $\,\,$ \\
$\,\,$0.0942$\,\,$ & $\,\,$0.2790$\,\,$ & $\,\,$\color{red} 0.5804\color{black} $\,\,$ & $\,\,$ 1  $\,\,$ \\
\end{pmatrix},
\end{equation*}

\begin{equation*}
\mathbf{w}^{\prime} =
\begin{pmatrix}
0.625557\\
0.211244\\
0.104260\\
0.058939
\end{pmatrix} =
0.997286\cdot
\begin{pmatrix}
0.627260\\
0.211819\\
\color{gr} 0.104543\color{black} \\
0.059100
\end{pmatrix},
\end{equation*}
\begin{equation*}
\left[ \frac{{w}^{\prime}_i}{{w}^{\prime}_j} \right] =
\begin{pmatrix}
$\,\,$ 1 $\,\,$ & $\,\,$2.9613$\,\,$ & $\,\,$\color{gr} \color{blue} 6\color{black} $\,\,$ & $\,\,$10.6136$\,\,$ \\
$\,\,$0.3377$\,\,$ & $\,\,$ 1 $\,\,$ & $\,\,$\color{gr} 2.0261\color{black} $\,\,$ & $\,\,$3.5841  $\,\,$ \\
$\,\,$\color{gr} \color{blue}  1/6\color{black} $\,\,$ & $\,\,$\color{gr} 0.4936\color{black} $\,\,$ & $\,\,$ 1 $\,\,$ & $\,\,$\color{gr} 1.7689\color{black}  $\,\,$ \\
$\,\,$0.0942$\,\,$ & $\,\,$0.2790$\,\,$ & $\,\,$\color{gr} 0.5653\color{black} $\,\,$ & $\,\,$ 1  $\,\,$ \\
\end{pmatrix},
\end{equation*}
\end{example}
\newpage
\begin{example}
\begin{equation*}
\mathbf{A} =
\begin{pmatrix}
$\,\,$ 1 $\,\,$ & $\,\,$5$\,\,$ & $\,\,$6$\,\,$ & $\,\,$7 $\,\,$ \\
$\,\,$ 1/5$\,\,$ & $\,\,$ 1 $\,\,$ & $\,\,$2$\,\,$ & $\,\,$6 $\,\,$ \\
$\,\,$ 1/6$\,\,$ & $\,\,$ 1/2$\,\,$ & $\,\,$ 1 $\,\,$ & $\,\,$2 $\,\,$ \\
$\,\,$ 1/7$\,\,$ & $\,\,$ 1/6$\,\,$ & $\,\,$ 1/2$\,\,$ & $\,\,$ 1  $\,\,$ \\
\end{pmatrix},
\qquad
\lambda_{\max} =
4.1832,
\qquad
CR = 0.0691
\end{equation*}

\begin{equation*}
\mathbf{w}^{EM} =
\begin{pmatrix}
0.636802\\
0.207344\\
\color{red} 0.100616\color{black} \\
0.055239
\end{pmatrix}\end{equation*}
\begin{equation*}
\left[ \frac{{w}^{EM}_i}{{w}^{EM}_j} \right] =
\begin{pmatrix}
$\,\,$ 1 $\,\,$ & $\,\,$3.0712$\,\,$ & $\,\,$\color{red} 6.3290\color{black} $\,\,$ & $\,\,$11.5282$\,\,$ \\
$\,\,$0.3256$\,\,$ & $\,\,$ 1 $\,\,$ & $\,\,$\color{red} 2.0608\color{black} $\,\,$ & $\,\,$3.7536  $\,\,$ \\
$\,\,$\color{red} 0.1580\color{black} $\,\,$ & $\,\,$\color{red} 0.4853\color{black} $\,\,$ & $\,\,$ 1 $\,\,$ & $\,\,$\color{red} 1.8215\color{black}  $\,\,$ \\
$\,\,$0.0867$\,\,$ & $\,\,$0.2664$\,\,$ & $\,\,$\color{red} 0.5490\color{black} $\,\,$ & $\,\,$ 1  $\,\,$ \\
\end{pmatrix},
\end{equation*}

\begin{equation*}
\mathbf{w}^{\prime} =
\begin{pmatrix}
0.634861\\
0.206712\\
0.103356\\
0.055070
\end{pmatrix} =
0.996953\cdot
\begin{pmatrix}
0.636802\\
0.207344\\
\color{gr} 0.103672\color{black} \\
0.055239
\end{pmatrix},
\end{equation*}
\begin{equation*}
\left[ \frac{{w}^{\prime}_i}{{w}^{\prime}_j} \right] =
\begin{pmatrix}
$\,\,$ 1 $\,\,$ & $\,\,$3.0712$\,\,$ & $\,\,$\color{gr} 6.1425\color{black} $\,\,$ & $\,\,$11.5282$\,\,$ \\
$\,\,$0.3256$\,\,$ & $\,\,$ 1 $\,\,$ & $\,\,$\color{gr} \color{blue} 2\color{black} $\,\,$ & $\,\,$3.7536  $\,\,$ \\
$\,\,$\color{gr} 0.1628\color{black} $\,\,$ & $\,\,$\color{gr} \color{blue}  1/2\color{black} $\,\,$ & $\,\,$ 1 $\,\,$ & $\,\,$\color{gr} 1.8768\color{black}  $\,\,$ \\
$\,\,$0.0867$\,\,$ & $\,\,$0.2664$\,\,$ & $\,\,$\color{gr} 0.5328\color{black} $\,\,$ & $\,\,$ 1  $\,\,$ \\
\end{pmatrix},
\end{equation*}
\end{example}
\newpage
\begin{example}
\begin{equation*}
\mathbf{A} =
\begin{pmatrix}
$\,\,$ 1 $\,\,$ & $\,\,$5$\,\,$ & $\,\,$6$\,\,$ & $\,\,$7 $\,\,$ \\
$\,\,$ 1/5$\,\,$ & $\,\,$ 1 $\,\,$ & $\,\,$2$\,\,$ & $\,\,$7 $\,\,$ \\
$\,\,$ 1/6$\,\,$ & $\,\,$ 1/2$\,\,$ & $\,\,$ 1 $\,\,$ & $\,\,$2 $\,\,$ \\
$\,\,$ 1/7$\,\,$ & $\,\,$ 1/7$\,\,$ & $\,\,$ 1/2$\,\,$ & $\,\,$ 1  $\,\,$ \\
\end{pmatrix},
\qquad
\lambda_{\max} =
4.2251,
\qquad
CR = 0.0849
\end{equation*}

\begin{equation*}
\mathbf{w}^{EM} =
\begin{pmatrix}
0.632785\\
0.215405\\
\color{red} 0.098905\color{black} \\
0.052905
\end{pmatrix}\end{equation*}
\begin{equation*}
\left[ \frac{{w}^{EM}_i}{{w}^{EM}_j} \right] =
\begin{pmatrix}
$\,\,$ 1 $\,\,$ & $\,\,$2.9377$\,\,$ & $\,\,$\color{red} 6.3979\color{black} $\,\,$ & $\,\,$11.9608$\,\,$ \\
$\,\,$0.3404$\,\,$ & $\,\,$ 1 $\,\,$ & $\,\,$\color{red} 2.1779\color{black} $\,\,$ & $\,\,$4.0716  $\,\,$ \\
$\,\,$\color{red} 0.1563\color{black} $\,\,$ & $\,\,$\color{red} 0.4592\color{black} $\,\,$ & $\,\,$ 1 $\,\,$ & $\,\,$\color{red} 1.8695\color{black}  $\,\,$ \\
$\,\,$0.0836$\,\,$ & $\,\,$0.2456$\,\,$ & $\,\,$\color{red} 0.5349\color{black} $\,\,$ & $\,\,$ 1  $\,\,$ \\
\end{pmatrix},
\end{equation*}

\begin{equation*}
\mathbf{w}^{\prime} =
\begin{pmatrix}
0.628662\\
0.214001\\
0.104777\\
0.052560
\end{pmatrix} =
0.993483\cdot
\begin{pmatrix}
0.632785\\
0.215405\\
\color{gr} 0.105464\color{black} \\
0.052905
\end{pmatrix},
\end{equation*}
\begin{equation*}
\left[ \frac{{w}^{\prime}_i}{{w}^{\prime}_j} \right] =
\begin{pmatrix}
$\,\,$ 1 $\,\,$ & $\,\,$2.9377$\,\,$ & $\,\,$\color{gr} \color{blue} 6\color{black} $\,\,$ & $\,\,$11.9608$\,\,$ \\
$\,\,$0.3404$\,\,$ & $\,\,$ 1 $\,\,$ & $\,\,$\color{gr} 2.0424\color{black} $\,\,$ & $\,\,$4.0716  $\,\,$ \\
$\,\,$\color{gr} \color{blue}  1/6\color{black} $\,\,$ & $\,\,$\color{gr} 0.4896\color{black} $\,\,$ & $\,\,$ 1 $\,\,$ & $\,\,$\color{gr} 1.9935\color{black}  $\,\,$ \\
$\,\,$0.0836$\,\,$ & $\,\,$0.2456$\,\,$ & $\,\,$\color{gr} 0.5016\color{black} $\,\,$ & $\,\,$ 1  $\,\,$ \\
\end{pmatrix},
\end{equation*}
\end{example}
\newpage
\begin{example}
\begin{equation*}
\mathbf{A} =
\begin{pmatrix}
$\,\,$ 1 $\,\,$ & $\,\,$5$\,\,$ & $\,\,$6$\,\,$ & $\,\,$8 $\,\,$ \\
$\,\,$ 1/5$\,\,$ & $\,\,$ 1 $\,\,$ & $\,\,$2$\,\,$ & $\,\,$6 $\,\,$ \\
$\,\,$ 1/6$\,\,$ & $\,\,$ 1/2$\,\,$ & $\,\,$ 1 $\,\,$ & $\,\,$2 $\,\,$ \\
$\,\,$ 1/8$\,\,$ & $\,\,$ 1/6$\,\,$ & $\,\,$ 1/2$\,\,$ & $\,\,$ 1  $\,\,$ \\
\end{pmatrix},
\qquad
\lambda_{\max} =
4.1502,
\qquad
CR = 0.0566
\end{equation*}

\begin{equation*}
\mathbf{w}^{EM} =
\begin{pmatrix}
0.644887\\
0.203447\\
\color{red} 0.099518\color{black} \\
0.052148
\end{pmatrix}\end{equation*}
\begin{equation*}
\left[ \frac{{w}^{EM}_i}{{w}^{EM}_j} \right] =
\begin{pmatrix}
$\,\,$ 1 $\,\,$ & $\,\,$3.1698$\,\,$ & $\,\,$\color{red} 6.4801\color{black} $\,\,$ & $\,\,$12.3664$\,\,$ \\
$\,\,$0.3155$\,\,$ & $\,\,$ 1 $\,\,$ & $\,\,$\color{red} 2.0443\color{black} $\,\,$ & $\,\,$3.9013  $\,\,$ \\
$\,\,$\color{red} 0.1543\color{black} $\,\,$ & $\,\,$\color{red} 0.4892\color{black} $\,\,$ & $\,\,$ 1 $\,\,$ & $\,\,$\color{red} 1.9084\color{black}  $\,\,$ \\
$\,\,$0.0809$\,\,$ & $\,\,$0.2563$\,\,$ & $\,\,$\color{red} 0.5240\color{black} $\,\,$ & $\,\,$ 1  $\,\,$ \\
\end{pmatrix},
\end{equation*}

\begin{equation*}
\mathbf{w}^{\prime} =
\begin{pmatrix}
0.643467\\
0.203000\\
0.101500\\
0.052033
\end{pmatrix} =
0.997799\cdot
\begin{pmatrix}
0.644887\\
0.203447\\
\color{gr} 0.101724\color{black} \\
0.052148
\end{pmatrix},
\end{equation*}
\begin{equation*}
\left[ \frac{{w}^{\prime}_i}{{w}^{\prime}_j} \right] =
\begin{pmatrix}
$\,\,$ 1 $\,\,$ & $\,\,$3.1698$\,\,$ & $\,\,$\color{gr} 6.3396\color{black} $\,\,$ & $\,\,$12.3664$\,\,$ \\
$\,\,$0.3155$\,\,$ & $\,\,$ 1 $\,\,$ & $\,\,$\color{gr} \color{blue} 2\color{black} $\,\,$ & $\,\,$3.9013  $\,\,$ \\
$\,\,$\color{gr} 0.1577\color{black} $\,\,$ & $\,\,$\color{gr} \color{blue}  1/2\color{black} $\,\,$ & $\,\,$ 1 $\,\,$ & $\,\,$\color{gr} 1.9507\color{black}  $\,\,$ \\
$\,\,$0.0809$\,\,$ & $\,\,$0.2563$\,\,$ & $\,\,$\color{gr} 0.5126\color{black} $\,\,$ & $\,\,$ 1  $\,\,$ \\
\end{pmatrix},
\end{equation*}
\end{example}
\newpage
\begin{example}
\begin{equation*}
\mathbf{A} =
\begin{pmatrix}
$\,\,$ 1 $\,\,$ & $\,\,$5$\,\,$ & $\,\,$6$\,\,$ & $\,\,$8 $\,\,$ \\
$\,\,$ 1/5$\,\,$ & $\,\,$ 1 $\,\,$ & $\,\,$2$\,\,$ & $\,\,$7 $\,\,$ \\
$\,\,$ 1/6$\,\,$ & $\,\,$ 1/2$\,\,$ & $\,\,$ 1 $\,\,$ & $\,\,$2 $\,\,$ \\
$\,\,$ 1/8$\,\,$ & $\,\,$ 1/7$\,\,$ & $\,\,$ 1/2$\,\,$ & $\,\,$ 1  $\,\,$ \\
\end{pmatrix},
\qquad
\lambda_{\max} =
4.1888,
\qquad
CR = 0.0712
\end{equation*}

\begin{equation*}
\mathbf{w}^{EM} =
\begin{pmatrix}
0.640850\\
0.211261\\
\color{red} 0.097945\color{black} \\
0.049944
\end{pmatrix}\end{equation*}
\begin{equation*}
\left[ \frac{{w}^{EM}_i}{{w}^{EM}_j} \right] =
\begin{pmatrix}
$\,\,$ 1 $\,\,$ & $\,\,$3.0334$\,\,$ & $\,\,$\color{red} 6.5429\color{black} $\,\,$ & $\,\,$12.8315$\,\,$ \\
$\,\,$0.3297$\,\,$ & $\,\,$ 1 $\,\,$ & $\,\,$\color{red} 2.1569\color{black} $\,\,$ & $\,\,$4.2300  $\,\,$ \\
$\,\,$\color{red} 0.1528\color{black} $\,\,$ & $\,\,$\color{red} 0.4636\color{black} $\,\,$ & $\,\,$ 1 $\,\,$ & $\,\,$\color{red} 1.9611\color{black}  $\,\,$ \\
$\,\,$0.0779$\,\,$ & $\,\,$0.2364$\,\,$ & $\,\,$\color{red} 0.5099\color{black} $\,\,$ & $\,\,$ 1  $\,\,$ \\
\end{pmatrix},
\end{equation*}

\begin{equation*}
\mathbf{w}^{\prime} =
\begin{pmatrix}
0.639608\\
0.210852\\
0.099694\\
0.049847
\end{pmatrix} =
0.998062\cdot
\begin{pmatrix}
0.640850\\
0.211261\\
\color{gr} 0.099887\color{black} \\
0.049944
\end{pmatrix},
\end{equation*}
\begin{equation*}
\left[ \frac{{w}^{\prime}_i}{{w}^{\prime}_j} \right] =
\begin{pmatrix}
$\,\,$ 1 $\,\,$ & $\,\,$3.0334$\,\,$ & $\,\,$\color{gr} 6.4157\color{black} $\,\,$ & $\,\,$12.8315$\,\,$ \\
$\,\,$0.3297$\,\,$ & $\,\,$ 1 $\,\,$ & $\,\,$\color{gr} 2.1150\color{black} $\,\,$ & $\,\,$4.2300  $\,\,$ \\
$\,\,$\color{gr} 0.1559\color{black} $\,\,$ & $\,\,$\color{gr} 0.4728\color{black} $\,\,$ & $\,\,$ 1 $\,\,$ & $\,\,$\color{gr} \color{blue} 2\color{black}  $\,\,$ \\
$\,\,$0.0779$\,\,$ & $\,\,$0.2364$\,\,$ & $\,\,$\color{gr} \color{blue}  1/2\color{black} $\,\,$ & $\,\,$ 1  $\,\,$ \\
\end{pmatrix},
\end{equation*}
\end{example}
\newpage
\begin{example}
\begin{equation*}
\mathbf{A} =
\begin{pmatrix}
$\,\,$ 1 $\,\,$ & $\,\,$5$\,\,$ & $\,\,$6$\,\,$ & $\,\,$9 $\,\,$ \\
$\,\,$ 1/5$\,\,$ & $\,\,$ 1 $\,\,$ & $\,\,$2$\,\,$ & $\,\,$6 $\,\,$ \\
$\,\,$ 1/6$\,\,$ & $\,\,$ 1/2$\,\,$ & $\,\,$ 1 $\,\,$ & $\,\,$2 $\,\,$ \\
$\,\,$ 1/9$\,\,$ & $\,\,$ 1/6$\,\,$ & $\,\,$ 1/2$\,\,$ & $\,\,$ 1  $\,\,$ \\
\end{pmatrix},
\qquad
\lambda_{\max} =
4.1252,
\qquad
CR = 0.0472
\end{equation*}

\begin{equation*}
\mathbf{w}^{EM} =
\begin{pmatrix}
0.651909\\
0.199985\\
\color{red} 0.098504\color{black} \\
0.049602
\end{pmatrix}\end{equation*}
\begin{equation*}
\left[ \frac{{w}^{EM}_i}{{w}^{EM}_j} \right] =
\begin{pmatrix}
$\,\,$ 1 $\,\,$ & $\,\,$3.2598$\,\,$ & $\,\,$\color{red} 6.6181\color{black} $\,\,$ & $\,\,$13.1428$\,\,$ \\
$\,\,$0.3068$\,\,$ & $\,\,$ 1 $\,\,$ & $\,\,$\color{red} 2.0302\color{black} $\,\,$ & $\,\,$4.0318  $\,\,$ \\
$\,\,$\color{red} 0.1511\color{black} $\,\,$ & $\,\,$\color{red} 0.4926\color{black} $\,\,$ & $\,\,$ 1 $\,\,$ & $\,\,$\color{red} 1.9859\color{black}  $\,\,$ \\
$\,\,$0.0761$\,\,$ & $\,\,$0.2480$\,\,$ & $\,\,$\color{red} 0.5036\color{black} $\,\,$ & $\,\,$ 1  $\,\,$ \\
\end{pmatrix},
\end{equation*}

\begin{equation*}
\mathbf{w}^{\prime} =
\begin{pmatrix}
0.651453\\
0.199845\\
0.099134\\
0.049567
\end{pmatrix} =
0.999301\cdot
\begin{pmatrix}
0.651909\\
0.199985\\
\color{gr} 0.099204\color{black} \\
0.049602
\end{pmatrix},
\end{equation*}
\begin{equation*}
\left[ \frac{{w}^{\prime}_i}{{w}^{\prime}_j} \right] =
\begin{pmatrix}
$\,\,$ 1 $\,\,$ & $\,\,$3.2598$\,\,$ & $\,\,$\color{gr} 6.5714\color{black} $\,\,$ & $\,\,$13.1428$\,\,$ \\
$\,\,$0.3068$\,\,$ & $\,\,$ 1 $\,\,$ & $\,\,$\color{gr} 2.0159\color{black} $\,\,$ & $\,\,$4.0318  $\,\,$ \\
$\,\,$\color{gr} 0.1522\color{black} $\,\,$ & $\,\,$\color{gr} 0.4961\color{black} $\,\,$ & $\,\,$ 1 $\,\,$ & $\,\,$\color{gr} \color{blue} 2\color{black}  $\,\,$ \\
$\,\,$0.0761$\,\,$ & $\,\,$0.2480$\,\,$ & $\,\,$\color{gr} \color{blue}  1/2\color{black} $\,\,$ & $\,\,$ 1  $\,\,$ \\
\end{pmatrix},
\end{equation*}
\end{example}
\newpage
\begin{example}
\begin{equation*}
\mathbf{A} =
\begin{pmatrix}
$\,\,$ 1 $\,\,$ & $\,\,$5$\,\,$ & $\,\,$6$\,\,$ & $\,\,$9 $\,\,$ \\
$\,\,$ 1/5$\,\,$ & $\,\,$ 1 $\,\,$ & $\,\,$2$\,\,$ & $\,\,$9 $\,\,$ \\
$\,\,$ 1/6$\,\,$ & $\,\,$ 1/2$\,\,$ & $\,\,$ 1 $\,\,$ & $\,\,$3 $\,\,$ \\
$\,\,$ 1/9$\,\,$ & $\,\,$ 1/9$\,\,$ & $\,\,$ 1/3$\,\,$ & $\,\,$ 1  $\,\,$ \\
\end{pmatrix},
\qquad
\lambda_{\max} =
4.2277,
\qquad
CR = 0.0859
\end{equation*}

\begin{equation*}
\mathbf{w}^{EM} =
\begin{pmatrix}
0.639865\\
0.216076\\
\color{red} 0.103868\color{black} \\
0.040192
\end{pmatrix}\end{equation*}
\begin{equation*}
\left[ \frac{{w}^{EM}_i}{{w}^{EM}_j} \right] =
\begin{pmatrix}
$\,\,$ 1 $\,\,$ & $\,\,$2.9613$\,\,$ & $\,\,$\color{red} 6.1604\color{black} $\,\,$ & $\,\,$15.9204$\,\,$ \\
$\,\,$0.3377$\,\,$ & $\,\,$ 1 $\,\,$ & $\,\,$\color{red} 2.0803\color{black} $\,\,$ & $\,\,$5.3761  $\,\,$ \\
$\,\,$\color{red} 0.1623\color{black} $\,\,$ & $\,\,$\color{red} 0.4807\color{black} $\,\,$ & $\,\,$ 1 $\,\,$ & $\,\,$\color{red} 2.5843\color{black}  $\,\,$ \\
$\,\,$0.0628$\,\,$ & $\,\,$0.1860$\,\,$ & $\,\,$\color{red} 0.3869\color{black} $\,\,$ & $\,\,$ 1  $\,\,$ \\
\end{pmatrix},
\end{equation*}

\begin{equation*}
\mathbf{w}^{\prime} =
\begin{pmatrix}
0.638094\\
0.215477\\
0.106349\\
0.040080
\end{pmatrix} =
0.997231\cdot
\begin{pmatrix}
0.639865\\
0.216076\\
\color{gr} 0.106644\color{black} \\
0.040192
\end{pmatrix},
\end{equation*}
\begin{equation*}
\left[ \frac{{w}^{\prime}_i}{{w}^{\prime}_j} \right] =
\begin{pmatrix}
$\,\,$ 1 $\,\,$ & $\,\,$2.9613$\,\,$ & $\,\,$\color{gr} \color{blue} 6\color{black} $\,\,$ & $\,\,$15.9204$\,\,$ \\
$\,\,$0.3377$\,\,$ & $\,\,$ 1 $\,\,$ & $\,\,$\color{gr} 2.0261\color{black} $\,\,$ & $\,\,$5.3761  $\,\,$ \\
$\,\,$\color{gr} \color{blue}  1/6\color{black} $\,\,$ & $\,\,$\color{gr} 0.4936\color{black} $\,\,$ & $\,\,$ 1 $\,\,$ & $\,\,$\color{gr} 2.6534\color{black}  $\,\,$ \\
$\,\,$0.0628$\,\,$ & $\,\,$0.1860$\,\,$ & $\,\,$\color{gr} 0.3769\color{black} $\,\,$ & $\,\,$ 1  $\,\,$ \\
\end{pmatrix},
\end{equation*}
\end{example}
\newpage
\begin{example}
\begin{equation*}
\mathbf{A} =
\begin{pmatrix}
$\,\,$ 1 $\,\,$ & $\,\,$5$\,\,$ & $\,\,$6$\,\,$ & $\,\,$9 $\,\,$ \\
$\,\,$ 1/5$\,\,$ & $\,\,$ 1 $\,\,$ & $\,\,$5$\,\,$ & $\,\,$3 $\,\,$ \\
$\,\,$ 1/6$\,\,$ & $\,\,$ 1/5$\,\,$ & $\,\,$ 1 $\,\,$ & $\,\,$1 $\,\,$ \\
$\,\,$ 1/9$\,\,$ & $\,\,$ 1/3$\,\,$ & $\,\,$ 1 $\,\,$ & $\,\,$ 1  $\,\,$ \\
\end{pmatrix},
\qquad
\lambda_{\max} =
4.1758,
\qquad
CR = 0.0663
\end{equation*}

\begin{equation*}
\mathbf{w}^{EM} =
\begin{pmatrix}
0.652581\\
0.212115\\
0.068607\\
\color{red} 0.066698\color{black}
\end{pmatrix}\end{equation*}
\begin{equation*}
\left[ \frac{{w}^{EM}_i}{{w}^{EM}_j} \right] =
\begin{pmatrix}
$\,\,$ 1 $\,\,$ & $\,\,$3.0765$\,\,$ & $\,\,$9.5119$\,\,$ & $\,\,$\color{red} 9.7842\color{black} $\,\,$ \\
$\,\,$0.3250$\,\,$ & $\,\,$ 1 $\,\,$ & $\,\,$3.0917$\,\,$ & $\,\,$\color{red} 3.1802\color{black}   $\,\,$ \\
$\,\,$0.1051$\,\,$ & $\,\,$0.3234$\,\,$ & $\,\,$ 1 $\,\,$ & $\,\,$\color{red} 1.0286\color{black}  $\,\,$ \\
$\,\,$\color{red} 0.1022\color{black} $\,\,$ & $\,\,$\color{red} 0.3144\color{black} $\,\,$ & $\,\,$\color{red} 0.9722\color{black} $\,\,$ & $\,\,$ 1  $\,\,$ \\
\end{pmatrix},
\end{equation*}

\begin{equation*}
\mathbf{w}^{\prime} =
\begin{pmatrix}
0.651337\\
0.211711\\
0.068476\\
0.068476
\end{pmatrix} =
0.998094\cdot
\begin{pmatrix}
0.652581\\
0.212115\\
0.068607\\
\color{gr} 0.068607\color{black}
\end{pmatrix},
\end{equation*}
\begin{equation*}
\left[ \frac{{w}^{\prime}_i}{{w}^{\prime}_j} \right] =
\begin{pmatrix}
$\,\,$ 1 $\,\,$ & $\,\,$3.0765$\,\,$ & $\,\,$9.5119$\,\,$ & $\,\,$\color{gr} 9.5119\color{black} $\,\,$ \\
$\,\,$0.3250$\,\,$ & $\,\,$ 1 $\,\,$ & $\,\,$3.0917$\,\,$ & $\,\,$\color{gr} 3.0917\color{black}   $\,\,$ \\
$\,\,$0.1051$\,\,$ & $\,\,$0.3234$\,\,$ & $\,\,$ 1 $\,\,$ & $\,\,$\color{gr} \color{blue} 1\color{black}  $\,\,$ \\
$\,\,$\color{gr} 0.1051\color{black} $\,\,$ & $\,\,$\color{gr} 0.3234\color{black} $\,\,$ & $\,\,$\color{gr} \color{blue} 1\color{black} $\,\,$ & $\,\,$ 1  $\,\,$ \\
\end{pmatrix},
\end{equation*}
\end{example}
\newpage
\begin{example}
\begin{equation*}
\mathbf{A} =
\begin{pmatrix}
$\,\,$ 1 $\,\,$ & $\,\,$5$\,\,$ & $\,\,$7$\,\,$ & $\,\,$8 $\,\,$ \\
$\,\,$ 1/5$\,\,$ & $\,\,$ 1 $\,\,$ & $\,\,$2$\,\,$ & $\,\,$7 $\,\,$ \\
$\,\,$ 1/7$\,\,$ & $\,\,$ 1/2$\,\,$ & $\,\,$ 1 $\,\,$ & $\,\,$2 $\,\,$ \\
$\,\,$ 1/8$\,\,$ & $\,\,$ 1/7$\,\,$ & $\,\,$ 1/2$\,\,$ & $\,\,$ 1  $\,\,$ \\
\end{pmatrix},
\qquad
\lambda_{\max} =
4.1897,
\qquad
CR = 0.0715
\end{equation*}

\begin{equation*}
\mathbf{w}^{EM} =
\begin{pmatrix}
0.651171\\
0.207015\\
\color{red} 0.092521\color{black} \\
0.049293
\end{pmatrix}\end{equation*}
\begin{equation*}
\left[ \frac{{w}^{EM}_i}{{w}^{EM}_j} \right] =
\begin{pmatrix}
$\,\,$ 1 $\,\,$ & $\,\,$3.1455$\,\,$ & $\,\,$\color{red} 7.0381\color{black} $\,\,$ & $\,\,$13.2103$\,\,$ \\
$\,\,$0.3179$\,\,$ & $\,\,$ 1 $\,\,$ & $\,\,$\color{red} 2.2375\color{black} $\,\,$ & $\,\,$4.1997  $\,\,$ \\
$\,\,$\color{red} 0.1421\color{black} $\,\,$ & $\,\,$\color{red} 0.4469\color{black} $\,\,$ & $\,\,$ 1 $\,\,$ & $\,\,$\color{red} 1.8770\color{black}  $\,\,$ \\
$\,\,$0.0757$\,\,$ & $\,\,$0.2381$\,\,$ & $\,\,$\color{red} 0.5328\color{black} $\,\,$ & $\,\,$ 1  $\,\,$ \\
\end{pmatrix},
\end{equation*}

\begin{equation*}
\mathbf{w}^{\prime} =
\begin{pmatrix}
0.650843\\
0.206911\\
0.092978\\
0.049268
\end{pmatrix} =
0.999497\cdot
\begin{pmatrix}
0.651171\\
0.207015\\
\color{gr} 0.093024\color{black} \\
0.049293
\end{pmatrix},
\end{equation*}
\begin{equation*}
\left[ \frac{{w}^{\prime}_i}{{w}^{\prime}_j} \right] =
\begin{pmatrix}
$\,\,$ 1 $\,\,$ & $\,\,$3.1455$\,\,$ & $\,\,$\color{gr} \color{blue} 7\color{black} $\,\,$ & $\,\,$13.2103$\,\,$ \\
$\,\,$0.3179$\,\,$ & $\,\,$ 1 $\,\,$ & $\,\,$\color{gr} 2.2254\color{black} $\,\,$ & $\,\,$4.1997  $\,\,$ \\
$\,\,$\color{gr} \color{blue}  1/7\color{black} $\,\,$ & $\,\,$\color{gr} 0.4494\color{black} $\,\,$ & $\,\,$ 1 $\,\,$ & $\,\,$\color{gr} 1.8872\color{black}  $\,\,$ \\
$\,\,$0.0757$\,\,$ & $\,\,$0.2381$\,\,$ & $\,\,$\color{gr} 0.5299\color{black} $\,\,$ & $\,\,$ 1  $\,\,$ \\
\end{pmatrix},
\end{equation*}
\end{example}
\newpage
\begin{example}
\begin{equation*}
\mathbf{A} =
\begin{pmatrix}
$\,\,$ 1 $\,\,$ & $\,\,$5$\,\,$ & $\,\,$7$\,\,$ & $\,\,$8 $\,\,$ \\
$\,\,$ 1/5$\,\,$ & $\,\,$ 1 $\,\,$ & $\,\,$2$\,\,$ & $\,\,$8 $\,\,$ \\
$\,\,$ 1/7$\,\,$ & $\,\,$ 1/2$\,\,$ & $\,\,$ 1 $\,\,$ & $\,\,$2 $\,\,$ \\
$\,\,$ 1/8$\,\,$ & $\,\,$ 1/8$\,\,$ & $\,\,$ 1/2$\,\,$ & $\,\,$ 1  $\,\,$ \\
\end{pmatrix},
\qquad
\lambda_{\max} =
4.2287,
\qquad
CR = 0.0862
\end{equation*}

\begin{equation*}
\mathbf{w}^{EM} =
\begin{pmatrix}
0.647107\\
0.214210\\
\color{red} 0.091211\color{black} \\
0.047472
\end{pmatrix}\end{equation*}
\begin{equation*}
\left[ \frac{{w}^{EM}_i}{{w}^{EM}_j} \right] =
\begin{pmatrix}
$\,\,$ 1 $\,\,$ & $\,\,$3.0209$\,\,$ & $\,\,$\color{red} 7.0946\color{black} $\,\,$ & $\,\,$13.6315$\,\,$ \\
$\,\,$0.3310$\,\,$ & $\,\,$ 1 $\,\,$ & $\,\,$\color{red} 2.3485\color{black} $\,\,$ & $\,\,$4.5124  $\,\,$ \\
$\,\,$\color{red} 0.1410\color{black} $\,\,$ & $\,\,$\color{red} 0.4258\color{black} $\,\,$ & $\,\,$ 1 $\,\,$ & $\,\,$\color{red} 1.9214\color{black}  $\,\,$ \\
$\,\,$0.0734$\,\,$ & $\,\,$0.2216$\,\,$ & $\,\,$\color{red} 0.5205\color{black} $\,\,$ & $\,\,$ 1  $\,\,$ \\
\end{pmatrix},
\end{equation*}

\begin{equation*}
\mathbf{w}^{\prime} =
\begin{pmatrix}
0.646310\\
0.213946\\
0.092330\\
0.047413
\end{pmatrix} =
0.998769\cdot
\begin{pmatrix}
0.647107\\
0.214210\\
\color{gr} 0.092444\color{black} \\
0.047472
\end{pmatrix},
\end{equation*}
\begin{equation*}
\left[ \frac{{w}^{\prime}_i}{{w}^{\prime}_j} \right] =
\begin{pmatrix}
$\,\,$ 1 $\,\,$ & $\,\,$3.0209$\,\,$ & $\,\,$\color{gr} \color{blue} 7\color{black} $\,\,$ & $\,\,$13.6315$\,\,$ \\
$\,\,$0.3310$\,\,$ & $\,\,$ 1 $\,\,$ & $\,\,$\color{gr} 2.3172\color{black} $\,\,$ & $\,\,$4.5124  $\,\,$ \\
$\,\,$\color{gr} \color{blue}  1/7\color{black} $\,\,$ & $\,\,$\color{gr} 0.4316\color{black} $\,\,$ & $\,\,$ 1 $\,\,$ & $\,\,$\color{gr} 1.9474\color{black}  $\,\,$ \\
$\,\,$0.0734$\,\,$ & $\,\,$0.2216$\,\,$ & $\,\,$\color{gr} 0.5135\color{black} $\,\,$ & $\,\,$ 1  $\,\,$ \\
\end{pmatrix},
\end{equation*}
\end{example}
\newpage
\begin{example}
\begin{equation*}
\mathbf{A} =
\begin{pmatrix}
$\,\,$ 1 $\,\,$ & $\,\,$5$\,\,$ & $\,\,$7$\,\,$ & $\,\,$9 $\,\,$ \\
$\,\,$ 1/5$\,\,$ & $\,\,$ 1 $\,\,$ & $\,\,$2$\,\,$ & $\,\,$6 $\,\,$ \\
$\,\,$ 1/7$\,\,$ & $\,\,$ 1/2$\,\,$ & $\,\,$ 1 $\,\,$ & $\,\,$2 $\,\,$ \\
$\,\,$ 1/9$\,\,$ & $\,\,$ 1/6$\,\,$ & $\,\,$ 1/2$\,\,$ & $\,\,$ 1  $\,\,$ \\
\end{pmatrix},
\qquad
\lambda_{\max} =
4.1239,
\qquad
CR = 0.0467
\end{equation*}

\begin{equation*}
\mathbf{w}^{EM} =
\begin{pmatrix}
0.662413\\
0.195784\\
\color{red} 0.092923\color{black} \\
0.048880
\end{pmatrix}\end{equation*}
\begin{equation*}
\left[ \frac{{w}^{EM}_i}{{w}^{EM}_j} \right] =
\begin{pmatrix}
$\,\,$ 1 $\,\,$ & $\,\,$3.3834$\,\,$ & $\,\,$\color{red} 7.1286\color{black} $\,\,$ & $\,\,$13.5520$\,\,$ \\
$\,\,$0.2956$\,\,$ & $\,\,$ 1 $\,\,$ & $\,\,$\color{red} 2.1069\color{black} $\,\,$ & $\,\,$4.0054  $\,\,$ \\
$\,\,$\color{red} 0.1403\color{black} $\,\,$ & $\,\,$\color{red} 0.4746\color{black} $\,\,$ & $\,\,$ 1 $\,\,$ & $\,\,$\color{red} 1.9011\color{black}  $\,\,$ \\
$\,\,$0.0738$\,\,$ & $\,\,$0.2497$\,\,$ & $\,\,$\color{red} 0.5260\color{black} $\,\,$ & $\,\,$ 1  $\,\,$ \\
\end{pmatrix},
\end{equation*}

\begin{equation*}
\mathbf{w}^{\prime} =
\begin{pmatrix}
0.661284\\
0.195450\\
0.094469\\
0.048796
\end{pmatrix} =
0.998296\cdot
\begin{pmatrix}
0.662413\\
0.195784\\
\color{gr} 0.094630\color{black} \\
0.048880
\end{pmatrix},
\end{equation*}
\begin{equation*}
\left[ \frac{{w}^{\prime}_i}{{w}^{\prime}_j} \right] =
\begin{pmatrix}
$\,\,$ 1 $\,\,$ & $\,\,$3.3834$\,\,$ & $\,\,$\color{gr} \color{blue} 7\color{black} $\,\,$ & $\,\,$13.5520$\,\,$ \\
$\,\,$0.2956$\,\,$ & $\,\,$ 1 $\,\,$ & $\,\,$\color{gr} 2.0689\color{black} $\,\,$ & $\,\,$4.0054  $\,\,$ \\
$\,\,$\color{gr} \color{blue}  1/7\color{black} $\,\,$ & $\,\,$\color{gr} 0.4833\color{black} $\,\,$ & $\,\,$ 1 $\,\,$ & $\,\,$\color{gr} 1.9360\color{black}  $\,\,$ \\
$\,\,$0.0738$\,\,$ & $\,\,$0.2497$\,\,$ & $\,\,$\color{gr} 0.5165\color{black} $\,\,$ & $\,\,$ 1  $\,\,$ \\
\end{pmatrix},
\end{equation*}
\end{example}
\newpage
\begin{example}
\begin{equation*}
\mathbf{A} =
\begin{pmatrix}
$\,\,$ 1 $\,\,$ & $\,\,$5$\,\,$ & $\,\,$7$\,\,$ & $\,\,$9 $\,\,$ \\
$\,\,$ 1/5$\,\,$ & $\,\,$ 1 $\,\,$ & $\,\,$2$\,\,$ & $\,\,$7 $\,\,$ \\
$\,\,$ 1/7$\,\,$ & $\,\,$ 1/2$\,\,$ & $\,\,$ 1 $\,\,$ & $\,\,$2 $\,\,$ \\
$\,\,$ 1/9$\,\,$ & $\,\,$ 1/7$\,\,$ & $\,\,$ 1/2$\,\,$ & $\,\,$ 1  $\,\,$ \\
\end{pmatrix},
\qquad
\lambda_{\max} =
4.1597,
\qquad
CR = 0.0602
\end{equation*}

\begin{equation*}
\mathbf{w}^{EM} =
\begin{pmatrix}
0.658179\\
0.203396\\
\color{red} 0.091590\color{black} \\
0.046835
\end{pmatrix}\end{equation*}
\begin{equation*}
\left[ \frac{{w}^{EM}_i}{{w}^{EM}_j} \right] =
\begin{pmatrix}
$\,\,$ 1 $\,\,$ & $\,\,$3.2360$\,\,$ & $\,\,$\color{red} 7.1861\color{black} $\,\,$ & $\,\,$14.0532$\,\,$ \\
$\,\,$0.3090$\,\,$ & $\,\,$ 1 $\,\,$ & $\,\,$\color{red} 2.2207\color{black} $\,\,$ & $\,\,$4.3428  $\,\,$ \\
$\,\,$\color{red} 0.1392\color{black} $\,\,$ & $\,\,$\color{red} 0.4503\color{black} $\,\,$ & $\,\,$ 1 $\,\,$ & $\,\,$\color{red} 1.9556\color{black}  $\,\,$ \\
$\,\,$0.0712$\,\,$ & $\,\,$0.2303$\,\,$ & $\,\,$\color{red} 0.5114\color{black} $\,\,$ & $\,\,$ 1  $\,\,$ \\
\end{pmatrix},
\end{equation*}

\begin{equation*}
\mathbf{w}^{\prime} =
\begin{pmatrix}
0.656813\\
0.202973\\
0.093476\\
0.046738
\end{pmatrix} =
0.997924\cdot
\begin{pmatrix}
0.658179\\
0.203396\\
\color{gr} 0.093670\color{black} \\
0.046835
\end{pmatrix},
\end{equation*}
\begin{equation*}
\left[ \frac{{w}^{\prime}_i}{{w}^{\prime}_j} \right] =
\begin{pmatrix}
$\,\,$ 1 $\,\,$ & $\,\,$3.2360$\,\,$ & $\,\,$\color{gr} 7.0266\color{black} $\,\,$ & $\,\,$14.0532$\,\,$ \\
$\,\,$0.3090$\,\,$ & $\,\,$ 1 $\,\,$ & $\,\,$\color{gr} 2.1714\color{black} $\,\,$ & $\,\,$4.3428  $\,\,$ \\
$\,\,$\color{gr} 0.1423\color{black} $\,\,$ & $\,\,$\color{gr} 0.4605\color{black} $\,\,$ & $\,\,$ 1 $\,\,$ & $\,\,$\color{gr} \color{blue} 2\color{black}  $\,\,$ \\
$\,\,$0.0712$\,\,$ & $\,\,$0.2303$\,\,$ & $\,\,$\color{gr} \color{blue}  1/2\color{black} $\,\,$ & $\,\,$ 1  $\,\,$ \\
\end{pmatrix},
\end{equation*}
\end{example}
\newpage
\begin{example}
\begin{equation*}
\mathbf{A} =
\begin{pmatrix}
$\,\,$ 1 $\,\,$ & $\,\,$6$\,\,$ & $\,\,$3$\,\,$ & $\,\,$6 $\,\,$ \\
$\,\,$ 1/6$\,\,$ & $\,\,$ 1 $\,\,$ & $\,\,$1$\,\,$ & $\,\,$5 $\,\,$ \\
$\,\,$ 1/3$\,\,$ & $\,\,$ 1 $\,\,$ & $\,\,$ 1 $\,\,$ & $\,\,$3 $\,\,$ \\
$\,\,$ 1/6$\,\,$ & $\,\,$ 1/5$\,\,$ & $\,\,$ 1/3$\,\,$ & $\,\,$ 1  $\,\,$ \\
\end{pmatrix},
\qquad
\lambda_{\max} =
4.2277,
\qquad
CR = 0.0859
\end{equation*}

\begin{equation*}
\mathbf{w}^{EM} =
\begin{pmatrix}
0.594697\\
0.175330\\
\color{red} 0.170765\color{black} \\
0.059207
\end{pmatrix}\end{equation*}
\begin{equation*}
\left[ \frac{{w}^{EM}_i}{{w}^{EM}_j} \right] =
\begin{pmatrix}
$\,\,$ 1 $\,\,$ & $\,\,$3.3919$\,\,$ & $\,\,$\color{red} 3.4825\color{black} $\,\,$ & $\,\,$10.0444$\,\,$ \\
$\,\,$0.2948$\,\,$ & $\,\,$ 1 $\,\,$ & $\,\,$\color{red} 1.0267\color{black} $\,\,$ & $\,\,$2.9613  $\,\,$ \\
$\,\,$\color{red} 0.2871\color{black} $\,\,$ & $\,\,$\color{red} 0.9740\color{black} $\,\,$ & $\,\,$ 1 $\,\,$ & $\,\,$\color{red} 2.8842\color{black}  $\,\,$ \\
$\,\,$0.0996$\,\,$ & $\,\,$0.3377$\,\,$ & $\,\,$\color{red} 0.3467\color{black} $\,\,$ & $\,\,$ 1  $\,\,$ \\
\end{pmatrix},
\end{equation*}

\begin{equation*}
\mathbf{w}^{\prime} =
\begin{pmatrix}
0.591995\\
0.174533\\
0.174533\\
0.058938
\end{pmatrix} =
0.995456\cdot
\begin{pmatrix}
0.594697\\
0.175330\\
\color{gr} 0.175330\color{black} \\
0.059207
\end{pmatrix},
\end{equation*}
\begin{equation*}
\left[ \frac{{w}^{\prime}_i}{{w}^{\prime}_j} \right] =
\begin{pmatrix}
$\,\,$ 1 $\,\,$ & $\,\,$3.3919$\,\,$ & $\,\,$\color{gr} 3.3919\color{black} $\,\,$ & $\,\,$10.0444$\,\,$ \\
$\,\,$0.2948$\,\,$ & $\,\,$ 1 $\,\,$ & $\,\,$\color{gr} \color{blue} 1\color{black} $\,\,$ & $\,\,$2.9613  $\,\,$ \\
$\,\,$\color{gr} 0.2948\color{black} $\,\,$ & $\,\,$\color{gr} \color{blue} 1\color{black} $\,\,$ & $\,\,$ 1 $\,\,$ & $\,\,$\color{gr} 2.9613\color{black}  $\,\,$ \\
$\,\,$0.0996$\,\,$ & $\,\,$0.3377$\,\,$ & $\,\,$\color{gr} 0.3377\color{black} $\,\,$ & $\,\,$ 1  $\,\,$ \\
\end{pmatrix},
\end{equation*}
\end{example}
\newpage
\begin{example}
\begin{equation*}
\mathbf{A} =
\begin{pmatrix}
$\,\,$ 1 $\,\,$ & $\,\,$6$\,\,$ & $\,\,$3$\,\,$ & $\,\,$8 $\,\,$ \\
$\,\,$ 1/6$\,\,$ & $\,\,$ 1 $\,\,$ & $\,\,$1$\,\,$ & $\,\,$7 $\,\,$ \\
$\,\,$ 1/3$\,\,$ & $\,\,$ 1 $\,\,$ & $\,\,$ 1 $\,\,$ & $\,\,$4 $\,\,$ \\
$\,\,$ 1/8$\,\,$ & $\,\,$ 1/7$\,\,$ & $\,\,$ 1/4$\,\,$ & $\,\,$ 1  $\,\,$ \\
\end{pmatrix},
\qquad
\lambda_{\max} =
4.2421,
\qquad
CR = 0.0913
\end{equation*}

\begin{equation*}
\mathbf{w}^{EM} =
\begin{pmatrix}
0.602851\\
0.180220\\
\color{red} 0.172448\color{black} \\
0.044481
\end{pmatrix}\end{equation*}
\begin{equation*}
\left[ \frac{{w}^{EM}_i}{{w}^{EM}_j} \right] =
\begin{pmatrix}
$\,\,$ 1 $\,\,$ & $\,\,$3.3451$\,\,$ & $\,\,$\color{red} 3.4958\color{black} $\,\,$ & $\,\,$13.5529$\,\,$ \\
$\,\,$0.2989$\,\,$ & $\,\,$ 1 $\,\,$ & $\,\,$\color{red} 1.0451\color{black} $\,\,$ & $\,\,$4.0516  $\,\,$ \\
$\,\,$\color{red} 0.2861\color{black} $\,\,$ & $\,\,$\color{red} 0.9569\color{black} $\,\,$ & $\,\,$ 1 $\,\,$ & $\,\,$\color{red} 3.8768\color{black}  $\,\,$ \\
$\,\,$0.0738$\,\,$ & $\,\,$0.2468$\,\,$ & $\,\,$\color{red} 0.2579\color{black} $\,\,$ & $\,\,$ 1  $\,\,$ \\
\end{pmatrix},
\end{equation*}

\begin{equation*}
\mathbf{w}^{\prime} =
\begin{pmatrix}
0.599567\\
0.179238\\
0.176956\\
0.044239
\end{pmatrix} =
0.994552\cdot
\begin{pmatrix}
0.602851\\
0.180220\\
\color{gr} 0.177926\color{black} \\
0.044481
\end{pmatrix},
\end{equation*}
\begin{equation*}
\left[ \frac{{w}^{\prime}_i}{{w}^{\prime}_j} \right] =
\begin{pmatrix}
$\,\,$ 1 $\,\,$ & $\,\,$3.3451$\,\,$ & $\,\,$\color{gr} 3.3882\color{black} $\,\,$ & $\,\,$13.5529$\,\,$ \\
$\,\,$0.2989$\,\,$ & $\,\,$ 1 $\,\,$ & $\,\,$\color{gr} 1.0129\color{black} $\,\,$ & $\,\,$4.0516  $\,\,$ \\
$\,\,$\color{gr} 0.2951\color{black} $\,\,$ & $\,\,$\color{gr} 0.9873\color{black} $\,\,$ & $\,\,$ 1 $\,\,$ & $\,\,$\color{gr} \color{blue} 4\color{black}  $\,\,$ \\
$\,\,$0.0738$\,\,$ & $\,\,$0.2468$\,\,$ & $\,\,$\color{gr} \color{blue}  1/4\color{black} $\,\,$ & $\,\,$ 1  $\,\,$ \\
\end{pmatrix},
\end{equation*}
\end{example}
\newpage
\begin{example}
\begin{equation*}
\mathbf{A} =
\begin{pmatrix}
$\,\,$ 1 $\,\,$ & $\,\,$6$\,\,$ & $\,\,$3$\,\,$ & $\,\,$9 $\,\,$ \\
$\,\,$ 1/6$\,\,$ & $\,\,$ 1 $\,\,$ & $\,\,$1$\,\,$ & $\,\,$8 $\,\,$ \\
$\,\,$ 1/3$\,\,$ & $\,\,$ 1 $\,\,$ & $\,\,$ 1 $\,\,$ & $\,\,$5 $\,\,$ \\
$\,\,$ 1/9$\,\,$ & $\,\,$ 1/8$\,\,$ & $\,\,$ 1/5$\,\,$ & $\,\,$ 1  $\,\,$ \\
\end{pmatrix},
\qquad
\lambda_{\max} =
4.2460,
\qquad
CR = 0.0928
\end{equation*}

\begin{equation*}
\mathbf{w}^{EM} =
\begin{pmatrix}
0.604005\\
0.180490\\
\color{red} 0.176976\color{black} \\
0.038529
\end{pmatrix}\end{equation*}
\begin{equation*}
\left[ \frac{{w}^{EM}_i}{{w}^{EM}_j} \right] =
\begin{pmatrix}
$\,\,$ 1 $\,\,$ & $\,\,$3.3465$\,\,$ & $\,\,$\color{red} 3.4129\color{black} $\,\,$ & $\,\,$15.6765$\,\,$ \\
$\,\,$0.2988$\,\,$ & $\,\,$ 1 $\,\,$ & $\,\,$\color{red} 1.0199\color{black} $\,\,$ & $\,\,$4.6845  $\,\,$ \\
$\,\,$\color{red} 0.2930\color{black} $\,\,$ & $\,\,$\color{red} 0.9805\color{black} $\,\,$ & $\,\,$ 1 $\,\,$ & $\,\,$\color{red} 4.5933\color{black}  $\,\,$ \\
$\,\,$0.0638$\,\,$ & $\,\,$0.2135$\,\,$ & $\,\,$\color{red} 0.2177\color{black} $\,\,$ & $\,\,$ 1  $\,\,$ \\
\end{pmatrix},
\end{equation*}

\begin{equation*}
\mathbf{w}^{\prime} =
\begin{pmatrix}
0.601890\\
0.179858\\
0.179858\\
0.038394
\end{pmatrix} =
0.996498\cdot
\begin{pmatrix}
0.604005\\
0.180490\\
\color{gr} 0.180490\color{black} \\
0.038529
\end{pmatrix},
\end{equation*}
\begin{equation*}
\left[ \frac{{w}^{\prime}_i}{{w}^{\prime}_j} \right] =
\begin{pmatrix}
$\,\,$ 1 $\,\,$ & $\,\,$3.3465$\,\,$ & $\,\,$\color{gr} 3.3465\color{black} $\,\,$ & $\,\,$15.6765$\,\,$ \\
$\,\,$0.2988$\,\,$ & $\,\,$ 1 $\,\,$ & $\,\,$\color{gr} \color{blue} 1\color{black} $\,\,$ & $\,\,$4.6845  $\,\,$ \\
$\,\,$\color{gr} 0.2988\color{black} $\,\,$ & $\,\,$\color{gr} \color{blue} 1\color{black} $\,\,$ & $\,\,$ 1 $\,\,$ & $\,\,$\color{gr} 4.6845\color{black}  $\,\,$ \\
$\,\,$0.0638$\,\,$ & $\,\,$0.2135$\,\,$ & $\,\,$\color{gr} 0.2135\color{black} $\,\,$ & $\,\,$ 1  $\,\,$ \\
\end{pmatrix},
\end{equation*}
\end{example}
\newpage
\begin{example}
\begin{equation*}
\mathbf{A} =
\begin{pmatrix}
$\,\,$ 1 $\,\,$ & $\,\,$6$\,\,$ & $\,\,$4$\,\,$ & $\,\,$5 $\,\,$ \\
$\,\,$ 1/6$\,\,$ & $\,\,$ 1 $\,\,$ & $\,\,$1$\,\,$ & $\,\,$3 $\,\,$ \\
$\,\,$ 1/4$\,\,$ & $\,\,$ 1 $\,\,$ & $\,\,$ 1 $\,\,$ & $\,\,$2 $\,\,$ \\
$\,\,$ 1/5$\,\,$ & $\,\,$ 1/3$\,\,$ & $\,\,$ 1/2$\,\,$ & $\,\,$ 1  $\,\,$ \\
\end{pmatrix},
\qquad
\lambda_{\max} =
4.1406,
\qquad
CR = 0.0530
\end{equation*}

\begin{equation*}
\mathbf{w}^{EM} =
\begin{pmatrix}
0.615046\\
0.156134\\
\color{red} 0.149309\color{black} \\
0.079511
\end{pmatrix}\end{equation*}
\begin{equation*}
\left[ \frac{{w}^{EM}_i}{{w}^{EM}_j} \right] =
\begin{pmatrix}
$\,\,$ 1 $\,\,$ & $\,\,$3.9392$\,\,$ & $\,\,$\color{red} 4.1193\color{black} $\,\,$ & $\,\,$7.7354$\,\,$ \\
$\,\,$0.2539$\,\,$ & $\,\,$ 1 $\,\,$ & $\,\,$\color{red} 1.0457\color{black} $\,\,$ & $\,\,$1.9637  $\,\,$ \\
$\,\,$\color{red} 0.2428\color{black} $\,\,$ & $\,\,$\color{red} 0.9563\color{black} $\,\,$ & $\,\,$ 1 $\,\,$ & $\,\,$\color{red} 1.8779\color{black}  $\,\,$ \\
$\,\,$0.1293$\,\,$ & $\,\,$0.5092$\,\,$ & $\,\,$\color{red} 0.5325\color{black} $\,\,$ & $\,\,$ 1  $\,\,$ \\
\end{pmatrix},
\end{equation*}

\begin{equation*}
\mathbf{w}^{\prime} =
\begin{pmatrix}
0.612320\\
0.155442\\
0.153080\\
0.079158
\end{pmatrix} =
0.995568\cdot
\begin{pmatrix}
0.615046\\
0.156134\\
\color{gr} 0.153762\color{black} \\
0.079511
\end{pmatrix},
\end{equation*}
\begin{equation*}
\left[ \frac{{w}^{\prime}_i}{{w}^{\prime}_j} \right] =
\begin{pmatrix}
$\,\,$ 1 $\,\,$ & $\,\,$3.9392$\,\,$ & $\,\,$\color{gr} \color{blue} 4\color{black} $\,\,$ & $\,\,$7.7354$\,\,$ \\
$\,\,$0.2539$\,\,$ & $\,\,$ 1 $\,\,$ & $\,\,$\color{gr} 1.0154\color{black} $\,\,$ & $\,\,$1.9637  $\,\,$ \\
$\,\,$\color{gr} \color{blue}  1/4\color{black} $\,\,$ & $\,\,$\color{gr} 0.9848\color{black} $\,\,$ & $\,\,$ 1 $\,\,$ & $\,\,$\color{gr} 1.9338\color{black}  $\,\,$ \\
$\,\,$0.1293$\,\,$ & $\,\,$0.5092$\,\,$ & $\,\,$\color{gr} 0.5171\color{black} $\,\,$ & $\,\,$ 1  $\,\,$ \\
\end{pmatrix},
\end{equation*}
\end{example}
\newpage
\begin{example}
\begin{equation*}
\mathbf{A} =
\begin{pmatrix}
$\,\,$ 1 $\,\,$ & $\,\,$6$\,\,$ & $\,\,$4$\,\,$ & $\,\,$5 $\,\,$ \\
$\,\,$ 1/6$\,\,$ & $\,\,$ 1 $\,\,$ & $\,\,$1$\,\,$ & $\,\,$4 $\,\,$ \\
$\,\,$ 1/4$\,\,$ & $\,\,$ 1 $\,\,$ & $\,\,$ 1 $\,\,$ & $\,\,$2 $\,\,$ \\
$\,\,$ 1/5$\,\,$ & $\,\,$ 1/4$\,\,$ & $\,\,$ 1/2$\,\,$ & $\,\,$ 1  $\,\,$ \\
\end{pmatrix},
\qquad
\lambda_{\max} =
4.2162,
\qquad
CR = 0.0815
\end{equation*}

\begin{equation*}
\mathbf{w}^{EM} =
\begin{pmatrix}
0.611344\\
0.168882\\
\color{red} 0.145942\color{black} \\
0.073832
\end{pmatrix}\end{equation*}
\begin{equation*}
\left[ \frac{{w}^{EM}_i}{{w}^{EM}_j} \right] =
\begin{pmatrix}
$\,\,$ 1 $\,\,$ & $\,\,$3.6200$\,\,$ & $\,\,$\color{red} 4.1890\color{black} $\,\,$ & $\,\,$8.2802$\,\,$ \\
$\,\,$0.2762$\,\,$ & $\,\,$ 1 $\,\,$ & $\,\,$\color{red} 1.1572\color{black} $\,\,$ & $\,\,$2.2874  $\,\,$ \\
$\,\,$\color{red} 0.2387\color{black} $\,\,$ & $\,\,$\color{red} 0.8642\color{black} $\,\,$ & $\,\,$ 1 $\,\,$ & $\,\,$\color{red} 1.9767\color{black}  $\,\,$ \\
$\,\,$0.1208$\,\,$ & $\,\,$0.4372$\,\,$ & $\,\,$\color{red} 0.5059\color{black} $\,\,$ & $\,\,$ 1  $\,\,$ \\
\end{pmatrix},
\end{equation*}

\begin{equation*}
\mathbf{w}^{\prime} =
\begin{pmatrix}
0.610293\\
0.168591\\
0.147410\\
0.073705
\end{pmatrix} =
0.998281\cdot
\begin{pmatrix}
0.611344\\
0.168882\\
\color{gr} 0.147664\color{black} \\
0.073832
\end{pmatrix},
\end{equation*}
\begin{equation*}
\left[ \frac{{w}^{\prime}_i}{{w}^{\prime}_j} \right] =
\begin{pmatrix}
$\,\,$ 1 $\,\,$ & $\,\,$3.6200$\,\,$ & $\,\,$\color{gr} 4.1401\color{black} $\,\,$ & $\,\,$8.2802$\,\,$ \\
$\,\,$0.2762$\,\,$ & $\,\,$ 1 $\,\,$ & $\,\,$\color{gr} 1.1437\color{black} $\,\,$ & $\,\,$2.2874  $\,\,$ \\
$\,\,$\color{gr} 0.2415\color{black} $\,\,$ & $\,\,$\color{gr} 0.8744\color{black} $\,\,$ & $\,\,$ 1 $\,\,$ & $\,\,$\color{gr} \color{blue} 2\color{black}  $\,\,$ \\
$\,\,$0.1208$\,\,$ & $\,\,$0.4372$\,\,$ & $\,\,$\color{gr} \color{blue}  1/2\color{black} $\,\,$ & $\,\,$ 1  $\,\,$ \\
\end{pmatrix},
\end{equation*}
\end{example}
\newpage
\begin{example}
\begin{equation*}
\mathbf{A} =
\begin{pmatrix}
$\,\,$ 1 $\,\,$ & $\,\,$6$\,\,$ & $\,\,$4$\,\,$ & $\,\,$6 $\,\,$ \\
$\,\,$ 1/6$\,\,$ & $\,\,$ 1 $\,\,$ & $\,\,$1$\,\,$ & $\,\,$3 $\,\,$ \\
$\,\,$ 1/4$\,\,$ & $\,\,$ 1 $\,\,$ & $\,\,$ 1 $\,\,$ & $\,\,$2 $\,\,$ \\
$\,\,$ 1/6$\,\,$ & $\,\,$ 1/3$\,\,$ & $\,\,$ 1/2$\,\,$ & $\,\,$ 1  $\,\,$ \\
\end{pmatrix},
\qquad
\lambda_{\max} =
4.1031,
\qquad
CR = 0.0389
\end{equation*}

\begin{equation*}
\mathbf{w}^{EM} =
\begin{pmatrix}
0.626791\\
0.152357\\
\color{red} 0.147117\color{black} \\
0.073735
\end{pmatrix}\end{equation*}
\begin{equation*}
\left[ \frac{{w}^{EM}_i}{{w}^{EM}_j} \right] =
\begin{pmatrix}
$\,\,$ 1 $\,\,$ & $\,\,$4.1140$\,\,$ & $\,\,$\color{red} 4.2605\color{black} $\,\,$ & $\,\,$8.5006$\,\,$ \\
$\,\,$0.2431$\,\,$ & $\,\,$ 1 $\,\,$ & $\,\,$\color{red} 1.0356\color{black} $\,\,$ & $\,\,$2.0663  $\,\,$ \\
$\,\,$\color{red} 0.2347\color{black} $\,\,$ & $\,\,$\color{red} 0.9656\color{black} $\,\,$ & $\,\,$ 1 $\,\,$ & $\,\,$\color{red} 1.9952\color{black}  $\,\,$ \\
$\,\,$0.1176$\,\,$ & $\,\,$0.4840$\,\,$ & $\,\,$\color{red} 0.5012\color{black} $\,\,$ & $\,\,$ 1  $\,\,$ \\
\end{pmatrix},
\end{equation*}

\begin{equation*}
\mathbf{w}^{\prime} =
\begin{pmatrix}
0.626570\\
0.152304\\
0.147418\\
0.073709
\end{pmatrix} =
0.999648\cdot
\begin{pmatrix}
0.626791\\
0.152357\\
\color{gr} 0.147470\color{black} \\
0.073735
\end{pmatrix},
\end{equation*}
\begin{equation*}
\left[ \frac{{w}^{\prime}_i}{{w}^{\prime}_j} \right] =
\begin{pmatrix}
$\,\,$ 1 $\,\,$ & $\,\,$4.1140$\,\,$ & $\,\,$\color{gr} 4.2503\color{black} $\,\,$ & $\,\,$8.5006$\,\,$ \\
$\,\,$0.2431$\,\,$ & $\,\,$ 1 $\,\,$ & $\,\,$\color{gr} 1.0331\color{black} $\,\,$ & $\,\,$2.0663  $\,\,$ \\
$\,\,$\color{gr} 0.2353\color{black} $\,\,$ & $\,\,$\color{gr} 0.9679\color{black} $\,\,$ & $\,\,$ 1 $\,\,$ & $\,\,$\color{gr} \color{blue} 2\color{black}  $\,\,$ \\
$\,\,$0.1176$\,\,$ & $\,\,$0.4840$\,\,$ & $\,\,$\color{gr} \color{blue}  1/2\color{black} $\,\,$ & $\,\,$ 1  $\,\,$ \\
\end{pmatrix},
\end{equation*}
\end{example}
\newpage
\begin{example}
\begin{equation*}
\mathbf{A} =
\begin{pmatrix}
$\,\,$ 1 $\,\,$ & $\,\,$6$\,\,$ & $\,\,$4$\,\,$ & $\,\,$7 $\,\,$ \\
$\,\,$ 1/6$\,\,$ & $\,\,$ 1 $\,\,$ & $\,\,$1$\,\,$ & $\,\,$4 $\,\,$ \\
$\,\,$ 1/4$\,\,$ & $\,\,$ 1 $\,\,$ & $\,\,$ 1 $\,\,$ & $\,\,$3 $\,\,$ \\
$\,\,$ 1/7$\,\,$ & $\,\,$ 1/4$\,\,$ & $\,\,$ 1/3$\,\,$ & $\,\,$ 1  $\,\,$ \\
\end{pmatrix},
\qquad
\lambda_{\max} =
4.1317,
\qquad
CR = 0.0496
\end{equation*}

\begin{equation*}
\mathbf{w}^{EM} =
\begin{pmatrix}
0.629265\\
0.157105\\
\color{red} 0.155800\color{black} \\
0.057830
\end{pmatrix}\end{equation*}
\begin{equation*}
\left[ \frac{{w}^{EM}_i}{{w}^{EM}_j} \right] =
\begin{pmatrix}
$\,\,$ 1 $\,\,$ & $\,\,$4.0054$\,\,$ & $\,\,$\color{red} 4.0389\color{black} $\,\,$ & $\,\,$10.8812$\,\,$ \\
$\,\,$0.2497$\,\,$ & $\,\,$ 1 $\,\,$ & $\,\,$\color{red} 1.0084\color{black} $\,\,$ & $\,\,$2.7167  $\,\,$ \\
$\,\,$\color{red} 0.2476\color{black} $\,\,$ & $\,\,$\color{red} 0.9917\color{black} $\,\,$ & $\,\,$ 1 $\,\,$ & $\,\,$\color{red} 2.6941\color{black}  $\,\,$ \\
$\,\,$0.0919$\,\,$ & $\,\,$0.3681$\,\,$ & $\,\,$\color{red} 0.3712\color{black} $\,\,$ & $\,\,$ 1  $\,\,$ \\
\end{pmatrix},
\end{equation*}

\begin{equation*}
\mathbf{w}^{\prime} =
\begin{pmatrix}
0.628445\\
0.156900\\
0.156900\\
0.057755
\end{pmatrix} =
0.998697\cdot
\begin{pmatrix}
0.629265\\
0.157105\\
\color{gr} 0.157105\color{black} \\
0.057830
\end{pmatrix},
\end{equation*}
\begin{equation*}
\left[ \frac{{w}^{\prime}_i}{{w}^{\prime}_j} \right] =
\begin{pmatrix}
$\,\,$ 1 $\,\,$ & $\,\,$4.0054$\,\,$ & $\,\,$\color{gr} 4.0054\color{black} $\,\,$ & $\,\,$10.8812$\,\,$ \\
$\,\,$0.2497$\,\,$ & $\,\,$ 1 $\,\,$ & $\,\,$\color{gr} \color{blue} 1\color{black} $\,\,$ & $\,\,$2.7167  $\,\,$ \\
$\,\,$\color{gr} 0.2497\color{black} $\,\,$ & $\,\,$\color{gr} \color{blue} 1\color{black} $\,\,$ & $\,\,$ 1 $\,\,$ & $\,\,$\color{gr} 2.7167\color{black}  $\,\,$ \\
$\,\,$0.0919$\,\,$ & $\,\,$0.3681$\,\,$ & $\,\,$\color{gr} 0.3681\color{black} $\,\,$ & $\,\,$ 1  $\,\,$ \\
\end{pmatrix},
\end{equation*}
\end{example}
\newpage
\begin{example}
\begin{equation*}
\mathbf{A} =
\begin{pmatrix}
$\,\,$ 1 $\,\,$ & $\,\,$6$\,\,$ & $\,\,$4$\,\,$ & $\,\,$7 $\,\,$ \\
$\,\,$ 1/6$\,\,$ & $\,\,$ 1 $\,\,$ & $\,\,$1$\,\,$ & $\,\,$5 $\,\,$ \\
$\,\,$ 1/4$\,\,$ & $\,\,$ 1 $\,\,$ & $\,\,$ 1 $\,\,$ & $\,\,$3 $\,\,$ \\
$\,\,$ 1/7$\,\,$ & $\,\,$ 1/5$\,\,$ & $\,\,$ 1/3$\,\,$ & $\,\,$ 1  $\,\,$ \\
\end{pmatrix},
\qquad
\lambda_{\max} =
4.1832,
\qquad
CR = 0.0691
\end{equation*}

\begin{equation*}
\mathbf{w}^{EM} =
\begin{pmatrix}
0.626007\\
0.166522\\
\color{red} 0.152904\color{black} \\
0.054568
\end{pmatrix}\end{equation*}
\begin{equation*}
\left[ \frac{{w}^{EM}_i}{{w}^{EM}_j} \right] =
\begin{pmatrix}
$\,\,$ 1 $\,\,$ & $\,\,$3.7593$\,\,$ & $\,\,$\color{red} 4.0941\color{black} $\,\,$ & $\,\,$11.4721$\,\,$ \\
$\,\,$0.2660$\,\,$ & $\,\,$ 1 $\,\,$ & $\,\,$\color{red} 1.0891\color{black} $\,\,$ & $\,\,$3.0517  $\,\,$ \\
$\,\,$\color{red} 0.2443\color{black} $\,\,$ & $\,\,$\color{red} 0.9182\color{black} $\,\,$ & $\,\,$ 1 $\,\,$ & $\,\,$\color{red} 2.8021\color{black}  $\,\,$ \\
$\,\,$0.0872$\,\,$ & $\,\,$0.3277$\,\,$ & $\,\,$\color{red} 0.3569\color{black} $\,\,$ & $\,\,$ 1  $\,\,$ \\
\end{pmatrix},
\end{equation*}

\begin{equation*}
\mathbf{w}^{\prime} =
\begin{pmatrix}
0.623762\\
0.165925\\
0.155941\\
0.054372
\end{pmatrix} =
0.996415\cdot
\begin{pmatrix}
0.626007\\
0.166522\\
\color{gr} 0.156502\color{black} \\
0.054568
\end{pmatrix},
\end{equation*}
\begin{equation*}
\left[ \frac{{w}^{\prime}_i}{{w}^{\prime}_j} \right] =
\begin{pmatrix}
$\,\,$ 1 $\,\,$ & $\,\,$3.7593$\,\,$ & $\,\,$\color{gr} \color{blue} 4\color{black} $\,\,$ & $\,\,$11.4721$\,\,$ \\
$\,\,$0.2660$\,\,$ & $\,\,$ 1 $\,\,$ & $\,\,$\color{gr} 1.0640\color{black} $\,\,$ & $\,\,$3.0517  $\,\,$ \\
$\,\,$\color{gr} \color{blue}  1/4\color{black} $\,\,$ & $\,\,$\color{gr} 0.9398\color{black} $\,\,$ & $\,\,$ 1 $\,\,$ & $\,\,$\color{gr} 2.8680\color{black}  $\,\,$ \\
$\,\,$0.0872$\,\,$ & $\,\,$0.3277$\,\,$ & $\,\,$\color{gr} 0.3487\color{black} $\,\,$ & $\,\,$ 1  $\,\,$ \\
\end{pmatrix},
\end{equation*}
\end{example}
\newpage
\begin{example}
\begin{equation*}
\mathbf{A} =
\begin{pmatrix}
$\,\,$ 1 $\,\,$ & $\,\,$6$\,\,$ & $\,\,$4$\,\,$ & $\,\,$7 $\,\,$ \\
$\,\,$ 1/6$\,\,$ & $\,\,$ 1 $\,\,$ & $\,\,$1$\,\,$ & $\,\,$6 $\,\,$ \\
$\,\,$ 1/4$\,\,$ & $\,\,$ 1 $\,\,$ & $\,\,$ 1 $\,\,$ & $\,\,$3 $\,\,$ \\
$\,\,$ 1/7$\,\,$ & $\,\,$ 1/6$\,\,$ & $\,\,$ 1/3$\,\,$ & $\,\,$ 1  $\,\,$ \\
\end{pmatrix},
\qquad
\lambda_{\max} =
4.2359,
\qquad
CR = 0.0890
\end{equation*}

\begin{equation*}
\mathbf{w}^{EM} =
\begin{pmatrix}
0.622703\\
0.174939\\
\color{red} 0.150368\color{black} \\
0.051990
\end{pmatrix}\end{equation*}
\begin{equation*}
\left[ \frac{{w}^{EM}_i}{{w}^{EM}_j} \right] =
\begin{pmatrix}
$\,\,$ 1 $\,\,$ & $\,\,$3.5596$\,\,$ & $\,\,$\color{red} 4.1412\color{black} $\,\,$ & $\,\,$11.9774$\,\,$ \\
$\,\,$0.2809$\,\,$ & $\,\,$ 1 $\,\,$ & $\,\,$\color{red} 1.1634\color{black} $\,\,$ & $\,\,$3.3649  $\,\,$ \\
$\,\,$\color{red} 0.2415\color{black} $\,\,$ & $\,\,$\color{red} 0.8595\color{black} $\,\,$ & $\,\,$ 1 $\,\,$ & $\,\,$\color{red} 2.8923\color{black}  $\,\,$ \\
$\,\,$0.0835$\,\,$ & $\,\,$0.2972$\,\,$ & $\,\,$\color{red} 0.3458\color{black} $\,\,$ & $\,\,$ 1  $\,\,$ \\
\end{pmatrix},
\end{equation*}

\begin{equation*}
\mathbf{w}^{\prime} =
\begin{pmatrix}
0.619416\\
0.174015\\
0.154854\\
0.051715
\end{pmatrix} =
0.994721\cdot
\begin{pmatrix}
0.622703\\
0.174939\\
\color{gr} 0.155676\color{black} \\
0.051990
\end{pmatrix},
\end{equation*}
\begin{equation*}
\left[ \frac{{w}^{\prime}_i}{{w}^{\prime}_j} \right] =
\begin{pmatrix}
$\,\,$ 1 $\,\,$ & $\,\,$3.5596$\,\,$ & $\,\,$\color{gr} \color{blue} 4\color{black} $\,\,$ & $\,\,$11.9774$\,\,$ \\
$\,\,$0.2809$\,\,$ & $\,\,$ 1 $\,\,$ & $\,\,$\color{gr} 1.1237\color{black} $\,\,$ & $\,\,$3.3649  $\,\,$ \\
$\,\,$\color{gr} \color{blue}  1/4\color{black} $\,\,$ & $\,\,$\color{gr} 0.8899\color{black} $\,\,$ & $\,\,$ 1 $\,\,$ & $\,\,$\color{gr} 2.9943\color{black}  $\,\,$ \\
$\,\,$0.0835$\,\,$ & $\,\,$0.2972$\,\,$ & $\,\,$\color{gr} 0.3340\color{black} $\,\,$ & $\,\,$ 1  $\,\,$ \\
\end{pmatrix},
\end{equation*}
\end{example}
\newpage
\begin{example}
\begin{equation*}
\mathbf{A} =
\begin{pmatrix}
$\,\,$ 1 $\,\,$ & $\,\,$6$\,\,$ & $\,\,$4$\,\,$ & $\,\,$7 $\,\,$ \\
$\,\,$ 1/6$\,\,$ & $\,\,$ 1 $\,\,$ & $\,\,$3$\,\,$ & $\,\,$2 $\,\,$ \\
$\,\,$ 1/4$\,\,$ & $\,\,$ 1/3$\,\,$ & $\,\,$ 1 $\,\,$ & $\,\,$1 $\,\,$ \\
$\,\,$ 1/7$\,\,$ & $\,\,$ 1/2$\,\,$ & $\,\,$ 1 $\,\,$ & $\,\,$ 1  $\,\,$ \\
\end{pmatrix},
\qquad
\lambda_{\max} =
4.1964,
\qquad
CR = 0.0741
\end{equation*}

\begin{equation*}
\mathbf{w}^{EM} =
\begin{pmatrix}
0.641021\\
0.177121\\
0.095595\\
\color{red} 0.086263\color{black}
\end{pmatrix}\end{equation*}
\begin{equation*}
\left[ \frac{{w}^{EM}_i}{{w}^{EM}_j} \right] =
\begin{pmatrix}
$\,\,$ 1 $\,\,$ & $\,\,$3.6191$\,\,$ & $\,\,$6.7056$\,\,$ & $\,\,$\color{red} 7.4310\color{black} $\,\,$ \\
$\,\,$0.2763$\,\,$ & $\,\,$ 1 $\,\,$ & $\,\,$1.8528$\,\,$ & $\,\,$\color{red} 2.0533\color{black}   $\,\,$ \\
$\,\,$0.1491$\,\,$ & $\,\,$0.5397$\,\,$ & $\,\,$ 1 $\,\,$ & $\,\,$\color{red} 1.1082\color{black}  $\,\,$ \\
$\,\,$\color{red} 0.1346\color{black} $\,\,$ & $\,\,$\color{red} 0.4870\color{black} $\,\,$ & $\,\,$\color{red} 0.9024\color{black} $\,\,$ & $\,\,$ 1  $\,\,$ \\
\end{pmatrix},
\end{equation*}

\begin{equation*}
\mathbf{w}^{\prime} =
\begin{pmatrix}
0.639551\\
0.176715\\
0.095376\\
0.088358
\end{pmatrix} =
0.997708\cdot
\begin{pmatrix}
0.641021\\
0.177121\\
0.095595\\
\color{gr} 0.088561\color{black}
\end{pmatrix},
\end{equation*}
\begin{equation*}
\left[ \frac{{w}^{\prime}_i}{{w}^{\prime}_j} \right] =
\begin{pmatrix}
$\,\,$ 1 $\,\,$ & $\,\,$3.6191$\,\,$ & $\,\,$6.7056$\,\,$ & $\,\,$\color{gr} 7.2382\color{black} $\,\,$ \\
$\,\,$0.2763$\,\,$ & $\,\,$ 1 $\,\,$ & $\,\,$1.8528$\,\,$ & $\,\,$\color{gr} \color{blue} 2\color{black}   $\,\,$ \\
$\,\,$0.1491$\,\,$ & $\,\,$0.5397$\,\,$ & $\,\,$ 1 $\,\,$ & $\,\,$\color{gr} 1.0794\color{black}  $\,\,$ \\
$\,\,$\color{gr} 0.1382\color{black} $\,\,$ & $\,\,$\color{gr} \color{blue}  1/2\color{black} $\,\,$ & $\,\,$\color{gr} 0.9264\color{black} $\,\,$ & $\,\,$ 1  $\,\,$ \\
\end{pmatrix},
\end{equation*}
\end{example}
\newpage
\begin{example}
\begin{equation*}
\mathbf{A} =
\begin{pmatrix}
$\,\,$ 1 $\,\,$ & $\,\,$6$\,\,$ & $\,\,$4$\,\,$ & $\,\,$8 $\,\,$ \\
$\,\,$ 1/6$\,\,$ & $\,\,$ 1 $\,\,$ & $\,\,$1$\,\,$ & $\,\,$4 $\,\,$ \\
$\,\,$ 1/4$\,\,$ & $\,\,$ 1 $\,\,$ & $\,\,$ 1 $\,\,$ & $\,\,$3 $\,\,$ \\
$\,\,$ 1/8$\,\,$ & $\,\,$ 1/4$\,\,$ & $\,\,$ 1/3$\,\,$ & $\,\,$ 1  $\,\,$ \\
\end{pmatrix},
\qquad
\lambda_{\max} =
4.1031,
\qquad
CR = 0.0389
\end{equation*}

\begin{equation*}
\mathbf{w}^{EM} =
\begin{pmatrix}
0.637313\\
0.154217\\
\color{red} 0.153848\color{black} \\
0.054622
\end{pmatrix}\end{equation*}
\begin{equation*}
\left[ \frac{{w}^{EM}_i}{{w}^{EM}_j} \right] =
\begin{pmatrix}
$\,\,$ 1 $\,\,$ & $\,\,$4.1326$\,\,$ & $\,\,$\color{red} 4.1425\color{black} $\,\,$ & $\,\,$11.6676$\,\,$ \\
$\,\,$0.2420$\,\,$ & $\,\,$ 1 $\,\,$ & $\,\,$\color{red} 1.0024\color{black} $\,\,$ & $\,\,$2.8233  $\,\,$ \\
$\,\,$\color{red} 0.2414\color{black} $\,\,$ & $\,\,$\color{red} 0.9976\color{black} $\,\,$ & $\,\,$ 1 $\,\,$ & $\,\,$\color{red} 2.8166\color{black}  $\,\,$ \\
$\,\,$0.0857$\,\,$ & $\,\,$0.3542$\,\,$ & $\,\,$\color{red} 0.3550\color{black} $\,\,$ & $\,\,$ 1  $\,\,$ \\
\end{pmatrix},
\end{equation*}

\begin{equation*}
\mathbf{w}^{\prime} =
\begin{pmatrix}
0.637078\\
0.154160\\
0.154160\\
0.054602
\end{pmatrix} =
0.999631\cdot
\begin{pmatrix}
0.637313\\
0.154217\\
\color{gr} 0.154217\color{black} \\
0.054622
\end{pmatrix},
\end{equation*}
\begin{equation*}
\left[ \frac{{w}^{\prime}_i}{{w}^{\prime}_j} \right] =
\begin{pmatrix}
$\,\,$ 1 $\,\,$ & $\,\,$4.1326$\,\,$ & $\,\,$\color{gr} 4.1326\color{black} $\,\,$ & $\,\,$11.6676$\,\,$ \\
$\,\,$0.2420$\,\,$ & $\,\,$ 1 $\,\,$ & $\,\,$\color{gr} \color{blue} 1\color{black} $\,\,$ & $\,\,$2.8233  $\,\,$ \\
$\,\,$\color{gr} 0.2420\color{black} $\,\,$ & $\,\,$\color{gr} \color{blue} 1\color{black} $\,\,$ & $\,\,$ 1 $\,\,$ & $\,\,$\color{gr} 2.8233\color{black}  $\,\,$ \\
$\,\,$0.0857$\,\,$ & $\,\,$0.3542$\,\,$ & $\,\,$\color{gr} 0.3542\color{black} $\,\,$ & $\,\,$ 1  $\,\,$ \\
\end{pmatrix},
\end{equation*}
\end{example}
\newpage
\begin{example}
\begin{equation*}
\mathbf{A} =
\begin{pmatrix}
$\,\,$ 1 $\,\,$ & $\,\,$6$\,\,$ & $\,\,$4$\,\,$ & $\,\,$8 $\,\,$ \\
$\,\,$ 1/6$\,\,$ & $\,\,$ 1 $\,\,$ & $\,\,$1$\,\,$ & $\,\,$5 $\,\,$ \\
$\,\,$ 1/4$\,\,$ & $\,\,$ 1 $\,\,$ & $\,\,$ 1 $\,\,$ & $\,\,$3 $\,\,$ \\
$\,\,$ 1/8$\,\,$ & $\,\,$ 1/5$\,\,$ & $\,\,$ 1/3$\,\,$ & $\,\,$ 1  $\,\,$ \\
\end{pmatrix},
\qquad
\lambda_{\max} =
4.1502,
\qquad
CR = 0.0566
\end{equation*}

\begin{equation*}
\mathbf{w}^{EM} =
\begin{pmatrix}
0.633932\\
0.163322\\
\color{red} 0.151221\color{black} \\
0.051525
\end{pmatrix}\end{equation*}
\begin{equation*}
\left[ \frac{{w}^{EM}_i}{{w}^{EM}_j} \right] =
\begin{pmatrix}
$\,\,$ 1 $\,\,$ & $\,\,$3.8815$\,\,$ & $\,\,$\color{red} 4.1921\color{black} $\,\,$ & $\,\,$12.3035$\,\,$ \\
$\,\,$0.2576$\,\,$ & $\,\,$ 1 $\,\,$ & $\,\,$\color{red} 1.0800\color{black} $\,\,$ & $\,\,$3.1698  $\,\,$ \\
$\,\,$\color{red} 0.2385\color{black} $\,\,$ & $\,\,$\color{red} 0.9259\color{black} $\,\,$ & $\,\,$ 1 $\,\,$ & $\,\,$\color{red} 2.9349\color{black}  $\,\,$ \\
$\,\,$0.0813$\,\,$ & $\,\,$0.3155$\,\,$ & $\,\,$\color{red} 0.3407\color{black} $\,\,$ & $\,\,$ 1  $\,\,$ \\
\end{pmatrix},
\end{equation*}

\begin{equation*}
\mathbf{w}^{\prime} =
\begin{pmatrix}
0.631814\\
0.162777\\
0.154057\\
0.051352
\end{pmatrix} =
0.996659\cdot
\begin{pmatrix}
0.633932\\
0.163322\\
\color{gr} 0.154574\color{black} \\
0.051525
\end{pmatrix},
\end{equation*}
\begin{equation*}
\left[ \frac{{w}^{\prime}_i}{{w}^{\prime}_j} \right] =
\begin{pmatrix}
$\,\,$ 1 $\,\,$ & $\,\,$3.8815$\,\,$ & $\,\,$\color{gr} 4.1012\color{black} $\,\,$ & $\,\,$12.3035$\,\,$ \\
$\,\,$0.2576$\,\,$ & $\,\,$ 1 $\,\,$ & $\,\,$\color{gr} 1.0566\color{black} $\,\,$ & $\,\,$3.1698  $\,\,$ \\
$\,\,$\color{gr} 0.2438\color{black} $\,\,$ & $\,\,$\color{gr} 0.9464\color{black} $\,\,$ & $\,\,$ 1 $\,\,$ & $\,\,$\color{gr} \color{blue} 3\color{black}  $\,\,$ \\
$\,\,$0.0813$\,\,$ & $\,\,$0.3155$\,\,$ & $\,\,$\color{gr} \color{blue}  1/3\color{black} $\,\,$ & $\,\,$ 1  $\,\,$ \\
\end{pmatrix},
\end{equation*}
\end{example}
\newpage
\begin{example}
\begin{equation*}
\mathbf{A} =
\begin{pmatrix}
$\,\,$ 1 $\,\,$ & $\,\,$6$\,\,$ & $\,\,$4$\,\,$ & $\,\,$9 $\,\,$ \\
$\,\,$ 1/6$\,\,$ & $\,\,$ 1 $\,\,$ & $\,\,$1$\,\,$ & $\,\,$6 $\,\,$ \\
$\,\,$ 1/4$\,\,$ & $\,\,$ 1 $\,\,$ & $\,\,$ 1 $\,\,$ & $\,\,$4 $\,\,$ \\
$\,\,$ 1/9$\,\,$ & $\,\,$ 1/6$\,\,$ & $\,\,$ 1/4$\,\,$ & $\,\,$ 1  $\,\,$ \\
\end{pmatrix},
\qquad
\lambda_{\max} =
4.1664,
\qquad
CR = 0.0627
\end{equation*}

\begin{equation*}
\mathbf{w}^{EM} =
\begin{pmatrix}
0.634494\\
0.165136\\
\color{red} 0.157016\color{black} \\
0.043354
\end{pmatrix}\end{equation*}
\begin{equation*}
\left[ \frac{{w}^{EM}_i}{{w}^{EM}_j} \right] =
\begin{pmatrix}
$\,\,$ 1 $\,\,$ & $\,\,$3.8422$\,\,$ & $\,\,$\color{red} 4.0410\color{black} $\,\,$ & $\,\,$14.6352$\,\,$ \\
$\,\,$0.2603$\,\,$ & $\,\,$ 1 $\,\,$ & $\,\,$\color{red} 1.0517\color{black} $\,\,$ & $\,\,$3.8090  $\,\,$ \\
$\,\,$\color{red} 0.2475\color{black} $\,\,$ & $\,\,$\color{red} 0.9508\color{black} $\,\,$ & $\,\,$ 1 $\,\,$ & $\,\,$\color{red} 3.6217\color{black}  $\,\,$ \\
$\,\,$0.0683$\,\,$ & $\,\,$0.2625$\,\,$ & $\,\,$\color{red} 0.2761\color{black} $\,\,$ & $\,\,$ 1  $\,\,$ \\
\end{pmatrix},
\end{equation*}

\begin{equation*}
\mathbf{w}^{\prime} =
\begin{pmatrix}
0.633476\\
0.164871\\
0.158369\\
0.043284
\end{pmatrix} =
0.998395\cdot
\begin{pmatrix}
0.634494\\
0.165136\\
\color{gr} 0.158624\color{black} \\
0.043354
\end{pmatrix},
\end{equation*}
\begin{equation*}
\left[ \frac{{w}^{\prime}_i}{{w}^{\prime}_j} \right] =
\begin{pmatrix}
$\,\,$ 1 $\,\,$ & $\,\,$3.8422$\,\,$ & $\,\,$\color{gr} \color{blue} 4\color{black} $\,\,$ & $\,\,$14.6352$\,\,$ \\
$\,\,$0.2603$\,\,$ & $\,\,$ 1 $\,\,$ & $\,\,$\color{gr} 1.0411\color{black} $\,\,$ & $\,\,$3.8090  $\,\,$ \\
$\,\,$\color{gr} \color{blue}  1/4\color{black} $\,\,$ & $\,\,$\color{gr} 0.9606\color{black} $\,\,$ & $\,\,$ 1 $\,\,$ & $\,\,$\color{gr} 3.6588\color{black}  $\,\,$ \\
$\,\,$0.0683$\,\,$ & $\,\,$0.2625$\,\,$ & $\,\,$\color{gr} 0.2733\color{black} $\,\,$ & $\,\,$ 1  $\,\,$ \\
\end{pmatrix},
\end{equation*}
\end{example}
\newpage
\begin{example}
\begin{equation*}
\mathbf{A} =
\begin{pmatrix}
$\,\,$ 1 $\,\,$ & $\,\,$6$\,\,$ & $\,\,$4$\,\,$ & $\,\,$9 $\,\,$ \\
$\,\,$ 1/6$\,\,$ & $\,\,$ 1 $\,\,$ & $\,\,$1$\,\,$ & $\,\,$7 $\,\,$ \\
$\,\,$ 1/4$\,\,$ & $\,\,$ 1 $\,\,$ & $\,\,$ 1 $\,\,$ & $\,\,$4 $\,\,$ \\
$\,\,$ 1/9$\,\,$ & $\,\,$ 1/7$\,\,$ & $\,\,$ 1/4$\,\,$ & $\,\,$ 1  $\,\,$ \\
\end{pmatrix},
\qquad
\lambda_{\max} =
4.2065,
\qquad
CR = 0.0779
\end{equation*}

\begin{equation*}
\mathbf{w}^{EM} =
\begin{pmatrix}
0.631646\\
0.171952\\
\color{red} 0.154786\color{black} \\
0.041616
\end{pmatrix}\end{equation*}
\begin{equation*}
\left[ \frac{{w}^{EM}_i}{{w}^{EM}_j} \right] =
\begin{pmatrix}
$\,\,$ 1 $\,\,$ & $\,\,$3.6734$\,\,$ & $\,\,$\color{red} 4.0808\color{black} $\,\,$ & $\,\,$15.1779$\,\,$ \\
$\,\,$0.2722$\,\,$ & $\,\,$ 1 $\,\,$ & $\,\,$\color{red} 1.1109\color{black} $\,\,$ & $\,\,$4.1319  $\,\,$ \\
$\,\,$\color{red} 0.2451\color{black} $\,\,$ & $\,\,$\color{red} 0.9002\color{black} $\,\,$ & $\,\,$ 1 $\,\,$ & $\,\,$\color{red} 3.7194\color{black}  $\,\,$ \\
$\,\,$0.0659$\,\,$ & $\,\,$0.2420$\,\,$ & $\,\,$\color{red} 0.2689\color{black} $\,\,$ & $\,\,$ 1  $\,\,$ \\
\end{pmatrix},
\end{equation*}

\begin{equation*}
\mathbf{w}^{\prime} =
\begin{pmatrix}
0.629678\\
0.171416\\
0.157419\\
0.041486
\end{pmatrix} =
0.996884\cdot
\begin{pmatrix}
0.631646\\
0.171952\\
\color{gr} 0.157911\color{black} \\
0.041616
\end{pmatrix},
\end{equation*}
\begin{equation*}
\left[ \frac{{w}^{\prime}_i}{{w}^{\prime}_j} \right] =
\begin{pmatrix}
$\,\,$ 1 $\,\,$ & $\,\,$3.6734$\,\,$ & $\,\,$\color{gr} \color{blue} 4\color{black} $\,\,$ & $\,\,$15.1779$\,\,$ \\
$\,\,$0.2722$\,\,$ & $\,\,$ 1 $\,\,$ & $\,\,$\color{gr} 1.0889\color{black} $\,\,$ & $\,\,$4.1319  $\,\,$ \\
$\,\,$\color{gr} \color{blue}  1/4\color{black} $\,\,$ & $\,\,$\color{gr} 0.9183\color{black} $\,\,$ & $\,\,$ 1 $\,\,$ & $\,\,$\color{gr} 3.7945\color{black}  $\,\,$ \\
$\,\,$0.0659$\,\,$ & $\,\,$0.2420$\,\,$ & $\,\,$\color{gr} 0.2635\color{black} $\,\,$ & $\,\,$ 1  $\,\,$ \\
\end{pmatrix},
\end{equation*}
\end{example}
\newpage
\begin{example}
\begin{equation*}
\mathbf{A} =
\begin{pmatrix}
$\,\,$ 1 $\,\,$ & $\,\,$6$\,\,$ & $\,\,$4$\,\,$ & $\,\,$9 $\,\,$ \\
$\,\,$ 1/6$\,\,$ & $\,\,$ 1 $\,\,$ & $\,\,$1$\,\,$ & $\,\,$8 $\,\,$ \\
$\,\,$ 1/4$\,\,$ & $\,\,$ 1 $\,\,$ & $\,\,$ 1 $\,\,$ & $\,\,$4 $\,\,$ \\
$\,\,$ 1/9$\,\,$ & $\,\,$ 1/8$\,\,$ & $\,\,$ 1/4$\,\,$ & $\,\,$ 1  $\,\,$ \\
\end{pmatrix},
\qquad
\lambda_{\max} =
4.2469,
\qquad
CR = 0.0931
\end{equation*}

\begin{equation*}
\mathbf{w}^{EM} =
\begin{pmatrix}
0.628847\\
0.178240\\
\color{red} 0.152769\color{black} \\
0.040144
\end{pmatrix}\end{equation*}
\begin{equation*}
\left[ \frac{{w}^{EM}_i}{{w}^{EM}_j} \right] =
\begin{pmatrix}
$\,\,$ 1 $\,\,$ & $\,\,$3.5281$\,\,$ & $\,\,$\color{red} 4.1163\color{black} $\,\,$ & $\,\,$15.6648$\,\,$ \\
$\,\,$0.2834$\,\,$ & $\,\,$ 1 $\,\,$ & $\,\,$\color{red} 1.1667\color{black} $\,\,$ & $\,\,$4.4400  $\,\,$ \\
$\,\,$\color{red} 0.2429\color{black} $\,\,$ & $\,\,$\color{red} 0.8571\color{black} $\,\,$ & $\,\,$ 1 $\,\,$ & $\,\,$\color{red} 3.8055\color{black}  $\,\,$ \\
$\,\,$0.0638$\,\,$ & $\,\,$0.2252$\,\,$ & $\,\,$\color{red} 0.2628\color{black} $\,\,$ & $\,\,$ 1  $\,\,$ \\
\end{pmatrix},
\end{equation*}

\begin{equation*}
\mathbf{w}^{\prime} =
\begin{pmatrix}
0.626066\\
0.177451\\
0.156516\\
0.039966
\end{pmatrix} =
0.995577\cdot
\begin{pmatrix}
0.628847\\
0.178240\\
\color{gr} 0.157212\color{black} \\
0.040144
\end{pmatrix},
\end{equation*}
\begin{equation*}
\left[ \frac{{w}^{\prime}_i}{{w}^{\prime}_j} \right] =
\begin{pmatrix}
$\,\,$ 1 $\,\,$ & $\,\,$3.5281$\,\,$ & $\,\,$\color{gr} \color{blue} 4\color{black} $\,\,$ & $\,\,$15.6648$\,\,$ \\
$\,\,$0.2834$\,\,$ & $\,\,$ 1 $\,\,$ & $\,\,$\color{gr} 1.1338\color{black} $\,\,$ & $\,\,$4.4400  $\,\,$ \\
$\,\,$\color{gr} \color{blue}  1/4\color{black} $\,\,$ & $\,\,$\color{gr} 0.8820\color{black} $\,\,$ & $\,\,$ 1 $\,\,$ & $\,\,$\color{gr} 3.9162\color{black}  $\,\,$ \\
$\,\,$0.0638$\,\,$ & $\,\,$0.2252$\,\,$ & $\,\,$\color{gr} 0.2554\color{black} $\,\,$ & $\,\,$ 1  $\,\,$ \\
\end{pmatrix},
\end{equation*}
\end{example}
\newpage
\begin{example}
\begin{equation*}
\mathbf{A} =
\begin{pmatrix}
$\,\,$ 1 $\,\,$ & $\,\,$6$\,\,$ & $\,\,$5$\,\,$ & $\,\,$7 $\,\,$ \\
$\,\,$ 1/6$\,\,$ & $\,\,$ 1 $\,\,$ & $\,\,$3$\,\,$ & $\,\,$2 $\,\,$ \\
$\,\,$ 1/5$\,\,$ & $\,\,$ 1/3$\,\,$ & $\,\,$ 1 $\,\,$ & $\,\,$1 $\,\,$ \\
$\,\,$ 1/7$\,\,$ & $\,\,$ 1/2$\,\,$ & $\,\,$ 1 $\,\,$ & $\,\,$ 1  $\,\,$ \\
\end{pmatrix},
\qquad
\lambda_{\max} =
4.1417,
\qquad
CR = 0.0534
\end{equation*}

\begin{equation*}
\mathbf{w}^{EM} =
\begin{pmatrix}
0.656097\\
0.171963\\
0.087036\\
\color{red} 0.084904\color{black}
\end{pmatrix}\end{equation*}
\begin{equation*}
\left[ \frac{{w}^{EM}_i}{{w}^{EM}_j} \right] =
\begin{pmatrix}
$\,\,$ 1 $\,\,$ & $\,\,$3.8153$\,\,$ & $\,\,$7.5382$\,\,$ & $\,\,$\color{red} 7.7275\color{black} $\,\,$ \\
$\,\,$0.2621$\,\,$ & $\,\,$ 1 $\,\,$ & $\,\,$1.9758$\,\,$ & $\,\,$\color{red} 2.0254\color{black}   $\,\,$ \\
$\,\,$0.1327$\,\,$ & $\,\,$0.5061$\,\,$ & $\,\,$ 1 $\,\,$ & $\,\,$\color{red} 1.0251\color{black}  $\,\,$ \\
$\,\,$\color{red} 0.1294\color{black} $\,\,$ & $\,\,$\color{red} 0.4937\color{black} $\,\,$ & $\,\,$\color{red} 0.9755\color{black} $\,\,$ & $\,\,$ 1  $\,\,$ \\
\end{pmatrix},
\end{equation*}

\begin{equation*}
\mathbf{w}^{\prime} =
\begin{pmatrix}
0.655390\\
0.171778\\
0.086942\\
0.085889
\end{pmatrix} =
0.998923\cdot
\begin{pmatrix}
0.656097\\
0.171963\\
0.087036\\
\color{gr} 0.085982\color{black}
\end{pmatrix},
\end{equation*}
\begin{equation*}
\left[ \frac{{w}^{\prime}_i}{{w}^{\prime}_j} \right] =
\begin{pmatrix}
$\,\,$ 1 $\,\,$ & $\,\,$3.8153$\,\,$ & $\,\,$7.5382$\,\,$ & $\,\,$\color{gr} 7.6307\color{black} $\,\,$ \\
$\,\,$0.2621$\,\,$ & $\,\,$ 1 $\,\,$ & $\,\,$1.9758$\,\,$ & $\,\,$\color{gr} \color{blue} 2\color{black}   $\,\,$ \\
$\,\,$0.1327$\,\,$ & $\,\,$0.5061$\,\,$ & $\,\,$ 1 $\,\,$ & $\,\,$\color{gr} 1.0123\color{black}  $\,\,$ \\
$\,\,$\color{gr} 0.1311\color{black} $\,\,$ & $\,\,$\color{gr} \color{blue}  1/2\color{black} $\,\,$ & $\,\,$\color{gr} 0.9879\color{black} $\,\,$ & $\,\,$ 1  $\,\,$ \\
\end{pmatrix},
\end{equation*}
\end{example}
\newpage
\begin{example}
\begin{equation*}
\mathbf{A} =
\begin{pmatrix}
$\,\,$ 1 $\,\,$ & $\,\,$6$\,\,$ & $\,\,$5$\,\,$ & $\,\,$8 $\,\,$ \\
$\,\,$ 1/6$\,\,$ & $\,\,$ 1 $\,\,$ & $\,\,$3$\,\,$ & $\,\,$2 $\,\,$ \\
$\,\,$ 1/5$\,\,$ & $\,\,$ 1/3$\,\,$ & $\,\,$ 1 $\,\,$ & $\,\,$1 $\,\,$ \\
$\,\,$ 1/8$\,\,$ & $\,\,$ 1/2$\,\,$ & $\,\,$ 1 $\,\,$ & $\,\,$ 1  $\,\,$ \\
\end{pmatrix},
\qquad
\lambda_{\max} =
4.1406,
\qquad
CR = 0.0530
\end{equation*}

\begin{equation*}
\mathbf{w}^{EM} =
\begin{pmatrix}
0.664667\\
0.168730\\
0.085925\\
\color{red} 0.080678\color{black}
\end{pmatrix}\end{equation*}
\begin{equation*}
\left[ \frac{{w}^{EM}_i}{{w}^{EM}_j} \right] =
\begin{pmatrix}
$\,\,$ 1 $\,\,$ & $\,\,$3.9392$\,\,$ & $\,\,$7.7354$\,\,$ & $\,\,$\color{red} 8.2385\color{black} $\,\,$ \\
$\,\,$0.2539$\,\,$ & $\,\,$ 1 $\,\,$ & $\,\,$1.9637$\,\,$ & $\,\,$\color{red} 2.0914\color{black}   $\,\,$ \\
$\,\,$0.1293$\,\,$ & $\,\,$0.5092$\,\,$ & $\,\,$ 1 $\,\,$ & $\,\,$\color{red} 1.0650\color{black}  $\,\,$ \\
$\,\,$\color{red} 0.1214\color{black} $\,\,$ & $\,\,$\color{red} 0.4781\color{black} $\,\,$ & $\,\,$\color{red} 0.9389\color{black} $\,\,$ & $\,\,$ 1  $\,\,$ \\
\end{pmatrix},
\end{equation*}

\begin{equation*}
\mathbf{w}^{\prime} =
\begin{pmatrix}
0.663071\\
0.168325\\
0.085719\\
0.082884
\end{pmatrix} =
0.997600\cdot
\begin{pmatrix}
0.664667\\
0.168730\\
0.085925\\
\color{gr} 0.083083\color{black}
\end{pmatrix},
\end{equation*}
\begin{equation*}
\left[ \frac{{w}^{\prime}_i}{{w}^{\prime}_j} \right] =
\begin{pmatrix}
$\,\,$ 1 $\,\,$ & $\,\,$3.9392$\,\,$ & $\,\,$7.7354$\,\,$ & $\,\,$\color{gr} \color{blue} 8\color{black} $\,\,$ \\
$\,\,$0.2539$\,\,$ & $\,\,$ 1 $\,\,$ & $\,\,$1.9637$\,\,$ & $\,\,$\color{gr} 2.0309\color{black}   $\,\,$ \\
$\,\,$0.1293$\,\,$ & $\,\,$0.5092$\,\,$ & $\,\,$ 1 $\,\,$ & $\,\,$\color{gr} 1.0342\color{black}  $\,\,$ \\
$\,\,$\color{gr} \color{blue}  1/8\color{black} $\,\,$ & $\,\,$\color{gr} 0.4924\color{black} $\,\,$ & $\,\,$\color{gr} 0.9669\color{black} $\,\,$ & $\,\,$ 1  $\,\,$ \\
\end{pmatrix},
\end{equation*}
\end{example}
\newpage
\begin{example}
\begin{equation*}
\mathbf{A} =
\begin{pmatrix}
$\,\,$ 1 $\,\,$ & $\,\,$6$\,\,$ & $\,\,$5$\,\,$ & $\,\,$8 $\,\,$ \\
$\,\,$ 1/6$\,\,$ & $\,\,$ 1 $\,\,$ & $\,\,$4$\,\,$ & $\,\,$2 $\,\,$ \\
$\,\,$ 1/5$\,\,$ & $\,\,$ 1/4$\,\,$ & $\,\,$ 1 $\,\,$ & $\,\,$1 $\,\,$ \\
$\,\,$ 1/8$\,\,$ & $\,\,$ 1/2$\,\,$ & $\,\,$ 1 $\,\,$ & $\,\,$ 1  $\,\,$ \\
\end{pmatrix},
\qquad
\lambda_{\max} =
4.2162,
\qquad
CR = 0.0815
\end{equation*}

\begin{equation*}
\mathbf{w}^{EM} =
\begin{pmatrix}
0.659466\\
0.182175\\
0.079644\\
\color{red} 0.078715\color{black}
\end{pmatrix}\end{equation*}
\begin{equation*}
\left[ \frac{{w}^{EM}_i}{{w}^{EM}_j} \right] =
\begin{pmatrix}
$\,\,$ 1 $\,\,$ & $\,\,$3.6200$\,\,$ & $\,\,$8.2802$\,\,$ & $\,\,$\color{red} 8.3779\color{black} $\,\,$ \\
$\,\,$0.2762$\,\,$ & $\,\,$ 1 $\,\,$ & $\,\,$2.2874$\,\,$ & $\,\,$\color{red} 2.3144\color{black}   $\,\,$ \\
$\,\,$0.1208$\,\,$ & $\,\,$0.4372$\,\,$ & $\,\,$ 1 $\,\,$ & $\,\,$\color{red} 1.0118\color{black}  $\,\,$ \\
$\,\,$\color{red} 0.1194\color{black} $\,\,$ & $\,\,$\color{red} 0.4321\color{black} $\,\,$ & $\,\,$\color{red} 0.9883\color{black} $\,\,$ & $\,\,$ 1  $\,\,$ \\
\end{pmatrix},
\end{equation*}

\begin{equation*}
\mathbf{w}^{\prime} =
\begin{pmatrix}
0.658854\\
0.182006\\
0.079570\\
0.079570
\end{pmatrix} =
0.999072\cdot
\begin{pmatrix}
0.659466\\
0.182175\\
0.079644\\
\color{gr} 0.079644\color{black}
\end{pmatrix},
\end{equation*}
\begin{equation*}
\left[ \frac{{w}^{\prime}_i}{{w}^{\prime}_j} \right] =
\begin{pmatrix}
$\,\,$ 1 $\,\,$ & $\,\,$3.6200$\,\,$ & $\,\,$8.2802$\,\,$ & $\,\,$\color{gr} 8.2802\color{black} $\,\,$ \\
$\,\,$0.2762$\,\,$ & $\,\,$ 1 $\,\,$ & $\,\,$2.2874$\,\,$ & $\,\,$\color{gr} 2.2874\color{black}   $\,\,$ \\
$\,\,$0.1208$\,\,$ & $\,\,$0.4372$\,\,$ & $\,\,$ 1 $\,\,$ & $\,\,$\color{gr} \color{blue} 1\color{black}  $\,\,$ \\
$\,\,$\color{gr} 0.1208\color{black} $\,\,$ & $\,\,$\color{gr} 0.4372\color{black} $\,\,$ & $\,\,$\color{gr} \color{blue} 1\color{black} $\,\,$ & $\,\,$ 1  $\,\,$ \\
\end{pmatrix},
\end{equation*}
\end{example}
\newpage
\begin{example}
\begin{equation*}
\mathbf{A} =
\begin{pmatrix}
$\,\,$ 1 $\,\,$ & $\,\,$6$\,\,$ & $\,\,$6$\,\,$ & $\,\,$8 $\,\,$ \\
$\,\,$ 1/6$\,\,$ & $\,\,$ 1 $\,\,$ & $\,\,$2$\,\,$ & $\,\,$7 $\,\,$ \\
$\,\,$ 1/6$\,\,$ & $\,\,$ 1/2$\,\,$ & $\,\,$ 1 $\,\,$ & $\,\,$2 $\,\,$ \\
$\,\,$ 1/8$\,\,$ & $\,\,$ 1/7$\,\,$ & $\,\,$ 1/2$\,\,$ & $\,\,$ 1  $\,\,$ \\
\end{pmatrix},
\qquad
\lambda_{\max} =
4.2421,
\qquad
CR = 0.0913
\end{equation*}

\begin{equation*}
\mathbf{w}^{EM} =
\begin{pmatrix}
0.659736\\
0.197225\\
\color{red} 0.094360\color{black} \\
0.048679
\end{pmatrix}\end{equation*}
\begin{equation*}
\left[ \frac{{w}^{EM}_i}{{w}^{EM}_j} \right] =
\begin{pmatrix}
$\,\,$ 1 $\,\,$ & $\,\,$3.3451$\,\,$ & $\,\,$\color{red} 6.9917\color{black} $\,\,$ & $\,\,$13.5529$\,\,$ \\
$\,\,$0.2989$\,\,$ & $\,\,$ 1 $\,\,$ & $\,\,$\color{red} 2.0901\color{black} $\,\,$ & $\,\,$4.0516  $\,\,$ \\
$\,\,$\color{red} 0.1430\color{black} $\,\,$ & $\,\,$\color{red} 0.4784\color{black} $\,\,$ & $\,\,$ 1 $\,\,$ & $\,\,$\color{red} 1.9384\color{black}  $\,\,$ \\
$\,\,$0.0738$\,\,$ & $\,\,$0.2468$\,\,$ & $\,\,$\color{red} 0.5159\color{black} $\,\,$ & $\,\,$ 1  $\,\,$ \\
\end{pmatrix},
\end{equation*}

\begin{equation*}
\mathbf{w}^{\prime} =
\begin{pmatrix}
0.657764\\
0.196636\\
0.097066\\
0.048533
\end{pmatrix} =
0.997012\cdot
\begin{pmatrix}
0.659736\\
0.197225\\
\color{gr} 0.097357\color{black} \\
0.048679
\end{pmatrix},
\end{equation*}
\begin{equation*}
\left[ \frac{{w}^{\prime}_i}{{w}^{\prime}_j} \right] =
\begin{pmatrix}
$\,\,$ 1 $\,\,$ & $\,\,$3.3451$\,\,$ & $\,\,$\color{gr} 6.7764\color{black} $\,\,$ & $\,\,$13.5529$\,\,$ \\
$\,\,$0.2989$\,\,$ & $\,\,$ 1 $\,\,$ & $\,\,$\color{gr} 2.0258\color{black} $\,\,$ & $\,\,$4.0516  $\,\,$ \\
$\,\,$\color{gr} 0.1476\color{black} $\,\,$ & $\,\,$\color{gr} 0.4936\color{black} $\,\,$ & $\,\,$ 1 $\,\,$ & $\,\,$\color{gr} \color{blue} 2\color{black}  $\,\,$ \\
$\,\,$0.0738$\,\,$ & $\,\,$0.2468$\,\,$ & $\,\,$\color{gr} \color{blue}  1/2\color{black} $\,\,$ & $\,\,$ 1  $\,\,$ \\
\end{pmatrix},
\end{equation*}
\end{example}
\newpage
\begin{example}
\begin{equation*}
\mathbf{A} =
\begin{pmatrix}
$\,\,$ 1 $\,\,$ & $\,\,$6$\,\,$ & $\,\,$6$\,\,$ & $\,\,$8 $\,\,$ \\
$\,\,$ 1/6$\,\,$ & $\,\,$ 1 $\,\,$ & $\,\,$3$\,\,$ & $\,\,$2 $\,\,$ \\
$\,\,$ 1/6$\,\,$ & $\,\,$ 1/3$\,\,$ & $\,\,$ 1 $\,\,$ & $\,\,$1 $\,\,$ \\
$\,\,$ 1/8$\,\,$ & $\,\,$ 1/2$\,\,$ & $\,\,$ 1 $\,\,$ & $\,\,$ 1  $\,\,$ \\
\end{pmatrix},
\qquad
\lambda_{\max} =
4.1031,
\qquad
CR = 0.0389
\end{equation*}

\begin{equation*}
\mathbf{w}^{EM} =
\begin{pmatrix}
0.676557\\
0.164454\\
0.079589\\
\color{red} 0.079399\color{black}
\end{pmatrix}\end{equation*}
\begin{equation*}
\left[ \frac{{w}^{EM}_i}{{w}^{EM}_j} \right] =
\begin{pmatrix}
$\,\,$ 1 $\,\,$ & $\,\,$4.1140$\,\,$ & $\,\,$8.5006$\,\,$ & $\,\,$\color{red} 8.5210\color{black} $\,\,$ \\
$\,\,$0.2431$\,\,$ & $\,\,$ 1 $\,\,$ & $\,\,$2.0663$\,\,$ & $\,\,$\color{red} 2.0712\color{black}   $\,\,$ \\
$\,\,$0.1176$\,\,$ & $\,\,$0.4840$\,\,$ & $\,\,$ 1 $\,\,$ & $\,\,$\color{red} 1.0024\color{black}  $\,\,$ \\
$\,\,$\color{red} 0.1174\color{black} $\,\,$ & $\,\,$\color{red} 0.4828\color{black} $\,\,$ & $\,\,$\color{red} 0.9976\color{black} $\,\,$ & $\,\,$ 1  $\,\,$ \\
\end{pmatrix},
\end{equation*}

\begin{equation*}
\mathbf{w}^{\prime} =
\begin{pmatrix}
0.676429\\
0.164423\\
0.079574\\
0.079574
\end{pmatrix} =
0.999810\cdot
\begin{pmatrix}
0.676557\\
0.164454\\
0.079589\\
\color{gr} 0.079589\color{black}
\end{pmatrix},
\end{equation*}
\begin{equation*}
\left[ \frac{{w}^{\prime}_i}{{w}^{\prime}_j} \right] =
\begin{pmatrix}
$\,\,$ 1 $\,\,$ & $\,\,$4.1140$\,\,$ & $\,\,$8.5006$\,\,$ & $\,\,$\color{gr} 8.5006\color{black} $\,\,$ \\
$\,\,$0.2431$\,\,$ & $\,\,$ 1 $\,\,$ & $\,\,$2.0663$\,\,$ & $\,\,$\color{gr} 2.0663\color{black}   $\,\,$ \\
$\,\,$0.1176$\,\,$ & $\,\,$0.4840$\,\,$ & $\,\,$ 1 $\,\,$ & $\,\,$\color{gr} \color{blue} 1\color{black}  $\,\,$ \\
$\,\,$\color{gr} 0.1176\color{black} $\,\,$ & $\,\,$\color{gr} 0.4840\color{black} $\,\,$ & $\,\,$\color{gr} \color{blue} 1\color{black} $\,\,$ & $\,\,$ 1  $\,\,$ \\
\end{pmatrix},
\end{equation*}
\end{example}
\newpage
\begin{example}
\begin{equation*}
\mathbf{A} =
\begin{pmatrix}
$\,\,$ 1 $\,\,$ & $\,\,$6$\,\,$ & $\,\,$6$\,\,$ & $\,\,$9 $\,\,$ \\
$\,\,$ 1/6$\,\,$ & $\,\,$ 1 $\,\,$ & $\,\,$3$\,\,$ & $\,\,$2 $\,\,$ \\
$\,\,$ 1/6$\,\,$ & $\,\,$ 1/3$\,\,$ & $\,\,$ 1 $\,\,$ & $\,\,$1 $\,\,$ \\
$\,\,$ 1/9$\,\,$ & $\,\,$ 1/2$\,\,$ & $\,\,$ 1 $\,\,$ & $\,\,$ 1  $\,\,$ \\
\end{pmatrix},
\qquad
\lambda_{\max} =
4.1031,
\qquad
CR = 0.0389
\end{equation*}

\begin{equation*}
\mathbf{w}^{EM} =
\begin{pmatrix}
0.684103\\
0.161536\\
0.078531\\
\color{red} 0.075830\color{black}
\end{pmatrix}\end{equation*}
\begin{equation*}
\left[ \frac{{w}^{EM}_i}{{w}^{EM}_j} \right] =
\begin{pmatrix}
$\,\,$ 1 $\,\,$ & $\,\,$4.2350$\,\,$ & $\,\,$8.7113$\,\,$ & $\,\,$\color{red} 9.0216\color{black} $\,\,$ \\
$\,\,$0.2361$\,\,$ & $\,\,$ 1 $\,\,$ & $\,\,$2.0570$\,\,$ & $\,\,$\color{red} 2.1302\color{black}   $\,\,$ \\
$\,\,$0.1148$\,\,$ & $\,\,$0.4862$\,\,$ & $\,\,$ 1 $\,\,$ & $\,\,$\color{red} 1.0356\color{black}  $\,\,$ \\
$\,\,$\color{red} 0.1108\color{black} $\,\,$ & $\,\,$\color{red} 0.4694\color{black} $\,\,$ & $\,\,$\color{red} 0.9656\color{black} $\,\,$ & $\,\,$ 1  $\,\,$ \\
\end{pmatrix},
\end{equation*}

\begin{equation*}
\mathbf{w}^{\prime} =
\begin{pmatrix}
0.683979\\
0.161507\\
0.078517\\
0.075998
\end{pmatrix} =
0.999818\cdot
\begin{pmatrix}
0.684103\\
0.161536\\
0.078531\\
\color{gr} 0.076011\color{black}
\end{pmatrix},
\end{equation*}
\begin{equation*}
\left[ \frac{{w}^{\prime}_i}{{w}^{\prime}_j} \right] =
\begin{pmatrix}
$\,\,$ 1 $\,\,$ & $\,\,$4.2350$\,\,$ & $\,\,$8.7113$\,\,$ & $\,\,$\color{gr} \color{blue} 9\color{black} $\,\,$ \\
$\,\,$0.2361$\,\,$ & $\,\,$ 1 $\,\,$ & $\,\,$2.0570$\,\,$ & $\,\,$\color{gr} 2.1252\color{black}   $\,\,$ \\
$\,\,$0.1148$\,\,$ & $\,\,$0.4862$\,\,$ & $\,\,$ 1 $\,\,$ & $\,\,$\color{gr} 1.0331\color{black}  $\,\,$ \\
$\,\,$\color{gr} \color{blue}  1/9\color{black} $\,\,$ & $\,\,$\color{gr} 0.4706\color{black} $\,\,$ & $\,\,$\color{gr} 0.9679\color{black} $\,\,$ & $\,\,$ 1  $\,\,$ \\
\end{pmatrix},
\end{equation*}
\end{example}
\newpage
\begin{example}
\begin{equation*}
\mathbf{A} =
\begin{pmatrix}
$\,\,$ 1 $\,\,$ & $\,\,$6$\,\,$ & $\,\,$6$\,\,$ & $\,\,$9 $\,\,$ \\
$\,\,$ 1/6$\,\,$ & $\,\,$ 1 $\,\,$ & $\,\,$5$\,\,$ & $\,\,$3 $\,\,$ \\
$\,\,$ 1/6$\,\,$ & $\,\,$ 1/5$\,\,$ & $\,\,$ 1 $\,\,$ & $\,\,$1 $\,\,$ \\
$\,\,$ 1/9$\,\,$ & $\,\,$ 1/3$\,\,$ & $\,\,$ 1 $\,\,$ & $\,\,$ 1  $\,\,$ \\
\end{pmatrix},
\qquad
\lambda_{\max} =
4.2277,
\qquad
CR = 0.0859
\end{equation*}

\begin{equation*}
\mathbf{w}^{EM} =
\begin{pmatrix}
0.671098\\
0.197855\\
0.066813\\
\color{red} 0.064235\color{black}
\end{pmatrix}\end{equation*}
\begin{equation*}
\left[ \frac{{w}^{EM}_i}{{w}^{EM}_j} \right] =
\begin{pmatrix}
$\,\,$ 1 $\,\,$ & $\,\,$3.3919$\,\,$ & $\,\,$10.0444$\,\,$ & $\,\,$\color{red} 10.4476\color{black} $\,\,$ \\
$\,\,$0.2948$\,\,$ & $\,\,$ 1 $\,\,$ & $\,\,$2.9613$\,\,$ & $\,\,$\color{red} 3.0802\color{black}   $\,\,$ \\
$\,\,$0.0996$\,\,$ & $\,\,$0.3377$\,\,$ & $\,\,$ 1 $\,\,$ & $\,\,$\color{red} 1.0401\color{black}  $\,\,$ \\
$\,\,$\color{red} 0.0957\color{black} $\,\,$ & $\,\,$\color{red} 0.3247\color{black} $\,\,$ & $\,\,$\color{red} 0.9614\color{black} $\,\,$ & $\,\,$ 1  $\,\,$ \\
\end{pmatrix},
\end{equation*}

\begin{equation*}
\mathbf{w}^{\prime} =
\begin{pmatrix}
0.669947\\
0.197515\\
0.066699\\
0.065838
\end{pmatrix} =
0.998286\cdot
\begin{pmatrix}
0.671098\\
0.197855\\
0.066813\\
\color{gr} 0.065952\color{black}
\end{pmatrix},
\end{equation*}
\begin{equation*}
\left[ \frac{{w}^{\prime}_i}{{w}^{\prime}_j} \right] =
\begin{pmatrix}
$\,\,$ 1 $\,\,$ & $\,\,$3.3919$\,\,$ & $\,\,$10.0444$\,\,$ & $\,\,$\color{gr} 10.1756\color{black} $\,\,$ \\
$\,\,$0.2948$\,\,$ & $\,\,$ 1 $\,\,$ & $\,\,$2.9613$\,\,$ & $\,\,$\color{gr} \color{blue} 3\color{black}   $\,\,$ \\
$\,\,$0.0996$\,\,$ & $\,\,$0.3377$\,\,$ & $\,\,$ 1 $\,\,$ & $\,\,$\color{gr} 1.0131\color{black}  $\,\,$ \\
$\,\,$\color{gr} 0.0983\color{black} $\,\,$ & $\,\,$\color{gr} \color{blue}  1/3\color{black} $\,\,$ & $\,\,$\color{gr} 0.9871\color{black} $\,\,$ & $\,\,$ 1  $\,\,$ \\
\end{pmatrix},
\end{equation*}
\end{example}
\newpage
\begin{example}
\begin{equation*}
\mathbf{A} =
\begin{pmatrix}
$\,\,$ 1 $\,\,$ & $\,\,$6$\,\,$ & $\,\,$7$\,\,$ & $\,\,$7 $\,\,$ \\
$\,\,$ 1/6$\,\,$ & $\,\,$ 1 $\,\,$ & $\,\,$2$\,\,$ & $\,\,$6 $\,\,$ \\
$\,\,$ 1/7$\,\,$ & $\,\,$ 1/2$\,\,$ & $\,\,$ 1 $\,\,$ & $\,\,$2 $\,\,$ \\
$\,\,$ 1/7$\,\,$ & $\,\,$ 1/6$\,\,$ & $\,\,$ 1/2$\,\,$ & $\,\,$ 1  $\,\,$ \\
\end{pmatrix},
\qquad
\lambda_{\max} =
4.2359,
\qquad
CR = 0.0890
\end{equation*}

\begin{equation*}
\mathbf{w}^{EM} =
\begin{pmatrix}
0.665328\\
0.189735\\
\color{red} 0.091633\color{black} \\
0.053303
\end{pmatrix}\end{equation*}
\begin{equation*}
\left[ \frac{{w}^{EM}_i}{{w}^{EM}_j} \right] =
\begin{pmatrix}
$\,\,$ 1 $\,\,$ & $\,\,$3.5066$\,\,$ & $\,\,$\color{red} 7.2608\color{black} $\,\,$ & $\,\,$12.4820$\,\,$ \\
$\,\,$0.2852$\,\,$ & $\,\,$ 1 $\,\,$ & $\,\,$\color{red} 2.0706\color{black} $\,\,$ & $\,\,$3.5596  $\,\,$ \\
$\,\,$\color{red} 0.1377\color{black} $\,\,$ & $\,\,$\color{red} 0.4830\color{black} $\,\,$ & $\,\,$ 1 $\,\,$ & $\,\,$\color{red} 1.7191\color{black}  $\,\,$ \\
$\,\,$0.0801$\,\,$ & $\,\,$0.2809$\,\,$ & $\,\,$\color{red} 0.5817\color{black} $\,\,$ & $\,\,$ 1  $\,\,$ \\
\end{pmatrix},
\end{equation*}

\begin{equation*}
\mathbf{w}^{\prime} =
\begin{pmatrix}
0.663183\\
0.189124\\
0.094562\\
0.053131
\end{pmatrix} =
0.996776\cdot
\begin{pmatrix}
0.665328\\
0.189735\\
\color{gr} 0.094868\color{black} \\
0.053303
\end{pmatrix},
\end{equation*}
\begin{equation*}
\left[ \frac{{w}^{\prime}_i}{{w}^{\prime}_j} \right] =
\begin{pmatrix}
$\,\,$ 1 $\,\,$ & $\,\,$3.5066$\,\,$ & $\,\,$\color{gr} 7.0132\color{black} $\,\,$ & $\,\,$12.4820$\,\,$ \\
$\,\,$0.2852$\,\,$ & $\,\,$ 1 $\,\,$ & $\,\,$\color{gr} \color{blue} 2\color{black} $\,\,$ & $\,\,$3.5596  $\,\,$ \\
$\,\,$\color{gr} 0.1426\color{black} $\,\,$ & $\,\,$\color{gr} \color{blue}  1/2\color{black} $\,\,$ & $\,\,$ 1 $\,\,$ & $\,\,$\color{gr} 1.7798\color{black}  $\,\,$ \\
$\,\,$0.0801$\,\,$ & $\,\,$0.2809$\,\,$ & $\,\,$\color{gr} 0.5619\color{black} $\,\,$ & $\,\,$ 1  $\,\,$ \\
\end{pmatrix},
\end{equation*}
\end{example}
\newpage
\begin{example}
\begin{equation*}
\mathbf{A} =
\begin{pmatrix}
$\,\,$ 1 $\,\,$ & $\,\,$6$\,\,$ & $\,\,$7$\,\,$ & $\,\,$8 $\,\,$ \\
$\,\,$ 1/6$\,\,$ & $\,\,$ 1 $\,\,$ & $\,\,$2$\,\,$ & $\,\,$6 $\,\,$ \\
$\,\,$ 1/7$\,\,$ & $\,\,$ 1/2$\,\,$ & $\,\,$ 1 $\,\,$ & $\,\,$2 $\,\,$ \\
$\,\,$ 1/8$\,\,$ & $\,\,$ 1/6$\,\,$ & $\,\,$ 1/2$\,\,$ & $\,\,$ 1  $\,\,$ \\
\end{pmatrix},
\qquad
\lambda_{\max} =
4.1964,
\qquad
CR = 0.0741
\end{equation*}

\begin{equation*}
\mathbf{w}^{EM} =
\begin{pmatrix}
0.673198\\
0.186012\\
\color{red} 0.090593\color{black} \\
0.050197
\end{pmatrix}\end{equation*}
\begin{equation*}
\left[ \frac{{w}^{EM}_i}{{w}^{EM}_j} \right] =
\begin{pmatrix}
$\,\,$ 1 $\,\,$ & $\,\,$3.6191$\,\,$ & $\,\,$\color{red} 7.4310\color{black} $\,\,$ & $\,\,$13.4112$\,\,$ \\
$\,\,$0.2763$\,\,$ & $\,\,$ 1 $\,\,$ & $\,\,$\color{red} 2.0533\color{black} $\,\,$ & $\,\,$3.7057  $\,\,$ \\
$\,\,$\color{red} 0.1346\color{black} $\,\,$ & $\,\,$\color{red} 0.4870\color{black} $\,\,$ & $\,\,$ 1 $\,\,$ & $\,\,$\color{red} 1.8048\color{black}  $\,\,$ \\
$\,\,$0.0746$\,\,$ & $\,\,$0.2699$\,\,$ & $\,\,$\color{red} 0.5541\color{black} $\,\,$ & $\,\,$ 1  $\,\,$ \\
\end{pmatrix},
\end{equation*}

\begin{equation*}
\mathbf{w}^{\prime} =
\begin{pmatrix}
0.671577\\
0.185564\\
0.092782\\
0.050076
\end{pmatrix} =
0.997593\cdot
\begin{pmatrix}
0.673198\\
0.186012\\
\color{gr} 0.093006\color{black} \\
0.050197
\end{pmatrix},
\end{equation*}
\begin{equation*}
\left[ \frac{{w}^{\prime}_i}{{w}^{\prime}_j} \right] =
\begin{pmatrix}
$\,\,$ 1 $\,\,$ & $\,\,$3.6191$\,\,$ & $\,\,$\color{gr} 7.2382\color{black} $\,\,$ & $\,\,$13.4112$\,\,$ \\
$\,\,$0.2763$\,\,$ & $\,\,$ 1 $\,\,$ & $\,\,$\color{gr} \color{blue} 2\color{black} $\,\,$ & $\,\,$3.7057  $\,\,$ \\
$\,\,$\color{gr} 0.1382\color{black} $\,\,$ & $\,\,$\color{gr} \color{blue}  1/2\color{black} $\,\,$ & $\,\,$ 1 $\,\,$ & $\,\,$\color{gr} 1.8528\color{black}  $\,\,$ \\
$\,\,$0.0746$\,\,$ & $\,\,$0.2699$\,\,$ & $\,\,$\color{gr} 0.5397\color{black} $\,\,$ & $\,\,$ 1  $\,\,$ \\
\end{pmatrix},
\end{equation*}
\end{example}
\newpage
\begin{example}
\begin{equation*}
\mathbf{A} =
\begin{pmatrix}
$\,\,$ 1 $\,\,$ & $\,\,$6$\,\,$ & $\,\,$7$\,\,$ & $\,\,$8 $\,\,$ \\
$\,\,$ 1/6$\,\,$ & $\,\,$ 1 $\,\,$ & $\,\,$2$\,\,$ & $\,\,$7 $\,\,$ \\
$\,\,$ 1/7$\,\,$ & $\,\,$ 1/2$\,\,$ & $\,\,$ 1 $\,\,$ & $\,\,$2 $\,\,$ \\
$\,\,$ 1/8$\,\,$ & $\,\,$ 1/7$\,\,$ & $\,\,$ 1/2$\,\,$ & $\,\,$ 1  $\,\,$ \\
\end{pmatrix},
\qquad
\lambda_{\max} =
4.2395,
\qquad
CR = 0.0903
\end{equation*}

\begin{equation*}
\mathbf{w}^{EM} =
\begin{pmatrix}
0.669441\\
0.193383\\
\color{red} 0.089070\color{black} \\
0.048107
\end{pmatrix}\end{equation*}
\begin{equation*}
\left[ \frac{{w}^{EM}_i}{{w}^{EM}_j} \right] =
\begin{pmatrix}
$\,\,$ 1 $\,\,$ & $\,\,$3.4617$\,\,$ & $\,\,$\color{red} 7.5159\color{black} $\,\,$ & $\,\,$13.9157$\,\,$ \\
$\,\,$0.2889$\,\,$ & $\,\,$ 1 $\,\,$ & $\,\,$\color{red} 2.1711\color{black} $\,\,$ & $\,\,$4.0199  $\,\,$ \\
$\,\,$\color{red} 0.1331\color{black} $\,\,$ & $\,\,$\color{red} 0.4606\color{black} $\,\,$ & $\,\,$ 1 $\,\,$ & $\,\,$\color{red} 1.8515\color{black}  $\,\,$ \\
$\,\,$0.0719$\,\,$ & $\,\,$0.2488$\,\,$ & $\,\,$\color{red} 0.5401\color{black} $\,\,$ & $\,\,$ 1  $\,\,$ \\
\end{pmatrix},
\end{equation*}

\begin{equation*}
\mathbf{w}^{\prime} =
\begin{pmatrix}
0.665075\\
0.192122\\
0.095011\\
0.047793
\end{pmatrix} =
0.993478\cdot
\begin{pmatrix}
0.669441\\
0.193383\\
\color{gr} 0.095634\color{black} \\
0.048107
\end{pmatrix},
\end{equation*}
\begin{equation*}
\left[ \frac{{w}^{\prime}_i}{{w}^{\prime}_j} \right] =
\begin{pmatrix}
$\,\,$ 1 $\,\,$ & $\,\,$3.4617$\,\,$ & $\,\,$\color{gr} \color{blue} 7\color{black} $\,\,$ & $\,\,$13.9157$\,\,$ \\
$\,\,$0.2889$\,\,$ & $\,\,$ 1 $\,\,$ & $\,\,$\color{gr} 2.0221\color{black} $\,\,$ & $\,\,$4.0199  $\,\,$ \\
$\,\,$\color{gr} \color{blue}  1/7\color{black} $\,\,$ & $\,\,$\color{gr} 0.4945\color{black} $\,\,$ & $\,\,$ 1 $\,\,$ & $\,\,$\color{gr} 1.9880\color{black}  $\,\,$ \\
$\,\,$0.0719$\,\,$ & $\,\,$0.2488$\,\,$ & $\,\,$\color{gr} 0.5030\color{black} $\,\,$ & $\,\,$ 1  $\,\,$ \\
\end{pmatrix},
\end{equation*}
\end{example}
\newpage
\begin{example}
\begin{equation*}
\mathbf{A} =
\begin{pmatrix}
$\,\,$ 1 $\,\,$ & $\,\,$6$\,\,$ & $\,\,$7$\,\,$ & $\,\,$9 $\,\,$ \\
$\,\,$ 1/6$\,\,$ & $\,\,$ 1 $\,\,$ & $\,\,$2$\,\,$ & $\,\,$6 $\,\,$ \\
$\,\,$ 1/7$\,\,$ & $\,\,$ 1/2$\,\,$ & $\,\,$ 1 $\,\,$ & $\,\,$2 $\,\,$ \\
$\,\,$ 1/9$\,\,$ & $\,\,$ 1/6$\,\,$ & $\,\,$ 1/2$\,\,$ & $\,\,$ 1  $\,\,$ \\
\end{pmatrix},
\qquad
\lambda_{\max} =
4.1658,
\qquad
CR = 0.0625
\end{equation*}

\begin{equation*}
\mathbf{w}^{EM} =
\begin{pmatrix}
0.679983\\
0.182729\\
\color{red} 0.089644\color{black} \\
0.047644
\end{pmatrix}\end{equation*}
\begin{equation*}
\left[ \frac{{w}^{EM}_i}{{w}^{EM}_j} \right] =
\begin{pmatrix}
$\,\,$ 1 $\,\,$ & $\,\,$3.7213$\,\,$ & $\,\,$\color{red} 7.5854\color{black} $\,\,$ & $\,\,$14.2722$\,\,$ \\
$\,\,$0.2687$\,\,$ & $\,\,$ 1 $\,\,$ & $\,\,$\color{red} 2.0384\color{black} $\,\,$ & $\,\,$3.8353  $\,\,$ \\
$\,\,$\color{red} 0.1318\color{black} $\,\,$ & $\,\,$\color{red} 0.4906\color{black} $\,\,$ & $\,\,$ 1 $\,\,$ & $\,\,$\color{red} 1.8815\color{black}  $\,\,$ \\
$\,\,$0.0701$\,\,$ & $\,\,$0.2607$\,\,$ & $\,\,$\color{red} 0.5315\color{black} $\,\,$ & $\,\,$ 1  $\,\,$ \\
\end{pmatrix},
\end{equation*}

\begin{equation*}
\mathbf{w}^{\prime} =
\begin{pmatrix}
0.678815\\
0.182415\\
0.091208\\
0.047562
\end{pmatrix} =
0.998282\cdot
\begin{pmatrix}
0.679983\\
0.182729\\
\color{gr} 0.091365\color{black} \\
0.047644
\end{pmatrix},
\end{equation*}
\begin{equation*}
\left[ \frac{{w}^{\prime}_i}{{w}^{\prime}_j} \right] =
\begin{pmatrix}
$\,\,$ 1 $\,\,$ & $\,\,$3.7213$\,\,$ & $\,\,$\color{gr} 7.4425\color{black} $\,\,$ & $\,\,$14.2722$\,\,$ \\
$\,\,$0.2687$\,\,$ & $\,\,$ 1 $\,\,$ & $\,\,$\color{gr} \color{blue} 2\color{black} $\,\,$ & $\,\,$3.8353  $\,\,$ \\
$\,\,$\color{gr} 0.1344\color{black} $\,\,$ & $\,\,$\color{gr} \color{blue}  1/2\color{black} $\,\,$ & $\,\,$ 1 $\,\,$ & $\,\,$\color{gr} 1.9177\color{black}  $\,\,$ \\
$\,\,$0.0701$\,\,$ & $\,\,$0.2607$\,\,$ & $\,\,$\color{gr} 0.5215\color{black} $\,\,$ & $\,\,$ 1  $\,\,$ \\
\end{pmatrix},
\end{equation*}
\end{example}
\newpage
\begin{example}
\begin{equation*}
\mathbf{A} =
\begin{pmatrix}
$\,\,$ 1 $\,\,$ & $\,\,$6$\,\,$ & $\,\,$7$\,\,$ & $\,\,$9 $\,\,$ \\
$\,\,$ 1/6$\,\,$ & $\,\,$ 1 $\,\,$ & $\,\,$2$\,\,$ & $\,\,$7 $\,\,$ \\
$\,\,$ 1/7$\,\,$ & $\,\,$ 1/2$\,\,$ & $\,\,$ 1 $\,\,$ & $\,\,$2 $\,\,$ \\
$\,\,$ 1/9$\,\,$ & $\,\,$ 1/7$\,\,$ & $\,\,$ 1/2$\,\,$ & $\,\,$ 1  $\,\,$ \\
\end{pmatrix},
\qquad
\lambda_{\max} =
4.2059,
\qquad
CR = 0.0776
\end{equation*}

\begin{equation*}
\mathbf{w}^{EM} =
\begin{pmatrix}
0.676217\\
0.189892\\
\color{red} 0.088232\color{black} \\
0.045659
\end{pmatrix}\end{equation*}
\begin{equation*}
\left[ \frac{{w}^{EM}_i}{{w}^{EM}_j} \right] =
\begin{pmatrix}
$\,\,$ 1 $\,\,$ & $\,\,$3.5611$\,\,$ & $\,\,$\color{red} 7.6641\color{black} $\,\,$ & $\,\,$14.8102$\,\,$ \\
$\,\,$0.2808$\,\,$ & $\,\,$ 1 $\,\,$ & $\,\,$\color{red} 2.1522\color{black} $\,\,$ & $\,\,$4.1589  $\,\,$ \\
$\,\,$\color{red} 0.1305\color{black} $\,\,$ & $\,\,$\color{red} 0.4646\color{black} $\,\,$ & $\,\,$ 1 $\,\,$ & $\,\,$\color{red} 1.9324\color{black}  $\,\,$ \\
$\,\,$0.0675$\,\,$ & $\,\,$0.2404$\,\,$ & $\,\,$\color{red} 0.5175\color{black} $\,\,$ & $\,\,$ 1  $\,\,$ \\
\end{pmatrix},
\end{equation*}

\begin{equation*}
\mathbf{w}^{\prime} =
\begin{pmatrix}
0.674137\\
0.189308\\
0.091037\\
0.045518
\end{pmatrix} =
0.996924\cdot
\begin{pmatrix}
0.676217\\
0.189892\\
\color{gr} 0.091318\color{black} \\
0.045659
\end{pmatrix},
\end{equation*}
\begin{equation*}
\left[ \frac{{w}^{\prime}_i}{{w}^{\prime}_j} \right] =
\begin{pmatrix}
$\,\,$ 1 $\,\,$ & $\,\,$3.5611$\,\,$ & $\,\,$\color{gr} 7.4051\color{black} $\,\,$ & $\,\,$14.8102$\,\,$ \\
$\,\,$0.2808$\,\,$ & $\,\,$ 1 $\,\,$ & $\,\,$\color{gr} 2.0795\color{black} $\,\,$ & $\,\,$4.1589  $\,\,$ \\
$\,\,$\color{gr} 0.1350\color{black} $\,\,$ & $\,\,$\color{gr} 0.4809\color{black} $\,\,$ & $\,\,$ 1 $\,\,$ & $\,\,$\color{gr} \color{blue} 2\color{black}  $\,\,$ \\
$\,\,$0.0675$\,\,$ & $\,\,$0.2404$\,\,$ & $\,\,$\color{gr} \color{blue}  1/2\color{black} $\,\,$ & $\,\,$ 1  $\,\,$ \\
\end{pmatrix},
\end{equation*}
\end{example}
\newpage
\begin{example}
\begin{equation*}
\mathbf{A} =
\begin{pmatrix}
$\,\,$ 1 $\,\,$ & $\,\,$6$\,\,$ & $\,\,$7$\,\,$ & $\,\,$9 $\,\,$ \\
$\,\,$ 1/6$\,\,$ & $\,\,$ 1 $\,\,$ & $\,\,$2$\,\,$ & $\,\,$8 $\,\,$ \\
$\,\,$ 1/7$\,\,$ & $\,\,$ 1/2$\,\,$ & $\,\,$ 1 $\,\,$ & $\,\,$2 $\,\,$ \\
$\,\,$ 1/9$\,\,$ & $\,\,$ 1/8$\,\,$ & $\,\,$ 1/2$\,\,$ & $\,\,$ 1  $\,\,$ \\
\end{pmatrix},
\qquad
\lambda_{\max} =
4.2463,
\qquad
CR = 0.0929
\end{equation*}

\begin{equation*}
\mathbf{w}^{EM} =
\begin{pmatrix}
0.672582\\
0.196483\\
\color{red} 0.086956\color{black} \\
0.043979
\end{pmatrix}\end{equation*}
\begin{equation*}
\left[ \frac{{w}^{EM}_i}{{w}^{EM}_j} \right] =
\begin{pmatrix}
$\,\,$ 1 $\,\,$ & $\,\,$3.4231$\,\,$ & $\,\,$\color{red} 7.7348\color{black} $\,\,$ & $\,\,$15.2932$\,\,$ \\
$\,\,$0.2921$\,\,$ & $\,\,$ 1 $\,\,$ & $\,\,$\color{red} 2.2596\color{black} $\,\,$ & $\,\,$4.4676  $\,\,$ \\
$\,\,$\color{red} 0.1293\color{black} $\,\,$ & $\,\,$\color{red} 0.4426\color{black} $\,\,$ & $\,\,$ 1 $\,\,$ & $\,\,$\color{red} 1.9772\color{black}  $\,\,$ \\
$\,\,$0.0654$\,\,$ & $\,\,$0.2238$\,\,$ & $\,\,$\color{red} 0.5058\color{black} $\,\,$ & $\,\,$ 1  $\,\,$ \\
\end{pmatrix},
\end{equation*}

\begin{equation*}
\mathbf{w}^{\prime} =
\begin{pmatrix}
0.671908\\
0.196286\\
0.087870\\
0.043935
\end{pmatrix} =
0.998998\cdot
\begin{pmatrix}
0.672582\\
0.196483\\
\color{gr} 0.087958\color{black} \\
0.043979
\end{pmatrix},
\end{equation*}
\begin{equation*}
\left[ \frac{{w}^{\prime}_i}{{w}^{\prime}_j} \right] =
\begin{pmatrix}
$\,\,$ 1 $\,\,$ & $\,\,$3.4231$\,\,$ & $\,\,$\color{gr} 7.6466\color{black} $\,\,$ & $\,\,$15.2932$\,\,$ \\
$\,\,$0.2921$\,\,$ & $\,\,$ 1 $\,\,$ & $\,\,$\color{gr} 2.2338\color{black} $\,\,$ & $\,\,$4.4676  $\,\,$ \\
$\,\,$\color{gr} 0.1308\color{black} $\,\,$ & $\,\,$\color{gr} 0.4477\color{black} $\,\,$ & $\,\,$ 1 $\,\,$ & $\,\,$\color{gr} \color{blue} 2\color{black}  $\,\,$ \\
$\,\,$0.0654$\,\,$ & $\,\,$0.2238$\,\,$ & $\,\,$\color{gr} \color{blue}  1/2\color{black} $\,\,$ & $\,\,$ 1  $\,\,$ \\
\end{pmatrix},
\end{equation*}
\end{example}
\newpage
\begin{example}
\begin{equation*}
\mathbf{A} =
\begin{pmatrix}
$\,\,$ 1 $\,\,$ & $\,\,$6$\,\,$ & $\,\,$8$\,\,$ & $\,\,$9 $\,\,$ \\
$\,\,$ 1/6$\,\,$ & $\,\,$ 1 $\,\,$ & $\,\,$2$\,\,$ & $\,\,$6 $\,\,$ \\
$\,\,$ 1/8$\,\,$ & $\,\,$ 1/2$\,\,$ & $\,\,$ 1 $\,\,$ & $\,\,$2 $\,\,$ \\
$\,\,$ 1/9$\,\,$ & $\,\,$ 1/6$\,\,$ & $\,\,$ 1/2$\,\,$ & $\,\,$ 1  $\,\,$ \\
\end{pmatrix},
\qquad
\lambda_{\max} =
4.1664,
\qquad
CR = 0.0627
\end{equation*}

\begin{equation*}
\mathbf{w}^{EM} =
\begin{pmatrix}
0.688551\\
0.179205\\
\color{red} 0.085196\color{black} \\
0.047047
\end{pmatrix}\end{equation*}
\begin{equation*}
\left[ \frac{{w}^{EM}_i}{{w}^{EM}_j} \right] =
\begin{pmatrix}
$\,\,$ 1 $\,\,$ & $\,\,$3.8422$\,\,$ & $\,\,$\color{red} 8.0819\color{black} $\,\,$ & $\,\,$14.6352$\,\,$ \\
$\,\,$0.2603$\,\,$ & $\,\,$ 1 $\,\,$ & $\,\,$\color{red} 2.1034\color{black} $\,\,$ & $\,\,$3.8090  $\,\,$ \\
$\,\,$\color{red} 0.1237\color{black} $\,\,$ & $\,\,$\color{red} 0.4754\color{black} $\,\,$ & $\,\,$ 1 $\,\,$ & $\,\,$\color{red} 1.8109\color{black}  $\,\,$ \\
$\,\,$0.0683$\,\,$ & $\,\,$0.2625$\,\,$ & $\,\,$\color{red} 0.5522\color{black} $\,\,$ & $\,\,$ 1  $\,\,$ \\
\end{pmatrix},
\end{equation*}

\begin{equation*}
\mathbf{w}^{\prime} =
\begin{pmatrix}
0.687951\\
0.179049\\
0.085994\\
0.047006
\end{pmatrix} =
0.999128\cdot
\begin{pmatrix}
0.688551\\
0.179205\\
\color{gr} 0.086069\color{black} \\
0.047047
\end{pmatrix},
\end{equation*}
\begin{equation*}
\left[ \frac{{w}^{\prime}_i}{{w}^{\prime}_j} \right] =
\begin{pmatrix}
$\,\,$ 1 $\,\,$ & $\,\,$3.8422$\,\,$ & $\,\,$\color{gr} \color{blue} 8\color{black} $\,\,$ & $\,\,$14.6352$\,\,$ \\
$\,\,$0.2603$\,\,$ & $\,\,$ 1 $\,\,$ & $\,\,$\color{gr} 2.0821\color{black} $\,\,$ & $\,\,$3.8090  $\,\,$ \\
$\,\,$\color{gr} \color{blue}  1/8\color{black} $\,\,$ & $\,\,$\color{gr} 0.4803\color{black} $\,\,$ & $\,\,$ 1 $\,\,$ & $\,\,$\color{gr} 1.8294\color{black}  $\,\,$ \\
$\,\,$0.0683$\,\,$ & $\,\,$0.2625$\,\,$ & $\,\,$\color{gr} 0.5466\color{black} $\,\,$ & $\,\,$ 1  $\,\,$ \\
\end{pmatrix},
\end{equation*}
\end{example}
\newpage
\begin{example}
\begin{equation*}
\mathbf{A} =
\begin{pmatrix}
$\,\,$ 1 $\,\,$ & $\,\,$6$\,\,$ & $\,\,$8$\,\,$ & $\,\,$9 $\,\,$ \\
$\,\,$ 1/6$\,\,$ & $\,\,$ 1 $\,\,$ & $\,\,$2$\,\,$ & $\,\,$7 $\,\,$ \\
$\,\,$ 1/8$\,\,$ & $\,\,$ 1/2$\,\,$ & $\,\,$ 1 $\,\,$ & $\,\,$2 $\,\,$ \\
$\,\,$ 1/9$\,\,$ & $\,\,$ 1/7$\,\,$ & $\,\,$ 1/2$\,\,$ & $\,\,$ 1  $\,\,$ \\
\end{pmatrix},
\qquad
\lambda_{\max} =
4.2065,
\qquad
CR = 0.0779
\end{equation*}

\begin{equation*}
\mathbf{w}^{EM} =
\begin{pmatrix}
0.684631\\
0.186376\\
\color{red} 0.083885\color{black} \\
0.045107
\end{pmatrix}\end{equation*}
\begin{equation*}
\left[ \frac{{w}^{EM}_i}{{w}^{EM}_j} \right] =
\begin{pmatrix}
$\,\,$ 1 $\,\,$ & $\,\,$3.6734$\,\,$ & $\,\,$\color{red} 8.1615\color{black} $\,\,$ & $\,\,$15.1779$\,\,$ \\
$\,\,$0.2722$\,\,$ & $\,\,$ 1 $\,\,$ & $\,\,$\color{red} 2.2218\color{black} $\,\,$ & $\,\,$4.1319  $\,\,$ \\
$\,\,$\color{red} 0.1225\color{black} $\,\,$ & $\,\,$\color{red} 0.4501\color{black} $\,\,$ & $\,\,$ 1 $\,\,$ & $\,\,$\color{red} 1.8597\color{black}  $\,\,$ \\
$\,\,$0.0659$\,\,$ & $\,\,$0.2420$\,\,$ & $\,\,$\color{red} 0.5377\color{black} $\,\,$ & $\,\,$ 1  $\,\,$ \\
\end{pmatrix},
\end{equation*}

\begin{equation*}
\mathbf{w}^{\prime} =
\begin{pmatrix}
0.683474\\
0.186061\\
0.085434\\
0.045031
\end{pmatrix} =
0.998309\cdot
\begin{pmatrix}
0.684631\\
0.186376\\
\color{gr} 0.085579\color{black} \\
0.045107
\end{pmatrix},
\end{equation*}
\begin{equation*}
\left[ \frac{{w}^{\prime}_i}{{w}^{\prime}_j} \right] =
\begin{pmatrix}
$\,\,$ 1 $\,\,$ & $\,\,$3.6734$\,\,$ & $\,\,$\color{gr} \color{blue} 8\color{black} $\,\,$ & $\,\,$15.1779$\,\,$ \\
$\,\,$0.2722$\,\,$ & $\,\,$ 1 $\,\,$ & $\,\,$\color{gr} 2.1778\color{black} $\,\,$ & $\,\,$4.1319  $\,\,$ \\
$\,\,$\color{gr} \color{blue}  1/8\color{black} $\,\,$ & $\,\,$\color{gr} 0.4592\color{black} $\,\,$ & $\,\,$ 1 $\,\,$ & $\,\,$\color{gr} 1.8972\color{black}  $\,\,$ \\
$\,\,$0.0659$\,\,$ & $\,\,$0.2420$\,\,$ & $\,\,$\color{gr} 0.5271\color{black} $\,\,$ & $\,\,$ 1  $\,\,$ \\
\end{pmatrix},
\end{equation*}
\end{example}
\newpage
\begin{example}
\begin{equation*}
\mathbf{A} =
\begin{pmatrix}
$\,\,$ 1 $\,\,$ & $\,\,$6$\,\,$ & $\,\,$8$\,\,$ & $\,\,$9 $\,\,$ \\
$\,\,$ 1/6$\,\,$ & $\,\,$ 1 $\,\,$ & $\,\,$2$\,\,$ & $\,\,$8 $\,\,$ \\
$\,\,$ 1/8$\,\,$ & $\,\,$ 1/2$\,\,$ & $\,\,$ 1 $\,\,$ & $\,\,$2 $\,\,$ \\
$\,\,$ 1/9$\,\,$ & $\,\,$ 1/8$\,\,$ & $\,\,$ 1/2$\,\,$ & $\,\,$ 1  $\,\,$ \\
\end{pmatrix},
\qquad
\lambda_{\max} =
4.2469,
\qquad
CR = 0.0931
\end{equation*}

\begin{equation*}
\mathbf{w}^{EM} =
\begin{pmatrix}
0.680854\\
0.192981\\
\color{red} 0.082702\color{black} \\
0.043464
\end{pmatrix}\end{equation*}
\begin{equation*}
\left[ \frac{{w}^{EM}_i}{{w}^{EM}_j} \right] =
\begin{pmatrix}
$\,\,$ 1 $\,\,$ & $\,\,$3.5281$\,\,$ & $\,\,$\color{red} 8.2327\color{black} $\,\,$ & $\,\,$15.6648$\,\,$ \\
$\,\,$0.2834$\,\,$ & $\,\,$ 1 $\,\,$ & $\,\,$\color{red} 2.3335\color{black} $\,\,$ & $\,\,$4.4400  $\,\,$ \\
$\,\,$\color{red} 0.1215\color{black} $\,\,$ & $\,\,$\color{red} 0.4285\color{black} $\,\,$ & $\,\,$ 1 $\,\,$ & $\,\,$\color{red} 1.9028\color{black}  $\,\,$ \\
$\,\,$0.0638$\,\,$ & $\,\,$0.2252$\,\,$ & $\,\,$\color{red} 0.5256\color{black} $\,\,$ & $\,\,$ 1  $\,\,$ \\
\end{pmatrix},
\end{equation*}

\begin{equation*}
\mathbf{w}^{\prime} =
\begin{pmatrix}
0.679220\\
0.192518\\
0.084903\\
0.043360
\end{pmatrix} =
0.997601\cdot
\begin{pmatrix}
0.680854\\
0.192981\\
\color{gr} 0.085107\color{black} \\
0.043464
\end{pmatrix},
\end{equation*}
\begin{equation*}
\left[ \frac{{w}^{\prime}_i}{{w}^{\prime}_j} \right] =
\begin{pmatrix}
$\,\,$ 1 $\,\,$ & $\,\,$3.5281$\,\,$ & $\,\,$\color{gr} \color{blue} 8\color{black} $\,\,$ & $\,\,$15.6648$\,\,$ \\
$\,\,$0.2834$\,\,$ & $\,\,$ 1 $\,\,$ & $\,\,$\color{gr} 2.2675\color{black} $\,\,$ & $\,\,$4.4400  $\,\,$ \\
$\,\,$\color{gr} \color{blue}  1/8\color{black} $\,\,$ & $\,\,$\color{gr} 0.4410\color{black} $\,\,$ & $\,\,$ 1 $\,\,$ & $\,\,$\color{gr} 1.9581\color{black}  $\,\,$ \\
$\,\,$0.0638$\,\,$ & $\,\,$0.2252$\,\,$ & $\,\,$\color{gr} 0.5107\color{black} $\,\,$ & $\,\,$ 1  $\,\,$ \\
\end{pmatrix},
\end{equation*}
\end{example}
\newpage
\begin{example}
\begin{equation*}
\mathbf{A} =
\begin{pmatrix}
$\,\,$ 1 $\,\,$ & $\,\,$7$\,\,$ & $\,\,$4$\,\,$ & $\,\,$4 $\,\,$ \\
$\,\,$ 1/7$\,\,$ & $\,\,$ 1 $\,\,$ & $\,\,$1$\,\,$ & $\,\,$3 $\,\,$ \\
$\,\,$ 1/4$\,\,$ & $\,\,$ 1 $\,\,$ & $\,\,$ 1 $\,\,$ & $\,\,$2 $\,\,$ \\
$\,\,$ 1/4$\,\,$ & $\,\,$ 1/3$\,\,$ & $\,\,$ 1/2$\,\,$ & $\,\,$ 1  $\,\,$ \\
\end{pmatrix},
\qquad
\lambda_{\max} =
4.2421,
\qquad
CR = 0.0913
\end{equation*}

\begin{equation*}
\mathbf{w}^{EM} =
\begin{pmatrix}
0.615227\\
0.151849\\
\color{red} 0.147174\color{black} \\
0.085750
\end{pmatrix}\end{equation*}
\begin{equation*}
\left[ \frac{{w}^{EM}_i}{{w}^{EM}_j} \right] =
\begin{pmatrix}
$\,\,$ 1 $\,\,$ & $\,\,$4.0516$\,\,$ & $\,\,$\color{red} 4.1803\color{black} $\,\,$ & $\,\,$7.1747$\,\,$ \\
$\,\,$0.2468$\,\,$ & $\,\,$ 1 $\,\,$ & $\,\,$\color{red} 1.0318\color{black} $\,\,$ & $\,\,$1.7708  $\,\,$ \\
$\,\,$\color{red} 0.2392\color{black} $\,\,$ & $\,\,$\color{red} 0.9692\color{black} $\,\,$ & $\,\,$ 1 $\,\,$ & $\,\,$\color{red} 1.7163\color{black}  $\,\,$ \\
$\,\,$0.1394$\,\,$ & $\,\,$0.5647$\,\,$ & $\,\,$\color{red} 0.5826\color{black} $\,\,$ & $\,\,$ 1  $\,\,$ \\
\end{pmatrix},
\end{equation*}

\begin{equation*}
\mathbf{w}^{\prime} =
\begin{pmatrix}
0.612364\\
0.151143\\
0.151143\\
0.085351
\end{pmatrix} =
0.995347\cdot
\begin{pmatrix}
0.615227\\
0.151849\\
\color{gr} 0.151849\color{black} \\
0.085750
\end{pmatrix},
\end{equation*}
\begin{equation*}
\left[ \frac{{w}^{\prime}_i}{{w}^{\prime}_j} \right] =
\begin{pmatrix}
$\,\,$ 1 $\,\,$ & $\,\,$4.0516$\,\,$ & $\,\,$\color{gr} 4.0516\color{black} $\,\,$ & $\,\,$7.1747$\,\,$ \\
$\,\,$0.2468$\,\,$ & $\,\,$ 1 $\,\,$ & $\,\,$\color{gr} \color{blue} 1\color{black} $\,\,$ & $\,\,$1.7708  $\,\,$ \\
$\,\,$\color{gr} 0.2468\color{black} $\,\,$ & $\,\,$\color{gr} \color{blue} 1\color{black} $\,\,$ & $\,\,$ 1 $\,\,$ & $\,\,$\color{gr} 1.7708\color{black}  $\,\,$ \\
$\,\,$0.1394$\,\,$ & $\,\,$0.5647$\,\,$ & $\,\,$\color{gr} 0.5647\color{black} $\,\,$ & $\,\,$ 1  $\,\,$ \\
\end{pmatrix},
\end{equation*}
\end{example}
\newpage
\begin{example}
\begin{equation*}
\mathbf{A} =
\begin{pmatrix}
$\,\,$ 1 $\,\,$ & $\,\,$7$\,\,$ & $\,\,$4$\,\,$ & $\,\,$5 $\,\,$ \\
$\,\,$ 1/7$\,\,$ & $\,\,$ 1 $\,\,$ & $\,\,$1$\,\,$ & $\,\,$3 $\,\,$ \\
$\,\,$ 1/4$\,\,$ & $\,\,$ 1 $\,\,$ & $\,\,$ 1 $\,\,$ & $\,\,$2 $\,\,$ \\
$\,\,$ 1/5$\,\,$ & $\,\,$ 1/3$\,\,$ & $\,\,$ 1/2$\,\,$ & $\,\,$ 1  $\,\,$ \\
\end{pmatrix},
\qquad
\lambda_{\max} =
4.1782,
\qquad
CR = 0.0672
\end{equation*}

\begin{equation*}
\mathbf{w}^{EM} =
\begin{pmatrix}
0.629718\\
0.147441\\
\color{red} 0.144946\color{black} \\
0.077895
\end{pmatrix}\end{equation*}
\begin{equation*}
\left[ \frac{{w}^{EM}_i}{{w}^{EM}_j} \right] =
\begin{pmatrix}
$\,\,$ 1 $\,\,$ & $\,\,$4.2710$\,\,$ & $\,\,$\color{red} 4.3445\color{black} $\,\,$ & $\,\,$8.0842$\,\,$ \\
$\,\,$0.2341$\,\,$ & $\,\,$ 1 $\,\,$ & $\,\,$\color{red} 1.0172\color{black} $\,\,$ & $\,\,$1.8928  $\,\,$ \\
$\,\,$\color{red} 0.2302\color{black} $\,\,$ & $\,\,$\color{red} 0.9831\color{black} $\,\,$ & $\,\,$ 1 $\,\,$ & $\,\,$\color{red} 1.8608\color{black}  $\,\,$ \\
$\,\,$0.1237$\,\,$ & $\,\,$0.5283$\,\,$ & $\,\,$\color{red} 0.5374\color{black} $\,\,$ & $\,\,$ 1  $\,\,$ \\
\end{pmatrix},
\end{equation*}

\begin{equation*}
\mathbf{w}^{\prime} =
\begin{pmatrix}
0.628151\\
0.147074\\
0.147074\\
0.077701
\end{pmatrix} =
0.997511\cdot
\begin{pmatrix}
0.629718\\
0.147441\\
\color{gr} 0.147441\color{black} \\
0.077895
\end{pmatrix},
\end{equation*}
\begin{equation*}
\left[ \frac{{w}^{\prime}_i}{{w}^{\prime}_j} \right] =
\begin{pmatrix}
$\,\,$ 1 $\,\,$ & $\,\,$4.2710$\,\,$ & $\,\,$\color{gr} 4.2710\color{black} $\,\,$ & $\,\,$8.0842$\,\,$ \\
$\,\,$0.2341$\,\,$ & $\,\,$ 1 $\,\,$ & $\,\,$\color{gr} \color{blue} 1\color{black} $\,\,$ & $\,\,$1.8928  $\,\,$ \\
$\,\,$\color{gr} 0.2341\color{black} $\,\,$ & $\,\,$\color{gr} \color{blue} 1\color{black} $\,\,$ & $\,\,$ 1 $\,\,$ & $\,\,$\color{gr} 1.8928\color{black}  $\,\,$ \\
$\,\,$0.1237$\,\,$ & $\,\,$0.5283$\,\,$ & $\,\,$\color{gr} 0.5283\color{black} $\,\,$ & $\,\,$ 1  $\,\,$ \\
\end{pmatrix},
\end{equation*}
\end{example}
\newpage
\begin{example}
\begin{equation*}
\mathbf{A} =
\begin{pmatrix}
$\,\,$ 1 $\,\,$ & $\,\,$7$\,\,$ & $\,\,$4$\,\,$ & $\,\,$5 $\,\,$ \\
$\,\,$ 1/7$\,\,$ & $\,\,$ 1 $\,\,$ & $\,\,$1$\,\,$ & $\,\,$4 $\,\,$ \\
$\,\,$ 1/4$\,\,$ & $\,\,$ 1 $\,\,$ & $\,\,$ 1 $\,\,$ & $\,\,$2 $\,\,$ \\
$\,\,$ 1/5$\,\,$ & $\,\,$ 1/4$\,\,$ & $\,\,$ 1/2$\,\,$ & $\,\,$ 1  $\,\,$ \\
\end{pmatrix},
\qquad
\lambda_{\max} =
4.2610,
\qquad
CR = 0.0984
\end{equation*}

\begin{equation*}
\mathbf{w}^{EM} =
\begin{pmatrix}
0.626771\\
0.159541\\
\color{red} 0.141345\color{black} \\
0.072344
\end{pmatrix}\end{equation*}
\begin{equation*}
\left[ \frac{{w}^{EM}_i}{{w}^{EM}_j} \right] =
\begin{pmatrix}
$\,\,$ 1 $\,\,$ & $\,\,$3.9286$\,\,$ & $\,\,$\color{red} 4.4343\color{black} $\,\,$ & $\,\,$8.6638$\,\,$ \\
$\,\,$0.2545$\,\,$ & $\,\,$ 1 $\,\,$ & $\,\,$\color{red} 1.1287\color{black} $\,\,$ & $\,\,$2.2053  $\,\,$ \\
$\,\,$\color{red} 0.2255\color{black} $\,\,$ & $\,\,$\color{red} 0.8859\color{black} $\,\,$ & $\,\,$ 1 $\,\,$ & $\,\,$\color{red} 1.9538\color{black}  $\,\,$ \\
$\,\,$0.1154$\,\,$ & $\,\,$0.4535$\,\,$ & $\,\,$\color{red} 0.5118\color{black} $\,\,$ & $\,\,$ 1  $\,\,$ \\
\end{pmatrix},
\end{equation*}

\begin{equation*}
\mathbf{w}^{\prime} =
\begin{pmatrix}
0.624682\\
0.159009\\
0.144206\\
0.072103
\end{pmatrix} =
0.996668\cdot
\begin{pmatrix}
0.626771\\
0.159541\\
\color{gr} 0.144688\color{black} \\
0.072344
\end{pmatrix},
\end{equation*}
\begin{equation*}
\left[ \frac{{w}^{\prime}_i}{{w}^{\prime}_j} \right] =
\begin{pmatrix}
$\,\,$ 1 $\,\,$ & $\,\,$3.9286$\,\,$ & $\,\,$\color{gr} 4.3319\color{black} $\,\,$ & $\,\,$8.6638$\,\,$ \\
$\,\,$0.2545$\,\,$ & $\,\,$ 1 $\,\,$ & $\,\,$\color{gr} 1.1027\color{black} $\,\,$ & $\,\,$2.2053  $\,\,$ \\
$\,\,$\color{gr} 0.2308\color{black} $\,\,$ & $\,\,$\color{gr} 0.9069\color{black} $\,\,$ & $\,\,$ 1 $\,\,$ & $\,\,$\color{gr} \color{blue} 2\color{black}  $\,\,$ \\
$\,\,$0.1154$\,\,$ & $\,\,$0.4535$\,\,$ & $\,\,$\color{gr} \color{blue}  1/2\color{black} $\,\,$ & $\,\,$ 1  $\,\,$ \\
\end{pmatrix},
\end{equation*}
\end{example}
\newpage
\begin{example}
\begin{equation*}
\mathbf{A} =
\begin{pmatrix}
$\,\,$ 1 $\,\,$ & $\,\,$7$\,\,$ & $\,\,$4$\,\,$ & $\,\,$6 $\,\,$ \\
$\,\,$ 1/7$\,\,$ & $\,\,$ 1 $\,\,$ & $\,\,$1$\,\,$ & $\,\,$3 $\,\,$ \\
$\,\,$ 1/4$\,\,$ & $\,\,$ 1 $\,\,$ & $\,\,$ 1 $\,\,$ & $\,\,$2 $\,\,$ \\
$\,\,$ 1/6$\,\,$ & $\,\,$ 1/3$\,\,$ & $\,\,$ 1/2$\,\,$ & $\,\,$ 1  $\,\,$ \\
\end{pmatrix},
\qquad
\lambda_{\max} =
4.1365,
\qquad
CR = 0.0515
\end{equation*}

\begin{equation*}
\mathbf{w}^{EM} =
\begin{pmatrix}
0.641153\\
0.143772\\
\color{red} 0.142940\color{black} \\
0.072135
\end{pmatrix}\end{equation*}
\begin{equation*}
\left[ \frac{{w}^{EM}_i}{{w}^{EM}_j} \right] =
\begin{pmatrix}
$\,\,$ 1 $\,\,$ & $\,\,$4.4595$\,\,$ & $\,\,$\color{red} 4.4855\color{black} $\,\,$ & $\,\,$8.8882$\,\,$ \\
$\,\,$0.2242$\,\,$ & $\,\,$ 1 $\,\,$ & $\,\,$\color{red} 1.0058\color{black} $\,\,$ & $\,\,$1.9931  $\,\,$ \\
$\,\,$\color{red} 0.2229\color{black} $\,\,$ & $\,\,$\color{red} 0.9942\color{black} $\,\,$ & $\,\,$ 1 $\,\,$ & $\,\,$\color{red} 1.9815\color{black}  $\,\,$ \\
$\,\,$0.1125$\,\,$ & $\,\,$0.5017$\,\,$ & $\,\,$\color{red} 0.5047\color{black} $\,\,$ & $\,\,$ 1  $\,\,$ \\
\end{pmatrix},
\end{equation*}

\begin{equation*}
\mathbf{w}^{\prime} =
\begin{pmatrix}
0.640620\\
0.143652\\
0.143652\\
0.072076
\end{pmatrix} =
0.999169\cdot
\begin{pmatrix}
0.641153\\
0.143772\\
\color{gr} 0.143772\color{black} \\
0.072135
\end{pmatrix},
\end{equation*}
\begin{equation*}
\left[ \frac{{w}^{\prime}_i}{{w}^{\prime}_j} \right] =
\begin{pmatrix}
$\,\,$ 1 $\,\,$ & $\,\,$4.4595$\,\,$ & $\,\,$\color{gr} 4.4595\color{black} $\,\,$ & $\,\,$8.8882$\,\,$ \\
$\,\,$0.2242$\,\,$ & $\,\,$ 1 $\,\,$ & $\,\,$\color{gr} \color{blue} 1\color{black} $\,\,$ & $\,\,$1.9931  $\,\,$ \\
$\,\,$\color{gr} 0.2242\color{black} $\,\,$ & $\,\,$\color{gr} \color{blue} 1\color{black} $\,\,$ & $\,\,$ 1 $\,\,$ & $\,\,$\color{gr} 1.9931\color{black}  $\,\,$ \\
$\,\,$0.1125$\,\,$ & $\,\,$0.5017$\,\,$ & $\,\,$\color{gr} 0.5017\color{black} $\,\,$ & $\,\,$ 1  $\,\,$ \\
\end{pmatrix},
\end{equation*}
\end{example}
\newpage
\begin{example}
\begin{equation*}
\mathbf{A} =
\begin{pmatrix}
$\,\,$ 1 $\,\,$ & $\,\,$7$\,\,$ & $\,\,$4$\,\,$ & $\,\,$7 $\,\,$ \\
$\,\,$ 1/7$\,\,$ & $\,\,$ 1 $\,\,$ & $\,\,$1$\,\,$ & $\,\,$5 $\,\,$ \\
$\,\,$ 1/4$\,\,$ & $\,\,$ 1 $\,\,$ & $\,\,$ 1 $\,\,$ & $\,\,$3 $\,\,$ \\
$\,\,$ 1/7$\,\,$ & $\,\,$ 1/5$\,\,$ & $\,\,$ 1/3$\,\,$ & $\,\,$ 1  $\,\,$ \\
\end{pmatrix},
\qquad
\lambda_{\max} =
4.2251,
\qquad
CR = 0.0849
\end{equation*}

\begin{equation*}
\mathbf{w}^{EM} =
\begin{pmatrix}
0.641114\\
0.157236\\
\color{red} 0.148185\color{black} \\
0.053465
\end{pmatrix}\end{equation*}
\begin{equation*}
\left[ \frac{{w}^{EM}_i}{{w}^{EM}_j} \right] =
\begin{pmatrix}
$\,\,$ 1 $\,\,$ & $\,\,$4.0774$\,\,$ & $\,\,$\color{red} 4.3264\color{black} $\,\,$ & $\,\,$11.9912$\,\,$ \\
$\,\,$0.2453$\,\,$ & $\,\,$ 1 $\,\,$ & $\,\,$\color{red} 1.0611\color{black} $\,\,$ & $\,\,$2.9409  $\,\,$ \\
$\,\,$\color{red} 0.2311\color{black} $\,\,$ & $\,\,$\color{red} 0.9424\color{black} $\,\,$ & $\,\,$ 1 $\,\,$ & $\,\,$\color{red} 2.7716\color{black}  $\,\,$ \\
$\,\,$0.0834$\,\,$ & $\,\,$0.3400$\,\,$ & $\,\,$\color{red} 0.3608\color{black} $\,\,$ & $\,\,$ 1  $\,\,$ \\
\end{pmatrix},
\end{equation*}

\begin{equation*}
\mathbf{w}^{\prime} =
\begin{pmatrix}
0.635363\\
0.155825\\
0.155825\\
0.052986
\end{pmatrix} =
0.991031\cdot
\begin{pmatrix}
0.641114\\
0.157236\\
\color{gr} 0.157236\color{black} \\
0.053465
\end{pmatrix},
\end{equation*}
\begin{equation*}
\left[ \frac{{w}^{\prime}_i}{{w}^{\prime}_j} \right] =
\begin{pmatrix}
$\,\,$ 1 $\,\,$ & $\,\,$4.0774$\,\,$ & $\,\,$\color{gr} 4.0774\color{black} $\,\,$ & $\,\,$11.9912$\,\,$ \\
$\,\,$0.2453$\,\,$ & $\,\,$ 1 $\,\,$ & $\,\,$\color{gr} \color{blue} 1\color{black} $\,\,$ & $\,\,$2.9409  $\,\,$ \\
$\,\,$\color{gr} 0.2453\color{black} $\,\,$ & $\,\,$\color{gr} \color{blue} 1\color{black} $\,\,$ & $\,\,$ 1 $\,\,$ & $\,\,$\color{gr} 2.9409\color{black}  $\,\,$ \\
$\,\,$0.0834$\,\,$ & $\,\,$0.3400$\,\,$ & $\,\,$\color{gr} 0.3400\color{black} $\,\,$ & $\,\,$ 1  $\,\,$ \\
\end{pmatrix},
\end{equation*}
\end{example}
\newpage
\begin{example}
\begin{equation*}
\mathbf{A} =
\begin{pmatrix}
$\,\,$ 1 $\,\,$ & $\,\,$7$\,\,$ & $\,\,$4$\,\,$ & $\,\,$8 $\,\,$ \\
$\,\,$ 1/7$\,\,$ & $\,\,$ 1 $\,\,$ & $\,\,$1$\,\,$ & $\,\,$5 $\,\,$ \\
$\,\,$ 1/4$\,\,$ & $\,\,$ 1 $\,\,$ & $\,\,$ 1 $\,\,$ & $\,\,$3 $\,\,$ \\
$\,\,$ 1/8$\,\,$ & $\,\,$ 1/5$\,\,$ & $\,\,$ 1/3$\,\,$ & $\,\,$ 1  $\,\,$ \\
\end{pmatrix},
\qquad
\lambda_{\max} =
4.1888,
\qquad
CR = 0.0712
\end{equation*}

\begin{equation*}
\mathbf{w}^{EM} =
\begin{pmatrix}
0.648802\\
0.154126\\
\color{red} 0.146643\color{black} \\
0.050429
\end{pmatrix}\end{equation*}
\begin{equation*}
\left[ \frac{{w}^{EM}_i}{{w}^{EM}_j} \right] =
\begin{pmatrix}
$\,\,$ 1 $\,\,$ & $\,\,$4.2096$\,\,$ & $\,\,$\color{red} 4.4244\color{black} $\,\,$ & $\,\,$12.8657$\,\,$ \\
$\,\,$0.2376$\,\,$ & $\,\,$ 1 $\,\,$ & $\,\,$\color{red} 1.0510\color{black} $\,\,$ & $\,\,$3.0563  $\,\,$ \\
$\,\,$\color{red} 0.2260\color{black} $\,\,$ & $\,\,$\color{red} 0.9515\color{black} $\,\,$ & $\,\,$ 1 $\,\,$ & $\,\,$\color{red} 2.9079\color{black}  $\,\,$ \\
$\,\,$0.0777$\,\,$ & $\,\,$0.3272$\,\,$ & $\,\,$\color{red} 0.3439\color{black} $\,\,$ & $\,\,$ 1  $\,\,$ \\
\end{pmatrix},
\end{equation*}

\begin{equation*}
\mathbf{w}^{\prime} =
\begin{pmatrix}
0.645803\\
0.153414\\
0.150587\\
0.050196
\end{pmatrix} =
0.995378\cdot
\begin{pmatrix}
0.648802\\
0.154126\\
\color{gr} 0.151287\color{black} \\
0.050429
\end{pmatrix},
\end{equation*}
\begin{equation*}
\left[ \frac{{w}^{\prime}_i}{{w}^{\prime}_j} \right] =
\begin{pmatrix}
$\,\,$ 1 $\,\,$ & $\,\,$4.2096$\,\,$ & $\,\,$\color{gr} 4.2886\color{black} $\,\,$ & $\,\,$12.8657$\,\,$ \\
$\,\,$0.2376$\,\,$ & $\,\,$ 1 $\,\,$ & $\,\,$\color{gr} 1.0188\color{black} $\,\,$ & $\,\,$3.0563  $\,\,$ \\
$\,\,$\color{gr} 0.2332\color{black} $\,\,$ & $\,\,$\color{gr} 0.9816\color{black} $\,\,$ & $\,\,$ 1 $\,\,$ & $\,\,$\color{gr} \color{blue} 3\color{black}  $\,\,$ \\
$\,\,$0.0777$\,\,$ & $\,\,$0.3272$\,\,$ & $\,\,$\color{gr} \color{blue}  1/3\color{black} $\,\,$ & $\,\,$ 1  $\,\,$ \\
\end{pmatrix},
\end{equation*}
\end{example}
\newpage
\begin{example}
\begin{equation*}
\mathbf{A} =
\begin{pmatrix}
$\,\,$ 1 $\,\,$ & $\,\,$7$\,\,$ & $\,\,$4$\,\,$ & $\,\,$8 $\,\,$ \\
$\,\,$ 1/7$\,\,$ & $\,\,$ 1 $\,\,$ & $\,\,$1$\,\,$ & $\,\,$6 $\,\,$ \\
$\,\,$ 1/4$\,\,$ & $\,\,$ 1 $\,\,$ & $\,\,$ 1 $\,\,$ & $\,\,$4 $\,\,$ \\
$\,\,$ 1/8$\,\,$ & $\,\,$ 1/6$\,\,$ & $\,\,$ 1/4$\,\,$ & $\,\,$ 1  $\,\,$ \\
\end{pmatrix},
\qquad
\lambda_{\max} =
4.2421,
\qquad
CR = 0.0913
\end{equation*}

\begin{equation*}
\mathbf{w}^{EM} =
\begin{pmatrix}
0.642787\\
0.158651\\
\color{red} 0.153767\color{black} \\
0.044795
\end{pmatrix}\end{equation*}
\begin{equation*}
\left[ \frac{{w}^{EM}_i}{{w}^{EM}_j} \right] =
\begin{pmatrix}
$\,\,$ 1 $\,\,$ & $\,\,$4.0516$\,\,$ & $\,\,$\color{red} 4.1803\color{black} $\,\,$ & $\,\,$14.3494$\,\,$ \\
$\,\,$0.2468$\,\,$ & $\,\,$ 1 $\,\,$ & $\,\,$\color{red} 1.0318\color{black} $\,\,$ & $\,\,$3.5417  $\,\,$ \\
$\,\,$\color{red} 0.2392\color{black} $\,\,$ & $\,\,$\color{red} 0.9692\color{black} $\,\,$ & $\,\,$ 1 $\,\,$ & $\,\,$\color{red} 3.4326\color{black}  $\,\,$ \\
$\,\,$0.0697$\,\,$ & $\,\,$0.2824$\,\,$ & $\,\,$\color{red} 0.2913\color{black} $\,\,$ & $\,\,$ 1  $\,\,$ \\
\end{pmatrix},
\end{equation*}

\begin{equation*}
\mathbf{w}^{\prime} =
\begin{pmatrix}
0.639662\\
0.157880\\
0.157880\\
0.044578
\end{pmatrix} =
0.995139\cdot
\begin{pmatrix}
0.642787\\
0.158651\\
\color{gr} 0.158651\color{black} \\
0.044795
\end{pmatrix},
\end{equation*}
\begin{equation*}
\left[ \frac{{w}^{\prime}_i}{{w}^{\prime}_j} \right] =
\begin{pmatrix}
$\,\,$ 1 $\,\,$ & $\,\,$4.0516$\,\,$ & $\,\,$\color{gr} 4.0516\color{black} $\,\,$ & $\,\,$14.3494$\,\,$ \\
$\,\,$0.2468$\,\,$ & $\,\,$ 1 $\,\,$ & $\,\,$\color{gr} \color{blue} 1\color{black} $\,\,$ & $\,\,$3.5417  $\,\,$ \\
$\,\,$\color{gr} 0.2468\color{black} $\,\,$ & $\,\,$\color{gr} \color{blue} 1\color{black} $\,\,$ & $\,\,$ 1 $\,\,$ & $\,\,$\color{gr} 3.5417\color{black}  $\,\,$ \\
$\,\,$0.0697$\,\,$ & $\,\,$0.2824$\,\,$ & $\,\,$\color{gr} 0.2824\color{black} $\,\,$ & $\,\,$ 1  $\,\,$ \\
\end{pmatrix},
\end{equation*}
\end{example}
\newpage
\begin{example}
\begin{equation*}
\mathbf{A} =
\begin{pmatrix}
$\,\,$ 1 $\,\,$ & $\,\,$7$\,\,$ & $\,\,$4$\,\,$ & $\,\,$8 $\,\,$ \\
$\,\,$ 1/7$\,\,$ & $\,\,$ 1 $\,\,$ & $\,\,$3$\,\,$ & $\,\,$2 $\,\,$ \\
$\,\,$ 1/4$\,\,$ & $\,\,$ 1/3$\,\,$ & $\,\,$ 1 $\,\,$ & $\,\,$1 $\,\,$ \\
$\,\,$ 1/8$\,\,$ & $\,\,$ 1/2$\,\,$ & $\,\,$ 1 $\,\,$ & $\,\,$ 1  $\,\,$ \\
\end{pmatrix},
\qquad
\lambda_{\max} =
4.2421,
\qquad
CR = 0.0913
\end{equation*}

\begin{equation*}
\mathbf{w}^{EM} =
\begin{pmatrix}
0.664096\\
0.163911\\
0.092561\\
\color{red} 0.079432\color{black}
\end{pmatrix}\end{equation*}
\begin{equation*}
\left[ \frac{{w}^{EM}_i}{{w}^{EM}_j} \right] =
\begin{pmatrix}
$\,\,$ 1 $\,\,$ & $\,\,$4.0516$\,\,$ & $\,\,$7.1747$\,\,$ & $\,\,$\color{red} 8.3605\color{black} $\,\,$ \\
$\,\,$0.2468$\,\,$ & $\,\,$ 1 $\,\,$ & $\,\,$1.7708$\,\,$ & $\,\,$\color{red} 2.0635\color{black}   $\,\,$ \\
$\,\,$0.1394$\,\,$ & $\,\,$0.5647$\,\,$ & $\,\,$ 1 $\,\,$ & $\,\,$\color{red} 1.1653\color{black}  $\,\,$ \\
$\,\,$\color{red} 0.1196\color{black} $\,\,$ & $\,\,$\color{red} 0.4846\color{black} $\,\,$ & $\,\,$\color{red} 0.8582\color{black} $\,\,$ & $\,\,$ 1  $\,\,$ \\
\end{pmatrix},
\end{equation*}

\begin{equation*}
\mathbf{w}^{\prime} =
\begin{pmatrix}
0.662425\\
0.163498\\
0.092328\\
0.081749
\end{pmatrix} =
0.997483\cdot
\begin{pmatrix}
0.664096\\
0.163911\\
0.092561\\
\color{gr} 0.081955\color{black}
\end{pmatrix},
\end{equation*}
\begin{equation*}
\left[ \frac{{w}^{\prime}_i}{{w}^{\prime}_j} \right] =
\begin{pmatrix}
$\,\,$ 1 $\,\,$ & $\,\,$4.0516$\,\,$ & $\,\,$7.1747$\,\,$ & $\,\,$\color{gr} 8.1031\color{black} $\,\,$ \\
$\,\,$0.2468$\,\,$ & $\,\,$ 1 $\,\,$ & $\,\,$1.7708$\,\,$ & $\,\,$\color{gr} \color{blue} 2\color{black}   $\,\,$ \\
$\,\,$0.1394$\,\,$ & $\,\,$0.5647$\,\,$ & $\,\,$ 1 $\,\,$ & $\,\,$\color{gr} 1.1294\color{black}  $\,\,$ \\
$\,\,$\color{gr} 0.1234\color{black} $\,\,$ & $\,\,$\color{gr} \color{blue}  1/2\color{black} $\,\,$ & $\,\,$\color{gr} 0.8854\color{black} $\,\,$ & $\,\,$ 1  $\,\,$ \\
\end{pmatrix},
\end{equation*}
\end{example}
\newpage
\begin{example}
\begin{equation*}
\mathbf{A} =
\begin{pmatrix}
$\,\,$ 1 $\,\,$ & $\,\,$7$\,\,$ & $\,\,$4$\,\,$ & $\,\,$9 $\,\,$ \\
$\,\,$ 1/7$\,\,$ & $\,\,$ 1 $\,\,$ & $\,\,$1$\,\,$ & $\,\,$6 $\,\,$ \\
$\,\,$ 1/4$\,\,$ & $\,\,$ 1 $\,\,$ & $\,\,$ 1 $\,\,$ & $\,\,$4 $\,\,$ \\
$\,\,$ 1/9$\,\,$ & $\,\,$ 1/6$\,\,$ & $\,\,$ 1/4$\,\,$ & $\,\,$ 1  $\,\,$ \\
\end{pmatrix},
\qquad
\lambda_{\max} =
4.2065,
\qquad
CR = 0.0779
\end{equation*}

\begin{equation*}
\mathbf{w}^{EM} =
\begin{pmatrix}
0.649412\\
0.155883\\
\color{red} 0.152230\color{black} \\
0.042474
\end{pmatrix}\end{equation*}
\begin{equation*}
\left[ \frac{{w}^{EM}_i}{{w}^{EM}_j} \right] =
\begin{pmatrix}
$\,\,$ 1 $\,\,$ & $\,\,$4.1660$\,\,$ & $\,\,$\color{red} 4.2660\color{black} $\,\,$ & $\,\,$15.2896$\,\,$ \\
$\,\,$0.2400$\,\,$ & $\,\,$ 1 $\,\,$ & $\,\,$\color{red} 1.0240\color{black} $\,\,$ & $\,\,$3.6701  $\,\,$ \\
$\,\,$\color{red} 0.2344\color{black} $\,\,$ & $\,\,$\color{red} 0.9766\color{black} $\,\,$ & $\,\,$ 1 $\,\,$ & $\,\,$\color{red} 3.5841\color{black}  $\,\,$ \\
$\,\,$0.0654$\,\,$ & $\,\,$0.2725$\,\,$ & $\,\,$\color{red} 0.2790\color{black} $\,\,$ & $\,\,$ 1  $\,\,$ \\
\end{pmatrix},
\end{equation*}

\begin{equation*}
\mathbf{w}^{\prime} =
\begin{pmatrix}
0.647048\\
0.155316\\
0.155316\\
0.042319
\end{pmatrix} =
0.996360\cdot
\begin{pmatrix}
0.649412\\
0.155883\\
\color{gr} 0.155883\color{black} \\
0.042474
\end{pmatrix},
\end{equation*}
\begin{equation*}
\left[ \frac{{w}^{\prime}_i}{{w}^{\prime}_j} \right] =
\begin{pmatrix}
$\,\,$ 1 $\,\,$ & $\,\,$4.1660$\,\,$ & $\,\,$\color{gr} 4.1660\color{black} $\,\,$ & $\,\,$15.2896$\,\,$ \\
$\,\,$0.2400$\,\,$ & $\,\,$ 1 $\,\,$ & $\,\,$\color{gr} \color{blue} 1\color{black} $\,\,$ & $\,\,$3.6701  $\,\,$ \\
$\,\,$\color{gr} 0.2400\color{black} $\,\,$ & $\,\,$\color{gr} \color{blue} 1\color{black} $\,\,$ & $\,\,$ 1 $\,\,$ & $\,\,$\color{gr} 3.6701\color{black}  $\,\,$ \\
$\,\,$0.0654$\,\,$ & $\,\,$0.2725$\,\,$ & $\,\,$\color{gr} 0.2725\color{black} $\,\,$ & $\,\,$ 1  $\,\,$ \\
\end{pmatrix},
\end{equation*}
\end{example}
\newpage
\begin{example}
\begin{equation*}
\mathbf{A} =
\begin{pmatrix}
$\,\,$ 1 $\,\,$ & $\,\,$7$\,\,$ & $\,\,$4$\,\,$ & $\,\,$9 $\,\,$ \\
$\,\,$ 1/7$\,\,$ & $\,\,$ 1 $\,\,$ & $\,\,$1$\,\,$ & $\,\,$7 $\,\,$ \\
$\,\,$ 1/4$\,\,$ & $\,\,$ 1 $\,\,$ & $\,\,$ 1 $\,\,$ & $\,\,$4 $\,\,$ \\
$\,\,$ 1/9$\,\,$ & $\,\,$ 1/7$\,\,$ & $\,\,$ 1/4$\,\,$ & $\,\,$ 1  $\,\,$ \\
\end{pmatrix},
\qquad
\lambda_{\max} =
4.2506,
\qquad
CR = 0.0945
\end{equation*}

\begin{equation*}
\mathbf{w}^{EM} =
\begin{pmatrix}
0.646978\\
0.162358\\
\color{red} 0.149886\color{black} \\
0.040778
\end{pmatrix}\end{equation*}
\begin{equation*}
\left[ \frac{{w}^{EM}_i}{{w}^{EM}_j} \right] =
\begin{pmatrix}
$\,\,$ 1 $\,\,$ & $\,\,$3.9849$\,\,$ & $\,\,$\color{red} 4.3165\color{black} $\,\,$ & $\,\,$15.8659$\,\,$ \\
$\,\,$0.2509$\,\,$ & $\,\,$ 1 $\,\,$ & $\,\,$\color{red} 1.0832\color{black} $\,\,$ & $\,\,$3.9815  $\,\,$ \\
$\,\,$\color{red} 0.2317\color{black} $\,\,$ & $\,\,$\color{red} 0.9232\color{black} $\,\,$ & $\,\,$ 1 $\,\,$ & $\,\,$\color{red} 3.6757\color{black}  $\,\,$ \\
$\,\,$0.0630$\,\,$ & $\,\,$0.2512$\,\,$ & $\,\,$\color{red} 0.2721\color{black} $\,\,$ & $\,\,$ 1  $\,\,$ \\
\end{pmatrix},
\end{equation*}

\begin{equation*}
\mathbf{w}^{\prime} =
\begin{pmatrix}
0.639396\\
0.160455\\
0.159849\\
0.040300
\end{pmatrix} =
0.988280\cdot
\begin{pmatrix}
0.646978\\
0.162358\\
\color{gr} 0.161745\color{black} \\
0.040778
\end{pmatrix},
\end{equation*}
\begin{equation*}
\left[ \frac{{w}^{\prime}_i}{{w}^{\prime}_j} \right] =
\begin{pmatrix}
$\,\,$ 1 $\,\,$ & $\,\,$3.9849$\,\,$ & $\,\,$\color{gr} \color{blue} 4\color{black} $\,\,$ & $\,\,$15.8659$\,\,$ \\
$\,\,$0.2509$\,\,$ & $\,\,$ 1 $\,\,$ & $\,\,$\color{gr} 1.0038\color{black} $\,\,$ & $\,\,$3.9815  $\,\,$ \\
$\,\,$\color{gr} \color{blue}  1/4\color{black} $\,\,$ & $\,\,$\color{gr} 0.9962\color{black} $\,\,$ & $\,\,$ 1 $\,\,$ & $\,\,$\color{gr} 3.9665\color{black}  $\,\,$ \\
$\,\,$0.0630$\,\,$ & $\,\,$0.2512$\,\,$ & $\,\,$\color{gr} 0.2521\color{black} $\,\,$ & $\,\,$ 1  $\,\,$ \\
\end{pmatrix},
\end{equation*}
\end{example}
\newpage
\begin{example}
\begin{equation*}
\mathbf{A} =
\begin{pmatrix}
$\,\,$ 1 $\,\,$ & $\,\,$7$\,\,$ & $\,\,$4$\,\,$ & $\,\,$9 $\,\,$ \\
$\,\,$ 1/7$\,\,$ & $\,\,$ 1 $\,\,$ & $\,\,$1$\,\,$ & $\,\,$7 $\,\,$ \\
$\,\,$ 1/4$\,\,$ & $\,\,$ 1 $\,\,$ & $\,\,$ 1 $\,\,$ & $\,\,$5 $\,\,$ \\
$\,\,$ 1/9$\,\,$ & $\,\,$ 1/7$\,\,$ & $\,\,$ 1/5$\,\,$ & $\,\,$ 1  $\,\,$ \\
\end{pmatrix},
\qquad
\lambda_{\max} =
4.2574,
\qquad
CR = 0.0971
\end{equation*}

\begin{equation*}
\mathbf{w}^{EM} =
\begin{pmatrix}
0.643823\\
0.159739\\
\color{red} 0.157783\color{black} \\
0.038654
\end{pmatrix}\end{equation*}
\begin{equation*}
\left[ \frac{{w}^{EM}_i}{{w}^{EM}_j} \right] =
\begin{pmatrix}
$\,\,$ 1 $\,\,$ & $\,\,$4.0305$\,\,$ & $\,\,$\color{red} 4.0804\color{black} $\,\,$ & $\,\,$16.6560$\,\,$ \\
$\,\,$0.2481$\,\,$ & $\,\,$ 1 $\,\,$ & $\,\,$\color{red} 1.0124\color{black} $\,\,$ & $\,\,$4.1325  $\,\,$ \\
$\,\,$\color{red} 0.2451\color{black} $\,\,$ & $\,\,$\color{red} 0.9878\color{black} $\,\,$ & $\,\,$ 1 $\,\,$ & $\,\,$\color{red} 4.0819\color{black}  $\,\,$ \\
$\,\,$0.0600$\,\,$ & $\,\,$0.2420$\,\,$ & $\,\,$\color{red} 0.2450\color{black} $\,\,$ & $\,\,$ 1  $\,\,$ \\
\end{pmatrix},
\end{equation*}

\begin{equation*}
\mathbf{w}^{\prime} =
\begin{pmatrix}
0.642566\\
0.159428\\
0.159428\\
0.038579
\end{pmatrix} =
0.998048\cdot
\begin{pmatrix}
0.643823\\
0.159739\\
\color{gr} 0.159739\color{black} \\
0.038654
\end{pmatrix},
\end{equation*}
\begin{equation*}
\left[ \frac{{w}^{\prime}_i}{{w}^{\prime}_j} \right] =
\begin{pmatrix}
$\,\,$ 1 $\,\,$ & $\,\,$4.0305$\,\,$ & $\,\,$\color{gr} 4.0305\color{black} $\,\,$ & $\,\,$16.6560$\,\,$ \\
$\,\,$0.2481$\,\,$ & $\,\,$ 1 $\,\,$ & $\,\,$\color{gr} \color{blue} 1\color{black} $\,\,$ & $\,\,$4.1325  $\,\,$ \\
$\,\,$\color{gr} 0.2481\color{black} $\,\,$ & $\,\,$\color{gr} \color{blue} 1\color{black} $\,\,$ & $\,\,$ 1 $\,\,$ & $\,\,$\color{gr} 4.1325\color{black}  $\,\,$ \\
$\,\,$0.0600$\,\,$ & $\,\,$0.2420$\,\,$ & $\,\,$\color{gr} 0.2420\color{black} $\,\,$ & $\,\,$ 1  $\,\,$ \\
\end{pmatrix},
\end{equation*}
\end{example}
\newpage
\begin{example}
\begin{equation*}
\mathbf{A} =
\begin{pmatrix}
$\,\,$ 1 $\,\,$ & $\,\,$7$\,\,$ & $\,\,$5$\,\,$ & $\,\,$6 $\,\,$ \\
$\,\,$ 1/7$\,\,$ & $\,\,$ 1 $\,\,$ & $\,\,$1$\,\,$ & $\,\,$4 $\,\,$ \\
$\,\,$ 1/5$\,\,$ & $\,\,$ 1 $\,\,$ & $\,\,$ 1 $\,\,$ & $\,\,$2 $\,\,$ \\
$\,\,$ 1/6$\,\,$ & $\,\,$ 1/4$\,\,$ & $\,\,$ 1/2$\,\,$ & $\,\,$ 1  $\,\,$ \\
\end{pmatrix},
\qquad
\lambda_{\max} =
4.2095,
\qquad
CR = 0.0790
\end{equation*}

\begin{equation*}
\mathbf{w}^{EM} =
\begin{pmatrix}
0.653934\\
0.151318\\
\color{red} 0.128919\color{black} \\
0.065829
\end{pmatrix}\end{equation*}
\begin{equation*}
\left[ \frac{{w}^{EM}_i}{{w}^{EM}_j} \right] =
\begin{pmatrix}
$\,\,$ 1 $\,\,$ & $\,\,$4.3216$\,\,$ & $\,\,$\color{red} 5.0725\color{black} $\,\,$ & $\,\,$9.9338$\,\,$ \\
$\,\,$0.2314$\,\,$ & $\,\,$ 1 $\,\,$ & $\,\,$\color{red} 1.1737\color{black} $\,\,$ & $\,\,$2.2987  $\,\,$ \\
$\,\,$\color{red} 0.1971\color{black} $\,\,$ & $\,\,$\color{red} 0.8520\color{black} $\,\,$ & $\,\,$ 1 $\,\,$ & $\,\,$\color{red} 1.9584\color{black}  $\,\,$ \\
$\,\,$0.1007$\,\,$ & $\,\,$0.4350$\,\,$ & $\,\,$\color{red} 0.5106\color{black} $\,\,$ & $\,\,$ 1  $\,\,$ \\
\end{pmatrix},
\end{equation*}

\begin{equation*}
\mathbf{w}^{\prime} =
\begin{pmatrix}
0.652715\\
0.151036\\
0.130543\\
0.065706
\end{pmatrix} =
0.998135\cdot
\begin{pmatrix}
0.653934\\
0.151318\\
\color{gr} 0.130787\color{black} \\
0.065829
\end{pmatrix},
\end{equation*}
\begin{equation*}
\left[ \frac{{w}^{\prime}_i}{{w}^{\prime}_j} \right] =
\begin{pmatrix}
$\,\,$ 1 $\,\,$ & $\,\,$4.3216$\,\,$ & $\,\,$\color{gr} \color{blue} 5\color{black} $\,\,$ & $\,\,$9.9338$\,\,$ \\
$\,\,$0.2314$\,\,$ & $\,\,$ 1 $\,\,$ & $\,\,$\color{gr} 1.1570\color{black} $\,\,$ & $\,\,$2.2987  $\,\,$ \\
$\,\,$\color{gr} \color{blue}  1/5\color{black} $\,\,$ & $\,\,$\color{gr} 0.8643\color{black} $\,\,$ & $\,\,$ 1 $\,\,$ & $\,\,$\color{gr} 1.9868\color{black}  $\,\,$ \\
$\,\,$0.1007$\,\,$ & $\,\,$0.4350$\,\,$ & $\,\,$\color{gr} 0.5033\color{black} $\,\,$ & $\,\,$ 1  $\,\,$ \\
\end{pmatrix},
\end{equation*}
\end{example}
\newpage
\begin{example}
\begin{equation*}
\mathbf{A} =
\begin{pmatrix}
$\,\,$ 1 $\,\,$ & $\,\,$7$\,\,$ & $\,\,$5$\,\,$ & $\,\,$7 $\,\,$ \\
$\,\,$ 1/7$\,\,$ & $\,\,$ 1 $\,\,$ & $\,\,$1$\,\,$ & $\,\,$3 $\,\,$ \\
$\,\,$ 1/5$\,\,$ & $\,\,$ 1 $\,\,$ & $\,\,$ 1 $\,\,$ & $\,\,$2 $\,\,$ \\
$\,\,$ 1/7$\,\,$ & $\,\,$ 1/3$\,\,$ & $\,\,$ 1/2$\,\,$ & $\,\,$ 1  $\,\,$ \\
\end{pmatrix},
\qquad
\lambda_{\max} =
4.1027,
\qquad
CR = 0.0387
\end{equation*}

\begin{equation*}
\mathbf{w}^{EM} =
\begin{pmatrix}
0.667185\\
0.136683\\
\color{red} 0.129810\color{black} \\
0.066322
\end{pmatrix}\end{equation*}
\begin{equation*}
\left[ \frac{{w}^{EM}_i}{{w}^{EM}_j} \right] =
\begin{pmatrix}
$\,\,$ 1 $\,\,$ & $\,\,$4.8813$\,\,$ & $\,\,$\color{red} 5.1397\color{black} $\,\,$ & $\,\,$10.0598$\,\,$ \\
$\,\,$0.2049$\,\,$ & $\,\,$ 1 $\,\,$ & $\,\,$\color{red} 1.0529\color{black} $\,\,$ & $\,\,$2.0609  $\,\,$ \\
$\,\,$\color{red} 0.1946\color{black} $\,\,$ & $\,\,$\color{red} 0.9497\color{black} $\,\,$ & $\,\,$ 1 $\,\,$ & $\,\,$\color{red} 1.9573\color{black}  $\,\,$ \\
$\,\,$0.0994$\,\,$ & $\,\,$0.4852$\,\,$ & $\,\,$\color{red} 0.5109\color{black} $\,\,$ & $\,\,$ 1  $\,\,$ \\
\end{pmatrix},
\end{equation*}

\begin{equation*}
\mathbf{w}^{\prime} =
\begin{pmatrix}
0.665300\\
0.136297\\
0.132269\\
0.066135
\end{pmatrix} =
0.997174\cdot
\begin{pmatrix}
0.667185\\
0.136683\\
\color{gr} 0.132644\color{black} \\
0.066322
\end{pmatrix},
\end{equation*}
\begin{equation*}
\left[ \frac{{w}^{\prime}_i}{{w}^{\prime}_j} \right] =
\begin{pmatrix}
$\,\,$ 1 $\,\,$ & $\,\,$4.8813$\,\,$ & $\,\,$\color{gr} 5.0299\color{black} $\,\,$ & $\,\,$10.0598$\,\,$ \\
$\,\,$0.2049$\,\,$ & $\,\,$ 1 $\,\,$ & $\,\,$\color{gr} 1.0304\color{black} $\,\,$ & $\,\,$2.0609  $\,\,$ \\
$\,\,$\color{gr} 0.1988\color{black} $\,\,$ & $\,\,$\color{gr} 0.9704\color{black} $\,\,$ & $\,\,$ 1 $\,\,$ & $\,\,$\color{gr} \color{blue} 2\color{black}  $\,\,$ \\
$\,\,$0.0994$\,\,$ & $\,\,$0.4852$\,\,$ & $\,\,$\color{gr} \color{blue}  1/2\color{black} $\,\,$ & $\,\,$ 1  $\,\,$ \\
\end{pmatrix},
\end{equation*}
\end{example}
\newpage
\begin{example}
\begin{equation*}
\mathbf{A} =
\begin{pmatrix}
$\,\,$ 1 $\,\,$ & $\,\,$7$\,\,$ & $\,\,$5$\,\,$ & $\,\,$8 $\,\,$ \\
$\,\,$ 1/7$\,\,$ & $\,\,$ 1 $\,\,$ & $\,\,$3$\,\,$ & $\,\,$2 $\,\,$ \\
$\,\,$ 1/5$\,\,$ & $\,\,$ 1/3$\,\,$ & $\,\,$ 1 $\,\,$ & $\,\,$1 $\,\,$ \\
$\,\,$ 1/8$\,\,$ & $\,\,$ 1/2$\,\,$ & $\,\,$ 1 $\,\,$ & $\,\,$ 1  $\,\,$ \\
\end{pmatrix},
\qquad
\lambda_{\max} =
4.1782,
\qquad
CR = 0.0672
\end{equation*}

\begin{equation*}
\mathbf{w}^{EM} =
\begin{pmatrix}
0.678922\\
0.158961\\
0.083982\\
\color{red} 0.078136\color{black}
\end{pmatrix}\end{equation*}
\begin{equation*}
\left[ \frac{{w}^{EM}_i}{{w}^{EM}_j} \right] =
\begin{pmatrix}
$\,\,$ 1 $\,\,$ & $\,\,$4.2710$\,\,$ & $\,\,$8.0842$\,\,$ & $\,\,$\color{red} 8.6890\color{black} $\,\,$ \\
$\,\,$0.2341$\,\,$ & $\,\,$ 1 $\,\,$ & $\,\,$1.8928$\,\,$ & $\,\,$\color{red} 2.0344\color{black}   $\,\,$ \\
$\,\,$0.1237$\,\,$ & $\,\,$0.5283$\,\,$ & $\,\,$ 1 $\,\,$ & $\,\,$\color{red} 1.0748\color{black}  $\,\,$ \\
$\,\,$\color{red} 0.1151\color{black} $\,\,$ & $\,\,$\color{red} 0.4915\color{black} $\,\,$ & $\,\,$\color{red} 0.9304\color{black} $\,\,$ & $\,\,$ 1  $\,\,$ \\
\end{pmatrix},
\end{equation*}

\begin{equation*}
\mathbf{w}^{\prime} =
\begin{pmatrix}
0.678010\\
0.158748\\
0.083869\\
0.079374
\end{pmatrix} =
0.998657\cdot
\begin{pmatrix}
0.678922\\
0.158961\\
0.083982\\
\color{gr} 0.079481\color{black}
\end{pmatrix},
\end{equation*}
\begin{equation*}
\left[ \frac{{w}^{\prime}_i}{{w}^{\prime}_j} \right] =
\begin{pmatrix}
$\,\,$ 1 $\,\,$ & $\,\,$4.2710$\,\,$ & $\,\,$8.0842$\,\,$ & $\,\,$\color{gr} 8.5420\color{black} $\,\,$ \\
$\,\,$0.2341$\,\,$ & $\,\,$ 1 $\,\,$ & $\,\,$1.8928$\,\,$ & $\,\,$\color{gr} \color{blue} 2\color{black}   $\,\,$ \\
$\,\,$0.1237$\,\,$ & $\,\,$0.5283$\,\,$ & $\,\,$ 1 $\,\,$ & $\,\,$\color{gr} 1.0566\color{black}  $\,\,$ \\
$\,\,$\color{gr} 0.1171\color{black} $\,\,$ & $\,\,$\color{gr} \color{blue}  1/2\color{black} $\,\,$ & $\,\,$\color{gr} 0.9464\color{black} $\,\,$ & $\,\,$ 1  $\,\,$ \\
\end{pmatrix},
\end{equation*}
\end{example}
\newpage
\begin{example}
\begin{equation*}
\mathbf{A} =
\begin{pmatrix}
$\,\,$ 1 $\,\,$ & $\,\,$7$\,\,$ & $\,\,$5$\,\,$ & $\,\,$8 $\,\,$ \\
$\,\,$ 1/7$\,\,$ & $\,\,$ 1 $\,\,$ & $\,\,$4$\,\,$ & $\,\,$2 $\,\,$ \\
$\,\,$ 1/5$\,\,$ & $\,\,$ 1/4$\,\,$ & $\,\,$ 1 $\,\,$ & $\,\,$1 $\,\,$ \\
$\,\,$ 1/8$\,\,$ & $\,\,$ 1/2$\,\,$ & $\,\,$ 1 $\,\,$ & $\,\,$ 1  $\,\,$ \\
\end{pmatrix},
\qquad
\lambda_{\max} =
4.2610,
\qquad
CR = 0.0984
\end{equation*}

\begin{equation*}
\mathbf{w}^{EM} =
\begin{pmatrix}
0.674434\\
0.171673\\
0.077845\\
\color{red} 0.076047\color{black}
\end{pmatrix}\end{equation*}
\begin{equation*}
\left[ \frac{{w}^{EM}_i}{{w}^{EM}_j} \right] =
\begin{pmatrix}
$\,\,$ 1 $\,\,$ & $\,\,$3.9286$\,\,$ & $\,\,$8.6638$\,\,$ & $\,\,$\color{red} 8.8687\color{black} $\,\,$ \\
$\,\,$0.2545$\,\,$ & $\,\,$ 1 $\,\,$ & $\,\,$2.2053$\,\,$ & $\,\,$\color{red} 2.2575\color{black}   $\,\,$ \\
$\,\,$0.1154$\,\,$ & $\,\,$0.4535$\,\,$ & $\,\,$ 1 $\,\,$ & $\,\,$\color{red} 1.0237\color{black}  $\,\,$ \\
$\,\,$\color{red} 0.1128\color{black} $\,\,$ & $\,\,$\color{red} 0.4430\color{black} $\,\,$ & $\,\,$\color{red} 0.9769\color{black} $\,\,$ & $\,\,$ 1  $\,\,$ \\
\end{pmatrix},
\end{equation*}

\begin{equation*}
\mathbf{w}^{\prime} =
\begin{pmatrix}
0.673224\\
0.171365\\
0.077706\\
0.077706
\end{pmatrix} =
0.998205\cdot
\begin{pmatrix}
0.674434\\
0.171673\\
0.077845\\
\color{gr} 0.077845\color{black}
\end{pmatrix},
\end{equation*}
\begin{equation*}
\left[ \frac{{w}^{\prime}_i}{{w}^{\prime}_j} \right] =
\begin{pmatrix}
$\,\,$ 1 $\,\,$ & $\,\,$3.9286$\,\,$ & $\,\,$8.6638$\,\,$ & $\,\,$\color{gr} 8.6638\color{black} $\,\,$ \\
$\,\,$0.2545$\,\,$ & $\,\,$ 1 $\,\,$ & $\,\,$2.2053$\,\,$ & $\,\,$\color{gr} 2.2053\color{black}   $\,\,$ \\
$\,\,$0.1154$\,\,$ & $\,\,$0.4535$\,\,$ & $\,\,$ 1 $\,\,$ & $\,\,$\color{gr} \color{blue} 1\color{black}  $\,\,$ \\
$\,\,$\color{gr} 0.1154\color{black} $\,\,$ & $\,\,$\color{gr} 0.4535\color{black} $\,\,$ & $\,\,$\color{gr} \color{blue} 1\color{black} $\,\,$ & $\,\,$ 1  $\,\,$ \\
\end{pmatrix},
\end{equation*}
\end{example}
\newpage
\begin{example}
\begin{equation*}
\mathbf{A} =
\begin{pmatrix}
$\,\,$ 1 $\,\,$ & $\,\,$7$\,\,$ & $\,\,$5$\,\,$ & $\,\,$9 $\,\,$ \\
$\,\,$ 1/7$\,\,$ & $\,\,$ 1 $\,\,$ & $\,\,$1$\,\,$ & $\,\,$5 $\,\,$ \\
$\,\,$ 1/5$\,\,$ & $\,\,$ 1 $\,\,$ & $\,\,$ 1 $\,\,$ & $\,\,$3 $\,\,$ \\
$\,\,$ 1/9$\,\,$ & $\,\,$ 1/5$\,\,$ & $\,\,$ 1/3$\,\,$ & $\,\,$ 1  $\,\,$ \\
\end{pmatrix},
\qquad
\lambda_{\max} =
4.1596,
\qquad
CR = 0.0602
\end{equation*}

\begin{equation*}
\mathbf{w}^{EM} =
\begin{pmatrix}
0.671926\\
0.147205\\
\color{red} 0.133807\color{black} \\
0.047063
\end{pmatrix}\end{equation*}
\begin{equation*}
\left[ \frac{{w}^{EM}_i}{{w}^{EM}_j} \right] =
\begin{pmatrix}
$\,\,$ 1 $\,\,$ & $\,\,$4.5646$\,\,$ & $\,\,$\color{red} 5.0216\color{black} $\,\,$ & $\,\,$14.2771$\,\,$ \\
$\,\,$0.2191$\,\,$ & $\,\,$ 1 $\,\,$ & $\,\,$\color{red} 1.1001\color{black} $\,\,$ & $\,\,$3.1278  $\,\,$ \\
$\,\,$\color{red} 0.1991\color{black} $\,\,$ & $\,\,$\color{red} 0.9090\color{black} $\,\,$ & $\,\,$ 1 $\,\,$ & $\,\,$\color{red} 2.8431\color{black}  $\,\,$ \\
$\,\,$0.0700$\,\,$ & $\,\,$0.3197$\,\,$ & $\,\,$\color{red} 0.3517\color{black} $\,\,$ & $\,\,$ 1  $\,\,$ \\
\end{pmatrix},
\end{equation*}

\begin{equation*}
\mathbf{w}^{\prime} =
\begin{pmatrix}
0.671537\\
0.147119\\
0.134307\\
0.047036
\end{pmatrix} =
0.999422\cdot
\begin{pmatrix}
0.671926\\
0.147205\\
\color{gr} 0.134385\color{black} \\
0.047063
\end{pmatrix},
\end{equation*}
\begin{equation*}
\left[ \frac{{w}^{\prime}_i}{{w}^{\prime}_j} \right] =
\begin{pmatrix}
$\,\,$ 1 $\,\,$ & $\,\,$4.5646$\,\,$ & $\,\,$\color{gr} \color{blue} 5\color{black} $\,\,$ & $\,\,$14.2771$\,\,$ \\
$\,\,$0.2191$\,\,$ & $\,\,$ 1 $\,\,$ & $\,\,$\color{gr} 1.0954\color{black} $\,\,$ & $\,\,$3.1278  $\,\,$ \\
$\,\,$\color{gr} \color{blue}  1/5\color{black} $\,\,$ & $\,\,$\color{gr} 0.9129\color{black} $\,\,$ & $\,\,$ 1 $\,\,$ & $\,\,$\color{gr} 2.8554\color{black}  $\,\,$ \\
$\,\,$0.0700$\,\,$ & $\,\,$0.3197$\,\,$ & $\,\,$\color{gr} 0.3502\color{black} $\,\,$ & $\,\,$ 1  $\,\,$ \\
\end{pmatrix},
\end{equation*}
\end{example}
\newpage
\begin{example}
\begin{equation*}
\mathbf{A} =
\begin{pmatrix}
$\,\,$ 1 $\,\,$ & $\,\,$7$\,\,$ & $\,\,$5$\,\,$ & $\,\,$9 $\,\,$ \\
$\,\,$ 1/7$\,\,$ & $\,\,$ 1 $\,\,$ & $\,\,$1$\,\,$ & $\,\,$6 $\,\,$ \\
$\,\,$ 1/5$\,\,$ & $\,\,$ 1 $\,\,$ & $\,\,$ 1 $\,\,$ & $\,\,$3 $\,\,$ \\
$\,\,$ 1/9$\,\,$ & $\,\,$ 1/6$\,\,$ & $\,\,$ 1/3$\,\,$ & $\,\,$ 1  $\,\,$ \\
\end{pmatrix},
\qquad
\lambda_{\max} =
4.2095,
\qquad
CR = 0.0790
\end{equation*}

\begin{equation*}
\mathbf{w}^{EM} =
\begin{pmatrix}
0.668605\\
0.154713\\
\color{red} 0.131811\color{black} \\
0.044871
\end{pmatrix}\end{equation*}
\begin{equation*}
\left[ \frac{{w}^{EM}_i}{{w}^{EM}_j} \right] =
\begin{pmatrix}
$\,\,$ 1 $\,\,$ & $\,\,$4.3216$\,\,$ & $\,\,$\color{red} 5.0725\color{black} $\,\,$ & $\,\,$14.9007$\,\,$ \\
$\,\,$0.2314$\,\,$ & $\,\,$ 1 $\,\,$ & $\,\,$\color{red} 1.1737\color{black} $\,\,$ & $\,\,$3.4480  $\,\,$ \\
$\,\,$\color{red} 0.1971\color{black} $\,\,$ & $\,\,$\color{red} 0.8520\color{black} $\,\,$ & $\,\,$ 1 $\,\,$ & $\,\,$\color{red} 2.9376\color{black}  $\,\,$ \\
$\,\,$0.0671$\,\,$ & $\,\,$0.2900$\,\,$ & $\,\,$\color{red} 0.3404\color{black} $\,\,$ & $\,\,$ 1  $\,\,$ \\
\end{pmatrix},
\end{equation*}

\begin{equation*}
\mathbf{w}^{\prime} =
\begin{pmatrix}
0.667331\\
0.154418\\
0.133466\\
0.044785
\end{pmatrix} =
0.998094\cdot
\begin{pmatrix}
0.668605\\
0.154713\\
\color{gr} 0.133721\color{black} \\
0.044871
\end{pmatrix},
\end{equation*}
\begin{equation*}
\left[ \frac{{w}^{\prime}_i}{{w}^{\prime}_j} \right] =
\begin{pmatrix}
$\,\,$ 1 $\,\,$ & $\,\,$4.3216$\,\,$ & $\,\,$\color{gr} \color{blue} 5\color{black} $\,\,$ & $\,\,$14.9007$\,\,$ \\
$\,\,$0.2314$\,\,$ & $\,\,$ 1 $\,\,$ & $\,\,$\color{gr} 1.1570\color{black} $\,\,$ & $\,\,$3.4480  $\,\,$ \\
$\,\,$\color{gr} \color{blue}  1/5\color{black} $\,\,$ & $\,\,$\color{gr} 0.8643\color{black} $\,\,$ & $\,\,$ 1 $\,\,$ & $\,\,$\color{gr} 2.9801\color{black}  $\,\,$ \\
$\,\,$0.0671$\,\,$ & $\,\,$0.2900$\,\,$ & $\,\,$\color{gr} 0.3356\color{black} $\,\,$ & $\,\,$ 1  $\,\,$ \\
\end{pmatrix},
\end{equation*}
\end{example}
\newpage
\begin{example}
\begin{equation*}
\mathbf{A} =
\begin{pmatrix}
$\,\,$ 1 $\,\,$ & $\,\,$7$\,\,$ & $\,\,$5$\,\,$ & $\,\,$9 $\,\,$ \\
$\,\,$ 1/7$\,\,$ & $\,\,$ 1 $\,\,$ & $\,\,$3$\,\,$ & $\,\,$2 $\,\,$ \\
$\,\,$ 1/5$\,\,$ & $\,\,$ 1/3$\,\,$ & $\,\,$ 1 $\,\,$ & $\,\,$1 $\,\,$ \\
$\,\,$ 1/9$\,\,$ & $\,\,$ 1/2$\,\,$ & $\,\,$ 1 $\,\,$ & $\,\,$ 1  $\,\,$ \\
\end{pmatrix},
\qquad
\lambda_{\max} =
4.1786,
\qquad
CR = 0.0673
\end{equation*}

\begin{equation*}
\mathbf{w}^{EM} =
\begin{pmatrix}
0.686075\\
0.156202\\
0.083044\\
\color{red} 0.074680\color{black}
\end{pmatrix}\end{equation*}
\begin{equation*}
\left[ \frac{{w}^{EM}_i}{{w}^{EM}_j} \right] =
\begin{pmatrix}
$\,\,$ 1 $\,\,$ & $\,\,$4.3922$\,\,$ & $\,\,$8.2616$\,\,$ & $\,\,$\color{red} 9.1869\color{black} $\,\,$ \\
$\,\,$0.2277$\,\,$ & $\,\,$ 1 $\,\,$ & $\,\,$1.8810$\,\,$ & $\,\,$\color{red} 2.0916\color{black}   $\,\,$ \\
$\,\,$0.1210$\,\,$ & $\,\,$0.5316$\,\,$ & $\,\,$ 1 $\,\,$ & $\,\,$\color{red} 1.1120\color{black}  $\,\,$ \\
$\,\,$\color{red} 0.1089\color{black} $\,\,$ & $\,\,$\color{red} 0.4781\color{black} $\,\,$ & $\,\,$\color{red} 0.8993\color{black} $\,\,$ & $\,\,$ 1  $\,\,$ \\
\end{pmatrix},
\end{equation*}

\begin{equation*}
\mathbf{w}^{\prime} =
\begin{pmatrix}
0.685012\\
0.155960\\
0.082915\\
0.076112
\end{pmatrix} =
0.998451\cdot
\begin{pmatrix}
0.686075\\
0.156202\\
0.083044\\
\color{gr} 0.076231\color{black}
\end{pmatrix},
\end{equation*}
\begin{equation*}
\left[ \frac{{w}^{\prime}_i}{{w}^{\prime}_j} \right] =
\begin{pmatrix}
$\,\,$ 1 $\,\,$ & $\,\,$4.3922$\,\,$ & $\,\,$8.2616$\,\,$ & $\,\,$\color{gr} \color{blue} 9\color{black} $\,\,$ \\
$\,\,$0.2277$\,\,$ & $\,\,$ 1 $\,\,$ & $\,\,$1.8810$\,\,$ & $\,\,$\color{gr} 2.0491\color{black}   $\,\,$ \\
$\,\,$0.1210$\,\,$ & $\,\,$0.5316$\,\,$ & $\,\,$ 1 $\,\,$ & $\,\,$\color{gr} 1.0894\color{black}  $\,\,$ \\
$\,\,$\color{gr} \color{blue}  1/9\color{black} $\,\,$ & $\,\,$\color{gr} 0.4880\color{black} $\,\,$ & $\,\,$\color{gr} 0.9180\color{black} $\,\,$ & $\,\,$ 1  $\,\,$ \\
\end{pmatrix},
\end{equation*}
\end{example}
\newpage
\begin{example}
\begin{equation*}
\mathbf{A} =
\begin{pmatrix}
$\,\,$ 1 $\,\,$ & $\,\,$7$\,\,$ & $\,\,$5$\,\,$ & $\,\,$9 $\,\,$ \\
$\,\,$ 1/7$\,\,$ & $\,\,$ 1 $\,\,$ & $\,\,$4$\,\,$ & $\,\,$2 $\,\,$ \\
$\,\,$ 1/5$\,\,$ & $\,\,$ 1/4$\,\,$ & $\,\,$ 1 $\,\,$ & $\,\,$1 $\,\,$ \\
$\,\,$ 1/9$\,\,$ & $\,\,$ 1/2$\,\,$ & $\,\,$ 1 $\,\,$ & $\,\,$ 1  $\,\,$ \\
\end{pmatrix},
\qquad
\lambda_{\max} =
4.2614,
\qquad
CR = 0.0986
\end{equation*}

\begin{equation*}
\mathbf{w}^{EM} =
\begin{pmatrix}
0.681330\\
0.168915\\
0.077029\\
\color{red} 0.072726\color{black}
\end{pmatrix}\end{equation*}
\begin{equation*}
\left[ \frac{{w}^{EM}_i}{{w}^{EM}_j} \right] =
\begin{pmatrix}
$\,\,$ 1 $\,\,$ & $\,\,$4.0336$\,\,$ & $\,\,$8.8451$\,\,$ & $\,\,$\color{red} 9.3684\color{black} $\,\,$ \\
$\,\,$0.2479$\,\,$ & $\,\,$ 1 $\,\,$ & $\,\,$2.1929$\,\,$ & $\,\,$\color{red} 2.3226\color{black}   $\,\,$ \\
$\,\,$0.1131$\,\,$ & $\,\,$0.4560$\,\,$ & $\,\,$ 1 $\,\,$ & $\,\,$\color{red} 1.0592\color{black}  $\,\,$ \\
$\,\,$\color{red} 0.1067\color{black} $\,\,$ & $\,\,$\color{red} 0.4305\color{black} $\,\,$ & $\,\,$\color{red} 0.9441\color{black} $\,\,$ & $\,\,$ 1  $\,\,$ \\
\end{pmatrix},
\end{equation*}

\begin{equation*}
\mathbf{w}^{\prime} =
\begin{pmatrix}
0.679307\\
0.168414\\
0.076800\\
0.075479
\end{pmatrix} =
0.997032\cdot
\begin{pmatrix}
0.681330\\
0.168915\\
0.077029\\
\color{gr} 0.075703\color{black}
\end{pmatrix},
\end{equation*}
\begin{equation*}
\left[ \frac{{w}^{\prime}_i}{{w}^{\prime}_j} \right] =
\begin{pmatrix}
$\,\,$ 1 $\,\,$ & $\,\,$4.0336$\,\,$ & $\,\,$8.8451$\,\,$ & $\,\,$\color{gr} \color{blue} 9\color{black} $\,\,$ \\
$\,\,$0.2479$\,\,$ & $\,\,$ 1 $\,\,$ & $\,\,$2.1929$\,\,$ & $\,\,$\color{gr} 2.2313\color{black}   $\,\,$ \\
$\,\,$0.1131$\,\,$ & $\,\,$0.4560$\,\,$ & $\,\,$ 1 $\,\,$ & $\,\,$\color{gr} 1.0175\color{black}  $\,\,$ \\
$\,\,$\color{gr} \color{blue}  1/9\color{black} $\,\,$ & $\,\,$\color{gr} 0.4482\color{black} $\,\,$ & $\,\,$\color{gr} 0.9828\color{black} $\,\,$ & $\,\,$ 1  $\,\,$ \\
\end{pmatrix},
\end{equation*}
\end{example}
\newpage
\begin{example}
\begin{equation*}
\mathbf{A} =
\begin{pmatrix}
$\,\,$ 1 $\,\,$ & $\,\,$7$\,\,$ & $\,\,$6$\,\,$ & $\,\,$8 $\,\,$ \\
$\,\,$ 1/7$\,\,$ & $\,\,$ 1 $\,\,$ & $\,\,$3$\,\,$ & $\,\,$2 $\,\,$ \\
$\,\,$ 1/6$\,\,$ & $\,\,$ 1/3$\,\,$ & $\,\,$ 1 $\,\,$ & $\,\,$1 $\,\,$ \\
$\,\,$ 1/8$\,\,$ & $\,\,$ 1/2$\,\,$ & $\,\,$ 1 $\,\,$ & $\,\,$ 1  $\,\,$ \\
\end{pmatrix},
\qquad
\lambda_{\max} =
4.1365,
\qquad
CR = 0.0515
\end{equation*}

\begin{equation*}
\mathbf{w}^{EM} =
\begin{pmatrix}
0.690503\\
0.154838\\
0.077688\\
\color{red} 0.076971\color{black}
\end{pmatrix}\end{equation*}
\begin{equation*}
\left[ \frac{{w}^{EM}_i}{{w}^{EM}_j} \right] =
\begin{pmatrix}
$\,\,$ 1 $\,\,$ & $\,\,$4.4595$\,\,$ & $\,\,$8.8882$\,\,$ & $\,\,$\color{red} 8.9709\color{black} $\,\,$ \\
$\,\,$0.2242$\,\,$ & $\,\,$ 1 $\,\,$ & $\,\,$1.9931$\,\,$ & $\,\,$\color{red} 2.0116\color{black}   $\,\,$ \\
$\,\,$0.1125$\,\,$ & $\,\,$0.5017$\,\,$ & $\,\,$ 1 $\,\,$ & $\,\,$\color{red} 1.0093\color{black}  $\,\,$ \\
$\,\,$\color{red} 0.1115\color{black} $\,\,$ & $\,\,$\color{red} 0.4971\color{black} $\,\,$ & $\,\,$\color{red} 0.9908\color{black} $\,\,$ & $\,\,$ 1  $\,\,$ \\
\end{pmatrix},
\end{equation*}

\begin{equation*}
\mathbf{w}^{\prime} =
\begin{pmatrix}
0.690194\\
0.154769\\
0.077653\\
0.077384
\end{pmatrix} =
0.999552\cdot
\begin{pmatrix}
0.690503\\
0.154838\\
0.077688\\
\color{gr} 0.077419\color{black}
\end{pmatrix},
\end{equation*}
\begin{equation*}
\left[ \frac{{w}^{\prime}_i}{{w}^{\prime}_j} \right] =
\begin{pmatrix}
$\,\,$ 1 $\,\,$ & $\,\,$4.4595$\,\,$ & $\,\,$8.8882$\,\,$ & $\,\,$\color{gr} 8.9190\color{black} $\,\,$ \\
$\,\,$0.2242$\,\,$ & $\,\,$ 1 $\,\,$ & $\,\,$1.9931$\,\,$ & $\,\,$\color{gr} \color{blue} 2\color{black}   $\,\,$ \\
$\,\,$0.1125$\,\,$ & $\,\,$0.5017$\,\,$ & $\,\,$ 1 $\,\,$ & $\,\,$\color{gr} 1.0035\color{black}  $\,\,$ \\
$\,\,$\color{gr} 0.1121\color{black} $\,\,$ & $\,\,$\color{gr} \color{blue}  1/2\color{black} $\,\,$ & $\,\,$\color{gr} 0.9965\color{black} $\,\,$ & $\,\,$ 1  $\,\,$ \\
\end{pmatrix},
\end{equation*}
\end{example}
\newpage
\begin{example}
\begin{equation*}
\mathbf{A} =
\begin{pmatrix}
$\,\,$ 1 $\,\,$ & $\,\,$7$\,\,$ & $\,\,$6$\,\,$ & $\,\,$9 $\,\,$ \\
$\,\,$ 1/7$\,\,$ & $\,\,$ 1 $\,\,$ & $\,\,$3$\,\,$ & $\,\,$2 $\,\,$ \\
$\,\,$ 1/6$\,\,$ & $\,\,$ 1/3$\,\,$ & $\,\,$ 1 $\,\,$ & $\,\,$1 $\,\,$ \\
$\,\,$ 1/9$\,\,$ & $\,\,$ 1/2$\,\,$ & $\,\,$ 1 $\,\,$ & $\,\,$ 1  $\,\,$ \\
\end{pmatrix},
\qquad
\lambda_{\max} =
4.1342,
\qquad
CR = 0.0506
\end{equation*}

\begin{equation*}
\mathbf{w}^{EM} =
\begin{pmatrix}
0.697662\\
0.152130\\
0.076725\\
\color{red} 0.073483\color{black}
\end{pmatrix}\end{equation*}
\begin{equation*}
\left[ \frac{{w}^{EM}_i}{{w}^{EM}_j} \right] =
\begin{pmatrix}
$\,\,$ 1 $\,\,$ & $\,\,$4.5859$\,\,$ & $\,\,$9.0931$\,\,$ & $\,\,$\color{red} 9.4943\color{black} $\,\,$ \\
$\,\,$0.2181$\,\,$ & $\,\,$ 1 $\,\,$ & $\,\,$1.9828$\,\,$ & $\,\,$\color{red} 2.0703\color{black}   $\,\,$ \\
$\,\,$0.1100$\,\,$ & $\,\,$0.5043$\,\,$ & $\,\,$ 1 $\,\,$ & $\,\,$\color{red} 1.0441\color{black}  $\,\,$ \\
$\,\,$\color{red} 0.1053\color{black} $\,\,$ & $\,\,$\color{red} 0.4830\color{black} $\,\,$ & $\,\,$\color{red} 0.9577\color{black} $\,\,$ & $\,\,$ 1  $\,\,$ \\
\end{pmatrix},
\end{equation*}

\begin{equation*}
\mathbf{w}^{\prime} =
\begin{pmatrix}
0.695865\\
0.151739\\
0.076527\\
0.075869
\end{pmatrix} =
0.997424\cdot
\begin{pmatrix}
0.697662\\
0.152130\\
0.076725\\
\color{gr} 0.076065\color{black}
\end{pmatrix},
\end{equation*}
\begin{equation*}
\left[ \frac{{w}^{\prime}_i}{{w}^{\prime}_j} \right] =
\begin{pmatrix}
$\,\,$ 1 $\,\,$ & $\,\,$4.5859$\,\,$ & $\,\,$9.0931$\,\,$ & $\,\,$\color{gr} 9.1719\color{black} $\,\,$ \\
$\,\,$0.2181$\,\,$ & $\,\,$ 1 $\,\,$ & $\,\,$1.9828$\,\,$ & $\,\,$\color{gr} \color{blue} 2\color{black}   $\,\,$ \\
$\,\,$0.1100$\,\,$ & $\,\,$0.5043$\,\,$ & $\,\,$ 1 $\,\,$ & $\,\,$\color{gr} 1.0087\color{black}  $\,\,$ \\
$\,\,$\color{gr} 0.1090\color{black} $\,\,$ & $\,\,$\color{gr} \color{blue}  1/2\color{black} $\,\,$ & $\,\,$\color{gr} 0.9914\color{black} $\,\,$ & $\,\,$ 1  $\,\,$ \\
\end{pmatrix},
\end{equation*}
\end{example}
\newpage
\begin{example}
\begin{equation*}
\mathbf{A} =
\begin{pmatrix}
$\,\,$ 1 $\,\,$ & $\,\,$7$\,\,$ & $\,\,$7$\,\,$ & $\,\,$9 $\,\,$ \\
$\,\,$ 1/7$\,\,$ & $\,\,$ 1 $\,\,$ & $\,\,$2$\,\,$ & $\,\,$7 $\,\,$ \\
$\,\,$ 1/7$\,\,$ & $\,\,$ 1/2$\,\,$ & $\,\,$ 1 $\,\,$ & $\,\,$2 $\,\,$ \\
$\,\,$ 1/9$\,\,$ & $\,\,$ 1/7$\,\,$ & $\,\,$ 1/2$\,\,$ & $\,\,$ 1  $\,\,$ \\
\end{pmatrix},
\qquad
\lambda_{\max} =
4.2526,
\qquad
CR = 0.0952
\end{equation*}

\begin{equation*}
\mathbf{w}^{EM} =
\begin{pmatrix}
0.691450\\
0.178723\\
\color{red} 0.085252\color{black} \\
0.044575
\end{pmatrix}\end{equation*}
\begin{equation*}
\left[ \frac{{w}^{EM}_i}{{w}^{EM}_j} \right] =
\begin{pmatrix}
$\,\,$ 1 $\,\,$ & $\,\,$3.8688$\,\,$ & $\,\,$\color{red} 8.1106\color{black} $\,\,$ & $\,\,$15.5119$\,\,$ \\
$\,\,$0.2585$\,\,$ & $\,\,$ 1 $\,\,$ & $\,\,$\color{red} 2.0964\color{black} $\,\,$ & $\,\,$4.0094  $\,\,$ \\
$\,\,$\color{red} 0.1233\color{black} $\,\,$ & $\,\,$\color{red} 0.4770\color{black} $\,\,$ & $\,\,$ 1 $\,\,$ & $\,\,$\color{red} 1.9125\color{black}  $\,\,$ \\
$\,\,$0.0645$\,\,$ & $\,\,$0.2494$\,\,$ & $\,\,$\color{red} 0.5229\color{black} $\,\,$ & $\,\,$ 1  $\,\,$ \\
\end{pmatrix},
\end{equation*}

\begin{equation*}
\mathbf{w}^{\prime} =
\begin{pmatrix}
0.688765\\
0.178028\\
0.088805\\
0.044402
\end{pmatrix} =
0.996117\cdot
\begin{pmatrix}
0.691450\\
0.178723\\
\color{gr} 0.089151\color{black} \\
0.044575
\end{pmatrix},
\end{equation*}
\begin{equation*}
\left[ \frac{{w}^{\prime}_i}{{w}^{\prime}_j} \right] =
\begin{pmatrix}
$\,\,$ 1 $\,\,$ & $\,\,$3.8688$\,\,$ & $\,\,$\color{gr} 7.7559\color{black} $\,\,$ & $\,\,$15.5119$\,\,$ \\
$\,\,$0.2585$\,\,$ & $\,\,$ 1 $\,\,$ & $\,\,$\color{gr} 2.0047\color{black} $\,\,$ & $\,\,$4.0094  $\,\,$ \\
$\,\,$\color{gr} 0.1289\color{black} $\,\,$ & $\,\,$\color{gr} 0.4988\color{black} $\,\,$ & $\,\,$ 1 $\,\,$ & $\,\,$\color{gr} \color{blue} 2\color{black}  $\,\,$ \\
$\,\,$0.0645$\,\,$ & $\,\,$0.2494$\,\,$ & $\,\,$\color{gr} \color{blue}  1/2\color{black} $\,\,$ & $\,\,$ 1  $\,\,$ \\
\end{pmatrix},
\end{equation*}
\end{example}
\newpage
\begin{example}
\begin{equation*}
\mathbf{A} =
\begin{pmatrix}
$\,\,$ 1 $\,\,$ & $\,\,$7$\,\,$ & $\,\,$8$\,\,$ & $\,\,$8 $\,\,$ \\
$\,\,$ 1/7$\,\,$ & $\,\,$ 1 $\,\,$ & $\,\,$2$\,\,$ & $\,\,$6 $\,\,$ \\
$\,\,$ 1/8$\,\,$ & $\,\,$ 1/2$\,\,$ & $\,\,$ 1 $\,\,$ & $\,\,$2 $\,\,$ \\
$\,\,$ 1/8$\,\,$ & $\,\,$ 1/6$\,\,$ & $\,\,$ 1/2$\,\,$ & $\,\,$ 1  $\,\,$ \\
\end{pmatrix},
\qquad
\lambda_{\max} =
4.2421,
\qquad
CR = 0.0913
\end{equation*}

\begin{equation*}
\mathbf{w}^{EM} =
\begin{pmatrix}
0.696322\\
0.171865\\
\color{red} 0.083287\color{black} \\
0.048526
\end{pmatrix}\end{equation*}
\begin{equation*}
\left[ \frac{{w}^{EM}_i}{{w}^{EM}_j} \right] =
\begin{pmatrix}
$\,\,$ 1 $\,\,$ & $\,\,$4.0516$\,\,$ & $\,\,$\color{red} 8.3605\color{black} $\,\,$ & $\,\,$14.3494$\,\,$ \\
$\,\,$0.2468$\,\,$ & $\,\,$ 1 $\,\,$ & $\,\,$\color{red} 2.0635\color{black} $\,\,$ & $\,\,$3.5417  $\,\,$ \\
$\,\,$\color{red} 0.1196\color{black} $\,\,$ & $\,\,$\color{red} 0.4846\color{black} $\,\,$ & $\,\,$ 1 $\,\,$ & $\,\,$\color{red} 1.7163\color{black}  $\,\,$ \\
$\,\,$0.0697$\,\,$ & $\,\,$0.2824$\,\,$ & $\,\,$\color{red} 0.5826\color{black} $\,\,$ & $\,\,$ 1  $\,\,$ \\
\end{pmatrix},
\end{equation*}

\begin{equation*}
\mathbf{w}^{\prime} =
\begin{pmatrix}
0.694485\\
0.171411\\
0.085706\\
0.048398
\end{pmatrix} =
0.997361\cdot
\begin{pmatrix}
0.696322\\
0.171865\\
\color{gr} 0.085932\color{black} \\
0.048526
\end{pmatrix},
\end{equation*}
\begin{equation*}
\left[ \frac{{w}^{\prime}_i}{{w}^{\prime}_j} \right] =
\begin{pmatrix}
$\,\,$ 1 $\,\,$ & $\,\,$4.0516$\,\,$ & $\,\,$\color{gr} 8.1031\color{black} $\,\,$ & $\,\,$14.3494$\,\,$ \\
$\,\,$0.2468$\,\,$ & $\,\,$ 1 $\,\,$ & $\,\,$\color{gr} \color{blue} 2\color{black} $\,\,$ & $\,\,$3.5417  $\,\,$ \\
$\,\,$\color{gr} 0.1234\color{black} $\,\,$ & $\,\,$\color{gr} \color{blue}  1/2\color{black} $\,\,$ & $\,\,$ 1 $\,\,$ & $\,\,$\color{gr} 1.7708\color{black}  $\,\,$ \\
$\,\,$0.0697$\,\,$ & $\,\,$0.2824$\,\,$ & $\,\,$\color{gr} 0.5647\color{black} $\,\,$ & $\,\,$ 1  $\,\,$ \\
\end{pmatrix},
\end{equation*}
\end{example}
\newpage
\begin{example}
\begin{equation*}
\mathbf{A} =
\begin{pmatrix}
$\,\,$ 1 $\,\,$ & $\,\,$7$\,\,$ & $\,\,$8$\,\,$ & $\,\,$9 $\,\,$ \\
$\,\,$ 1/7$\,\,$ & $\,\,$ 1 $\,\,$ & $\,\,$2$\,\,$ & $\,\,$6 $\,\,$ \\
$\,\,$ 1/8$\,\,$ & $\,\,$ 1/2$\,\,$ & $\,\,$ 1 $\,\,$ & $\,\,$2 $\,\,$ \\
$\,\,$ 1/9$\,\,$ & $\,\,$ 1/6$\,\,$ & $\,\,$ 1/2$\,\,$ & $\,\,$ 1  $\,\,$ \\
\end{pmatrix},
\qquad
\lambda_{\max} =
4.2065,
\qquad
CR = 0.0779
\end{equation*}

\begin{equation*}
\mathbf{w}^{EM} =
\begin{pmatrix}
0.702915\\
0.168726\\
\color{red} 0.082386\color{black} \\
0.045973
\end{pmatrix}\end{equation*}
\begin{equation*}
\left[ \frac{{w}^{EM}_i}{{w}^{EM}_j} \right] =
\begin{pmatrix}
$\,\,$ 1 $\,\,$ & $\,\,$4.1660$\,\,$ & $\,\,$\color{red} 8.5320\color{black} $\,\,$ & $\,\,$15.2896$\,\,$ \\
$\,\,$0.2400$\,\,$ & $\,\,$ 1 $\,\,$ & $\,\,$\color{red} 2.0480\color{black} $\,\,$ & $\,\,$3.6701  $\,\,$ \\
$\,\,$\color{red} 0.1172\color{black} $\,\,$ & $\,\,$\color{red} 0.4883\color{black} $\,\,$ & $\,\,$ 1 $\,\,$ & $\,\,$\color{red} 1.7920\color{black}  $\,\,$ \\
$\,\,$0.0654$\,\,$ & $\,\,$0.2725$\,\,$ & $\,\,$\color{red} 0.5580\color{black} $\,\,$ & $\,\,$ 1  $\,\,$ \\
\end{pmatrix},
\end{equation*}

\begin{equation*}
\mathbf{w}^{\prime} =
\begin{pmatrix}
0.701528\\
0.168393\\
0.084197\\
0.045883
\end{pmatrix} =
0.998027\cdot
\begin{pmatrix}
0.702915\\
0.168726\\
\color{gr} 0.084363\color{black} \\
0.045973
\end{pmatrix},
\end{equation*}
\begin{equation*}
\left[ \frac{{w}^{\prime}_i}{{w}^{\prime}_j} \right] =
\begin{pmatrix}
$\,\,$ 1 $\,\,$ & $\,\,$4.1660$\,\,$ & $\,\,$\color{gr} 8.3320\color{black} $\,\,$ & $\,\,$15.2896$\,\,$ \\
$\,\,$0.2400$\,\,$ & $\,\,$ 1 $\,\,$ & $\,\,$\color{gr} \color{blue} 2\color{black} $\,\,$ & $\,\,$3.6701  $\,\,$ \\
$\,\,$\color{gr} 0.1200\color{black} $\,\,$ & $\,\,$\color{gr} \color{blue}  1/2\color{black} $\,\,$ & $\,\,$ 1 $\,\,$ & $\,\,$\color{gr} 1.8350\color{black}  $\,\,$ \\
$\,\,$0.0654$\,\,$ & $\,\,$0.2725$\,\,$ & $\,\,$\color{gr} 0.5449\color{black} $\,\,$ & $\,\,$ 1  $\,\,$ \\
\end{pmatrix},
\end{equation*}
\end{example}
\newpage
\begin{example}
\begin{equation*}
\mathbf{A} =
\begin{pmatrix}
$\,\,$ 1 $\,\,$ & $\,\,$7$\,\,$ & $\,\,$8$\,\,$ & $\,\,$9 $\,\,$ \\
$\,\,$ 1/7$\,\,$ & $\,\,$ 1 $\,\,$ & $\,\,$2$\,\,$ & $\,\,$7 $\,\,$ \\
$\,\,$ 1/8$\,\,$ & $\,\,$ 1/2$\,\,$ & $\,\,$ 1 $\,\,$ & $\,\,$2 $\,\,$ \\
$\,\,$ 1/9$\,\,$ & $\,\,$ 1/7$\,\,$ & $\,\,$ 1/2$\,\,$ & $\,\,$ 1  $\,\,$ \\
\end{pmatrix},
\qquad
\lambda_{\max} =
4.2506,
\qquad
CR = 0.0945
\end{equation*}

\begin{equation*}
\mathbf{w}^{EM} =
\begin{pmatrix}
0.699393\\
0.175511\\
\color{red} 0.081014\color{black} \\
0.044082
\end{pmatrix}\end{equation*}
\begin{equation*}
\left[ \frac{{w}^{EM}_i}{{w}^{EM}_j} \right] =
\begin{pmatrix}
$\,\,$ 1 $\,\,$ & $\,\,$3.9849$\,\,$ & $\,\,$\color{red} 8.6330\color{black} $\,\,$ & $\,\,$15.8659$\,\,$ \\
$\,\,$0.2509$\,\,$ & $\,\,$ 1 $\,\,$ & $\,\,$\color{red} 2.1664\color{black} $\,\,$ & $\,\,$3.9815  $\,\,$ \\
$\,\,$\color{red} 0.1158\color{black} $\,\,$ & $\,\,$\color{red} 0.4616\color{black} $\,\,$ & $\,\,$ 1 $\,\,$ & $\,\,$\color{red} 1.8378\color{black}  $\,\,$ \\
$\,\,$0.0630$\,\,$ & $\,\,$0.2512$\,\,$ & $\,\,$\color{red} 0.5441\color{black} $\,\,$ & $\,\,$ 1  $\,\,$ \\
\end{pmatrix},
\end{equation*}

\begin{equation*}
\mathbf{w}^{\prime} =
\begin{pmatrix}
0.694938\\
0.174394\\
0.086867\\
0.043801
\end{pmatrix} =
0.993631\cdot
\begin{pmatrix}
0.699393\\
0.175511\\
\color{gr} 0.087424\color{black} \\
0.044082
\end{pmatrix},
\end{equation*}
\begin{equation*}
\left[ \frac{{w}^{\prime}_i}{{w}^{\prime}_j} \right] =
\begin{pmatrix}
$\,\,$ 1 $\,\,$ & $\,\,$3.9849$\,\,$ & $\,\,$\color{gr} \color{blue} 8\color{black} $\,\,$ & $\,\,$15.8659$\,\,$ \\
$\,\,$0.2509$\,\,$ & $\,\,$ 1 $\,\,$ & $\,\,$\color{gr} 2.0076\color{black} $\,\,$ & $\,\,$3.9815  $\,\,$ \\
$\,\,$\color{gr} \color{blue}  1/8\color{black} $\,\,$ & $\,\,$\color{gr} 0.4981\color{black} $\,\,$ & $\,\,$ 1 $\,\,$ & $\,\,$\color{gr} 1.9832\color{black}  $\,\,$ \\
$\,\,$0.0630$\,\,$ & $\,\,$0.2512$\,\,$ & $\,\,$\color{gr} 0.5042\color{black} $\,\,$ & $\,\,$ 1  $\,\,$ \\
\end{pmatrix},
\end{equation*}
\end{example}
\newpage
\begin{example}
\begin{equation*}
\mathbf{A} =
\begin{pmatrix}
$\,\,$ 1 $\,\,$ & $\,\,$7$\,\,$ & $\,\,$9$\,\,$ & $\,\,$9 $\,\,$ \\
$\,\,$ 1/7$\,\,$ & $\,\,$ 1 $\,\,$ & $\,\,$2$\,\,$ & $\,\,$6 $\,\,$ \\
$\,\,$ 1/9$\,\,$ & $\,\,$ 1/2$\,\,$ & $\,\,$ 1 $\,\,$ & $\,\,$2 $\,\,$ \\
$\,\,$ 1/9$\,\,$ & $\,\,$ 1/6$\,\,$ & $\,\,$ 1/2$\,\,$ & $\,\,$ 1  $\,\,$ \\
\end{pmatrix},
\qquad
\lambda_{\max} =
4.2086,
\qquad
CR = 0.0786
\end{equation*}

\begin{equation*}
\mathbf{w}^{EM} =
\begin{pmatrix}
0.710037\\
0.165735\\
\color{red} 0.078758\color{black} \\
0.045470
\end{pmatrix}\end{equation*}
\begin{equation*}
\left[ \frac{{w}^{EM}_i}{{w}^{EM}_j} \right] =
\begin{pmatrix}
$\,\,$ 1 $\,\,$ & $\,\,$4.2842$\,\,$ & $\,\,$\color{red} 9.0154\color{black} $\,\,$ & $\,\,$15.6154$\,\,$ \\
$\,\,$0.2334$\,\,$ & $\,\,$ 1 $\,\,$ & $\,\,$\color{red} 2.1044\color{black} $\,\,$ & $\,\,$3.6449  $\,\,$ \\
$\,\,$\color{red} 0.1109\color{black} $\,\,$ & $\,\,$\color{red} 0.4752\color{black} $\,\,$ & $\,\,$ 1 $\,\,$ & $\,\,$\color{red} 1.7321\color{black}  $\,\,$ \\
$\,\,$0.0640$\,\,$ & $\,\,$0.2744$\,\,$ & $\,\,$\color{red} 0.5773\color{black} $\,\,$ & $\,\,$ 1  $\,\,$ \\
\end{pmatrix},
\end{equation*}

\begin{equation*}
\mathbf{w}^{\prime} =
\begin{pmatrix}
0.709941\\
0.165713\\
0.078882\\
0.045464
\end{pmatrix} =
0.999865\cdot
\begin{pmatrix}
0.710037\\
0.165735\\
\color{gr} 0.078893\color{black} \\
0.045470
\end{pmatrix},
\end{equation*}
\begin{equation*}
\left[ \frac{{w}^{\prime}_i}{{w}^{\prime}_j} \right] =
\begin{pmatrix}
$\,\,$ 1 $\,\,$ & $\,\,$4.2842$\,\,$ & $\,\,$\color{gr} \color{blue} 9\color{black} $\,\,$ & $\,\,$15.6154$\,\,$ \\
$\,\,$0.2334$\,\,$ & $\,\,$ 1 $\,\,$ & $\,\,$\color{gr} 2.1008\color{black} $\,\,$ & $\,\,$3.6449  $\,\,$ \\
$\,\,$\color{gr} \color{blue}  1/9\color{black} $\,\,$ & $\,\,$\color{gr} 0.4760\color{black} $\,\,$ & $\,\,$ 1 $\,\,$ & $\,\,$\color{gr} 1.7350\color{black}  $\,\,$ \\
$\,\,$0.0640$\,\,$ & $\,\,$0.2744$\,\,$ & $\,\,$\color{gr} 0.5764\color{black} $\,\,$ & $\,\,$ 1  $\,\,$ \\
\end{pmatrix},
\end{equation*}
\end{example}
\newpage
\begin{example}
\begin{equation*}
\mathbf{A} =
\begin{pmatrix}
$\,\,$ 1 $\,\,$ & $\,\,$7$\,\,$ & $\,\,$9$\,\,$ & $\,\,$9 $\,\,$ \\
$\,\,$ 1/7$\,\,$ & $\,\,$ 1 $\,\,$ & $\,\,$2$\,\,$ & $\,\,$7 $\,\,$ \\
$\,\,$ 1/9$\,\,$ & $\,\,$ 1/2$\,\,$ & $\,\,$ 1 $\,\,$ & $\,\,$2 $\,\,$ \\
$\,\,$ 1/9$\,\,$ & $\,\,$ 1/7$\,\,$ & $\,\,$ 1/2$\,\,$ & $\,\,$ 1  $\,\,$ \\
\end{pmatrix},
\qquad
\lambda_{\max} =
4.2526,
\qquad
CR = 0.0952
\end{equation*}

\begin{equation*}
\mathbf{w}^{EM} =
\begin{pmatrix}
0.706376\\
0.172533\\
\color{red} 0.077473\color{black} \\
0.043618
\end{pmatrix}\end{equation*}
\begin{equation*}
\left[ \frac{{w}^{EM}_i}{{w}^{EM}_j} \right] =
\begin{pmatrix}
$\,\,$ 1 $\,\,$ & $\,\,$4.0941$\,\,$ & $\,\,$\color{red} 9.1177\color{black} $\,\,$ & $\,\,$16.1947$\,\,$ \\
$\,\,$0.2443$\,\,$ & $\,\,$ 1 $\,\,$ & $\,\,$\color{red} 2.2270\color{black} $\,\,$ & $\,\,$3.9556  $\,\,$ \\
$\,\,$\color{red} 0.1097\color{black} $\,\,$ & $\,\,$\color{red} 0.4490\color{black} $\,\,$ & $\,\,$ 1 $\,\,$ & $\,\,$\color{red} 1.7762\color{black}  $\,\,$ \\
$\,\,$0.0617$\,\,$ & $\,\,$0.2528$\,\,$ & $\,\,$\color{red} 0.5630\color{black} $\,\,$ & $\,\,$ 1  $\,\,$ \\
\end{pmatrix},
\end{equation*}

\begin{equation*}
\mathbf{w}^{\prime} =
\begin{pmatrix}
0.705661\\
0.172359\\
0.078407\\
0.043574
\end{pmatrix} =
0.998988\cdot
\begin{pmatrix}
0.706376\\
0.172533\\
\color{gr} 0.078486\color{black} \\
0.043618
\end{pmatrix},
\end{equation*}
\begin{equation*}
\left[ \frac{{w}^{\prime}_i}{{w}^{\prime}_j} \right] =
\begin{pmatrix}
$\,\,$ 1 $\,\,$ & $\,\,$4.0941$\,\,$ & $\,\,$\color{gr} \color{blue} 9\color{black} $\,\,$ & $\,\,$16.1947$\,\,$ \\
$\,\,$0.2443$\,\,$ & $\,\,$ 1 $\,\,$ & $\,\,$\color{gr} 2.1983\color{black} $\,\,$ & $\,\,$3.9556  $\,\,$ \\
$\,\,$\color{gr} \color{blue}  1/9\color{black} $\,\,$ & $\,\,$\color{gr} 0.4549\color{black} $\,\,$ & $\,\,$ 1 $\,\,$ & $\,\,$\color{gr} 1.7994\color{black}  $\,\,$ \\
$\,\,$0.0617$\,\,$ & $\,\,$0.2528$\,\,$ & $\,\,$\color{gr} 0.5557\color{black} $\,\,$ & $\,\,$ 1  $\,\,$ \\
\end{pmatrix},
\end{equation*}
\end{example}
\newpage
\begin{example}
\begin{equation*}
\mathbf{A} =
\begin{pmatrix}
$\,\,$ 1 $\,\,$ & $\,\,$8$\,\,$ & $\,\,$3$\,\,$ & $\,\,$9 $\,\,$ \\
$\,\,$ 1/8$\,\,$ & $\,\,$ 1 $\,\,$ & $\,\,$2$\,\,$ & $\,\,$2 $\,\,$ \\
$\,\,$ 1/3$\,\,$ & $\,\,$ 1/2$\,\,$ & $\,\,$ 1 $\,\,$ & $\,\,$2 $\,\,$ \\
$\,\,$ 1/9$\,\,$ & $\,\,$ 1/2$\,\,$ & $\,\,$ 1/2$\,\,$ & $\,\,$ 1  $\,\,$ \\
\end{pmatrix},
\qquad
\lambda_{\max} =
4.2469,
\qquad
CR = 0.0931
\end{equation*}

\begin{equation*}
\mathbf{w}^{EM} =
\begin{pmatrix}
0.659227\\
0.145600\\
0.130151\\
\color{red} 0.065023\color{black}
\end{pmatrix}\end{equation*}
\begin{equation*}
\left[ \frac{{w}^{EM}_i}{{w}^{EM}_j} \right] =
\begin{pmatrix}
$\,\,$ 1 $\,\,$ & $\,\,$4.5277$\,\,$ & $\,\,$5.0651$\,\,$ & $\,\,$\color{red} 10.1384\color{black} $\,\,$ \\
$\,\,$0.2209$\,\,$ & $\,\,$ 1 $\,\,$ & $\,\,$1.1187$\,\,$ & $\,\,$\color{red} 2.2392\color{black}   $\,\,$ \\
$\,\,$0.1974$\,\,$ & $\,\,$0.8939$\,\,$ & $\,\,$ 1 $\,\,$ & $\,\,$\color{red} 2.0016\color{black}  $\,\,$ \\
$\,\,$\color{red} 0.0986\color{black} $\,\,$ & $\,\,$\color{red} 0.4466\color{black} $\,\,$ & $\,\,$\color{red} 0.4996\color{black} $\,\,$ & $\,\,$ 1  $\,\,$ \\
\end{pmatrix},
\end{equation*}

\begin{equation*}
\mathbf{w}^{\prime} =
\begin{pmatrix}
0.659192\\
0.145592\\
0.130144\\
0.065072
\end{pmatrix} =
0.999947\cdot
\begin{pmatrix}
0.659227\\
0.145600\\
0.130151\\
\color{gr} 0.065075\color{black}
\end{pmatrix},
\end{equation*}
\begin{equation*}
\left[ \frac{{w}^{\prime}_i}{{w}^{\prime}_j} \right] =
\begin{pmatrix}
$\,\,$ 1 $\,\,$ & $\,\,$4.5277$\,\,$ & $\,\,$5.0651$\,\,$ & $\,\,$\color{gr} 10.1302\color{black} $\,\,$ \\
$\,\,$0.2209$\,\,$ & $\,\,$ 1 $\,\,$ & $\,\,$1.1187$\,\,$ & $\,\,$\color{gr} 2.2374\color{black}   $\,\,$ \\
$\,\,$0.1974$\,\,$ & $\,\,$0.8939$\,\,$ & $\,\,$ 1 $\,\,$ & $\,\,$\color{gr} \color{blue} 2\color{black}  $\,\,$ \\
$\,\,$\color{gr} 0.0987\color{black} $\,\,$ & $\,\,$\color{gr} 0.4469\color{black} $\,\,$ & $\,\,$\color{gr} \color{blue}  1/2\color{black} $\,\,$ & $\,\,$ 1  $\,\,$ \\
\end{pmatrix},
\end{equation*}
\end{example}
\newpage
\begin{example}
\begin{equation*}
\mathbf{A} =
\begin{pmatrix}
$\,\,$ 1 $\,\,$ & $\,\,$8$\,\,$ & $\,\,$4$\,\,$ & $\,\,$8 $\,\,$ \\
$\,\,$ 1/8$\,\,$ & $\,\,$ 1 $\,\,$ & $\,\,$1$\,\,$ & $\,\,$5 $\,\,$ \\
$\,\,$ 1/4$\,\,$ & $\,\,$ 1 $\,\,$ & $\,\,$ 1 $\,\,$ & $\,\,$3 $\,\,$ \\
$\,\,$ 1/8$\,\,$ & $\,\,$ 1/5$\,\,$ & $\,\,$ 1/3$\,\,$ & $\,\,$ 1  $\,\,$ \\
\end{pmatrix},
\qquad
\lambda_{\max} =
4.2277,
\qquad
CR = 0.0859
\end{equation*}

\begin{equation*}
\mathbf{w}^{EM} =
\begin{pmatrix}
0.661750\\
0.146324\\
\color{red} 0.142514\color{black} \\
0.049412
\end{pmatrix}\end{equation*}
\begin{equation*}
\left[ \frac{{w}^{EM}_i}{{w}^{EM}_j} \right] =
\begin{pmatrix}
$\,\,$ 1 $\,\,$ & $\,\,$4.5225$\,\,$ & $\,\,$\color{red} 4.6434\color{black} $\,\,$ & $\,\,$13.3925$\,\,$ \\
$\,\,$0.2211$\,\,$ & $\,\,$ 1 $\,\,$ & $\,\,$\color{red} 1.0267\color{black} $\,\,$ & $\,\,$2.9613  $\,\,$ \\
$\,\,$\color{red} 0.2154\color{black} $\,\,$ & $\,\,$\color{red} 0.9740\color{black} $\,\,$ & $\,\,$ 1 $\,\,$ & $\,\,$\color{red} 2.8842\color{black}  $\,\,$ \\
$\,\,$0.0747$\,\,$ & $\,\,$0.3377$\,\,$ & $\,\,$\color{red} 0.3467\color{black} $\,\,$ & $\,\,$ 1  $\,\,$ \\
\end{pmatrix},
\end{equation*}

\begin{equation*}
\mathbf{w}^{\prime} =
\begin{pmatrix}
0.659238\\
0.145769\\
0.145769\\
0.049224
\end{pmatrix} =
0.996205\cdot
\begin{pmatrix}
0.661750\\
0.146324\\
\color{gr} 0.146324\color{black} \\
0.049412
\end{pmatrix},
\end{equation*}
\begin{equation*}
\left[ \frac{{w}^{\prime}_i}{{w}^{\prime}_j} \right] =
\begin{pmatrix}
$\,\,$ 1 $\,\,$ & $\,\,$4.5225$\,\,$ & $\,\,$\color{gr} 4.5225\color{black} $\,\,$ & $\,\,$13.3925$\,\,$ \\
$\,\,$0.2211$\,\,$ & $\,\,$ 1 $\,\,$ & $\,\,$\color{gr} \color{blue} 1\color{black} $\,\,$ & $\,\,$2.9613  $\,\,$ \\
$\,\,$\color{gr} 0.2211\color{black} $\,\,$ & $\,\,$\color{gr} \color{blue} 1\color{black} $\,\,$ & $\,\,$ 1 $\,\,$ & $\,\,$\color{gr} 2.9613\color{black}  $\,\,$ \\
$\,\,$0.0747$\,\,$ & $\,\,$0.3377$\,\,$ & $\,\,$\color{gr} 0.3377\color{black} $\,\,$ & $\,\,$ 1  $\,\,$ \\
\end{pmatrix},
\end{equation*}
\end{example}
\newpage
\begin{example}
\begin{equation*}
\mathbf{A} =
\begin{pmatrix}
$\,\,$ 1 $\,\,$ & $\,\,$8$\,\,$ & $\,\,$4$\,\,$ & $\,\,$9 $\,\,$ \\
$\,\,$ 1/8$\,\,$ & $\,\,$ 1 $\,\,$ & $\,\,$1$\,\,$ & $\,\,$6 $\,\,$ \\
$\,\,$ 1/4$\,\,$ & $\,\,$ 1 $\,\,$ & $\,\,$ 1 $\,\,$ & $\,\,$4 $\,\,$ \\
$\,\,$ 1/9$\,\,$ & $\,\,$ 1/6$\,\,$ & $\,\,$ 1/4$\,\,$ & $\,\,$ 1  $\,\,$ \\
\end{pmatrix},
\qquad
\lambda_{\max} =
4.2469,
\qquad
CR = 0.0931
\end{equation*}

\begin{equation*}
\mathbf{w}^{EM} =
\begin{pmatrix}
0.662405\\
0.148030\\
\color{red} 0.147910\color{black} \\
0.041655
\end{pmatrix}\end{equation*}
\begin{equation*}
\left[ \frac{{w}^{EM}_i}{{w}^{EM}_j} \right] =
\begin{pmatrix}
$\,\,$ 1 $\,\,$ & $\,\,$4.4748$\,\,$ & $\,\,$\color{red} 4.4784\color{black} $\,\,$ & $\,\,$15.9022$\,\,$ \\
$\,\,$0.2235$\,\,$ & $\,\,$ 1 $\,\,$ & $\,\,$\color{red} 1.0008\color{black} $\,\,$ & $\,\,$3.5537  $\,\,$ \\
$\,\,$\color{red} 0.2233\color{black} $\,\,$ & $\,\,$\color{red} 0.9992\color{black} $\,\,$ & $\,\,$ 1 $\,\,$ & $\,\,$\color{red} 3.5508\color{black}  $\,\,$ \\
$\,\,$0.0629$\,\,$ & $\,\,$0.2814$\,\,$ & $\,\,$\color{red} 0.2816\color{black} $\,\,$ & $\,\,$ 1  $\,\,$ \\
\end{pmatrix},
\end{equation*}

\begin{equation*}
\mathbf{w}^{\prime} =
\begin{pmatrix}
0.662326\\
0.148012\\
0.148012\\
0.041650
\end{pmatrix} =
0.999880\cdot
\begin{pmatrix}
0.662405\\
0.148030\\
\color{gr} 0.148030\color{black} \\
0.041655
\end{pmatrix},
\end{equation*}
\begin{equation*}
\left[ \frac{{w}^{\prime}_i}{{w}^{\prime}_j} \right] =
\begin{pmatrix}
$\,\,$ 1 $\,\,$ & $\,\,$4.4748$\,\,$ & $\,\,$\color{gr} 4.4748\color{black} $\,\,$ & $\,\,$15.9022$\,\,$ \\
$\,\,$0.2235$\,\,$ & $\,\,$ 1 $\,\,$ & $\,\,$\color{gr} \color{blue} 1\color{black} $\,\,$ & $\,\,$3.5537  $\,\,$ \\
$\,\,$\color{gr} 0.2235\color{black} $\,\,$ & $\,\,$\color{gr} \color{blue} 1\color{black} $\,\,$ & $\,\,$ 1 $\,\,$ & $\,\,$\color{gr} 3.5537\color{black}  $\,\,$ \\
$\,\,$0.0629$\,\,$ & $\,\,$0.2814$\,\,$ & $\,\,$\color{gr} 0.2814\color{black} $\,\,$ & $\,\,$ 1  $\,\,$ \\
\end{pmatrix},
\end{equation*}
\end{example}
\newpage
\begin{example}
\begin{equation*}
\mathbf{A} =
\begin{pmatrix}
$\,\,$ 1 $\,\,$ & $\,\,$8$\,\,$ & $\,\,$5$\,\,$ & $\,\,$5 $\,\,$ \\
$\,\,$ 1/8$\,\,$ & $\,\,$ 1 $\,\,$ & $\,\,$1$\,\,$ & $\,\,$3 $\,\,$ \\
$\,\,$ 1/5$\,\,$ & $\,\,$ 1 $\,\,$ & $\,\,$ 1 $\,\,$ & $\,\,$2 $\,\,$ \\
$\,\,$ 1/5$\,\,$ & $\,\,$ 1/3$\,\,$ & $\,\,$ 1/2$\,\,$ & $\,\,$ 1  $\,\,$ \\
\end{pmatrix},
\qquad
\lambda_{\max} =
4.2162,
\qquad
CR = 0.0815
\end{equation*}

\begin{equation*}
\mathbf{w}^{EM} =
\begin{pmatrix}
0.658310\\
0.136273\\
\color{red} 0.130127\color{black} \\
0.075290
\end{pmatrix}\end{equation*}
\begin{equation*}
\left[ \frac{{w}^{EM}_i}{{w}^{EM}_j} \right] =
\begin{pmatrix}
$\,\,$ 1 $\,\,$ & $\,\,$4.8308$\,\,$ & $\,\,$\color{red} 5.0590\color{black} $\,\,$ & $\,\,$8.7436$\,\,$ \\
$\,\,$0.2070$\,\,$ & $\,\,$ 1 $\,\,$ & $\,\,$\color{red} 1.0472\color{black} $\,\,$ & $\,\,$1.8100  $\,\,$ \\
$\,\,$\color{red} 0.1977\color{black} $\,\,$ & $\,\,$\color{red} 0.9549\color{black} $\,\,$ & $\,\,$ 1 $\,\,$ & $\,\,$\color{red} 1.7283\color{black}  $\,\,$ \\
$\,\,$0.1144$\,\,$ & $\,\,$0.5525$\,\,$ & $\,\,$\color{red} 0.5786\color{black} $\,\,$ & $\,\,$ 1  $\,\,$ \\
\end{pmatrix},
\end{equation*}

\begin{equation*}
\mathbf{w}^{\prime} =
\begin{pmatrix}
0.657301\\
0.136065\\
0.131460\\
0.075175
\end{pmatrix} =
0.998467\cdot
\begin{pmatrix}
0.658310\\
0.136273\\
\color{gr} 0.131662\color{black} \\
0.075290
\end{pmatrix},
\end{equation*}
\begin{equation*}
\left[ \frac{{w}^{\prime}_i}{{w}^{\prime}_j} \right] =
\begin{pmatrix}
$\,\,$ 1 $\,\,$ & $\,\,$4.8308$\,\,$ & $\,\,$\color{gr} \color{blue} 5\color{black} $\,\,$ & $\,\,$8.7436$\,\,$ \\
$\,\,$0.2070$\,\,$ & $\,\,$ 1 $\,\,$ & $\,\,$\color{gr} 1.0350\color{black} $\,\,$ & $\,\,$1.8100  $\,\,$ \\
$\,\,$\color{gr} \color{blue}  1/5\color{black} $\,\,$ & $\,\,$\color{gr} 0.9662\color{black} $\,\,$ & $\,\,$ 1 $\,\,$ & $\,\,$\color{gr} 1.7487\color{black}  $\,\,$ \\
$\,\,$0.1144$\,\,$ & $\,\,$0.5525$\,\,$ & $\,\,$\color{gr} 0.5718\color{black} $\,\,$ & $\,\,$ 1  $\,\,$ \\
\end{pmatrix},
\end{equation*}
\end{example}
\newpage
\begin{example}
\begin{equation*}
\mathbf{A} =
\begin{pmatrix}
$\,\,$ 1 $\,\,$ & $\,\,$8$\,\,$ & $\,\,$5$\,\,$ & $\,\,$6 $\,\,$ \\
$\,\,$ 1/8$\,\,$ & $\,\,$ 1 $\,\,$ & $\,\,$1$\,\,$ & $\,\,$3 $\,\,$ \\
$\,\,$ 1/5$\,\,$ & $\,\,$ 1 $\,\,$ & $\,\,$ 1 $\,\,$ & $\,\,$2 $\,\,$ \\
$\,\,$ 1/6$\,\,$ & $\,\,$ 1/3$\,\,$ & $\,\,$ 1/2$\,\,$ & $\,\,$ 1  $\,\,$ \\
\end{pmatrix},
\qquad
\lambda_{\max} =
4.1655,
\qquad
CR = 0.0624
\end{equation*}

\begin{equation*}
\mathbf{w}^{EM} =
\begin{pmatrix}
0.669616\\
0.132765\\
\color{red} 0.128143\color{black} \\
0.069476
\end{pmatrix}\end{equation*}
\begin{equation*}
\left[ \frac{{w}^{EM}_i}{{w}^{EM}_j} \right] =
\begin{pmatrix}
$\,\,$ 1 $\,\,$ & $\,\,$5.0436$\,\,$ & $\,\,$\color{red} 5.2255\color{black} $\,\,$ & $\,\,$9.6381$\,\,$ \\
$\,\,$0.1983$\,\,$ & $\,\,$ 1 $\,\,$ & $\,\,$\color{red} 1.0361\color{black} $\,\,$ & $\,\,$1.9109  $\,\,$ \\
$\,\,$\color{red} 0.1914\color{black} $\,\,$ & $\,\,$\color{red} 0.9652\color{black} $\,\,$ & $\,\,$ 1 $\,\,$ & $\,\,$\color{red} 1.8444\color{black}  $\,\,$ \\
$\,\,$0.1038$\,\,$ & $\,\,$0.5233$\,\,$ & $\,\,$\color{red} 0.5422\color{black} $\,\,$ & $\,\,$ 1  $\,\,$ \\
\end{pmatrix},
\end{equation*}

\begin{equation*}
\mathbf{w}^{\prime} =
\begin{pmatrix}
0.666535\\
0.132154\\
0.132154\\
0.069156
\end{pmatrix} =
0.995399\cdot
\begin{pmatrix}
0.669616\\
0.132765\\
\color{gr} 0.132765\color{black} \\
0.069476
\end{pmatrix},
\end{equation*}
\begin{equation*}
\left[ \frac{{w}^{\prime}_i}{{w}^{\prime}_j} \right] =
\begin{pmatrix}
$\,\,$ 1 $\,\,$ & $\,\,$5.0436$\,\,$ & $\,\,$\color{gr} 5.0436\color{black} $\,\,$ & $\,\,$9.6381$\,\,$ \\
$\,\,$0.1983$\,\,$ & $\,\,$ 1 $\,\,$ & $\,\,$\color{gr} \color{blue} 1\color{black} $\,\,$ & $\,\,$1.9109  $\,\,$ \\
$\,\,$\color{gr} 0.1983\color{black} $\,\,$ & $\,\,$\color{gr} \color{blue} 1\color{black} $\,\,$ & $\,\,$ 1 $\,\,$ & $\,\,$\color{gr} 1.9109\color{black}  $\,\,$ \\
$\,\,$0.1038$\,\,$ & $\,\,$0.5233$\,\,$ & $\,\,$\color{gr} 0.5233\color{black} $\,\,$ & $\,\,$ 1  $\,\,$ \\
\end{pmatrix},
\end{equation*}
\end{example}
\newpage
\begin{example}
\begin{equation*}
\mathbf{A} =
\begin{pmatrix}
$\,\,$ 1 $\,\,$ & $\,\,$8$\,\,$ & $\,\,$5$\,\,$ & $\,\,$6 $\,\,$ \\
$\,\,$ 1/8$\,\,$ & $\,\,$ 1 $\,\,$ & $\,\,$1$\,\,$ & $\,\,$4 $\,\,$ \\
$\,\,$ 1/5$\,\,$ & $\,\,$ 1 $\,\,$ & $\,\,$ 1 $\,\,$ & $\,\,$2 $\,\,$ \\
$\,\,$ 1/6$\,\,$ & $\,\,$ 1/4$\,\,$ & $\,\,$ 1/2$\,\,$ & $\,\,$ 1  $\,\,$ \\
\end{pmatrix},
\qquad
\lambda_{\max} =
4.2460,
\qquad
CR = 0.0928
\end{equation*}

\begin{equation*}
\mathbf{w}^{EM} =
\begin{pmatrix}
0.666494\\
0.143789\\
\color{red} 0.125146\color{black} \\
0.064572
\end{pmatrix}\end{equation*}
\begin{equation*}
\left[ \frac{{w}^{EM}_i}{{w}^{EM}_j} \right] =
\begin{pmatrix}
$\,\,$ 1 $\,\,$ & $\,\,$4.6352$\,\,$ & $\,\,$\color{red} 5.3257\color{black} $\,\,$ & $\,\,$10.3218$\,\,$ \\
$\,\,$0.2157$\,\,$ & $\,\,$ 1 $\,\,$ & $\,\,$\color{red} 1.1490\color{black} $\,\,$ & $\,\,$2.2268  $\,\,$ \\
$\,\,$\color{red} 0.1878\color{black} $\,\,$ & $\,\,$\color{red} 0.8703\color{black} $\,\,$ & $\,\,$ 1 $\,\,$ & $\,\,$\color{red} 1.9381\color{black}  $\,\,$ \\
$\,\,$0.0969$\,\,$ & $\,\,$0.4491$\,\,$ & $\,\,$\color{red} 0.5160\color{black} $\,\,$ & $\,\,$ 1  $\,\,$ \\
\end{pmatrix},
\end{equation*}

\begin{equation*}
\mathbf{w}^{\prime} =
\begin{pmatrix}
0.663840\\
0.143216\\
0.128629\\
0.064315
\end{pmatrix} =
0.996019\cdot
\begin{pmatrix}
0.666494\\
0.143789\\
\color{gr} 0.129143\color{black} \\
0.064572
\end{pmatrix},
\end{equation*}
\begin{equation*}
\left[ \frac{{w}^{\prime}_i}{{w}^{\prime}_j} \right] =
\begin{pmatrix}
$\,\,$ 1 $\,\,$ & $\,\,$4.6352$\,\,$ & $\,\,$\color{gr} 5.1609\color{black} $\,\,$ & $\,\,$10.3218$\,\,$ \\
$\,\,$0.2157$\,\,$ & $\,\,$ 1 $\,\,$ & $\,\,$\color{gr} 1.1134\color{black} $\,\,$ & $\,\,$2.2268  $\,\,$ \\
$\,\,$\color{gr} 0.1938\color{black} $\,\,$ & $\,\,$\color{gr} 0.8981\color{black} $\,\,$ & $\,\,$ 1 $\,\,$ & $\,\,$\color{gr} \color{blue} 2\color{black}  $\,\,$ \\
$\,\,$0.0969$\,\,$ & $\,\,$0.4491$\,\,$ & $\,\,$\color{gr} \color{blue}  1/2\color{black} $\,\,$ & $\,\,$ 1  $\,\,$ \\
\end{pmatrix},
\end{equation*}
\end{example}
\newpage
\begin{example}
\begin{equation*}
\mathbf{A} =
\begin{pmatrix}
$\,\,$ 1 $\,\,$ & $\,\,$8$\,\,$ & $\,\,$5$\,\,$ & $\,\,$7 $\,\,$ \\
$\,\,$ 1/8$\,\,$ & $\,\,$ 1 $\,\,$ & $\,\,$1$\,\,$ & $\,\,$3 $\,\,$ \\
$\,\,$ 1/5$\,\,$ & $\,\,$ 1 $\,\,$ & $\,\,$ 1 $\,\,$ & $\,\,$2 $\,\,$ \\
$\,\,$ 1/7$\,\,$ & $\,\,$ 1/3$\,\,$ & $\,\,$ 1/2$\,\,$ & $\,\,$ 1  $\,\,$ \\
\end{pmatrix},
\qquad
\lambda_{\max} =
4.1301,
\qquad
CR = 0.0490
\end{equation*}

\begin{equation*}
\mathbf{w}^{EM} =
\begin{pmatrix}
0.678875\\
0.129771\\
\color{red} 0.126364\color{black} \\
0.064990
\end{pmatrix}\end{equation*}
\begin{equation*}
\left[ \frac{{w}^{EM}_i}{{w}^{EM}_j} \right] =
\begin{pmatrix}
$\,\,$ 1 $\,\,$ & $\,\,$5.2313$\,\,$ & $\,\,$\color{red} 5.3724\color{black} $\,\,$ & $\,\,$10.4459$\,\,$ \\
$\,\,$0.1912$\,\,$ & $\,\,$ 1 $\,\,$ & $\,\,$\color{red} 1.0270\color{black} $\,\,$ & $\,\,$1.9968  $\,\,$ \\
$\,\,$\color{red} 0.1861\color{black} $\,\,$ & $\,\,$\color{red} 0.9737\color{black} $\,\,$ & $\,\,$ 1 $\,\,$ & $\,\,$\color{red} 1.9444\color{black}  $\,\,$ \\
$\,\,$0.0957$\,\,$ & $\,\,$0.5008$\,\,$ & $\,\,$\color{red} 0.5143\color{black} $\,\,$ & $\,\,$ 1  $\,\,$ \\
\end{pmatrix},
\end{equation*}

\begin{equation*}
\mathbf{w}^{\prime} =
\begin{pmatrix}
0.676570\\
0.129331\\
0.129331\\
0.064769
\end{pmatrix} =
0.996604\cdot
\begin{pmatrix}
0.678875\\
0.129771\\
\color{gr} 0.129771\color{black} \\
0.064990
\end{pmatrix},
\end{equation*}
\begin{equation*}
\left[ \frac{{w}^{\prime}_i}{{w}^{\prime}_j} \right] =
\begin{pmatrix}
$\,\,$ 1 $\,\,$ & $\,\,$5.2313$\,\,$ & $\,\,$\color{gr} 5.2313\color{black} $\,\,$ & $\,\,$10.4459$\,\,$ \\
$\,\,$0.1912$\,\,$ & $\,\,$ 1 $\,\,$ & $\,\,$\color{gr} \color{blue} 1\color{black} $\,\,$ & $\,\,$1.9968  $\,\,$ \\
$\,\,$\color{gr} 0.1912\color{black} $\,\,$ & $\,\,$\color{gr} \color{blue} 1\color{black} $\,\,$ & $\,\,$ 1 $\,\,$ & $\,\,$\color{gr} 1.9968\color{black}  $\,\,$ \\
$\,\,$0.0957$\,\,$ & $\,\,$0.5008$\,\,$ & $\,\,$\color{gr} 0.5008\color{black} $\,\,$ & $\,\,$ 1  $\,\,$ \\
\end{pmatrix},
\end{equation*}
\end{example}
\newpage
\begin{example}
\begin{equation*}
\mathbf{A} =
\begin{pmatrix}
$\,\,$ 1 $\,\,$ & $\,\,$8$\,\,$ & $\,\,$5$\,\,$ & $\,\,$8 $\,\,$ \\
$\,\,$ 1/8$\,\,$ & $\,\,$ 1 $\,\,$ & $\,\,$1$\,\,$ & $\,\,$4 $\,\,$ \\
$\,\,$ 1/5$\,\,$ & $\,\,$ 1 $\,\,$ & $\,\,$ 1 $\,\,$ & $\,\,$3 $\,\,$ \\
$\,\,$ 1/8$\,\,$ & $\,\,$ 1/4$\,\,$ & $\,\,$ 1/3$\,\,$ & $\,\,$ 1  $\,\,$ \\
\end{pmatrix},
\qquad
\lambda_{\max} =
4.1689,
\qquad
CR = 0.0637
\end{equation*}

\begin{equation*}
\mathbf{w}^{EM} =
\begin{pmatrix}
0.680270\\
0.134165\\
\color{red} 0.134046\color{black} \\
0.051519
\end{pmatrix}\end{equation*}
\begin{equation*}
\left[ \frac{{w}^{EM}_i}{{w}^{EM}_j} \right] =
\begin{pmatrix}
$\,\,$ 1 $\,\,$ & $\,\,$5.0704$\,\,$ & $\,\,$\color{red} 5.0749\color{black} $\,\,$ & $\,\,$13.2043$\,\,$ \\
$\,\,$0.1972$\,\,$ & $\,\,$ 1 $\,\,$ & $\,\,$\color{red} 1.0009\color{black} $\,\,$ & $\,\,$2.6042  $\,\,$ \\
$\,\,$\color{red} 0.1970\color{black} $\,\,$ & $\,\,$\color{red} 0.9991\color{black} $\,\,$ & $\,\,$ 1 $\,\,$ & $\,\,$\color{red} 2.6019\color{black}  $\,\,$ \\
$\,\,$0.0757$\,\,$ & $\,\,$0.3840$\,\,$ & $\,\,$\color{red} 0.3843\color{black} $\,\,$ & $\,\,$ 1  $\,\,$ \\
\end{pmatrix},
\end{equation*}

\begin{equation*}
\mathbf{w}^{\prime} =
\begin{pmatrix}
0.680189\\
0.134149\\
0.134149\\
0.051513
\end{pmatrix} =
0.999880\cdot
\begin{pmatrix}
0.680270\\
0.134165\\
\color{gr} 0.134165\color{black} \\
0.051519
\end{pmatrix},
\end{equation*}
\begin{equation*}
\left[ \frac{{w}^{\prime}_i}{{w}^{\prime}_j} \right] =
\begin{pmatrix}
$\,\,$ 1 $\,\,$ & $\,\,$5.0704$\,\,$ & $\,\,$\color{gr} 5.0704\color{black} $\,\,$ & $\,\,$13.2043$\,\,$ \\
$\,\,$0.1972$\,\,$ & $\,\,$ 1 $\,\,$ & $\,\,$\color{gr} \color{blue} 1\color{black} $\,\,$ & $\,\,$2.6042  $\,\,$ \\
$\,\,$\color{gr} 0.1972\color{black} $\,\,$ & $\,\,$\color{gr} \color{blue} 1\color{black} $\,\,$ & $\,\,$ 1 $\,\,$ & $\,\,$\color{gr} 2.6042\color{black}  $\,\,$ \\
$\,\,$0.0757$\,\,$ & $\,\,$0.3840$\,\,$ & $\,\,$\color{gr} 0.3840\color{black} $\,\,$ & $\,\,$ 1  $\,\,$ \\
\end{pmatrix},
\end{equation*}
\end{example}
\newpage
\begin{example}
\begin{equation*}
\mathbf{A} =
\begin{pmatrix}
$\,\,$ 1 $\,\,$ & $\,\,$8$\,\,$ & $\,\,$5$\,\,$ & $\,\,$8 $\,\,$ \\
$\,\,$ 1/8$\,\,$ & $\,\,$ 1 $\,\,$ & $\,\,$1$\,\,$ & $\,\,$5 $\,\,$ \\
$\,\,$ 1/5$\,\,$ & $\,\,$ 1 $\,\,$ & $\,\,$ 1 $\,\,$ & $\,\,$3 $\,\,$ \\
$\,\,$ 1/8$\,\,$ & $\,\,$ 1/5$\,\,$ & $\,\,$ 1/3$\,\,$ & $\,\,$ 1  $\,\,$ \\
\end{pmatrix},
\qquad
\lambda_{\max} =
4.2259,
\qquad
CR = 0.0852
\end{equation*}

\begin{equation*}
\mathbf{w}^{EM} =
\begin{pmatrix}
0.677522\\
0.142411\\
\color{red} 0.131406\color{black} \\
0.048661
\end{pmatrix}\end{equation*}
\begin{equation*}
\left[ \frac{{w}^{EM}_i}{{w}^{EM}_j} \right] =
\begin{pmatrix}
$\,\,$ 1 $\,\,$ & $\,\,$4.7575$\,\,$ & $\,\,$\color{red} 5.1560\color{black} $\,\,$ & $\,\,$13.9233$\,\,$ \\
$\,\,$0.2102$\,\,$ & $\,\,$ 1 $\,\,$ & $\,\,$\color{red} 1.0838\color{black} $\,\,$ & $\,\,$2.9266  $\,\,$ \\
$\,\,$\color{red} 0.1940\color{black} $\,\,$ & $\,\,$\color{red} 0.9227\color{black} $\,\,$ & $\,\,$ 1 $\,\,$ & $\,\,$\color{red} 2.7004\color{black}  $\,\,$ \\
$\,\,$0.0718$\,\,$ & $\,\,$0.3417$\,\,$ & $\,\,$\color{red} 0.3703\color{black} $\,\,$ & $\,\,$ 1  $\,\,$ \\
\end{pmatrix},
\end{equation*}

\begin{equation*}
\mathbf{w}^{\prime} =
\begin{pmatrix}
0.674757\\
0.141830\\
0.134951\\
0.048462
\end{pmatrix} =
0.995918\cdot
\begin{pmatrix}
0.677522\\
0.142411\\
\color{gr} 0.135504\color{black} \\
0.048661
\end{pmatrix},
\end{equation*}
\begin{equation*}
\left[ \frac{{w}^{\prime}_i}{{w}^{\prime}_j} \right] =
\begin{pmatrix}
$\,\,$ 1 $\,\,$ & $\,\,$4.7575$\,\,$ & $\,\,$\color{gr} \color{blue} 5\color{black} $\,\,$ & $\,\,$13.9233$\,\,$ \\
$\,\,$0.2102$\,\,$ & $\,\,$ 1 $\,\,$ & $\,\,$\color{gr} 1.0510\color{black} $\,\,$ & $\,\,$2.9266  $\,\,$ \\
$\,\,$\color{gr} \color{blue}  1/5\color{black} $\,\,$ & $\,\,$\color{gr} 0.9515\color{black} $\,\,$ & $\,\,$ 1 $\,\,$ & $\,\,$\color{gr} 2.7847\color{black}  $\,\,$ \\
$\,\,$0.0718$\,\,$ & $\,\,$0.3417$\,\,$ & $\,\,$\color{gr} 0.3591\color{black} $\,\,$ & $\,\,$ 1  $\,\,$ \\
\end{pmatrix},
\end{equation*}
\end{example}
\newpage
\begin{example}
\begin{equation*}
\mathbf{A} =
\begin{pmatrix}
$\,\,$ 1 $\,\,$ & $\,\,$8$\,\,$ & $\,\,$5$\,\,$ & $\,\,$9 $\,\,$ \\
$\,\,$ 1/8$\,\,$ & $\,\,$ 1 $\,\,$ & $\,\,$1$\,\,$ & $\,\,$5 $\,\,$ \\
$\,\,$ 1/5$\,\,$ & $\,\,$ 1 $\,\,$ & $\,\,$ 1 $\,\,$ & $\,\,$3 $\,\,$ \\
$\,\,$ 1/9$\,\,$ & $\,\,$ 1/5$\,\,$ & $\,\,$ 1/3$\,\,$ & $\,\,$ 1  $\,\,$ \\
\end{pmatrix},
\qquad
\lambda_{\max} =
4.1922,
\qquad
CR = 0.0725
\end{equation*}

\begin{equation*}
\mathbf{w}^{EM} =
\begin{pmatrix}
0.684055\\
0.139789\\
\color{red} 0.130013\color{black} \\
0.046144
\end{pmatrix}\end{equation*}
\begin{equation*}
\left[ \frac{{w}^{EM}_i}{{w}^{EM}_j} \right] =
\begin{pmatrix}
$\,\,$ 1 $\,\,$ & $\,\,$4.8935$\,\,$ & $\,\,$\color{red} 5.2615\color{black} $\,\,$ & $\,\,$14.8245$\,\,$ \\
$\,\,$0.2044$\,\,$ & $\,\,$ 1 $\,\,$ & $\,\,$\color{red} 1.0752\color{black} $\,\,$ & $\,\,$3.0294  $\,\,$ \\
$\,\,$\color{red} 0.1901\color{black} $\,\,$ & $\,\,$\color{red} 0.9301\color{black} $\,\,$ & $\,\,$ 1 $\,\,$ & $\,\,$\color{red} 2.8176\color{black}  $\,\,$ \\
$\,\,$0.0675$\,\,$ & $\,\,$0.3301$\,\,$ & $\,\,$\color{red} 0.3549\color{black} $\,\,$ & $\,\,$ 1  $\,\,$ \\
\end{pmatrix},
\end{equation*}

\begin{equation*}
\mathbf{w}^{\prime} =
\begin{pmatrix}
0.679436\\
0.138845\\
0.135887\\
0.045832
\end{pmatrix} =
0.993247\cdot
\begin{pmatrix}
0.684055\\
0.139789\\
\color{gr} 0.136811\color{black} \\
0.046144
\end{pmatrix},
\end{equation*}
\begin{equation*}
\left[ \frac{{w}^{\prime}_i}{{w}^{\prime}_j} \right] =
\begin{pmatrix}
$\,\,$ 1 $\,\,$ & $\,\,$4.8935$\,\,$ & $\,\,$\color{gr} \color{blue} 5\color{black} $\,\,$ & $\,\,$14.8245$\,\,$ \\
$\,\,$0.2044$\,\,$ & $\,\,$ 1 $\,\,$ & $\,\,$\color{gr} 1.0218\color{black} $\,\,$ & $\,\,$3.0294  $\,\,$ \\
$\,\,$\color{gr} \color{blue}  1/5\color{black} $\,\,$ & $\,\,$\color{gr} 0.9787\color{black} $\,\,$ & $\,\,$ 1 $\,\,$ & $\,\,$\color{gr} 2.9649\color{black}  $\,\,$ \\
$\,\,$0.0675$\,\,$ & $\,\,$0.3301$\,\,$ & $\,\,$\color{gr} 0.3373\color{black} $\,\,$ & $\,\,$ 1  $\,\,$ \\
\end{pmatrix},
\end{equation*}
\end{example}
\newpage
\begin{example}
\begin{equation*}
\mathbf{A} =
\begin{pmatrix}
$\,\,$ 1 $\,\,$ & $\,\,$8$\,\,$ & $\,\,$5$\,\,$ & $\,\,$9 $\,\,$ \\
$\,\,$ 1/8$\,\,$ & $\,\,$ 1 $\,\,$ & $\,\,$1$\,\,$ & $\,\,$6 $\,\,$ \\
$\,\,$ 1/5$\,\,$ & $\,\,$ 1 $\,\,$ & $\,\,$ 1 $\,\,$ & $\,\,$3 $\,\,$ \\
$\,\,$ 1/9$\,\,$ & $\,\,$ 1/6$\,\,$ & $\,\,$ 1/3$\,\,$ & $\,\,$ 1  $\,\,$ \\
\end{pmatrix},
\qquad
\lambda_{\max} =
4.2460,
\qquad
CR = 0.0928
\end{equation*}

\begin{equation*}
\mathbf{w}^{EM} =
\begin{pmatrix}
0.681155\\
0.146952\\
\color{red} 0.127899\color{black} \\
0.043995
\end{pmatrix}\end{equation*}
\begin{equation*}
\left[ \frac{{w}^{EM}_i}{{w}^{EM}_j} \right] =
\begin{pmatrix}
$\,\,$ 1 $\,\,$ & $\,\,$4.6352$\,\,$ & $\,\,$\color{red} 5.3257\color{black} $\,\,$ & $\,\,$15.4827$\,\,$ \\
$\,\,$0.2157$\,\,$ & $\,\,$ 1 $\,\,$ & $\,\,$\color{red} 1.1490\color{black} $\,\,$ & $\,\,$3.3402  $\,\,$ \\
$\,\,$\color{red} 0.1878\color{black} $\,\,$ & $\,\,$\color{red} 0.8703\color{black} $\,\,$ & $\,\,$ 1 $\,\,$ & $\,\,$\color{red} 2.9071\color{black}  $\,\,$ \\
$\,\,$0.0646$\,\,$ & $\,\,$0.2994$\,\,$ & $\,\,$\color{red} 0.3440\color{black} $\,\,$ & $\,\,$ 1  $\,\,$ \\
\end{pmatrix},
\end{equation*}

\begin{equation*}
\mathbf{w}^{\prime} =
\begin{pmatrix}
0.678384\\
0.146354\\
0.131447\\
0.043816
\end{pmatrix} =
0.995932\cdot
\begin{pmatrix}
0.681155\\
0.146952\\
\color{gr} 0.131984\color{black} \\
0.043995
\end{pmatrix},
\end{equation*}
\begin{equation*}
\left[ \frac{{w}^{\prime}_i}{{w}^{\prime}_j} \right] =
\begin{pmatrix}
$\,\,$ 1 $\,\,$ & $\,\,$4.6352$\,\,$ & $\,\,$\color{gr} 5.1609\color{black} $\,\,$ & $\,\,$15.4827$\,\,$ \\
$\,\,$0.2157$\,\,$ & $\,\,$ 1 $\,\,$ & $\,\,$\color{gr} 1.1134\color{black} $\,\,$ & $\,\,$3.3402  $\,\,$ \\
$\,\,$\color{gr} 0.1938\color{black} $\,\,$ & $\,\,$\color{gr} 0.8981\color{black} $\,\,$ & $\,\,$ 1 $\,\,$ & $\,\,$\color{gr} \color{blue} 3\color{black}  $\,\,$ \\
$\,\,$0.0646$\,\,$ & $\,\,$0.2994$\,\,$ & $\,\,$\color{gr} \color{blue}  1/3\color{black} $\,\,$ & $\,\,$ 1  $\,\,$ \\
\end{pmatrix},
\end{equation*}
\end{example}
\newpage
\begin{example}
\begin{equation*}
\mathbf{A} =
\begin{pmatrix}
$\,\,$ 1 $\,\,$ & $\,\,$8$\,\,$ & $\,\,$5$\,\,$ & $\,\,$9 $\,\,$ \\
$\,\,$ 1/8$\,\,$ & $\,\,$ 1 $\,\,$ & $\,\,$3$\,\,$ & $\,\,$2 $\,\,$ \\
$\,\,$ 1/5$\,\,$ & $\,\,$ 1/3$\,\,$ & $\,\,$ 1 $\,\,$ & $\,\,$1 $\,\,$ \\
$\,\,$ 1/9$\,\,$ & $\,\,$ 1/2$\,\,$ & $\,\,$ 1 $\,\,$ & $\,\,$ 1  $\,\,$ \\
\end{pmatrix},
\qquad
\lambda_{\max} =
4.2146,
\qquad
CR = 0.0809
\end{equation*}

\begin{equation*}
\mathbf{w}^{EM} =
\begin{pmatrix}
0.698059\\
0.148136\\
0.081335\\
\color{red} 0.072470\color{black}
\end{pmatrix}\end{equation*}
\begin{equation*}
\left[ \frac{{w}^{EM}_i}{{w}^{EM}_j} \right] =
\begin{pmatrix}
$\,\,$ 1 $\,\,$ & $\,\,$4.7123$\,\,$ & $\,\,$8.5825$\,\,$ & $\,\,$\color{red} 9.6323\color{black} $\,\,$ \\
$\,\,$0.2122$\,\,$ & $\,\,$ 1 $\,\,$ & $\,\,$1.8213$\,\,$ & $\,\,$\color{red} 2.0441\color{black}   $\,\,$ \\
$\,\,$0.1165$\,\,$ & $\,\,$0.5491$\,\,$ & $\,\,$ 1 $\,\,$ & $\,\,$\color{red} 1.1223\color{black}  $\,\,$ \\
$\,\,$\color{red} 0.1038\color{black} $\,\,$ & $\,\,$\color{red} 0.4892\color{black} $\,\,$ & $\,\,$\color{red} 0.8910\color{black} $\,\,$ & $\,\,$ 1  $\,\,$ \\
\end{pmatrix},
\end{equation*}

\begin{equation*}
\mathbf{w}^{\prime} =
\begin{pmatrix}
0.696945\\
0.147900\\
0.081205\\
0.073950
\end{pmatrix} =
0.998405\cdot
\begin{pmatrix}
0.698059\\
0.148136\\
0.081335\\
\color{gr} 0.074068\color{black}
\end{pmatrix},
\end{equation*}
\begin{equation*}
\left[ \frac{{w}^{\prime}_i}{{w}^{\prime}_j} \right] =
\begin{pmatrix}
$\,\,$ 1 $\,\,$ & $\,\,$4.7123$\,\,$ & $\,\,$8.5825$\,\,$ & $\,\,$\color{gr} 9.4246\color{black} $\,\,$ \\
$\,\,$0.2122$\,\,$ & $\,\,$ 1 $\,\,$ & $\,\,$1.8213$\,\,$ & $\,\,$\color{gr} \color{blue} 2\color{black}   $\,\,$ \\
$\,\,$0.1165$\,\,$ & $\,\,$0.5491$\,\,$ & $\,\,$ 1 $\,\,$ & $\,\,$\color{gr} 1.0981\color{black}  $\,\,$ \\
$\,\,$\color{gr} 0.1061\color{black} $\,\,$ & $\,\,$\color{gr} \color{blue}  1/2\color{black} $\,\,$ & $\,\,$\color{gr} 0.9107\color{black} $\,\,$ & $\,\,$ 1  $\,\,$ \\
\end{pmatrix},
\end{equation*}
\end{example}
\newpage
\begin{example}
\begin{equation*}
\mathbf{A} =
\begin{pmatrix}
$\,\,$ 1 $\,\,$ & $\,\,$8$\,\,$ & $\,\,$6$\,\,$ & $\,\,$8 $\,\,$ \\
$\,\,$ 1/8$\,\,$ & $\,\,$ 1 $\,\,$ & $\,\,$1$\,\,$ & $\,\,$3 $\,\,$ \\
$\,\,$ 1/6$\,\,$ & $\,\,$ 1 $\,\,$ & $\,\,$ 1 $\,\,$ & $\,\,$2 $\,\,$ \\
$\,\,$ 1/8$\,\,$ & $\,\,$ 1/3$\,\,$ & $\,\,$ 1/2$\,\,$ & $\,\,$ 1  $\,\,$ \\
\end{pmatrix},
\qquad
\lambda_{\max} =
4.1031,
\qquad
CR = 0.0389
\end{equation*}

\begin{equation*}
\mathbf{w}^{EM} =
\begin{pmatrix}
0.699565\\
0.123890\\
\color{red} 0.116315\color{black} \\
0.060229
\end{pmatrix}\end{equation*}
\begin{equation*}
\left[ \frac{{w}^{EM}_i}{{w}^{EM}_j} \right] =
\begin{pmatrix}
$\,\,$ 1 $\,\,$ & $\,\,$5.6467$\,\,$ & $\,\,$\color{red} 6.0144\color{black} $\,\,$ & $\,\,$11.6150$\,\,$ \\
$\,\,$0.1771$\,\,$ & $\,\,$ 1 $\,\,$ & $\,\,$\color{red} 1.0651\color{black} $\,\,$ & $\,\,$2.0570  $\,\,$ \\
$\,\,$\color{red} 0.1663\color{black} $\,\,$ & $\,\,$\color{red} 0.9389\color{black} $\,\,$ & $\,\,$ 1 $\,\,$ & $\,\,$\color{red} 1.9312\color{black}  $\,\,$ \\
$\,\,$0.0861$\,\,$ & $\,\,$0.4862$\,\,$ & $\,\,$\color{red} 0.5178\color{black} $\,\,$ & $\,\,$ 1  $\,\,$ \\
\end{pmatrix},
\end{equation*}

\begin{equation*}
\mathbf{w}^{\prime} =
\begin{pmatrix}
0.699370\\
0.123856\\
0.116562\\
0.060213
\end{pmatrix} =
0.999721\cdot
\begin{pmatrix}
0.699565\\
0.123890\\
\color{gr} 0.116594\color{black} \\
0.060229
\end{pmatrix},
\end{equation*}
\begin{equation*}
\left[ \frac{{w}^{\prime}_i}{{w}^{\prime}_j} \right] =
\begin{pmatrix}
$\,\,$ 1 $\,\,$ & $\,\,$5.6467$\,\,$ & $\,\,$\color{gr} \color{blue} 6\color{black} $\,\,$ & $\,\,$11.6150$\,\,$ \\
$\,\,$0.1771$\,\,$ & $\,\,$ 1 $\,\,$ & $\,\,$\color{gr} 1.0626\color{black} $\,\,$ & $\,\,$2.0570  $\,\,$ \\
$\,\,$\color{gr} \color{blue}  1/6\color{black} $\,\,$ & $\,\,$\color{gr} 0.9411\color{black} $\,\,$ & $\,\,$ 1 $\,\,$ & $\,\,$\color{gr} 1.9358\color{black}  $\,\,$ \\
$\,\,$0.0861$\,\,$ & $\,\,$0.4862$\,\,$ & $\,\,$\color{gr} 0.5166\color{black} $\,\,$ & $\,\,$ 1  $\,\,$ \\
\end{pmatrix},
\end{equation*}
\end{example}
\newpage
\begin{example}
\begin{equation*}
\mathbf{A} =
\begin{pmatrix}
$\,\,$ 1 $\,\,$ & $\,\,$8$\,\,$ & $\,\,$6$\,\,$ & $\,\,$9 $\,\,$ \\
$\,\,$ 1/8$\,\,$ & $\,\,$ 1 $\,\,$ & $\,\,$3$\,\,$ & $\,\,$2 $\,\,$ \\
$\,\,$ 1/6$\,\,$ & $\,\,$ 1/3$\,\,$ & $\,\,$ 1 $\,\,$ & $\,\,$1 $\,\,$ \\
$\,\,$ 1/9$\,\,$ & $\,\,$ 1/2$\,\,$ & $\,\,$ 1 $\,\,$ & $\,\,$ 1  $\,\,$ \\
\end{pmatrix},
\qquad
\lambda_{\max} =
4.1664,
\qquad
CR = 0.0627
\end{equation*}

\begin{equation*}
\mathbf{w}^{EM} =
\begin{pmatrix}
0.709382\\
0.144194\\
0.075057\\
\color{red} 0.071366\color{black}
\end{pmatrix}\end{equation*}
\begin{equation*}
\left[ \frac{{w}^{EM}_i}{{w}^{EM}_j} \right] =
\begin{pmatrix}
$\,\,$ 1 $\,\,$ & $\,\,$4.9196$\,\,$ & $\,\,$9.4512$\,\,$ & $\,\,$\color{red} 9.9400\color{black} $\,\,$ \\
$\,\,$0.2033$\,\,$ & $\,\,$ 1 $\,\,$ & $\,\,$1.9211$\,\,$ & $\,\,$\color{red} 2.0205\color{black}   $\,\,$ \\
$\,\,$0.1058$\,\,$ & $\,\,$0.5205$\,\,$ & $\,\,$ 1 $\,\,$ & $\,\,$\color{red} 1.0517\color{black}  $\,\,$ \\
$\,\,$\color{red} 0.1006\color{black} $\,\,$ & $\,\,$\color{red} 0.4949\color{black} $\,\,$ & $\,\,$\color{red} 0.9508\color{black} $\,\,$ & $\,\,$ 1  $\,\,$ \\
\end{pmatrix},
\end{equation*}

\begin{equation*}
\mathbf{w}^{\prime} =
\begin{pmatrix}
0.708864\\
0.144089\\
0.075002\\
0.072044
\end{pmatrix} =
0.999270\cdot
\begin{pmatrix}
0.709382\\
0.144194\\
0.075057\\
\color{gr} 0.072097\color{black}
\end{pmatrix},
\end{equation*}
\begin{equation*}
\left[ \frac{{w}^{\prime}_i}{{w}^{\prime}_j} \right] =
\begin{pmatrix}
$\,\,$ 1 $\,\,$ & $\,\,$4.9196$\,\,$ & $\,\,$9.4512$\,\,$ & $\,\,$\color{gr} 9.8393\color{black} $\,\,$ \\
$\,\,$0.2033$\,\,$ & $\,\,$ 1 $\,\,$ & $\,\,$1.9211$\,\,$ & $\,\,$\color{gr} \color{blue} 2\color{black}   $\,\,$ \\
$\,\,$0.1058$\,\,$ & $\,\,$0.5205$\,\,$ & $\,\,$ 1 $\,\,$ & $\,\,$\color{gr} 1.0411\color{black}  $\,\,$ \\
$\,\,$\color{gr} 0.1016\color{black} $\,\,$ & $\,\,$\color{gr} \color{blue}  1/2\color{black} $\,\,$ & $\,\,$\color{gr} 0.9606\color{black} $\,\,$ & $\,\,$ 1  $\,\,$ \\
\end{pmatrix},
\end{equation*}
\end{example}
\newpage
\begin{example}
\begin{equation*}
\mathbf{A} =
\begin{pmatrix}
$\,\,$ 1 $\,\,$ & $\,\,$8$\,\,$ & $\,\,$6$\,\,$ & $\,\,$9 $\,\,$ \\
$\,\,$ 1/8$\,\,$ & $\,\,$ 1 $\,\,$ & $\,\,$4$\,\,$ & $\,\,$2 $\,\,$ \\
$\,\,$ 1/6$\,\,$ & $\,\,$ 1/4$\,\,$ & $\,\,$ 1 $\,\,$ & $\,\,$1 $\,\,$ \\
$\,\,$ 1/9$\,\,$ & $\,\,$ 1/2$\,\,$ & $\,\,$ 1 $\,\,$ & $\,\,$ 1  $\,\,$ \\
\end{pmatrix},
\qquad
\lambda_{\max} =
4.2469,
\qquad
CR = 0.0931
\end{equation*}

\begin{equation*}
\mathbf{w}^{EM} =
\begin{pmatrix}
0.705113\\
0.155734\\
0.069605\\
\color{red} 0.069548\color{black}
\end{pmatrix}\end{equation*}
\begin{equation*}
\left[ \frac{{w}^{EM}_i}{{w}^{EM}_j} \right] =
\begin{pmatrix}
$\,\,$ 1 $\,\,$ & $\,\,$4.5277$\,\,$ & $\,\,$10.1302$\,\,$ & $\,\,$\color{red} 10.1384\color{black} $\,\,$ \\
$\,\,$0.2209$\,\,$ & $\,\,$ 1 $\,\,$ & $\,\,$2.2374$\,\,$ & $\,\,$\color{red} 2.2392\color{black}   $\,\,$ \\
$\,\,$0.0987$\,\,$ & $\,\,$0.4469$\,\,$ & $\,\,$ 1 $\,\,$ & $\,\,$\color{red} 1.0008\color{black}  $\,\,$ \\
$\,\,$\color{red} 0.0986\color{black} $\,\,$ & $\,\,$\color{red} 0.4466\color{black} $\,\,$ & $\,\,$\color{red} 0.9992\color{black} $\,\,$ & $\,\,$ 1  $\,\,$ \\
\end{pmatrix},
\end{equation*}

\begin{equation*}
\mathbf{w}^{\prime} =
\begin{pmatrix}
0.705073\\
0.155725\\
0.069601\\
0.069601
\end{pmatrix} =
0.999944\cdot
\begin{pmatrix}
0.705113\\
0.155734\\
0.069605\\
\color{gr} 0.069605\color{black}
\end{pmatrix},
\end{equation*}
\begin{equation*}
\left[ \frac{{w}^{\prime}_i}{{w}^{\prime}_j} \right] =
\begin{pmatrix}
$\,\,$ 1 $\,\,$ & $\,\,$4.5277$\,\,$ & $\,\,$10.1302$\,\,$ & $\,\,$\color{gr} 10.1302\color{black} $\,\,$ \\
$\,\,$0.2209$\,\,$ & $\,\,$ 1 $\,\,$ & $\,\,$2.2374$\,\,$ & $\,\,$\color{gr} 2.2374\color{black}   $\,\,$ \\
$\,\,$0.0987$\,\,$ & $\,\,$0.4469$\,\,$ & $\,\,$ 1 $\,\,$ & $\,\,$\color{gr} \color{blue} 1\color{black}  $\,\,$ \\
$\,\,$\color{gr} 0.0987\color{black} $\,\,$ & $\,\,$\color{gr} 0.4469\color{black} $\,\,$ & $\,\,$\color{gr} \color{blue} 1\color{black} $\,\,$ & $\,\,$ 1  $\,\,$ \\
\end{pmatrix},
\end{equation*}
\end{example}
\newpage
\begin{example}
\begin{equation*}
\mathbf{A} =
\begin{pmatrix}
$\,\,$ 1 $\,\,$ & $\,\,$8$\,\,$ & $\,\,$8$\,\,$ & $\,\,$9 $\,\,$ \\
$\,\,$ 1/8$\,\,$ & $\,\,$ 1 $\,\,$ & $\,\,$2$\,\,$ & $\,\,$6 $\,\,$ \\
$\,\,$ 1/8$\,\,$ & $\,\,$ 1/2$\,\,$ & $\,\,$ 1 $\,\,$ & $\,\,$2 $\,\,$ \\
$\,\,$ 1/9$\,\,$ & $\,\,$ 1/6$\,\,$ & $\,\,$ 1/2$\,\,$ & $\,\,$ 1  $\,\,$ \\
\end{pmatrix},
\qquad
\lambda_{\max} =
4.2469,
\qquad
CR = 0.0931
\end{equation*}

\begin{equation*}
\mathbf{w}^{EM} =
\begin{pmatrix}
0.715306\\
0.159852\\
\color{red} 0.079861\color{black} \\
0.044981
\end{pmatrix}\end{equation*}
\begin{equation*}
\left[ \frac{{w}^{EM}_i}{{w}^{EM}_j} \right] =
\begin{pmatrix}
$\,\,$ 1 $\,\,$ & $\,\,$4.4748$\,\,$ & $\,\,$\color{red} 8.9569\color{black} $\,\,$ & $\,\,$15.9022$\,\,$ \\
$\,\,$0.2235$\,\,$ & $\,\,$ 1 $\,\,$ & $\,\,$\color{red} 2.0016\color{black} $\,\,$ & $\,\,$3.5537  $\,\,$ \\
$\,\,$\color{red} 0.1116\color{black} $\,\,$ & $\,\,$\color{red} 0.4996\color{black} $\,\,$ & $\,\,$ 1 $\,\,$ & $\,\,$\color{red} 1.7754\color{black}  $\,\,$ \\
$\,\,$0.0629$\,\,$ & $\,\,$0.2814$\,\,$ & $\,\,$\color{red} 0.5632\color{black} $\,\,$ & $\,\,$ 1  $\,\,$ \\
\end{pmatrix},
\end{equation*}

\begin{equation*}
\mathbf{w}^{\prime} =
\begin{pmatrix}
0.715259\\
0.159841\\
0.079921\\
0.044979
\end{pmatrix} =
0.999935\cdot
\begin{pmatrix}
0.715306\\
0.159852\\
\color{gr} 0.079926\color{black} \\
0.044981
\end{pmatrix},
\end{equation*}
\begin{equation*}
\left[ \frac{{w}^{\prime}_i}{{w}^{\prime}_j} \right] =
\begin{pmatrix}
$\,\,$ 1 $\,\,$ & $\,\,$4.4748$\,\,$ & $\,\,$\color{gr} 8.9496\color{black} $\,\,$ & $\,\,$15.9022$\,\,$ \\
$\,\,$0.2235$\,\,$ & $\,\,$ 1 $\,\,$ & $\,\,$\color{gr} \color{blue} 2\color{black} $\,\,$ & $\,\,$3.5537  $\,\,$ \\
$\,\,$\color{gr} 0.1117\color{black} $\,\,$ & $\,\,$\color{gr} \color{blue}  1/2\color{black} $\,\,$ & $\,\,$ 1 $\,\,$ & $\,\,$\color{gr} 1.7769\color{black}  $\,\,$ \\
$\,\,$0.0629$\,\,$ & $\,\,$0.2814$\,\,$ & $\,\,$\color{gr} 0.5628\color{black} $\,\,$ & $\,\,$ 1  $\,\,$ \\
\end{pmatrix},
\end{equation*}
\end{example}
\newpage
\begin{example}
\begin{equation*}
\mathbf{A} =
\begin{pmatrix}
$\,\,$ 1 $\,\,$ & $\,\,$8$\,\,$ & $\,\,$9$\,\,$ & $\,\,$9 $\,\,$ \\
$\,\,$ 1/8$\,\,$ & $\,\,$ 1 $\,\,$ & $\,\,$2$\,\,$ & $\,\,$6 $\,\,$ \\
$\,\,$ 1/9$\,\,$ & $\,\,$ 1/2$\,\,$ & $\,\,$ 1 $\,\,$ & $\,\,$2 $\,\,$ \\
$\,\,$ 1/9$\,\,$ & $\,\,$ 1/6$\,\,$ & $\,\,$ 1/2$\,\,$ & $\,\,$ 1  $\,\,$ \\
\end{pmatrix},
\qquad
\lambda_{\max} =
4.2469,
\qquad
CR = 0.0931
\end{equation*}

\begin{equation*}
\mathbf{w}^{EM} =
\begin{pmatrix}
0.722052\\
0.157094\\
\color{red} 0.076327\color{black} \\
0.044527
\end{pmatrix}\end{equation*}
\begin{equation*}
\left[ \frac{{w}^{EM}_i}{{w}^{EM}_j} \right] =
\begin{pmatrix}
$\,\,$ 1 $\,\,$ & $\,\,$4.5963$\,\,$ & $\,\,$\color{red} 9.4599\color{black} $\,\,$ & $\,\,$16.2162$\,\,$ \\
$\,\,$0.2176$\,\,$ & $\,\,$ 1 $\,\,$ & $\,\,$\color{red} 2.0582\color{black} $\,\,$ & $\,\,$3.5281  $\,\,$ \\
$\,\,$\color{red} 0.1057\color{black} $\,\,$ & $\,\,$\color{red} 0.4859\color{black} $\,\,$ & $\,\,$ 1 $\,\,$ & $\,\,$\color{red} 1.7142\color{black}  $\,\,$ \\
$\,\,$0.0617$\,\,$ & $\,\,$0.2834$\,\,$ & $\,\,$\color{red} 0.5834\color{black} $\,\,$ & $\,\,$ 1  $\,\,$ \\
\end{pmatrix},
\end{equation*}

\begin{equation*}
\mathbf{w}^{\prime} =
\begin{pmatrix}
0.720453\\
0.156746\\
0.078373\\
0.044428
\end{pmatrix} =
0.997785\cdot
\begin{pmatrix}
0.722052\\
0.157094\\
\color{gr} 0.078547\color{black} \\
0.044527
\end{pmatrix},
\end{equation*}
\begin{equation*}
\left[ \frac{{w}^{\prime}_i}{{w}^{\prime}_j} \right] =
\begin{pmatrix}
$\,\,$ 1 $\,\,$ & $\,\,$4.5963$\,\,$ & $\,\,$\color{gr} 9.1926\color{black} $\,\,$ & $\,\,$16.2162$\,\,$ \\
$\,\,$0.2176$\,\,$ & $\,\,$ 1 $\,\,$ & $\,\,$\color{gr} \color{blue} 2\color{black} $\,\,$ & $\,\,$3.5281  $\,\,$ \\
$\,\,$\color{gr} 0.1088\color{black} $\,\,$ & $\,\,$\color{gr} \color{blue}  1/2\color{black} $\,\,$ & $\,\,$ 1 $\,\,$ & $\,\,$\color{gr} 1.7640\color{black}  $\,\,$ \\
$\,\,$0.0617$\,\,$ & $\,\,$0.2834$\,\,$ & $\,\,$\color{gr} 0.5669\color{black} $\,\,$ & $\,\,$ 1  $\,\,$ \\
\end{pmatrix},
\end{equation*}
\end{example}
\newpage
\begin{example}
\begin{equation*}
\mathbf{A} =
\begin{pmatrix}
$\,\,$ 1 $\,\,$ & $\,\,$9$\,\,$ & $\,\,$5$\,\,$ & $\,\,$5 $\,\,$ \\
$\,\,$ 1/9$\,\,$ & $\,\,$ 1 $\,\,$ & $\,\,$1$\,\,$ & $\,\,$3 $\,\,$ \\
$\,\,$ 1/5$\,\,$ & $\,\,$ 1 $\,\,$ & $\,\,$ 1 $\,\,$ & $\,\,$2 $\,\,$ \\
$\,\,$ 1/5$\,\,$ & $\,\,$ 1/3$\,\,$ & $\,\,$ 1/2$\,\,$ & $\,\,$ 1  $\,\,$ \\
\end{pmatrix},
\qquad
\lambda_{\max} =
4.2507,
\qquad
CR = 0.0946
\end{equation*}

\begin{equation*}
\mathbf{w}^{EM} =
\begin{pmatrix}
0.669105\\
0.130154\\
\color{red} 0.126736\color{black} \\
0.074006
\end{pmatrix}\end{equation*}
\begin{equation*}
\left[ \frac{{w}^{EM}_i}{{w}^{EM}_j} \right] =
\begin{pmatrix}
$\,\,$ 1 $\,\,$ & $\,\,$5.1409$\,\,$ & $\,\,$\color{red} 5.2795\color{black} $\,\,$ & $\,\,$9.0413$\,\,$ \\
$\,\,$0.1945$\,\,$ & $\,\,$ 1 $\,\,$ & $\,\,$\color{red} 1.0270\color{black} $\,\,$ & $\,\,$1.7587  $\,\,$ \\
$\,\,$\color{red} 0.1894\color{black} $\,\,$ & $\,\,$\color{red} 0.9737\color{black} $\,\,$ & $\,\,$ 1 $\,\,$ & $\,\,$\color{red} 1.7125\color{black}  $\,\,$ \\
$\,\,$0.1106$\,\,$ & $\,\,$0.5686$\,\,$ & $\,\,$\color{red} 0.5839\color{black} $\,\,$ & $\,\,$ 1  $\,\,$ \\
\end{pmatrix},
\end{equation*}

\begin{equation*}
\mathbf{w}^{\prime} =
\begin{pmatrix}
0.666826\\
0.129710\\
0.129710\\
0.073753
\end{pmatrix} =
0.996594\cdot
\begin{pmatrix}
0.669105\\
0.130154\\
\color{gr} 0.130154\color{black} \\
0.074006
\end{pmatrix},
\end{equation*}
\begin{equation*}
\left[ \frac{{w}^{\prime}_i}{{w}^{\prime}_j} \right] =
\begin{pmatrix}
$\,\,$ 1 $\,\,$ & $\,\,$5.1409$\,\,$ & $\,\,$\color{gr} 5.1409\color{black} $\,\,$ & $\,\,$9.0413$\,\,$ \\
$\,\,$0.1945$\,\,$ & $\,\,$ 1 $\,\,$ & $\,\,$\color{gr} \color{blue} 1\color{black} $\,\,$ & $\,\,$1.7587  $\,\,$ \\
$\,\,$\color{gr} 0.1945\color{black} $\,\,$ & $\,\,$\color{gr} \color{blue} 1\color{black} $\,\,$ & $\,\,$ 1 $\,\,$ & $\,\,$\color{gr} 1.7587\color{black}  $\,\,$ \\
$\,\,$0.1106$\,\,$ & $\,\,$0.5686$\,\,$ & $\,\,$\color{gr} 0.5686\color{black} $\,\,$ & $\,\,$ 1  $\,\,$ \\
\end{pmatrix},
\end{equation*}
\end{example}
\newpage
\begin{example}
\begin{equation*}
\mathbf{A} =
\begin{pmatrix}
$\,\,$ 1 $\,\,$ & $\,\,$9$\,\,$ & $\,\,$5$\,\,$ & $\,\,$6 $\,\,$ \\
$\,\,$ 1/9$\,\,$ & $\,\,$ 1 $\,\,$ & $\,\,$1$\,\,$ & $\,\,$3 $\,\,$ \\
$\,\,$ 1/5$\,\,$ & $\,\,$ 1 $\,\,$ & $\,\,$ 1 $\,\,$ & $\,\,$2 $\,\,$ \\
$\,\,$ 1/6$\,\,$ & $\,\,$ 1/3$\,\,$ & $\,\,$ 1/2$\,\,$ & $\,\,$ 1  $\,\,$ \\
\end{pmatrix},
\qquad
\lambda_{\max} =
4.1966,
\qquad
CR = 0.0741
\end{equation*}

\begin{equation*}
\mathbf{w}^{EM} =
\begin{pmatrix}
0.680178\\
0.126728\\
\color{red} 0.124881\color{black} \\
0.068213
\end{pmatrix}\end{equation*}
\begin{equation*}
\left[ \frac{{w}^{EM}_i}{{w}^{EM}_j} \right] =
\begin{pmatrix}
$\,\,$ 1 $\,\,$ & $\,\,$5.3672$\,\,$ & $\,\,$\color{red} 5.4466\color{black} $\,\,$ & $\,\,$9.9714$\,\,$ \\
$\,\,$0.1863$\,\,$ & $\,\,$ 1 $\,\,$ & $\,\,$\color{red} 1.0148\color{black} $\,\,$ & $\,\,$1.8578  $\,\,$ \\
$\,\,$\color{red} 0.1836\color{black} $\,\,$ & $\,\,$\color{red} 0.9854\color{black} $\,\,$ & $\,\,$ 1 $\,\,$ & $\,\,$\color{red} 1.8308\color{black}  $\,\,$ \\
$\,\,$0.1003$\,\,$ & $\,\,$0.5383$\,\,$ & $\,\,$\color{red} 0.5462\color{black} $\,\,$ & $\,\,$ 1  $\,\,$ \\
\end{pmatrix},
\end{equation*}

\begin{equation*}
\mathbf{w}^{\prime} =
\begin{pmatrix}
0.678924\\
0.126494\\
0.126494\\
0.068087
\end{pmatrix} =
0.998156\cdot
\begin{pmatrix}
0.680178\\
0.126728\\
\color{gr} 0.126728\color{black} \\
0.068213
\end{pmatrix},
\end{equation*}
\begin{equation*}
\left[ \frac{{w}^{\prime}_i}{{w}^{\prime}_j} \right] =
\begin{pmatrix}
$\,\,$ 1 $\,\,$ & $\,\,$5.3672$\,\,$ & $\,\,$\color{gr} 5.3672\color{black} $\,\,$ & $\,\,$9.9714$\,\,$ \\
$\,\,$0.1863$\,\,$ & $\,\,$ 1 $\,\,$ & $\,\,$\color{gr} \color{blue} 1\color{black} $\,\,$ & $\,\,$1.8578  $\,\,$ \\
$\,\,$\color{gr} 0.1863\color{black} $\,\,$ & $\,\,$\color{gr} \color{blue} 1\color{black} $\,\,$ & $\,\,$ 1 $\,\,$ & $\,\,$\color{gr} 1.8578\color{black}  $\,\,$ \\
$\,\,$0.1003$\,\,$ & $\,\,$0.5383$\,\,$ & $\,\,$\color{gr} 0.5383\color{black} $\,\,$ & $\,\,$ 1  $\,\,$ \\
\end{pmatrix},
\end{equation*}
\end{example}
\newpage
\begin{example}
\begin{equation*}
\mathbf{A} =
\begin{pmatrix}
$\,\,$ 1 $\,\,$ & $\,\,$9$\,\,$ & $\,\,$5$\,\,$ & $\,\,$7 $\,\,$ \\
$\,\,$ 1/9$\,\,$ & $\,\,$ 1 $\,\,$ & $\,\,$1$\,\,$ & $\,\,$3 $\,\,$ \\
$\,\,$ 1/5$\,\,$ & $\,\,$ 1 $\,\,$ & $\,\,$ 1 $\,\,$ & $\,\,$2 $\,\,$ \\
$\,\,$ 1/7$\,\,$ & $\,\,$ 1/3$\,\,$ & $\,\,$ 1/2$\,\,$ & $\,\,$ 1  $\,\,$ \\
\end{pmatrix},
\qquad
\lambda_{\max} =
4.1583,
\qquad
CR = 0.0597
\end{equation*}

\begin{equation*}
\mathbf{w}^{EM} =
\begin{pmatrix}
0.689212\\
0.123818\\
\color{red} 0.123220\color{black} \\
0.063751
\end{pmatrix}\end{equation*}
\begin{equation*}
\left[ \frac{{w}^{EM}_i}{{w}^{EM}_j} \right] =
\begin{pmatrix}
$\,\,$ 1 $\,\,$ & $\,\,$5.5663$\,\,$ & $\,\,$\color{red} 5.5933\color{black} $\,\,$ & $\,\,$10.8111$\,\,$ \\
$\,\,$0.1797$\,\,$ & $\,\,$ 1 $\,\,$ & $\,\,$\color{red} 1.0049\color{black} $\,\,$ & $\,\,$1.9422  $\,\,$ \\
$\,\,$\color{red} 0.1788\color{black} $\,\,$ & $\,\,$\color{red} 0.9952\color{black} $\,\,$ & $\,\,$ 1 $\,\,$ & $\,\,$\color{red} 1.9328\color{black}  $\,\,$ \\
$\,\,$0.0925$\,\,$ & $\,\,$0.5149$\,\,$ & $\,\,$\color{red} 0.5174\color{black} $\,\,$ & $\,\,$ 1  $\,\,$ \\
\end{pmatrix},
\end{equation*}

\begin{equation*}
\mathbf{w}^{\prime} =
\begin{pmatrix}
0.688800\\
0.123744\\
0.123744\\
0.063712
\end{pmatrix} =
0.999402\cdot
\begin{pmatrix}
0.689212\\
0.123818\\
\color{gr} 0.123818\color{black} \\
0.063751
\end{pmatrix},
\end{equation*}
\begin{equation*}
\left[ \frac{{w}^{\prime}_i}{{w}^{\prime}_j} \right] =
\begin{pmatrix}
$\,\,$ 1 $\,\,$ & $\,\,$5.5663$\,\,$ & $\,\,$\color{gr} 5.5663\color{black} $\,\,$ & $\,\,$10.8111$\,\,$ \\
$\,\,$0.1797$\,\,$ & $\,\,$ 1 $\,\,$ & $\,\,$\color{gr} \color{blue} 1\color{black} $\,\,$ & $\,\,$1.9422  $\,\,$ \\
$\,\,$\color{gr} 0.1797\color{black} $\,\,$ & $\,\,$\color{gr} \color{blue} 1\color{black} $\,\,$ & $\,\,$ 1 $\,\,$ & $\,\,$\color{gr} 1.9422\color{black}  $\,\,$ \\
$\,\,$0.0925$\,\,$ & $\,\,$0.5149$\,\,$ & $\,\,$\color{gr} 0.5149\color{black} $\,\,$ & $\,\,$ 1  $\,\,$ \\
\end{pmatrix},
\end{equation*}
\end{example}
\newpage
\begin{example}
\begin{equation*}
\mathbf{A} =
\begin{pmatrix}
$\,\,$ 1 $\,\,$ & $\,\,$9$\,\,$ & $\,\,$5$\,\,$ & $\,\,$8 $\,\,$ \\
$\,\,$ 1/9$\,\,$ & $\,\,$ 1 $\,\,$ & $\,\,$1$\,\,$ & $\,\,$5 $\,\,$ \\
$\,\,$ 1/5$\,\,$ & $\,\,$ 1 $\,\,$ & $\,\,$ 1 $\,\,$ & $\,\,$3 $\,\,$ \\
$\,\,$ 1/8$\,\,$ & $\,\,$ 1/5$\,\,$ & $\,\,$ 1/3$\,\,$ & $\,\,$ 1  $\,\,$ \\
\end{pmatrix},
\qquad
\lambda_{\max} =
4.2612,
\qquad
CR = 0.0985
\end{equation*}

\begin{equation*}
\mathbf{w}^{EM} =
\begin{pmatrix}
0.688410\\
0.135934\\
\color{red} 0.127864\color{black} \\
0.047792
\end{pmatrix}\end{equation*}
\begin{equation*}
\left[ \frac{{w}^{EM}_i}{{w}^{EM}_j} \right] =
\begin{pmatrix}
$\,\,$ 1 $\,\,$ & $\,\,$5.0643$\,\,$ & $\,\,$\color{red} 5.3839\color{black} $\,\,$ & $\,\,$14.4044$\,\,$ \\
$\,\,$0.1975$\,\,$ & $\,\,$ 1 $\,\,$ & $\,\,$\color{red} 1.0631\color{black} $\,\,$ & $\,\,$2.8443  $\,\,$ \\
$\,\,$\color{red} 0.1857\color{black} $\,\,$ & $\,\,$\color{red} 0.9406\color{black} $\,\,$ & $\,\,$ 1 $\,\,$ & $\,\,$\color{red} 2.6754\color{black}  $\,\,$ \\
$\,\,$0.0694$\,\,$ & $\,\,$0.3516$\,\,$ & $\,\,$\color{red} 0.3738\color{black} $\,\,$ & $\,\,$ 1  $\,\,$ \\
\end{pmatrix},
\end{equation*}

\begin{equation*}
\mathbf{w}^{\prime} =
\begin{pmatrix}
0.682899\\
0.134846\\
0.134846\\
0.047409
\end{pmatrix} =
0.991994\cdot
\begin{pmatrix}
0.688410\\
0.135934\\
\color{gr} 0.135934\color{black} \\
0.047792
\end{pmatrix},
\end{equation*}
\begin{equation*}
\left[ \frac{{w}^{\prime}_i}{{w}^{\prime}_j} \right] =
\begin{pmatrix}
$\,\,$ 1 $\,\,$ & $\,\,$5.0643$\,\,$ & $\,\,$\color{gr} 5.0643\color{black} $\,\,$ & $\,\,$14.4044$\,\,$ \\
$\,\,$0.1975$\,\,$ & $\,\,$ 1 $\,\,$ & $\,\,$\color{gr} \color{blue} 1\color{black} $\,\,$ & $\,\,$2.8443  $\,\,$ \\
$\,\,$\color{gr} 0.1975\color{black} $\,\,$ & $\,\,$\color{gr} \color{blue} 1\color{black} $\,\,$ & $\,\,$ 1 $\,\,$ & $\,\,$\color{gr} 2.8443\color{black}  $\,\,$ \\
$\,\,$0.0694$\,\,$ & $\,\,$0.3516$\,\,$ & $\,\,$\color{gr} 0.3516\color{black} $\,\,$ & $\,\,$ 1  $\,\,$ \\
\end{pmatrix},
\end{equation*}
\end{example}
\newpage
\begin{example}
\begin{equation*}
\mathbf{A} =
\begin{pmatrix}
$\,\,$ 1 $\,\,$ & $\,\,$9$\,\,$ & $\,\,$5$\,\,$ & $\,\,$9 $\,\,$ \\
$\,\,$ 1/9$\,\,$ & $\,\,$ 1 $\,\,$ & $\,\,$1$\,\,$ & $\,\,$5 $\,\,$ \\
$\,\,$ 1/5$\,\,$ & $\,\,$ 1 $\,\,$ & $\,\,$ 1 $\,\,$ & $\,\,$3 $\,\,$ \\
$\,\,$ 1/9$\,\,$ & $\,\,$ 1/5$\,\,$ & $\,\,$ 1/3$\,\,$ & $\,\,$ 1  $\,\,$ \\
\end{pmatrix},
\qquad
\lambda_{\max} =
4.2253,
\qquad
CR = 0.0849
\end{equation*}

\begin{equation*}
\mathbf{w}^{EM} =
\begin{pmatrix}
0.694771\\
0.133381\\
\color{red} 0.126562\color{black} \\
0.045286
\end{pmatrix}\end{equation*}
\begin{equation*}
\left[ \frac{{w}^{EM}_i}{{w}^{EM}_j} \right] =
\begin{pmatrix}
$\,\,$ 1 $\,\,$ & $\,\,$5.2089$\,\,$ & $\,\,$\color{red} 5.4896\color{black} $\,\,$ & $\,\,$15.3417$\,\,$ \\
$\,\,$0.1920$\,\,$ & $\,\,$ 1 $\,\,$ & $\,\,$\color{red} 1.0539\color{black} $\,\,$ & $\,\,$2.9453  $\,\,$ \\
$\,\,$\color{red} 0.1822\color{black} $\,\,$ & $\,\,$\color{red} 0.9489\color{black} $\,\,$ & $\,\,$ 1 $\,\,$ & $\,\,$\color{red} 2.7947\color{black}  $\,\,$ \\
$\,\,$0.0652$\,\,$ & $\,\,$0.3395$\,\,$ & $\,\,$\color{red} 0.3578\color{black} $\,\,$ & $\,\,$ 1  $\,\,$ \\
\end{pmatrix},
\end{equation*}

\begin{equation*}
\mathbf{w}^{\prime} =
\begin{pmatrix}
0.690065\\
0.132478\\
0.132478\\
0.044980
\end{pmatrix} =
0.993226\cdot
\begin{pmatrix}
0.694771\\
0.133381\\
\color{gr} 0.133381\color{black} \\
0.045286
\end{pmatrix},
\end{equation*}
\begin{equation*}
\left[ \frac{{w}^{\prime}_i}{{w}^{\prime}_j} \right] =
\begin{pmatrix}
$\,\,$ 1 $\,\,$ & $\,\,$5.2089$\,\,$ & $\,\,$\color{gr} 5.2089\color{black} $\,\,$ & $\,\,$15.3417$\,\,$ \\
$\,\,$0.1920$\,\,$ & $\,\,$ 1 $\,\,$ & $\,\,$\color{gr} \color{blue} 1\color{black} $\,\,$ & $\,\,$2.9453  $\,\,$ \\
$\,\,$\color{gr} 0.1920\color{black} $\,\,$ & $\,\,$\color{gr} \color{blue} 1\color{black} $\,\,$ & $\,\,$ 1 $\,\,$ & $\,\,$\color{gr} 2.9453\color{black}  $\,\,$ \\
$\,\,$0.0652$\,\,$ & $\,\,$0.3395$\,\,$ & $\,\,$\color{gr} 0.3395\color{black} $\,\,$ & $\,\,$ 1  $\,\,$ \\
\end{pmatrix},
\end{equation*}
\end{example}
\newpage
\begin{example}
\begin{equation*}
\mathbf{A} =
\begin{pmatrix}
$\,\,$ 1 $\,\,$ & $\,\,$9$\,\,$ & $\,\,$5$\,\,$ & $\,\,$9 $\,\,$ \\
$\,\,$ 1/9$\,\,$ & $\,\,$ 1 $\,\,$ & $\,\,$3$\,\,$ & $\,\,$2 $\,\,$ \\
$\,\,$ 1/5$\,\,$ & $\,\,$ 1/3$\,\,$ & $\,\,$ 1 $\,\,$ & $\,\,$1 $\,\,$ \\
$\,\,$ 1/9$\,\,$ & $\,\,$ 1/2$\,\,$ & $\,\,$ 1 $\,\,$ & $\,\,$ 1  $\,\,$ \\
\end{pmatrix},
\qquad
\lambda_{\max} =
4.2507,
\qquad
CR = 0.0946
\end{equation*}

\begin{equation*}
\mathbf{w}^{EM} =
\begin{pmatrix}
0.708610\\
0.141173\\
0.079750\\
\color{red} 0.070467\color{black}
\end{pmatrix}\end{equation*}
\begin{equation*}
\left[ \frac{{w}^{EM}_i}{{w}^{EM}_j} \right] =
\begin{pmatrix}
$\,\,$ 1 $\,\,$ & $\,\,$5.0194$\,\,$ & $\,\,$8.8854$\,\,$ & $\,\,$\color{red} 10.0559\color{black} $\,\,$ \\
$\,\,$0.1992$\,\,$ & $\,\,$ 1 $\,\,$ & $\,\,$1.7702$\,\,$ & $\,\,$\color{red} 2.0034\color{black}   $\,\,$ \\
$\,\,$0.1125$\,\,$ & $\,\,$0.5649$\,\,$ & $\,\,$ 1 $\,\,$ & $\,\,$\color{red} 1.1317\color{black}  $\,\,$ \\
$\,\,$\color{red} 0.0994\color{black} $\,\,$ & $\,\,$\color{red} 0.4992\color{black} $\,\,$ & $\,\,$\color{red} 0.8836\color{black} $\,\,$ & $\,\,$ 1  $\,\,$ \\
\end{pmatrix},
\end{equation*}

\begin{equation*}
\mathbf{w}^{\prime} =
\begin{pmatrix}
0.708525\\
0.141156\\
0.079740\\
0.070578
\end{pmatrix} =
0.999881\cdot
\begin{pmatrix}
0.708610\\
0.141173\\
0.079750\\
\color{gr} 0.070587\color{black}
\end{pmatrix},
\end{equation*}
\begin{equation*}
\left[ \frac{{w}^{\prime}_i}{{w}^{\prime}_j} \right] =
\begin{pmatrix}
$\,\,$ 1 $\,\,$ & $\,\,$5.0194$\,\,$ & $\,\,$8.8854$\,\,$ & $\,\,$\color{gr} 10.0389\color{black} $\,\,$ \\
$\,\,$0.1992$\,\,$ & $\,\,$ 1 $\,\,$ & $\,\,$1.7702$\,\,$ & $\,\,$\color{gr} \color{blue} 2\color{black}   $\,\,$ \\
$\,\,$0.1125$\,\,$ & $\,\,$0.5649$\,\,$ & $\,\,$ 1 $\,\,$ & $\,\,$\color{gr} 1.1298\color{black}  $\,\,$ \\
$\,\,$\color{gr} 0.0996\color{black} $\,\,$ & $\,\,$\color{gr} \color{blue}  1/2\color{black} $\,\,$ & $\,\,$\color{gr} 0.8851\color{black} $\,\,$ & $\,\,$ 1  $\,\,$ \\
\end{pmatrix},
\end{equation*}
\end{example}
\newpage
\begin{example}
\begin{equation*}
\mathbf{A} =
\begin{pmatrix}
$\,\,$ 1 $\,\,$ & $\,\,$9$\,\,$ & $\,\,$6$\,\,$ & $\,\,$7 $\,\,$ \\
$\,\,$ 1/9$\,\,$ & $\,\,$ 1 $\,\,$ & $\,\,$1$\,\,$ & $\,\,$3 $\,\,$ \\
$\,\,$ 1/6$\,\,$ & $\,\,$ 1 $\,\,$ & $\,\,$ 1 $\,\,$ & $\,\,$2 $\,\,$ \\
$\,\,$ 1/7$\,\,$ & $\,\,$ 1/3$\,\,$ & $\,\,$ 1/2$\,\,$ & $\,\,$ 1  $\,\,$ \\
\end{pmatrix},
\qquad
\lambda_{\max} =
4.1571,
\qquad
CR = 0.0593
\end{equation*}

\begin{equation*}
\mathbf{w}^{EM} =
\begin{pmatrix}
0.701615\\
0.120695\\
\color{red} 0.114988\color{black} \\
0.062701
\end{pmatrix}\end{equation*}
\begin{equation*}
\left[ \frac{{w}^{EM}_i}{{w}^{EM}_j} \right] =
\begin{pmatrix}
$\,\,$ 1 $\,\,$ & $\,\,$5.8131$\,\,$ & $\,\,$\color{red} 6.1016\color{black} $\,\,$ & $\,\,$11.1898$\,\,$ \\
$\,\,$0.1720$\,\,$ & $\,\,$ 1 $\,\,$ & $\,\,$\color{red} 1.0496\color{black} $\,\,$ & $\,\,$1.9249  $\,\,$ \\
$\,\,$\color{red} 0.1639\color{black} $\,\,$ & $\,\,$\color{red} 0.9527\color{black} $\,\,$ & $\,\,$ 1 $\,\,$ & $\,\,$\color{red} 1.8339\color{black}  $\,\,$ \\
$\,\,$0.0894$\,\,$ & $\,\,$0.5195$\,\,$ & $\,\,$\color{red} 0.5453\color{black} $\,\,$ & $\,\,$ 1  $\,\,$ \\
\end{pmatrix},
\end{equation*}

\begin{equation*}
\mathbf{w}^{\prime} =
\begin{pmatrix}
0.700252\\
0.120460\\
0.116709\\
0.062579
\end{pmatrix} =
0.998056\cdot
\begin{pmatrix}
0.701615\\
0.120695\\
\color{gr} 0.116936\color{black} \\
0.062701
\end{pmatrix},
\end{equation*}
\begin{equation*}
\left[ \frac{{w}^{\prime}_i}{{w}^{\prime}_j} \right] =
\begin{pmatrix}
$\,\,$ 1 $\,\,$ & $\,\,$5.8131$\,\,$ & $\,\,$\color{gr} \color{blue} 6\color{black} $\,\,$ & $\,\,$11.1898$\,\,$ \\
$\,\,$0.1720$\,\,$ & $\,\,$ 1 $\,\,$ & $\,\,$\color{gr} 1.0321\color{black} $\,\,$ & $\,\,$1.9249  $\,\,$ \\
$\,\,$\color{gr} \color{blue}  1/6\color{black} $\,\,$ & $\,\,$\color{gr} 0.9689\color{black} $\,\,$ & $\,\,$ 1 $\,\,$ & $\,\,$\color{gr} 1.8650\color{black}  $\,\,$ \\
$\,\,$0.0894$\,\,$ & $\,\,$0.5195$\,\,$ & $\,\,$\color{gr} 0.5362\color{black} $\,\,$ & $\,\,$ 1  $\,\,$ \\
\end{pmatrix},
\end{equation*}
\end{example}
\newpage
\begin{example}
\begin{equation*}
\mathbf{A} =
\begin{pmatrix}
$\,\,$ 1 $\,\,$ & $\,\,$9$\,\,$ & $\,\,$6$\,\,$ & $\,\,$7 $\,\,$ \\
$\,\,$ 1/9$\,\,$ & $\,\,$ 1 $\,\,$ & $\,\,$1$\,\,$ & $\,\,$4 $\,\,$ \\
$\,\,$ 1/6$\,\,$ & $\,\,$ 1 $\,\,$ & $\,\,$ 1 $\,\,$ & $\,\,$2 $\,\,$ \\
$\,\,$ 1/7$\,\,$ & $\,\,$ 1/4$\,\,$ & $\,\,$ 1/2$\,\,$ & $\,\,$ 1  $\,\,$ \\
\end{pmatrix},
\qquad
\lambda_{\max} =
4.2359,
\qquad
CR = 0.0890
\end{equation*}

\begin{equation*}
\mathbf{w}^{EM} =
\begin{pmatrix}
0.698439\\
0.130810\\
\color{red} 0.112438\color{black} \\
0.058313
\end{pmatrix}\end{equation*}
\begin{equation*}
\left[ \frac{{w}^{EM}_i}{{w}^{EM}_j} \right] =
\begin{pmatrix}
$\,\,$ 1 $\,\,$ & $\,\,$5.3393$\,\,$ & $\,\,$\color{red} 6.2118\color{black} $\,\,$ & $\,\,$11.9774$\,\,$ \\
$\,\,$0.1873$\,\,$ & $\,\,$ 1 $\,\,$ & $\,\,$\color{red} 1.1634\color{black} $\,\,$ & $\,\,$2.2432  $\,\,$ \\
$\,\,$\color{red} 0.1610\color{black} $\,\,$ & $\,\,$\color{red} 0.8595\color{black} $\,\,$ & $\,\,$ 1 $\,\,$ & $\,\,$\color{red} 1.9282\color{black}  $\,\,$ \\
$\,\,$0.0835$\,\,$ & $\,\,$0.4458$\,\,$ & $\,\,$\color{red} 0.5186\color{black} $\,\,$ & $\,\,$ 1  $\,\,$ \\
\end{pmatrix},
\end{equation*}

\begin{equation*}
\mathbf{w}^{\prime} =
\begin{pmatrix}
0.695678\\
0.130293\\
0.115946\\
0.058083
\end{pmatrix} =
0.996047\cdot
\begin{pmatrix}
0.698439\\
0.130810\\
\color{gr} 0.116406\color{black} \\
0.058313
\end{pmatrix},
\end{equation*}
\begin{equation*}
\left[ \frac{{w}^{\prime}_i}{{w}^{\prime}_j} \right] =
\begin{pmatrix}
$\,\,$ 1 $\,\,$ & $\,\,$5.3393$\,\,$ & $\,\,$\color{gr} \color{blue} 6\color{black} $\,\,$ & $\,\,$11.9774$\,\,$ \\
$\,\,$0.1873$\,\,$ & $\,\,$ 1 $\,\,$ & $\,\,$\color{gr} 1.1237\color{black} $\,\,$ & $\,\,$2.2432  $\,\,$ \\
$\,\,$\color{gr} \color{blue}  1/6\color{black} $\,\,$ & $\,\,$\color{gr} 0.8899\color{black} $\,\,$ & $\,\,$ 1 $\,\,$ & $\,\,$\color{gr} 1.9962\color{black}  $\,\,$ \\
$\,\,$0.0835$\,\,$ & $\,\,$0.4458$\,\,$ & $\,\,$\color{gr} 0.5009\color{black} $\,\,$ & $\,\,$ 1  $\,\,$ \\
\end{pmatrix},
\end{equation*}
\end{example}
\newpage
\begin{example}
\begin{equation*}
\mathbf{A} =
\begin{pmatrix}
$\,\,$ 1 $\,\,$ & $\,\,$9$\,\,$ & $\,\,$6$\,\,$ & $\,\,$8 $\,\,$ \\
$\,\,$ 1/9$\,\,$ & $\,\,$ 1 $\,\,$ & $\,\,$1$\,\,$ & $\,\,$3 $\,\,$ \\
$\,\,$ 1/6$\,\,$ & $\,\,$ 1 $\,\,$ & $\,\,$ 1 $\,\,$ & $\,\,$2 $\,\,$ \\
$\,\,$ 1/8$\,\,$ & $\,\,$ 1/3$\,\,$ & $\,\,$ 1/2$\,\,$ & $\,\,$ 1  $\,\,$ \\
\end{pmatrix},
\qquad
\lambda_{\max} =
4.1263,
\qquad
CR = 0.0476
\end{equation*}

\begin{equation*}
\mathbf{w}^{EM} =
\begin{pmatrix}
0.709253\\
0.118209\\
\color{red} 0.113434\color{black} \\
0.059104
\end{pmatrix}\end{equation*}
\begin{equation*}
\left[ \frac{{w}^{EM}_i}{{w}^{EM}_j} \right] =
\begin{pmatrix}
$\,\,$ 1 $\,\,$ & $\,\,$6$\,\,$ & $\,\,$\color{red} 6.2526\color{black} $\,\,$ & $\,\,$12$\,\,$ \\
$\,\,$1/6$\,\,$ & $\,\,$ 1 $\,\,$ & $\,\,$\color{red} 1.0421\color{black} $\,\,$ & $\,\,$2  $\,\,$ \\
$\,\,$\color{red} 0.1599\color{black} $\,\,$ & $\,\,$\color{red} 0.9596\color{black} $\,\,$ & $\,\,$ 1 $\,\,$ & $\,\,$\color{red} 1.9192\color{black}  $\,\,$ \\
$\,\,$1/12$\,\,$ & $\,\,$1/2$\,\,$ & $\,\,$\color{red} 0.5210\color{black} $\,\,$ & $\,\,$ 1  $\,\,$ \\
\end{pmatrix},
\end{equation*}

\begin{equation*}
\mathbf{w}^{\prime} =
\begin{pmatrix}
0.705882\\
0.117647\\
0.117647\\
0.058824
\end{pmatrix} =
0.995248\cdot
\begin{pmatrix}
0.709253\\
0.118209\\
\color{gr} 0.118209\color{black} \\
0.059104
\end{pmatrix},
\end{equation*}
\begin{equation*}
\left[ \frac{{w}^{\prime}_i}{{w}^{\prime}_j} \right] =
\begin{pmatrix}
$\,\,$ 1 $\,\,$ & $\,\,$6$\,\,$ & $\,\,$\color{blue} 6\color{black} $\,\,$ & $\,\,$12$\,\,$ \\
$\,\,$1/6$\,\,$ & $\,\,$ 1 $\,\,$ & $\,\,$\color{blue} 1\color{black} $\,\,$ & $\,\,$2  $\,\,$ \\
$\,\,$\color{blue} 1/6\color{black} $\,\,$ & $\,\,$\color{blue} 1\color{black} $\,\,$ & $\,\,$ 1 $\,\,$ & $\,\,$\color{gr} \color{blue} 2\color{black}  $\,\,$ \\
$\,\,$1/12$\,\,$ & $\,\,$1/2$\,\,$ & $\,\,$\color{gr} \color{blue}  1/2\color{black} $\,\,$ & $\,\,$ 1  $\,\,$ \\
\end{pmatrix},
\end{equation*}
\end{example}
\newpage
\begin{example}
\begin{equation*}
\mathbf{A} =
\begin{pmatrix}
$\,\,$ 1 $\,\,$ & $\,\,$9$\,\,$ & $\,\,$6$\,\,$ & $\,\,$9 $\,\,$ \\
$\,\,$ 1/9$\,\,$ & $\,\,$ 1 $\,\,$ & $\,\,$1$\,\,$ & $\,\,$3 $\,\,$ \\
$\,\,$ 1/6$\,\,$ & $\,\,$ 1 $\,\,$ & $\,\,$ 1 $\,\,$ & $\,\,$2 $\,\,$ \\
$\,\,$ 1/9$\,\,$ & $\,\,$ 1/3$\,\,$ & $\,\,$ 1/2$\,\,$ & $\,\,$ 1  $\,\,$ \\
\end{pmatrix},
\qquad
\lambda_{\max} =
4.1031,
\qquad
CR = 0.0389
\end{equation*}

\begin{equation*}
\mathbf{w}^{EM} =
\begin{pmatrix}
0.715844\\
0.116003\\
\color{red} 0.112013\color{black} \\
0.056141
\end{pmatrix}\end{equation*}
\begin{equation*}
\left[ \frac{{w}^{EM}_i}{{w}^{EM}_j} \right] =
\begin{pmatrix}
$\,\,$ 1 $\,\,$ & $\,\,$6.1709$\,\,$ & $\,\,$\color{red} 6.3907\color{black} $\,\,$ & $\,\,$12.7509$\,\,$ \\
$\,\,$0.1621$\,\,$ & $\,\,$ 1 $\,\,$ & $\,\,$\color{red} 1.0356\color{black} $\,\,$ & $\,\,$2.0663  $\,\,$ \\
$\,\,$\color{red} 0.1565\color{black} $\,\,$ & $\,\,$\color{red} 0.9656\color{black} $\,\,$ & $\,\,$ 1 $\,\,$ & $\,\,$\color{red} 1.9952\color{black}  $\,\,$ \\
$\,\,$0.0784$\,\,$ & $\,\,$0.4840$\,\,$ & $\,\,$\color{red} 0.5012\color{black} $\,\,$ & $\,\,$ 1  $\,\,$ \\
\end{pmatrix},
\end{equation*}

\begin{equation*}
\mathbf{w}^{\prime} =
\begin{pmatrix}
0.715652\\
0.115972\\
0.112251\\
0.056126
\end{pmatrix} =
0.999732\cdot
\begin{pmatrix}
0.715844\\
0.116003\\
\color{gr} 0.112281\color{black} \\
0.056141
\end{pmatrix},
\end{equation*}
\begin{equation*}
\left[ \frac{{w}^{\prime}_i}{{w}^{\prime}_j} \right] =
\begin{pmatrix}
$\,\,$ 1 $\,\,$ & $\,\,$6.1709$\,\,$ & $\,\,$\color{gr} 6.3755\color{black} $\,\,$ & $\,\,$12.7509$\,\,$ \\
$\,\,$0.1621$\,\,$ & $\,\,$ 1 $\,\,$ & $\,\,$\color{gr} 1.0331\color{black} $\,\,$ & $\,\,$2.0663  $\,\,$ \\
$\,\,$\color{gr} 0.1569\color{black} $\,\,$ & $\,\,$\color{gr} 0.9679\color{black} $\,\,$ & $\,\,$ 1 $\,\,$ & $\,\,$\color{gr} \color{blue} 2\color{black}  $\,\,$ \\
$\,\,$0.0784$\,\,$ & $\,\,$0.4840$\,\,$ & $\,\,$\color{gr} \color{blue}  1/2\color{black} $\,\,$ & $\,\,$ 1  $\,\,$ \\
\end{pmatrix},
\end{equation*}
\end{example}
\newpage
\begin{example}
\begin{equation*}
\mathbf{A} =
\begin{pmatrix}
$\,\,$ 1 $\,\,$ & $\,\,$1$\,\,$ & $\,\,$2$\,\,$ & $\,\,$3 $\,\,$ \\
$\,\,$ 1 $\,\,$ & $\,\,$ 1 $\,\,$ & $\,\,$3$\,\,$ & $\,\,$2 $\,\,$ \\
$\,\,$ 1/2$\,\,$ & $\,\,$ 1/3$\,\,$ & $\,\,$ 1 $\,\,$ & $\,\,$ 1/2 $\,\,$ \\
$\,\,$ 1/3$\,\,$ & $\,\,$ 1/2$\,\,$ & $\,\,$2$\,\,$ & $\,\,$ 1  $\,\,$ \\
\end{pmatrix},
\qquad
\lambda_{\max} =
4.1031,
\qquad
CR = 0.0389
\end{equation*}

\begin{equation*}
\mathbf{w}^{EM} =
\begin{pmatrix}
0.358116\\
\color{red} 0.345798\color{black} \\
0.122773\\
0.173313
\end{pmatrix}\end{equation*}
\begin{equation*}
\left[ \frac{{w}^{EM}_i}{{w}^{EM}_j} \right] =
\begin{pmatrix}
$\,\,$ 1 $\,\,$ & $\,\,$\color{red} 1.0356\color{black} $\,\,$ & $\,\,$2.9169$\,\,$ & $\,\,$2.0663$\,\,$ \\
$\,\,$\color{red} 0.9656\color{black} $\,\,$ & $\,\,$ 1 $\,\,$ & $\,\,$\color{red} 2.8166\color{black} $\,\,$ & $\,\,$\color{red} 1.9952\color{black}   $\,\,$ \\
$\,\,$0.3428$\,\,$ & $\,\,$\color{red} 0.3550\color{black} $\,\,$ & $\,\,$ 1 $\,\,$ & $\,\,$0.7084 $\,\,$ \\
$\,\,$0.4840$\,\,$ & $\,\,$\color{red} 0.5012\color{black} $\,\,$ & $\,\,$1.4117$\,\,$ & $\,\,$ 1  $\,\,$ \\
\end{pmatrix},
\end{equation*}

\begin{equation*}
\mathbf{w}^{\prime} =
\begin{pmatrix}
0.357819\\
0.346340\\
0.122671\\
0.173170
\end{pmatrix} =
0.999172\cdot
\begin{pmatrix}
0.358116\\
\color{gr} 0.346627\color{black} \\
0.122773\\
0.173313
\end{pmatrix},
\end{equation*}
\begin{equation*}
\left[ \frac{{w}^{\prime}_i}{{w}^{\prime}_j} \right] =
\begin{pmatrix}
$\,\,$ 1 $\,\,$ & $\,\,$\color{gr} 1.0331\color{black} $\,\,$ & $\,\,$2.9169$\,\,$ & $\,\,$2.0663$\,\,$ \\
$\,\,$\color{gr} 0.9679\color{black} $\,\,$ & $\,\,$ 1 $\,\,$ & $\,\,$\color{gr} 2.8233\color{black} $\,\,$ & $\,\,$\color{gr} \color{blue} 2\color{black}   $\,\,$ \\
$\,\,$0.3428$\,\,$ & $\,\,$\color{gr} 0.3542\color{black} $\,\,$ & $\,\,$ 1 $\,\,$ & $\,\,$0.7084 $\,\,$ \\
$\,\,$0.4840$\,\,$ & $\,\,$\color{gr} \color{blue}  1/2\color{black} $\,\,$ & $\,\,$1.4117$\,\,$ & $\,\,$ 1  $\,\,$ \\
\end{pmatrix},
\end{equation*}
\end{example}
\newpage
\begin{example}
\begin{equation*}
\mathbf{A} =
\begin{pmatrix}
$\,\,$ 1 $\,\,$ & $\,\,$1$\,\,$ & $\,\,$3$\,\,$ & $\,\,$3 $\,\,$ \\
$\,\,$ 1 $\,\,$ & $\,\,$ 1 $\,\,$ & $\,\,$4$\,\,$ & $\,\,$2 $\,\,$ \\
$\,\,$ 1/3$\,\,$ & $\,\,$ 1/4$\,\,$ & $\,\,$ 1 $\,\,$ & $\,\,$ 1/3 $\,\,$ \\
$\,\,$ 1/3$\,\,$ & $\,\,$ 1/2$\,\,$ & $\,\,$3$\,\,$ & $\,\,$ 1  $\,\,$ \\
\end{pmatrix},
\qquad
\lambda_{\max} =
4.1031,
\qquad
CR = 0.0389
\end{equation*}

\begin{equation*}
\mathbf{w}^{EM} =
\begin{pmatrix}
0.375899\\
\color{red} 0.352916\color{black} \\
0.088441\\
0.182744
\end{pmatrix}\end{equation*}
\begin{equation*}
\left[ \frac{{w}^{EM}_i}{{w}^{EM}_j} \right] =
\begin{pmatrix}
$\,\,$ 1 $\,\,$ & $\,\,$\color{red} 1.0651\color{black} $\,\,$ & $\,\,$4.2503$\,\,$ & $\,\,$2.0570$\,\,$ \\
$\,\,$\color{red} 0.9389\color{black} $\,\,$ & $\,\,$ 1 $\,\,$ & $\,\,$\color{red} 3.9904\color{black} $\,\,$ & $\,\,$\color{red} 1.9312\color{black}   $\,\,$ \\
$\,\,$0.2353$\,\,$ & $\,\,$\color{red} 0.2506\color{black} $\,\,$ & $\,\,$ 1 $\,\,$ & $\,\,$0.4840 $\,\,$ \\
$\,\,$0.4862$\,\,$ & $\,\,$\color{red} 0.5178\color{black} $\,\,$ & $\,\,$2.0663$\,\,$ & $\,\,$ 1  $\,\,$ \\
\end{pmatrix},
\end{equation*}

\begin{equation*}
\mathbf{w}^{\prime} =
\begin{pmatrix}
0.375582\\
0.353463\\
0.088366\\
0.182589
\end{pmatrix} =
0.999155\cdot
\begin{pmatrix}
0.375899\\
\color{gr} 0.353762\color{black} \\
0.088441\\
0.182744
\end{pmatrix},
\end{equation*}
\begin{equation*}
\left[ \frac{{w}^{\prime}_i}{{w}^{\prime}_j} \right] =
\begin{pmatrix}
$\,\,$ 1 $\,\,$ & $\,\,$\color{gr} 1.0626\color{black} $\,\,$ & $\,\,$4.2503$\,\,$ & $\,\,$2.0570$\,\,$ \\
$\,\,$\color{gr} 0.9411\color{black} $\,\,$ & $\,\,$ 1 $\,\,$ & $\,\,$\color{gr} \color{blue} 4\color{black} $\,\,$ & $\,\,$\color{gr} 1.9358\color{black}   $\,\,$ \\
$\,\,$0.2353$\,\,$ & $\,\,$\color{gr} \color{blue}  1/4\color{black} $\,\,$ & $\,\,$ 1 $\,\,$ & $\,\,$0.4840 $\,\,$ \\
$\,\,$0.4862$\,\,$ & $\,\,$\color{gr} 0.5166\color{black} $\,\,$ & $\,\,$2.0663$\,\,$ & $\,\,$ 1  $\,\,$ \\
\end{pmatrix},
\end{equation*}
\end{example}
\newpage
\begin{example}
\begin{equation*}
\mathbf{A} =
\begin{pmatrix}
$\,\,$ 1 $\,\,$ & $\,\,$1$\,\,$ & $\,\,$3$\,\,$ & $\,\,$3 $\,\,$ \\
$\,\,$ 1 $\,\,$ & $\,\,$ 1 $\,\,$ & $\,\,$5$\,\,$ & $\,\,$2 $\,\,$ \\
$\,\,$ 1/3$\,\,$ & $\,\,$ 1/5$\,\,$ & $\,\,$ 1 $\,\,$ & $\,\,$ 1/4 $\,\,$ \\
$\,\,$ 1/3$\,\,$ & $\,\,$ 1/2$\,\,$ & $\,\,$4$\,\,$ & $\,\,$ 1  $\,\,$ \\
\end{pmatrix},
\qquad
\lambda_{\max} =
4.1655,
\qquad
CR = 0.0624
\end{equation*}

\begin{equation*}
\mathbf{w}^{EM} =
\begin{pmatrix}
0.369702\\
\color{red} 0.360296\color{black} \\
0.076941\\
0.193062
\end{pmatrix}\end{equation*}
\begin{equation*}
\left[ \frac{{w}^{EM}_i}{{w}^{EM}_j} \right] =
\begin{pmatrix}
$\,\,$ 1 $\,\,$ & $\,\,$\color{red} 1.0261\color{black} $\,\,$ & $\,\,$4.8050$\,\,$ & $\,\,$1.9149$\,\,$ \\
$\,\,$\color{red} 0.9746\color{black} $\,\,$ & $\,\,$ 1 $\,\,$ & $\,\,$\color{red} 4.6828\color{black} $\,\,$ & $\,\,$\color{red} 1.8662\color{black}   $\,\,$ \\
$\,\,$0.2081$\,\,$ & $\,\,$\color{red} 0.2135\color{black} $\,\,$ & $\,\,$ 1 $\,\,$ & $\,\,$0.3985 $\,\,$ \\
$\,\,$0.5222$\,\,$ & $\,\,$\color{red} 0.5358\color{black} $\,\,$ & $\,\,$2.5092$\,\,$ & $\,\,$ 1  $\,\,$ \\
\end{pmatrix},
\end{equation*}

\begin{equation*}
\mathbf{w}^{\prime} =
\begin{pmatrix}
0.366257\\
0.366257\\
0.076224\\
0.191263
\end{pmatrix} =
0.990682\cdot
\begin{pmatrix}
0.369702\\
\color{gr} 0.369702\color{black} \\
0.076941\\
0.193062
\end{pmatrix},
\end{equation*}
\begin{equation*}
\left[ \frac{{w}^{\prime}_i}{{w}^{\prime}_j} \right] =
\begin{pmatrix}
$\,\,$ 1 $\,\,$ & $\,\,$\color{gr} \color{blue} 1\color{black} $\,\,$ & $\,\,$4.8050$\,\,$ & $\,\,$1.9149$\,\,$ \\
$\,\,$\color{gr} \color{blue} 1\color{black} $\,\,$ & $\,\,$ 1 $\,\,$ & $\,\,$\color{gr} 4.8050\color{black} $\,\,$ & $\,\,$\color{gr} 1.9149\color{black}   $\,\,$ \\
$\,\,$0.2081$\,\,$ & $\,\,$\color{gr} 0.2081\color{black} $\,\,$ & $\,\,$ 1 $\,\,$ & $\,\,$0.3985 $\,\,$ \\
$\,\,$0.5222$\,\,$ & $\,\,$\color{gr} 0.5222\color{black} $\,\,$ & $\,\,$2.5092$\,\,$ & $\,\,$ 1  $\,\,$ \\
\end{pmatrix},
\end{equation*}
\end{example}
\newpage
\begin{example}
\begin{equation*}
\mathbf{A} =
\begin{pmatrix}
$\,\,$ 1 $\,\,$ & $\,\,$1$\,\,$ & $\,\,$3$\,\,$ & $\,\,$3 $\,\,$ \\
$\,\,$ 1 $\,\,$ & $\,\,$ 1 $\,\,$ & $\,\,$5$\,\,$ & $\,\,$2 $\,\,$ \\
$\,\,$ 1/3$\,\,$ & $\,\,$ 1/5$\,\,$ & $\,\,$ 1 $\,\,$ & $\,\,$ 1/5 $\,\,$ \\
$\,\,$ 1/3$\,\,$ & $\,\,$ 1/2$\,\,$ & $\,\,$5$\,\,$ & $\,\,$ 1  $\,\,$ \\
\end{pmatrix},
\qquad
\lambda_{\max} =
4.2277,
\qquad
CR = 0.0859
\end{equation*}

\begin{equation*}
\mathbf{w}^{EM} =
\begin{pmatrix}
0.368074\\
\color{red} 0.353868\color{black} \\
0.072665\\
0.205393
\end{pmatrix}\end{equation*}
\begin{equation*}
\left[ \frac{{w}^{EM}_i}{{w}^{EM}_j} \right] =
\begin{pmatrix}
$\,\,$ 1 $\,\,$ & $\,\,$\color{red} 1.0401\color{black} $\,\,$ & $\,\,$5.0653$\,\,$ & $\,\,$1.7920$\,\,$ \\
$\,\,$\color{red} 0.9614\color{black} $\,\,$ & $\,\,$ 1 $\,\,$ & $\,\,$\color{red} 4.8698\color{black} $\,\,$ & $\,\,$\color{red} 1.7229\color{black}   $\,\,$ \\
$\,\,$0.1974$\,\,$ & $\,\,$\color{red} 0.2053\color{black} $\,\,$ & $\,\,$ 1 $\,\,$ & $\,\,$0.3538 $\,\,$ \\
$\,\,$0.5580$\,\,$ & $\,\,$\color{red} 0.5804\color{black} $\,\,$ & $\,\,$2.8266$\,\,$ & $\,\,$ 1  $\,\,$ \\
\end{pmatrix},
\end{equation*}

\begin{equation*}
\mathbf{w}^{\prime} =
\begin{pmatrix}
0.364625\\
0.359922\\
0.071984\\
0.203468
\end{pmatrix} =
0.990630\cdot
\begin{pmatrix}
0.368074\\
\color{gr} 0.363326\color{black} \\
0.072665\\
0.205393
\end{pmatrix},
\end{equation*}
\begin{equation*}
\left[ \frac{{w}^{\prime}_i}{{w}^{\prime}_j} \right] =
\begin{pmatrix}
$\,\,$ 1 $\,\,$ & $\,\,$\color{gr} 1.0131\color{black} $\,\,$ & $\,\,$5.0653$\,\,$ & $\,\,$1.7920$\,\,$ \\
$\,\,$\color{gr} 0.9871\color{black} $\,\,$ & $\,\,$ 1 $\,\,$ & $\,\,$\color{gr} \color{blue} 5\color{black} $\,\,$ & $\,\,$\color{gr} 1.7689\color{black}   $\,\,$ \\
$\,\,$0.1974$\,\,$ & $\,\,$\color{gr} \color{blue}  1/5\color{black} $\,\,$ & $\,\,$ 1 $\,\,$ & $\,\,$0.3538 $\,\,$ \\
$\,\,$0.5580$\,\,$ & $\,\,$\color{gr} 0.5653\color{black} $\,\,$ & $\,\,$2.8266$\,\,$ & $\,\,$ 1  $\,\,$ \\
\end{pmatrix},
\end{equation*}
\end{example}
\newpage
\begin{example}
\begin{equation*}
\mathbf{A} =
\begin{pmatrix}
$\,\,$ 1 $\,\,$ & $\,\,$1$\,\,$ & $\,\,$3$\,\,$ & $\,\,$4 $\,\,$ \\
$\,\,$ 1 $\,\,$ & $\,\,$ 1 $\,\,$ & $\,\,$5$\,\,$ & $\,\,$2 $\,\,$ \\
$\,\,$ 1/3$\,\,$ & $\,\,$ 1/5$\,\,$ & $\,\,$ 1 $\,\,$ & $\,\,$ 1/4 $\,\,$ \\
$\,\,$ 1/4$\,\,$ & $\,\,$ 1/2$\,\,$ & $\,\,$4$\,\,$ & $\,\,$ 1  $\,\,$ \\
\end{pmatrix},
\qquad
\lambda_{\max} =
4.2460,
\qquad
CR = 0.0928
\end{equation*}

\begin{equation*}
\mathbf{w}^{EM} =
\begin{pmatrix}
0.397044\\
\color{red} 0.349006\color{black} \\
0.075982\\
0.177968
\end{pmatrix}\end{equation*}
\begin{equation*}
\left[ \frac{{w}^{EM}_i}{{w}^{EM}_j} \right] =
\begin{pmatrix}
$\,\,$ 1 $\,\,$ & $\,\,$\color{red} 1.1376\color{black} $\,\,$ & $\,\,$5.2255$\,\,$ & $\,\,$2.2310$\,\,$ \\
$\,\,$\color{red} 0.8790\color{black} $\,\,$ & $\,\,$ 1 $\,\,$ & $\,\,$\color{red} 4.5933\color{black} $\,\,$ & $\,\,$\color{red} 1.9611\color{black}   $\,\,$ \\
$\,\,$0.1914$\,\,$ & $\,\,$\color{red} 0.2177\color{black} $\,\,$ & $\,\,$ 1 $\,\,$ & $\,\,$0.4269 $\,\,$ \\
$\,\,$0.4482$\,\,$ & $\,\,$\color{red} 0.5099\color{black} $\,\,$ & $\,\,$2.3422$\,\,$ & $\,\,$ 1  $\,\,$ \\
\end{pmatrix},
\end{equation*}

\begin{equation*}
\mathbf{w}^{\prime} =
\begin{pmatrix}
0.394312\\
0.353486\\
0.075459\\
0.176743
\end{pmatrix} =
0.993118\cdot
\begin{pmatrix}
0.397044\\
\color{gr} 0.355936\color{black} \\
0.075982\\
0.177968
\end{pmatrix},
\end{equation*}
\begin{equation*}
\left[ \frac{{w}^{\prime}_i}{{w}^{\prime}_j} \right] =
\begin{pmatrix}
$\,\,$ 1 $\,\,$ & $\,\,$\color{gr} 1.1155\color{black} $\,\,$ & $\,\,$5.2255$\,\,$ & $\,\,$2.2310$\,\,$ \\
$\,\,$\color{gr} 0.8965\color{black} $\,\,$ & $\,\,$ 1 $\,\,$ & $\,\,$\color{gr} 4.6845\color{black} $\,\,$ & $\,\,$\color{gr} \color{blue} 2\color{black}   $\,\,$ \\
$\,\,$0.1914$\,\,$ & $\,\,$\color{gr} 0.2135\color{black} $\,\,$ & $\,\,$ 1 $\,\,$ & $\,\,$0.4269 $\,\,$ \\
$\,\,$0.4482$\,\,$ & $\,\,$\color{gr} \color{blue}  1/2\color{black} $\,\,$ & $\,\,$2.3422$\,\,$ & $\,\,$ 1  $\,\,$ \\
\end{pmatrix},
\end{equation*}
\end{example}
\newpage
\begin{example}
\begin{equation*}
\mathbf{A} =
\begin{pmatrix}
$\,\,$ 1 $\,\,$ & $\,\,$1$\,\,$ & $\,\,$3$\,\,$ & $\,\,$5 $\,\,$ \\
$\,\,$ 1 $\,\,$ & $\,\,$ 1 $\,\,$ & $\,\,$4$\,\,$ & $\,\,$3 $\,\,$ \\
$\,\,$ 1/3$\,\,$ & $\,\,$ 1/4$\,\,$ & $\,\,$ 1 $\,\,$ & $\,\,$ 1/2 $\,\,$ \\
$\,\,$ 1/5$\,\,$ & $\,\,$ 1/3$\,\,$ & $\,\,$2$\,\,$ & $\,\,$ 1  $\,\,$ \\
\end{pmatrix},
\qquad
\lambda_{\max} =
4.1252,
\qquad
CR = 0.0472
\end{equation*}

\begin{equation*}
\mathbf{w}^{EM} =
\begin{pmatrix}
0.409562\\
\color{red} 0.371311\color{black} \\
0.093487\\
0.125641
\end{pmatrix}\end{equation*}
\begin{equation*}
\left[ \frac{{w}^{EM}_i}{{w}^{EM}_j} \right] =
\begin{pmatrix}
$\,\,$ 1 $\,\,$ & $\,\,$\color{red} 1.1030\color{black} $\,\,$ & $\,\,$4.3809$\,\,$ & $\,\,$3.2598$\,\,$ \\
$\,\,$\color{red} 0.9066\color{black} $\,\,$ & $\,\,$ 1 $\,\,$ & $\,\,$\color{red} 3.9718\color{black} $\,\,$ & $\,\,$\color{red} 2.9553\color{black}   $\,\,$ \\
$\,\,$0.2283$\,\,$ & $\,\,$\color{red} 0.2518\color{black} $\,\,$ & $\,\,$ 1 $\,\,$ & $\,\,$0.7441 $\,\,$ \\
$\,\,$0.3068$\,\,$ & $\,\,$\color{red} 0.3384\color{black} $\,\,$ & $\,\,$1.3439$\,\,$ & $\,\,$ 1  $\,\,$ \\
\end{pmatrix},
\end{equation*}

\begin{equation*}
\mathbf{w}^{\prime} =
\begin{pmatrix}
0.408484\\
0.372964\\
0.093241\\
0.125310
\end{pmatrix} =
0.997370\cdot
\begin{pmatrix}
0.409562\\
\color{gr} 0.373948\color{black} \\
0.093487\\
0.125641
\end{pmatrix},
\end{equation*}
\begin{equation*}
\left[ \frac{{w}^{\prime}_i}{{w}^{\prime}_j} \right] =
\begin{pmatrix}
$\,\,$ 1 $\,\,$ & $\,\,$\color{gr} 1.0952\color{black} $\,\,$ & $\,\,$4.3809$\,\,$ & $\,\,$3.2598$\,\,$ \\
$\,\,$\color{gr} 0.9130\color{black} $\,\,$ & $\,\,$ 1 $\,\,$ & $\,\,$\color{gr} \color{blue} 4\color{black} $\,\,$ & $\,\,$\color{gr} 2.9763\color{black}   $\,\,$ \\
$\,\,$0.2283$\,\,$ & $\,\,$\color{gr} \color{blue}  1/4\color{black} $\,\,$ & $\,\,$ 1 $\,\,$ & $\,\,$0.7441 $\,\,$ \\
$\,\,$0.3068$\,\,$ & $\,\,$\color{gr} 0.3360\color{black} $\,\,$ & $\,\,$1.3439$\,\,$ & $\,\,$ 1  $\,\,$ \\
\end{pmatrix},
\end{equation*}
\end{example}
\newpage
\begin{example}
\begin{equation*}
\mathbf{A} =
\begin{pmatrix}
$\,\,$ 1 $\,\,$ & $\,\,$1$\,\,$ & $\,\,$3$\,\,$ & $\,\,$5 $\,\,$ \\
$\,\,$ 1 $\,\,$ & $\,\,$ 1 $\,\,$ & $\,\,$5$\,\,$ & $\,\,$3 $\,\,$ \\
$\,\,$ 1/3$\,\,$ & $\,\,$ 1/5$\,\,$ & $\,\,$ 1 $\,\,$ & $\,\,$ 1/3 $\,\,$ \\
$\,\,$ 1/5$\,\,$ & $\,\,$ 1/3$\,\,$ & $\,\,$3$\,\,$ & $\,\,$ 1  $\,\,$ \\
\end{pmatrix},
\qquad
\lambda_{\max} =
4.2253,
\qquad
CR = 0.0849
\end{equation*}

\begin{equation*}
\mathbf{w}^{EM} =
\begin{pmatrix}
0.405047\\
\color{red} 0.377329\color{black} \\
0.079532\\
0.138092
\end{pmatrix}\end{equation*}
\begin{equation*}
\left[ \frac{{w}^{EM}_i}{{w}^{EM}_j} \right] =
\begin{pmatrix}
$\,\,$ 1 $\,\,$ & $\,\,$\color{red} 1.0735\color{black} $\,\,$ & $\,\,$5.0929$\,\,$ & $\,\,$2.9332$\,\,$ \\
$\,\,$\color{red} 0.9316\color{black} $\,\,$ & $\,\,$ 1 $\,\,$ & $\,\,$\color{red} 4.7444\color{black} $\,\,$ & $\,\,$\color{red} 2.7324\color{black}   $\,\,$ \\
$\,\,$0.1964$\,\,$ & $\,\,$\color{red} 0.2108\color{black} $\,\,$ & $\,\,$ 1 $\,\,$ & $\,\,$0.5759 $\,\,$ \\
$\,\,$0.3409$\,\,$ & $\,\,$\color{red} 0.3660\color{black} $\,\,$ & $\,\,$1.7363$\,\,$ & $\,\,$ 1  $\,\,$ \\
\end{pmatrix},
\end{equation*}

\begin{equation*}
\mathbf{w}^{\prime} =
\begin{pmatrix}
0.396976\\
0.389737\\
0.077947\\
0.135340
\end{pmatrix} =
0.980073\cdot
\begin{pmatrix}
0.405047\\
\color{gr} 0.397661\color{black} \\
0.079532\\
0.138092
\end{pmatrix},
\end{equation*}
\begin{equation*}
\left[ \frac{{w}^{\prime}_i}{{w}^{\prime}_j} \right] =
\begin{pmatrix}
$\,\,$ 1 $\,\,$ & $\,\,$\color{gr} 1.0186\color{black} $\,\,$ & $\,\,$5.0929$\,\,$ & $\,\,$2.9332$\,\,$ \\
$\,\,$\color{gr} 0.9818\color{black} $\,\,$ & $\,\,$ 1 $\,\,$ & $\,\,$\color{gr} \color{blue} 5\color{black} $\,\,$ & $\,\,$\color{gr} 2.8797\color{black}   $\,\,$ \\
$\,\,$0.1964$\,\,$ & $\,\,$\color{gr} \color{blue}  1/5\color{black} $\,\,$ & $\,\,$ 1 $\,\,$ & $\,\,$0.5759 $\,\,$ \\
$\,\,$0.3409$\,\,$ & $\,\,$\color{gr} 0.3473\color{black} $\,\,$ & $\,\,$1.7363$\,\,$ & $\,\,$ 1  $\,\,$ \\
\end{pmatrix},
\end{equation*}
\end{example}
\newpage
\begin{example}
\begin{equation*}
\mathbf{A} =
\begin{pmatrix}
$\,\,$ 1 $\,\,$ & $\,\,$1$\,\,$ & $\,\,$4$\,\,$ & $\,\,$3 $\,\,$ \\
$\,\,$ 1 $\,\,$ & $\,\,$ 1 $\,\,$ & $\,\,$6$\,\,$ & $\,\,$2 $\,\,$ \\
$\,\,$ 1/4$\,\,$ & $\,\,$ 1/6$\,\,$ & $\,\,$ 1 $\,\,$ & $\,\,$ 1/4 $\,\,$ \\
$\,\,$ 1/3$\,\,$ & $\,\,$ 1/2$\,\,$ & $\,\,$4$\,\,$ & $\,\,$ 1  $\,\,$ \\
\end{pmatrix},
\qquad
\lambda_{\max} =
4.1031,
\qquad
CR = 0.0389
\end{equation*}

\begin{equation*}
\mathbf{w}^{EM} =
\begin{pmatrix}
0.381537\\
\color{red} 0.368414\color{black} \\
0.065401\\
0.184648
\end{pmatrix}\end{equation*}
\begin{equation*}
\left[ \frac{{w}^{EM}_i}{{w}^{EM}_j} \right] =
\begin{pmatrix}
$\,\,$ 1 $\,\,$ & $\,\,$\color{red} 1.0356\color{black} $\,\,$ & $\,\,$5.8338$\,\,$ & $\,\,$2.0663$\,\,$ \\
$\,\,$\color{red} 0.9656\color{black} $\,\,$ & $\,\,$ 1 $\,\,$ & $\,\,$\color{red} 5.6331\color{black} $\,\,$ & $\,\,$\color{red} 1.9952\color{black}   $\,\,$ \\
$\,\,$0.1714$\,\,$ & $\,\,$\color{red} 0.1775\color{black} $\,\,$ & $\,\,$ 1 $\,\,$ & $\,\,$0.3542 $\,\,$ \\
$\,\,$0.4840$\,\,$ & $\,\,$\color{red} 0.5012\color{black} $\,\,$ & $\,\,$2.8233$\,\,$ & $\,\,$ 1  $\,\,$ \\
\end{pmatrix},
\end{equation*}

\begin{equation*}
\mathbf{w}^{\prime} =
\begin{pmatrix}
0.381200\\
0.368971\\
0.065343\\
0.184485
\end{pmatrix} =
0.999118\cdot
\begin{pmatrix}
0.381537\\
\color{gr} 0.369297\color{black} \\
0.065401\\
0.184648
\end{pmatrix},
\end{equation*}
\begin{equation*}
\left[ \frac{{w}^{\prime}_i}{{w}^{\prime}_j} \right] =
\begin{pmatrix}
$\,\,$ 1 $\,\,$ & $\,\,$\color{gr} 1.0331\color{black} $\,\,$ & $\,\,$5.8338$\,\,$ & $\,\,$2.0663$\,\,$ \\
$\,\,$\color{gr} 0.9679\color{black} $\,\,$ & $\,\,$ 1 $\,\,$ & $\,\,$\color{gr} 5.6467\color{black} $\,\,$ & $\,\,$\color{gr} \color{blue} 2\color{black}   $\,\,$ \\
$\,\,$0.1714$\,\,$ & $\,\,$\color{gr} 0.1771\color{black} $\,\,$ & $\,\,$ 1 $\,\,$ & $\,\,$0.3542 $\,\,$ \\
$\,\,$0.4840$\,\,$ & $\,\,$\color{gr} \color{blue}  1/2\color{black} $\,\,$ & $\,\,$2.8233$\,\,$ & $\,\,$ 1  $\,\,$ \\
\end{pmatrix},
\end{equation*}
\end{example}
\newpage
\begin{example}
\begin{equation*}
\mathbf{A} =
\begin{pmatrix}
$\,\,$ 1 $\,\,$ & $\,\,$1$\,\,$ & $\,\,$4$\,\,$ & $\,\,$3 $\,\,$ \\
$\,\,$ 1 $\,\,$ & $\,\,$ 1 $\,\,$ & $\,\,$6$\,\,$ & $\,\,$2 $\,\,$ \\
$\,\,$ 1/4$\,\,$ & $\,\,$ 1/6$\,\,$ & $\,\,$ 1 $\,\,$ & $\,\,$ 1/5 $\,\,$ \\
$\,\,$ 1/3$\,\,$ & $\,\,$ 1/2$\,\,$ & $\,\,$5$\,\,$ & $\,\,$ 1  $\,\,$ \\
\end{pmatrix},
\qquad
\lambda_{\max} =
4.1502,
\qquad
CR = 0.0566
\end{equation*}

\begin{equation*}
\mathbf{w}^{EM} =
\begin{pmatrix}
0.379938\\
\color{red} 0.362530\color{black} \\
0.061761\\
0.195770
\end{pmatrix}\end{equation*}
\begin{equation*}
\left[ \frac{{w}^{EM}_i}{{w}^{EM}_j} \right] =
\begin{pmatrix}
$\,\,$ 1 $\,\,$ & $\,\,$\color{red} 1.0480\color{black} $\,\,$ & $\,\,$6.1517$\,\,$ & $\,\,$1.9407$\,\,$ \\
$\,\,$\color{red} 0.9542\color{black} $\,\,$ & $\,\,$ 1 $\,\,$ & $\,\,$\color{red} 5.8699\color{black} $\,\,$ & $\,\,$\color{red} 1.8518\color{black}   $\,\,$ \\
$\,\,$0.1626$\,\,$ & $\,\,$\color{red} 0.1704\color{black} $\,\,$ & $\,\,$ 1 $\,\,$ & $\,\,$0.3155 $\,\,$ \\
$\,\,$0.5153$\,\,$ & $\,\,$\color{red} 0.5400\color{black} $\,\,$ & $\,\,$3.1698$\,\,$ & $\,\,$ 1  $\,\,$ \\
\end{pmatrix},
\end{equation*}

\begin{equation*}
\mathbf{w}^{\prime} =
\begin{pmatrix}
0.376909\\
0.367612\\
0.061269\\
0.194209
\end{pmatrix} =
0.992028\cdot
\begin{pmatrix}
0.379938\\
\color{gr} 0.370567\color{black} \\
0.061761\\
0.195770
\end{pmatrix},
\end{equation*}
\begin{equation*}
\left[ \frac{{w}^{\prime}_i}{{w}^{\prime}_j} \right] =
\begin{pmatrix}
$\,\,$ 1 $\,\,$ & $\,\,$\color{gr} 1.0253\color{black} $\,\,$ & $\,\,$6.1517$\,\,$ & $\,\,$1.9407$\,\,$ \\
$\,\,$\color{gr} 0.9753\color{black} $\,\,$ & $\,\,$ 1 $\,\,$ & $\,\,$\color{gr} \color{blue} 6\color{black} $\,\,$ & $\,\,$\color{gr} 1.8929\color{black}   $\,\,$ \\
$\,\,$0.1626$\,\,$ & $\,\,$\color{gr} \color{blue}  1/6\color{black} $\,\,$ & $\,\,$ 1 $\,\,$ & $\,\,$0.3155 $\,\,$ \\
$\,\,$0.5153$\,\,$ & $\,\,$\color{gr} 0.5283\color{black} $\,\,$ & $\,\,$3.1698$\,\,$ & $\,\,$ 1  $\,\,$ \\
\end{pmatrix},
\end{equation*}
\end{example}
\newpage
\begin{example}
\begin{equation*}
\mathbf{A} =
\begin{pmatrix}
$\,\,$ 1 $\,\,$ & $\,\,$1$\,\,$ & $\,\,$4$\,\,$ & $\,\,$3 $\,\,$ \\
$\,\,$ 1 $\,\,$ & $\,\,$ 1 $\,\,$ & $\,\,$7$\,\,$ & $\,\,$2 $\,\,$ \\
$\,\,$ 1/4$\,\,$ & $\,\,$ 1/7$\,\,$ & $\,\,$ 1 $\,\,$ & $\,\,$ 1/5 $\,\,$ \\
$\,\,$ 1/3$\,\,$ & $\,\,$ 1/2$\,\,$ & $\,\,$5$\,\,$ & $\,\,$ 1  $\,\,$ \\
\end{pmatrix},
\qquad
\lambda_{\max} =
4.1512,
\qquad
CR = 0.0570
\end{equation*}

\begin{equation*}
\mathbf{w}^{EM} =
\begin{pmatrix}
0.376198\\
\color{red} 0.372443\color{black} \\
0.058943\\
0.192416
\end{pmatrix}\end{equation*}
\begin{equation*}
\left[ \frac{{w}^{EM}_i}{{w}^{EM}_j} \right] =
\begin{pmatrix}
$\,\,$ 1 $\,\,$ & $\,\,$\color{red} 1.0101\color{black} $\,\,$ & $\,\,$6.3824$\,\,$ & $\,\,$1.9551$\,\,$ \\
$\,\,$\color{red} 0.9900\color{black} $\,\,$ & $\,\,$ 1 $\,\,$ & $\,\,$\color{red} 6.3187\color{black} $\,\,$ & $\,\,$\color{red} 1.9356\color{black}   $\,\,$ \\
$\,\,$0.1567$\,\,$ & $\,\,$\color{red} 0.1583\color{black} $\,\,$ & $\,\,$ 1 $\,\,$ & $\,\,$0.3063 $\,\,$ \\
$\,\,$0.5115$\,\,$ & $\,\,$\color{red} 0.5166\color{black} $\,\,$ & $\,\,$3.2645$\,\,$ & $\,\,$ 1  $\,\,$ \\
\end{pmatrix},
\end{equation*}

\begin{equation*}
\mathbf{w}^{\prime} =
\begin{pmatrix}
0.374791\\
0.374791\\
0.058722\\
0.191696
\end{pmatrix} =
0.996259\cdot
\begin{pmatrix}
0.376198\\
\color{gr} 0.376198\color{black} \\
0.058943\\
0.192416
\end{pmatrix},
\end{equation*}
\begin{equation*}
\left[ \frac{{w}^{\prime}_i}{{w}^{\prime}_j} \right] =
\begin{pmatrix}
$\,\,$ 1 $\,\,$ & $\,\,$\color{gr} \color{blue} 1\color{black} $\,\,$ & $\,\,$6.3824$\,\,$ & $\,\,$1.9551$\,\,$ \\
$\,\,$\color{gr} \color{blue} 1\color{black} $\,\,$ & $\,\,$ 1 $\,\,$ & $\,\,$\color{gr} 6.3824\color{black} $\,\,$ & $\,\,$\color{gr} 1.9551\color{black}   $\,\,$ \\
$\,\,$0.1567$\,\,$ & $\,\,$\color{gr} 0.1567\color{black} $\,\,$ & $\,\,$ 1 $\,\,$ & $\,\,$0.3063 $\,\,$ \\
$\,\,$0.5115$\,\,$ & $\,\,$\color{gr} 0.5115\color{black} $\,\,$ & $\,\,$3.2645$\,\,$ & $\,\,$ 1  $\,\,$ \\
\end{pmatrix},
\end{equation*}
\end{example}
\newpage
\begin{example}
\begin{equation*}
\mathbf{A} =
\begin{pmatrix}
$\,\,$ 1 $\,\,$ & $\,\,$1$\,\,$ & $\,\,$4$\,\,$ & $\,\,$3 $\,\,$ \\
$\,\,$ 1 $\,\,$ & $\,\,$ 1 $\,\,$ & $\,\,$7$\,\,$ & $\,\,$2 $\,\,$ \\
$\,\,$ 1/4$\,\,$ & $\,\,$ 1/7$\,\,$ & $\,\,$ 1 $\,\,$ & $\,\,$ 1/6 $\,\,$ \\
$\,\,$ 1/3$\,\,$ & $\,\,$ 1/2$\,\,$ & $\,\,$6$\,\,$ & $\,\,$ 1  $\,\,$ \\
\end{pmatrix},
\qquad
\lambda_{\max} =
4.1964,
\qquad
CR = 0.0741
\end{equation*}

\begin{equation*}
\mathbf{w}^{EM} =
\begin{pmatrix}
0.374815\\
\color{red} 0.366876\color{black} \\
0.056248\\
0.202061
\end{pmatrix}\end{equation*}
\begin{equation*}
\left[ \frac{{w}^{EM}_i}{{w}^{EM}_j} \right] =
\begin{pmatrix}
$\,\,$ 1 $\,\,$ & $\,\,$\color{red} 1.0216\color{black} $\,\,$ & $\,\,$6.6636$\,\,$ & $\,\,$1.8550$\,\,$ \\
$\,\,$\color{red} 0.9788\color{black} $\,\,$ & $\,\,$ 1 $\,\,$ & $\,\,$\color{red} 6.5224\color{black} $\,\,$ & $\,\,$\color{red} 1.8157\color{black}   $\,\,$ \\
$\,\,$0.1501$\,\,$ & $\,\,$\color{red} 0.1533\color{black} $\,\,$ & $\,\,$ 1 $\,\,$ & $\,\,$0.2784 $\,\,$ \\
$\,\,$0.5391$\,\,$ & $\,\,$\color{red} 0.5508\color{black} $\,\,$ & $\,\,$3.5923$\,\,$ & $\,\,$ 1  $\,\,$ \\
\end{pmatrix},
\end{equation*}

\begin{equation*}
\mathbf{w}^{\prime} =
\begin{pmatrix}
0.371862\\
0.371862\\
0.055805\\
0.200470
\end{pmatrix} =
0.992123\cdot
\begin{pmatrix}
0.374815\\
\color{gr} 0.374815\color{black} \\
0.056248\\
0.202061
\end{pmatrix},
\end{equation*}
\begin{equation*}
\left[ \frac{{w}^{\prime}_i}{{w}^{\prime}_j} \right] =
\begin{pmatrix}
$\,\,$ 1 $\,\,$ & $\,\,$\color{gr} \color{blue} 1\color{black} $\,\,$ & $\,\,$6.6636$\,\,$ & $\,\,$1.8550$\,\,$ \\
$\,\,$\color{gr} \color{blue} 1\color{black} $\,\,$ & $\,\,$ 1 $\,\,$ & $\,\,$\color{gr} 6.6636\color{black} $\,\,$ & $\,\,$\color{gr} 1.8550\color{black}   $\,\,$ \\
$\,\,$0.1501$\,\,$ & $\,\,$\color{gr} 0.1501\color{black} $\,\,$ & $\,\,$ 1 $\,\,$ & $\,\,$0.2784 $\,\,$ \\
$\,\,$0.5391$\,\,$ & $\,\,$\color{gr} 0.5391\color{black} $\,\,$ & $\,\,$3.5923$\,\,$ & $\,\,$ 1  $\,\,$ \\
\end{pmatrix},
\end{equation*}
\end{example}
\newpage
\begin{example}
\begin{equation*}
\mathbf{A} =
\begin{pmatrix}
$\,\,$ 1 $\,\,$ & $\,\,$1$\,\,$ & $\,\,$4$\,\,$ & $\,\,$3 $\,\,$ \\
$\,\,$ 1 $\,\,$ & $\,\,$ 1 $\,\,$ & $\,\,$7$\,\,$ & $\,\,$2 $\,\,$ \\
$\,\,$ 1/4$\,\,$ & $\,\,$ 1/7$\,\,$ & $\,\,$ 1 $\,\,$ & $\,\,$ 1/7 $\,\,$ \\
$\,\,$ 1/3$\,\,$ & $\,\,$ 1/2$\,\,$ & $\,\,$7$\,\,$ & $\,\,$ 1  $\,\,$ \\
\end{pmatrix},
\qquad
\lambda_{\max} =
4.2421,
\qquad
CR = 0.0913
\end{equation*}

\begin{equation*}
\mathbf{w}^{EM} =
\begin{pmatrix}
0.373327\\
\color{red} 0.361833\color{black} \\
0.054020\\
0.210819
\end{pmatrix}\end{equation*}
\begin{equation*}
\left[ \frac{{w}^{EM}_i}{{w}^{EM}_j} \right] =
\begin{pmatrix}
$\,\,$ 1 $\,\,$ & $\,\,$\color{red} 1.0318\color{black} $\,\,$ & $\,\,$6.9109$\,\,$ & $\,\,$1.7708$\,\,$ \\
$\,\,$\color{red} 0.9692\color{black} $\,\,$ & $\,\,$ 1 $\,\,$ & $\,\,$\color{red} 6.6981\color{black} $\,\,$ & $\,\,$\color{red} 1.7163\color{black}   $\,\,$ \\
$\,\,$0.1447$\,\,$ & $\,\,$\color{red} 0.1493\color{black} $\,\,$ & $\,\,$ 1 $\,\,$ & $\,\,$0.2562 $\,\,$ \\
$\,\,$0.5647$\,\,$ & $\,\,$\color{red} 0.5826\color{black} $\,\,$ & $\,\,$3.9026$\,\,$ & $\,\,$ 1  $\,\,$ \\
\end{pmatrix},
\end{equation*}

\begin{equation*}
\mathbf{w}^{\prime} =
\begin{pmatrix}
0.369085\\
0.369085\\
0.053406\\
0.208423
\end{pmatrix} =
0.988637\cdot
\begin{pmatrix}
0.373327\\
\color{gr} 0.373327\color{black} \\
0.054020\\
0.210819
\end{pmatrix},
\end{equation*}
\begin{equation*}
\left[ \frac{{w}^{\prime}_i}{{w}^{\prime}_j} \right] =
\begin{pmatrix}
$\,\,$ 1 $\,\,$ & $\,\,$\color{gr} \color{blue} 1\color{black} $\,\,$ & $\,\,$6.9109$\,\,$ & $\,\,$1.7708$\,\,$ \\
$\,\,$\color{gr} \color{blue} 1\color{black} $\,\,$ & $\,\,$ 1 $\,\,$ & $\,\,$\color{gr} 6.9109\color{black} $\,\,$ & $\,\,$\color{gr} 1.7708\color{black}   $\,\,$ \\
$\,\,$0.1447$\,\,$ & $\,\,$\color{gr} 0.1447\color{black} $\,\,$ & $\,\,$ 1 $\,\,$ & $\,\,$0.2562 $\,\,$ \\
$\,\,$0.5647$\,\,$ & $\,\,$\color{gr} 0.5647\color{black} $\,\,$ & $\,\,$3.9026$\,\,$ & $\,\,$ 1  $\,\,$ \\
\end{pmatrix},
\end{equation*}
\end{example}
\newpage
\begin{example}
\begin{equation*}
\mathbf{A} =
\begin{pmatrix}
$\,\,$ 1 $\,\,$ & $\,\,$1$\,\,$ & $\,\,$4$\,\,$ & $\,\,$4 $\,\,$ \\
$\,\,$ 1 $\,\,$ & $\,\,$ 1 $\,\,$ & $\,\,$6$\,\,$ & $\,\,$2 $\,\,$ \\
$\,\,$ 1/4$\,\,$ & $\,\,$ 1/6$\,\,$ & $\,\,$ 1 $\,\,$ & $\,\,$ 1/5 $\,\,$ \\
$\,\,$ 1/4$\,\,$ & $\,\,$ 1/2$\,\,$ & $\,\,$5$\,\,$ & $\,\,$ 1  $\,\,$ \\
\end{pmatrix},
\qquad
\lambda_{\max} =
4.2277,
\qquad
CR = 0.0859
\end{equation*}

\begin{equation*}
\mathbf{w}^{EM} =
\begin{pmatrix}
0.407662\\
\color{red} 0.351177\color{black} \\
0.060879\\
0.180282
\end{pmatrix}\end{equation*}
\begin{equation*}
\left[ \frac{{w}^{EM}_i}{{w}^{EM}_j} \right] =
\begin{pmatrix}
$\,\,$ 1 $\,\,$ & $\,\,$\color{red} 1.1608\color{black} $\,\,$ & $\,\,$6.6962$\,\,$ & $\,\,$2.2612$\,\,$ \\
$\,\,$\color{red} 0.8614\color{black} $\,\,$ & $\,\,$ 1 $\,\,$ & $\,\,$\color{red} 5.7684\color{black} $\,\,$ & $\,\,$\color{red} 1.9479\color{black}   $\,\,$ \\
$\,\,$0.1493$\,\,$ & $\,\,$\color{red} 0.1734\color{black} $\,\,$ & $\,\,$ 1 $\,\,$ & $\,\,$0.3377 $\,\,$ \\
$\,\,$0.4422$\,\,$ & $\,\,$\color{red} 0.5134\color{black} $\,\,$ & $\,\,$2.9613$\,\,$ & $\,\,$ 1  $\,\,$ \\
\end{pmatrix},
\end{equation*}

\begin{equation*}
\mathbf{w}^{\prime} =
\begin{pmatrix}
0.403871\\
0.357211\\
0.060313\\
0.178605
\end{pmatrix} =
0.990700\cdot
\begin{pmatrix}
0.407662\\
\color{gr} 0.360564\color{black} \\
0.060879\\
0.180282
\end{pmatrix},
\end{equation*}
\begin{equation*}
\left[ \frac{{w}^{\prime}_i}{{w}^{\prime}_j} \right] =
\begin{pmatrix}
$\,\,$ 1 $\,\,$ & $\,\,$\color{gr} 1.1306\color{black} $\,\,$ & $\,\,$6.6962$\,\,$ & $\,\,$2.2612$\,\,$ \\
$\,\,$\color{gr} 0.8845\color{black} $\,\,$ & $\,\,$ 1 $\,\,$ & $\,\,$\color{gr} 5.9226\color{black} $\,\,$ & $\,\,$\color{gr} \color{blue} 2\color{black}   $\,\,$ \\
$\,\,$0.1493$\,\,$ & $\,\,$\color{gr} 0.1688\color{black} $\,\,$ & $\,\,$ 1 $\,\,$ & $\,\,$0.3377 $\,\,$ \\
$\,\,$0.4422$\,\,$ & $\,\,$\color{gr} \color{blue}  1/2\color{black} $\,\,$ & $\,\,$2.9613$\,\,$ & $\,\,$ 1  $\,\,$ \\
\end{pmatrix},
\end{equation*}
\end{example}
\newpage
\begin{example}
\begin{equation*}
\mathbf{A} =
\begin{pmatrix}
$\,\,$ 1 $\,\,$ & $\,\,$1$\,\,$ & $\,\,$4$\,\,$ & $\,\,$4 $\,\,$ \\
$\,\,$ 1 $\,\,$ & $\,\,$ 1 $\,\,$ & $\,\,$6$\,\,$ & $\,\,$3 $\,\,$ \\
$\,\,$ 1/4$\,\,$ & $\,\,$ 1/6$\,\,$ & $\,\,$ 1 $\,\,$ & $\,\,$ 1/3 $\,\,$ \\
$\,\,$ 1/4$\,\,$ & $\,\,$ 1/3$\,\,$ & $\,\,$3$\,\,$ & $\,\,$ 1  $\,\,$ \\
\end{pmatrix},
\qquad
\lambda_{\max} =
4.1031,
\qquad
CR = 0.0389
\end{equation*}

\begin{equation*}
\mathbf{w}^{EM} =
\begin{pmatrix}
0.396198\\
\color{red} 0.395250\color{black} \\
0.068222\\
0.140330
\end{pmatrix}\end{equation*}
\begin{equation*}
\left[ \frac{{w}^{EM}_i}{{w}^{EM}_j} \right] =
\begin{pmatrix}
$\,\,$ 1 $\,\,$ & $\,\,$\color{red} 1.0024\color{black} $\,\,$ & $\,\,$5.8075$\,\,$ & $\,\,$2.8233$\,\,$ \\
$\,\,$\color{red} 0.9976\color{black} $\,\,$ & $\,\,$ 1 $\,\,$ & $\,\,$\color{red} 5.7936\color{black} $\,\,$ & $\,\,$\color{red} 2.8166\color{black}   $\,\,$ \\
$\,\,$0.1722$\,\,$ & $\,\,$\color{red} 0.1726\color{black} $\,\,$ & $\,\,$ 1 $\,\,$ & $\,\,$0.4862 $\,\,$ \\
$\,\,$0.3542$\,\,$ & $\,\,$\color{red} 0.3550\color{black} $\,\,$ & $\,\,$2.0570$\,\,$ & $\,\,$ 1  $\,\,$ \\
\end{pmatrix},
\end{equation*}

\begin{equation*}
\mathbf{w}^{\prime} =
\begin{pmatrix}
0.395823\\
0.395823\\
0.068157\\
0.140197
\end{pmatrix} =
0.999054\cdot
\begin{pmatrix}
0.396198\\
\color{gr} 0.396198\color{black} \\
0.068222\\
0.140330
\end{pmatrix},
\end{equation*}
\begin{equation*}
\left[ \frac{{w}^{\prime}_i}{{w}^{\prime}_j} \right] =
\begin{pmatrix}
$\,\,$ 1 $\,\,$ & $\,\,$\color{gr} \color{blue} 1\color{black} $\,\,$ & $\,\,$5.8075$\,\,$ & $\,\,$2.8233$\,\,$ \\
$\,\,$\color{gr} \color{blue} 1\color{black} $\,\,$ & $\,\,$ 1 $\,\,$ & $\,\,$\color{gr} 5.8075\color{black} $\,\,$ & $\,\,$\color{gr} 2.8233\color{black}   $\,\,$ \\
$\,\,$0.1722$\,\,$ & $\,\,$\color{gr} 0.1722\color{black} $\,\,$ & $\,\,$ 1 $\,\,$ & $\,\,$0.4862 $\,\,$ \\
$\,\,$0.3542$\,\,$ & $\,\,$\color{gr} 0.3542\color{black} $\,\,$ & $\,\,$2.0570$\,\,$ & $\,\,$ 1  $\,\,$ \\
\end{pmatrix},
\end{equation*}
\end{example}
\newpage
\begin{example}
\begin{equation*}
\mathbf{A} =
\begin{pmatrix}
$\,\,$ 1 $\,\,$ & $\,\,$1$\,\,$ & $\,\,$4$\,\,$ & $\,\,$5 $\,\,$ \\
$\,\,$ 1 $\,\,$ & $\,\,$ 1 $\,\,$ & $\,\,$6$\,\,$ & $\,\,$3 $\,\,$ \\
$\,\,$ 1/4$\,\,$ & $\,\,$ 1/6$\,\,$ & $\,\,$ 1 $\,\,$ & $\,\,$ 1/3 $\,\,$ \\
$\,\,$ 1/5$\,\,$ & $\,\,$ 1/3$\,\,$ & $\,\,$3$\,\,$ & $\,\,$ 1  $\,\,$ \\
\end{pmatrix},
\qquad
\lambda_{\max} =
4.1502,
\qquad
CR = 0.0566
\end{equation*}

\begin{equation*}
\mathbf{w}^{EM} =
\begin{pmatrix}
0.416127\\
\color{red} 0.385295\color{black} \\
0.067299\\
0.131279
\end{pmatrix}\end{equation*}
\begin{equation*}
\left[ \frac{{w}^{EM}_i}{{w}^{EM}_j} \right] =
\begin{pmatrix}
$\,\,$ 1 $\,\,$ & $\,\,$\color{red} 1.0800\color{black} $\,\,$ & $\,\,$6.1832$\,\,$ & $\,\,$3.1698$\,\,$ \\
$\,\,$\color{red} 0.9259\color{black} $\,\,$ & $\,\,$ 1 $\,\,$ & $\,\,$\color{red} 5.7251\color{black} $\,\,$ & $\,\,$\color{red} 2.9349\color{black}   $\,\,$ \\
$\,\,$0.1617$\,\,$ & $\,\,$\color{red} 0.1747\color{black} $\,\,$ & $\,\,$ 1 $\,\,$ & $\,\,$0.5126 $\,\,$ \\
$\,\,$0.3155$\,\,$ & $\,\,$\color{red} 0.3407\color{black} $\,\,$ & $\,\,$1.9507$\,\,$ & $\,\,$ 1  $\,\,$ \\
\end{pmatrix},
\end{equation*}

\begin{equation*}
\mathbf{w}^{\prime} =
\begin{pmatrix}
0.412603\\
0.390501\\
0.066729\\
0.130167
\end{pmatrix} =
0.991531\cdot
\begin{pmatrix}
0.416127\\
\color{gr} 0.393836\color{black} \\
0.067299\\
0.131279
\end{pmatrix},
\end{equation*}
\begin{equation*}
\left[ \frac{{w}^{\prime}_i}{{w}^{\prime}_j} \right] =
\begin{pmatrix}
$\,\,$ 1 $\,\,$ & $\,\,$\color{gr} 1.0566\color{black} $\,\,$ & $\,\,$6.1832$\,\,$ & $\,\,$3.1698$\,\,$ \\
$\,\,$\color{gr} 0.9464\color{black} $\,\,$ & $\,\,$ 1 $\,\,$ & $\,\,$\color{gr} 5.8520\color{black} $\,\,$ & $\,\,$\color{gr} \color{blue} 3\color{black}   $\,\,$ \\
$\,\,$0.1617$\,\,$ & $\,\,$\color{gr} 0.1709\color{black} $\,\,$ & $\,\,$ 1 $\,\,$ & $\,\,$0.5126 $\,\,$ \\
$\,\,$0.3155$\,\,$ & $\,\,$\color{gr} \color{blue}  1/3\color{black} $\,\,$ & $\,\,$1.9507$\,\,$ & $\,\,$ 1  $\,\,$ \\
\end{pmatrix},
\end{equation*}
\end{example}
\newpage
\begin{example}
\begin{equation*}
\mathbf{A} =
\begin{pmatrix}
$\,\,$ 1 $\,\,$ & $\,\,$1$\,\,$ & $\,\,$4$\,\,$ & $\,\,$5 $\,\,$ \\
$\,\,$ 1 $\,\,$ & $\,\,$ 1 $\,\,$ & $\,\,$7$\,\,$ & $\,\,$3 $\,\,$ \\
$\,\,$ 1/4$\,\,$ & $\,\,$ 1/7$\,\,$ & $\,\,$ 1 $\,\,$ & $\,\,$ 1/4 $\,\,$ \\
$\,\,$ 1/5$\,\,$ & $\,\,$ 1/3$\,\,$ & $\,\,$4$\,\,$ & $\,\,$ 1  $\,\,$ \\
\end{pmatrix},
\qquad
\lambda_{\max} =
4.2251,
\qquad
CR = 0.0849
\end{equation*}

\begin{equation*}
\mathbf{w}^{EM} =
\begin{pmatrix}
0.411846\\
\color{red} 0.388140\color{black} \\
0.059974\\
0.140041
\end{pmatrix}\end{equation*}
\begin{equation*}
\left[ \frac{{w}^{EM}_i}{{w}^{EM}_j} \right] =
\begin{pmatrix}
$\,\,$ 1 $\,\,$ & $\,\,$\color{red} 1.0611\color{black} $\,\,$ & $\,\,$6.8671$\,\,$ & $\,\,$2.9409$\,\,$ \\
$\,\,$\color{red} 0.9424\color{black} $\,\,$ & $\,\,$ 1 $\,\,$ & $\,\,$\color{red} 6.4718\color{black} $\,\,$ & $\,\,$\color{red} 2.7716\color{black}   $\,\,$ \\
$\,\,$0.1456$\,\,$ & $\,\,$\color{red} 0.1545\color{black} $\,\,$ & $\,\,$ 1 $\,\,$ & $\,\,$0.4283 $\,\,$ \\
$\,\,$0.3400$\,\,$ & $\,\,$\color{red} 0.3608\color{black} $\,\,$ & $\,\,$2.3350$\,\,$ & $\,\,$ 1  $\,\,$ \\
\end{pmatrix},
\end{equation*}

\begin{equation*}
\mathbf{w}^{\prime} =
\begin{pmatrix}
0.402309\\
0.402309\\
0.058585\\
0.136798
\end{pmatrix} =
0.976843\cdot
\begin{pmatrix}
0.411846\\
\color{gr} 0.411846\color{black} \\
0.059974\\
0.140041
\end{pmatrix},
\end{equation*}
\begin{equation*}
\left[ \frac{{w}^{\prime}_i}{{w}^{\prime}_j} \right] =
\begin{pmatrix}
$\,\,$ 1 $\,\,$ & $\,\,$\color{gr} \color{blue} 1\color{black} $\,\,$ & $\,\,$6.8671$\,\,$ & $\,\,$2.9409$\,\,$ \\
$\,\,$\color{gr} \color{blue} 1\color{black} $\,\,$ & $\,\,$ 1 $\,\,$ & $\,\,$\color{gr} 6.8671\color{black} $\,\,$ & $\,\,$\color{gr} 2.9409\color{black}   $\,\,$ \\
$\,\,$0.1456$\,\,$ & $\,\,$\color{gr} 0.1456\color{black} $\,\,$ & $\,\,$ 1 $\,\,$ & $\,\,$0.4283 $\,\,$ \\
$\,\,$0.3400$\,\,$ & $\,\,$\color{gr} 0.3400\color{black} $\,\,$ & $\,\,$2.3350$\,\,$ & $\,\,$ 1  $\,\,$ \\
\end{pmatrix},
\end{equation*}
\end{example}
\newpage
\begin{example}
\begin{equation*}
\mathbf{A} =
\begin{pmatrix}
$\,\,$ 1 $\,\,$ & $\,\,$1$\,\,$ & $\,\,$4$\,\,$ & $\,\,$5 $\,\,$ \\
$\,\,$ 1 $\,\,$ & $\,\,$ 1 $\,\,$ & $\,\,$8$\,\,$ & $\,\,$3 $\,\,$ \\
$\,\,$ 1/4$\,\,$ & $\,\,$ 1/8$\,\,$ & $\,\,$ 1 $\,\,$ & $\,\,$ 1/4 $\,\,$ \\
$\,\,$ 1/5$\,\,$ & $\,\,$ 1/3$\,\,$ & $\,\,$4$\,\,$ & $\,\,$ 1  $\,\,$ \\
\end{pmatrix},
\qquad
\lambda_{\max} =
4.2277,
\qquad
CR = 0.0859
\end{equation*}

\begin{equation*}
\mathbf{w}^{EM} =
\begin{pmatrix}
0.407667\\
\color{red} 0.397054\color{black} \\
0.057615\\
0.137665
\end{pmatrix}\end{equation*}
\begin{equation*}
\left[ \frac{{w}^{EM}_i}{{w}^{EM}_j} \right] =
\begin{pmatrix}
$\,\,$ 1 $\,\,$ & $\,\,$\color{red} 1.0267\color{black} $\,\,$ & $\,\,$7.0757$\,\,$ & $\,\,$2.9613$\,\,$ \\
$\,\,$\color{red} 0.9740\color{black} $\,\,$ & $\,\,$ 1 $\,\,$ & $\,\,$\color{red} 6.8915\color{black} $\,\,$ & $\,\,$\color{red} 2.8842\color{black}   $\,\,$ \\
$\,\,$0.1413$\,\,$ & $\,\,$\color{red} 0.1451\color{black} $\,\,$ & $\,\,$ 1 $\,\,$ & $\,\,$0.4185 $\,\,$ \\
$\,\,$0.3377$\,\,$ & $\,\,$\color{red} 0.3467\color{black} $\,\,$ & $\,\,$2.3894$\,\,$ & $\,\,$ 1  $\,\,$ \\
\end{pmatrix},
\end{equation*}

\begin{equation*}
\mathbf{w}^{\prime} =
\begin{pmatrix}
0.403386\\
0.403386\\
0.057010\\
0.136219
\end{pmatrix} =
0.989498\cdot
\begin{pmatrix}
0.407667\\
\color{gr} 0.407667\color{black} \\
0.057615\\
0.137665
\end{pmatrix},
\end{equation*}
\begin{equation*}
\left[ \frac{{w}^{\prime}_i}{{w}^{\prime}_j} \right] =
\begin{pmatrix}
$\,\,$ 1 $\,\,$ & $\,\,$\color{gr} \color{blue} 1\color{black} $\,\,$ & $\,\,$7.0757$\,\,$ & $\,\,$2.9613$\,\,$ \\
$\,\,$\color{gr} \color{blue} 1\color{black} $\,\,$ & $\,\,$ 1 $\,\,$ & $\,\,$\color{gr} 7.0757\color{black} $\,\,$ & $\,\,$\color{gr} 2.9613\color{black}   $\,\,$ \\
$\,\,$0.1413$\,\,$ & $\,\,$\color{gr} 0.1413\color{black} $\,\,$ & $\,\,$ 1 $\,\,$ & $\,\,$0.4185 $\,\,$ \\
$\,\,$0.3377$\,\,$ & $\,\,$\color{gr} 0.3377\color{black} $\,\,$ & $\,\,$2.3894$\,\,$ & $\,\,$ 1  $\,\,$ \\
\end{pmatrix},
\end{equation*}
\end{example}
\newpage
\begin{example}
\begin{equation*}
\mathbf{A} =
\begin{pmatrix}
$\,\,$ 1 $\,\,$ & $\,\,$1$\,\,$ & $\,\,$4$\,\,$ & $\,\,$6 $\,\,$ \\
$\,\,$ 1 $\,\,$ & $\,\,$ 1 $\,\,$ & $\,\,$6$\,\,$ & $\,\,$4 $\,\,$ \\
$\,\,$ 1/4$\,\,$ & $\,\,$ 1/6$\,\,$ & $\,\,$ 1 $\,\,$ & $\,\,$ 1/2 $\,\,$ \\
$\,\,$ 1/6$\,\,$ & $\,\,$ 1/4$\,\,$ & $\,\,$2$\,\,$ & $\,\,$ 1  $\,\,$ \\
\end{pmatrix},
\qquad
\lambda_{\max} =
4.1031,
\qquad
CR = 0.0389
\end{equation*}

\begin{equation*}
\mathbf{w}^{EM} =
\begin{pmatrix}
0.420345\\
\color{red} 0.405887\color{black} \\
0.072053\\
0.101715
\end{pmatrix}\end{equation*}
\begin{equation*}
\left[ \frac{{w}^{EM}_i}{{w}^{EM}_j} \right] =
\begin{pmatrix}
$\,\,$ 1 $\,\,$ & $\,\,$\color{red} 1.0356\color{black} $\,\,$ & $\,\,$5.8338$\,\,$ & $\,\,$4.1326$\,\,$ \\
$\,\,$\color{red} 0.9656\color{black} $\,\,$ & $\,\,$ 1 $\,\,$ & $\,\,$\color{red} 5.6331\color{black} $\,\,$ & $\,\,$\color{red} 3.9904\color{black}   $\,\,$ \\
$\,\,$0.1714$\,\,$ & $\,\,$\color{red} 0.1775\color{black} $\,\,$ & $\,\,$ 1 $\,\,$ & $\,\,$0.7084 $\,\,$ \\
$\,\,$0.2420$\,\,$ & $\,\,$\color{red} 0.2506\color{black} $\,\,$ & $\,\,$1.4117$\,\,$ & $\,\,$ 1  $\,\,$ \\
\end{pmatrix},
\end{equation*}

\begin{equation*}
\mathbf{w}^{\prime} =
\begin{pmatrix}
0.419936\\
0.406464\\
0.071983\\
0.101616
\end{pmatrix} =
0.999028\cdot
\begin{pmatrix}
0.420345\\
\color{gr} 0.406860\color{black} \\
0.072053\\
0.101715
\end{pmatrix},
\end{equation*}
\begin{equation*}
\left[ \frac{{w}^{\prime}_i}{{w}^{\prime}_j} \right] =
\begin{pmatrix}
$\,\,$ 1 $\,\,$ & $\,\,$\color{gr} 1.0331\color{black} $\,\,$ & $\,\,$5.8338$\,\,$ & $\,\,$4.1326$\,\,$ \\
$\,\,$\color{gr} 0.9679\color{black} $\,\,$ & $\,\,$ 1 $\,\,$ & $\,\,$\color{gr} 5.6467\color{black} $\,\,$ & $\,\,$\color{gr} \color{blue} 4\color{black}   $\,\,$ \\
$\,\,$0.1714$\,\,$ & $\,\,$\color{gr} 0.1771\color{black} $\,\,$ & $\,\,$ 1 $\,\,$ & $\,\,$0.7084 $\,\,$ \\
$\,\,$0.2420$\,\,$ & $\,\,$\color{gr} \color{blue}  1/4\color{black} $\,\,$ & $\,\,$1.4117$\,\,$ & $\,\,$ 1  $\,\,$ \\
\end{pmatrix},
\end{equation*}
\end{example}
\newpage
\begin{example}
\begin{equation*}
\mathbf{A} =
\begin{pmatrix}
$\,\,$ 1 $\,\,$ & $\,\,$1$\,\,$ & $\,\,$4$\,\,$ & $\,\,$7 $\,\,$ \\
$\,\,$ 1 $\,\,$ & $\,\,$ 1 $\,\,$ & $\,\,$7$\,\,$ & $\,\,$4 $\,\,$ \\
$\,\,$ 1/4$\,\,$ & $\,\,$ 1/7$\,\,$ & $\,\,$ 1 $\,\,$ & $\,\,$ 1/3 $\,\,$ \\
$\,\,$ 1/7$\,\,$ & $\,\,$ 1/4$\,\,$ & $\,\,$3$\,\,$ & $\,\,$ 1  $\,\,$ \\
\end{pmatrix},
\qquad
\lambda_{\max} =
4.2395,
\qquad
CR = 0.0903
\end{equation*}

\begin{equation*}
\mathbf{w}^{EM} =
\begin{pmatrix}
0.431442\\
\color{red} 0.399407\color{black} \\
0.061941\\
0.107211
\end{pmatrix}\end{equation*}
\begin{equation*}
\left[ \frac{{w}^{EM}_i}{{w}^{EM}_j} \right] =
\begin{pmatrix}
$\,\,$ 1 $\,\,$ & $\,\,$\color{red} 1.0802\color{black} $\,\,$ & $\,\,$6.9654$\,\,$ & $\,\,$4.0242$\,\,$ \\
$\,\,$\color{red} 0.9257\color{black} $\,\,$ & $\,\,$ 1 $\,\,$ & $\,\,$\color{red} 6.4482\color{black} $\,\,$ & $\,\,$\color{red} 3.7254\color{black}   $\,\,$ \\
$\,\,$0.1436$\,\,$ & $\,\,$\color{red} 0.1551\color{black} $\,\,$ & $\,\,$ 1 $\,\,$ & $\,\,$0.5777 $\,\,$ \\
$\,\,$0.2485$\,\,$ & $\,\,$\color{red} 0.2684\color{black} $\,\,$ & $\,\,$1.7309$\,\,$ & $\,\,$ 1  $\,\,$ \\
\end{pmatrix},
\end{equation*}

\begin{equation*}
\mathbf{w}^{\prime} =
\begin{pmatrix}
0.419104\\
0.416581\\
0.060169\\
0.104145
\end{pmatrix} =
0.971404\cdot
\begin{pmatrix}
0.431442\\
\color{gr} 0.428844\color{black} \\
0.061941\\
0.107211
\end{pmatrix},
\end{equation*}
\begin{equation*}
\left[ \frac{{w}^{\prime}_i}{{w}^{\prime}_j} \right] =
\begin{pmatrix}
$\,\,$ 1 $\,\,$ & $\,\,$\color{gr} 1.0061\color{black} $\,\,$ & $\,\,$6.9654$\,\,$ & $\,\,$4.0242$\,\,$ \\
$\,\,$\color{gr} 0.9940\color{black} $\,\,$ & $\,\,$ 1 $\,\,$ & $\,\,$\color{gr} 6.9235\color{black} $\,\,$ & $\,\,$\color{gr} \color{blue} 4\color{black}   $\,\,$ \\
$\,\,$0.1436$\,\,$ & $\,\,$\color{gr} 0.1444\color{black} $\,\,$ & $\,\,$ 1 $\,\,$ & $\,\,$0.5777 $\,\,$ \\
$\,\,$0.2485$\,\,$ & $\,\,$\color{gr} \color{blue}  1/4\color{black} $\,\,$ & $\,\,$1.7309$\,\,$ & $\,\,$ 1  $\,\,$ \\
\end{pmatrix},
\end{equation*}
\end{example}
\newpage
\begin{example}
\begin{equation*}
\mathbf{A} =
\begin{pmatrix}
$\,\,$ 1 $\,\,$ & $\,\,$1$\,\,$ & $\,\,$5$\,\,$ & $\,\,$3 $\,\,$ \\
$\,\,$ 1 $\,\,$ & $\,\,$ 1 $\,\,$ & $\,\,$7$\,\,$ & $\,\,$2 $\,\,$ \\
$\,\,$ 1/5$\,\,$ & $\,\,$ 1/7$\,\,$ & $\,\,$ 1 $\,\,$ & $\,\,$ 1/5 $\,\,$ \\
$\,\,$ 1/3$\,\,$ & $\,\,$ 1/2$\,\,$ & $\,\,$5$\,\,$ & $\,\,$ 1  $\,\,$ \\
\end{pmatrix},
\qquad
\lambda_{\max} =
4.1027,
\qquad
CR = 0.0387
\end{equation*}

\begin{equation*}
\mathbf{w}^{EM} =
\begin{pmatrix}
0.388439\\
\color{red} 0.368907\color{black} \\
0.054173\\
0.188480
\end{pmatrix}\end{equation*}
\begin{equation*}
\left[ \frac{{w}^{EM}_i}{{w}^{EM}_j} \right] =
\begin{pmatrix}
$\,\,$ 1 $\,\,$ & $\,\,$\color{red} 1.0529\color{black} $\,\,$ & $\,\,$7.1703$\,\,$ & $\,\,$2.0609$\,\,$ \\
$\,\,$\color{red} 0.9497\color{black} $\,\,$ & $\,\,$ 1 $\,\,$ & $\,\,$\color{red} 6.8097\color{black} $\,\,$ & $\,\,$\color{red} 1.9573\color{black}   $\,\,$ \\
$\,\,$0.1395$\,\,$ & $\,\,$\color{red} 0.1468\color{black} $\,\,$ & $\,\,$ 1 $\,\,$ & $\,\,$0.2874 $\,\,$ \\
$\,\,$0.4852$\,\,$ & $\,\,$\color{red} 0.5109\color{black} $\,\,$ & $\,\,$3.4792$\,\,$ & $\,\,$ 1  $\,\,$ \\
\end{pmatrix},
\end{equation*}

\begin{equation*}
\mathbf{w}^{\prime} =
\begin{pmatrix}
0.385336\\
0.373949\\
0.053741\\
0.186975
\end{pmatrix} =
0.992011\cdot
\begin{pmatrix}
0.388439\\
\color{gr} 0.376961\color{black} \\
0.054173\\
0.188480
\end{pmatrix},
\end{equation*}
\begin{equation*}
\left[ \frac{{w}^{\prime}_i}{{w}^{\prime}_j} \right] =
\begin{pmatrix}
$\,\,$ 1 $\,\,$ & $\,\,$\color{gr} 1.0304\color{black} $\,\,$ & $\,\,$7.1703$\,\,$ & $\,\,$2.0609$\,\,$ \\
$\,\,$\color{gr} 0.9704\color{black} $\,\,$ & $\,\,$ 1 $\,\,$ & $\,\,$\color{gr} 6.9584\color{black} $\,\,$ & $\,\,$\color{gr} \color{blue} 2\color{black}   $\,\,$ \\
$\,\,$0.1395$\,\,$ & $\,\,$\color{gr} 0.1437\color{black} $\,\,$ & $\,\,$ 1 $\,\,$ & $\,\,$0.2874 $\,\,$ \\
$\,\,$0.4852$\,\,$ & $\,\,$\color{gr} \color{blue}  1/2\color{black} $\,\,$ & $\,\,$3.4792$\,\,$ & $\,\,$ 1  $\,\,$ \\
\end{pmatrix},
\end{equation*}
\end{example}
\newpage
\begin{example}
\begin{equation*}
\mathbf{A} =
\begin{pmatrix}
$\,\,$ 1 $\,\,$ & $\,\,$1$\,\,$ & $\,\,$5$\,\,$ & $\,\,$3 $\,\,$ \\
$\,\,$ 1 $\,\,$ & $\,\,$ 1 $\,\,$ & $\,\,$8$\,\,$ & $\,\,$2 $\,\,$ \\
$\,\,$ 1/5$\,\,$ & $\,\,$ 1/8$\,\,$ & $\,\,$ 1 $\,\,$ & $\,\,$ 1/6 $\,\,$ \\
$\,\,$ 1/3$\,\,$ & $\,\,$ 1/2$\,\,$ & $\,\,$6$\,\,$ & $\,\,$ 1  $\,\,$ \\
\end{pmatrix},
\qquad
\lambda_{\max} =
4.1406,
\qquad
CR = 0.0530
\end{equation*}

\begin{equation*}
\mathbf{w}^{EM} =
\begin{pmatrix}
0.383430\\
\color{red} 0.372328\color{black} \\
0.049568\\
0.194673
\end{pmatrix}\end{equation*}
\begin{equation*}
\left[ \frac{{w}^{EM}_i}{{w}^{EM}_j} \right] =
\begin{pmatrix}
$\,\,$ 1 $\,\,$ & $\,\,$\color{red} 1.0298\color{black} $\,\,$ & $\,\,$7.7354$\,\,$ & $\,\,$1.9696$\,\,$ \\
$\,\,$\color{red} 0.9710\color{black} $\,\,$ & $\,\,$ 1 $\,\,$ & $\,\,$\color{red} 7.5114\color{black} $\,\,$ & $\,\,$\color{red} 1.9126\color{black}   $\,\,$ \\
$\,\,$0.1293$\,\,$ & $\,\,$\color{red} 0.1331\color{black} $\,\,$ & $\,\,$ 1 $\,\,$ & $\,\,$0.2546 $\,\,$ \\
$\,\,$0.5077$\,\,$ & $\,\,$\color{red} 0.5229\color{black} $\,\,$ & $\,\,$3.9274$\,\,$ & $\,\,$ 1  $\,\,$ \\
\end{pmatrix},
\end{equation*}

\begin{equation*}
\mathbf{w}^{\prime} =
\begin{pmatrix}
0.379220\\
0.379220\\
0.049024\\
0.192536
\end{pmatrix} =
0.989020\cdot
\begin{pmatrix}
0.383430\\
\color{gr} 0.383430\color{black} \\
0.049568\\
0.194673
\end{pmatrix},
\end{equation*}
\begin{equation*}
\left[ \frac{{w}^{\prime}_i}{{w}^{\prime}_j} \right] =
\begin{pmatrix}
$\,\,$ 1 $\,\,$ & $\,\,$\color{gr} \color{blue} 1\color{black} $\,\,$ & $\,\,$7.7354$\,\,$ & $\,\,$1.9696$\,\,$ \\
$\,\,$\color{gr} \color{blue} 1\color{black} $\,\,$ & $\,\,$ 1 $\,\,$ & $\,\,$\color{gr} 7.7354\color{black} $\,\,$ & $\,\,$\color{gr} 1.9696\color{black}   $\,\,$ \\
$\,\,$0.1293$\,\,$ & $\,\,$\color{gr} 0.1293\color{black} $\,\,$ & $\,\,$ 1 $\,\,$ & $\,\,$0.2546 $\,\,$ \\
$\,\,$0.5077$\,\,$ & $\,\,$\color{gr} 0.5077\color{black} $\,\,$ & $\,\,$3.9274$\,\,$ & $\,\,$ 1  $\,\,$ \\
\end{pmatrix},
\end{equation*}
\end{example}
\newpage
\begin{example}
\begin{equation*}
\mathbf{A} =
\begin{pmatrix}
$\,\,$ 1 $\,\,$ & $\,\,$1$\,\,$ & $\,\,$5$\,\,$ & $\,\,$3 $\,\,$ \\
$\,\,$ 1 $\,\,$ & $\,\,$ 1 $\,\,$ & $\,\,$8$\,\,$ & $\,\,$2 $\,\,$ \\
$\,\,$ 1/5$\,\,$ & $\,\,$ 1/8$\,\,$ & $\,\,$ 1 $\,\,$ & $\,\,$ 1/7 $\,\,$ \\
$\,\,$ 1/3$\,\,$ & $\,\,$ 1/2$\,\,$ & $\,\,$7$\,\,$ & $\,\,$ 1  $\,\,$ \\
\end{pmatrix},
\qquad
\lambda_{\max} =
4.1782,
\qquad
CR = 0.0672
\end{equation*}

\begin{equation*}
\mathbf{w}^{EM} =
\begin{pmatrix}
0.381982\\
\color{red} 0.367638\color{black} \\
0.047612\\
0.202768
\end{pmatrix}\end{equation*}
\begin{equation*}
\left[ \frac{{w}^{EM}_i}{{w}^{EM}_j} \right] =
\begin{pmatrix}
$\,\,$ 1 $\,\,$ & $\,\,$\color{red} 1.0390\color{black} $\,\,$ & $\,\,$8.0228$\,\,$ & $\,\,$1.8838$\,\,$ \\
$\,\,$\color{red} 0.9624\color{black} $\,\,$ & $\,\,$ 1 $\,\,$ & $\,\,$\color{red} 7.7216\color{black} $\,\,$ & $\,\,$\color{red} 1.8131\color{black}   $\,\,$ \\
$\,\,$0.1246$\,\,$ & $\,\,$\color{red} 0.1295\color{black} $\,\,$ & $\,\,$ 1 $\,\,$ & $\,\,$0.2348 $\,\,$ \\
$\,\,$0.5308$\,\,$ & $\,\,$\color{red} 0.5515\color{black} $\,\,$ & $\,\,$4.2588$\,\,$ & $\,\,$ 1  $\,\,$ \\
\end{pmatrix},
\end{equation*}

\begin{equation*}
\mathbf{w}^{\prime} =
\begin{pmatrix}
0.376985\\
0.375911\\
0.046989\\
0.200115
\end{pmatrix} =
0.986917\cdot
\begin{pmatrix}
0.381982\\
\color{gr} 0.380895\color{black} \\
0.047612\\
0.202768
\end{pmatrix},
\end{equation*}
\begin{equation*}
\left[ \frac{{w}^{\prime}_i}{{w}^{\prime}_j} \right] =
\begin{pmatrix}
$\,\,$ 1 $\,\,$ & $\,\,$\color{gr} 1.0029\color{black} $\,\,$ & $\,\,$8.0228$\,\,$ & $\,\,$1.8838$\,\,$ \\
$\,\,$\color{gr} 0.9972\color{black} $\,\,$ & $\,\,$ 1 $\,\,$ & $\,\,$\color{gr} \color{blue} 8\color{black} $\,\,$ & $\,\,$\color{gr} 1.8785\color{black}   $\,\,$ \\
$\,\,$0.1246$\,\,$ & $\,\,$\color{gr} \color{blue}  1/8\color{black} $\,\,$ & $\,\,$ 1 $\,\,$ & $\,\,$0.2348 $\,\,$ \\
$\,\,$0.5308$\,\,$ & $\,\,$\color{gr} 0.5323\color{black} $\,\,$ & $\,\,$4.2588$\,\,$ & $\,\,$ 1  $\,\,$ \\
\end{pmatrix},
\end{equation*}
\end{example}
\newpage
\begin{example}
\begin{equation*}
\mathbf{A} =
\begin{pmatrix}
$\,\,$ 1 $\,\,$ & $\,\,$1$\,\,$ & $\,\,$5$\,\,$ & $\,\,$3 $\,\,$ \\
$\,\,$ 1 $\,\,$ & $\,\,$ 1 $\,\,$ & $\,\,$8$\,\,$ & $\,\,$2 $\,\,$ \\
$\,\,$ 1/5$\,\,$ & $\,\,$ 1/8$\,\,$ & $\,\,$ 1 $\,\,$ & $\,\,$ 1/8 $\,\,$ \\
$\,\,$ 1/3$\,\,$ & $\,\,$ 1/2$\,\,$ & $\,\,$8$\,\,$ & $\,\,$ 1  $\,\,$ \\
\end{pmatrix},
\qquad
\lambda_{\max} =
4.2162,
\qquad
CR = 0.0815
\end{equation*}

\begin{equation*}
\mathbf{w}^{EM} =
\begin{pmatrix}
0.380495\\
\color{red} 0.363332\color{black} \\
0.045952\\
0.210221
\end{pmatrix}\end{equation*}
\begin{equation*}
\left[ \frac{{w}^{EM}_i}{{w}^{EM}_j} \right] =
\begin{pmatrix}
$\,\,$ 1 $\,\,$ & $\,\,$\color{red} 1.0472\color{black} $\,\,$ & $\,\,$8.2802$\,\,$ & $\,\,$1.8100$\,\,$ \\
$\,\,$\color{red} 0.9549\color{black} $\,\,$ & $\,\,$ 1 $\,\,$ & $\,\,$\color{red} 7.9067\color{black} $\,\,$ & $\,\,$\color{red} 1.7283\color{black}   $\,\,$ \\
$\,\,$0.1208$\,\,$ & $\,\,$\color{red} 0.1265\color{black} $\,\,$ & $\,\,$ 1 $\,\,$ & $\,\,$0.2186 $\,\,$ \\
$\,\,$0.5525$\,\,$ & $\,\,$\color{red} 0.5786\color{black} $\,\,$ & $\,\,$4.5748$\,\,$ & $\,\,$ 1  $\,\,$ \\
\end{pmatrix},
\end{equation*}

\begin{equation*}
\mathbf{w}^{\prime} =
\begin{pmatrix}
0.378871\\
0.366050\\
0.045756\\
0.209323
\end{pmatrix} =
0.995731\cdot
\begin{pmatrix}
0.380495\\
\color{gr} 0.367619\color{black} \\
0.045952\\
0.210221
\end{pmatrix},
\end{equation*}
\begin{equation*}
\left[ \frac{{w}^{\prime}_i}{{w}^{\prime}_j} \right] =
\begin{pmatrix}
$\,\,$ 1 $\,\,$ & $\,\,$\color{gr} 1.0350\color{black} $\,\,$ & $\,\,$8.2802$\,\,$ & $\,\,$1.8100$\,\,$ \\
$\,\,$\color{gr} 0.9662\color{black} $\,\,$ & $\,\,$ 1 $\,\,$ & $\,\,$\color{gr} \color{blue} 8\color{black} $\,\,$ & $\,\,$\color{gr} 1.7487\color{black}   $\,\,$ \\
$\,\,$0.1208$\,\,$ & $\,\,$\color{gr} \color{blue}  1/8\color{black} $\,\,$ & $\,\,$ 1 $\,\,$ & $\,\,$0.2186 $\,\,$ \\
$\,\,$0.5525$\,\,$ & $\,\,$\color{gr} 0.5718\color{black} $\,\,$ & $\,\,$4.5748$\,\,$ & $\,\,$ 1  $\,\,$ \\
\end{pmatrix},
\end{equation*}
\end{example}
\newpage
\begin{example}
\begin{equation*}
\mathbf{A} =
\begin{pmatrix}
$\,\,$ 1 $\,\,$ & $\,\,$1$\,\,$ & $\,\,$5$\,\,$ & $\,\,$3 $\,\,$ \\
$\,\,$ 1 $\,\,$ & $\,\,$ 1 $\,\,$ & $\,\,$9$\,\,$ & $\,\,$2 $\,\,$ \\
$\,\,$ 1/5$\,\,$ & $\,\,$ 1/9$\,\,$ & $\,\,$ 1 $\,\,$ & $\,\,$ 1/6 $\,\,$ \\
$\,\,$ 1/3$\,\,$ & $\,\,$ 1/2$\,\,$ & $\,\,$6$\,\,$ & $\,\,$ 1  $\,\,$ \\
\end{pmatrix},
\qquad
\lambda_{\max} =
4.1433,
\qquad
CR = 0.0540
\end{equation*}

\begin{equation*}
\mathbf{w}^{EM} =
\begin{pmatrix}
0.380189\\
\color{red} 0.379996\color{black} \\
0.047804\\
0.192011
\end{pmatrix}\end{equation*}
\begin{equation*}
\left[ \frac{{w}^{EM}_i}{{w}^{EM}_j} \right] =
\begin{pmatrix}
$\,\,$ 1 $\,\,$ & $\,\,$\color{red} 1.0005\color{black} $\,\,$ & $\,\,$7.9531$\,\,$ & $\,\,$1.9800$\,\,$ \\
$\,\,$\color{red} 0.9995\color{black} $\,\,$ & $\,\,$ 1 $\,\,$ & $\,\,$\color{red} 7.9491\color{black} $\,\,$ & $\,\,$\color{red} 1.9790\color{black}   $\,\,$ \\
$\,\,$0.1257$\,\,$ & $\,\,$\color{red} 0.1258\color{black} $\,\,$ & $\,\,$ 1 $\,\,$ & $\,\,$0.2490 $\,\,$ \\
$\,\,$0.5050$\,\,$ & $\,\,$\color{red} 0.5053\color{black} $\,\,$ & $\,\,$4.0167$\,\,$ & $\,\,$ 1  $\,\,$ \\
\end{pmatrix},
\end{equation*}

\begin{equation*}
\mathbf{w}^{\prime} =
\begin{pmatrix}
0.380116\\
0.380116\\
0.047794\\
0.191974
\end{pmatrix} =
0.999808\cdot
\begin{pmatrix}
0.380189\\
\color{gr} 0.380189\color{black} \\
0.047804\\
0.192011
\end{pmatrix},
\end{equation*}
\begin{equation*}
\left[ \frac{{w}^{\prime}_i}{{w}^{\prime}_j} \right] =
\begin{pmatrix}
$\,\,$ 1 $\,\,$ & $\,\,$\color{gr} \color{blue} 1\color{black} $\,\,$ & $\,\,$7.9531$\,\,$ & $\,\,$1.9800$\,\,$ \\
$\,\,$\color{gr} \color{blue} 1\color{black} $\,\,$ & $\,\,$ 1 $\,\,$ & $\,\,$\color{gr} 7.9531\color{black} $\,\,$ & $\,\,$\color{gr} 1.9800\color{black}   $\,\,$ \\
$\,\,$0.1257$\,\,$ & $\,\,$\color{gr} 0.1257\color{black} $\,\,$ & $\,\,$ 1 $\,\,$ & $\,\,$0.2490 $\,\,$ \\
$\,\,$0.5050$\,\,$ & $\,\,$\color{gr} 0.5050\color{black} $\,\,$ & $\,\,$4.0167$\,\,$ & $\,\,$ 1  $\,\,$ \\
\end{pmatrix},
\end{equation*}
\end{example}
\newpage
\begin{example}
\begin{equation*}
\mathbf{A} =
\begin{pmatrix}
$\,\,$ 1 $\,\,$ & $\,\,$1$\,\,$ & $\,\,$5$\,\,$ & $\,\,$3 $\,\,$ \\
$\,\,$ 1 $\,\,$ & $\,\,$ 1 $\,\,$ & $\,\,$9$\,\,$ & $\,\,$2 $\,\,$ \\
$\,\,$ 1/5$\,\,$ & $\,\,$ 1/9$\,\,$ & $\,\,$ 1 $\,\,$ & $\,\,$ 1/7 $\,\,$ \\
$\,\,$ 1/3$\,\,$ & $\,\,$ 1/2$\,\,$ & $\,\,$7$\,\,$ & $\,\,$ 1  $\,\,$ \\
\end{pmatrix},
\qquad
\lambda_{\max} =
4.1786,
\qquad
CR = 0.0673
\end{equation*}

\begin{equation*}
\mathbf{w}^{EM} =
\begin{pmatrix}
0.378992\\
\color{red} 0.375127\color{black} \\
0.045945\\
0.199935
\end{pmatrix}\end{equation*}
\begin{equation*}
\left[ \frac{{w}^{EM}_i}{{w}^{EM}_j} \right] =
\begin{pmatrix}
$\,\,$ 1 $\,\,$ & $\,\,$\color{red} 1.0103\color{black} $\,\,$ & $\,\,$8.2488$\,\,$ & $\,\,$1.8956$\,\,$ \\
$\,\,$\color{red} 0.9898\color{black} $\,\,$ & $\,\,$ 1 $\,\,$ & $\,\,$\color{red} 8.1646\color{black} $\,\,$ & $\,\,$\color{red} 1.8762\color{black}   $\,\,$ \\
$\,\,$0.1212$\,\,$ & $\,\,$\color{red} 0.1225\color{black} $\,\,$ & $\,\,$ 1 $\,\,$ & $\,\,$0.2298 $\,\,$ \\
$\,\,$0.5275$\,\,$ & $\,\,$\color{red} 0.5330\color{black} $\,\,$ & $\,\,$4.3516$\,\,$ & $\,\,$ 1  $\,\,$ \\
\end{pmatrix},
\end{equation*}

\begin{equation*}
\mathbf{w}^{\prime} =
\begin{pmatrix}
0.377533\\
0.377533\\
0.045768\\
0.199166
\end{pmatrix} =
0.996149\cdot
\begin{pmatrix}
0.378992\\
\color{gr} 0.378992\color{black} \\
0.045945\\
0.199935
\end{pmatrix},
\end{equation*}
\begin{equation*}
\left[ \frac{{w}^{\prime}_i}{{w}^{\prime}_j} \right] =
\begin{pmatrix}
$\,\,$ 1 $\,\,$ & $\,\,$\color{gr} \color{blue} 1\color{black} $\,\,$ & $\,\,$8.2488$\,\,$ & $\,\,$1.8956$\,\,$ \\
$\,\,$\color{gr} \color{blue} 1\color{black} $\,\,$ & $\,\,$ 1 $\,\,$ & $\,\,$\color{gr} 8.2488\color{black} $\,\,$ & $\,\,$\color{gr} 1.8956\color{black}   $\,\,$ \\
$\,\,$0.1212$\,\,$ & $\,\,$\color{gr} 0.1212\color{black} $\,\,$ & $\,\,$ 1 $\,\,$ & $\,\,$0.2298 $\,\,$ \\
$\,\,$0.5275$\,\,$ & $\,\,$\color{gr} 0.5275\color{black} $\,\,$ & $\,\,$4.3516$\,\,$ & $\,\,$ 1  $\,\,$ \\
\end{pmatrix},
\end{equation*}
\end{example}
\newpage
\begin{example}
\begin{equation*}
\mathbf{A} =
\begin{pmatrix}
$\,\,$ 1 $\,\,$ & $\,\,$1$\,\,$ & $\,\,$5$\,\,$ & $\,\,$3 $\,\,$ \\
$\,\,$ 1 $\,\,$ & $\,\,$ 1 $\,\,$ & $\,\,$9$\,\,$ & $\,\,$2 $\,\,$ \\
$\,\,$ 1/5$\,\,$ & $\,\,$ 1/9$\,\,$ & $\,\,$ 1 $\,\,$ & $\,\,$ 1/8 $\,\,$ \\
$\,\,$ 1/3$\,\,$ & $\,\,$ 1/2$\,\,$ & $\,\,$8$\,\,$ & $\,\,$ 1  $\,\,$ \\
\end{pmatrix},
\qquad
\lambda_{\max} =
4.2146,
\qquad
CR = 0.0809
\end{equation*}

\begin{equation*}
\mathbf{w}^{EM} =
\begin{pmatrix}
0.377724\\
\color{red} 0.370663\color{black} \\
0.044371\\
0.207242
\end{pmatrix}\end{equation*}
\begin{equation*}
\left[ \frac{{w}^{EM}_i}{{w}^{EM}_j} \right] =
\begin{pmatrix}
$\,\,$ 1 $\,\,$ & $\,\,$\color{red} 1.0190\color{black} $\,\,$ & $\,\,$8.5129$\,\,$ & $\,\,$1.8226$\,\,$ \\
$\,\,$\color{red} 0.9813\color{black} $\,\,$ & $\,\,$ 1 $\,\,$ & $\,\,$\color{red} 8.3538\color{black} $\,\,$ & $\,\,$\color{red} 1.7886\color{black}   $\,\,$ \\
$\,\,$0.1175$\,\,$ & $\,\,$\color{red} 0.1197\color{black} $\,\,$ & $\,\,$ 1 $\,\,$ & $\,\,$0.2141 $\,\,$ \\
$\,\,$0.5487$\,\,$ & $\,\,$\color{red} 0.5591\color{black} $\,\,$ & $\,\,$4.6707$\,\,$ & $\,\,$ 1  $\,\,$ \\
\end{pmatrix},
\end{equation*}

\begin{equation*}
\mathbf{w}^{\prime} =
\begin{pmatrix}
0.375076\\
0.375076\\
0.044060\\
0.205789
\end{pmatrix} =
0.992989\cdot
\begin{pmatrix}
0.377724\\
\color{gr} 0.377724\color{black} \\
0.044371\\
0.207242
\end{pmatrix},
\end{equation*}
\begin{equation*}
\left[ \frac{{w}^{\prime}_i}{{w}^{\prime}_j} \right] =
\begin{pmatrix}
$\,\,$ 1 $\,\,$ & $\,\,$\color{gr} \color{blue} 1\color{black} $\,\,$ & $\,\,$8.5129$\,\,$ & $\,\,$1.8226$\,\,$ \\
$\,\,$\color{gr} \color{blue} 1\color{black} $\,\,$ & $\,\,$ 1 $\,\,$ & $\,\,$\color{gr} 8.5129\color{black} $\,\,$ & $\,\,$\color{gr} 1.8226\color{black}   $\,\,$ \\
$\,\,$0.1175$\,\,$ & $\,\,$\color{gr} 0.1175\color{black} $\,\,$ & $\,\,$ 1 $\,\,$ & $\,\,$0.2141 $\,\,$ \\
$\,\,$0.5487$\,\,$ & $\,\,$\color{gr} 0.5487\color{black} $\,\,$ & $\,\,$4.6707$\,\,$ & $\,\,$ 1  $\,\,$ \\
\end{pmatrix},
\end{equation*}
\end{example}
\newpage
\begin{example}
\begin{equation*}
\mathbf{A} =
\begin{pmatrix}
$\,\,$ 1 $\,\,$ & $\,\,$1$\,\,$ & $\,\,$5$\,\,$ & $\,\,$3 $\,\,$ \\
$\,\,$ 1 $\,\,$ & $\,\,$ 1 $\,\,$ & $\,\,$9$\,\,$ & $\,\,$2 $\,\,$ \\
$\,\,$ 1/5$\,\,$ & $\,\,$ 1/9$\,\,$ & $\,\,$ 1 $\,\,$ & $\,\,$ 1/9 $\,\,$ \\
$\,\,$ 1/3$\,\,$ & $\,\,$ 1/2$\,\,$ & $\,\,$9$\,\,$ & $\,\,$ 1  $\,\,$ \\
\end{pmatrix},
\qquad
\lambda_{\max} =
4.2507,
\qquad
CR = 0.0946
\end{equation*}

\begin{equation*}
\mathbf{w}^{EM} =
\begin{pmatrix}
0.376424\\
\color{red} 0.366538\color{black} \\
0.043003\\
0.214035
\end{pmatrix}\end{equation*}
\begin{equation*}
\left[ \frac{{w}^{EM}_i}{{w}^{EM}_j} \right] =
\begin{pmatrix}
$\,\,$ 1 $\,\,$ & $\,\,$\color{red} 1.0270\color{black} $\,\,$ & $\,\,$8.7534$\,\,$ & $\,\,$1.7587$\,\,$ \\
$\,\,$\color{red} 0.9737\color{black} $\,\,$ & $\,\,$ 1 $\,\,$ & $\,\,$\color{red} 8.5235\color{black} $\,\,$ & $\,\,$\color{red} 1.7125\color{black}   $\,\,$ \\
$\,\,$0.1142$\,\,$ & $\,\,$\color{red} 0.1173\color{black} $\,\,$ & $\,\,$ 1 $\,\,$ & $\,\,$0.2009 $\,\,$ \\
$\,\,$0.5686$\,\,$ & $\,\,$\color{red} 0.5839\color{black} $\,\,$ & $\,\,$4.9772$\,\,$ & $\,\,$ 1  $\,\,$ \\
\end{pmatrix},
\end{equation*}

\begin{equation*}
\mathbf{w}^{\prime} =
\begin{pmatrix}
0.372739\\
0.372739\\
0.042582\\
0.211940
\end{pmatrix} =
0.990211\cdot
\begin{pmatrix}
0.376424\\
\color{gr} 0.376424\color{black} \\
0.043003\\
0.214035
\end{pmatrix},
\end{equation*}
\begin{equation*}
\left[ \frac{{w}^{\prime}_i}{{w}^{\prime}_j} \right] =
\begin{pmatrix}
$\,\,$ 1 $\,\,$ & $\,\,$\color{gr} \color{blue} 1\color{black} $\,\,$ & $\,\,$8.7534$\,\,$ & $\,\,$1.7587$\,\,$ \\
$\,\,$\color{gr} \color{blue} 1\color{black} $\,\,$ & $\,\,$ 1 $\,\,$ & $\,\,$\color{gr} 8.7534\color{black} $\,\,$ & $\,\,$\color{gr} 1.7587\color{black}   $\,\,$ \\
$\,\,$0.1142$\,\,$ & $\,\,$\color{gr} 0.1142\color{black} $\,\,$ & $\,\,$ 1 $\,\,$ & $\,\,$0.2009 $\,\,$ \\
$\,\,$0.5686$\,\,$ & $\,\,$\color{gr} 0.5686\color{black} $\,\,$ & $\,\,$4.9772$\,\,$ & $\,\,$ 1  $\,\,$ \\
\end{pmatrix},
\end{equation*}
\end{example}
\newpage
\begin{example}
\begin{equation*}
\mathbf{A} =
\begin{pmatrix}
$\,\,$ 1 $\,\,$ & $\,\,$1$\,\,$ & $\,\,$5$\,\,$ & $\,\,$4 $\,\,$ \\
$\,\,$ 1 $\,\,$ & $\,\,$ 1 $\,\,$ & $\,\,$7$\,\,$ & $\,\,$2 $\,\,$ \\
$\,\,$ 1/5$\,\,$ & $\,\,$ 1/7$\,\,$ & $\,\,$ 1 $\,\,$ & $\,\,$ 1/6 $\,\,$ \\
$\,\,$ 1/4$\,\,$ & $\,\,$ 1/2$\,\,$ & $\,\,$6$\,\,$ & $\,\,$ 1  $\,\,$ \\
\end{pmatrix},
\qquad
\lambda_{\max} =
4.2174,
\qquad
CR = 0.0820
\end{equation*}

\begin{equation*}
\mathbf{w}^{EM} =
\begin{pmatrix}
0.414707\\
\color{red} 0.352584\color{black} \\
0.050855\\
0.181854
\end{pmatrix}\end{equation*}
\begin{equation*}
\left[ \frac{{w}^{EM}_i}{{w}^{EM}_j} \right] =
\begin{pmatrix}
$\,\,$ 1 $\,\,$ & $\,\,$\color{red} 1.1762\color{black} $\,\,$ & $\,\,$8.1547$\,\,$ & $\,\,$2.2804$\,\,$ \\
$\,\,$\color{red} 0.8502\color{black} $\,\,$ & $\,\,$ 1 $\,\,$ & $\,\,$\color{red} 6.9332\color{black} $\,\,$ & $\,\,$\color{red} 1.9388\color{black}   $\,\,$ \\
$\,\,$0.1226$\,\,$ & $\,\,$\color{red} 0.1442\color{black} $\,\,$ & $\,\,$ 1 $\,\,$ & $\,\,$0.2796 $\,\,$ \\
$\,\,$0.4385$\,\,$ & $\,\,$\color{red} 0.5158\color{black} $\,\,$ & $\,\,$3.5760$\,\,$ & $\,\,$ 1  $\,\,$ \\
\end{pmatrix},
\end{equation*}

\begin{equation*}
\mathbf{w}^{\prime} =
\begin{pmatrix}
0.413302\\
0.354777\\
0.050682\\
0.181238
\end{pmatrix} =
0.996612\cdot
\begin{pmatrix}
0.414707\\
\color{gr} 0.355983\color{black} \\
0.050855\\
0.181854
\end{pmatrix},
\end{equation*}
\begin{equation*}
\left[ \frac{{w}^{\prime}_i}{{w}^{\prime}_j} \right] =
\begin{pmatrix}
$\,\,$ 1 $\,\,$ & $\,\,$\color{gr} 1.1650\color{black} $\,\,$ & $\,\,$8.1547$\,\,$ & $\,\,$2.2804$\,\,$ \\
$\,\,$\color{gr} 0.8584\color{black} $\,\,$ & $\,\,$ 1 $\,\,$ & $\,\,$\color{gr} \color{blue} 7\color{black} $\,\,$ & $\,\,$\color{gr} 1.9575\color{black}   $\,\,$ \\
$\,\,$0.1226$\,\,$ & $\,\,$\color{gr} \color{blue}  1/7\color{black} $\,\,$ & $\,\,$ 1 $\,\,$ & $\,\,$0.2796 $\,\,$ \\
$\,\,$0.4385$\,\,$ & $\,\,$\color{gr} 0.5109\color{black} $\,\,$ & $\,\,$3.5760$\,\,$ & $\,\,$ 1  $\,\,$ \\
\end{pmatrix},
\end{equation*}
\end{example}
\newpage
\begin{example}
\begin{equation*}
\mathbf{A} =
\begin{pmatrix}
$\,\,$ 1 $\,\,$ & $\,\,$1$\,\,$ & $\,\,$5$\,\,$ & $\,\,$4 $\,\,$ \\
$\,\,$ 1 $\,\,$ & $\,\,$ 1 $\,\,$ & $\,\,$7$\,\,$ & $\,\,$3 $\,\,$ \\
$\,\,$ 1/5$\,\,$ & $\,\,$ 1/7$\,\,$ & $\,\,$ 1 $\,\,$ & $\,\,$ 1/3 $\,\,$ \\
$\,\,$ 1/4$\,\,$ & $\,\,$ 1/3$\,\,$ & $\,\,$3$\,\,$ & $\,\,$ 1  $\,\,$ \\
\end{pmatrix},
\qquad
\lambda_{\max} =
4.0649,
\qquad
CR = 0.0245
\end{equation*}

\begin{equation*}
\mathbf{w}^{EM} =
\begin{pmatrix}
0.404498\\
\color{red} 0.400674\color{black} \\
0.059761\\
0.135067
\end{pmatrix}\end{equation*}
\begin{equation*}
\left[ \frac{{w}^{EM}_i}{{w}^{EM}_j} \right] =
\begin{pmatrix}
$\,\,$ 1 $\,\,$ & $\,\,$\color{red} 1.0095\color{black} $\,\,$ & $\,\,$6.7686$\,\,$ & $\,\,$2.9948$\,\,$ \\
$\,\,$\color{red} 0.9905\color{black} $\,\,$ & $\,\,$ 1 $\,\,$ & $\,\,$\color{red} 6.7046\color{black} $\,\,$ & $\,\,$\color{red} 2.9665\color{black}   $\,\,$ \\
$\,\,$0.1477$\,\,$ & $\,\,$\color{red} 0.1492\color{black} $\,\,$ & $\,\,$ 1 $\,\,$ & $\,\,$0.4425 $\,\,$ \\
$\,\,$0.3339$\,\,$ & $\,\,$\color{red} 0.3371\color{black} $\,\,$ & $\,\,$2.2601$\,\,$ & $\,\,$ 1  $\,\,$ \\
\end{pmatrix},
\end{equation*}

\begin{equation*}
\mathbf{w}^{\prime} =
\begin{pmatrix}
0.402957\\
0.402957\\
0.059533\\
0.134552
\end{pmatrix} =
0.996190\cdot
\begin{pmatrix}
0.404498\\
\color{gr} 0.404498\color{black} \\
0.059761\\
0.135067
\end{pmatrix},
\end{equation*}
\begin{equation*}
\left[ \frac{{w}^{\prime}_i}{{w}^{\prime}_j} \right] =
\begin{pmatrix}
$\,\,$ 1 $\,\,$ & $\,\,$\color{gr} \color{blue} 1\color{black} $\,\,$ & $\,\,$6.7686$\,\,$ & $\,\,$2.9948$\,\,$ \\
$\,\,$\color{gr} \color{blue} 1\color{black} $\,\,$ & $\,\,$ 1 $\,\,$ & $\,\,$\color{gr} 6.7686\color{black} $\,\,$ & $\,\,$\color{gr} 2.9948\color{black}   $\,\,$ \\
$\,\,$0.1477$\,\,$ & $\,\,$\color{gr} 0.1477\color{black} $\,\,$ & $\,\,$ 1 $\,\,$ & $\,\,$0.4425 $\,\,$ \\
$\,\,$0.3339$\,\,$ & $\,\,$\color{gr} 0.3339\color{black} $\,\,$ & $\,\,$2.2601$\,\,$ & $\,\,$ 1  $\,\,$ \\
\end{pmatrix},
\end{equation*}
\end{example}
\newpage
\begin{example}
\begin{equation*}
\mathbf{A} =
\begin{pmatrix}
$\,\,$ 1 $\,\,$ & $\,\,$1$\,\,$ & $\,\,$5$\,\,$ & $\,\,$4 $\,\,$ \\
$\,\,$ 1 $\,\,$ & $\,\,$ 1 $\,\,$ & $\,\,$8$\,\,$ & $\,\,$2 $\,\,$ \\
$\,\,$ 1/5$\,\,$ & $\,\,$ 1/8$\,\,$ & $\,\,$ 1 $\,\,$ & $\,\,$ 1/7 $\,\,$ \\
$\,\,$ 1/4$\,\,$ & $\,\,$ 1/2$\,\,$ & $\,\,$7$\,\,$ & $\,\,$ 1  $\,\,$ \\
\end{pmatrix},
\qquad
\lambda_{\max} =
4.2610,
\qquad
CR = 0.0984
\end{equation*}

\begin{equation*}
\mathbf{w}^{EM} =
\begin{pmatrix}
0.410379\\
\color{red} 0.355747\color{black} \\
0.046993\\
0.186882
\end{pmatrix}\end{equation*}
\begin{equation*}
\left[ \frac{{w}^{EM}_i}{{w}^{EM}_j} \right] =
\begin{pmatrix}
$\,\,$ 1 $\,\,$ & $\,\,$\color{red} 1.1536\color{black} $\,\,$ & $\,\,$8.7329$\,\,$ & $\,\,$2.1959$\,\,$ \\
$\,\,$\color{red} 0.8669\color{black} $\,\,$ & $\,\,$ 1 $\,\,$ & $\,\,$\color{red} 7.5703\color{black} $\,\,$ & $\,\,$\color{red} 1.9036\color{black}   $\,\,$ \\
$\,\,$0.1145$\,\,$ & $\,\,$\color{red} 0.1321\color{black} $\,\,$ & $\,\,$ 1 $\,\,$ & $\,\,$0.2515 $\,\,$ \\
$\,\,$0.4554$\,\,$ & $\,\,$\color{red} 0.5253\color{black} $\,\,$ & $\,\,$3.9768$\,\,$ & $\,\,$ 1  $\,\,$ \\
\end{pmatrix},
\end{equation*}

\begin{equation*}
\mathbf{w}^{\prime} =
\begin{pmatrix}
0.403116\\
0.367148\\
0.046161\\
0.183574
\end{pmatrix} =
0.982303\cdot
\begin{pmatrix}
0.410379\\
\color{gr} 0.373763\color{black} \\
0.046993\\
0.186882
\end{pmatrix},
\end{equation*}
\begin{equation*}
\left[ \frac{{w}^{\prime}_i}{{w}^{\prime}_j} \right] =
\begin{pmatrix}
$\,\,$ 1 $\,\,$ & $\,\,$\color{gr} 1.0980\color{black} $\,\,$ & $\,\,$8.7329$\,\,$ & $\,\,$2.1959$\,\,$ \\
$\,\,$\color{gr} 0.9108\color{black} $\,\,$ & $\,\,$ 1 $\,\,$ & $\,\,$\color{gr} 7.9537\color{black} $\,\,$ & $\,\,$\color{gr} \color{blue} 2\color{black}   $\,\,$ \\
$\,\,$0.1145$\,\,$ & $\,\,$\color{gr} 0.1257\color{black} $\,\,$ & $\,\,$ 1 $\,\,$ & $\,\,$0.2515 $\,\,$ \\
$\,\,$0.4554$\,\,$ & $\,\,$\color{gr} \color{blue}  1/2\color{black} $\,\,$ & $\,\,$3.9768$\,\,$ & $\,\,$ 1  $\,\,$ \\
\end{pmatrix},
\end{equation*}
\end{example}
\newpage
\begin{example}
\begin{equation*}
\mathbf{A} =
\begin{pmatrix}
$\,\,$ 1 $\,\,$ & $\,\,$1$\,\,$ & $\,\,$5$\,\,$ & $\,\,$4 $\,\,$ \\
$\,\,$ 1 $\,\,$ & $\,\,$ 1 $\,\,$ & $\,\,$8$\,\,$ & $\,\,$3 $\,\,$ \\
$\,\,$ 1/5$\,\,$ & $\,\,$ 1/8$\,\,$ & $\,\,$ 1 $\,\,$ & $\,\,$ 1/5 $\,\,$ \\
$\,\,$ 1/4$\,\,$ & $\,\,$ 1/3$\,\,$ & $\,\,$5$\,\,$ & $\,\,$ 1  $\,\,$ \\
\end{pmatrix},
\qquad
\lambda_{\max} =
4.1689,
\qquad
CR = 0.0637
\end{equation*}

\begin{equation*}
\mathbf{w}^{EM} =
\begin{pmatrix}
0.398428\\
\color{red} 0.398073\color{black} \\
0.050505\\
0.152994
\end{pmatrix}\end{equation*}
\begin{equation*}
\left[ \frac{{w}^{EM}_i}{{w}^{EM}_j} \right] =
\begin{pmatrix}
$\,\,$ 1 $\,\,$ & $\,\,$\color{red} 1.0009\color{black} $\,\,$ & $\,\,$7.8889$\,\,$ & $\,\,$2.6042$\,\,$ \\
$\,\,$\color{red} 0.9991\color{black} $\,\,$ & $\,\,$ 1 $\,\,$ & $\,\,$\color{red} 7.8819\color{black} $\,\,$ & $\,\,$\color{red} 2.6019\color{black}   $\,\,$ \\
$\,\,$0.1268$\,\,$ & $\,\,$\color{red} 0.1269\color{black} $\,\,$ & $\,\,$ 1 $\,\,$ & $\,\,$0.3301 $\,\,$ \\
$\,\,$0.3840$\,\,$ & $\,\,$\color{red} 0.3843\color{black} $\,\,$ & $\,\,$3.0293$\,\,$ & $\,\,$ 1  $\,\,$ \\
\end{pmatrix},
\end{equation*}

\begin{equation*}
\mathbf{w}^{\prime} =
\begin{pmatrix}
0.398287\\
0.398287\\
0.050487\\
0.152940
\end{pmatrix} =
0.999645\cdot
\begin{pmatrix}
0.398428\\
\color{gr} 0.398428\color{black} \\
0.050505\\
0.152994
\end{pmatrix},
\end{equation*}
\begin{equation*}
\left[ \frac{{w}^{\prime}_i}{{w}^{\prime}_j} \right] =
\begin{pmatrix}
$\,\,$ 1 $\,\,$ & $\,\,$\color{gr} \color{blue} 1\color{black} $\,\,$ & $\,\,$7.8889$\,\,$ & $\,\,$2.6042$\,\,$ \\
$\,\,$\color{gr} \color{blue} 1\color{black} $\,\,$ & $\,\,$ 1 $\,\,$ & $\,\,$\color{gr} 7.8889\color{black} $\,\,$ & $\,\,$\color{gr} 2.6042\color{black}   $\,\,$ \\
$\,\,$0.1268$\,\,$ & $\,\,$\color{gr} 0.1268\color{black} $\,\,$ & $\,\,$ 1 $\,\,$ & $\,\,$0.3301 $\,\,$ \\
$\,\,$0.3840$\,\,$ & $\,\,$\color{gr} 0.3840\color{black} $\,\,$ & $\,\,$3.0293$\,\,$ & $\,\,$ 1  $\,\,$ \\
\end{pmatrix},
\end{equation*}
\end{example}
\newpage
\begin{example}
\begin{equation*}
\mathbf{A} =
\begin{pmatrix}
$\,\,$ 1 $\,\,$ & $\,\,$1$\,\,$ & $\,\,$5$\,\,$ & $\,\,$4 $\,\,$ \\
$\,\,$ 1 $\,\,$ & $\,\,$ 1 $\,\,$ & $\,\,$9$\,\,$ & $\,\,$2 $\,\,$ \\
$\,\,$ 1/5$\,\,$ & $\,\,$ 1/9$\,\,$ & $\,\,$ 1 $\,\,$ & $\,\,$ 1/7 $\,\,$ \\
$\,\,$ 1/4$\,\,$ & $\,\,$ 1/2$\,\,$ & $\,\,$7$\,\,$ & $\,\,$ 1  $\,\,$ \\
\end{pmatrix},
\qquad
\lambda_{\max} =
4.2614,
\qquad
CR = 0.0986
\end{equation*}

\begin{equation*}
\mathbf{w}^{EM} =
\begin{pmatrix}
0.407069\\
\color{red} 0.363174\color{black} \\
0.045410\\
0.184347
\end{pmatrix}\end{equation*}
\begin{equation*}
\left[ \frac{{w}^{EM}_i}{{w}^{EM}_j} \right] =
\begin{pmatrix}
$\,\,$ 1 $\,\,$ & $\,\,$\color{red} 1.1209\color{black} $\,\,$ & $\,\,$8.9642$\,\,$ & $\,\,$2.2082$\,\,$ \\
$\,\,$\color{red} 0.8922\color{black} $\,\,$ & $\,\,$ 1 $\,\,$ & $\,\,$\color{red} 7.9976\color{black} $\,\,$ & $\,\,$\color{red} 1.9701\color{black}   $\,\,$ \\
$\,\,$0.1116$\,\,$ & $\,\,$\color{red} 0.1250\color{black} $\,\,$ & $\,\,$ 1 $\,\,$ & $\,\,$0.2463 $\,\,$ \\
$\,\,$0.4529$\,\,$ & $\,\,$\color{red} 0.5076\color{black} $\,\,$ & $\,\,$4.0596$\,\,$ & $\,\,$ 1  $\,\,$ \\
\end{pmatrix},
\end{equation*}

\begin{equation*}
\mathbf{w}^{\prime} =
\begin{pmatrix}
0.404834\\
0.366670\\
0.045161\\
0.183335
\end{pmatrix} =
0.994511\cdot
\begin{pmatrix}
0.407069\\
\color{gr} 0.368693\color{black} \\
0.045410\\
0.184347
\end{pmatrix},
\end{equation*}
\begin{equation*}
\left[ \frac{{w}^{\prime}_i}{{w}^{\prime}_j} \right] =
\begin{pmatrix}
$\,\,$ 1 $\,\,$ & $\,\,$\color{gr} 1.1041\color{black} $\,\,$ & $\,\,$8.9642$\,\,$ & $\,\,$2.2082$\,\,$ \\
$\,\,$\color{gr} 0.9057\color{black} $\,\,$ & $\,\,$ 1 $\,\,$ & $\,\,$\color{gr} 8.1191\color{black} $\,\,$ & $\,\,$\color{gr} \color{blue} 2\color{black}   $\,\,$ \\
$\,\,$0.1116$\,\,$ & $\,\,$\color{gr} 0.1232\color{black} $\,\,$ & $\,\,$ 1 $\,\,$ & $\,\,$0.2463 $\,\,$ \\
$\,\,$0.4529$\,\,$ & $\,\,$\color{gr} \color{blue}  1/2\color{black} $\,\,$ & $\,\,$4.0596$\,\,$ & $\,\,$ 1  $\,\,$ \\
\end{pmatrix},
\end{equation*}
\end{example}
\newpage
\begin{example}
\begin{equation*}
\mathbf{A} =
\begin{pmatrix}
$\,\,$ 1 $\,\,$ & $\,\,$1$\,\,$ & $\,\,$5$\,\,$ & $\,\,$5 $\,\,$ \\
$\,\,$ 1 $\,\,$ & $\,\,$ 1 $\,\,$ & $\,\,$8$\,\,$ & $\,\,$3 $\,\,$ \\
$\,\,$ 1/5$\,\,$ & $\,\,$ 1/8$\,\,$ & $\,\,$ 1 $\,\,$ & $\,\,$ 1/4 $\,\,$ \\
$\,\,$ 1/5$\,\,$ & $\,\,$ 1/3$\,\,$ & $\,\,$4$\,\,$ & $\,\,$ 1  $\,\,$ \\
\end{pmatrix},
\qquad
\lambda_{\max} =
4.1655,
\qquad
CR = 0.0624
\end{equation*}

\begin{equation*}
\mathbf{w}^{EM} =
\begin{pmatrix}
0.419753\\
\color{red} 0.393122\color{black} \\
0.052663\\
0.134462
\end{pmatrix}\end{equation*}
\begin{equation*}
\left[ \frac{{w}^{EM}_i}{{w}^{EM}_j} \right] =
\begin{pmatrix}
$\,\,$ 1 $\,\,$ & $\,\,$\color{red} 1.0677\color{black} $\,\,$ & $\,\,$7.9706$\,\,$ & $\,\,$3.1217$\,\,$ \\
$\,\,$\color{red} 0.9366\color{black} $\,\,$ & $\,\,$ 1 $\,\,$ & $\,\,$\color{red} 7.4649\color{black} $\,\,$ & $\,\,$\color{red} 2.9237\color{black}   $\,\,$ \\
$\,\,$0.1255$\,\,$ & $\,\,$\color{red} 0.1340\color{black} $\,\,$ & $\,\,$ 1 $\,\,$ & $\,\,$0.3917 $\,\,$ \\
$\,\,$0.3203$\,\,$ & $\,\,$\color{red} 0.3420\color{black} $\,\,$ & $\,\,$2.5533$\,\,$ & $\,\,$ 1  $\,\,$ \\
\end{pmatrix},
\end{equation*}

\begin{equation*}
\mathbf{w}^{\prime} =
\begin{pmatrix}
0.415489\\
0.399287\\
0.052128\\
0.133096
\end{pmatrix} =
0.989841\cdot
\begin{pmatrix}
0.419753\\
\color{gr} 0.403385\color{black} \\
0.052663\\
0.134462
\end{pmatrix},
\end{equation*}
\begin{equation*}
\left[ \frac{{w}^{\prime}_i}{{w}^{\prime}_j} \right] =
\begin{pmatrix}
$\,\,$ 1 $\,\,$ & $\,\,$\color{gr} 1.0406\color{black} $\,\,$ & $\,\,$7.9706$\,\,$ & $\,\,$3.1217$\,\,$ \\
$\,\,$\color{gr} 0.9610\color{black} $\,\,$ & $\,\,$ 1 $\,\,$ & $\,\,$\color{gr} 7.6598\color{black} $\,\,$ & $\,\,$\color{gr} \color{blue} 3\color{black}   $\,\,$ \\
$\,\,$0.1255$\,\,$ & $\,\,$\color{gr} 0.1306\color{black} $\,\,$ & $\,\,$ 1 $\,\,$ & $\,\,$0.3917 $\,\,$ \\
$\,\,$0.3203$\,\,$ & $\,\,$\color{gr} \color{blue}  1/3\color{black} $\,\,$ & $\,\,$2.5533$\,\,$ & $\,\,$ 1  $\,\,$ \\
\end{pmatrix},
\end{equation*}
\end{example}
\newpage
\begin{example}
\begin{equation*}
\mathbf{A} =
\begin{pmatrix}
$\,\,$ 1 $\,\,$ & $\,\,$1$\,\,$ & $\,\,$5$\,\,$ & $\,\,$5 $\,\,$ \\
$\,\,$ 1 $\,\,$ & $\,\,$ 1 $\,\,$ & $\,\,$8$\,\,$ & $\,\,$3 $\,\,$ \\
$\,\,$ 1/5$\,\,$ & $\,\,$ 1/8$\,\,$ & $\,\,$ 1 $\,\,$ & $\,\,$ 1/5 $\,\,$ \\
$\,\,$ 1/5$\,\,$ & $\,\,$ 1/3$\,\,$ & $\,\,$5$\,\,$ & $\,\,$ 1  $\,\,$ \\
\end{pmatrix},
\qquad
\lambda_{\max} =
4.2259,
\qquad
CR = 0.0852
\end{equation*}

\begin{equation*}
\mathbf{w}^{EM} =
\begin{pmatrix}
0.419577\\
\color{red} 0.387152\color{black} \\
0.049904\\
0.143367
\end{pmatrix}\end{equation*}
\begin{equation*}
\left[ \frac{{w}^{EM}_i}{{w}^{EM}_j} \right] =
\begin{pmatrix}
$\,\,$ 1 $\,\,$ & $\,\,$\color{red} 1.0838\color{black} $\,\,$ & $\,\,$8.4078$\,\,$ & $\,\,$2.9266$\,\,$ \\
$\,\,$\color{red} 0.9227\color{black} $\,\,$ & $\,\,$ 1 $\,\,$ & $\,\,$\color{red} 7.7580\color{black} $\,\,$ & $\,\,$\color{red} 2.7004\color{black}   $\,\,$ \\
$\,\,$0.1189$\,\,$ & $\,\,$\color{red} 0.1289\color{black} $\,\,$ & $\,\,$ 1 $\,\,$ & $\,\,$0.3481 $\,\,$ \\
$\,\,$0.3417$\,\,$ & $\,\,$\color{red} 0.3703\color{black} $\,\,$ & $\,\,$2.8729$\,\,$ & $\,\,$ 1  $\,\,$ \\
\end{pmatrix},
\end{equation*}

\begin{equation*}
\mathbf{w}^{\prime} =
\begin{pmatrix}
0.414571\\
0.394465\\
0.049308\\
0.141656
\end{pmatrix} =
0.988068\cdot
\begin{pmatrix}
0.419577\\
\color{gr} 0.399229\color{black} \\
0.049904\\
0.143367
\end{pmatrix},
\end{equation*}
\begin{equation*}
\left[ \frac{{w}^{\prime}_i}{{w}^{\prime}_j} \right] =
\begin{pmatrix}
$\,\,$ 1 $\,\,$ & $\,\,$\color{gr} 1.0510\color{black} $\,\,$ & $\,\,$8.4078$\,\,$ & $\,\,$2.9266$\,\,$ \\
$\,\,$\color{gr} 0.9515\color{black} $\,\,$ & $\,\,$ 1 $\,\,$ & $\,\,$\color{gr} \color{blue} 8\color{black} $\,\,$ & $\,\,$\color{gr} 2.7847\color{black}   $\,\,$ \\
$\,\,$0.1189$\,\,$ & $\,\,$\color{gr} \color{blue}  1/8\color{black} $\,\,$ & $\,\,$ 1 $\,\,$ & $\,\,$0.3481 $\,\,$ \\
$\,\,$0.3417$\,\,$ & $\,\,$\color{gr} 0.3591\color{black} $\,\,$ & $\,\,$2.8729$\,\,$ & $\,\,$ 1  $\,\,$ \\
\end{pmatrix},
\end{equation*}
\end{example}
\newpage
\begin{example}
\begin{equation*}
\mathbf{A} =
\begin{pmatrix}
$\,\,$ 1 $\,\,$ & $\,\,$1$\,\,$ & $\,\,$5$\,\,$ & $\,\,$5 $\,\,$ \\
$\,\,$ 1 $\,\,$ & $\,\,$ 1 $\,\,$ & $\,\,$9$\,\,$ & $\,\,$3 $\,\,$ \\
$\,\,$ 1/5$\,\,$ & $\,\,$ 1/9$\,\,$ & $\,\,$ 1 $\,\,$ & $\,\,$ 1/5 $\,\,$ \\
$\,\,$ 1/5$\,\,$ & $\,\,$ 1/3$\,\,$ & $\,\,$5$\,\,$ & $\,\,$ 1  $\,\,$ \\
\end{pmatrix},
\qquad
\lambda_{\max} =
4.2253,
\qquad
CR = 0.0849
\end{equation*}

\begin{equation*}
\mathbf{w}^{EM} =
\begin{pmatrix}
0.415948\\
\color{red} 0.394680\color{black} \\
0.048147\\
0.141225
\end{pmatrix}\end{equation*}
\begin{equation*}
\left[ \frac{{w}^{EM}_i}{{w}^{EM}_j} \right] =
\begin{pmatrix}
$\,\,$ 1 $\,\,$ & $\,\,$\color{red} 1.0539\color{black} $\,\,$ & $\,\,$8.6391$\,\,$ & $\,\,$2.9453$\,\,$ \\
$\,\,$\color{red} 0.9489\color{black} $\,\,$ & $\,\,$ 1 $\,\,$ & $\,\,$\color{red} 8.1973\color{black} $\,\,$ & $\,\,$\color{red} 2.7947\color{black}   $\,\,$ \\
$\,\,$0.1158$\,\,$ & $\,\,$\color{red} 0.1220\color{black} $\,\,$ & $\,\,$ 1 $\,\,$ & $\,\,$0.3409 $\,\,$ \\
$\,\,$0.3395$\,\,$ & $\,\,$\color{red} 0.3578\color{black} $\,\,$ & $\,\,$2.9332$\,\,$ & $\,\,$ 1  $\,\,$ \\
\end{pmatrix},
\end{equation*}

\begin{equation*}
\mathbf{w}^{\prime} =
\begin{pmatrix}
0.407286\\
0.407286\\
0.047145\\
0.138284
\end{pmatrix} =
0.979176\cdot
\begin{pmatrix}
0.415948\\
\color{gr} 0.415948\color{black} \\
0.048147\\
0.141225
\end{pmatrix},
\end{equation*}
\begin{equation*}
\left[ \frac{{w}^{\prime}_i}{{w}^{\prime}_j} \right] =
\begin{pmatrix}
$\,\,$ 1 $\,\,$ & $\,\,$\color{gr} \color{blue} 1\color{black} $\,\,$ & $\,\,$8.6391$\,\,$ & $\,\,$2.9453$\,\,$ \\
$\,\,$\color{gr} \color{blue} 1\color{black} $\,\,$ & $\,\,$ 1 $\,\,$ & $\,\,$\color{gr} 8.6391\color{black} $\,\,$ & $\,\,$\color{gr} 2.9453\color{black}   $\,\,$ \\
$\,\,$0.1158$\,\,$ & $\,\,$\color{gr} 0.1158\color{black} $\,\,$ & $\,\,$ 1 $\,\,$ & $\,\,$0.3409 $\,\,$ \\
$\,\,$0.3395$\,\,$ & $\,\,$\color{gr} 0.3395\color{black} $\,\,$ & $\,\,$2.9332$\,\,$ & $\,\,$ 1  $\,\,$ \\
\end{pmatrix},
\end{equation*}
\end{example}
\newpage
\begin{example}
\begin{equation*}
\mathbf{A} =
\begin{pmatrix}
$\,\,$ 1 $\,\,$ & $\,\,$1$\,\,$ & $\,\,$5$\,\,$ & $\,\,$6 $\,\,$ \\
$\,\,$ 1 $\,\,$ & $\,\,$ 1 $\,\,$ & $\,\,$7$\,\,$ & $\,\,$3 $\,\,$ \\
$\,\,$ 1/5$\,\,$ & $\,\,$ 1/7$\,\,$ & $\,\,$ 1 $\,\,$ & $\,\,$ 1/4 $\,\,$ \\
$\,\,$ 1/6$\,\,$ & $\,\,$ 1/3$\,\,$ & $\,\,$4$\,\,$ & $\,\,$ 1  $\,\,$ \\
\end{pmatrix},
\qquad
\lambda_{\max} =
4.2174,
\qquad
CR = 0.0820
\end{equation*}

\begin{equation*}
\mathbf{w}^{EM} =
\begin{pmatrix}
0.441468\\
\color{red} 0.375336\color{black} \\
0.054136\\
0.129060
\end{pmatrix}\end{equation*}
\begin{equation*}
\left[ \frac{{w}^{EM}_i}{{w}^{EM}_j} \right] =
\begin{pmatrix}
$\,\,$ 1 $\,\,$ & $\,\,$\color{red} 1.1762\color{black} $\,\,$ & $\,\,$8.1547$\,\,$ & $\,\,$3.4207$\,\,$ \\
$\,\,$\color{red} 0.8502\color{black} $\,\,$ & $\,\,$ 1 $\,\,$ & $\,\,$\color{red} 6.9332\color{black} $\,\,$ & $\,\,$\color{red} 2.9082\color{black}   $\,\,$ \\
$\,\,$0.1226$\,\,$ & $\,\,$\color{red} 0.1442\color{black} $\,\,$ & $\,\,$ 1 $\,\,$ & $\,\,$0.4195 $\,\,$ \\
$\,\,$0.2923$\,\,$ & $\,\,$\color{red} 0.3439\color{black} $\,\,$ & $\,\,$2.3840$\,\,$ & $\,\,$ 1  $\,\,$ \\
\end{pmatrix},
\end{equation*}

\begin{equation*}
\mathbf{w}^{\prime} =
\begin{pmatrix}
0.439876\\
0.377588\\
0.053941\\
0.128594
\end{pmatrix} =
0.996394\cdot
\begin{pmatrix}
0.441468\\
\color{gr} 0.378955\color{black} \\
0.054136\\
0.129060
\end{pmatrix},
\end{equation*}
\begin{equation*}
\left[ \frac{{w}^{\prime}_i}{{w}^{\prime}_j} \right] =
\begin{pmatrix}
$\,\,$ 1 $\,\,$ & $\,\,$\color{gr} 1.1650\color{black} $\,\,$ & $\,\,$8.1547$\,\,$ & $\,\,$3.4207$\,\,$ \\
$\,\,$\color{gr} 0.8584\color{black} $\,\,$ & $\,\,$ 1 $\,\,$ & $\,\,$\color{gr} \color{blue} 7\color{black} $\,\,$ & $\,\,$\color{gr} 2.9363\color{black}   $\,\,$ \\
$\,\,$0.1226$\,\,$ & $\,\,$\color{gr} \color{blue}  1/7\color{black} $\,\,$ & $\,\,$ 1 $\,\,$ & $\,\,$0.4195 $\,\,$ \\
$\,\,$0.2923$\,\,$ & $\,\,$\color{gr} 0.3406\color{black} $\,\,$ & $\,\,$2.3840$\,\,$ & $\,\,$ 1  $\,\,$ \\
\end{pmatrix},
\end{equation*}
\end{example}
\newpage
\begin{example}
\begin{equation*}
\mathbf{A} =
\begin{pmatrix}
$\,\,$ 1 $\,\,$ & $\,\,$1$\,\,$ & $\,\,$5$\,\,$ & $\,\,$6 $\,\,$ \\
$\,\,$ 1 $\,\,$ & $\,\,$ 1 $\,\,$ & $\,\,$8$\,\,$ & $\,\,$4 $\,\,$ \\
$\,\,$ 1/5$\,\,$ & $\,\,$ 1/8$\,\,$ & $\,\,$ 1 $\,\,$ & $\,\,$ 1/3 $\,\,$ \\
$\,\,$ 1/6$\,\,$ & $\,\,$ 1/4$\,\,$ & $\,\,$3$\,\,$ & $\,\,$ 1  $\,\,$ \\
\end{pmatrix},
\qquad
\lambda_{\max} =
4.1406,
\qquad
CR = 0.0530
\end{equation*}

\begin{equation*}
\mathbf{w}^{EM} =
\begin{pmatrix}
0.424777\\
\color{red} 0.412478\color{black} \\
0.054913\\
0.107833
\end{pmatrix}\end{equation*}
\begin{equation*}
\left[ \frac{{w}^{EM}_i}{{w}^{EM}_j} \right] =
\begin{pmatrix}
$\,\,$ 1 $\,\,$ & $\,\,$\color{red} 1.0298\color{black} $\,\,$ & $\,\,$7.7354$\,\,$ & $\,\,$3.9392$\,\,$ \\
$\,\,$\color{red} 0.9710\color{black} $\,\,$ & $\,\,$ 1 $\,\,$ & $\,\,$\color{red} 7.5114\color{black} $\,\,$ & $\,\,$\color{red} 3.8252\color{black}   $\,\,$ \\
$\,\,$0.1293$\,\,$ & $\,\,$\color{red} 0.1331\color{black} $\,\,$ & $\,\,$ 1 $\,\,$ & $\,\,$0.5092 $\,\,$ \\
$\,\,$0.2539$\,\,$ & $\,\,$\color{red} 0.2614\color{black} $\,\,$ & $\,\,$1.9637$\,\,$ & $\,\,$ 1  $\,\,$ \\
\end{pmatrix},
\end{equation*}

\begin{equation*}
\mathbf{w}^{\prime} =
\begin{pmatrix}
0.419616\\
0.419616\\
0.054246\\
0.106522
\end{pmatrix} =
0.987850\cdot
\begin{pmatrix}
0.424777\\
\color{gr} 0.424777\color{black} \\
0.054913\\
0.107833
\end{pmatrix},
\end{equation*}
\begin{equation*}
\left[ \frac{{w}^{\prime}_i}{{w}^{\prime}_j} \right] =
\begin{pmatrix}
$\,\,$ 1 $\,\,$ & $\,\,$\color{gr} \color{blue} 1\color{black} $\,\,$ & $\,\,$7.7354$\,\,$ & $\,\,$3.9392$\,\,$ \\
$\,\,$\color{gr} \color{blue} 1\color{black} $\,\,$ & $\,\,$ 1 $\,\,$ & $\,\,$\color{gr} 7.7354\color{black} $\,\,$ & $\,\,$\color{gr} 3.9392\color{black}   $\,\,$ \\
$\,\,$0.1293$\,\,$ & $\,\,$\color{gr} 0.1293\color{black} $\,\,$ & $\,\,$ 1 $\,\,$ & $\,\,$0.5092 $\,\,$ \\
$\,\,$0.2539$\,\,$ & $\,\,$\color{gr} 0.2539\color{black} $\,\,$ & $\,\,$1.9637$\,\,$ & $\,\,$ 1  $\,\,$ \\
\end{pmatrix},
\end{equation*}
\end{example}
\newpage
\begin{example}
\begin{equation*}
\mathbf{A} =
\begin{pmatrix}
$\,\,$ 1 $\,\,$ & $\,\,$1$\,\,$ & $\,\,$5$\,\,$ & $\,\,$6 $\,\,$ \\
$\,\,$ 1 $\,\,$ & $\,\,$ 1 $\,\,$ & $\,\,$8$\,\,$ & $\,\,$4 $\,\,$ \\
$\,\,$ 1/5$\,\,$ & $\,\,$ 1/8$\,\,$ & $\,\,$ 1 $\,\,$ & $\,\,$ 1/4 $\,\,$ \\
$\,\,$ 1/6$\,\,$ & $\,\,$ 1/4$\,\,$ & $\,\,$4$\,\,$ & $\,\,$ 1  $\,\,$ \\
\end{pmatrix},
\qquad
\lambda_{\max} =
4.2162,
\qquad
CR = 0.0815
\end{equation*}

\begin{equation*}
\mathbf{w}^{EM} =
\begin{pmatrix}
0.425187\\
\color{red} 0.406007\color{black} \\
0.051350\\
0.117456
\end{pmatrix}\end{equation*}
\begin{equation*}
\left[ \frac{{w}^{EM}_i}{{w}^{EM}_j} \right] =
\begin{pmatrix}
$\,\,$ 1 $\,\,$ & $\,\,$\color{red} 1.0472\color{black} $\,\,$ & $\,\,$8.2802$\,\,$ & $\,\,$3.6200$\,\,$ \\
$\,\,$\color{red} 0.9549\color{black} $\,\,$ & $\,\,$ 1 $\,\,$ & $\,\,$\color{red} 7.9067\color{black} $\,\,$ & $\,\,$\color{red} 3.4567\color{black}   $\,\,$ \\
$\,\,$0.1208$\,\,$ & $\,\,$\color{red} 0.1265\color{black} $\,\,$ & $\,\,$ 1 $\,\,$ & $\,\,$0.4372 $\,\,$ \\
$\,\,$0.2762$\,\,$ & $\,\,$\color{red} 0.2893\color{black} $\,\,$ & $\,\,$2.2874$\,\,$ & $\,\,$ 1  $\,\,$ \\
\end{pmatrix},
\end{equation*}

\begin{equation*}
\mathbf{w}^{\prime} =
\begin{pmatrix}
0.423159\\
0.408840\\
0.051105\\
0.116896
\end{pmatrix} =
0.995232\cdot
\begin{pmatrix}
0.425187\\
\color{gr} 0.410798\color{black} \\
0.051350\\
0.117456
\end{pmatrix},
\end{equation*}
\begin{equation*}
\left[ \frac{{w}^{\prime}_i}{{w}^{\prime}_j} \right] =
\begin{pmatrix}
$\,\,$ 1 $\,\,$ & $\,\,$\color{gr} 1.0350\color{black} $\,\,$ & $\,\,$8.2802$\,\,$ & $\,\,$3.6200$\,\,$ \\
$\,\,$\color{gr} 0.9662\color{black} $\,\,$ & $\,\,$ 1 $\,\,$ & $\,\,$\color{gr} \color{blue} 8\color{black} $\,\,$ & $\,\,$\color{gr} 3.4975\color{black}   $\,\,$ \\
$\,\,$0.1208$\,\,$ & $\,\,$\color{gr} \color{blue}  1/8\color{black} $\,\,$ & $\,\,$ 1 $\,\,$ & $\,\,$0.4372 $\,\,$ \\
$\,\,$0.2762$\,\,$ & $\,\,$\color{gr} 0.2859\color{black} $\,\,$ & $\,\,$2.2874$\,\,$ & $\,\,$ 1  $\,\,$ \\
\end{pmatrix},
\end{equation*}
\end{example}
\newpage
\begin{example}
\begin{equation*}
\mathbf{A} =
\begin{pmatrix}
$\,\,$ 1 $\,\,$ & $\,\,$1$\,\,$ & $\,\,$5$\,\,$ & $\,\,$6 $\,\,$ \\
$\,\,$ 1 $\,\,$ & $\,\,$ 1 $\,\,$ & $\,\,$9$\,\,$ & $\,\,$4 $\,\,$ \\
$\,\,$ 1/5$\,\,$ & $\,\,$ 1/9$\,\,$ & $\,\,$ 1 $\,\,$ & $\,\,$ 1/3 $\,\,$ \\
$\,\,$ 1/6$\,\,$ & $\,\,$ 1/4$\,\,$ & $\,\,$3$\,\,$ & $\,\,$ 1  $\,\,$ \\
\end{pmatrix},
\qquad
\lambda_{\max} =
4.1433,
\qquad
CR = 0.0540
\end{equation*}

\begin{equation*}
\mathbf{w}^{EM} =
\begin{pmatrix}
0.420565\\
\color{red} 0.420353\color{black} \\
0.052880\\
0.106202
\end{pmatrix}\end{equation*}
\begin{equation*}
\left[ \frac{{w}^{EM}_i}{{w}^{EM}_j} \right] =
\begin{pmatrix}
$\,\,$ 1 $\,\,$ & $\,\,$\color{red} 1.0005\color{black} $\,\,$ & $\,\,$7.9531$\,\,$ & $\,\,$3.9601$\,\,$ \\
$\,\,$\color{red} 0.9995\color{black} $\,\,$ & $\,\,$ 1 $\,\,$ & $\,\,$\color{red} 7.9491\color{black} $\,\,$ & $\,\,$\color{red} 3.9581\color{black}   $\,\,$ \\
$\,\,$0.1257$\,\,$ & $\,\,$\color{red} 0.1258\color{black} $\,\,$ & $\,\,$ 1 $\,\,$ & $\,\,$0.4979 $\,\,$ \\
$\,\,$0.2525$\,\,$ & $\,\,$\color{red} 0.2526\color{black} $\,\,$ & $\,\,$2.0083$\,\,$ & $\,\,$ 1  $\,\,$ \\
\end{pmatrix},
\end{equation*}

\begin{equation*}
\mathbf{w}^{\prime} =
\begin{pmatrix}
0.420476\\
0.420476\\
0.052869\\
0.106179
\end{pmatrix} =
0.999787\cdot
\begin{pmatrix}
0.420565\\
\color{gr} 0.420565\color{black} \\
0.052880\\
0.106202
\end{pmatrix},
\end{equation*}
\begin{equation*}
\left[ \frac{{w}^{\prime}_i}{{w}^{\prime}_j} \right] =
\begin{pmatrix}
$\,\,$ 1 $\,\,$ & $\,\,$\color{gr} \color{blue} 1\color{black} $\,\,$ & $\,\,$7.9531$\,\,$ & $\,\,$3.9601$\,\,$ \\
$\,\,$\color{gr} \color{blue} 1\color{black} $\,\,$ & $\,\,$ 1 $\,\,$ & $\,\,$\color{gr} 7.9531\color{black} $\,\,$ & $\,\,$\color{gr} 3.9601\color{black}   $\,\,$ \\
$\,\,$0.1257$\,\,$ & $\,\,$\color{gr} 0.1257\color{black} $\,\,$ & $\,\,$ 1 $\,\,$ & $\,\,$0.4979 $\,\,$ \\
$\,\,$0.2525$\,\,$ & $\,\,$\color{gr} 0.2525\color{black} $\,\,$ & $\,\,$2.0083$\,\,$ & $\,\,$ 1  $\,\,$ \\
\end{pmatrix},
\end{equation*}
\end{example}
\newpage
\begin{example}
\begin{equation*}
\mathbf{A} =
\begin{pmatrix}
$\,\,$ 1 $\,\,$ & $\,\,$1$\,\,$ & $\,\,$5$\,\,$ & $\,\,$6 $\,\,$ \\
$\,\,$ 1 $\,\,$ & $\,\,$ 1 $\,\,$ & $\,\,$9$\,\,$ & $\,\,$4 $\,\,$ \\
$\,\,$ 1/5$\,\,$ & $\,\,$ 1/9$\,\,$ & $\,\,$ 1 $\,\,$ & $\,\,$ 1/4 $\,\,$ \\
$\,\,$ 1/6$\,\,$ & $\,\,$ 1/4$\,\,$ & $\,\,$4$\,\,$ & $\,\,$ 1  $\,\,$ \\
\end{pmatrix},
\qquad
\lambda_{\max} =
4.2146,
\qquad
CR = 0.0809
\end{equation*}

\begin{equation*}
\mathbf{w}^{EM} =
\begin{pmatrix}
0.421389\\
\color{red} 0.413512\color{black} \\
0.049500\\
0.115599
\end{pmatrix}\end{equation*}
\begin{equation*}
\left[ \frac{{w}^{EM}_i}{{w}^{EM}_j} \right] =
\begin{pmatrix}
$\,\,$ 1 $\,\,$ & $\,\,$\color{red} 1.0190\color{black} $\,\,$ & $\,\,$8.5129$\,\,$ & $\,\,$3.6453$\,\,$ \\
$\,\,$\color{red} 0.9813\color{black} $\,\,$ & $\,\,$ 1 $\,\,$ & $\,\,$\color{red} 8.3538\color{black} $\,\,$ & $\,\,$\color{red} 3.5771\color{black}   $\,\,$ \\
$\,\,$0.1175$\,\,$ & $\,\,$\color{red} 0.1197\color{black} $\,\,$ & $\,\,$ 1 $\,\,$ & $\,\,$0.4282 $\,\,$ \\
$\,\,$0.2743$\,\,$ & $\,\,$\color{red} 0.2796\color{black} $\,\,$ & $\,\,$2.3353$\,\,$ & $\,\,$ 1  $\,\,$ \\
\end{pmatrix},
\end{equation*}

\begin{equation*}
\mathbf{w}^{\prime} =
\begin{pmatrix}
0.418096\\
0.418096\\
0.049113\\
0.114696
\end{pmatrix} =
0.992184\cdot
\begin{pmatrix}
0.421389\\
\color{gr} 0.421389\color{black} \\
0.049500\\
0.115599
\end{pmatrix},
\end{equation*}
\begin{equation*}
\left[ \frac{{w}^{\prime}_i}{{w}^{\prime}_j} \right] =
\begin{pmatrix}
$\,\,$ 1 $\,\,$ & $\,\,$\color{gr} \color{blue} 1\color{black} $\,\,$ & $\,\,$8.5129$\,\,$ & $\,\,$3.6453$\,\,$ \\
$\,\,$\color{gr} \color{blue} 1\color{black} $\,\,$ & $\,\,$ 1 $\,\,$ & $\,\,$\color{gr} 8.5129\color{black} $\,\,$ & $\,\,$\color{gr} 3.6453\color{black}   $\,\,$ \\
$\,\,$0.1175$\,\,$ & $\,\,$\color{gr} 0.1175\color{black} $\,\,$ & $\,\,$ 1 $\,\,$ & $\,\,$0.4282 $\,\,$ \\
$\,\,$0.2743$\,\,$ & $\,\,$\color{gr} 0.2743\color{black} $\,\,$ & $\,\,$2.3353$\,\,$ & $\,\,$ 1  $\,\,$ \\
\end{pmatrix},
\end{equation*}
\end{example}
\newpage
\begin{example}
\begin{equation*}
\mathbf{A} =
\begin{pmatrix}
$\,\,$ 1 $\,\,$ & $\,\,$1$\,\,$ & $\,\,$5$\,\,$ & $\,\,$7 $\,\,$ \\
$\,\,$ 1 $\,\,$ & $\,\,$ 1 $\,\,$ & $\,\,$7$\,\,$ & $\,\,$4 $\,\,$ \\
$\,\,$ 1/5$\,\,$ & $\,\,$ 1/7$\,\,$ & $\,\,$ 1 $\,\,$ & $\,\,$ 1/3 $\,\,$ \\
$\,\,$ 1/7$\,\,$ & $\,\,$ 1/4$\,\,$ & $\,\,$3$\,\,$ & $\,\,$ 1  $\,\,$ \\
\end{pmatrix},
\qquad
\lambda_{\max} =
4.1793,
\qquad
CR = 0.0676
\end{equation*}

\begin{equation*}
\mathbf{w}^{EM} =
\begin{pmatrix}
0.443409\\
\color{red} 0.395527\color{black} \\
0.056616\\
0.104448
\end{pmatrix}\end{equation*}
\begin{equation*}
\left[ \frac{{w}^{EM}_i}{{w}^{EM}_j} \right] =
\begin{pmatrix}
$\,\,$ 1 $\,\,$ & $\,\,$\color{red} 1.1211\color{black} $\,\,$ & $\,\,$7.8318$\,\,$ & $\,\,$4.2453$\,\,$ \\
$\,\,$\color{red} 0.8920\color{black} $\,\,$ & $\,\,$ 1 $\,\,$ & $\,\,$\color{red} 6.9861\color{black} $\,\,$ & $\,\,$\color{red} 3.7868\color{black}   $\,\,$ \\
$\,\,$0.1277$\,\,$ & $\,\,$\color{red} 0.1431\color{black} $\,\,$ & $\,\,$ 1 $\,\,$ & $\,\,$0.5421 $\,\,$ \\
$\,\,$0.2356$\,\,$ & $\,\,$\color{red} 0.2641\color{black} $\,\,$ & $\,\,$1.8448$\,\,$ & $\,\,$ 1  $\,\,$ \\
\end{pmatrix},
\end{equation*}

\begin{equation*}
\mathbf{w}^{\prime} =
\begin{pmatrix}
0.443060\\
0.396002\\
0.056572\\
0.104366
\end{pmatrix} =
0.999215\cdot
\begin{pmatrix}
0.443409\\
\color{gr} 0.396313\color{black} \\
0.056616\\
0.104448
\end{pmatrix},
\end{equation*}
\begin{equation*}
\left[ \frac{{w}^{\prime}_i}{{w}^{\prime}_j} \right] =
\begin{pmatrix}
$\,\,$ 1 $\,\,$ & $\,\,$\color{gr} 1.1188\color{black} $\,\,$ & $\,\,$7.8318$\,\,$ & $\,\,$4.2453$\,\,$ \\
$\,\,$\color{gr} 0.8938\color{black} $\,\,$ & $\,\,$ 1 $\,\,$ & $\,\,$\color{gr} \color{blue} 7\color{black} $\,\,$ & $\,\,$\color{gr} 3.7944\color{black}   $\,\,$ \\
$\,\,$0.1277$\,\,$ & $\,\,$\color{gr} \color{blue}  1/7\color{black} $\,\,$ & $\,\,$ 1 $\,\,$ & $\,\,$0.5421 $\,\,$ \\
$\,\,$0.2356$\,\,$ & $\,\,$\color{gr} 0.2635\color{black} $\,\,$ & $\,\,$1.8448$\,\,$ & $\,\,$ 1  $\,\,$ \\
\end{pmatrix},
\end{equation*}
\end{example}
\newpage
\begin{example}
\begin{equation*}
\mathbf{A} =
\begin{pmatrix}
$\,\,$ 1 $\,\,$ & $\,\,$1$\,\,$ & $\,\,$5$\,\,$ & $\,\,$7 $\,\,$ \\
$\,\,$ 1 $\,\,$ & $\,\,$ 1 $\,\,$ & $\,\,$7$\,\,$ & $\,\,$5 $\,\,$ \\
$\,\,$ 1/5$\,\,$ & $\,\,$ 1/7$\,\,$ & $\,\,$ 1 $\,\,$ & $\,\,$ 1/2 $\,\,$ \\
$\,\,$ 1/7$\,\,$ & $\,\,$ 1/5$\,\,$ & $\,\,$2$\,\,$ & $\,\,$ 1  $\,\,$ \\
\end{pmatrix},
\qquad
\lambda_{\max} =
4.0899,
\qquad
CR = 0.0339
\end{equation*}

\begin{equation*}
\mathbf{w}^{EM} =
\begin{pmatrix}
0.432065\\
\color{red} 0.419610\color{black} \\
0.061428\\
0.086897
\end{pmatrix}\end{equation*}
\begin{equation*}
\left[ \frac{{w}^{EM}_i}{{w}^{EM}_j} \right] =
\begin{pmatrix}
$\,\,$ 1 $\,\,$ & $\,\,$\color{red} 1.0297\color{black} $\,\,$ & $\,\,$7.0337$\,\,$ & $\,\,$4.9721$\,\,$ \\
$\,\,$\color{red} 0.9712\color{black} $\,\,$ & $\,\,$ 1 $\,\,$ & $\,\,$\color{red} 6.8309\color{black} $\,\,$ & $\,\,$\color{red} 4.8288\color{black}   $\,\,$ \\
$\,\,$0.1422$\,\,$ & $\,\,$\color{red} 0.1464\color{black} $\,\,$ & $\,\,$ 1 $\,\,$ & $\,\,$0.7069 $\,\,$ \\
$\,\,$0.2011$\,\,$ & $\,\,$\color{red} 0.2071\color{black} $\,\,$ & $\,\,$1.4146$\,\,$ & $\,\,$ 1  $\,\,$ \\
\end{pmatrix},
\end{equation*}

\begin{equation*}
\mathbf{w}^{\prime} =
\begin{pmatrix}
0.427623\\
0.425576\\
0.060797\\
0.086004
\end{pmatrix} =
0.989720\cdot
\begin{pmatrix}
0.432065\\
\color{gr} 0.429997\color{black} \\
0.061428\\
0.086897
\end{pmatrix},
\end{equation*}
\begin{equation*}
\left[ \frac{{w}^{\prime}_i}{{w}^{\prime}_j} \right] =
\begin{pmatrix}
$\,\,$ 1 $\,\,$ & $\,\,$\color{gr} 1.0048\color{black} $\,\,$ & $\,\,$7.0337$\,\,$ & $\,\,$4.9721$\,\,$ \\
$\,\,$\color{gr} 0.9952\color{black} $\,\,$ & $\,\,$ 1 $\,\,$ & $\,\,$\color{gr} \color{blue} 7\color{black} $\,\,$ & $\,\,$\color{gr} 4.9483\color{black}   $\,\,$ \\
$\,\,$0.1422$\,\,$ & $\,\,$\color{gr} \color{blue}  1/7\color{black} $\,\,$ & $\,\,$ 1 $\,\,$ & $\,\,$0.7069 $\,\,$ \\
$\,\,$0.2011$\,\,$ & $\,\,$\color{gr} 0.2021\color{black} $\,\,$ & $\,\,$1.4146$\,\,$ & $\,\,$ 1  $\,\,$ \\
\end{pmatrix},
\end{equation*}
\end{example}
\newpage
\begin{example}
\begin{equation*}
\mathbf{A} =
\begin{pmatrix}
$\,\,$ 1 $\,\,$ & $\,\,$1$\,\,$ & $\,\,$5$\,\,$ & $\,\,$7 $\,\,$ \\
$\,\,$ 1 $\,\,$ & $\,\,$ 1 $\,\,$ & $\,\,$8$\,\,$ & $\,\,$4 $\,\,$ \\
$\,\,$ 1/5$\,\,$ & $\,\,$ 1/8$\,\,$ & $\,\,$ 1 $\,\,$ & $\,\,$ 1/3 $\,\,$ \\
$\,\,$ 1/7$\,\,$ & $\,\,$ 1/4$\,\,$ & $\,\,$3$\,\,$ & $\,\,$ 1  $\,\,$ \\
\end{pmatrix},
\qquad
\lambda_{\max} =
4.1782,
\qquad
CR = 0.0672
\end{equation*}

\begin{equation*}
\mathbf{w}^{EM} =
\begin{pmatrix}
0.438878\\
\color{red} 0.404076\color{black} \\
0.054289\\
0.102758
\end{pmatrix}\end{equation*}
\begin{equation*}
\left[ \frac{{w}^{EM}_i}{{w}^{EM}_j} \right] =
\begin{pmatrix}
$\,\,$ 1 $\,\,$ & $\,\,$\color{red} 1.0861\color{black} $\,\,$ & $\,\,$8.0842$\,\,$ & $\,\,$4.2710$\,\,$ \\
$\,\,$\color{red} 0.9207\color{black} $\,\,$ & $\,\,$ 1 $\,\,$ & $\,\,$\color{red} 7.4431\color{black} $\,\,$ & $\,\,$\color{red} 3.9323\color{black}   $\,\,$ \\
$\,\,$0.1237$\,\,$ & $\,\,$\color{red} 0.1344\color{black} $\,\,$ & $\,\,$ 1 $\,\,$ & $\,\,$0.5283 $\,\,$ \\
$\,\,$0.2341$\,\,$ & $\,\,$\color{red} 0.2543\color{black} $\,\,$ & $\,\,$1.8928$\,\,$ & $\,\,$ 1  $\,\,$ \\
\end{pmatrix},
\end{equation*}

\begin{equation*}
\mathbf{w}^{\prime} =
\begin{pmatrix}
0.435846\\
0.408192\\
0.053913\\
0.102048
\end{pmatrix} =
0.993092\cdot
\begin{pmatrix}
0.438878\\
\color{gr} 0.411032\color{black} \\
0.054289\\
0.102758
\end{pmatrix},
\end{equation*}
\begin{equation*}
\left[ \frac{{w}^{\prime}_i}{{w}^{\prime}_j} \right] =
\begin{pmatrix}
$\,\,$ 1 $\,\,$ & $\,\,$\color{gr} 1.0677\color{black} $\,\,$ & $\,\,$8.0842$\,\,$ & $\,\,$4.2710$\,\,$ \\
$\,\,$\color{gr} 0.9366\color{black} $\,\,$ & $\,\,$ 1 $\,\,$ & $\,\,$\color{gr} 7.5712\color{black} $\,\,$ & $\,\,$\color{gr} \color{blue} 4\color{black}   $\,\,$ \\
$\,\,$0.1237$\,\,$ & $\,\,$\color{gr} 0.1321\color{black} $\,\,$ & $\,\,$ 1 $\,\,$ & $\,\,$0.5283 $\,\,$ \\
$\,\,$0.2341$\,\,$ & $\,\,$\color{gr} \color{blue}  1/4\color{black} $\,\,$ & $\,\,$1.8928$\,\,$ & $\,\,$ 1  $\,\,$ \\
\end{pmatrix},
\end{equation*}
\end{example}
\newpage
\begin{example}
\begin{equation*}
\mathbf{A} =
\begin{pmatrix}
$\,\,$ 1 $\,\,$ & $\,\,$1$\,\,$ & $\,\,$5$\,\,$ & $\,\,$7 $\,\,$ \\
$\,\,$ 1 $\,\,$ & $\,\,$ 1 $\,\,$ & $\,\,$8$\,\,$ & $\,\,$4 $\,\,$ \\
$\,\,$ 1/5$\,\,$ & $\,\,$ 1/8$\,\,$ & $\,\,$ 1 $\,\,$ & $\,\,$ 1/4 $\,\,$ \\
$\,\,$ 1/7$\,\,$ & $\,\,$ 1/4$\,\,$ & $\,\,$4$\,\,$ & $\,\,$ 1  $\,\,$ \\
\end{pmatrix},
\qquad
\lambda_{\max} =
4.2610,
\qquad
CR = 0.0984
\end{equation*}

\begin{equation*}
\mathbf{w}^{EM} =
\begin{pmatrix}
0.440137\\
\color{red} 0.397026\color{black} \\
0.050802\\
0.112034
\end{pmatrix}\end{equation*}
\begin{equation*}
\left[ \frac{{w}^{EM}_i}{{w}^{EM}_j} \right] =
\begin{pmatrix}
$\,\,$ 1 $\,\,$ & $\,\,$\color{red} 1.1086\color{black} $\,\,$ & $\,\,$8.6638$\,\,$ & $\,\,$3.9286$\,\,$ \\
$\,\,$\color{red} 0.9020\color{black} $\,\,$ & $\,\,$ 1 $\,\,$ & $\,\,$\color{red} 7.8152\color{black} $\,\,$ & $\,\,$\color{red} 3.5438\color{black}   $\,\,$ \\
$\,\,$0.1154$\,\,$ & $\,\,$\color{red} 0.1280\color{black} $\,\,$ & $\,\,$ 1 $\,\,$ & $\,\,$0.4535 $\,\,$ \\
$\,\,$0.2545$\,\,$ & $\,\,$\color{red} 0.2822\color{black} $\,\,$ & $\,\,$2.2053$\,\,$ & $\,\,$ 1  $\,\,$ \\
\end{pmatrix},
\end{equation*}

\begin{equation*}
\mathbf{w}^{\prime} =
\begin{pmatrix}
0.436043\\
0.402636\\
0.050329\\
0.110992
\end{pmatrix} =
0.990697\cdot
\begin{pmatrix}
0.440137\\
\color{gr} 0.406417\color{black} \\
0.050802\\
0.112034
\end{pmatrix},
\end{equation*}
\begin{equation*}
\left[ \frac{{w}^{\prime}_i}{{w}^{\prime}_j} \right] =
\begin{pmatrix}
$\,\,$ 1 $\,\,$ & $\,\,$\color{gr} 1.0830\color{black} $\,\,$ & $\,\,$8.6638$\,\,$ & $\,\,$3.9286$\,\,$ \\
$\,\,$\color{gr} 0.9234\color{black} $\,\,$ & $\,\,$ 1 $\,\,$ & $\,\,$\color{gr} \color{blue} 8\color{black} $\,\,$ & $\,\,$\color{gr} 3.6276\color{black}   $\,\,$ \\
$\,\,$0.1154$\,\,$ & $\,\,$\color{gr} \color{blue}  1/8\color{black} $\,\,$ & $\,\,$ 1 $\,\,$ & $\,\,$0.4535 $\,\,$ \\
$\,\,$0.2545$\,\,$ & $\,\,$\color{gr} 0.2757\color{black} $\,\,$ & $\,\,$2.2053$\,\,$ & $\,\,$ 1  $\,\,$ \\
\end{pmatrix},
\end{equation*}
\end{example}
\newpage
\begin{example}
\begin{equation*}
\mathbf{A} =
\begin{pmatrix}
$\,\,$ 1 $\,\,$ & $\,\,$1$\,\,$ & $\,\,$5$\,\,$ & $\,\,$7 $\,\,$ \\
$\,\,$ 1 $\,\,$ & $\,\,$ 1 $\,\,$ & $\,\,$9$\,\,$ & $\,\,$4 $\,\,$ \\
$\,\,$ 1/5$\,\,$ & $\,\,$ 1/9$\,\,$ & $\,\,$ 1 $\,\,$ & $\,\,$ 1/4 $\,\,$ \\
$\,\,$ 1/7$\,\,$ & $\,\,$ 1/4$\,\,$ & $\,\,$4$\,\,$ & $\,\,$ 1  $\,\,$ \\
\end{pmatrix},
\qquad
\lambda_{\max} =
4.2594,
\qquad
CR = 0.0978
\end{equation*}

\begin{equation*}
\mathbf{w}^{EM} =
\begin{pmatrix}
0.436173\\
\color{red} 0.404518\color{black} \\
0.049014\\
0.110295
\end{pmatrix}\end{equation*}
\begin{equation*}
\left[ \frac{{w}^{EM}_i}{{w}^{EM}_j} \right] =
\begin{pmatrix}
$\,\,$ 1 $\,\,$ & $\,\,$\color{red} 1.0783\color{black} $\,\,$ & $\,\,$8.8990$\,\,$ & $\,\,$3.9546$\,\,$ \\
$\,\,$\color{red} 0.9274\color{black} $\,\,$ & $\,\,$ 1 $\,\,$ & $\,\,$\color{red} 8.2531\color{black} $\,\,$ & $\,\,$\color{red} 3.6676\color{black}   $\,\,$ \\
$\,\,$0.1124$\,\,$ & $\,\,$\color{red} 0.1212\color{black} $\,\,$ & $\,\,$ 1 $\,\,$ & $\,\,$0.4444 $\,\,$ \\
$\,\,$0.2529$\,\,$ & $\,\,$\color{red} 0.2727\color{black} $\,\,$ & $\,\,$2.2503$\,\,$ & $\,\,$ 1  $\,\,$ \\
\end{pmatrix},
\end{equation*}

\begin{equation*}
\mathbf{w}^{\prime} =
\begin{pmatrix}
0.422789\\
0.422789\\
0.047510\\
0.106911
\end{pmatrix} =
0.969316\cdot
\begin{pmatrix}
0.436173\\
\color{gr} 0.436173\color{black} \\
0.049014\\
0.110295
\end{pmatrix},
\end{equation*}
\begin{equation*}
\left[ \frac{{w}^{\prime}_i}{{w}^{\prime}_j} \right] =
\begin{pmatrix}
$\,\,$ 1 $\,\,$ & $\,\,$\color{gr} \color{blue} 1\color{black} $\,\,$ & $\,\,$8.8990$\,\,$ & $\,\,$3.9546$\,\,$ \\
$\,\,$\color{gr} \color{blue} 1\color{black} $\,\,$ & $\,\,$ 1 $\,\,$ & $\,\,$\color{gr} 8.8990\color{black} $\,\,$ & $\,\,$\color{gr} 3.9546\color{black}   $\,\,$ \\
$\,\,$0.1124$\,\,$ & $\,\,$\color{gr} 0.1124\color{black} $\,\,$ & $\,\,$ 1 $\,\,$ & $\,\,$0.4444 $\,\,$ \\
$\,\,$0.2529$\,\,$ & $\,\,$\color{gr} 0.2529\color{black} $\,\,$ & $\,\,$2.2503$\,\,$ & $\,\,$ 1  $\,\,$ \\
\end{pmatrix},
\end{equation*}
\end{example}
\newpage
\begin{example}
\begin{equation*}
\mathbf{A} =
\begin{pmatrix}
$\,\,$ 1 $\,\,$ & $\,\,$1$\,\,$ & $\,\,$5$\,\,$ & $\,\,$8 $\,\,$ \\
$\,\,$ 1 $\,\,$ & $\,\,$ 1 $\,\,$ & $\,\,$7$\,\,$ & $\,\,$4 $\,\,$ \\
$\,\,$ 1/5$\,\,$ & $\,\,$ 1/7$\,\,$ & $\,\,$ 1 $\,\,$ & $\,\,$ 1/3 $\,\,$ \\
$\,\,$ 1/8$\,\,$ & $\,\,$ 1/4$\,\,$ & $\,\,$3$\,\,$ & $\,\,$ 1  $\,\,$ \\
\end{pmatrix},
\qquad
\lambda_{\max} =
4.2174,
\qquad
CR = 0.0820
\end{equation*}

\begin{equation*}
\mathbf{w}^{EM} =
\begin{pmatrix}
0.456187\\
\color{red} 0.387850\color{black} \\
0.055941\\
0.100022
\end{pmatrix}\end{equation*}
\begin{equation*}
\left[ \frac{{w}^{EM}_i}{{w}^{EM}_j} \right] =
\begin{pmatrix}
$\,\,$ 1 $\,\,$ & $\,\,$\color{red} 1.1762\color{black} $\,\,$ & $\,\,$8.1547$\,\,$ & $\,\,$4.5609$\,\,$ \\
$\,\,$\color{red} 0.8502\color{black} $\,\,$ & $\,\,$ 1 $\,\,$ & $\,\,$\color{red} 6.9332\color{black} $\,\,$ & $\,\,$\color{red} 3.8776\color{black}   $\,\,$ \\
$\,\,$0.1226$\,\,$ & $\,\,$\color{red} 0.1442\color{black} $\,\,$ & $\,\,$ 1 $\,\,$ & $\,\,$0.5593 $\,\,$ \\
$\,\,$0.2193$\,\,$ & $\,\,$\color{red} 0.2579\color{black} $\,\,$ & $\,\,$1.7880$\,\,$ & $\,\,$ 1  $\,\,$ \\
\end{pmatrix},
\end{equation*}

\begin{equation*}
\mathbf{w}^{\prime} =
\begin{pmatrix}
0.454487\\
0.390130\\
0.055733\\
0.099649
\end{pmatrix} =
0.996274\cdot
\begin{pmatrix}
0.456187\\
\color{gr} 0.391589\color{black} \\
0.055941\\
0.100022
\end{pmatrix},
\end{equation*}
\begin{equation*}
\left[ \frac{{w}^{\prime}_i}{{w}^{\prime}_j} \right] =
\begin{pmatrix}
$\,\,$ 1 $\,\,$ & $\,\,$\color{gr} 1.1650\color{black} $\,\,$ & $\,\,$8.1547$\,\,$ & $\,\,$4.5609$\,\,$ \\
$\,\,$\color{gr} 0.8584\color{black} $\,\,$ & $\,\,$ 1 $\,\,$ & $\,\,$\color{gr} \color{blue} 7\color{black} $\,\,$ & $\,\,$\color{gr} 3.9150\color{black}   $\,\,$ \\
$\,\,$0.1226$\,\,$ & $\,\,$\color{gr} \color{blue}  1/7\color{black} $\,\,$ & $\,\,$ 1 $\,\,$ & $\,\,$0.5593 $\,\,$ \\
$\,\,$0.2193$\,\,$ & $\,\,$\color{gr} 0.2554\color{black} $\,\,$ & $\,\,$1.7880$\,\,$ & $\,\,$ 1  $\,\,$ \\
\end{pmatrix},
\end{equation*}
\end{example}
\newpage
\begin{example}
\begin{equation*}
\mathbf{A} =
\begin{pmatrix}
$\,\,$ 1 $\,\,$ & $\,\,$1$\,\,$ & $\,\,$5$\,\,$ & $\,\,$8 $\,\,$ \\
$\,\,$ 1 $\,\,$ & $\,\,$ 1 $\,\,$ & $\,\,$7$\,\,$ & $\,\,$5 $\,\,$ \\
$\,\,$ 1/5$\,\,$ & $\,\,$ 1/7$\,\,$ & $\,\,$ 1 $\,\,$ & $\,\,$ 1/2 $\,\,$ \\
$\,\,$ 1/8$\,\,$ & $\,\,$ 1/5$\,\,$ & $\,\,$2$\,\,$ & $\,\,$ 1  $\,\,$ \\
\end{pmatrix},
\qquad
\lambda_{\max} =
4.1159,
\qquad
CR = 0.0437
\end{equation*}

\begin{equation*}
\mathbf{w}^{EM} =
\begin{pmatrix}
0.443586\\
\color{red} 0.412418\color{black} \\
0.060741\\
0.083255
\end{pmatrix}\end{equation*}
\begin{equation*}
\left[ \frac{{w}^{EM}_i}{{w}^{EM}_j} \right] =
\begin{pmatrix}
$\,\,$ 1 $\,\,$ & $\,\,$\color{red} 1.0756\color{black} $\,\,$ & $\,\,$7.3029$\,\,$ & $\,\,$5.3280$\,\,$ \\
$\,\,$\color{red} 0.9297\color{black} $\,\,$ & $\,\,$ 1 $\,\,$ & $\,\,$\color{red} 6.7898\color{black} $\,\,$ & $\,\,$\color{red} 4.9537\color{black}   $\,\,$ \\
$\,\,$0.1369$\,\,$ & $\,\,$\color{red} 0.1473\color{black} $\,\,$ & $\,\,$ 1 $\,\,$ & $\,\,$0.7296 $\,\,$ \\
$\,\,$0.1877$\,\,$ & $\,\,$\color{red} 0.2019\color{black} $\,\,$ & $\,\,$1.3707$\,\,$ & $\,\,$ 1  $\,\,$ \\
\end{pmatrix},
\end{equation*}

\begin{equation*}
\mathbf{w}^{\prime} =
\begin{pmatrix}
0.441881\\
0.414676\\
0.060507\\
0.082935
\end{pmatrix} =
0.996157\cdot
\begin{pmatrix}
0.443586\\
\color{gr} 0.416276\color{black} \\
0.060741\\
0.083255
\end{pmatrix},
\end{equation*}
\begin{equation*}
\left[ \frac{{w}^{\prime}_i}{{w}^{\prime}_j} \right] =
\begin{pmatrix}
$\,\,$ 1 $\,\,$ & $\,\,$\color{gr} 1.0656\color{black} $\,\,$ & $\,\,$7.3029$\,\,$ & $\,\,$5.3280$\,\,$ \\
$\,\,$\color{gr} 0.9384\color{black} $\,\,$ & $\,\,$ 1 $\,\,$ & $\,\,$\color{gr} 6.8533\color{black} $\,\,$ & $\,\,$\color{gr} \color{blue} 5\color{black}   $\,\,$ \\
$\,\,$0.1369$\,\,$ & $\,\,$\color{gr} 0.1459\color{black} $\,\,$ & $\,\,$ 1 $\,\,$ & $\,\,$0.7296 $\,\,$ \\
$\,\,$0.1877$\,\,$ & $\,\,$\color{gr} \color{blue}  1/5\color{black} $\,\,$ & $\,\,$1.3707$\,\,$ & $\,\,$ 1  $\,\,$ \\
\end{pmatrix},
\end{equation*}
\end{example}
\newpage
\begin{example}
\begin{equation*}
\mathbf{A} =
\begin{pmatrix}
$\,\,$ 1 $\,\,$ & $\,\,$1$\,\,$ & $\,\,$5$\,\,$ & $\,\,$8 $\,\,$ \\
$\,\,$ 1 $\,\,$ & $\,\,$ 1 $\,\,$ & $\,\,$8$\,\,$ & $\,\,$5 $\,\,$ \\
$\,\,$ 1/5$\,\,$ & $\,\,$ 1/8$\,\,$ & $\,\,$ 1 $\,\,$ & $\,\,$ 1/3 $\,\,$ \\
$\,\,$ 1/8$\,\,$ & $\,\,$ 1/5$\,\,$ & $\,\,$3$\,\,$ & $\,\,$ 1  $\,\,$ \\
\end{pmatrix},
\qquad
\lambda_{\max} =
4.2144,
\qquad
CR = 0.0808
\end{equation*}

\begin{equation*}
\mathbf{w}^{EM} =
\begin{pmatrix}
0.441225\\
\color{red} 0.413230\color{black} \\
0.053109\\
0.092436
\end{pmatrix}\end{equation*}
\begin{equation*}
\left[ \frac{{w}^{EM}_i}{{w}^{EM}_j} \right] =
\begin{pmatrix}
$\,\,$ 1 $\,\,$ & $\,\,$\color{red} 1.0677\color{black} $\,\,$ & $\,\,$8.3080$\,\,$ & $\,\,$4.7733$\,\,$ \\
$\,\,$\color{red} 0.9366\color{black} $\,\,$ & $\,\,$ 1 $\,\,$ & $\,\,$\color{red} 7.7808\color{black} $\,\,$ & $\,\,$\color{red} 4.4704\color{black}   $\,\,$ \\
$\,\,$0.1204$\,\,$ & $\,\,$\color{red} 0.1285\color{black} $\,\,$ & $\,\,$ 1 $\,\,$ & $\,\,$0.5745 $\,\,$ \\
$\,\,$0.2095$\,\,$ & $\,\,$\color{red} 0.2237\color{black} $\,\,$ & $\,\,$1.7405$\,\,$ & $\,\,$ 1  $\,\,$ \\
\end{pmatrix},
\end{equation*}

\begin{equation*}
\mathbf{w}^{\prime} =
\begin{pmatrix}
0.436149\\
0.419981\\
0.052498\\
0.091373
\end{pmatrix} =
0.988495\cdot
\begin{pmatrix}
0.441225\\
\color{gr} 0.424869\color{black} \\
0.053109\\
0.092436
\end{pmatrix},
\end{equation*}
\begin{equation*}
\left[ \frac{{w}^{\prime}_i}{{w}^{\prime}_j} \right] =
\begin{pmatrix}
$\,\,$ 1 $\,\,$ & $\,\,$\color{gr} 1.0385\color{black} $\,\,$ & $\,\,$8.3080$\,\,$ & $\,\,$4.7733$\,\,$ \\
$\,\,$\color{gr} 0.9629\color{black} $\,\,$ & $\,\,$ 1 $\,\,$ & $\,\,$\color{gr} \color{blue} 8\color{black} $\,\,$ & $\,\,$\color{gr} 4.5963\color{black}   $\,\,$ \\
$\,\,$0.1204$\,\,$ & $\,\,$\color{gr} \color{blue}  1/8\color{black} $\,\,$ & $\,\,$ 1 $\,\,$ & $\,\,$0.5745 $\,\,$ \\
$\,\,$0.2095$\,\,$ & $\,\,$\color{gr} 0.2176\color{black} $\,\,$ & $\,\,$1.7405$\,\,$ & $\,\,$ 1  $\,\,$ \\
\end{pmatrix},
\end{equation*}
\end{example}
\newpage
\begin{example}
\begin{equation*}
\mathbf{A} =
\begin{pmatrix}
$\,\,$ 1 $\,\,$ & $\,\,$1$\,\,$ & $\,\,$5$\,\,$ & $\,\,$9 $\,\,$ \\
$\,\,$ 1 $\,\,$ & $\,\,$ 1 $\,\,$ & $\,\,$7$\,\,$ & $\,\,$4 $\,\,$ \\
$\,\,$ 1/5$\,\,$ & $\,\,$ 1/7$\,\,$ & $\,\,$ 1 $\,\,$ & $\,\,$ 1/3 $\,\,$ \\
$\,\,$ 1/9$\,\,$ & $\,\,$ 1/4$\,\,$ & $\,\,$3$\,\,$ & $\,\,$ 1  $\,\,$ \\
\end{pmatrix},
\qquad
\lambda_{\max} =
4.2553,
\qquad
CR = 0.0963
\end{equation*}

\begin{equation*}
\mathbf{w}^{EM} =
\begin{pmatrix}
0.467772\\
\color{red} 0.380765\color{black} \\
0.055296\\
0.096167
\end{pmatrix}\end{equation*}
\begin{equation*}
\left[ \frac{{w}^{EM}_i}{{w}^{EM}_j} \right] =
\begin{pmatrix}
$\,\,$ 1 $\,\,$ & $\,\,$\color{red} 1.2285\color{black} $\,\,$ & $\,\,$8.4595$\,\,$ & $\,\,$4.8642$\,\,$ \\
$\,\,$\color{red} 0.8140\color{black} $\,\,$ & $\,\,$ 1 $\,\,$ & $\,\,$\color{red} 6.8860\color{black} $\,\,$ & $\,\,$\color{red} 3.9594\color{black}   $\,\,$ \\
$\,\,$0.1182$\,\,$ & $\,\,$\color{red} 0.1452\color{black} $\,\,$ & $\,\,$ 1 $\,\,$ & $\,\,$0.5750 $\,\,$ \\
$\,\,$0.2056$\,\,$ & $\,\,$\color{red} 0.2526\color{black} $\,\,$ & $\,\,$1.7391$\,\,$ & $\,\,$ 1  $\,\,$ \\
\end{pmatrix},
\end{equation*}

\begin{equation*}
\mathbf{w}^{\prime} =
\begin{pmatrix}
0.465954\\
0.383172\\
0.055081\\
0.095793
\end{pmatrix} =
0.996113\cdot
\begin{pmatrix}
0.467772\\
\color{gr} 0.384667\color{black} \\
0.055296\\
0.096167
\end{pmatrix},
\end{equation*}
\begin{equation*}
\left[ \frac{{w}^{\prime}_i}{{w}^{\prime}_j} \right] =
\begin{pmatrix}
$\,\,$ 1 $\,\,$ & $\,\,$\color{gr} 1.2160\color{black} $\,\,$ & $\,\,$8.4595$\,\,$ & $\,\,$4.8642$\,\,$ \\
$\,\,$\color{gr} 0.8223\color{black} $\,\,$ & $\,\,$ 1 $\,\,$ & $\,\,$\color{gr} 6.9565\color{black} $\,\,$ & $\,\,$\color{gr} \color{blue} 4\color{black}   $\,\,$ \\
$\,\,$0.1182$\,\,$ & $\,\,$\color{gr} 0.1437\color{black} $\,\,$ & $\,\,$ 1 $\,\,$ & $\,\,$0.5750 $\,\,$ \\
$\,\,$0.2056$\,\,$ & $\,\,$\color{gr} \color{blue}  1/4\color{black} $\,\,$ & $\,\,$1.7391$\,\,$ & $\,\,$ 1  $\,\,$ \\
\end{pmatrix},
\end{equation*}
\end{example}
\newpage
\begin{example}
\begin{equation*}
\mathbf{A} =
\begin{pmatrix}
$\,\,$ 1 $\,\,$ & $\,\,$1$\,\,$ & $\,\,$5$\,\,$ & $\,\,$9 $\,\,$ \\
$\,\,$ 1 $\,\,$ & $\,\,$ 1 $\,\,$ & $\,\,$8$\,\,$ & $\,\,$5 $\,\,$ \\
$\,\,$ 1/5$\,\,$ & $\,\,$ 1/8$\,\,$ & $\,\,$ 1 $\,\,$ & $\,\,$ 1/3 $\,\,$ \\
$\,\,$ 1/9$\,\,$ & $\,\,$ 1/5$\,\,$ & $\,\,$3$\,\,$ & $\,\,$ 1  $\,\,$ \\
\end{pmatrix},
\qquad
\lambda_{\max} =
4.2489,
\qquad
CR = 0.0939
\end{equation*}

\begin{equation*}
\mathbf{w}^{EM} =
\begin{pmatrix}
0.452517\\
\color{red} 0.405842\color{black} \\
0.052606\\
0.089035
\end{pmatrix}\end{equation*}
\begin{equation*}
\left[ \frac{{w}^{EM}_i}{{w}^{EM}_j} \right] =
\begin{pmatrix}
$\,\,$ 1 $\,\,$ & $\,\,$\color{red} 1.1150\color{black} $\,\,$ & $\,\,$8.6020$\,\,$ & $\,\,$5.0825$\,\,$ \\
$\,\,$\color{red} 0.8969\color{black} $\,\,$ & $\,\,$ 1 $\,\,$ & $\,\,$\color{red} 7.7147\color{black} $\,\,$ & $\,\,$\color{red} 4.5582\color{black}   $\,\,$ \\
$\,\,$0.1163$\,\,$ & $\,\,$\color{red} 0.1296\color{black} $\,\,$ & $\,\,$ 1 $\,\,$ & $\,\,$0.5908 $\,\,$ \\
$\,\,$0.1968$\,\,$ & $\,\,$\color{red} 0.2194\color{black} $\,\,$ & $\,\,$1.6925$\,\,$ & $\,\,$ 1  $\,\,$ \\
\end{pmatrix},
\end{equation*}

\begin{equation*}
\mathbf{w}^{\prime} =
\begin{pmatrix}
0.445827\\
0.414626\\
0.051828\\
0.087719
\end{pmatrix} =
0.985215\cdot
\begin{pmatrix}
0.452517\\
\color{gr} 0.420848\color{black} \\
0.052606\\
0.089035
\end{pmatrix},
\end{equation*}
\begin{equation*}
\left[ \frac{{w}^{\prime}_i}{{w}^{\prime}_j} \right] =
\begin{pmatrix}
$\,\,$ 1 $\,\,$ & $\,\,$\color{gr} 1.0753\color{black} $\,\,$ & $\,\,$8.6020$\,\,$ & $\,\,$5.0825$\,\,$ \\
$\,\,$\color{gr} 0.9300\color{black} $\,\,$ & $\,\,$ 1 $\,\,$ & $\,\,$\color{gr} \color{blue} 8\color{black} $\,\,$ & $\,\,$\color{gr} 4.7268\color{black}   $\,\,$ \\
$\,\,$0.1163$\,\,$ & $\,\,$\color{gr} \color{blue}  1/8\color{black} $\,\,$ & $\,\,$ 1 $\,\,$ & $\,\,$0.5908 $\,\,$ \\
$\,\,$0.1968$\,\,$ & $\,\,$\color{gr} 0.2116\color{black} $\,\,$ & $\,\,$1.6925$\,\,$ & $\,\,$ 1  $\,\,$ \\
\end{pmatrix},
\end{equation*}
\end{example}
\newpage
\begin{example}
\begin{equation*}
\mathbf{A} =
\begin{pmatrix}
$\,\,$ 1 $\,\,$ & $\,\,$1$\,\,$ & $\,\,$5$\,\,$ & $\,\,$9 $\,\,$ \\
$\,\,$ 1 $\,\,$ & $\,\,$ 1 $\,\,$ & $\,\,$9$\,\,$ & $\,\,$5 $\,\,$ \\
$\,\,$ 1/5$\,\,$ & $\,\,$ 1/9$\,\,$ & $\,\,$ 1 $\,\,$ & $\,\,$ 1/3 $\,\,$ \\
$\,\,$ 1/9$\,\,$ & $\,\,$ 1/5$\,\,$ & $\,\,$3$\,\,$ & $\,\,$ 1  $\,\,$ \\
\end{pmatrix},
\qquad
\lambda_{\max} =
4.2483,
\qquad
CR = 0.0936
\end{equation*}

\begin{equation*}
\mathbf{w}^{EM} =
\begin{pmatrix}
0.448190\\
\color{red} 0.413439\color{black} \\
0.050731\\
0.087640
\end{pmatrix}\end{equation*}
\begin{equation*}
\left[ \frac{{w}^{EM}_i}{{w}^{EM}_j} \right] =
\begin{pmatrix}
$\,\,$ 1 $\,\,$ & $\,\,$\color{red} 1.0841\color{black} $\,\,$ & $\,\,$8.8346$\,\,$ & $\,\,$5.1140$\,\,$ \\
$\,\,$\color{red} 0.9225\color{black} $\,\,$ & $\,\,$ 1 $\,\,$ & $\,\,$\color{red} 8.1496\color{black} $\,\,$ & $\,\,$\color{red} 4.7175\color{black}   $\,\,$ \\
$\,\,$0.1132$\,\,$ & $\,\,$\color{red} 0.1227\color{black} $\,\,$ & $\,\,$ 1 $\,\,$ & $\,\,$0.5789 $\,\,$ \\
$\,\,$0.1955$\,\,$ & $\,\,$\color{red} 0.2120\color{black} $\,\,$ & $\,\,$1.7275$\,\,$ & $\,\,$ 1  $\,\,$ \\
\end{pmatrix},
\end{equation*}

\begin{equation*}
\mathbf{w}^{\prime} =
\begin{pmatrix}
0.437361\\
0.427611\\
0.049505\\
0.085522
\end{pmatrix} =
0.975838\cdot
\begin{pmatrix}
0.448190\\
\color{gr} 0.438199\color{black} \\
0.050731\\
0.087640
\end{pmatrix},
\end{equation*}
\begin{equation*}
\left[ \frac{{w}^{\prime}_i}{{w}^{\prime}_j} \right] =
\begin{pmatrix}
$\,\,$ 1 $\,\,$ & $\,\,$\color{gr} 1.0228\color{black} $\,\,$ & $\,\,$8.8346$\,\,$ & $\,\,$5.1140$\,\,$ \\
$\,\,$\color{gr} 0.9777\color{black} $\,\,$ & $\,\,$ 1 $\,\,$ & $\,\,$\color{gr} 8.6377\color{black} $\,\,$ & $\,\,$\color{gr} \color{blue} 5\color{black}   $\,\,$ \\
$\,\,$0.1132$\,\,$ & $\,\,$\color{gr} 0.1158\color{black} $\,\,$ & $\,\,$ 1 $\,\,$ & $\,\,$0.5789 $\,\,$ \\
$\,\,$0.1955$\,\,$ & $\,\,$\color{gr} \color{blue}  1/5\color{black} $\,\,$ & $\,\,$1.7275$\,\,$ & $\,\,$ 1  $\,\,$ \\
\end{pmatrix},
\end{equation*}
\end{example}
\newpage
\begin{example}
\begin{equation*}
\mathbf{A} =
\begin{pmatrix}
$\,\,$ 1 $\,\,$ & $\,\,$1$\,\,$ & $\,\,$6$\,\,$ & $\,\,$3 $\,\,$ \\
$\,\,$ 1 $\,\,$ & $\,\,$ 1 $\,\,$ & $\,\,$8$\,\,$ & $\,\,$2 $\,\,$ \\
$\,\,$ 1/6$\,\,$ & $\,\,$ 1/8$\,\,$ & $\,\,$ 1 $\,\,$ & $\,\,$ 1/6 $\,\,$ \\
$\,\,$ 1/3$\,\,$ & $\,\,$ 1/2$\,\,$ & $\,\,$6$\,\,$ & $\,\,$ 1  $\,\,$ \\
\end{pmatrix},
\qquad
\lambda_{\max} =
4.1031,
\qquad
CR = 0.0389
\end{equation*}

\begin{equation*}
\mathbf{w}^{EM} =
\begin{pmatrix}
0.393291\\
\color{red} 0.369244\color{black} \\
0.046266\\
0.191199
\end{pmatrix}\end{equation*}
\begin{equation*}
\left[ \frac{{w}^{EM}_i}{{w}^{EM}_j} \right] =
\begin{pmatrix}
$\,\,$ 1 $\,\,$ & $\,\,$\color{red} 1.0651\color{black} $\,\,$ & $\,\,$8.5006$\,\,$ & $\,\,$2.0570$\,\,$ \\
$\,\,$\color{red} 0.9389\color{black} $\,\,$ & $\,\,$ 1 $\,\,$ & $\,\,$\color{red} 7.9809\color{black} $\,\,$ & $\,\,$\color{red} 1.9312\color{black}   $\,\,$ \\
$\,\,$0.1176$\,\,$ & $\,\,$\color{red} 0.1253\color{black} $\,\,$ & $\,\,$ 1 $\,\,$ & $\,\,$0.2420 $\,\,$ \\
$\,\,$0.4862$\,\,$ & $\,\,$\color{red} 0.5178\color{black} $\,\,$ & $\,\,$4.1326$\,\,$ & $\,\,$ 1  $\,\,$ \\
\end{pmatrix},
\end{equation*}

\begin{equation*}
\mathbf{w}^{\prime} =
\begin{pmatrix}
0.392943\\
0.369802\\
0.046225\\
0.191030
\end{pmatrix} =
0.999116\cdot
\begin{pmatrix}
0.393291\\
\color{gr} 0.370129\color{black} \\
0.046266\\
0.191199
\end{pmatrix},
\end{equation*}
\begin{equation*}
\left[ \frac{{w}^{\prime}_i}{{w}^{\prime}_j} \right] =
\begin{pmatrix}
$\,\,$ 1 $\,\,$ & $\,\,$\color{gr} 1.0626\color{black} $\,\,$ & $\,\,$8.5006$\,\,$ & $\,\,$2.0570$\,\,$ \\
$\,\,$\color{gr} 0.9411\color{black} $\,\,$ & $\,\,$ 1 $\,\,$ & $\,\,$\color{gr} \color{blue} 8\color{black} $\,\,$ & $\,\,$\color{gr} 1.9358\color{black}   $\,\,$ \\
$\,\,$0.1176$\,\,$ & $\,\,$\color{gr} \color{blue}  1/8\color{black} $\,\,$ & $\,\,$ 1 $\,\,$ & $\,\,$0.2420 $\,\,$ \\
$\,\,$0.4862$\,\,$ & $\,\,$\color{gr} 0.5166\color{black} $\,\,$ & $\,\,$4.1326$\,\,$ & $\,\,$ 1  $\,\,$ \\
\end{pmatrix},
\end{equation*}
\end{example}
\newpage
\begin{example}
\begin{equation*}
\mathbf{A} =
\begin{pmatrix}
$\,\,$ 1 $\,\,$ & $\,\,$1$\,\,$ & $\,\,$6$\,\,$ & $\,\,$3 $\,\,$ \\
$\,\,$ 1 $\,\,$ & $\,\,$ 1 $\,\,$ & $\,\,$9$\,\,$ & $\,\,$2 $\,\,$ \\
$\,\,$ 1/6$\,\,$ & $\,\,$ 1/9$\,\,$ & $\,\,$ 1 $\,\,$ & $\,\,$ 1/6 $\,\,$ \\
$\,\,$ 1/3$\,\,$ & $\,\,$ 1/2$\,\,$ & $\,\,$6$\,\,$ & $\,\,$ 1  $\,\,$ \\
\end{pmatrix},
\qquad
\lambda_{\max} =
4.1031,
\qquad
CR = 0.0389
\end{equation*}

\begin{equation*}
\mathbf{w}^{EM} =
\begin{pmatrix}
0.390040\\
\color{red} 0.376624\color{black} \\
0.044572\\
0.188763
\end{pmatrix}\end{equation*}
\begin{equation*}
\left[ \frac{{w}^{EM}_i}{{w}^{EM}_j} \right] =
\begin{pmatrix}
$\,\,$ 1 $\,\,$ & $\,\,$\color{red} 1.0356\color{black} $\,\,$ & $\,\,$8.7507$\,\,$ & $\,\,$2.0663$\,\,$ \\
$\,\,$\color{red} 0.9656\color{black} $\,\,$ & $\,\,$ 1 $\,\,$ & $\,\,$\color{red} 8.4497\color{black} $\,\,$ & $\,\,$\color{red} 1.9952\color{black}   $\,\,$ \\
$\,\,$0.1143$\,\,$ & $\,\,$\color{red} 0.1183\color{black} $\,\,$ & $\,\,$ 1 $\,\,$ & $\,\,$0.2361 $\,\,$ \\
$\,\,$0.4840$\,\,$ & $\,\,$\color{red} 0.5012\color{black} $\,\,$ & $\,\,$4.2350$\,\,$ & $\,\,$ 1  $\,\,$ \\
\end{pmatrix},
\end{equation*}

\begin{equation*}
\mathbf{w}^{\prime} =
\begin{pmatrix}
0.389688\\
0.377186\\
0.044532\\
0.188593
\end{pmatrix} =
0.999098\cdot
\begin{pmatrix}
0.390040\\
\color{gr} 0.377527\color{black} \\
0.044572\\
0.188763
\end{pmatrix},
\end{equation*}
\begin{equation*}
\left[ \frac{{w}^{\prime}_i}{{w}^{\prime}_j} \right] =
\begin{pmatrix}
$\,\,$ 1 $\,\,$ & $\,\,$\color{gr} 1.0331\color{black} $\,\,$ & $\,\,$8.7507$\,\,$ & $\,\,$2.0663$\,\,$ \\
$\,\,$\color{gr} 0.9679\color{black} $\,\,$ & $\,\,$ 1 $\,\,$ & $\,\,$\color{gr} 8.4700\color{black} $\,\,$ & $\,\,$\color{gr} \color{blue} 2\color{black}   $\,\,$ \\
$\,\,$0.1143$\,\,$ & $\,\,$\color{gr} 0.1181\color{black} $\,\,$ & $\,\,$ 1 $\,\,$ & $\,\,$0.2361 $\,\,$ \\
$\,\,$0.4840$\,\,$ & $\,\,$\color{gr} \color{blue}  1/2\color{black} $\,\,$ & $\,\,$4.2350$\,\,$ & $\,\,$ 1  $\,\,$ \\
\end{pmatrix},
\end{equation*}
\end{example}
\newpage
\begin{example}
\begin{equation*}
\mathbf{A} =
\begin{pmatrix}
$\,\,$ 1 $\,\,$ & $\,\,$1$\,\,$ & $\,\,$6$\,\,$ & $\,\,$3 $\,\,$ \\
$\,\,$ 1 $\,\,$ & $\,\,$ 1 $\,\,$ & $\,\,$9$\,\,$ & $\,\,$2 $\,\,$ \\
$\,\,$ 1/6$\,\,$ & $\,\,$ 1/9$\,\,$ & $\,\,$ 1 $\,\,$ & $\,\,$ 1/7 $\,\,$ \\
$\,\,$ 1/3$\,\,$ & $\,\,$ 1/2$\,\,$ & $\,\,$7$\,\,$ & $\,\,$ 1  $\,\,$ \\
\end{pmatrix},
\qquad
\lambda_{\max} =
4.1342,
\qquad
CR = 0.0506
\end{equation*}

\begin{equation*}
\mathbf{w}^{EM} =
\begin{pmatrix}
0.388641\\
\color{red} 0.372218\color{black} \\
0.042811\\
0.196330
\end{pmatrix}\end{equation*}
\begin{equation*}
\left[ \frac{{w}^{EM}_i}{{w}^{EM}_j} \right] =
\begin{pmatrix}
$\,\,$ 1 $\,\,$ & $\,\,$\color{red} 1.0441\color{black} $\,\,$ & $\,\,$9.0780$\,\,$ & $\,\,$1.9795$\,\,$ \\
$\,\,$\color{red} 0.9577\color{black} $\,\,$ & $\,\,$ 1 $\,\,$ & $\,\,$\color{red} 8.6944\color{black} $\,\,$ & $\,\,$\color{red} 1.8959\color{black}   $\,\,$ \\
$\,\,$0.1102$\,\,$ & $\,\,$\color{red} 0.1150\color{black} $\,\,$ & $\,\,$ 1 $\,\,$ & $\,\,$0.2181 $\,\,$ \\
$\,\,$0.5052$\,\,$ & $\,\,$\color{red} 0.5275\color{black} $\,\,$ & $\,\,$4.5859$\,\,$ & $\,\,$ 1  $\,\,$ \\
\end{pmatrix},
\end{equation*}

\begin{equation*}
\mathbf{w}^{\prime} =
\begin{pmatrix}
0.383622\\
0.380325\\
0.042258\\
0.193794
\end{pmatrix} =
0.987087\cdot
\begin{pmatrix}
0.388641\\
\color{gr} 0.385300\color{black} \\
0.042811\\
0.196330
\end{pmatrix},
\end{equation*}
\begin{equation*}
\left[ \frac{{w}^{\prime}_i}{{w}^{\prime}_j} \right] =
\begin{pmatrix}
$\,\,$ 1 $\,\,$ & $\,\,$\color{gr} 1.0087\color{black} $\,\,$ & $\,\,$9.0780$\,\,$ & $\,\,$1.9795$\,\,$ \\
$\,\,$\color{gr} 0.9914\color{black} $\,\,$ & $\,\,$ 1 $\,\,$ & $\,\,$\color{gr} \color{blue} 9\color{black} $\,\,$ & $\,\,$\color{gr} 1.9625\color{black}   $\,\,$ \\
$\,\,$0.1102$\,\,$ & $\,\,$\color{gr} \color{blue}  1/9\color{black} $\,\,$ & $\,\,$ 1 $\,\,$ & $\,\,$0.2181 $\,\,$ \\
$\,\,$0.5052$\,\,$ & $\,\,$\color{gr} 0.5095\color{black} $\,\,$ & $\,\,$4.5859$\,\,$ & $\,\,$ 1  $\,\,$ \\
\end{pmatrix},
\end{equation*}
\end{example}
\newpage
\begin{example}
\begin{equation*}
\mathbf{A} =
\begin{pmatrix}
$\,\,$ 1 $\,\,$ & $\,\,$1$\,\,$ & $\,\,$6$\,\,$ & $\,\,$3 $\,\,$ \\
$\,\,$ 1 $\,\,$ & $\,\,$ 1 $\,\,$ & $\,\,$9$\,\,$ & $\,\,$2 $\,\,$ \\
$\,\,$ 1/6$\,\,$ & $\,\,$ 1/9$\,\,$ & $\,\,$ 1 $\,\,$ & $\,\,$ 1/8 $\,\,$ \\
$\,\,$ 1/3$\,\,$ & $\,\,$ 1/2$\,\,$ & $\,\,$8$\,\,$ & $\,\,$ 1  $\,\,$ \\
\end{pmatrix},
\qquad
\lambda_{\max} =
4.1664,
\qquad
CR = 0.0627
\end{equation*}

\begin{equation*}
\mathbf{w}^{EM} =
\begin{pmatrix}
0.387203\\
\color{red} 0.368163\color{black} \\
0.041326\\
0.203308
\end{pmatrix}\end{equation*}
\begin{equation*}
\left[ \frac{{w}^{EM}_i}{{w}^{EM}_j} \right] =
\begin{pmatrix}
$\,\,$ 1 $\,\,$ & $\,\,$\color{red} 1.0517\color{black} $\,\,$ & $\,\,$9.3695$\,\,$ & $\,\,$1.9045$\,\,$ \\
$\,\,$\color{red} 0.9508\color{black} $\,\,$ & $\,\,$ 1 $\,\,$ & $\,\,$\color{red} 8.9088\color{black} $\,\,$ & $\,\,$\color{red} 1.8109\color{black}   $\,\,$ \\
$\,\,$0.1067$\,\,$ & $\,\,$\color{red} 0.1122\color{black} $\,\,$ & $\,\,$ 1 $\,\,$ & $\,\,$0.2033 $\,\,$ \\
$\,\,$0.5251$\,\,$ & $\,\,$\color{red} 0.5522\color{black} $\,\,$ & $\,\,$4.9196$\,\,$ & $\,\,$ 1  $\,\,$ \\
\end{pmatrix},
\end{equation*}

\begin{equation*}
\mathbf{w}^{\prime} =
\begin{pmatrix}
0.385749\\
0.370536\\
0.041171\\
0.202545
\end{pmatrix} =
0.996244\cdot
\begin{pmatrix}
0.387203\\
\color{gr} 0.371933\color{black} \\
0.041326\\
0.203308
\end{pmatrix},
\end{equation*}
\begin{equation*}
\left[ \frac{{w}^{\prime}_i}{{w}^{\prime}_j} \right] =
\begin{pmatrix}
$\,\,$ 1 $\,\,$ & $\,\,$\color{gr} 1.0411\color{black} $\,\,$ & $\,\,$9.3695$\,\,$ & $\,\,$1.9045$\,\,$ \\
$\,\,$\color{gr} 0.9606\color{black} $\,\,$ & $\,\,$ 1 $\,\,$ & $\,\,$\color{gr} \color{blue} 9\color{black} $\,\,$ & $\,\,$\color{gr} 1.8294\color{black}   $\,\,$ \\
$\,\,$0.1067$\,\,$ & $\,\,$\color{gr} \color{blue}  1/9\color{black} $\,\,$ & $\,\,$ 1 $\,\,$ & $\,\,$0.2033 $\,\,$ \\
$\,\,$0.5251$\,\,$ & $\,\,$\color{gr} 0.5466\color{black} $\,\,$ & $\,\,$4.9196$\,\,$ & $\,\,$ 1  $\,\,$ \\
\end{pmatrix},
\end{equation*}
\end{example}
\newpage
\begin{example}
\begin{equation*}
\mathbf{A} =
\begin{pmatrix}
$\,\,$ 1 $\,\,$ & $\,\,$1$\,\,$ & $\,\,$6$\,\,$ & $\,\,$4 $\,\,$ \\
$\,\,$ 1 $\,\,$ & $\,\,$ 1 $\,\,$ & $\,\,$9$\,\,$ & $\,\,$2 $\,\,$ \\
$\,\,$ 1/6$\,\,$ & $\,\,$ 1/9$\,\,$ & $\,\,$ 1 $\,\,$ & $\,\,$ 1/7 $\,\,$ \\
$\,\,$ 1/4$\,\,$ & $\,\,$ 1/2$\,\,$ & $\,\,$7$\,\,$ & $\,\,$ 1  $\,\,$ \\
\end{pmatrix},
\qquad
\lambda_{\max} =
4.2086,
\qquad
CR = 0.0786
\end{equation*}

\begin{equation*}
\mathbf{w}^{EM} =
\begin{pmatrix}
0.416485\\
\color{red} 0.360692\color{black} \\
0.042168\\
0.180655
\end{pmatrix}\end{equation*}
\begin{equation*}
\left[ \frac{{w}^{EM}_i}{{w}^{EM}_j} \right] =
\begin{pmatrix}
$\,\,$ 1 $\,\,$ & $\,\,$\color{red} 1.1547\color{black} $\,\,$ & $\,\,$9.8768$\,\,$ & $\,\,$2.3054$\,\,$ \\
$\,\,$\color{red} 0.8660\color{black} $\,\,$ & $\,\,$ 1 $\,\,$ & $\,\,$\color{red} 8.5537\color{black} $\,\,$ & $\,\,$\color{red} 1.9966\color{black}   $\,\,$ \\
$\,\,$0.1012$\,\,$ & $\,\,$\color{red} 0.1169\color{black} $\,\,$ & $\,\,$ 1 $\,\,$ & $\,\,$0.2334 $\,\,$ \\
$\,\,$0.4338$\,\,$ & $\,\,$\color{red} 0.5009\color{black} $\,\,$ & $\,\,$4.2842$\,\,$ & $\,\,$ 1  $\,\,$ \\
\end{pmatrix},
\end{equation*}

\begin{equation*}
\mathbf{w}^{\prime} =
\begin{pmatrix}
0.416228\\
0.361087\\
0.042142\\
0.180543
\end{pmatrix} =
0.999383\cdot
\begin{pmatrix}
0.416485\\
\color{gr} 0.361310\color{black} \\
0.042168\\
0.180655
\end{pmatrix},
\end{equation*}
\begin{equation*}
\left[ \frac{{w}^{\prime}_i}{{w}^{\prime}_j} \right] =
\begin{pmatrix}
$\,\,$ 1 $\,\,$ & $\,\,$\color{gr} 1.1527\color{black} $\,\,$ & $\,\,$9.8768$\,\,$ & $\,\,$2.3054$\,\,$ \\
$\,\,$\color{gr} 0.8675\color{black} $\,\,$ & $\,\,$ 1 $\,\,$ & $\,\,$\color{gr} 8.5683\color{black} $\,\,$ & $\,\,$\color{gr} \color{blue} 2\color{black}   $\,\,$ \\
$\,\,$0.1012$\,\,$ & $\,\,$\color{gr} 0.1167\color{black} $\,\,$ & $\,\,$ 1 $\,\,$ & $\,\,$0.2334 $\,\,$ \\
$\,\,$0.4338$\,\,$ & $\,\,$\color{gr} \color{blue}  1/2\color{black} $\,\,$ & $\,\,$4.2842$\,\,$ & $\,\,$ 1  $\,\,$ \\
\end{pmatrix},
\end{equation*}
\end{example}
\newpage
\begin{example}
\begin{equation*}
\mathbf{A} =
\begin{pmatrix}
$\,\,$ 1 $\,\,$ & $\,\,$1$\,\,$ & $\,\,$6$\,\,$ & $\,\,$4 $\,\,$ \\
$\,\,$ 1 $\,\,$ & $\,\,$ 1 $\,\,$ & $\,\,$9$\,\,$ & $\,\,$2 $\,\,$ \\
$\,\,$ 1/6$\,\,$ & $\,\,$ 1/9$\,\,$ & $\,\,$ 1 $\,\,$ & $\,\,$ 1/8 $\,\,$ \\
$\,\,$ 1/4$\,\,$ & $\,\,$ 1/2$\,\,$ & $\,\,$8$\,\,$ & $\,\,$ 1  $\,\,$ \\
\end{pmatrix},
\qquad
\lambda_{\max} =
4.2469,
\qquad
CR = 0.0931
\end{equation*}

\begin{equation*}
\mathbf{w}^{EM} =
\begin{pmatrix}
0.415705\\
\color{red} 0.356300\color{black} \\
0.040740\\
0.187254
\end{pmatrix}\end{equation*}
\begin{equation*}
\left[ \frac{{w}^{EM}_i}{{w}^{EM}_j} \right] =
\begin{pmatrix}
$\,\,$ 1 $\,\,$ & $\,\,$\color{red} 1.1667\color{black} $\,\,$ & $\,\,$10.2038$\,\,$ & $\,\,$2.2200$\,\,$ \\
$\,\,$\color{red} 0.8571\color{black} $\,\,$ & $\,\,$ 1 $\,\,$ & $\,\,$\color{red} 8.7457\color{black} $\,\,$ & $\,\,$\color{red} 1.9028\color{black}   $\,\,$ \\
$\,\,$0.0980$\,\,$ & $\,\,$\color{red} 0.1143\color{black} $\,\,$ & $\,\,$ 1 $\,\,$ & $\,\,$0.2176 $\,\,$ \\
$\,\,$0.4504$\,\,$ & $\,\,$\color{red} 0.5256\color{black} $\,\,$ & $\,\,$4.5963$\,\,$ & $\,\,$ 1  $\,\,$ \\
\end{pmatrix},
\end{equation*}

\begin{equation*}
\mathbf{w}^{\prime} =
\begin{pmatrix}
0.411442\\
0.362902\\
0.040322\\
0.185334
\end{pmatrix} =
0.989745\cdot
\begin{pmatrix}
0.415705\\
\color{gr} 0.366662\color{black} \\
0.040740\\
0.187254
\end{pmatrix},
\end{equation*}
\begin{equation*}
\left[ \frac{{w}^{\prime}_i}{{w}^{\prime}_j} \right] =
\begin{pmatrix}
$\,\,$ 1 $\,\,$ & $\,\,$\color{gr} 1.1338\color{black} $\,\,$ & $\,\,$10.2038$\,\,$ & $\,\,$2.2200$\,\,$ \\
$\,\,$\color{gr} 0.8820\color{black} $\,\,$ & $\,\,$ 1 $\,\,$ & $\,\,$\color{gr} \color{blue} 9\color{black} $\,\,$ & $\,\,$\color{gr} 1.9581\color{black}   $\,\,$ \\
$\,\,$0.0980$\,\,$ & $\,\,$\color{gr} \color{blue}  1/9\color{black} $\,\,$ & $\,\,$ 1 $\,\,$ & $\,\,$0.2176 $\,\,$ \\
$\,\,$0.4504$\,\,$ & $\,\,$\color{gr} 0.5107\color{black} $\,\,$ & $\,\,$4.5963$\,\,$ & $\,\,$ 1  $\,\,$ \\
\end{pmatrix},
\end{equation*}
\end{example}
\newpage
\begin{example}
\begin{equation*}
\mathbf{A} =
\begin{pmatrix}
$\,\,$ 1 $\,\,$ & $\,\,$1$\,\,$ & $\,\,$6$\,\,$ & $\,\,$4 $\,\,$ \\
$\,\,$ 1 $\,\,$ & $\,\,$ 1 $\,\,$ & $\,\,$9$\,\,$ & $\,\,$3 $\,\,$ \\
$\,\,$ 1/6$\,\,$ & $\,\,$ 1/9$\,\,$ & $\,\,$ 1 $\,\,$ & $\,\,$ 1/5 $\,\,$ \\
$\,\,$ 1/4$\,\,$ & $\,\,$ 1/3$\,\,$ & $\,\,$5$\,\,$ & $\,\,$ 1  $\,\,$ \\
\end{pmatrix},
\qquad
\lambda_{\max} =
4.1252,
\qquad
CR = 0.0472
\end{equation*}

\begin{equation*}
\mathbf{w}^{EM} =
\begin{pmatrix}
0.404855\\
\color{red} 0.401999\color{black} \\
0.045342\\
0.147804
\end{pmatrix}\end{equation*}
\begin{equation*}
\left[ \frac{{w}^{EM}_i}{{w}^{EM}_j} \right] =
\begin{pmatrix}
$\,\,$ 1 $\,\,$ & $\,\,$\color{red} 1.0071\color{black} $\,\,$ & $\,\,$8.9290$\,\,$ & $\,\,$2.7391$\,\,$ \\
$\,\,$\color{red} 0.9929\color{black} $\,\,$ & $\,\,$ 1 $\,\,$ & $\,\,$\color{red} 8.8660\color{black} $\,\,$ & $\,\,$\color{red} 2.7198\color{black}   $\,\,$ \\
$\,\,$0.1120$\,\,$ & $\,\,$\color{red} 0.1128\color{black} $\,\,$ & $\,\,$ 1 $\,\,$ & $\,\,$0.3068 $\,\,$ \\
$\,\,$0.3651$\,\,$ & $\,\,$\color{red} 0.3677\color{black} $\,\,$ & $\,\,$3.2598$\,\,$ & $\,\,$ 1  $\,\,$ \\
\end{pmatrix},
\end{equation*}

\begin{equation*}
\mathbf{w}^{\prime} =
\begin{pmatrix}
0.403702\\
0.403702\\
0.045213\\
0.147383
\end{pmatrix} =
0.997153\cdot
\begin{pmatrix}
0.404855\\
\color{gr} 0.404855\color{black} \\
0.045342\\
0.147804
\end{pmatrix},
\end{equation*}
\begin{equation*}
\left[ \frac{{w}^{\prime}_i}{{w}^{\prime}_j} \right] =
\begin{pmatrix}
$\,\,$ 1 $\,\,$ & $\,\,$\color{gr} \color{blue} 1\color{black} $\,\,$ & $\,\,$8.9290$\,\,$ & $\,\,$2.7391$\,\,$ \\
$\,\,$\color{gr} \color{blue} 1\color{black} $\,\,$ & $\,\,$ 1 $\,\,$ & $\,\,$\color{gr} 8.9290\color{black} $\,\,$ & $\,\,$\color{gr} 2.7391\color{black}   $\,\,$ \\
$\,\,$0.1120$\,\,$ & $\,\,$\color{gr} 0.1120\color{black} $\,\,$ & $\,\,$ 1 $\,\,$ & $\,\,$0.3068 $\,\,$ \\
$\,\,$0.3651$\,\,$ & $\,\,$\color{gr} 0.3651\color{black} $\,\,$ & $\,\,$3.2598$\,\,$ & $\,\,$ 1  $\,\,$ \\
\end{pmatrix},
\end{equation*}
\end{example}
\newpage
\begin{example}
\begin{equation*}
\mathbf{A} =
\begin{pmatrix}
$\,\,$ 1 $\,\,$ & $\,\,$1$\,\,$ & $\,\,$6$\,\,$ & $\,\,$5 $\,\,$ \\
$\,\,$ 1 $\,\,$ & $\,\,$ 1 $\,\,$ & $\,\,$8$\,\,$ & $\,\,$3 $\,\,$ \\
$\,\,$ 1/6$\,\,$ & $\,\,$ 1/8$\,\,$ & $\,\,$ 1 $\,\,$ & $\,\,$ 1/4 $\,\,$ \\
$\,\,$ 1/5$\,\,$ & $\,\,$ 1/3$\,\,$ & $\,\,$4$\,\,$ & $\,\,$ 1  $\,\,$ \\
\end{pmatrix},
\qquad
\lambda_{\max} =
4.1252,
\qquad
CR = 0.0472
\end{equation*}

\begin{equation*}
\mathbf{w}^{EM} =
\begin{pmatrix}
0.429645\\
\color{red} 0.389518\color{black} \\
0.049036\\
0.131802
\end{pmatrix}\end{equation*}
\begin{equation*}
\left[ \frac{{w}^{EM}_i}{{w}^{EM}_j} \right] =
\begin{pmatrix}
$\,\,$ 1 $\,\,$ & $\,\,$\color{red} 1.1030\color{black} $\,\,$ & $\,\,$8.7619$\,\,$ & $\,\,$3.2598$\,\,$ \\
$\,\,$\color{red} 0.9066\color{black} $\,\,$ & $\,\,$ 1 $\,\,$ & $\,\,$\color{red} 7.9436\color{black} $\,\,$ & $\,\,$\color{red} 2.9553\color{black}   $\,\,$ \\
$\,\,$0.1141$\,\,$ & $\,\,$\color{red} 0.1259\color{black} $\,\,$ & $\,\,$ 1 $\,\,$ & $\,\,$0.3720 $\,\,$ \\
$\,\,$0.3068$\,\,$ & $\,\,$\color{red} 0.3384\color{black} $\,\,$ & $\,\,$2.6879$\,\,$ & $\,\,$ 1  $\,\,$ \\
\end{pmatrix},
\end{equation*}

\begin{equation*}
\mathbf{w}^{\prime} =
\begin{pmatrix}
0.428459\\
0.391202\\
0.048900\\
0.131438
\end{pmatrix} =
0.997241\cdot
\begin{pmatrix}
0.429645\\
\color{gr} 0.392285\color{black} \\
0.049036\\
0.131802
\end{pmatrix},
\end{equation*}
\begin{equation*}
\left[ \frac{{w}^{\prime}_i}{{w}^{\prime}_j} \right] =
\begin{pmatrix}
$\,\,$ 1 $\,\,$ & $\,\,$\color{gr} 1.0952\color{black} $\,\,$ & $\,\,$8.7619$\,\,$ & $\,\,$3.2598$\,\,$ \\
$\,\,$\color{gr} 0.9130\color{black} $\,\,$ & $\,\,$ 1 $\,\,$ & $\,\,$\color{gr} \color{blue} 8\color{black} $\,\,$ & $\,\,$\color{gr} 2.9763\color{black}   $\,\,$ \\
$\,\,$0.1141$\,\,$ & $\,\,$\color{gr} \color{blue}  1/8\color{black} $\,\,$ & $\,\,$ 1 $\,\,$ & $\,\,$0.3720 $\,\,$ \\
$\,\,$0.3068$\,\,$ & $\,\,$\color{gr} 0.3360\color{black} $\,\,$ & $\,\,$2.6879$\,\,$ & $\,\,$ 1  $\,\,$ \\
\end{pmatrix},
\end{equation*}
\end{example}
\newpage
\begin{example}
\begin{equation*}
\mathbf{A} =
\begin{pmatrix}
$\,\,$ 1 $\,\,$ & $\,\,$1$\,\,$ & $\,\,$6$\,\,$ & $\,\,$5 $\,\,$ \\
$\,\,$ 1 $\,\,$ & $\,\,$ 1 $\,\,$ & $\,\,$9$\,\,$ & $\,\,$3 $\,\,$ \\
$\,\,$ 1/6$\,\,$ & $\,\,$ 1/9$\,\,$ & $\,\,$ 1 $\,\,$ & $\,\,$ 1/5 $\,\,$ \\
$\,\,$ 1/5$\,\,$ & $\,\,$ 1/3$\,\,$ & $\,\,$5$\,\,$ & $\,\,$ 1  $\,\,$ \\
\end{pmatrix},
\qquad
\lambda_{\max} =
4.1758,
\qquad
CR = 0.0663
\end{equation*}

\begin{equation*}
\mathbf{w}^{EM} =
\begin{pmatrix}
0.425528\\
\color{red} 0.391422\color{black} \\
0.044736\\
0.138313
\end{pmatrix}\end{equation*}
\begin{equation*}
\left[ \frac{{w}^{EM}_i}{{w}^{EM}_j} \right] =
\begin{pmatrix}
$\,\,$ 1 $\,\,$ & $\,\,$\color{red} 1.0871\color{black} $\,\,$ & $\,\,$9.5119$\,\,$ & $\,\,$3.0765$\,\,$ \\
$\,\,$\color{red} 0.9199\color{black} $\,\,$ & $\,\,$ 1 $\,\,$ & $\,\,$\color{red} 8.7495\color{black} $\,\,$ & $\,\,$\color{red} 2.8300\color{black}   $\,\,$ \\
$\,\,$0.1051$\,\,$ & $\,\,$\color{red} 0.1143\color{black} $\,\,$ & $\,\,$ 1 $\,\,$ & $\,\,$0.3234 $\,\,$ \\
$\,\,$0.3250$\,\,$ & $\,\,$\color{red} 0.3534\color{black} $\,\,$ & $\,\,$3.0917$\,\,$ & $\,\,$ 1  $\,\,$ \\
\end{pmatrix},
\end{equation*}

\begin{equation*}
\mathbf{w}^{\prime} =
\begin{pmatrix}
0.420813\\
0.398166\\
0.044241\\
0.136781
\end{pmatrix} =
0.988920\cdot
\begin{pmatrix}
0.425528\\
\color{gr} 0.402627\color{black} \\
0.044736\\
0.138313
\end{pmatrix},
\end{equation*}
\begin{equation*}
\left[ \frac{{w}^{\prime}_i}{{w}^{\prime}_j} \right] =
\begin{pmatrix}
$\,\,$ 1 $\,\,$ & $\,\,$\color{gr} 1.0569\color{black} $\,\,$ & $\,\,$9.5119$\,\,$ & $\,\,$3.0765$\,\,$ \\
$\,\,$\color{gr} 0.9462\color{black} $\,\,$ & $\,\,$ 1 $\,\,$ & $\,\,$\color{gr} \color{blue} 9\color{black} $\,\,$ & $\,\,$\color{gr} 2.9110\color{black}   $\,\,$ \\
$\,\,$0.1051$\,\,$ & $\,\,$\color{gr} \color{blue}  1/9\color{black} $\,\,$ & $\,\,$ 1 $\,\,$ & $\,\,$0.3234 $\,\,$ \\
$\,\,$0.3250$\,\,$ & $\,\,$\color{gr} 0.3435\color{black} $\,\,$ & $\,\,$3.0917$\,\,$ & $\,\,$ 1  $\,\,$ \\
\end{pmatrix},
\end{equation*}
\end{example}
\newpage
\begin{example}
\begin{equation*}
\mathbf{A} =
\begin{pmatrix}
$\,\,$ 1 $\,\,$ & $\,\,$1$\,\,$ & $\,\,$6$\,\,$ & $\,\,$6 $\,\,$ \\
$\,\,$ 1 $\,\,$ & $\,\,$ 1 $\,\,$ & $\,\,$8$\,\,$ & $\,\,$4 $\,\,$ \\
$\,\,$ 1/6$\,\,$ & $\,\,$ 1/8$\,\,$ & $\,\,$ 1 $\,\,$ & $\,\,$ 1/3 $\,\,$ \\
$\,\,$ 1/6$\,\,$ & $\,\,$ 1/4$\,\,$ & $\,\,$3$\,\,$ & $\,\,$ 1  $\,\,$ \\
\end{pmatrix},
\qquad
\lambda_{\max} =
4.1031,
\qquad
CR = 0.0389
\end{equation*}

\begin{equation*}
\mathbf{w}^{EM} =
\begin{pmatrix}
0.434864\\
\color{red} 0.408275\color{black} \\
0.051157\\
0.105705
\end{pmatrix}\end{equation*}
\begin{equation*}
\left[ \frac{{w}^{EM}_i}{{w}^{EM}_j} \right] =
\begin{pmatrix}
$\,\,$ 1 $\,\,$ & $\,\,$\color{red} 1.0651\color{black} $\,\,$ & $\,\,$8.5006$\,\,$ & $\,\,$4.1140$\,\,$ \\
$\,\,$\color{red} 0.9389\color{black} $\,\,$ & $\,\,$ 1 $\,\,$ & $\,\,$\color{red} 7.9809\color{black} $\,\,$ & $\,\,$\color{red} 3.8624\color{black}   $\,\,$ \\
$\,\,$0.1176$\,\,$ & $\,\,$\color{red} 0.1253\color{black} $\,\,$ & $\,\,$ 1 $\,\,$ & $\,\,$0.4840 $\,\,$ \\
$\,\,$0.2431$\,\,$ & $\,\,$\color{red} 0.2589\color{black} $\,\,$ & $\,\,$2.0663$\,\,$ & $\,\,$ 1  $\,\,$ \\
\end{pmatrix},
\end{equation*}

\begin{equation*}
\mathbf{w}^{\prime} =
\begin{pmatrix}
0.434438\\
0.408854\\
0.051107\\
0.105601
\end{pmatrix} =
0.999022\cdot
\begin{pmatrix}
0.434864\\
\color{gr} 0.409254\color{black} \\
0.051157\\
0.105705
\end{pmatrix},
\end{equation*}
\begin{equation*}
\left[ \frac{{w}^{\prime}_i}{{w}^{\prime}_j} \right] =
\begin{pmatrix}
$\,\,$ 1 $\,\,$ & $\,\,$\color{gr} 1.0626\color{black} $\,\,$ & $\,\,$8.5006$\,\,$ & $\,\,$4.1140$\,\,$ \\
$\,\,$\color{gr} 0.9411\color{black} $\,\,$ & $\,\,$ 1 $\,\,$ & $\,\,$\color{gr} \color{blue} 8\color{black} $\,\,$ & $\,\,$\color{gr} 3.8717\color{black}   $\,\,$ \\
$\,\,$0.1176$\,\,$ & $\,\,$\color{gr} \color{blue}  1/8\color{black} $\,\,$ & $\,\,$ 1 $\,\,$ & $\,\,$0.4840 $\,\,$ \\
$\,\,$0.2431$\,\,$ & $\,\,$\color{gr} 0.2583\color{black} $\,\,$ & $\,\,$2.0663$\,\,$ & $\,\,$ 1  $\,\,$ \\
\end{pmatrix},
\end{equation*}
\end{example}
\newpage
\begin{example}
\begin{equation*}
\mathbf{A} =
\begin{pmatrix}
$\,\,$ 1 $\,\,$ & $\,\,$1$\,\,$ & $\,\,$6$\,\,$ & $\,\,$6 $\,\,$ \\
$\,\,$ 1 $\,\,$ & $\,\,$ 1 $\,\,$ & $\,\,$9$\,\,$ & $\,\,$3 $\,\,$ \\
$\,\,$ 1/6$\,\,$ & $\,\,$ 1/9$\,\,$ & $\,\,$ 1 $\,\,$ & $\,\,$ 1/5 $\,\,$ \\
$\,\,$ 1/6$\,\,$ & $\,\,$ 1/3$\,\,$ & $\,\,$5$\,\,$ & $\,\,$ 1  $\,\,$ \\
\end{pmatrix},
\qquad
\lambda_{\max} =
4.2277,
\qquad
CR = 0.0859
\end{equation*}

\begin{equation*}
\mathbf{w}^{EM} =
\begin{pmatrix}
0.443298\\
\color{red} 0.381874\color{black} \\
0.044134\\
0.130694
\end{pmatrix}\end{equation*}
\begin{equation*}
\left[ \frac{{w}^{EM}_i}{{w}^{EM}_j} \right] =
\begin{pmatrix}
$\,\,$ 1 $\,\,$ & $\,\,$\color{red} 1.1608\color{black} $\,\,$ & $\,\,$10.0444$\,\,$ & $\,\,$3.3919$\,\,$ \\
$\,\,$\color{red} 0.8614\color{black} $\,\,$ & $\,\,$ 1 $\,\,$ & $\,\,$\color{red} 8.6526\color{black} $\,\,$ & $\,\,$\color{red} 2.9219\color{black}   $\,\,$ \\
$\,\,$0.0996$\,\,$ & $\,\,$\color{red} 0.1156\color{black} $\,\,$ & $\,\,$ 1 $\,\,$ & $\,\,$0.3377 $\,\,$ \\
$\,\,$0.2948$\,\,$ & $\,\,$\color{red} 0.3422\color{black} $\,\,$ & $\,\,$2.9613$\,\,$ & $\,\,$ 1  $\,\,$ \\
\end{pmatrix},
\end{equation*}

\begin{equation*}
\mathbf{w}^{\prime} =
\begin{pmatrix}
0.438818\\
0.388120\\
0.043688\\
0.129373
\end{pmatrix} =
0.989896\cdot
\begin{pmatrix}
0.443298\\
\color{gr} 0.392082\color{black} \\
0.044134\\
0.130694
\end{pmatrix},
\end{equation*}
\begin{equation*}
\left[ \frac{{w}^{\prime}_i}{{w}^{\prime}_j} \right] =
\begin{pmatrix}
$\,\,$ 1 $\,\,$ & $\,\,$\color{gr} 1.1306\color{black} $\,\,$ & $\,\,$10.0444$\,\,$ & $\,\,$3.3919$\,\,$ \\
$\,\,$\color{gr} 0.8845\color{black} $\,\,$ & $\,\,$ 1 $\,\,$ & $\,\,$\color{gr} 8.8839\color{black} $\,\,$ & $\,\,$\color{gr} \color{blue} 3\color{black}   $\,\,$ \\
$\,\,$0.0996$\,\,$ & $\,\,$\color{gr} 0.1126\color{black} $\,\,$ & $\,\,$ 1 $\,\,$ & $\,\,$0.3377 $\,\,$ \\
$\,\,$0.2948$\,\,$ & $\,\,$\color{gr} \color{blue}  1/3\color{black} $\,\,$ & $\,\,$2.9613$\,\,$ & $\,\,$ 1  $\,\,$ \\
\end{pmatrix},
\end{equation*}
\end{example}
\newpage
\begin{example}
\begin{equation*}
\mathbf{A} =
\begin{pmatrix}
$\,\,$ 1 $\,\,$ & $\,\,$1$\,\,$ & $\,\,$6$\,\,$ & $\,\,$6 $\,\,$ \\
$\,\,$ 1 $\,\,$ & $\,\,$ 1 $\,\,$ & $\,\,$9$\,\,$ & $\,\,$4 $\,\,$ \\
$\,\,$ 1/6$\,\,$ & $\,\,$ 1/9$\,\,$ & $\,\,$ 1 $\,\,$ & $\,\,$ 1/3 $\,\,$ \\
$\,\,$ 1/6$\,\,$ & $\,\,$ 1/4$\,\,$ & $\,\,$3$\,\,$ & $\,\,$ 1  $\,\,$ \\
\end{pmatrix},
\qquad
\lambda_{\max} =
4.1031,
\qquad
CR = 0.0389
\end{equation*}

\begin{equation*}
\mathbf{w}^{EM} =
\begin{pmatrix}
0.430689\\
\color{red} 0.415875\color{black} \\
0.049218\\
0.104218
\end{pmatrix}\end{equation*}
\begin{equation*}
\left[ \frac{{w}^{EM}_i}{{w}^{EM}_j} \right] =
\begin{pmatrix}
$\,\,$ 1 $\,\,$ & $\,\,$\color{red} 1.0356\color{black} $\,\,$ & $\,\,$8.7507$\,\,$ & $\,\,$4.1326$\,\,$ \\
$\,\,$\color{red} 0.9656\color{black} $\,\,$ & $\,\,$ 1 $\,\,$ & $\,\,$\color{red} 8.4497\color{black} $\,\,$ & $\,\,$\color{red} 3.9904\color{black}   $\,\,$ \\
$\,\,$0.1143$\,\,$ & $\,\,$\color{red} 0.1183\color{black} $\,\,$ & $\,\,$ 1 $\,\,$ & $\,\,$0.4723 $\,\,$ \\
$\,\,$0.2420$\,\,$ & $\,\,$\color{red} 0.2506\color{black} $\,\,$ & $\,\,$2.1175$\,\,$ & $\,\,$ 1  $\,\,$ \\
\end{pmatrix},
\end{equation*}

\begin{equation*}
\mathbf{w}^{\prime} =
\begin{pmatrix}
0.430260\\
0.416457\\
0.049169\\
0.104114
\end{pmatrix} =
0.999004\cdot
\begin{pmatrix}
0.430689\\
\color{gr} 0.416872\color{black} \\
0.049218\\
0.104218
\end{pmatrix},
\end{equation*}
\begin{equation*}
\left[ \frac{{w}^{\prime}_i}{{w}^{\prime}_j} \right] =
\begin{pmatrix}
$\,\,$ 1 $\,\,$ & $\,\,$\color{gr} 1.0331\color{black} $\,\,$ & $\,\,$8.7507$\,\,$ & $\,\,$4.1326$\,\,$ \\
$\,\,$\color{gr} 0.9679\color{black} $\,\,$ & $\,\,$ 1 $\,\,$ & $\,\,$\color{gr} 8.4700\color{black} $\,\,$ & $\,\,$\color{gr} \color{blue} 4\color{black}   $\,\,$ \\
$\,\,$0.1143$\,\,$ & $\,\,$\color{gr} 0.1181\color{black} $\,\,$ & $\,\,$ 1 $\,\,$ & $\,\,$0.4723 $\,\,$ \\
$\,\,$0.2420$\,\,$ & $\,\,$\color{gr} \color{blue}  1/4\color{black} $\,\,$ & $\,\,$2.1175$\,\,$ & $\,\,$ 1  $\,\,$ \\
\end{pmatrix},
\end{equation*}
\end{example}
\newpage
\begin{example}
\begin{equation*}
\mathbf{A} =
\begin{pmatrix}
$\,\,$ 1 $\,\,$ & $\,\,$1$\,\,$ & $\,\,$6$\,\,$ & $\,\,$6 $\,\,$ \\
$\,\,$ 1 $\,\,$ & $\,\,$ 1 $\,\,$ & $\,\,$9$\,\,$ & $\,\,$4 $\,\,$ \\
$\,\,$ 1/6$\,\,$ & $\,\,$ 1/9$\,\,$ & $\,\,$ 1 $\,\,$ & $\,\,$ 1/4 $\,\,$ \\
$\,\,$ 1/6$\,\,$ & $\,\,$ 1/4$\,\,$ & $\,\,$4$\,\,$ & $\,\,$ 1  $\,\,$ \\
\end{pmatrix},
\qquad
\lambda_{\max} =
4.1664,
\qquad
CR = 0.0627
\end{equation*}

\begin{equation*}
\mathbf{w}^{EM} =
\begin{pmatrix}
0.431018\\
\color{red} 0.409823\color{black} \\
0.046002\\
0.113157
\end{pmatrix}\end{equation*}
\begin{equation*}
\left[ \frac{{w}^{EM}_i}{{w}^{EM}_j} \right] =
\begin{pmatrix}
$\,\,$ 1 $\,\,$ & $\,\,$\color{red} 1.0517\color{black} $\,\,$ & $\,\,$9.3695$\,\,$ & $\,\,$3.8090$\,\,$ \\
$\,\,$\color{red} 0.9508\color{black} $\,\,$ & $\,\,$ 1 $\,\,$ & $\,\,$\color{red} 8.9088\color{black} $\,\,$ & $\,\,$\color{red} 3.6217\color{black}   $\,\,$ \\
$\,\,$0.1067$\,\,$ & $\,\,$\color{red} 0.1122\color{black} $\,\,$ & $\,\,$ 1 $\,\,$ & $\,\,$0.4065 $\,\,$ \\
$\,\,$0.2625$\,\,$ & $\,\,$\color{red} 0.2761\color{black} $\,\,$ & $\,\,$2.4598$\,\,$ & $\,\,$ 1  $\,\,$ \\
\end{pmatrix},
\end{equation*}

\begin{equation*}
\mathbf{w}^{\prime} =
\begin{pmatrix}
0.429217\\
0.412289\\
0.045810\\
0.112684
\end{pmatrix} =
0.995821\cdot
\begin{pmatrix}
0.431018\\
\color{gr} 0.414019\color{black} \\
0.046002\\
0.113157
\end{pmatrix},
\end{equation*}
\begin{equation*}
\left[ \frac{{w}^{\prime}_i}{{w}^{\prime}_j} \right] =
\begin{pmatrix}
$\,\,$ 1 $\,\,$ & $\,\,$\color{gr} 1.0411\color{black} $\,\,$ & $\,\,$9.3695$\,\,$ & $\,\,$3.8090$\,\,$ \\
$\,\,$\color{gr} 0.9606\color{black} $\,\,$ & $\,\,$ 1 $\,\,$ & $\,\,$\color{gr} \color{blue} 9\color{black} $\,\,$ & $\,\,$\color{gr} 3.6588\color{black}   $\,\,$ \\
$\,\,$0.1067$\,\,$ & $\,\,$\color{gr} \color{blue}  1/9\color{black} $\,\,$ & $\,\,$ 1 $\,\,$ & $\,\,$0.4065 $\,\,$ \\
$\,\,$0.2625$\,\,$ & $\,\,$\color{gr} 0.2733\color{black} $\,\,$ & $\,\,$2.4598$\,\,$ & $\,\,$ 1  $\,\,$ \\
\end{pmatrix},
\end{equation*}
\end{example}
\newpage
\begin{example}
\begin{equation*}
\mathbf{A} =
\begin{pmatrix}
$\,\,$ 1 $\,\,$ & $\,\,$1$\,\,$ & $\,\,$6$\,\,$ & $\,\,$7 $\,\,$ \\
$\,\,$ 1 $\,\,$ & $\,\,$ 1 $\,\,$ & $\,\,$8$\,\,$ & $\,\,$4 $\,\,$ \\
$\,\,$ 1/6$\,\,$ & $\,\,$ 1/8$\,\,$ & $\,\,$ 1 $\,\,$ & $\,\,$ 1/3 $\,\,$ \\
$\,\,$ 1/7$\,\,$ & $\,\,$ 1/4$\,\,$ & $\,\,$3$\,\,$ & $\,\,$ 1  $\,\,$ \\
\end{pmatrix},
\qquad
\lambda_{\max} =
4.1365,
\qquad
CR = 0.0515
\end{equation*}

\begin{equation*}
\mathbf{w}^{EM} =
\begin{pmatrix}
0.448729\\
\color{red} 0.400162\color{black} \\
0.050486\\
0.100623
\end{pmatrix}\end{equation*}
\begin{equation*}
\left[ \frac{{w}^{EM}_i}{{w}^{EM}_j} \right] =
\begin{pmatrix}
$\,\,$ 1 $\,\,$ & $\,\,$\color{red} 1.1214\color{black} $\,\,$ & $\,\,$8.8882$\,\,$ & $\,\,$4.4595$\,\,$ \\
$\,\,$\color{red} 0.8918\color{black} $\,\,$ & $\,\,$ 1 $\,\,$ & $\,\,$\color{red} 7.9262\color{black} $\,\,$ & $\,\,$\color{red} 3.9769\color{black}   $\,\,$ \\
$\,\,$0.1125$\,\,$ & $\,\,$\color{red} 0.1262\color{black} $\,\,$ & $\,\,$ 1 $\,\,$ & $\,\,$0.5017 $\,\,$ \\
$\,\,$0.2242$\,\,$ & $\,\,$\color{red} 0.2515\color{black} $\,\,$ & $\,\,$1.9931$\,\,$ & $\,\,$ 1  $\,\,$ \\
\end{pmatrix},
\end{equation*}

\begin{equation*}
\mathbf{w}^{\prime} =
\begin{pmatrix}
0.447687\\
0.401556\\
0.050369\\
0.100389
\end{pmatrix} =
0.997677\cdot
\begin{pmatrix}
0.448729\\
\color{gr} 0.402491\color{black} \\
0.050486\\
0.100623
\end{pmatrix},
\end{equation*}
\begin{equation*}
\left[ \frac{{w}^{\prime}_i}{{w}^{\prime}_j} \right] =
\begin{pmatrix}
$\,\,$ 1 $\,\,$ & $\,\,$\color{gr} 1.1149\color{black} $\,\,$ & $\,\,$8.8882$\,\,$ & $\,\,$4.4595$\,\,$ \\
$\,\,$\color{gr} 0.8970\color{black} $\,\,$ & $\,\,$ 1 $\,\,$ & $\,\,$\color{gr} 7.9723\color{black} $\,\,$ & $\,\,$\color{gr} \color{blue} 4\color{black}   $\,\,$ \\
$\,\,$0.1125$\,\,$ & $\,\,$\color{gr} 0.1254\color{black} $\,\,$ & $\,\,$ 1 $\,\,$ & $\,\,$0.5017 $\,\,$ \\
$\,\,$0.2242$\,\,$ & $\,\,$\color{gr} \color{blue}  1/4\color{black} $\,\,$ & $\,\,$1.9931$\,\,$ & $\,\,$ 1  $\,\,$ \\
\end{pmatrix},
\end{equation*}
\end{example}
\newpage
\begin{example}
\begin{equation*}
\mathbf{A} =
\begin{pmatrix}
$\,\,$ 1 $\,\,$ & $\,\,$1$\,\,$ & $\,\,$6$\,\,$ & $\,\,$7 $\,\,$ \\
$\,\,$ 1 $\,\,$ & $\,\,$ 1 $\,\,$ & $\,\,$9$\,\,$ & $\,\,$4 $\,\,$ \\
$\,\,$ 1/6$\,\,$ & $\,\,$ 1/9$\,\,$ & $\,\,$ 1 $\,\,$ & $\,\,$ 1/4 $\,\,$ \\
$\,\,$ 1/7$\,\,$ & $\,\,$ 1/4$\,\,$ & $\,\,$4$\,\,$ & $\,\,$ 1  $\,\,$ \\
\end{pmatrix},
\qquad
\lambda_{\max} =
4.2065,
\qquad
CR = 0.0779
\end{equation*}

\begin{equation*}
\mathbf{w}^{EM} =
\begin{pmatrix}
0.445588\\
\color{red} 0.401104\color{black} \\
0.045467\\
0.107842
\end{pmatrix}\end{equation*}
\begin{equation*}
\left[ \frac{{w}^{EM}_i}{{w}^{EM}_j} \right] =
\begin{pmatrix}
$\,\,$ 1 $\,\,$ & $\,\,$\color{red} 1.1109\color{black} $\,\,$ & $\,\,$9.8002$\,\,$ & $\,\,$4.1319$\,\,$ \\
$\,\,$\color{red} 0.9002\color{black} $\,\,$ & $\,\,$ 1 $\,\,$ & $\,\,$\color{red} 8.8219\color{black} $\,\,$ & $\,\,$\color{red} 3.7194\color{black}   $\,\,$ \\
$\,\,$0.1020$\,\,$ & $\,\,$\color{red} 0.1134\color{black} $\,\,$ & $\,\,$ 1 $\,\,$ & $\,\,$0.4216 $\,\,$ \\
$\,\,$0.2420$\,\,$ & $\,\,$\color{red} 0.2689\color{black} $\,\,$ & $\,\,$2.3719$\,\,$ & $\,\,$ 1  $\,\,$ \\
\end{pmatrix},
\end{equation*}

\begin{equation*}
\mathbf{w}^{\prime} =
\begin{pmatrix}
0.442008\\
0.405915\\
0.045102\\
0.106975
\end{pmatrix} =
0.991966\cdot
\begin{pmatrix}
0.445588\\
\color{gr} 0.409203\color{black} \\
0.045467\\
0.107842
\end{pmatrix},
\end{equation*}
\begin{equation*}
\left[ \frac{{w}^{\prime}_i}{{w}^{\prime}_j} \right] =
\begin{pmatrix}
$\,\,$ 1 $\,\,$ & $\,\,$\color{gr} 1.0889\color{black} $\,\,$ & $\,\,$9.8002$\,\,$ & $\,\,$4.1319$\,\,$ \\
$\,\,$\color{gr} 0.9183\color{black} $\,\,$ & $\,\,$ 1 $\,\,$ & $\,\,$\color{gr} \color{blue} 9\color{black} $\,\,$ & $\,\,$\color{gr} 3.7945\color{black}   $\,\,$ \\
$\,\,$0.1020$\,\,$ & $\,\,$\color{gr} \color{blue}  1/9\color{black} $\,\,$ & $\,\,$ 1 $\,\,$ & $\,\,$0.4216 $\,\,$ \\
$\,\,$0.2420$\,\,$ & $\,\,$\color{gr} 0.2635\color{black} $\,\,$ & $\,\,$2.3719$\,\,$ & $\,\,$ 1  $\,\,$ \\
\end{pmatrix},
\end{equation*}
\end{example}
\newpage
\begin{example}
\begin{equation*}
\mathbf{A} =
\begin{pmatrix}
$\,\,$ 1 $\,\,$ & $\,\,$1$\,\,$ & $\,\,$6$\,\,$ & $\,\,$7 $\,\,$ \\
$\,\,$ 1 $\,\,$ & $\,\,$ 1 $\,\,$ & $\,\,$9$\,\,$ & $\,\,$5 $\,\,$ \\
$\,\,$ 1/6$\,\,$ & $\,\,$ 1/9$\,\,$ & $\,\,$ 1 $\,\,$ & $\,\,$ 1/3 $\,\,$ \\
$\,\,$ 1/7$\,\,$ & $\,\,$ 1/5$\,\,$ & $\,\,$3$\,\,$ & $\,\,$ 1  $\,\,$ \\
\end{pmatrix},
\qquad
\lambda_{\max} =
4.1351,
\qquad
CR = 0.0509
\end{equation*}

\begin{equation*}
\mathbf{w}^{EM} =
\begin{pmatrix}
0.434602\\
\color{red} 0.424539\color{black} \\
0.048021\\
0.092838
\end{pmatrix}\end{equation*}
\begin{equation*}
\left[ \frac{{w}^{EM}_i}{{w}^{EM}_j} \right] =
\begin{pmatrix}
$\,\,$ 1 $\,\,$ & $\,\,$\color{red} 1.0237\color{black} $\,\,$ & $\,\,$9.0503$\,\,$ & $\,\,$4.6813$\,\,$ \\
$\,\,$\color{red} 0.9768\color{black} $\,\,$ & $\,\,$ 1 $\,\,$ & $\,\,$\color{red} 8.8407\color{black} $\,\,$ & $\,\,$\color{red} 4.5729\color{black}   $\,\,$ \\
$\,\,$0.1105$\,\,$ & $\,\,$\color{red} 0.1131\color{black} $\,\,$ & $\,\,$ 1 $\,\,$ & $\,\,$0.5173 $\,\,$ \\
$\,\,$0.2136$\,\,$ & $\,\,$\color{red} 0.2187\color{black} $\,\,$ & $\,\,$1.9333$\,\,$ & $\,\,$ 1  $\,\,$ \\
\end{pmatrix},
\end{equation*}

\begin{equation*}
\mathbf{w}^{\prime} =
\begin{pmatrix}
0.431304\\
0.428907\\
0.047656\\
0.092133
\end{pmatrix} =
0.992410\cdot
\begin{pmatrix}
0.434602\\
\color{gr} 0.432187\color{black} \\
0.048021\\
0.092838
\end{pmatrix},
\end{equation*}
\begin{equation*}
\left[ \frac{{w}^{\prime}_i}{{w}^{\prime}_j} \right] =
\begin{pmatrix}
$\,\,$ 1 $\,\,$ & $\,\,$\color{gr} 1.0056\color{black} $\,\,$ & $\,\,$9.0503$\,\,$ & $\,\,$4.6813$\,\,$ \\
$\,\,$\color{gr} 0.9944\color{black} $\,\,$ & $\,\,$ 1 $\,\,$ & $\,\,$\color{gr} \color{blue} 9\color{black} $\,\,$ & $\,\,$\color{gr} 4.6553\color{black}   $\,\,$ \\
$\,\,$0.1105$\,\,$ & $\,\,$\color{gr} \color{blue}  1/9\color{black} $\,\,$ & $\,\,$ 1 $\,\,$ & $\,\,$0.5173 $\,\,$ \\
$\,\,$0.2136$\,\,$ & $\,\,$\color{gr} 0.2148\color{black} $\,\,$ & $\,\,$1.9333$\,\,$ & $\,\,$ 1  $\,\,$ \\
\end{pmatrix},
\end{equation*}
\end{example}
\newpage
\begin{example}
\begin{equation*}
\mathbf{A} =
\begin{pmatrix}
$\,\,$ 1 $\,\,$ & $\,\,$1$\,\,$ & $\,\,$6$\,\,$ & $\,\,$8 $\,\,$ \\
$\,\,$ 1 $\,\,$ & $\,\,$ 1 $\,\,$ & $\,\,$9$\,\,$ & $\,\,$4 $\,\,$ \\
$\,\,$ 1/6$\,\,$ & $\,\,$ 1/9$\,\,$ & $\,\,$ 1 $\,\,$ & $\,\,$ 1/4 $\,\,$ \\
$\,\,$ 1/8$\,\,$ & $\,\,$ 1/4$\,\,$ & $\,\,$4$\,\,$ & $\,\,$ 1  $\,\,$ \\
\end{pmatrix},
\qquad
\lambda_{\max} =
4.2469,
\qquad
CR = 0.0931
\end{equation*}

\begin{equation*}
\mathbf{w}^{EM} =
\begin{pmatrix}
0.458647\\
\color{red} 0.393106\color{black} \\
0.044949\\
0.103299
\end{pmatrix}\end{equation*}
\begin{equation*}
\left[ \frac{{w}^{EM}_i}{{w}^{EM}_j} \right] =
\begin{pmatrix}
$\,\,$ 1 $\,\,$ & $\,\,$\color{red} 1.1667\color{black} $\,\,$ & $\,\,$10.2038$\,\,$ & $\,\,$4.4400$\,\,$ \\
$\,\,$\color{red} 0.8571\color{black} $\,\,$ & $\,\,$ 1 $\,\,$ & $\,\,$\color{red} 8.7457\color{black} $\,\,$ & $\,\,$\color{red} 3.8055\color{black}   $\,\,$ \\
$\,\,$0.0980$\,\,$ & $\,\,$\color{red} 0.1143\color{black} $\,\,$ & $\,\,$ 1 $\,\,$ & $\,\,$0.4351 $\,\,$ \\
$\,\,$0.2252$\,\,$ & $\,\,$\color{red} 0.2628\color{black} $\,\,$ & $\,\,$2.2982$\,\,$ & $\,\,$ 1  $\,\,$ \\
\end{pmatrix},
\end{equation*}

\begin{equation*}
\mathbf{w}^{\prime} =
\begin{pmatrix}
0.453463\\
0.399965\\
0.044441\\
0.102131
\end{pmatrix} =
0.988697\cdot
\begin{pmatrix}
0.458647\\
\color{gr} 0.404538\color{black} \\
0.044949\\
0.103299
\end{pmatrix},
\end{equation*}
\begin{equation*}
\left[ \frac{{w}^{\prime}_i}{{w}^{\prime}_j} \right] =
\begin{pmatrix}
$\,\,$ 1 $\,\,$ & $\,\,$\color{gr} 1.1338\color{black} $\,\,$ & $\,\,$10.2038$\,\,$ & $\,\,$4.4400$\,\,$ \\
$\,\,$\color{gr} 0.8820\color{black} $\,\,$ & $\,\,$ 1 $\,\,$ & $\,\,$\color{gr} \color{blue} 9\color{black} $\,\,$ & $\,\,$\color{gr} 3.9162\color{black}   $\,\,$ \\
$\,\,$0.0980$\,\,$ & $\,\,$\color{gr} \color{blue}  1/9\color{black} $\,\,$ & $\,\,$ 1 $\,\,$ & $\,\,$0.4351 $\,\,$ \\
$\,\,$0.2252$\,\,$ & $\,\,$\color{gr} 0.2554\color{black} $\,\,$ & $\,\,$2.2982$\,\,$ & $\,\,$ 1  $\,\,$ \\
\end{pmatrix},
\end{equation*}
\end{example}
\newpage
\begin{example}
\begin{equation*}
\mathbf{A} =
\begin{pmatrix}
$\,\,$ 1 $\,\,$ & $\,\,$1$\,\,$ & $\,\,$6$\,\,$ & $\,\,$8 $\,\,$ \\
$\,\,$ 1 $\,\,$ & $\,\,$ 1 $\,\,$ & $\,\,$9$\,\,$ & $\,\,$5 $\,\,$ \\
$\,\,$ 1/6$\,\,$ & $\,\,$ 1/9$\,\,$ & $\,\,$ 1 $\,\,$ & $\,\,$ 1/3 $\,\,$ \\
$\,\,$ 1/8$\,\,$ & $\,\,$ 1/5$\,\,$ & $\,\,$3$\,\,$ & $\,\,$ 1  $\,\,$ \\
\end{pmatrix},
\qquad
\lambda_{\max} =
4.1655,
\qquad
CR = 0.0624
\end{equation*}

\begin{equation*}
\mathbf{w}^{EM} =
\begin{pmatrix}
0.446676\\
\color{red} 0.416798\color{black} \\
0.047520\\
0.089007
\end{pmatrix}\end{equation*}
\begin{equation*}
\left[ \frac{{w}^{EM}_i}{{w}^{EM}_j} \right] =
\begin{pmatrix}
$\,\,$ 1 $\,\,$ & $\,\,$\color{red} 1.0717\color{black} $\,\,$ & $\,\,$9.3998$\,\,$ & $\,\,$5.0185$\,\,$ \\
$\,\,$\color{red} 0.9331\color{black} $\,\,$ & $\,\,$ 1 $\,\,$ & $\,\,$\color{red} 8.7710\color{black} $\,\,$ & $\,\,$\color{red} 4.6828\color{black}   $\,\,$ \\
$\,\,$0.1064$\,\,$ & $\,\,$\color{red} 0.1140\color{black} $\,\,$ & $\,\,$ 1 $\,\,$ & $\,\,$0.5339 $\,\,$ \\
$\,\,$0.1993$\,\,$ & $\,\,$\color{red} 0.2135\color{black} $\,\,$ & $\,\,$1.8730$\,\,$ & $\,\,$ 1  $\,\,$ \\
\end{pmatrix},
\end{equation*}

\begin{equation*}
\mathbf{w}^{\prime} =
\begin{pmatrix}
0.441868\\
0.423075\\
0.047008\\
0.088049
\end{pmatrix} =
0.989236\cdot
\begin{pmatrix}
0.446676\\
\color{gr} 0.427678\color{black} \\
0.047520\\
0.089007
\end{pmatrix},
\end{equation*}
\begin{equation*}
\left[ \frac{{w}^{\prime}_i}{{w}^{\prime}_j} \right] =
\begin{pmatrix}
$\,\,$ 1 $\,\,$ & $\,\,$\color{gr} 1.0444\color{black} $\,\,$ & $\,\,$9.3998$\,\,$ & $\,\,$5.0185$\,\,$ \\
$\,\,$\color{gr} 0.9575\color{black} $\,\,$ & $\,\,$ 1 $\,\,$ & $\,\,$\color{gr} \color{blue} 9\color{black} $\,\,$ & $\,\,$\color{gr} 4.8050\color{black}   $\,\,$ \\
$\,\,$0.1064$\,\,$ & $\,\,$\color{gr} \color{blue}  1/9\color{black} $\,\,$ & $\,\,$ 1 $\,\,$ & $\,\,$0.5339 $\,\,$ \\
$\,\,$0.1993$\,\,$ & $\,\,$\color{gr} 0.2081\color{black} $\,\,$ & $\,\,$1.8730$\,\,$ & $\,\,$ 1  $\,\,$ \\
\end{pmatrix},
\end{equation*}
\end{example}
\newpage
\begin{example}
\begin{equation*}
\mathbf{A} =
\begin{pmatrix}
$\,\,$ 1 $\,\,$ & $\,\,$1$\,\,$ & $\,\,$6$\,\,$ & $\,\,$9 $\,\,$ \\
$\,\,$ 1 $\,\,$ & $\,\,$ 1 $\,\,$ & $\,\,$8$\,\,$ & $\,\,$6 $\,\,$ \\
$\,\,$ 1/6$\,\,$ & $\,\,$ 1/8$\,\,$ & $\,\,$ 1 $\,\,$ & $\,\,$ 1/2 $\,\,$ \\
$\,\,$ 1/9$\,\,$ & $\,\,$ 1/6$\,\,$ & $\,\,$2$\,\,$ & $\,\,$ 1  $\,\,$ \\
\end{pmatrix},
\qquad
\lambda_{\max} =
4.1031,
\qquad
CR = 0.0389
\end{equation*}

\begin{equation*}
\mathbf{w}^{EM} =
\begin{pmatrix}
0.450745\\
\color{red} 0.423186\color{black} \\
0.053025\\
0.073043
\end{pmatrix}\end{equation*}
\begin{equation*}
\left[ \frac{{w}^{EM}_i}{{w}^{EM}_j} \right] =
\begin{pmatrix}
$\,\,$ 1 $\,\,$ & $\,\,$\color{red} 1.0651\color{black} $\,\,$ & $\,\,$8.5006$\,\,$ & $\,\,$6.1709$\,\,$ \\
$\,\,$\color{red} 0.9389\color{black} $\,\,$ & $\,\,$ 1 $\,\,$ & $\,\,$\color{red} 7.9809\color{black} $\,\,$ & $\,\,$\color{red} 5.7936\color{black}   $\,\,$ \\
$\,\,$0.1176$\,\,$ & $\,\,$\color{red} 0.1253\color{black} $\,\,$ & $\,\,$ 1 $\,\,$ & $\,\,$0.7259 $\,\,$ \\
$\,\,$0.1621$\,\,$ & $\,\,$\color{red} 0.1726\color{black} $\,\,$ & $\,\,$1.3775$\,\,$ & $\,\,$ 1  $\,\,$ \\
\end{pmatrix},
\end{equation*}

\begin{equation*}
\mathbf{w}^{\prime} =
\begin{pmatrix}
0.450289\\
0.423771\\
0.052971\\
0.072969
\end{pmatrix} =
0.998987\cdot
\begin{pmatrix}
0.450745\\
\color{gr} 0.424200\color{black} \\
0.053025\\
0.073043
\end{pmatrix},
\end{equation*}
\begin{equation*}
\left[ \frac{{w}^{\prime}_i}{{w}^{\prime}_j} \right] =
\begin{pmatrix}
$\,\,$ 1 $\,\,$ & $\,\,$\color{gr} 1.0626\color{black} $\,\,$ & $\,\,$8.5006$\,\,$ & $\,\,$6.1709$\,\,$ \\
$\,\,$\color{gr} 0.9411\color{black} $\,\,$ & $\,\,$ 1 $\,\,$ & $\,\,$\color{gr} \color{blue} 8\color{black} $\,\,$ & $\,\,$\color{gr} 5.8075\color{black}   $\,\,$ \\
$\,\,$0.1176$\,\,$ & $\,\,$\color{gr} \color{blue}  1/8\color{black} $\,\,$ & $\,\,$ 1 $\,\,$ & $\,\,$0.7259 $\,\,$ \\
$\,\,$0.1621$\,\,$ & $\,\,$\color{gr} 0.1722\color{black} $\,\,$ & $\,\,$1.3775$\,\,$ & $\,\,$ 1  $\,\,$ \\
\end{pmatrix},
\end{equation*}
\end{example}
\newpage
\begin{example}
\begin{equation*}
\mathbf{A} =
\begin{pmatrix}
$\,\,$ 1 $\,\,$ & $\,\,$1$\,\,$ & $\,\,$6$\,\,$ & $\,\,$9 $\,\,$ \\
$\,\,$ 1 $\,\,$ & $\,\,$ 1 $\,\,$ & $\,\,$9$\,\,$ & $\,\,$5 $\,\,$ \\
$\,\,$ 1/6$\,\,$ & $\,\,$ 1/9$\,\,$ & $\,\,$ 1 $\,\,$ & $\,\,$ 1/3 $\,\,$ \\
$\,\,$ 1/9$\,\,$ & $\,\,$ 1/5$\,\,$ & $\,\,$3$\,\,$ & $\,\,$ 1  $\,\,$ \\
\end{pmatrix},
\qquad
\lambda_{\max} =
4.1966,
\qquad
CR = 0.0741
\end{equation*}

\begin{equation*}
\mathbf{w}^{EM} =
\begin{pmatrix}
0.457664\\
\color{red} 0.409621\color{black} \\
0.047035\\
0.085680
\end{pmatrix}\end{equation*}
\begin{equation*}
\left[ \frac{{w}^{EM}_i}{{w}^{EM}_j} \right] =
\begin{pmatrix}
$\,\,$ 1 $\,\,$ & $\,\,$\color{red} 1.1173\color{black} $\,\,$ & $\,\,$9.7303$\,\,$ & $\,\,$5.3416$\,\,$ \\
$\,\,$\color{red} 0.8950\color{black} $\,\,$ & $\,\,$ 1 $\,\,$ & $\,\,$\color{red} 8.7088\color{black} $\,\,$ & $\,\,$\color{red} 4.7808\color{black}   $\,\,$ \\
$\,\,$0.1028$\,\,$ & $\,\,$\color{red} 0.1148\color{black} $\,\,$ & $\,\,$ 1 $\,\,$ & $\,\,$0.5490 $\,\,$ \\
$\,\,$0.1872$\,\,$ & $\,\,$\color{red} 0.2092\color{black} $\,\,$ & $\,\,$1.8216$\,\,$ & $\,\,$ 1  $\,\,$ \\
\end{pmatrix},
\end{equation*}

\begin{equation*}
\mathbf{w}^{\prime} =
\begin{pmatrix}
0.451481\\
0.417597\\
0.046400\\
0.084522
\end{pmatrix} =
0.986490\cdot
\begin{pmatrix}
0.457664\\
\color{gr} 0.423316\color{black} \\
0.047035\\
0.085680
\end{pmatrix},
\end{equation*}
\begin{equation*}
\left[ \frac{{w}^{\prime}_i}{{w}^{\prime}_j} \right] =
\begin{pmatrix}
$\,\,$ 1 $\,\,$ & $\,\,$\color{gr} 1.0811\color{black} $\,\,$ & $\,\,$9.7303$\,\,$ & $\,\,$5.3416$\,\,$ \\
$\,\,$\color{gr} 0.9250\color{black} $\,\,$ & $\,\,$ 1 $\,\,$ & $\,\,$\color{gr} \color{blue} 9\color{black} $\,\,$ & $\,\,$\color{gr} 4.9407\color{black}   $\,\,$ \\
$\,\,$0.1028$\,\,$ & $\,\,$\color{gr} \color{blue}  1/9\color{black} $\,\,$ & $\,\,$ 1 $\,\,$ & $\,\,$0.5490 $\,\,$ \\
$\,\,$0.1872$\,\,$ & $\,\,$\color{gr} 0.2024\color{black} $\,\,$ & $\,\,$1.8216$\,\,$ & $\,\,$ 1  $\,\,$ \\
\end{pmatrix},
\end{equation*}
\end{example}
\newpage
\begin{example}
\begin{equation*}
\mathbf{A} =
\begin{pmatrix}
$\,\,$ 1 $\,\,$ & $\,\,$1$\,\,$ & $\,\,$6$\,\,$ & $\,\,$9 $\,\,$ \\
$\,\,$ 1 $\,\,$ & $\,\,$ 1 $\,\,$ & $\,\,$9$\,\,$ & $\,\,$6 $\,\,$ \\
$\,\,$ 1/6$\,\,$ & $\,\,$ 1/9$\,\,$ & $\,\,$ 1 $\,\,$ & $\,\,$ 1/2 $\,\,$ \\
$\,\,$ 1/9$\,\,$ & $\,\,$ 1/6$\,\,$ & $\,\,$2$\,\,$ & $\,\,$ 1  $\,\,$ \\
\end{pmatrix},
\qquad
\lambda_{\max} =
4.1031,
\qquad
CR = 0.0389
\end{equation*}

\begin{equation*}
\mathbf{w}^{EM} =
\begin{pmatrix}
0.446189\\
\color{red} 0.430842\color{black} \\
0.050989\\
0.071979
\end{pmatrix}\end{equation*}
\begin{equation*}
\left[ \frac{{w}^{EM}_i}{{w}^{EM}_j} \right] =
\begin{pmatrix}
$\,\,$ 1 $\,\,$ & $\,\,$\color{red} 1.0356\color{black} $\,\,$ & $\,\,$8.7507$\,\,$ & $\,\,$6.1989$\,\,$ \\
$\,\,$\color{red} 0.9656\color{black} $\,\,$ & $\,\,$ 1 $\,\,$ & $\,\,$\color{red} 8.4497\color{black} $\,\,$ & $\,\,$\color{red} 5.9857\color{black}   $\,\,$ \\
$\,\,$0.1143$\,\,$ & $\,\,$\color{red} 0.1183\color{black} $\,\,$ & $\,\,$ 1 $\,\,$ & $\,\,$0.7084 $\,\,$ \\
$\,\,$0.1613$\,\,$ & $\,\,$\color{red} 0.1671\color{black} $\,\,$ & $\,\,$1.4117$\,\,$ & $\,\,$ 1  $\,\,$ \\
\end{pmatrix},
\end{equation*}

\begin{equation*}
\mathbf{w}^{\prime} =
\begin{pmatrix}
0.445729\\
0.431430\\
0.050936\\
0.071905
\end{pmatrix} =
0.998968\cdot
\begin{pmatrix}
0.446189\\
\color{gr} 0.431875\color{black} \\
0.050989\\
0.071979
\end{pmatrix},
\end{equation*}
\begin{equation*}
\left[ \frac{{w}^{\prime}_i}{{w}^{\prime}_j} \right] =
\begin{pmatrix}
$\,\,$ 1 $\,\,$ & $\,\,$\color{gr} 1.0331\color{black} $\,\,$ & $\,\,$8.7507$\,\,$ & $\,\,$6.1989$\,\,$ \\
$\,\,$\color{gr} 0.9679\color{black} $\,\,$ & $\,\,$ 1 $\,\,$ & $\,\,$\color{gr} 8.4700\color{black} $\,\,$ & $\,\,$\color{gr} \color{blue} 6\color{black}   $\,\,$ \\
$\,\,$0.1143$\,\,$ & $\,\,$\color{gr} 0.1181\color{black} $\,\,$ & $\,\,$ 1 $\,\,$ & $\,\,$0.7084 $\,\,$ \\
$\,\,$0.1613$\,\,$ & $\,\,$\color{gr} \color{blue}  1/6\color{black} $\,\,$ & $\,\,$1.4117$\,\,$ & $\,\,$ 1  $\,\,$ \\
\end{pmatrix},
\end{equation*}
\end{example}
\newpage
\begin{example}
\begin{equation*}
\mathbf{A} =
\begin{pmatrix}
$\,\,$ 1 $\,\,$ & $\,\,$1$\,\,$ & $\,\,$7$\,\,$ & $\,\,$4 $\,\,$ \\
$\,\,$ 1 $\,\,$ & $\,\,$ 1 $\,\,$ & $\,\,$9$\,\,$ & $\,\,$3 $\,\,$ \\
$\,\,$ 1/7$\,\,$ & $\,\,$ 1/9$\,\,$ & $\,\,$ 1 $\,\,$ & $\,\,$ 1/4 $\,\,$ \\
$\,\,$ 1/4$\,\,$ & $\,\,$ 1/3$\,\,$ & $\,\,$4$\,\,$ & $\,\,$ 1  $\,\,$ \\
\end{pmatrix},
\qquad
\lambda_{\max} =
4.0576,
\qquad
CR = 0.0217
\end{equation*}

\begin{equation*}
\mathbf{w}^{EM} =
\begin{pmatrix}
0.414620\\
\color{red} 0.403144\color{black} \\
0.045224\\
0.137012
\end{pmatrix}\end{equation*}
\begin{equation*}
\left[ \frac{{w}^{EM}_i}{{w}^{EM}_j} \right] =
\begin{pmatrix}
$\,\,$ 1 $\,\,$ & $\,\,$\color{red} 1.0285\color{black} $\,\,$ & $\,\,$9.1682$\,\,$ & $\,\,$3.0262$\,\,$ \\
$\,\,$\color{red} 0.9723\color{black} $\,\,$ & $\,\,$ 1 $\,\,$ & $\,\,$\color{red} 8.9144\color{black} $\,\,$ & $\,\,$\color{red} 2.9424\color{black}   $\,\,$ \\
$\,\,$0.1091$\,\,$ & $\,\,$\color{red} 0.1122\color{black} $\,\,$ & $\,\,$ 1 $\,\,$ & $\,\,$0.3301 $\,\,$ \\
$\,\,$0.3305$\,\,$ & $\,\,$\color{red} 0.3399\color{black} $\,\,$ & $\,\,$3.0296$\,\,$ & $\,\,$ 1  $\,\,$ \\
\end{pmatrix},
\end{equation*}

\begin{equation*}
\mathbf{w}^{\prime} =
\begin{pmatrix}
0.413022\\
0.405446\\
0.045050\\
0.136483
\end{pmatrix} =
0.996145\cdot
\begin{pmatrix}
0.414620\\
\color{gr} 0.407015\color{black} \\
0.045224\\
0.137012
\end{pmatrix},
\end{equation*}
\begin{equation*}
\left[ \frac{{w}^{\prime}_i}{{w}^{\prime}_j} \right] =
\begin{pmatrix}
$\,\,$ 1 $\,\,$ & $\,\,$\color{gr} 1.0187\color{black} $\,\,$ & $\,\,$9.1682$\,\,$ & $\,\,$3.0262$\,\,$ \\
$\,\,$\color{gr} 0.9817\color{black} $\,\,$ & $\,\,$ 1 $\,\,$ & $\,\,$\color{gr} \color{blue} 9\color{black} $\,\,$ & $\,\,$\color{gr} 2.9707\color{black}   $\,\,$ \\
$\,\,$0.1091$\,\,$ & $\,\,$\color{gr} \color{blue}  1/9\color{black} $\,\,$ & $\,\,$ 1 $\,\,$ & $\,\,$0.3301 $\,\,$ \\
$\,\,$0.3305$\,\,$ & $\,\,$\color{gr} 0.3366\color{black} $\,\,$ & $\,\,$3.0296$\,\,$ & $\,\,$ 1  $\,\,$ \\
\end{pmatrix},
\end{equation*}
\end{example}
\newpage
\begin{example}
\begin{equation*}
\mathbf{A} =
\begin{pmatrix}
$\,\,$ 1 $\,\,$ & $\,\,$1$\,\,$ & $\,\,$7$\,\,$ & $\,\,$5 $\,\,$ \\
$\,\,$ 1 $\,\,$ & $\,\,$ 1 $\,\,$ & $\,\,$9$\,\,$ & $\,\,$4 $\,\,$ \\
$\,\,$ 1/7$\,\,$ & $\,\,$ 1/9$\,\,$ & $\,\,$ 1 $\,\,$ & $\,\,$ 1/3 $\,\,$ \\
$\,\,$ 1/5$\,\,$ & $\,\,$ 1/4$\,\,$ & $\,\,$3$\,\,$ & $\,\,$ 1  $\,\,$ \\
\end{pmatrix},
\qquad
\lambda_{\max} =
4.0490,
\qquad
CR = 0.0185
\end{equation*}

\begin{equation*}
\mathbf{w}^{EM} =
\begin{pmatrix}
0.423969\\
\color{red} 0.420396\color{black} \\
0.047055\\
0.108580
\end{pmatrix}\end{equation*}
\begin{equation*}
\left[ \frac{{w}^{EM}_i}{{w}^{EM}_j} \right] =
\begin{pmatrix}
$\,\,$ 1 $\,\,$ & $\,\,$\color{red} 1.0085\color{black} $\,\,$ & $\,\,$9.0100$\,\,$ & $\,\,$3.9047$\,\,$ \\
$\,\,$\color{red} 0.9916\color{black} $\,\,$ & $\,\,$ 1 $\,\,$ & $\,\,$\color{red} 8.9341\color{black} $\,\,$ & $\,\,$\color{red} 3.8718\color{black}   $\,\,$ \\
$\,\,$0.1110$\,\,$ & $\,\,$\color{red} 0.1119\color{black} $\,\,$ & $\,\,$ 1 $\,\,$ & $\,\,$0.4334 $\,\,$ \\
$\,\,$0.2561$\,\,$ & $\,\,$\color{red} 0.2583\color{black} $\,\,$ & $\,\,$2.3075$\,\,$ & $\,\,$ 1  $\,\,$ \\
\end{pmatrix},
\end{equation*}

\begin{equation*}
\mathbf{w}^{\prime} =
\begin{pmatrix}
0.422659\\
0.422187\\
0.046910\\
0.108244
\end{pmatrix} =
0.996909\cdot
\begin{pmatrix}
0.423969\\
\color{gr} 0.423497\color{black} \\
0.047055\\
0.108580
\end{pmatrix},
\end{equation*}
\begin{equation*}
\left[ \frac{{w}^{\prime}_i}{{w}^{\prime}_j} \right] =
\begin{pmatrix}
$\,\,$ 1 $\,\,$ & $\,\,$\color{gr} 1.0011\color{black} $\,\,$ & $\,\,$9.0100$\,\,$ & $\,\,$3.9047$\,\,$ \\
$\,\,$\color{gr} 0.9989\color{black} $\,\,$ & $\,\,$ 1 $\,\,$ & $\,\,$\color{gr} \color{blue} 9\color{black} $\,\,$ & $\,\,$\color{gr} 3.9003\color{black}   $\,\,$ \\
$\,\,$0.1110$\,\,$ & $\,\,$\color{gr} \color{blue}  1/9\color{black} $\,\,$ & $\,\,$ 1 $\,\,$ & $\,\,$0.4334 $\,\,$ \\
$\,\,$0.2561$\,\,$ & $\,\,$\color{gr} 0.2564\color{black} $\,\,$ & $\,\,$2.3075$\,\,$ & $\,\,$ 1  $\,\,$ \\
\end{pmatrix},
\end{equation*}
\end{example}
\newpage
\begin{example}
\begin{equation*}
\mathbf{A} =
\begin{pmatrix}
$\,\,$ 1 $\,\,$ & $\,\,$1$\,\,$ & $\,\,$7$\,\,$ & $\,\,$8 $\,\,$ \\
$\,\,$ 1 $\,\,$ & $\,\,$ 1 $\,\,$ & $\,\,$9$\,\,$ & $\,\,$6 $\,\,$ \\
$\,\,$ 1/7$\,\,$ & $\,\,$ 1/9$\,\,$ & $\,\,$ 1 $\,\,$ & $\,\,$ 1/2 $\,\,$ \\
$\,\,$ 1/8$\,\,$ & $\,\,$ 1/6$\,\,$ & $\,\,$2$\,\,$ & $\,\,$ 1  $\,\,$ \\
\end{pmatrix},
\qquad
\lambda_{\max} =
4.0576,
\qquad
CR = 0.0217
\end{equation*}

\begin{equation*}
\mathbf{w}^{EM} =
\begin{pmatrix}
0.445113\\
\color{red} 0.432793\color{black} \\
0.048550\\
0.073544
\end{pmatrix}\end{equation*}
\begin{equation*}
\left[ \frac{{w}^{EM}_i}{{w}^{EM}_j} \right] =
\begin{pmatrix}
$\,\,$ 1 $\,\,$ & $\,\,$\color{red} 1.0285\color{black} $\,\,$ & $\,\,$9.1682$\,\,$ & $\,\,$6.0523$\,\,$ \\
$\,\,$\color{red} 0.9723\color{black} $\,\,$ & $\,\,$ 1 $\,\,$ & $\,\,$\color{red} 8.9144\color{black} $\,\,$ & $\,\,$\color{red} 5.8848\color{black}   $\,\,$ \\
$\,\,$0.1091$\,\,$ & $\,\,$\color{red} 0.1122\color{black} $\,\,$ & $\,\,$ 1 $\,\,$ & $\,\,$0.6601 $\,\,$ \\
$\,\,$0.1652$\,\,$ & $\,\,$\color{red} 0.1699\color{black} $\,\,$ & $\,\,$1.5148$\,\,$ & $\,\,$ 1  $\,\,$ \\
\end{pmatrix},
\end{equation*}

\begin{equation*}
\mathbf{w}^{\prime} =
\begin{pmatrix}
0.443271\\
0.435140\\
0.048349\\
0.073240
\end{pmatrix} =
0.995862\cdot
\begin{pmatrix}
0.445113\\
\color{gr} 0.436948\color{black} \\
0.048550\\
0.073544
\end{pmatrix},
\end{equation*}
\begin{equation*}
\left[ \frac{{w}^{\prime}_i}{{w}^{\prime}_j} \right] =
\begin{pmatrix}
$\,\,$ 1 $\,\,$ & $\,\,$\color{gr} 1.0187\color{black} $\,\,$ & $\,\,$9.1682$\,\,$ & $\,\,$6.0523$\,\,$ \\
$\,\,$\color{gr} 0.9817\color{black} $\,\,$ & $\,\,$ 1 $\,\,$ & $\,\,$\color{gr} \color{blue} 9\color{black} $\,\,$ & $\,\,$\color{gr} 5.9413\color{black}   $\,\,$ \\
$\,\,$0.1091$\,\,$ & $\,\,$\color{gr} \color{blue}  1/9\color{black} $\,\,$ & $\,\,$ 1 $\,\,$ & $\,\,$0.6601 $\,\,$ \\
$\,\,$0.1652$\,\,$ & $\,\,$\color{gr} 0.1683\color{black} $\,\,$ & $\,\,$1.5148$\,\,$ & $\,\,$ 1  $\,\,$ \\
\end{pmatrix},
\end{equation*}
\end{example}
\newpage
\begin{example}
\begin{equation*}
\mathbf{A} =
\begin{pmatrix}
$\,\,$ 1 $\,\,$ & $\,\,$1$\,\,$ & $\,\,$8$\,\,$ & $\,\,$6 $\,\,$ \\
$\,\,$ 1 $\,\,$ & $\,\,$ 1 $\,\,$ & $\,\,$6$\,\,$ & $\,\,$2 $\,\,$ \\
$\,\,$ 1/8$\,\,$ & $\,\,$ 1/6$\,\,$ & $\,\,$ 1 $\,\,$ & $\,\,$ 1/2 $\,\,$ \\
$\,\,$ 1/6$\,\,$ & $\,\,$ 1/2$\,\,$ & $\,\,$2$\,\,$ & $\,\,$ 1  $\,\,$ \\
\end{pmatrix},
\qquad
\lambda_{\max} =
4.1031,
\qquad
CR = 0.0389
\end{equation*}

\begin{equation*}
\mathbf{w}^{EM} =
\begin{pmatrix}
0.483965\\
0.341598\\
\color{red} 0.056797\color{black} \\
0.117640
\end{pmatrix}\end{equation*}
\begin{equation*}
\left[ \frac{{w}^{EM}_i}{{w}^{EM}_j} \right] =
\begin{pmatrix}
$\,\,$ 1 $\,\,$ & $\,\,$1.4168$\,\,$ & $\,\,$\color{red} 8.5210\color{black} $\,\,$ & $\,\,$4.1140$\,\,$ \\
$\,\,$0.7058$\,\,$ & $\,\,$ 1 $\,\,$ & $\,\,$\color{red} 6.0144\color{black} $\,\,$ & $\,\,$2.9038  $\,\,$ \\
$\,\,$\color{red} 0.1174\color{black} $\,\,$ & $\,\,$\color{red} 0.1663\color{black} $\,\,$ & $\,\,$ 1 $\,\,$ & $\,\,$\color{red} 0.4828\color{black}  $\,\,$ \\
$\,\,$0.2431$\,\,$ & $\,\,$0.3444$\,\,$ & $\,\,$\color{red} 2.0712\color{black} $\,\,$ & $\,\,$ 1  $\,\,$ \\
\end{pmatrix},
\end{equation*}

\begin{equation*}
\mathbf{w}^{\prime} =
\begin{pmatrix}
0.483899\\
0.341551\\
0.056925\\
0.117624
\end{pmatrix} =
0.999864\cdot
\begin{pmatrix}
0.483965\\
0.341598\\
\color{gr} 0.056933\color{black} \\
0.117640
\end{pmatrix},
\end{equation*}
\begin{equation*}
\left[ \frac{{w}^{\prime}_i}{{w}^{\prime}_j} \right] =
\begin{pmatrix}
$\,\,$ 1 $\,\,$ & $\,\,$1.4168$\,\,$ & $\,\,$\color{gr} 8.5006\color{black} $\,\,$ & $\,\,$4.1140$\,\,$ \\
$\,\,$0.7058$\,\,$ & $\,\,$ 1 $\,\,$ & $\,\,$\color{gr} \color{blue} 6\color{black} $\,\,$ & $\,\,$2.9038  $\,\,$ \\
$\,\,$\color{gr} 0.1176\color{black} $\,\,$ & $\,\,$\color{gr} \color{blue}  1/6\color{black} $\,\,$ & $\,\,$ 1 $\,\,$ & $\,\,$\color{gr} 0.4840\color{black}  $\,\,$ \\
$\,\,$0.2431$\,\,$ & $\,\,$0.3444$\,\,$ & $\,\,$\color{gr} 2.0663\color{black} $\,\,$ & $\,\,$ 1  $\,\,$ \\
\end{pmatrix},
\end{equation*}
\end{example}
\newpage
\begin{example}
\begin{equation*}
\mathbf{A} =
\begin{pmatrix}
$\,\,$ 1 $\,\,$ & $\,\,$1$\,\,$ & $\,\,$8$\,\,$ & $\,\,$7 $\,\,$ \\
$\,\,$ 1 $\,\,$ & $\,\,$ 1 $\,\,$ & $\,\,$6$\,\,$ & $\,\,$2 $\,\,$ \\
$\,\,$ 1/8$\,\,$ & $\,\,$ 1/6$\,\,$ & $\,\,$ 1 $\,\,$ & $\,\,$ 1/2 $\,\,$ \\
$\,\,$ 1/7$\,\,$ & $\,\,$ 1/2$\,\,$ & $\,\,$2$\,\,$ & $\,\,$ 1  $\,\,$ \\
\end{pmatrix},
\qquad
\lambda_{\max} =
4.1365,
\qquad
CR = 0.0515
\end{equation*}

\begin{equation*}
\mathbf{w}^{EM} =
\begin{pmatrix}
0.497323\\
0.335720\\
\color{red} 0.055437\color{black} \\
0.111519
\end{pmatrix}\end{equation*}
\begin{equation*}
\left[ \frac{{w}^{EM}_i}{{w}^{EM}_j} \right] =
\begin{pmatrix}
$\,\,$ 1 $\,\,$ & $\,\,$1.4814$\,\,$ & $\,\,$\color{red} 8.9709\color{black} $\,\,$ & $\,\,$4.4595$\,\,$ \\
$\,\,$0.6751$\,\,$ & $\,\,$ 1 $\,\,$ & $\,\,$\color{red} 6.0559\color{black} $\,\,$ & $\,\,$3.0104  $\,\,$ \\
$\,\,$\color{red} 0.1115\color{black} $\,\,$ & $\,\,$\color{red} 0.1651\color{black} $\,\,$ & $\,\,$ 1 $\,\,$ & $\,\,$\color{red} 0.4971\color{black}  $\,\,$ \\
$\,\,$0.2242$\,\,$ & $\,\,$0.3322$\,\,$ & $\,\,$\color{red} 2.0116\color{black} $\,\,$ & $\,\,$ 1  $\,\,$ \\
\end{pmatrix},
\end{equation*}

\begin{equation*}
\mathbf{w}^{\prime} =
\begin{pmatrix}
0.497163\\
0.335612\\
0.055742\\
0.111483
\end{pmatrix} =
0.999678\cdot
\begin{pmatrix}
0.497323\\
0.335720\\
\color{gr} 0.055760\color{black} \\
0.111519
\end{pmatrix},
\end{equation*}
\begin{equation*}
\left[ \frac{{w}^{\prime}_i}{{w}^{\prime}_j} \right] =
\begin{pmatrix}
$\,\,$ 1 $\,\,$ & $\,\,$1.4814$\,\,$ & $\,\,$\color{gr} 8.9190\color{black} $\,\,$ & $\,\,$4.4595$\,\,$ \\
$\,\,$0.6751$\,\,$ & $\,\,$ 1 $\,\,$ & $\,\,$\color{gr} 6.0208\color{black} $\,\,$ & $\,\,$3.0104  $\,\,$ \\
$\,\,$\color{gr} 0.1121\color{black} $\,\,$ & $\,\,$\color{gr} 0.1661\color{black} $\,\,$ & $\,\,$ 1 $\,\,$ & $\,\,$\color{gr} \color{blue}  1/2\color{black}  $\,\,$ \\
$\,\,$0.2242$\,\,$ & $\,\,$0.3322$\,\,$ & $\,\,$\color{gr} \color{blue} 2\color{black} $\,\,$ & $\,\,$ 1  $\,\,$ \\
\end{pmatrix},
\end{equation*}
\end{example}
\newpage
\begin{example}
\begin{equation*}
\mathbf{A} =
\begin{pmatrix}
$\,\,$ 1 $\,\,$ & $\,\,$1$\,\,$ & $\,\,$9$\,\,$ & $\,\,$3 $\,\,$ \\
$\,\,$ 1 $\,\,$ & $\,\,$ 1 $\,\,$ & $\,\,$3$\,\,$ & $\,\,$2 $\,\,$ \\
$\,\,$ 1/9$\,\,$ & $\,\,$ 1/3$\,\,$ & $\,\,$ 1 $\,\,$ & $\,\,$ 1/2 $\,\,$ \\
$\,\,$ 1/3$\,\,$ & $\,\,$ 1/2$\,\,$ & $\,\,$2$\,\,$ & $\,\,$ 1  $\,\,$ \\
\end{pmatrix},
\qquad
\lambda_{\max} =
4.1031,
\qquad
CR = 0.0389
\end{equation*}

\begin{equation*}
\mathbf{w}^{EM} =
\begin{pmatrix}
0.461075\\
0.316140\\
0.074380\\
\color{red} 0.148405\color{black}
\end{pmatrix}\end{equation*}
\begin{equation*}
\left[ \frac{{w}^{EM}_i}{{w}^{EM}_j} \right] =
\begin{pmatrix}
$\,\,$ 1 $\,\,$ & $\,\,$1.4585$\,\,$ & $\,\,$6.1989$\,\,$ & $\,\,$\color{red} 3.1069\color{black} $\,\,$ \\
$\,\,$0.6857$\,\,$ & $\,\,$ 1 $\,\,$ & $\,\,$4.2503$\,\,$ & $\,\,$\color{red} 2.1302\color{black}   $\,\,$ \\
$\,\,$0.1613$\,\,$ & $\,\,$0.2353$\,\,$ & $\,\,$ 1 $\,\,$ & $\,\,$\color{red} 0.5012\color{black}  $\,\,$ \\
$\,\,$\color{red} 0.3219\color{black} $\,\,$ & $\,\,$\color{red} 0.4694\color{black} $\,\,$ & $\,\,$\color{red} 1.9952\color{black} $\,\,$ & $\,\,$ 1  $\,\,$ \\
\end{pmatrix},
\end{equation*}

\begin{equation*}
\mathbf{w}^{\prime} =
\begin{pmatrix}
0.460911\\
0.316027\\
0.074354\\
0.148708
\end{pmatrix} =
0.999644\cdot
\begin{pmatrix}
0.461075\\
0.316140\\
0.074380\\
\color{gr} 0.148761\color{black}
\end{pmatrix},
\end{equation*}
\begin{equation*}
\left[ \frac{{w}^{\prime}_i}{{w}^{\prime}_j} \right] =
\begin{pmatrix}
$\,\,$ 1 $\,\,$ & $\,\,$1.4585$\,\,$ & $\,\,$6.1989$\,\,$ & $\,\,$\color{gr} 3.0994\color{black} $\,\,$ \\
$\,\,$0.6857$\,\,$ & $\,\,$ 1 $\,\,$ & $\,\,$4.2503$\,\,$ & $\,\,$\color{gr} 2.1252\color{black}   $\,\,$ \\
$\,\,$0.1613$\,\,$ & $\,\,$0.2353$\,\,$ & $\,\,$ 1 $\,\,$ & $\,\,$\color{gr} \color{blue}  1/2\color{black}  $\,\,$ \\
$\,\,$\color{gr} 0.3226\color{black} $\,\,$ & $\,\,$\color{gr} 0.4706\color{black} $\,\,$ & $\,\,$\color{gr} \color{blue} 2\color{black} $\,\,$ & $\,\,$ 1  $\,\,$ \\
\end{pmatrix},
\end{equation*}
\end{example}
\newpage
\begin{example}
\begin{equation*}
\mathbf{A} =
\begin{pmatrix}
$\,\,$ 1 $\,\,$ & $\,\,$1$\,\,$ & $\,\,$9$\,\,$ & $\,\,$6 $\,\,$ \\
$\,\,$ 1 $\,\,$ & $\,\,$ 1 $\,\,$ & $\,\,$6$\,\,$ & $\,\,$2 $\,\,$ \\
$\,\,$ 1/9$\,\,$ & $\,\,$ 1/6$\,\,$ & $\,\,$ 1 $\,\,$ & $\,\,$ 1/2 $\,\,$ \\
$\,\,$ 1/6$\,\,$ & $\,\,$ 1/2$\,\,$ & $\,\,$2$\,\,$ & $\,\,$ 1  $\,\,$ \\
\end{pmatrix},
\qquad
\lambda_{\max} =
4.1031,
\qquad
CR = 0.0389
\end{equation*}

\begin{equation*}
\mathbf{w}^{EM} =
\begin{pmatrix}
0.491223\\
0.338336\\
\color{red} 0.054450\color{black} \\
0.115992
\end{pmatrix}\end{equation*}
\begin{equation*}
\left[ \frac{{w}^{EM}_i}{{w}^{EM}_j} \right] =
\begin{pmatrix}
$\,\,$ 1 $\,\,$ & $\,\,$1.4519$\,\,$ & $\,\,$\color{red} 9.0216\color{black} $\,\,$ & $\,\,$4.2350$\,\,$ \\
$\,\,$0.6888$\,\,$ & $\,\,$ 1 $\,\,$ & $\,\,$\color{red} 6.2137\color{black} $\,\,$ & $\,\,$2.9169  $\,\,$ \\
$\,\,$\color{red} 0.1108\color{black} $\,\,$ & $\,\,$\color{red} 0.1609\color{black} $\,\,$ & $\,\,$ 1 $\,\,$ & $\,\,$\color{red} 0.4694\color{black}  $\,\,$ \\
$\,\,$0.2361$\,\,$ & $\,\,$0.3428$\,\,$ & $\,\,$\color{red} 2.1302\color{black} $\,\,$ & $\,\,$ 1  $\,\,$ \\
\end{pmatrix},
\end{equation*}

\begin{equation*}
\mathbf{w}^{\prime} =
\begin{pmatrix}
0.491159\\
0.338292\\
0.054573\\
0.115976
\end{pmatrix} =
0.999870\cdot
\begin{pmatrix}
0.491223\\
0.338336\\
\color{gr} 0.054580\color{black} \\
0.115992
\end{pmatrix},
\end{equation*}
\begin{equation*}
\left[ \frac{{w}^{\prime}_i}{{w}^{\prime}_j} \right] =
\begin{pmatrix}
$\,\,$ 1 $\,\,$ & $\,\,$1.4519$\,\,$ & $\,\,$\color{gr} \color{blue} 9\color{black} $\,\,$ & $\,\,$4.2350$\,\,$ \\
$\,\,$0.6888$\,\,$ & $\,\,$ 1 $\,\,$ & $\,\,$\color{gr} 6.1989\color{black} $\,\,$ & $\,\,$2.9169  $\,\,$ \\
$\,\,$\color{gr} \color{blue}  1/9\color{black} $\,\,$ & $\,\,$\color{gr} 0.1613\color{black} $\,\,$ & $\,\,$ 1 $\,\,$ & $\,\,$\color{gr} 0.4706\color{black}  $\,\,$ \\
$\,\,$0.2361$\,\,$ & $\,\,$0.3428$\,\,$ & $\,\,$\color{gr} 2.1252\color{black} $\,\,$ & $\,\,$ 1  $\,\,$ \\
\end{pmatrix},
\end{equation*}
\end{example}
\newpage
\begin{example}
\begin{equation*}
\mathbf{A} =
\begin{pmatrix}
$\,\,$ 1 $\,\,$ & $\,\,$1$\,\,$ & $\,\,$9$\,\,$ & $\,\,$7 $\,\,$ \\
$\,\,$ 1 $\,\,$ & $\,\,$ 1 $\,\,$ & $\,\,$6$\,\,$ & $\,\,$2 $\,\,$ \\
$\,\,$ 1/9$\,\,$ & $\,\,$ 1/6$\,\,$ & $\,\,$ 1 $\,\,$ & $\,\,$ 1/2 $\,\,$ \\
$\,\,$ 1/7$\,\,$ & $\,\,$ 1/2$\,\,$ & $\,\,$2$\,\,$ & $\,\,$ 1  $\,\,$ \\
\end{pmatrix},
\qquad
\lambda_{\max} =
4.1342,
\qquad
CR = 0.0506
\end{equation*}

\begin{equation*}
\mathbf{w}^{EM} =
\begin{pmatrix}
0.504228\\
0.332712\\
\color{red} 0.053109\color{black} \\
0.109951
\end{pmatrix}\end{equation*}
\begin{equation*}
\left[ \frac{{w}^{EM}_i}{{w}^{EM}_j} \right] =
\begin{pmatrix}
$\,\,$ 1 $\,\,$ & $\,\,$1.5155$\,\,$ & $\,\,$\color{red} 9.4943\color{black} $\,\,$ & $\,\,$4.5859$\,\,$ \\
$\,\,$0.6598$\,\,$ & $\,\,$ 1 $\,\,$ & $\,\,$\color{red} 6.2647\color{black} $\,\,$ & $\,\,$3.0260  $\,\,$ \\
$\,\,$\color{red} 0.1053\color{black} $\,\,$ & $\,\,$\color{red} 0.1596\color{black} $\,\,$ & $\,\,$ 1 $\,\,$ & $\,\,$\color{red} 0.4830\color{black}  $\,\,$ \\
$\,\,$0.2181$\,\,$ & $\,\,$0.3305$\,\,$ & $\,\,$\color{red} 2.0703\color{black} $\,\,$ & $\,\,$ 1  $\,\,$ \\
\end{pmatrix},
\end{equation*}

\begin{equation*}
\mathbf{w}^{\prime} =
\begin{pmatrix}
0.503289\\
0.332092\\
0.054873\\
0.109746
\end{pmatrix} =
0.998137\cdot
\begin{pmatrix}
0.504228\\
0.332712\\
\color{gr} 0.054975\color{black} \\
0.109951
\end{pmatrix},
\end{equation*}
\begin{equation*}
\left[ \frac{{w}^{\prime}_i}{{w}^{\prime}_j} \right] =
\begin{pmatrix}
$\,\,$ 1 $\,\,$ & $\,\,$1.5155$\,\,$ & $\,\,$\color{gr} 9.1719\color{black} $\,\,$ & $\,\,$4.5859$\,\,$ \\
$\,\,$0.6598$\,\,$ & $\,\,$ 1 $\,\,$ & $\,\,$\color{gr} 6.0520\color{black} $\,\,$ & $\,\,$3.0260  $\,\,$ \\
$\,\,$\color{gr} 0.1090\color{black} $\,\,$ & $\,\,$\color{gr} 0.1652\color{black} $\,\,$ & $\,\,$ 1 $\,\,$ & $\,\,$\color{gr} \color{blue}  1/2\color{black}  $\,\,$ \\
$\,\,$0.2181$\,\,$ & $\,\,$0.3305$\,\,$ & $\,\,$\color{gr} \color{blue} 2\color{black} $\,\,$ & $\,\,$ 1  $\,\,$ \\
\end{pmatrix},
\end{equation*}
\end{example}
\newpage
\begin{example}
\begin{equation*}
\mathbf{A} =
\begin{pmatrix}
$\,\,$ 1 $\,\,$ & $\,\,$1$\,\,$ & $\,\,$9$\,\,$ & $\,\,$8 $\,\,$ \\
$\,\,$ 1 $\,\,$ & $\,\,$ 1 $\,\,$ & $\,\,$6$\,\,$ & $\,\,$2 $\,\,$ \\
$\,\,$ 1/9$\,\,$ & $\,\,$ 1/6$\,\,$ & $\,\,$ 1 $\,\,$ & $\,\,$ 1/2 $\,\,$ \\
$\,\,$ 1/8$\,\,$ & $\,\,$ 1/2$\,\,$ & $\,\,$2$\,\,$ & $\,\,$ 1  $\,\,$ \\
\end{pmatrix},
\qquad
\lambda_{\max} =
4.1664,
\qquad
CR = 0.0627
\end{equation*}

\begin{equation*}
\mathbf{w}^{EM} =
\begin{pmatrix}
0.515807\\
0.327454\\
\color{red} 0.051892\color{black} \\
0.104847
\end{pmatrix}\end{equation*}
\begin{equation*}
\left[ \frac{{w}^{EM}_i}{{w}^{EM}_j} \right] =
\begin{pmatrix}
$\,\,$ 1 $\,\,$ & $\,\,$1.5752$\,\,$ & $\,\,$\color{red} 9.9400\color{black} $\,\,$ & $\,\,$4.9196$\,\,$ \\
$\,\,$0.6348$\,\,$ & $\,\,$ 1 $\,\,$ & $\,\,$\color{red} 6.3103\color{black} $\,\,$ & $\,\,$3.1232  $\,\,$ \\
$\,\,$\color{red} 0.1006\color{black} $\,\,$ & $\,\,$\color{red} 0.1585\color{black} $\,\,$ & $\,\,$ 1 $\,\,$ & $\,\,$\color{red} 0.4949\color{black}  $\,\,$ \\
$\,\,$0.2033$\,\,$ & $\,\,$0.3202$\,\,$ & $\,\,$\color{red} 2.0205\color{black} $\,\,$ & $\,\,$ 1  $\,\,$ \\
\end{pmatrix},
\end{equation*}

\begin{equation*}
\mathbf{w}^{\prime} =
\begin{pmatrix}
0.515533\\
0.327280\\
0.052395\\
0.104791
\end{pmatrix} =
0.999469\cdot
\begin{pmatrix}
0.515807\\
0.327454\\
\color{gr} 0.052423\color{black} \\
0.104847
\end{pmatrix},
\end{equation*}
\begin{equation*}
\left[ \frac{{w}^{\prime}_i}{{w}^{\prime}_j} \right] =
\begin{pmatrix}
$\,\,$ 1 $\,\,$ & $\,\,$1.5752$\,\,$ & $\,\,$\color{gr} 9.8393\color{black} $\,\,$ & $\,\,$4.9196$\,\,$ \\
$\,\,$0.6348$\,\,$ & $\,\,$ 1 $\,\,$ & $\,\,$\color{gr} 6.2463\color{black} $\,\,$ & $\,\,$3.1232  $\,\,$ \\
$\,\,$\color{gr} 0.1016\color{black} $\,\,$ & $\,\,$\color{gr} 0.1601\color{black} $\,\,$ & $\,\,$ 1 $\,\,$ & $\,\,$\color{gr} \color{blue}  1/2\color{black}  $\,\,$ \\
$\,\,$0.2033$\,\,$ & $\,\,$0.3202$\,\,$ & $\,\,$\color{gr} \color{blue} 2\color{black} $\,\,$ & $\,\,$ 1  $\,\,$ \\
\end{pmatrix},
\end{equation*}
\end{example}
\newpage
\begin{example}
\begin{equation*}
\mathbf{A} =
\begin{pmatrix}
$\,\,$ 1 $\,\,$ & $\,\,$2$\,\,$ & $\,\,$2$\,\,$ & $\,\,$1 $\,\,$ \\
$\,\,$ 1/2$\,\,$ & $\,\,$ 1 $\,\,$ & $\,\,$5$\,\,$ & $\,\,$1 $\,\,$ \\
$\,\,$ 1/2$\,\,$ & $\,\,$ 1/5$\,\,$ & $\,\,$ 1 $\,\,$ & $\,\,$ 1/3 $\,\,$ \\
$\,\,$ 1 $\,\,$ & $\,\,$ 1 $\,\,$ & $\,\,$3$\,\,$ & $\,\,$ 1  $\,\,$ \\
\end{pmatrix},
\qquad
\lambda_{\max} =
4.2277,
\qquad
CR = 0.0859
\end{equation*}

\begin{equation*}
\mathbf{w}^{EM} =
\begin{pmatrix}
0.328452\\
0.290505\\
0.098100\\
\color{red} 0.282942\color{black}
\end{pmatrix}\end{equation*}
\begin{equation*}
\left[ \frac{{w}^{EM}_i}{{w}^{EM}_j} \right] =
\begin{pmatrix}
$\,\,$ 1 $\,\,$ & $\,\,$1.1306$\,\,$ & $\,\,$3.3481$\,\,$ & $\,\,$\color{red} 1.1608\color{black} $\,\,$ \\
$\,\,$0.8845$\,\,$ & $\,\,$ 1 $\,\,$ & $\,\,$2.9613$\,\,$ & $\,\,$\color{red} 1.0267\color{black}   $\,\,$ \\
$\,\,$0.2987$\,\,$ & $\,\,$0.3377$\,\,$ & $\,\,$ 1 $\,\,$ & $\,\,$\color{red} 0.3467\color{black}  $\,\,$ \\
$\,\,$\color{red} 0.8614\color{black} $\,\,$ & $\,\,$\color{red} 0.9740\color{black} $\,\,$ & $\,\,$\color{red} 2.8842\color{black} $\,\,$ & $\,\,$ 1  $\,\,$ \\
\end{pmatrix},
\end{equation*}

\begin{equation*}
\mathbf{w}^{\prime} =
\begin{pmatrix}
0.325987\\
0.288325\\
0.097364\\
0.288325
\end{pmatrix} =
0.992494\cdot
\begin{pmatrix}
0.328452\\
0.290505\\
0.098100\\
\color{gr} 0.290505\color{black}
\end{pmatrix},
\end{equation*}
\begin{equation*}
\left[ \frac{{w}^{\prime}_i}{{w}^{\prime}_j} \right] =
\begin{pmatrix}
$\,\,$ 1 $\,\,$ & $\,\,$1.1306$\,\,$ & $\,\,$3.3481$\,\,$ & $\,\,$\color{gr} 1.1306\color{black} $\,\,$ \\
$\,\,$0.8845$\,\,$ & $\,\,$ 1 $\,\,$ & $\,\,$2.9613$\,\,$ & $\,\,$\color{gr} \color{blue} 1\color{black}   $\,\,$ \\
$\,\,$0.2987$\,\,$ & $\,\,$0.3377$\,\,$ & $\,\,$ 1 $\,\,$ & $\,\,$\color{gr} 0.3377\color{black}  $\,\,$ \\
$\,\,$\color{gr} 0.8845\color{black} $\,\,$ & $\,\,$\color{gr} \color{blue} 1\color{black} $\,\,$ & $\,\,$\color{gr} 2.9613\color{black} $\,\,$ & $\,\,$ 1  $\,\,$ \\
\end{pmatrix},
\end{equation*}
\end{example}
\newpage
\begin{example}
\begin{equation*}
\mathbf{A} =
\begin{pmatrix}
$\,\,$ 1 $\,\,$ & $\,\,$2$\,\,$ & $\,\,$3$\,\,$ & $\,\,$1 $\,\,$ \\
$\,\,$ 1/2$\,\,$ & $\,\,$ 1 $\,\,$ & $\,\,$8$\,\,$ & $\,\,$1 $\,\,$ \\
$\,\,$ 1/3$\,\,$ & $\,\,$ 1/8$\,\,$ & $\,\,$ 1 $\,\,$ & $\,\,$ 1/5 $\,\,$ \\
$\,\,$ 1 $\,\,$ & $\,\,$ 1 $\,\,$ & $\,\,$5$\,\,$ & $\,\,$ 1  $\,\,$ \\
\end{pmatrix},
\qquad
\lambda_{\max} =
4.2460,
\qquad
CR = 0.0928
\end{equation*}

\begin{equation*}
\mathbf{w}^{EM} =
\begin{pmatrix}
0.337058\\
0.302161\\
0.064503\\
\color{red} 0.296278\color{black}
\end{pmatrix}\end{equation*}
\begin{equation*}
\left[ \frac{{w}^{EM}_i}{{w}^{EM}_j} \right] =
\begin{pmatrix}
$\,\,$ 1 $\,\,$ & $\,\,$1.1155$\,\,$ & $\,\,$5.2255$\,\,$ & $\,\,$\color{red} 1.1376\color{black} $\,\,$ \\
$\,\,$0.8965$\,\,$ & $\,\,$ 1 $\,\,$ & $\,\,$4.6845$\,\,$ & $\,\,$\color{red} 1.0199\color{black}   $\,\,$ \\
$\,\,$0.1914$\,\,$ & $\,\,$0.2135$\,\,$ & $\,\,$ 1 $\,\,$ & $\,\,$\color{red} 0.2177\color{black}  $\,\,$ \\
$\,\,$\color{red} 0.8790\color{black} $\,\,$ & $\,\,$\color{red} 0.9805\color{black} $\,\,$ & $\,\,$\color{red} 4.5933\color{black} $\,\,$ & $\,\,$ 1  $\,\,$ \\
\end{pmatrix},
\end{equation*}

\begin{equation*}
\mathbf{w}^{\prime} =
\begin{pmatrix}
0.335087\\
0.300394\\
0.064125\\
0.300394
\end{pmatrix} =
0.994152\cdot
\begin{pmatrix}
0.337058\\
0.302161\\
0.064503\\
\color{gr} 0.302161\color{black}
\end{pmatrix},
\end{equation*}
\begin{equation*}
\left[ \frac{{w}^{\prime}_i}{{w}^{\prime}_j} \right] =
\begin{pmatrix}
$\,\,$ 1 $\,\,$ & $\,\,$1.1155$\,\,$ & $\,\,$5.2255$\,\,$ & $\,\,$\color{gr} 1.1155\color{black} $\,\,$ \\
$\,\,$0.8965$\,\,$ & $\,\,$ 1 $\,\,$ & $\,\,$4.6845$\,\,$ & $\,\,$\color{gr} \color{blue} 1\color{black}   $\,\,$ \\
$\,\,$0.1914$\,\,$ & $\,\,$0.2135$\,\,$ & $\,\,$ 1 $\,\,$ & $\,\,$\color{gr} 0.2135\color{black}  $\,\,$ \\
$\,\,$\color{gr} 0.8965\color{black} $\,\,$ & $\,\,$\color{gr} \color{blue} 1\color{black} $\,\,$ & $\,\,$\color{gr} 4.6845\color{black} $\,\,$ & $\,\,$ 1  $\,\,$ \\
\end{pmatrix},
\end{equation*}
\end{example}
\newpage
\begin{example}
\begin{equation*}
\mathbf{A} =
\begin{pmatrix}
$\,\,$ 1 $\,\,$ & $\,\,$2$\,\,$ & $\,\,$4$\,\,$ & $\,\,$3 $\,\,$ \\
$\,\,$ 1/2$\,\,$ & $\,\,$ 1 $\,\,$ & $\,\,$6$\,\,$ & $\,\,$2 $\,\,$ \\
$\,\,$ 1/4$\,\,$ & $\,\,$ 1/6$\,\,$ & $\,\,$ 1 $\,\,$ & $\,\,$ 1/2 $\,\,$ \\
$\,\,$ 1/3$\,\,$ & $\,\,$ 1/2$\,\,$ & $\,\,$2$\,\,$ & $\,\,$ 1  $\,\,$ \\
\end{pmatrix},
\qquad
\lambda_{\max} =
4.1031,
\qquad
CR = 0.0389
\end{equation*}

\begin{equation*}
\mathbf{w}^{EM} =
\begin{pmatrix}
0.451852\\
0.320085\\
0.077805\\
\color{red} 0.150257\color{black}
\end{pmatrix}\end{equation*}
\begin{equation*}
\left[ \frac{{w}^{EM}_i}{{w}^{EM}_j} \right] =
\begin{pmatrix}
$\,\,$ 1 $\,\,$ & $\,\,$1.4117$\,\,$ & $\,\,$5.8075$\,\,$ & $\,\,$\color{red} 3.0072\color{black} $\,\,$ \\
$\,\,$0.7084$\,\,$ & $\,\,$ 1 $\,\,$ & $\,\,$4.1140$\,\,$ & $\,\,$\color{red} 2.1302\color{black}   $\,\,$ \\
$\,\,$0.1722$\,\,$ & $\,\,$0.2431$\,\,$ & $\,\,$ 1 $\,\,$ & $\,\,$\color{red} 0.5178\color{black}  $\,\,$ \\
$\,\,$\color{red} 0.3325\color{black} $\,\,$ & $\,\,$\color{red} 0.4694\color{black} $\,\,$ & $\,\,$\color{red} 1.9312\color{black} $\,\,$ & $\,\,$ 1  $\,\,$ \\
\end{pmatrix},
\end{equation*}

\begin{equation*}
\mathbf{w}^{\prime} =
\begin{pmatrix}
0.451690\\
0.319970\\
0.077777\\
0.150563
\end{pmatrix} =
0.999640\cdot
\begin{pmatrix}
0.451852\\
0.320085\\
0.077805\\
\color{gr} 0.150617\color{black}
\end{pmatrix},
\end{equation*}
\begin{equation*}
\left[ \frac{{w}^{\prime}_i}{{w}^{\prime}_j} \right] =
\begin{pmatrix}
$\,\,$ 1 $\,\,$ & $\,\,$1.4117$\,\,$ & $\,\,$5.8075$\,\,$ & $\,\,$\color{gr} \color{blue} 3\color{black} $\,\,$ \\
$\,\,$0.7084$\,\,$ & $\,\,$ 1 $\,\,$ & $\,\,$4.1140$\,\,$ & $\,\,$\color{gr} 2.1252\color{black}   $\,\,$ \\
$\,\,$0.1722$\,\,$ & $\,\,$0.2431$\,\,$ & $\,\,$ 1 $\,\,$ & $\,\,$\color{gr} 0.5166\color{black}  $\,\,$ \\
$\,\,$\color{gr} \color{blue}  1/3\color{black} $\,\,$ & $\,\,$\color{gr} 0.4706\color{black} $\,\,$ & $\,\,$\color{gr} 1.9358\color{black} $\,\,$ & $\,\,$ 1  $\,\,$ \\
\end{pmatrix},
\end{equation*}
\end{example}
\newpage
\begin{example}
\begin{equation*}
\mathbf{A} =
\begin{pmatrix}
$\,\,$ 1 $\,\,$ & $\,\,$2$\,\,$ & $\,\,$4$\,\,$ & $\,\,$3 $\,\,$ \\
$\,\,$ 1/2$\,\,$ & $\,\,$ 1 $\,\,$ & $\,\,$7$\,\,$ & $\,\,$2 $\,\,$ \\
$\,\,$ 1/4$\,\,$ & $\,\,$ 1/7$\,\,$ & $\,\,$ 1 $\,\,$ & $\,\,$ 1/2 $\,\,$ \\
$\,\,$ 1/3$\,\,$ & $\,\,$ 1/2$\,\,$ & $\,\,$2$\,\,$ & $\,\,$ 1  $\,\,$ \\
\end{pmatrix},
\qquad
\lambda_{\max} =
4.1365,
\qquad
CR = 0.0515
\end{equation*}

\begin{equation*}
\mathbf{w}^{EM} =
\begin{pmatrix}
0.447021\\
0.331100\\
0.074246\\
\color{red} 0.147632\color{black}
\end{pmatrix}\end{equation*}
\begin{equation*}
\left[ \frac{{w}^{EM}_i}{{w}^{EM}_j} \right] =
\begin{pmatrix}
$\,\,$ 1 $\,\,$ & $\,\,$1.3501$\,\,$ & $\,\,$6.0208$\,\,$ & $\,\,$\color{red} 3.0279\color{black} $\,\,$ \\
$\,\,$0.7407$\,\,$ & $\,\,$ 1 $\,\,$ & $\,\,$4.4595$\,\,$ & $\,\,$\color{red} 2.2427\color{black}   $\,\,$ \\
$\,\,$0.1661$\,\,$ & $\,\,$0.2242$\,\,$ & $\,\,$ 1 $\,\,$ & $\,\,$\color{red} 0.5029\color{black}  $\,\,$ \\
$\,\,$\color{red} 0.3303\color{black} $\,\,$ & $\,\,$\color{red} 0.4459\color{black} $\,\,$ & $\,\,$\color{red} 1.9884\color{black} $\,\,$ & $\,\,$ 1  $\,\,$ \\
\end{pmatrix},
\end{equation*}

\begin{equation*}
\mathbf{w}^{\prime} =
\begin{pmatrix}
0.446638\\
0.330816\\
0.074182\\
0.148364
\end{pmatrix} =
0.999142\cdot
\begin{pmatrix}
0.447021\\
0.331100\\
0.074246\\
\color{gr} 0.148492\color{black}
\end{pmatrix},
\end{equation*}
\begin{equation*}
\left[ \frac{{w}^{\prime}_i}{{w}^{\prime}_j} \right] =
\begin{pmatrix}
$\,\,$ 1 $\,\,$ & $\,\,$1.3501$\,\,$ & $\,\,$6.0208$\,\,$ & $\,\,$\color{gr} 3.0104\color{black} $\,\,$ \\
$\,\,$0.7407$\,\,$ & $\,\,$ 1 $\,\,$ & $\,\,$4.4595$\,\,$ & $\,\,$\color{gr} 2.2298\color{black}   $\,\,$ \\
$\,\,$0.1661$\,\,$ & $\,\,$0.2242$\,\,$ & $\,\,$ 1 $\,\,$ & $\,\,$\color{gr} \color{blue}  1/2\color{black}  $\,\,$ \\
$\,\,$\color{gr} 0.3322\color{black} $\,\,$ & $\,\,$\color{gr} 0.4485\color{black} $\,\,$ & $\,\,$\color{gr} \color{blue} 2\color{black} $\,\,$ & $\,\,$ 1  $\,\,$ \\
\end{pmatrix},
\end{equation*}
\end{example}
\newpage
\begin{example}
\begin{equation*}
\mathbf{A} =
\begin{pmatrix}
$\,\,$ 1 $\,\,$ & $\,\,$2$\,\,$ & $\,\,$5$\,\,$ & $\,\,$4 $\,\,$ \\
$\,\,$ 1/2$\,\,$ & $\,\,$ 1 $\,\,$ & $\,\,$8$\,\,$ & $\,\,$3 $\,\,$ \\
$\,\,$ 1/5$\,\,$ & $\,\,$ 1/8$\,\,$ & $\,\,$ 1 $\,\,$ & $\,\,$ 1/2 $\,\,$ \\
$\,\,$ 1/4$\,\,$ & $\,\,$ 1/3$\,\,$ & $\,\,$2$\,\,$ & $\,\,$ 1  $\,\,$ \\
\end{pmatrix},
\qquad
\lambda_{\max} =
4.1163,
\qquad
CR = 0.0439
\end{equation*}

\begin{equation*}
\mathbf{w}^{EM} =
\begin{pmatrix}
0.472959\\
0.348577\\
0.062881\\
\color{red} 0.115583\color{black}
\end{pmatrix}\end{equation*}
\begin{equation*}
\left[ \frac{{w}^{EM}_i}{{w}^{EM}_j} \right] =
\begin{pmatrix}
$\,\,$ 1 $\,\,$ & $\,\,$1.3568$\,\,$ & $\,\,$7.5215$\,\,$ & $\,\,$\color{red} 4.0919\color{black} $\,\,$ \\
$\,\,$0.7370$\,\,$ & $\,\,$ 1 $\,\,$ & $\,\,$5.5435$\,\,$ & $\,\,$\color{red} 3.0158\color{black}   $\,\,$ \\
$\,\,$0.1330$\,\,$ & $\,\,$0.1804$\,\,$ & $\,\,$ 1 $\,\,$ & $\,\,$\color{red} 0.5440\color{black}  $\,\,$ \\
$\,\,$\color{red} 0.2444\color{black} $\,\,$ & $\,\,$\color{red} 0.3316\color{black} $\,\,$ & $\,\,$\color{red} 1.8381\color{black} $\,\,$ & $\,\,$ 1  $\,\,$ \\
\end{pmatrix},
\end{equation*}

\begin{equation*}
\mathbf{w}^{\prime} =
\begin{pmatrix}
0.472672\\
0.348365\\
0.062842\\
0.116122
\end{pmatrix} =
0.999391\cdot
\begin{pmatrix}
0.472959\\
0.348577\\
0.062881\\
\color{gr} 0.116192\color{black}
\end{pmatrix},
\end{equation*}
\begin{equation*}
\left[ \frac{{w}^{\prime}_i}{{w}^{\prime}_j} \right] =
\begin{pmatrix}
$\,\,$ 1 $\,\,$ & $\,\,$1.3568$\,\,$ & $\,\,$7.5215$\,\,$ & $\,\,$\color{gr} 4.0705\color{black} $\,\,$ \\
$\,\,$0.7370$\,\,$ & $\,\,$ 1 $\,\,$ & $\,\,$5.5435$\,\,$ & $\,\,$\color{gr} \color{blue} 3\color{black}   $\,\,$ \\
$\,\,$0.1330$\,\,$ & $\,\,$0.1804$\,\,$ & $\,\,$ 1 $\,\,$ & $\,\,$\color{gr} 0.5412\color{black}  $\,\,$ \\
$\,\,$\color{gr} 0.2457\color{black} $\,\,$ & $\,\,$\color{gr} \color{blue}  1/3\color{black} $\,\,$ & $\,\,$\color{gr} 1.8478\color{black} $\,\,$ & $\,\,$ 1  $\,\,$ \\
\end{pmatrix},
\end{equation*}
\end{example}
\newpage
\begin{example}
\begin{equation*}
\mathbf{A} =
\begin{pmatrix}
$\,\,$ 1 $\,\,$ & $\,\,$2$\,\,$ & $\,\,$5$\,\,$ & $\,\,$4 $\,\,$ \\
$\,\,$ 1/2$\,\,$ & $\,\,$ 1 $\,\,$ & $\,\,$9$\,\,$ & $\,\,$3 $\,\,$ \\
$\,\,$ 1/5$\,\,$ & $\,\,$ 1/9$\,\,$ & $\,\,$ 1 $\,\,$ & $\,\,$ 1/2 $\,\,$ \\
$\,\,$ 1/4$\,\,$ & $\,\,$ 1/3$\,\,$ & $\,\,$2$\,\,$ & $\,\,$ 1  $\,\,$ \\
\end{pmatrix},
\qquad
\lambda_{\max} =
4.1406,
\qquad
CR = 0.0530
\end{equation*}

\begin{equation*}
\mathbf{w}^{EM} =
\begin{pmatrix}
0.468689\\
0.356941\\
0.060590\\
\color{red} 0.113780\color{black}
\end{pmatrix}\end{equation*}
\begin{equation*}
\left[ \frac{{w}^{EM}_i}{{w}^{EM}_j} \right] =
\begin{pmatrix}
$\,\,$ 1 $\,\,$ & $\,\,$1.3131$\,\,$ & $\,\,$7.7354$\,\,$ & $\,\,$\color{red} 4.1193\color{black} $\,\,$ \\
$\,\,$0.7616$\,\,$ & $\,\,$ 1 $\,\,$ & $\,\,$5.8911$\,\,$ & $\,\,$\color{red} 3.1371\color{black}   $\,\,$ \\
$\,\,$0.1293$\,\,$ & $\,\,$0.1697$\,\,$ & $\,\,$ 1 $\,\,$ & $\,\,$\color{red} 0.5325\color{black}  $\,\,$ \\
$\,\,$\color{red} 0.2428\color{black} $\,\,$ & $\,\,$\color{red} 0.3188\color{black} $\,\,$ & $\,\,$\color{red} 1.8779\color{black} $\,\,$ & $\,\,$ 1  $\,\,$ \\
\end{pmatrix},
\end{equation*}

\begin{equation*}
\mathbf{w}^{\prime} =
\begin{pmatrix}
0.467105\\
0.355734\\
0.060385\\
0.116776
\end{pmatrix} =
0.996619\cdot
\begin{pmatrix}
0.468689\\
0.356941\\
0.060590\\
\color{gr} 0.117172\color{black}
\end{pmatrix},
\end{equation*}
\begin{equation*}
\left[ \frac{{w}^{\prime}_i}{{w}^{\prime}_j} \right] =
\begin{pmatrix}
$\,\,$ 1 $\,\,$ & $\,\,$1.3131$\,\,$ & $\,\,$7.7354$\,\,$ & $\,\,$\color{gr} \color{blue} 4\color{black} $\,\,$ \\
$\,\,$0.7616$\,\,$ & $\,\,$ 1 $\,\,$ & $\,\,$5.8911$\,\,$ & $\,\,$\color{gr} 3.0463\color{black}   $\,\,$ \\
$\,\,$0.1293$\,\,$ & $\,\,$0.1697$\,\,$ & $\,\,$ 1 $\,\,$ & $\,\,$\color{gr} 0.5171\color{black}  $\,\,$ \\
$\,\,$\color{gr} \color{blue}  1/4\color{black} $\,\,$ & $\,\,$\color{gr} 0.3283\color{black} $\,\,$ & $\,\,$\color{gr} 1.9338\color{black} $\,\,$ & $\,\,$ 1  $\,\,$ \\
\end{pmatrix},
\end{equation*}
\end{example}
\newpage
\begin{example}
\begin{equation*}
\mathbf{A} =
\begin{pmatrix}
$\,\,$ 1 $\,\,$ & $\,\,$2$\,\,$ & $\,\,$6$\,\,$ & $\,\,$2 $\,\,$ \\
$\,\,$ 1/2$\,\,$ & $\,\,$ 1 $\,\,$ & $\,\,$2$\,\,$ & $\,\,$2 $\,\,$ \\
$\,\,$ 1/6$\,\,$ & $\,\,$ 1/2$\,\,$ & $\,\,$ 1 $\,\,$ & $\,\,$ 1/5 $\,\,$ \\
$\,\,$ 1/2$\,\,$ & $\,\,$ 1/2$\,\,$ & $\,\,$5$\,\,$ & $\,\,$ 1  $\,\,$ \\
\end{pmatrix},
\qquad
\lambda_{\max} =
4.2277,
\qquad
CR = 0.0859
\end{equation*}

\begin{equation*}
\mathbf{w}^{EM} =
\begin{pmatrix}
\color{red} 0.441083\color{black} \\
0.256015\\
0.076465\\
0.226437
\end{pmatrix}\end{equation*}
\begin{equation*}
\left[ \frac{{w}^{EM}_i}{{w}^{EM}_j} \right] =
\begin{pmatrix}
$\,\,$ 1 $\,\,$ & $\,\,$\color{red} 1.7229\color{black} $\,\,$ & $\,\,$\color{red} 5.7684\color{black} $\,\,$ & $\,\,$\color{red} 1.9479\color{black} $\,\,$ \\
$\,\,$\color{red} 0.5804\color{black} $\,\,$ & $\,\,$ 1 $\,\,$ & $\,\,$3.3481$\,\,$ & $\,\,$1.1306  $\,\,$ \\
$\,\,$\color{red} 0.1734\color{black} $\,\,$ & $\,\,$0.2987$\,\,$ & $\,\,$ 1 $\,\,$ & $\,\,$0.3377 $\,\,$ \\
$\,\,$\color{red} 0.5134\color{black} $\,\,$ & $\,\,$0.8845$\,\,$ & $\,\,$2.9613$\,\,$ & $\,\,$ 1  $\,\,$ \\
\end{pmatrix},
\end{equation*}

\begin{equation*}
\mathbf{w}^{\prime} =
\begin{pmatrix}
0.447596\\
0.253032\\
0.075574\\
0.223798
\end{pmatrix} =
0.988347\cdot
\begin{pmatrix}
\color{gr} 0.452873\color{black} \\
0.256015\\
0.076465\\
0.226437
\end{pmatrix},
\end{equation*}
\begin{equation*}
\left[ \frac{{w}^{\prime}_i}{{w}^{\prime}_j} \right] =
\begin{pmatrix}
$\,\,$ 1 $\,\,$ & $\,\,$\color{gr} 1.7689\color{black} $\,\,$ & $\,\,$\color{gr} 5.9226\color{black} $\,\,$ & $\,\,$\color{gr} \color{blue} 2\color{black} $\,\,$ \\
$\,\,$\color{gr} 0.5653\color{black} $\,\,$ & $\,\,$ 1 $\,\,$ & $\,\,$3.3481$\,\,$ & $\,\,$1.1306  $\,\,$ \\
$\,\,$\color{gr} 0.1688\color{black} $\,\,$ & $\,\,$0.2987$\,\,$ & $\,\,$ 1 $\,\,$ & $\,\,$0.3377 $\,\,$ \\
$\,\,$\color{gr} \color{blue}  1/2\color{black} $\,\,$ & $\,\,$0.8845$\,\,$ & $\,\,$2.9613$\,\,$ & $\,\,$ 1  $\,\,$ \\
\end{pmatrix},
\end{equation*}
\end{example}
\newpage
\begin{example}
\begin{equation*}
\mathbf{A} =
\begin{pmatrix}
$\,\,$ 1 $\,\,$ & $\,\,$2$\,\,$ & $\,\,$6$\,\,$ & $\,\,$2 $\,\,$ \\
$\,\,$ 1/2$\,\,$ & $\,\,$ 1 $\,\,$ & $\,\,$4$\,\,$ & $\,\,$3 $\,\,$ \\
$\,\,$ 1/6$\,\,$ & $\,\,$ 1/4$\,\,$ & $\,\,$ 1 $\,\,$ & $\,\,$ 1/2 $\,\,$ \\
$\,\,$ 1/2$\,\,$ & $\,\,$ 1/3$\,\,$ & $\,\,$2$\,\,$ & $\,\,$ 1  $\,\,$ \\
\end{pmatrix},
\qquad
\lambda_{\max} =
4.1031,
\qquad
CR = 0.0389
\end{equation*}

\begin{equation*}
\mathbf{w}^{EM} =
\begin{pmatrix}
0.450646\\
0.319231\\
\color{red} 0.074928\color{black} \\
0.155194
\end{pmatrix}\end{equation*}
\begin{equation*}
\left[ \frac{{w}^{EM}_i}{{w}^{EM}_j} \right] =
\begin{pmatrix}
$\,\,$ 1 $\,\,$ & $\,\,$1.4117$\,\,$ & $\,\,$\color{red} 6.0144\color{black} $\,\,$ & $\,\,$2.9038$\,\,$ \\
$\,\,$0.7084$\,\,$ & $\,\,$ 1 $\,\,$ & $\,\,$\color{red} 4.2605\color{black} $\,\,$ & $\,\,$2.0570  $\,\,$ \\
$\,\,$\color{red} 0.1663\color{black} $\,\,$ & $\,\,$\color{red} 0.2347\color{black} $\,\,$ & $\,\,$ 1 $\,\,$ & $\,\,$\color{red} 0.4828\color{black}  $\,\,$ \\
$\,\,$0.3444$\,\,$ & $\,\,$0.4862$\,\,$ & $\,\,$\color{red} 2.0712\color{black} $\,\,$ & $\,\,$ 1  $\,\,$ \\
\end{pmatrix},
\end{equation*}

\begin{equation*}
\mathbf{w}^{\prime} =
\begin{pmatrix}
0.450566\\
0.319174\\
0.075094\\
0.155166
\end{pmatrix} =
0.999820\cdot
\begin{pmatrix}
0.450646\\
0.319231\\
\color{gr} 0.075108\color{black} \\
0.155194
\end{pmatrix},
\end{equation*}
\begin{equation*}
\left[ \frac{{w}^{\prime}_i}{{w}^{\prime}_j} \right] =
\begin{pmatrix}
$\,\,$ 1 $\,\,$ & $\,\,$1.4117$\,\,$ & $\,\,$\color{gr} \color{blue} 6\color{black} $\,\,$ & $\,\,$2.9038$\,\,$ \\
$\,\,$0.7084$\,\,$ & $\,\,$ 1 $\,\,$ & $\,\,$\color{gr} 4.2503\color{black} $\,\,$ & $\,\,$2.0570  $\,\,$ \\
$\,\,$\color{gr} \color{blue}  1/6\color{black} $\,\,$ & $\,\,$\color{gr} 0.2353\color{black} $\,\,$ & $\,\,$ 1 $\,\,$ & $\,\,$\color{gr} 0.4840\color{black}  $\,\,$ \\
$\,\,$0.3444$\,\,$ & $\,\,$0.4862$\,\,$ & $\,\,$\color{gr} 2.0663\color{black} $\,\,$ & $\,\,$ 1  $\,\,$ \\
\end{pmatrix},
\end{equation*}
\end{example}
\newpage
\begin{example}
\begin{equation*}
\mathbf{A} =
\begin{pmatrix}
$\,\,$ 1 $\,\,$ & $\,\,$2$\,\,$ & $\,\,$6$\,\,$ & $\,\,$2 $\,\,$ \\
$\,\,$ 1/2$\,\,$ & $\,\,$ 1 $\,\,$ & $\,\,$5$\,\,$ & $\,\,$4 $\,\,$ \\
$\,\,$ 1/6$\,\,$ & $\,\,$ 1/5$\,\,$ & $\,\,$ 1 $\,\,$ & $\,\,$ 1/2 $\,\,$ \\
$\,\,$ 1/2$\,\,$ & $\,\,$ 1/4$\,\,$ & $\,\,$2$\,\,$ & $\,\,$ 1  $\,\,$ \\
\end{pmatrix},
\qquad
\lambda_{\max} =
4.1655,
\qquad
CR = 0.0624
\end{equation*}

\begin{equation*}
\mathbf{w}^{EM} =
\begin{pmatrix}
0.439490\\
0.352986\\
\color{red} 0.067550\color{black} \\
0.139974
\end{pmatrix}\end{equation*}
\begin{equation*}
\left[ \frac{{w}^{EM}_i}{{w}^{EM}_j} \right] =
\begin{pmatrix}
$\,\,$ 1 $\,\,$ & $\,\,$1.2451$\,\,$ & $\,\,$\color{red} 6.5061\color{black} $\,\,$ & $\,\,$3.1398$\,\,$ \\
$\,\,$0.8032$\,\,$ & $\,\,$ 1 $\,\,$ & $\,\,$\color{red} 5.2255\color{black} $\,\,$ & $\,\,$2.5218  $\,\,$ \\
$\,\,$\color{red} 0.1537\color{black} $\,\,$ & $\,\,$\color{red} 0.1914\color{black} $\,\,$ & $\,\,$ 1 $\,\,$ & $\,\,$\color{red} 0.4826\color{black}  $\,\,$ \\
$\,\,$0.3185$\,\,$ & $\,\,$0.3965$\,\,$ & $\,\,$\color{red} 2.0721\color{black} $\,\,$ & $\,\,$ 1  $\,\,$ \\
\end{pmatrix},
\end{equation*}

\begin{equation*}
\mathbf{w}^{\prime} =
\begin{pmatrix}
0.438422\\
0.352128\\
0.069817\\
0.139633
\end{pmatrix} =
0.997569\cdot
\begin{pmatrix}
0.439490\\
0.352986\\
\color{gr} 0.069987\color{black} \\
0.139974
\end{pmatrix},
\end{equation*}
\begin{equation*}
\left[ \frac{{w}^{\prime}_i}{{w}^{\prime}_j} \right] =
\begin{pmatrix}
$\,\,$ 1 $\,\,$ & $\,\,$1.2451$\,\,$ & $\,\,$\color{gr} 6.2796\color{black} $\,\,$ & $\,\,$3.1398$\,\,$ \\
$\,\,$0.8032$\,\,$ & $\,\,$ 1 $\,\,$ & $\,\,$\color{gr} 5.0436\color{black} $\,\,$ & $\,\,$2.5218  $\,\,$ \\
$\,\,$\color{gr} 0.1592\color{black} $\,\,$ & $\,\,$\color{gr} 0.1983\color{black} $\,\,$ & $\,\,$ 1 $\,\,$ & $\,\,$\color{gr} \color{blue}  1/2\color{black}  $\,\,$ \\
$\,\,$0.3185$\,\,$ & $\,\,$0.3965$\,\,$ & $\,\,$\color{gr} \color{blue} 2\color{black} $\,\,$ & $\,\,$ 1  $\,\,$ \\
\end{pmatrix},
\end{equation*}
\end{example}
\newpage
\begin{example}
\begin{equation*}
\mathbf{A} =
\begin{pmatrix}
$\,\,$ 1 $\,\,$ & $\,\,$2$\,\,$ & $\,\,$6$\,\,$ & $\,\,$2 $\,\,$ \\
$\,\,$ 1/2$\,\,$ & $\,\,$ 1 $\,\,$ & $\,\,$6$\,\,$ & $\,\,$5 $\,\,$ \\
$\,\,$ 1/6$\,\,$ & $\,\,$ 1/6$\,\,$ & $\,\,$ 1 $\,\,$ & $\,\,$ 1/2 $\,\,$ \\
$\,\,$ 1/2$\,\,$ & $\,\,$ 1/5$\,\,$ & $\,\,$2$\,\,$ & $\,\,$ 1  $\,\,$ \\
\end{pmatrix},
\qquad
\lambda_{\max} =
4.2277,
\qquad
CR = 0.0859
\end{equation*}

\begin{equation*}
\mathbf{w}^{EM} =
\begin{pmatrix}
0.429790\\
0.380135\\
\color{red} 0.061706\color{black} \\
0.128368
\end{pmatrix}\end{equation*}
\begin{equation*}
\left[ \frac{{w}^{EM}_i}{{w}^{EM}_j} \right] =
\begin{pmatrix}
$\,\,$ 1 $\,\,$ & $\,\,$1.1306$\,\,$ & $\,\,$\color{red} 6.9651\color{black} $\,\,$ & $\,\,$3.3481$\,\,$ \\
$\,\,$0.8845$\,\,$ & $\,\,$ 1 $\,\,$ & $\,\,$\color{red} 6.1604\color{black} $\,\,$ & $\,\,$2.9613  $\,\,$ \\
$\,\,$\color{red} 0.1436\color{black} $\,\,$ & $\,\,$\color{red} 0.1623\color{black} $\,\,$ & $\,\,$ 1 $\,\,$ & $\,\,$\color{red} 0.4807\color{black}  $\,\,$ \\
$\,\,$0.2987$\,\,$ & $\,\,$0.3377$\,\,$ & $\,\,$\color{red} 2.0803\color{black} $\,\,$ & $\,\,$ 1  $\,\,$ \\
\end{pmatrix},
\end{equation*}

\begin{equation*}
\mathbf{w}^{\prime} =
\begin{pmatrix}
0.429083\\
0.379509\\
0.063252\\
0.128156
\end{pmatrix} =
0.998353\cdot
\begin{pmatrix}
0.429790\\
0.380135\\
\color{gr} 0.063356\color{black} \\
0.128368
\end{pmatrix},
\end{equation*}
\begin{equation*}
\left[ \frac{{w}^{\prime}_i}{{w}^{\prime}_j} \right] =
\begin{pmatrix}
$\,\,$ 1 $\,\,$ & $\,\,$1.1306$\,\,$ & $\,\,$\color{gr} 6.7837\color{black} $\,\,$ & $\,\,$3.3481$\,\,$ \\
$\,\,$0.8845$\,\,$ & $\,\,$ 1 $\,\,$ & $\,\,$\color{gr} \color{blue} 6\color{black} $\,\,$ & $\,\,$2.9613  $\,\,$ \\
$\,\,$\color{gr} 0.1474\color{black} $\,\,$ & $\,\,$\color{gr} \color{blue}  1/6\color{black} $\,\,$ & $\,\,$ 1 $\,\,$ & $\,\,$\color{gr} 0.4936\color{black}  $\,\,$ \\
$\,\,$0.2987$\,\,$ & $\,\,$0.3377$\,\,$ & $\,\,$\color{gr} 2.0261\color{black} $\,\,$ & $\,\,$ 1  $\,\,$ \\
\end{pmatrix},
\end{equation*}
\end{example}
\newpage
\begin{example}
\begin{equation*}
\mathbf{A} =
\begin{pmatrix}
$\,\,$ 1 $\,\,$ & $\,\,$2$\,\,$ & $\,\,$6$\,\,$ & $\,\,$3 $\,\,$ \\
$\,\,$ 1/2$\,\,$ & $\,\,$ 1 $\,\,$ & $\,\,$9$\,\,$ & $\,\,$2 $\,\,$ \\
$\,\,$ 1/6$\,\,$ & $\,\,$ 1/9$\,\,$ & $\,\,$ 1 $\,\,$ & $\,\,$ 1/3 $\,\,$ \\
$\,\,$ 1/3$\,\,$ & $\,\,$ 1/2$\,\,$ & $\,\,$3$\,\,$ & $\,\,$ 1  $\,\,$ \\
\end{pmatrix},
\qquad
\lambda_{\max} =
4.1031,
\qquad
CR = 0.0389
\end{equation*}

\begin{equation*}
\mathbf{w}^{EM} =
\begin{pmatrix}
0.463883\\
0.328608\\
0.053251\\
\color{red} 0.154258\color{black}
\end{pmatrix}\end{equation*}
\begin{equation*}
\left[ \frac{{w}^{EM}_i}{{w}^{EM}_j} \right] =
\begin{pmatrix}
$\,\,$ 1 $\,\,$ & $\,\,$1.4117$\,\,$ & $\,\,$8.7113$\,\,$ & $\,\,$\color{red} 3.0072\color{black} $\,\,$ \\
$\,\,$0.7084$\,\,$ & $\,\,$ 1 $\,\,$ & $\,\,$6.1709$\,\,$ & $\,\,$\color{red} 2.1302\color{black}   $\,\,$ \\
$\,\,$0.1148$\,\,$ & $\,\,$0.1621$\,\,$ & $\,\,$ 1 $\,\,$ & $\,\,$\color{red} 0.3452\color{black}  $\,\,$ \\
$\,\,$\color{red} 0.3325\color{black} $\,\,$ & $\,\,$\color{red} 0.4694\color{black} $\,\,$ & $\,\,$\color{red} 2.8968\color{black} $\,\,$ & $\,\,$ 1  $\,\,$ \\
\end{pmatrix},
\end{equation*}

\begin{equation*}
\mathbf{w}^{\prime} =
\begin{pmatrix}
0.463712\\
0.328486\\
0.053231\\
0.154571
\end{pmatrix} =
0.999630\cdot
\begin{pmatrix}
0.463883\\
0.328608\\
0.053251\\
\color{gr} 0.154628\color{black}
\end{pmatrix},
\end{equation*}
\begin{equation*}
\left[ \frac{{w}^{\prime}_i}{{w}^{\prime}_j} \right] =
\begin{pmatrix}
$\,\,$ 1 $\,\,$ & $\,\,$1.4117$\,\,$ & $\,\,$8.7113$\,\,$ & $\,\,$\color{gr} \color{blue} 3\color{black} $\,\,$ \\
$\,\,$0.7084$\,\,$ & $\,\,$ 1 $\,\,$ & $\,\,$6.1709$\,\,$ & $\,\,$\color{gr} 2.1252\color{black}   $\,\,$ \\
$\,\,$0.1148$\,\,$ & $\,\,$0.1621$\,\,$ & $\,\,$ 1 $\,\,$ & $\,\,$\color{gr} 0.3444\color{black}  $\,\,$ \\
$\,\,$\color{gr} \color{blue}  1/3\color{black} $\,\,$ & $\,\,$\color{gr} 0.4706\color{black} $\,\,$ & $\,\,$\color{gr} 2.9038\color{black} $\,\,$ & $\,\,$ 1  $\,\,$ \\
\end{pmatrix},
\end{equation*}
\end{example}
\newpage
\begin{example}
\begin{equation*}
\mathbf{A} =
\begin{pmatrix}
$\,\,$ 1 $\,\,$ & $\,\,$2$\,\,$ & $\,\,$6$\,\,$ & $\,\,$4 $\,\,$ \\
$\,\,$ 1/2$\,\,$ & $\,\,$ 1 $\,\,$ & $\,\,$9$\,\,$ & $\,\,$3 $\,\,$ \\
$\,\,$ 1/6$\,\,$ & $\,\,$ 1/9$\,\,$ & $\,\,$ 1 $\,\,$ & $\,\,$ 1/2 $\,\,$ \\
$\,\,$ 1/4$\,\,$ & $\,\,$ 1/3$\,\,$ & $\,\,$2$\,\,$ & $\,\,$ 1  $\,\,$ \\
\end{pmatrix},
\qquad
\lambda_{\max} =
4.1031,
\qquad
CR = 0.0389
\end{equation*}

\begin{equation*}
\mathbf{w}^{EM} =
\begin{pmatrix}
0.480405\\
0.350323\\
0.056514\\
\color{red} 0.112758\color{black}
\end{pmatrix}\end{equation*}
\begin{equation*}
\left[ \frac{{w}^{EM}_i}{{w}^{EM}_j} \right] =
\begin{pmatrix}
$\,\,$ 1 $\,\,$ & $\,\,$1.3713$\,\,$ & $\,\,$8.5006$\,\,$ & $\,\,$\color{red} 4.2605\color{black} $\,\,$ \\
$\,\,$0.7292$\,\,$ & $\,\,$ 1 $\,\,$ & $\,\,$6.1989$\,\,$ & $\,\,$\color{red} 3.1069\color{black}   $\,\,$ \\
$\,\,$0.1176$\,\,$ & $\,\,$0.1613$\,\,$ & $\,\,$ 1 $\,\,$ & $\,\,$\color{red} 0.5012\color{black}  $\,\,$ \\
$\,\,$\color{red} 0.2347\color{black} $\,\,$ & $\,\,$\color{red} 0.3219\color{black} $\,\,$ & $\,\,$\color{red} 1.9952\color{black} $\,\,$ & $\,\,$ 1  $\,\,$ \\
\end{pmatrix},
\end{equation*}

\begin{equation*}
\mathbf{w}^{\prime} =
\begin{pmatrix}
0.480275\\
0.350229\\
0.056499\\
0.112998
\end{pmatrix} =
0.999730\cdot
\begin{pmatrix}
0.480405\\
0.350323\\
0.056514\\
\color{gr} 0.113028\color{black}
\end{pmatrix},
\end{equation*}
\begin{equation*}
\left[ \frac{{w}^{\prime}_i}{{w}^{\prime}_j} \right] =
\begin{pmatrix}
$\,\,$ 1 $\,\,$ & $\,\,$1.3713$\,\,$ & $\,\,$8.5006$\,\,$ & $\,\,$\color{gr} 4.2503\color{black} $\,\,$ \\
$\,\,$0.7292$\,\,$ & $\,\,$ 1 $\,\,$ & $\,\,$6.1989$\,\,$ & $\,\,$\color{gr} 3.0994\color{black}   $\,\,$ \\
$\,\,$0.1176$\,\,$ & $\,\,$0.1613$\,\,$ & $\,\,$ 1 $\,\,$ & $\,\,$\color{gr} \color{blue}  1/2\color{black}  $\,\,$ \\
$\,\,$\color{gr} 0.2353\color{black} $\,\,$ & $\,\,$\color{gr} 0.3226\color{black} $\,\,$ & $\,\,$\color{gr} \color{blue} 2\color{black} $\,\,$ & $\,\,$ 1  $\,\,$ \\
\end{pmatrix},
\end{equation*}
\end{example}
\newpage
\begin{example}
\begin{equation*}
\mathbf{A} =
\begin{pmatrix}
$\,\,$ 1 $\,\,$ & $\,\,$2$\,\,$ & $\,\,$6$\,\,$ & $\,\,$5 $\,\,$ \\
$\,\,$ 1/2$\,\,$ & $\,\,$ 1 $\,\,$ & $\,\,$2$\,\,$ & $\,\,$5 $\,\,$ \\
$\,\,$ 1/6$\,\,$ & $\,\,$ 1/2$\,\,$ & $\,\,$ 1 $\,\,$ & $\,\,$ 1/2 $\,\,$ \\
$\,\,$ 1/5$\,\,$ & $\,\,$ 1/5$\,\,$ & $\,\,$2$\,\,$ & $\,\,$ 1  $\,\,$ \\
\end{pmatrix},
\qquad
\lambda_{\max} =
4.2277,
\qquad
CR = 0.0859
\end{equation*}

\begin{equation*}
\mathbf{w}^{EM} =
\begin{pmatrix}
\color{red} 0.510431\color{black} \\
0.296266\\
0.088487\\
0.104815
\end{pmatrix}\end{equation*}
\begin{equation*}
\left[ \frac{{w}^{EM}_i}{{w}^{EM}_j} \right] =
\begin{pmatrix}
$\,\,$ 1 $\,\,$ & $\,\,$\color{red} 1.7229\color{black} $\,\,$ & $\,\,$\color{red} 5.7684\color{black} $\,\,$ & $\,\,$\color{red} 4.8698\color{black} $\,\,$ \\
$\,\,$\color{red} 0.5804\color{black} $\,\,$ & $\,\,$ 1 $\,\,$ & $\,\,$3.3481$\,\,$ & $\,\,$2.8266  $\,\,$ \\
$\,\,$\color{red} 0.1734\color{black} $\,\,$ & $\,\,$0.2987$\,\,$ & $\,\,$ 1 $\,\,$ & $\,\,$0.8442 $\,\,$ \\
$\,\,$\color{red} 0.2053\color{black} $\,\,$ & $\,\,$0.3538$\,\,$ & $\,\,$1.1845$\,\,$ & $\,\,$ 1  $\,\,$ \\
\end{pmatrix},
\end{equation*}

\begin{equation*}
\mathbf{w}^{\prime} =
\begin{pmatrix}
0.517021\\
0.292278\\
0.087296\\
0.103404
\end{pmatrix} =
0.986540\cdot
\begin{pmatrix}
\color{gr} 0.524075\color{black} \\
0.296266\\
0.088487\\
0.104815
\end{pmatrix},
\end{equation*}
\begin{equation*}
\left[ \frac{{w}^{\prime}_i}{{w}^{\prime}_j} \right] =
\begin{pmatrix}
$\,\,$ 1 $\,\,$ & $\,\,$\color{gr} 1.7689\color{black} $\,\,$ & $\,\,$\color{gr} 5.9226\color{black} $\,\,$ & $\,\,$\color{gr} \color{blue} 5\color{black} $\,\,$ \\
$\,\,$\color{gr} 0.5653\color{black} $\,\,$ & $\,\,$ 1 $\,\,$ & $\,\,$3.3481$\,\,$ & $\,\,$2.8266  $\,\,$ \\
$\,\,$\color{gr} 0.1688\color{black} $\,\,$ & $\,\,$0.2987$\,\,$ & $\,\,$ 1 $\,\,$ & $\,\,$0.8442 $\,\,$ \\
$\,\,$\color{gr} \color{blue}  1/5\color{black} $\,\,$ & $\,\,$0.3538$\,\,$ & $\,\,$1.1845$\,\,$ & $\,\,$ 1  $\,\,$ \\
\end{pmatrix},
\end{equation*}
\end{example}
\newpage
\begin{example}
\begin{equation*}
\mathbf{A} =
\begin{pmatrix}
$\,\,$ 1 $\,\,$ & $\,\,$2$\,\,$ & $\,\,$7$\,\,$ & $\,\,$2 $\,\,$ \\
$\,\,$ 1/2$\,\,$ & $\,\,$ 1 $\,\,$ & $\,\,$5$\,\,$ & $\,\,$4 $\,\,$ \\
$\,\,$ 1/7$\,\,$ & $\,\,$ 1/5$\,\,$ & $\,\,$ 1 $\,\,$ & $\,\,$ 1/2 $\,\,$ \\
$\,\,$ 1/2$\,\,$ & $\,\,$ 1/4$\,\,$ & $\,\,$2$\,\,$ & $\,\,$ 1  $\,\,$ \\
\end{pmatrix},
\qquad
\lambda_{\max} =
4.1665,
\qquad
CR = 0.0628
\end{equation*}

\begin{equation*}
\mathbf{w}^{EM} =
\begin{pmatrix}
0.449205\\
0.347722\\
\color{red} 0.064163\color{black} \\
0.138910
\end{pmatrix}\end{equation*}
\begin{equation*}
\left[ \frac{{w}^{EM}_i}{{w}^{EM}_j} \right] =
\begin{pmatrix}
$\,\,$ 1 $\,\,$ & $\,\,$1.2919$\,\,$ & $\,\,$\color{red} 7.0010\color{black} $\,\,$ & $\,\,$3.2338$\,\,$ \\
$\,\,$0.7741$\,\,$ & $\,\,$ 1 $\,\,$ & $\,\,$\color{red} 5.4194\color{black} $\,\,$ & $\,\,$2.5032  $\,\,$ \\
$\,\,$\color{red} 0.1428\color{black} $\,\,$ & $\,\,$\color{red} 0.1845\color{black} $\,\,$ & $\,\,$ 1 $\,\,$ & $\,\,$\color{red} 0.4619\color{black}  $\,\,$ \\
$\,\,$0.3092$\,\,$ & $\,\,$0.3995$\,\,$ & $\,\,$\color{red} 2.1650\color{black} $\,\,$ & $\,\,$ 1  $\,\,$ \\
\end{pmatrix},
\end{equation*}

\begin{equation*}
\mathbf{w}^{\prime} =
\begin{pmatrix}
0.449201\\
0.347719\\
0.064172\\
0.138909
\end{pmatrix} =
0.999991\cdot
\begin{pmatrix}
0.449205\\
0.347722\\
\color{gr} 0.064172\color{black} \\
0.138910
\end{pmatrix},
\end{equation*}
\begin{equation*}
\left[ \frac{{w}^{\prime}_i}{{w}^{\prime}_j} \right] =
\begin{pmatrix}
$\,\,$ 1 $\,\,$ & $\,\,$1.2919$\,\,$ & $\,\,$\color{gr} \color{blue} 7\color{black} $\,\,$ & $\,\,$3.2338$\,\,$ \\
$\,\,$0.7741$\,\,$ & $\,\,$ 1 $\,\,$ & $\,\,$\color{gr} 5.4186\color{black} $\,\,$ & $\,\,$2.5032  $\,\,$ \\
$\,\,$\color{gr} \color{blue}  1/7\color{black} $\,\,$ & $\,\,$\color{gr} 0.1846\color{black} $\,\,$ & $\,\,$ 1 $\,\,$ & $\,\,$\color{gr} 0.4620\color{black}  $\,\,$ \\
$\,\,$0.3092$\,\,$ & $\,\,$0.3995$\,\,$ & $\,\,$\color{gr} 2.1647\color{black} $\,\,$ & $\,\,$ 1  $\,\,$ \\
\end{pmatrix},
\end{equation*}
\end{example}
\newpage
\begin{example}
\begin{equation*}
\mathbf{A} =
\begin{pmatrix}
$\,\,$ 1 $\,\,$ & $\,\,$2$\,\,$ & $\,\,$7$\,\,$ & $\,\,$2 $\,\,$ \\
$\,\,$ 1/2$\,\,$ & $\,\,$ 1 $\,\,$ & $\,\,$5$\,\,$ & $\,\,$5 $\,\,$ \\
$\,\,$ 1/7$\,\,$ & $\,\,$ 1/5$\,\,$ & $\,\,$ 1 $\,\,$ & $\,\,$ 1/2 $\,\,$ \\
$\,\,$ 1/2$\,\,$ & $\,\,$ 1/5$\,\,$ & $\,\,$2$\,\,$ & $\,\,$ 1  $\,\,$ \\
\end{pmatrix},
\qquad
\lambda_{\max} =
4.2287,
\qquad
CR = 0.0862
\end{equation*}

\begin{equation*}
\mathbf{w}^{EM} =
\begin{pmatrix}
0.442095\\
0.365864\\
\color{red} 0.062314\color{black} \\
0.129728
\end{pmatrix}\end{equation*}
\begin{equation*}
\left[ \frac{{w}^{EM}_i}{{w}^{EM}_j} \right] =
\begin{pmatrix}
$\,\,$ 1 $\,\,$ & $\,\,$1.2084$\,\,$ & $\,\,$\color{red} 7.0946\color{black} $\,\,$ & $\,\,$3.4079$\,\,$ \\
$\,\,$0.8276$\,\,$ & $\,\,$ 1 $\,\,$ & $\,\,$\color{red} 5.8713\color{black} $\,\,$ & $\,\,$2.8202  $\,\,$ \\
$\,\,$\color{red} 0.1410\color{black} $\,\,$ & $\,\,$\color{red} 0.1703\color{black} $\,\,$ & $\,\,$ 1 $\,\,$ & $\,\,$\color{red} 0.4803\color{black}  $\,\,$ \\
$\,\,$0.2934$\,\,$ & $\,\,$0.3546$\,\,$ & $\,\,$\color{red} 2.0818\color{black} $\,\,$ & $\,\,$ 1  $\,\,$ \\
\end{pmatrix},
\end{equation*}

\begin{equation*}
\mathbf{w}^{\prime} =
\begin{pmatrix}
0.441723\\
0.365556\\
0.063103\\
0.129618
\end{pmatrix} =
0.999159\cdot
\begin{pmatrix}
0.442095\\
0.365864\\
\color{gr} 0.063156\color{black} \\
0.129728
\end{pmatrix},
\end{equation*}
\begin{equation*}
\left[ \frac{{w}^{\prime}_i}{{w}^{\prime}_j} \right] =
\begin{pmatrix}
$\,\,$ 1 $\,\,$ & $\,\,$1.2084$\,\,$ & $\,\,$\color{gr} \color{blue} 7\color{black} $\,\,$ & $\,\,$3.4079$\,\,$ \\
$\,\,$0.8276$\,\,$ & $\,\,$ 1 $\,\,$ & $\,\,$\color{gr} 5.7930\color{black} $\,\,$ & $\,\,$2.8202  $\,\,$ \\
$\,\,$\color{gr} \color{blue}  1/7\color{black} $\,\,$ & $\,\,$\color{gr} 0.1726\color{black} $\,\,$ & $\,\,$ 1 $\,\,$ & $\,\,$\color{gr} 0.4868\color{black}  $\,\,$ \\
$\,\,$0.2934$\,\,$ & $\,\,$0.3546$\,\,$ & $\,\,$\color{gr} 2.0541\color{black} $\,\,$ & $\,\,$ 1  $\,\,$ \\
\end{pmatrix},
\end{equation*}
\end{example}
\newpage
\begin{example}
\begin{equation*}
\mathbf{A} =
\begin{pmatrix}
$\,\,$ 1 $\,\,$ & $\,\,$2$\,\,$ & $\,\,$7$\,\,$ & $\,\,$2 $\,\,$ \\
$\,\,$ 1/2$\,\,$ & $\,\,$ 1 $\,\,$ & $\,\,$6$\,\,$ & $\,\,$5 $\,\,$ \\
$\,\,$ 1/7$\,\,$ & $\,\,$ 1/6$\,\,$ & $\,\,$ 1 $\,\,$ & $\,\,$ 1/2 $\,\,$ \\
$\,\,$ 1/2$\,\,$ & $\,\,$ 1/5$\,\,$ & $\,\,$2$\,\,$ & $\,\,$ 1  $\,\,$ \\
\end{pmatrix},
\qquad
\lambda_{\max} =
4.2251,
\qquad
CR = 0.0849
\end{equation*}

\begin{equation*}
\mathbf{w}^{EM} =
\begin{pmatrix}
0.438846\\
0.374926\\
\color{red} 0.058601\color{black} \\
0.127628
\end{pmatrix}\end{equation*}
\begin{equation*}
\left[ \frac{{w}^{EM}_i}{{w}^{EM}_j} \right] =
\begin{pmatrix}
$\,\,$ 1 $\,\,$ & $\,\,$1.1705$\,\,$ & $\,\,$\color{red} 7.4887\color{black} $\,\,$ & $\,\,$3.4385$\,\,$ \\
$\,\,$0.8543$\,\,$ & $\,\,$ 1 $\,\,$ & $\,\,$\color{red} 6.3979\color{black} $\,\,$ & $\,\,$2.9377  $\,\,$ \\
$\,\,$\color{red} 0.1335\color{black} $\,\,$ & $\,\,$\color{red} 0.1563\color{black} $\,\,$ & $\,\,$ 1 $\,\,$ & $\,\,$\color{red} 0.4592\color{black}  $\,\,$ \\
$\,\,$0.2908$\,\,$ & $\,\,$0.3404$\,\,$ & $\,\,$\color{red} 2.1779\color{black} $\,\,$ & $\,\,$ 1  $\,\,$ \\
\end{pmatrix},
\end{equation*}

\begin{equation*}
\mathbf{w}^{\prime} =
\begin{pmatrix}
0.437147\\
0.373474\\
0.062246\\
0.127134
\end{pmatrix} =
0.996129\cdot
\begin{pmatrix}
0.438846\\
0.374926\\
\color{gr} 0.062488\color{black} \\
0.127628
\end{pmatrix},
\end{equation*}
\begin{equation*}
\left[ \frac{{w}^{\prime}_i}{{w}^{\prime}_j} \right] =
\begin{pmatrix}
$\,\,$ 1 $\,\,$ & $\,\,$1.1705$\,\,$ & $\,\,$\color{gr} 7.0229\color{black} $\,\,$ & $\,\,$3.4385$\,\,$ \\
$\,\,$0.8543$\,\,$ & $\,\,$ 1 $\,\,$ & $\,\,$\color{gr} \color{blue} 6\color{black} $\,\,$ & $\,\,$2.9377  $\,\,$ \\
$\,\,$\color{gr} 0.1424\color{black} $\,\,$ & $\,\,$\color{gr} \color{blue}  1/6\color{black} $\,\,$ & $\,\,$ 1 $\,\,$ & $\,\,$\color{gr} 0.4896\color{black}  $\,\,$ \\
$\,\,$0.2908$\,\,$ & $\,\,$0.3404$\,\,$ & $\,\,$\color{gr} 2.0424\color{black} $\,\,$ & $\,\,$ 1  $\,\,$ \\
\end{pmatrix},
\end{equation*}
\end{example}
\newpage
\begin{example}
\begin{equation*}
\mathbf{A} =
\begin{pmatrix}
$\,\,$ 1 $\,\,$ & $\,\,$2$\,\,$ & $\,\,$7$\,\,$ & $\,\,$3 $\,\,$ \\
$\,\,$ 1/2$\,\,$ & $\,\,$ 1 $\,\,$ & $\,\,$8$\,\,$ & $\,\,$2 $\,\,$ \\
$\,\,$ 1/7$\,\,$ & $\,\,$ 1/8$\,\,$ & $\,\,$ 1 $\,\,$ & $\,\,$ 1/3 $\,\,$ \\
$\,\,$ 1/3$\,\,$ & $\,\,$ 1/2$\,\,$ & $\,\,$3$\,\,$ & $\,\,$ 1  $\,\,$ \\
\end{pmatrix},
\qquad
\lambda_{\max} =
4.0576,
\qquad
CR = 0.0217
\end{equation*}

\begin{equation*}
\mathbf{w}^{EM} =
\begin{pmatrix}
0.477536\\
0.315605\\
0.052086\\
\color{red} 0.154773\color{black}
\end{pmatrix}\end{equation*}
\begin{equation*}
\left[ \frac{{w}^{EM}_i}{{w}^{EM}_j} \right] =
\begin{pmatrix}
$\,\,$ 1 $\,\,$ & $\,\,$1.5131$\,\,$ & $\,\,$9.1682$\,\,$ & $\,\,$\color{red} 3.0854\color{black} $\,\,$ \\
$\,\,$0.6609$\,\,$ & $\,\,$ 1 $\,\,$ & $\,\,$6.0593$\,\,$ & $\,\,$\color{red} 2.0391\color{black}   $\,\,$ \\
$\,\,$0.1091$\,\,$ & $\,\,$0.1650$\,\,$ & $\,\,$ 1 $\,\,$ & $\,\,$\color{red} 0.3365\color{black}  $\,\,$ \\
$\,\,$\color{red} 0.3241\color{black} $\,\,$ & $\,\,$\color{red} 0.4904\color{black} $\,\,$ & $\,\,$\color{red} 2.9715\color{black} $\,\,$ & $\,\,$ 1  $\,\,$ \\
\end{pmatrix},
\end{equation*}

\begin{equation*}
\mathbf{w}^{\prime} =
\begin{pmatrix}
0.476828\\
0.315136\\
0.052009\\
0.156027
\end{pmatrix} =
0.998516\cdot
\begin{pmatrix}
0.477536\\
0.315605\\
0.052086\\
\color{gr} 0.156259\color{black}
\end{pmatrix},
\end{equation*}
\begin{equation*}
\left[ \frac{{w}^{\prime}_i}{{w}^{\prime}_j} \right] =
\begin{pmatrix}
$\,\,$ 1 $\,\,$ & $\,\,$1.5131$\,\,$ & $\,\,$9.1682$\,\,$ & $\,\,$\color{gr} 3.0561\color{black} $\,\,$ \\
$\,\,$0.6609$\,\,$ & $\,\,$ 1 $\,\,$ & $\,\,$6.0593$\,\,$ & $\,\,$\color{gr} 2.0198\color{black}   $\,\,$ \\
$\,\,$0.1091$\,\,$ & $\,\,$0.1650$\,\,$ & $\,\,$ 1 $\,\,$ & $\,\,$\color{gr} \color{blue}  1/3\color{black}  $\,\,$ \\
$\,\,$\color{gr} 0.3272\color{black} $\,\,$ & $\,\,$\color{gr} 0.4951\color{black} $\,\,$ & $\,\,$\color{gr} \color{blue} 3\color{black} $\,\,$ & $\,\,$ 1  $\,\,$ \\
\end{pmatrix},
\end{equation*}
\end{example}
\newpage
\begin{example}
\begin{equation*}
\mathbf{A} =
\begin{pmatrix}
$\,\,$ 1 $\,\,$ & $\,\,$2$\,\,$ & $\,\,$7$\,\,$ & $\,\,$6 $\,\,$ \\
$\,\,$ 1/2$\,\,$ & $\,\,$ 1 $\,\,$ & $\,\,$6$\,\,$ & $\,\,$2 $\,\,$ \\
$\,\,$ 1/7$\,\,$ & $\,\,$ 1/6$\,\,$ & $\,\,$ 1 $\,\,$ & $\,\,$ 1/4 $\,\,$ \\
$\,\,$ 1/6$\,\,$ & $\,\,$ 1/2$\,\,$ & $\,\,$4$\,\,$ & $\,\,$ 1  $\,\,$ \\
\end{pmatrix},
\qquad
\lambda_{\max} =
4.1317,
\qquad
CR = 0.0496
\end{equation*}

\begin{equation*}
\mathbf{w}^{EM} =
\begin{pmatrix}
0.544441\\
\color{red} 0.269597\color{black} \\
0.050035\\
0.135927
\end{pmatrix}\end{equation*}
\begin{equation*}
\left[ \frac{{w}^{EM}_i}{{w}^{EM}_j} \right] =
\begin{pmatrix}
$\,\,$ 1 $\,\,$ & $\,\,$\color{red} 2.0195\color{black} $\,\,$ & $\,\,$10.8812$\,\,$ & $\,\,$4.0054$\,\,$ \\
$\,\,$\color{red} 0.4952\color{black} $\,\,$ & $\,\,$ 1 $\,\,$ & $\,\,$\color{red} 5.3882\color{black} $\,\,$ & $\,\,$\color{red} 1.9834\color{black}   $\,\,$ \\
$\,\,$0.0919$\,\,$ & $\,\,$\color{red} 0.1856\color{black} $\,\,$ & $\,\,$ 1 $\,\,$ & $\,\,$0.3681 $\,\,$ \\
$\,\,$0.2497$\,\,$ & $\,\,$\color{red} 0.5042\color{black} $\,\,$ & $\,\,$2.7167$\,\,$ & $\,\,$ 1  $\,\,$ \\
\end{pmatrix},
\end{equation*}

\begin{equation*}
\mathbf{w}^{\prime} =
\begin{pmatrix}
0.543214\\
0.271242\\
0.049922\\
0.135621
\end{pmatrix} =
0.997747\cdot
\begin{pmatrix}
0.544441\\
\color{gr} 0.271855\color{black} \\
0.050035\\
0.135927
\end{pmatrix},
\end{equation*}
\begin{equation*}
\left[ \frac{{w}^{\prime}_i}{{w}^{\prime}_j} \right] =
\begin{pmatrix}
$\,\,$ 1 $\,\,$ & $\,\,$\color{gr} 2.0027\color{black} $\,\,$ & $\,\,$10.8812$\,\,$ & $\,\,$4.0054$\,\,$ \\
$\,\,$\color{gr} 0.4993\color{black} $\,\,$ & $\,\,$ 1 $\,\,$ & $\,\,$\color{gr} 5.4333\color{black} $\,\,$ & $\,\,$\color{gr} \color{blue} 2\color{black}   $\,\,$ \\
$\,\,$0.0919$\,\,$ & $\,\,$\color{gr} 0.1840\color{black} $\,\,$ & $\,\,$ 1 $\,\,$ & $\,\,$0.3681 $\,\,$ \\
$\,\,$0.2497$\,\,$ & $\,\,$\color{gr} \color{blue}  1/2\color{black} $\,\,$ & $\,\,$2.7167$\,\,$ & $\,\,$ 1  $\,\,$ \\
\end{pmatrix},
\end{equation*}
\end{example}
\newpage
\begin{example}
\begin{equation*}
\mathbf{A} =
\begin{pmatrix}
$\,\,$ 1 $\,\,$ & $\,\,$2$\,\,$ & $\,\,$7$\,\,$ & $\,\,$6 $\,\,$ \\
$\,\,$ 1/2$\,\,$ & $\,\,$ 1 $\,\,$ & $\,\,$6$\,\,$ & $\,\,$2 $\,\,$ \\
$\,\,$ 1/7$\,\,$ & $\,\,$ 1/6$\,\,$ & $\,\,$ 1 $\,\,$ & $\,\,$ 1/5 $\,\,$ \\
$\,\,$ 1/6$\,\,$ & $\,\,$ 1/2$\,\,$ & $\,\,$5$\,\,$ & $\,\,$ 1  $\,\,$ \\
\end{pmatrix},
\qquad
\lambda_{\max} =
4.1832,
\qquad
CR = 0.0691
\end{equation*}

\begin{equation*}
\mathbf{w}^{EM} =
\begin{pmatrix}
0.542983\\
\color{red} 0.265250\color{black} \\
0.047331\\
0.144437
\end{pmatrix}\end{equation*}
\begin{equation*}
\left[ \frac{{w}^{EM}_i}{{w}^{EM}_j} \right] =
\begin{pmatrix}
$\,\,$ 1 $\,\,$ & $\,\,$\color{red} 2.0471\color{black} $\,\,$ & $\,\,$11.4721$\,\,$ & $\,\,$3.7593$\,\,$ \\
$\,\,$\color{red} 0.4885\color{black} $\,\,$ & $\,\,$ 1 $\,\,$ & $\,\,$\color{red} 5.6042\color{black} $\,\,$ & $\,\,$\color{red} 1.8364\color{black}   $\,\,$ \\
$\,\,$0.0872$\,\,$ & $\,\,$\color{red} 0.1784\color{black} $\,\,$ & $\,\,$ 1 $\,\,$ & $\,\,$0.3277 $\,\,$ \\
$\,\,$0.2660$\,\,$ & $\,\,$\color{red} 0.5445\color{black} $\,\,$ & $\,\,$3.0517$\,\,$ & $\,\,$ 1  $\,\,$ \\
\end{pmatrix},
\end{equation*}

\begin{equation*}
\mathbf{w}^{\prime} =
\begin{pmatrix}
0.539614\\
0.269807\\
0.047037\\
0.143541
\end{pmatrix} =
0.993797\cdot
\begin{pmatrix}
0.542983\\
\color{gr} 0.271491\color{black} \\
0.047331\\
0.144437
\end{pmatrix},
\end{equation*}
\begin{equation*}
\left[ \frac{{w}^{\prime}_i}{{w}^{\prime}_j} \right] =
\begin{pmatrix}
$\,\,$ 1 $\,\,$ & $\,\,$\color{gr} \color{blue} 2\color{black} $\,\,$ & $\,\,$11.4721$\,\,$ & $\,\,$3.7593$\,\,$ \\
$\,\,$\color{gr} \color{blue}  1/2\color{black} $\,\,$ & $\,\,$ 1 $\,\,$ & $\,\,$\color{gr} 5.7360\color{black} $\,\,$ & $\,\,$\color{gr} 1.8797\color{black}   $\,\,$ \\
$\,\,$0.0872$\,\,$ & $\,\,$\color{gr} 0.1743\color{black} $\,\,$ & $\,\,$ 1 $\,\,$ & $\,\,$0.3277 $\,\,$ \\
$\,\,$0.2660$\,\,$ & $\,\,$\color{gr} 0.5320\color{black} $\,\,$ & $\,\,$3.0517$\,\,$ & $\,\,$ 1  $\,\,$ \\
\end{pmatrix},
\end{equation*}
\end{example}
\newpage
\begin{example}
\begin{equation*}
\mathbf{A} =
\begin{pmatrix}
$\,\,$ 1 $\,\,$ & $\,\,$2$\,\,$ & $\,\,$7$\,\,$ & $\,\,$6 $\,\,$ \\
$\,\,$ 1/2$\,\,$ & $\,\,$ 1 $\,\,$ & $\,\,$6$\,\,$ & $\,\,$2 $\,\,$ \\
$\,\,$ 1/7$\,\,$ & $\,\,$ 1/6$\,\,$ & $\,\,$ 1 $\,\,$ & $\,\,$ 1/6 $\,\,$ \\
$\,\,$ 1/6$\,\,$ & $\,\,$ 1/2$\,\,$ & $\,\,$6$\,\,$ & $\,\,$ 1  $\,\,$ \\
\end{pmatrix},
\qquad
\lambda_{\max} =
4.2359,
\qquad
CR = 0.0890
\end{equation*}

\begin{equation*}
\mathbf{w}^{EM} =
\begin{pmatrix}
0.541308\\
\color{red} 0.261426\color{black} \\
0.045194\\
0.152072
\end{pmatrix}\end{equation*}
\begin{equation*}
\left[ \frac{{w}^{EM}_i}{{w}^{EM}_j} \right] =
\begin{pmatrix}
$\,\,$ 1 $\,\,$ & $\,\,$\color{red} 2.0706\color{black} $\,\,$ & $\,\,$11.9774$\,\,$ & $\,\,$3.5596$\,\,$ \\
$\,\,$\color{red} 0.4830\color{black} $\,\,$ & $\,\,$ 1 $\,\,$ & $\,\,$\color{red} 5.7845\color{black} $\,\,$ & $\,\,$\color{red} 1.7191\color{black}   $\,\,$ \\
$\,\,$0.0835$\,\,$ & $\,\,$\color{red} 0.1729\color{black} $\,\,$ & $\,\,$ 1 $\,\,$ & $\,\,$0.2972 $\,\,$ \\
$\,\,$0.2809$\,\,$ & $\,\,$\color{red} 0.5817\color{black} $\,\,$ & $\,\,$3.3649$\,\,$ & $\,\,$ 1  $\,\,$ \\
\end{pmatrix},
\end{equation*}

\begin{equation*}
\mathbf{w}^{\prime} =
\begin{pmatrix}
0.536358\\
0.268179\\
0.044781\\
0.150681
\end{pmatrix} =
0.990857\cdot
\begin{pmatrix}
0.541308\\
\color{gr} 0.270654\color{black} \\
0.045194\\
0.152072
\end{pmatrix},
\end{equation*}
\begin{equation*}
\left[ \frac{{w}^{\prime}_i}{{w}^{\prime}_j} \right] =
\begin{pmatrix}
$\,\,$ 1 $\,\,$ & $\,\,$\color{gr} \color{blue} 2\color{black} $\,\,$ & $\,\,$11.9774$\,\,$ & $\,\,$3.5596$\,\,$ \\
$\,\,$\color{gr} \color{blue}  1/2\color{black} $\,\,$ & $\,\,$ 1 $\,\,$ & $\,\,$\color{gr} 5.9887\color{black} $\,\,$ & $\,\,$\color{gr} 1.7798\color{black}   $\,\,$ \\
$\,\,$0.0835$\,\,$ & $\,\,$\color{gr} 0.1670\color{black} $\,\,$ & $\,\,$ 1 $\,\,$ & $\,\,$0.2972 $\,\,$ \\
$\,\,$0.2809$\,\,$ & $\,\,$\color{gr} 0.5619\color{black} $\,\,$ & $\,\,$3.3649$\,\,$ & $\,\,$ 1  $\,\,$ \\
\end{pmatrix},
\end{equation*}
\end{example}
\newpage
\begin{example}
\begin{equation*}
\mathbf{A} =
\begin{pmatrix}
$\,\,$ 1 $\,\,$ & $\,\,$2$\,\,$ & $\,\,$7$\,\,$ & $\,\,$7 $\,\,$ \\
$\,\,$ 1/2$\,\,$ & $\,\,$ 1 $\,\,$ & $\,\,$7$\,\,$ & $\,\,$2 $\,\,$ \\
$\,\,$ 1/7$\,\,$ & $\,\,$ 1/7$\,\,$ & $\,\,$ 1 $\,\,$ & $\,\,$ 1/5 $\,\,$ \\
$\,\,$ 1/7$\,\,$ & $\,\,$ 1/2$\,\,$ & $\,\,$5$\,\,$ & $\,\,$ 1  $\,\,$ \\
\end{pmatrix},
\qquad
\lambda_{\max} =
4.2287,
\qquad
CR = 0.0862
\end{equation*}

\begin{equation*}
\mathbf{w}^{EM} =
\begin{pmatrix}
0.554218\\
\color{red} 0.266217\color{black} \\
0.044658\\
0.134907
\end{pmatrix}\end{equation*}
\begin{equation*}
\left[ \frac{{w}^{EM}_i}{{w}^{EM}_j} \right] =
\begin{pmatrix}
$\,\,$ 1 $\,\,$ & $\,\,$\color{red} 2.0818\color{black} $\,\,$ & $\,\,$12.4103$\,\,$ & $\,\,$4.1081$\,\,$ \\
$\,\,$\color{red} 0.4803\color{black} $\,\,$ & $\,\,$ 1 $\,\,$ & $\,\,$\color{red} 5.9612\color{black} $\,\,$ & $\,\,$\color{red} 1.9733\color{black}   $\,\,$ \\
$\,\,$0.0806$\,\,$ & $\,\,$\color{red} 0.1678\color{black} $\,\,$ & $\,\,$ 1 $\,\,$ & $\,\,$0.3310 $\,\,$ \\
$\,\,$0.2434$\,\,$ & $\,\,$\color{red} 0.5068\color{black} $\,\,$ & $\,\,$3.0209$\,\,$ & $\,\,$ 1  $\,\,$ \\
\end{pmatrix},
\end{equation*}

\begin{equation*}
\mathbf{w}^{\prime} =
\begin{pmatrix}
0.552231\\
0.268847\\
0.044498\\
0.134424
\end{pmatrix} =
0.996415\cdot
\begin{pmatrix}
0.554218\\
\color{gr} 0.269815\color{black} \\
0.044658\\
0.134907
\end{pmatrix},
\end{equation*}
\begin{equation*}
\left[ \frac{{w}^{\prime}_i}{{w}^{\prime}_j} \right] =
\begin{pmatrix}
$\,\,$ 1 $\,\,$ & $\,\,$\color{gr} 2.0541\color{black} $\,\,$ & $\,\,$12.4103$\,\,$ & $\,\,$4.1081$\,\,$ \\
$\,\,$\color{gr} 0.4868\color{black} $\,\,$ & $\,\,$ 1 $\,\,$ & $\,\,$\color{gr} 6.0418\color{black} $\,\,$ & $\,\,$\color{gr} \color{blue} 2\color{black}   $\,\,$ \\
$\,\,$0.0806$\,\,$ & $\,\,$\color{gr} 0.1655\color{black} $\,\,$ & $\,\,$ 1 $\,\,$ & $\,\,$0.3310 $\,\,$ \\
$\,\,$0.2434$\,\,$ & $\,\,$\color{gr} \color{blue}  1/2\color{black} $\,\,$ & $\,\,$3.0209$\,\,$ & $\,\,$ 1  $\,\,$ \\
\end{pmatrix},
\end{equation*}
\end{example}
\newpage
\begin{example}
\begin{equation*}
\mathbf{A} =
\begin{pmatrix}
$\,\,$ 1 $\,\,$ & $\,\,$2$\,\,$ & $\,\,$8$\,\,$ & $\,\,$2 $\,\,$ \\
$\,\,$ 1/2$\,\,$ & $\,\,$ 1 $\,\,$ & $\,\,$6$\,\,$ & $\,\,$3 $\,\,$ \\
$\,\,$ 1/8$\,\,$ & $\,\,$ 1/6$\,\,$ & $\,\,$ 1 $\,\,$ & $\,\,$ 1/3 $\,\,$ \\
$\,\,$ 1/2$\,\,$ & $\,\,$ 1/3$\,\,$ & $\,\,$3$\,\,$ & $\,\,$ 1  $\,\,$ \\
\end{pmatrix},
\qquad
\lambda_{\max} =
4.1031,
\qquad
CR = 0.0389
\end{equation*}

\begin{equation*}
\mathbf{w}^{EM} =
\begin{pmatrix}
0.454649\\
0.331542\\
\color{red} 0.053356\color{black} \\
0.160453
\end{pmatrix}\end{equation*}
\begin{equation*}
\left[ \frac{{w}^{EM}_i}{{w}^{EM}_j} \right] =
\begin{pmatrix}
$\,\,$ 1 $\,\,$ & $\,\,$1.3713$\,\,$ & $\,\,$\color{red} 8.5210\color{black} $\,\,$ & $\,\,$2.8335$\,\,$ \\
$\,\,$0.7292$\,\,$ & $\,\,$ 1 $\,\,$ & $\,\,$\color{red} 6.2137\color{black} $\,\,$ & $\,\,$2.0663  $\,\,$ \\
$\,\,$\color{red} 0.1174\color{black} $\,\,$ & $\,\,$\color{red} 0.1609\color{black} $\,\,$ & $\,\,$ 1 $\,\,$ & $\,\,$\color{red} 0.3325\color{black}  $\,\,$ \\
$\,\,$0.3529$\,\,$ & $\,\,$0.4840$\,\,$ & $\,\,$\color{red} 3.0072\color{black} $\,\,$ & $\,\,$ 1  $\,\,$ \\
\end{pmatrix},
\end{equation*}

\begin{equation*}
\mathbf{w}^{\prime} =
\begin{pmatrix}
0.454591\\
0.331499\\
0.053477\\
0.160432
\end{pmatrix} =
0.999872\cdot
\begin{pmatrix}
0.454649\\
0.331542\\
\color{gr} 0.053484\color{black} \\
0.160453
\end{pmatrix},
\end{equation*}
\begin{equation*}
\left[ \frac{{w}^{\prime}_i}{{w}^{\prime}_j} \right] =
\begin{pmatrix}
$\,\,$ 1 $\,\,$ & $\,\,$1.3713$\,\,$ & $\,\,$\color{gr} 8.5006\color{black} $\,\,$ & $\,\,$2.8335$\,\,$ \\
$\,\,$0.7292$\,\,$ & $\,\,$ 1 $\,\,$ & $\,\,$\color{gr} 6.1989\color{black} $\,\,$ & $\,\,$2.0663  $\,\,$ \\
$\,\,$\color{gr} 0.1176\color{black} $\,\,$ & $\,\,$\color{gr} 0.1613\color{black} $\,\,$ & $\,\,$ 1 $\,\,$ & $\,\,$\color{gr} \color{blue}  1/3\color{black}  $\,\,$ \\
$\,\,$0.3529$\,\,$ & $\,\,$0.4840$\,\,$ & $\,\,$\color{gr} \color{blue} 3\color{black} $\,\,$ & $\,\,$ 1  $\,\,$ \\
\end{pmatrix},
\end{equation*}
\end{example}
\newpage
\begin{example}
\begin{equation*}
\mathbf{A} =
\begin{pmatrix}
$\,\,$ 1 $\,\,$ & $\,\,$2$\,\,$ & $\,\,$8$\,\,$ & $\,\,$2 $\,\,$ \\
$\,\,$ 1/2$\,\,$ & $\,\,$ 1 $\,\,$ & $\,\,$7$\,\,$ & $\,\,$5 $\,\,$ \\
$\,\,$ 1/8$\,\,$ & $\,\,$ 1/7$\,\,$ & $\,\,$ 1 $\,\,$ & $\,\,$ 1/2 $\,\,$ \\
$\,\,$ 1/2$\,\,$ & $\,\,$ 1/5$\,\,$ & $\,\,$2$\,\,$ & $\,\,$ 1  $\,\,$ \\
\end{pmatrix},
\qquad
\lambda_{\max} =
4.2287,
\qquad
CR = 0.0862
\end{equation*}

\begin{equation*}
\mathbf{w}^{EM} =
\begin{pmatrix}
0.443640\\
0.377967\\
\color{red} 0.053275\color{black} \\
0.125117
\end{pmatrix}\end{equation*}
\begin{equation*}
\left[ \frac{{w}^{EM}_i}{{w}^{EM}_j} \right] =
\begin{pmatrix}
$\,\,$ 1 $\,\,$ & $\,\,$1.1738$\,\,$ & $\,\,$\color{red} 8.3273\color{black} $\,\,$ & $\,\,$3.5458$\,\,$ \\
$\,\,$0.8520$\,\,$ & $\,\,$ 1 $\,\,$ & $\,\,$\color{red} 7.0946\color{black} $\,\,$ & $\,\,$3.0209  $\,\,$ \\
$\,\,$\color{red} 0.1201\color{black} $\,\,$ & $\,\,$\color{red} 0.1410\color{black} $\,\,$ & $\,\,$ 1 $\,\,$ & $\,\,$\color{red} 0.4258\color{black}  $\,\,$ \\
$\,\,$0.2820$\,\,$ & $\,\,$0.3310$\,\,$ & $\,\,$\color{red} 2.3485\color{black} $\,\,$ & $\,\,$ 1  $\,\,$ \\
\end{pmatrix},
\end{equation*}

\begin{equation*}
\mathbf{w}^{\prime} =
\begin{pmatrix}
0.443321\\
0.377695\\
0.053956\\
0.125027
\end{pmatrix} =
0.999281\cdot
\begin{pmatrix}
0.443640\\
0.377967\\
\color{gr} 0.053995\color{black} \\
0.125117
\end{pmatrix},
\end{equation*}
\begin{equation*}
\left[ \frac{{w}^{\prime}_i}{{w}^{\prime}_j} \right] =
\begin{pmatrix}
$\,\,$ 1 $\,\,$ & $\,\,$1.1738$\,\,$ & $\,\,$\color{gr} 8.2163\color{black} $\,\,$ & $\,\,$3.5458$\,\,$ \\
$\,\,$0.8520$\,\,$ & $\,\,$ 1 $\,\,$ & $\,\,$\color{gr} \color{blue} 7\color{black} $\,\,$ & $\,\,$3.0209  $\,\,$ \\
$\,\,$\color{gr} 0.1217\color{black} $\,\,$ & $\,\,$\color{gr} \color{blue}  1/7\color{black} $\,\,$ & $\,\,$ 1 $\,\,$ & $\,\,$\color{gr} 0.4316\color{black}  $\,\,$ \\
$\,\,$0.2820$\,\,$ & $\,\,$0.3310$\,\,$ & $\,\,$\color{gr} 2.3172\color{black} $\,\,$ & $\,\,$ 1  $\,\,$ \\
\end{pmatrix},
\end{equation*}
\end{example}
\newpage
\begin{example}
\begin{equation*}
\mathbf{A} =
\begin{pmatrix}
$\,\,$ 1 $\,\,$ & $\,\,$2$\,\,$ & $\,\,$8$\,\,$ & $\,\,$4 $\,\,$ \\
$\,\,$ 1/2$\,\,$ & $\,\,$ 1 $\,\,$ & $\,\,$3$\,\,$ & $\,\,$3 $\,\,$ \\
$\,\,$ 1/8$\,\,$ & $\,\,$ 1/3$\,\,$ & $\,\,$ 1 $\,\,$ & $\,\,$ 1/3 $\,\,$ \\
$\,\,$ 1/4$\,\,$ & $\,\,$ 1/3$\,\,$ & $\,\,$3$\,\,$ & $\,\,$ 1  $\,\,$ \\
\end{pmatrix},
\qquad
\lambda_{\max} =
4.1031,
\qquad
CR = 0.0389
\end{equation*}

\begin{equation*}
\mathbf{w}^{EM} =
\begin{pmatrix}
\color{red} 0.521712\color{black} \\
0.277844\\
0.065370\\
0.135074
\end{pmatrix}\end{equation*}
\begin{equation*}
\left[ \frac{{w}^{EM}_i}{{w}^{EM}_j} \right] =
\begin{pmatrix}
$\,\,$ 1 $\,\,$ & $\,\,$\color{red} 1.8777\color{black} $\,\,$ & $\,\,$\color{red} 7.9809\color{black} $\,\,$ & $\,\,$\color{red} 3.8624\color{black} $\,\,$ \\
$\,\,$\color{red} 0.5326\color{black} $\,\,$ & $\,\,$ 1 $\,\,$ & $\,\,$4.2503$\,\,$ & $\,\,$2.0570  $\,\,$ \\
$\,\,$\color{red} 0.1253\color{black} $\,\,$ & $\,\,$0.2353$\,\,$ & $\,\,$ 1 $\,\,$ & $\,\,$0.4840 $\,\,$ \\
$\,\,$\color{red} 0.2589\color{black} $\,\,$ & $\,\,$0.4862$\,\,$ & $\,\,$2.0663$\,\,$ & $\,\,$ 1  $\,\,$ \\
\end{pmatrix},
\end{equation*}

\begin{equation*}
\mathbf{w}^{\prime} =
\begin{pmatrix}
0.522309\\
0.277497\\
0.065289\\
0.134905
\end{pmatrix} =
0.998751\cdot
\begin{pmatrix}
\color{gr} 0.522962\color{black} \\
0.277844\\
0.065370\\
0.135074
\end{pmatrix},
\end{equation*}
\begin{equation*}
\left[ \frac{{w}^{\prime}_i}{{w}^{\prime}_j} \right] =
\begin{pmatrix}
$\,\,$ 1 $\,\,$ & $\,\,$\color{gr} 1.8822\color{black} $\,\,$ & $\,\,$\color{gr} \color{blue} 8\color{black} $\,\,$ & $\,\,$\color{gr} 3.8717\color{black} $\,\,$ \\
$\,\,$\color{gr} 0.5313\color{black} $\,\,$ & $\,\,$ 1 $\,\,$ & $\,\,$4.2503$\,\,$ & $\,\,$2.0570  $\,\,$ \\
$\,\,$\color{gr} \color{blue}  1/8\color{black} $\,\,$ & $\,\,$0.2353$\,\,$ & $\,\,$ 1 $\,\,$ & $\,\,$0.4840 $\,\,$ \\
$\,\,$\color{gr} 0.2583\color{black} $\,\,$ & $\,\,$0.4862$\,\,$ & $\,\,$2.0663$\,\,$ & $\,\,$ 1  $\,\,$ \\
\end{pmatrix},
\end{equation*}
\end{example}
\newpage
\begin{example}
\begin{equation*}
\mathbf{A} =
\begin{pmatrix}
$\,\,$ 1 $\,\,$ & $\,\,$2$\,\,$ & $\,\,$8$\,\,$ & $\,\,$5 $\,\,$ \\
$\,\,$ 1/2$\,\,$ & $\,\,$ 1 $\,\,$ & $\,\,$5$\,\,$ & $\,\,$2 $\,\,$ \\
$\,\,$ 1/8$\,\,$ & $\,\,$ 1/5$\,\,$ & $\,\,$ 1 $\,\,$ & $\,\,$ 1/3 $\,\,$ \\
$\,\,$ 1/5$\,\,$ & $\,\,$ 1/2$\,\,$ & $\,\,$3$\,\,$ & $\,\,$ 1  $\,\,$ \\
\end{pmatrix},
\qquad
\lambda_{\max} =
4.0332,
\qquad
CR = 0.0125
\end{equation*}

\begin{equation*}
\mathbf{w}^{EM} =
\begin{pmatrix}
0.542689\\
\color{red} 0.268294\color{black} \\
0.054804\\
0.134213
\end{pmatrix}\end{equation*}
\begin{equation*}
\left[ \frac{{w}^{EM}_i}{{w}^{EM}_j} \right] =
\begin{pmatrix}
$\,\,$ 1 $\,\,$ & $\,\,$\color{red} 2.0227\color{black} $\,\,$ & $\,\,$9.9023$\,\,$ & $\,\,$4.0435$\,\,$ \\
$\,\,$\color{red} 0.4944\color{black} $\,\,$ & $\,\,$ 1 $\,\,$ & $\,\,$\color{red} 4.8955\color{black} $\,\,$ & $\,\,$\color{red} 1.9990\color{black}   $\,\,$ \\
$\,\,$0.1010$\,\,$ & $\,\,$\color{red} 0.2043\color{black} $\,\,$ & $\,\,$ 1 $\,\,$ & $\,\,$0.4083 $\,\,$ \\
$\,\,$0.2473$\,\,$ & $\,\,$\color{red} 0.5002\color{black} $\,\,$ & $\,\,$2.4490$\,\,$ & $\,\,$ 1  $\,\,$ \\
\end{pmatrix},
\end{equation*}

\begin{equation*}
\mathbf{w}^{\prime} =
\begin{pmatrix}
0.542616\\
0.268391\\
0.054797\\
0.134196
\end{pmatrix} =
0.999867\cdot
\begin{pmatrix}
0.542689\\
\color{gr} 0.268427\color{black} \\
0.054804\\
0.134213
\end{pmatrix},
\end{equation*}
\begin{equation*}
\left[ \frac{{w}^{\prime}_i}{{w}^{\prime}_j} \right] =
\begin{pmatrix}
$\,\,$ 1 $\,\,$ & $\,\,$\color{gr} 2.0217\color{black} $\,\,$ & $\,\,$9.9023$\,\,$ & $\,\,$4.0435$\,\,$ \\
$\,\,$\color{gr} 0.4946\color{black} $\,\,$ & $\,\,$ 1 $\,\,$ & $\,\,$\color{gr} 4.8979\color{black} $\,\,$ & $\,\,$\color{gr} \color{blue} 2\color{black}   $\,\,$ \\
$\,\,$0.1010$\,\,$ & $\,\,$\color{gr} 0.2042\color{black} $\,\,$ & $\,\,$ 1 $\,\,$ & $\,\,$0.4083 $\,\,$ \\
$\,\,$0.2473$\,\,$ & $\,\,$\color{gr} \color{blue}  1/2\color{black} $\,\,$ & $\,\,$2.4490$\,\,$ & $\,\,$ 1  $\,\,$ \\
\end{pmatrix},
\end{equation*}
\end{example}
\newpage
\begin{example}
\begin{equation*}
\mathbf{A} =
\begin{pmatrix}
$\,\,$ 1 $\,\,$ & $\,\,$2$\,\,$ & $\,\,$8$\,\,$ & $\,\,$6 $\,\,$ \\
$\,\,$ 1/2$\,\,$ & $\,\,$ 1 $\,\,$ & $\,\,$3$\,\,$ & $\,\,$5 $\,\,$ \\
$\,\,$ 1/8$\,\,$ & $\,\,$ 1/3$\,\,$ & $\,\,$ 1 $\,\,$ & $\,\,$ 1/2 $\,\,$ \\
$\,\,$ 1/6$\,\,$ & $\,\,$ 1/5$\,\,$ & $\,\,$2$\,\,$ & $\,\,$ 1  $\,\,$ \\
\end{pmatrix},
\qquad
\lambda_{\max} =
4.1252,
\qquad
CR = 0.0472
\end{equation*}

\begin{equation*}
\mathbf{w}^{EM} =
\begin{pmatrix}
\color{red} 0.541541\color{black} \\
0.298664\\
0.068173\\
0.091621
\end{pmatrix}\end{equation*}
\begin{equation*}
\left[ \frac{{w}^{EM}_i}{{w}^{EM}_j} \right] =
\begin{pmatrix}
$\,\,$ 1 $\,\,$ & $\,\,$\color{red} 1.8132\color{black} $\,\,$ & $\,\,$\color{red} 7.9436\color{black} $\,\,$ & $\,\,$\color{red} 5.9107\color{black} $\,\,$ \\
$\,\,$\color{red} 0.5515\color{black} $\,\,$ & $\,\,$ 1 $\,\,$ & $\,\,$4.3809$\,\,$ & $\,\,$3.2598  $\,\,$ \\
$\,\,$\color{red} 0.1259\color{black} $\,\,$ & $\,\,$0.2283$\,\,$ & $\,\,$ 1 $\,\,$ & $\,\,$0.7441 $\,\,$ \\
$\,\,$\color{red} 0.1692\color{black} $\,\,$ & $\,\,$0.3068$\,\,$ & $\,\,$1.3439$\,\,$ & $\,\,$ 1  $\,\,$ \\
\end{pmatrix},
\end{equation*}

\begin{equation*}
\mathbf{w}^{\prime} =
\begin{pmatrix}
0.543298\\
0.297520\\
0.067912\\
0.091270
\end{pmatrix} =
0.996168\cdot
\begin{pmatrix}
\color{gr} 0.545388\color{black} \\
0.298664\\
0.068173\\
0.091621
\end{pmatrix},
\end{equation*}
\begin{equation*}
\left[ \frac{{w}^{\prime}_i}{{w}^{\prime}_j} \right] =
\begin{pmatrix}
$\,\,$ 1 $\,\,$ & $\,\,$\color{gr} 1.8261\color{black} $\,\,$ & $\,\,$\color{gr} \color{blue} 8\color{black} $\,\,$ & $\,\,$\color{gr} 5.9527\color{black} $\,\,$ \\
$\,\,$\color{gr} 0.5476\color{black} $\,\,$ & $\,\,$ 1 $\,\,$ & $\,\,$4.3809$\,\,$ & $\,\,$3.2598  $\,\,$ \\
$\,\,$\color{gr} \color{blue}  1/8\color{black} $\,\,$ & $\,\,$0.2283$\,\,$ & $\,\,$ 1 $\,\,$ & $\,\,$0.7441 $\,\,$ \\
$\,\,$\color{gr} 0.1680\color{black} $\,\,$ & $\,\,$0.3068$\,\,$ & $\,\,$1.3439$\,\,$ & $\,\,$ 1  $\,\,$ \\
\end{pmatrix},
\end{equation*}
\end{example}
\newpage
\begin{example}
\begin{equation*}
\mathbf{A} =
\begin{pmatrix}
$\,\,$ 1 $\,\,$ & $\,\,$2$\,\,$ & $\,\,$8$\,\,$ & $\,\,$6 $\,\,$ \\
$\,\,$ 1/2$\,\,$ & $\,\,$ 1 $\,\,$ & $\,\,$6$\,\,$ & $\,\,$2 $\,\,$ \\
$\,\,$ 1/8$\,\,$ & $\,\,$ 1/6$\,\,$ & $\,\,$ 1 $\,\,$ & $\,\,$ 1/4 $\,\,$ \\
$\,\,$ 1/6$\,\,$ & $\,\,$ 1/2$\,\,$ & $\,\,$4$\,\,$ & $\,\,$ 1  $\,\,$ \\
\end{pmatrix},
\qquad
\lambda_{\max} =
4.1031,
\qquad
CR = 0.0389
\end{equation*}

\begin{equation*}
\mathbf{w}^{EM} =
\begin{pmatrix}
0.552337\\
\color{red} 0.266669\color{black} \\
0.047339\\
0.133654
\end{pmatrix}\end{equation*}
\begin{equation*}
\left[ \frac{{w}^{EM}_i}{{w}^{EM}_j} \right] =
\begin{pmatrix}
$\,\,$ 1 $\,\,$ & $\,\,$\color{red} 2.0712\color{black} $\,\,$ & $\,\,$11.6676$\,\,$ & $\,\,$4.1326$\,\,$ \\
$\,\,$\color{red} 0.4828\color{black} $\,\,$ & $\,\,$ 1 $\,\,$ & $\,\,$\color{red} 5.6331\color{black} $\,\,$ & $\,\,$\color{red} 1.9952\color{black}   $\,\,$ \\
$\,\,$0.0857$\,\,$ & $\,\,$\color{red} 0.1775\color{black} $\,\,$ & $\,\,$ 1 $\,\,$ & $\,\,$0.3542 $\,\,$ \\
$\,\,$0.2420$\,\,$ & $\,\,$\color{red} 0.5012\color{black} $\,\,$ & $\,\,$2.8233$\,\,$ & $\,\,$ 1  $\,\,$ \\
\end{pmatrix},
\end{equation*}

\begin{equation*}
\mathbf{w}^{\prime} =
\begin{pmatrix}
0.551984\\
0.267138\\
0.047309\\
0.133569
\end{pmatrix} =
0.999361\cdot
\begin{pmatrix}
0.552337\\
\color{gr} 0.267309\color{black} \\
0.047339\\
0.133654
\end{pmatrix},
\end{equation*}
\begin{equation*}
\left[ \frac{{w}^{\prime}_i}{{w}^{\prime}_j} \right] =
\begin{pmatrix}
$\,\,$ 1 $\,\,$ & $\,\,$\color{gr} 2.0663\color{black} $\,\,$ & $\,\,$11.6676$\,\,$ & $\,\,$4.1326$\,\,$ \\
$\,\,$\color{gr} 0.4840\color{black} $\,\,$ & $\,\,$ 1 $\,\,$ & $\,\,$\color{gr} 5.6467\color{black} $\,\,$ & $\,\,$\color{gr} \color{blue} 2\color{black}   $\,\,$ \\
$\,\,$0.0857$\,\,$ & $\,\,$\color{gr} 0.1771\color{black} $\,\,$ & $\,\,$ 1 $\,\,$ & $\,\,$0.3542 $\,\,$ \\
$\,\,$0.2420$\,\,$ & $\,\,$\color{gr} \color{blue}  1/2\color{black} $\,\,$ & $\,\,$2.8233$\,\,$ & $\,\,$ 1  $\,\,$ \\
\end{pmatrix},
\end{equation*}
\end{example}
\newpage
\begin{example}
\begin{equation*}
\mathbf{A} =
\begin{pmatrix}
$\,\,$ 1 $\,\,$ & $\,\,$2$\,\,$ & $\,\,$8$\,\,$ & $\,\,$6 $\,\,$ \\
$\,\,$ 1/2$\,\,$ & $\,\,$ 1 $\,\,$ & $\,\,$6$\,\,$ & $\,\,$2 $\,\,$ \\
$\,\,$ 1/8$\,\,$ & $\,\,$ 1/6$\,\,$ & $\,\,$ 1 $\,\,$ & $\,\,$ 1/5 $\,\,$ \\
$\,\,$ 1/6$\,\,$ & $\,\,$ 1/2$\,\,$ & $\,\,$5$\,\,$ & $\,\,$ 1  $\,\,$ \\
\end{pmatrix},
\qquad
\lambda_{\max} =
4.1502,
\qquad
CR = 0.0566
\end{equation*}

\begin{equation*}
\mathbf{w}^{EM} =
\begin{pmatrix}
0.550660\\
\color{red} 0.262715\color{black} \\
0.044756\\
0.141869
\end{pmatrix}\end{equation*}
\begin{equation*}
\left[ \frac{{w}^{EM}_i}{{w}^{EM}_j} \right] =
\begin{pmatrix}
$\,\,$ 1 $\,\,$ & $\,\,$\color{red} 2.0960\color{black} $\,\,$ & $\,\,$12.3035$\,\,$ & $\,\,$3.8815$\,\,$ \\
$\,\,$\color{red} 0.4771\color{black} $\,\,$ & $\,\,$ 1 $\,\,$ & $\,\,$\color{red} 5.8699\color{black} $\,\,$ & $\,\,$\color{red} 1.8518\color{black}   $\,\,$ \\
$\,\,$0.0813$\,\,$ & $\,\,$\color{red} 0.1704\color{black} $\,\,$ & $\,\,$ 1 $\,\,$ & $\,\,$0.3155 $\,\,$ \\
$\,\,$0.2576$\,\,$ & $\,\,$\color{red} 0.5400\color{black} $\,\,$ & $\,\,$3.1698$\,\,$ & $\,\,$ 1  $\,\,$ \\
\end{pmatrix},
\end{equation*}

\begin{equation*}
\mathbf{w}^{\prime} =
\begin{pmatrix}
0.547472\\
0.266984\\
0.044497\\
0.141047
\end{pmatrix} =
0.994210\cdot
\begin{pmatrix}
0.550660\\
\color{gr} 0.268539\color{black} \\
0.044756\\
0.141869
\end{pmatrix},
\end{equation*}
\begin{equation*}
\left[ \frac{{w}^{\prime}_i}{{w}^{\prime}_j} \right] =
\begin{pmatrix}
$\,\,$ 1 $\,\,$ & $\,\,$\color{gr} 2.0506\color{black} $\,\,$ & $\,\,$12.3035$\,\,$ & $\,\,$3.8815$\,\,$ \\
$\,\,$\color{gr} 0.4877\color{black} $\,\,$ & $\,\,$ 1 $\,\,$ & $\,\,$\color{gr} \color{blue} 6\color{black} $\,\,$ & $\,\,$\color{gr} 1.8929\color{black}   $\,\,$ \\
$\,\,$0.0813$\,\,$ & $\,\,$\color{gr} \color{blue}  1/6\color{black} $\,\,$ & $\,\,$ 1 $\,\,$ & $\,\,$0.3155 $\,\,$ \\
$\,\,$0.2576$\,\,$ & $\,\,$\color{gr} 0.5283\color{black} $\,\,$ & $\,\,$3.1698$\,\,$ & $\,\,$ 1  $\,\,$ \\
\end{pmatrix},
\end{equation*}
\end{example}
\newpage
\begin{example}
\begin{equation*}
\mathbf{A} =
\begin{pmatrix}
$\,\,$ 1 $\,\,$ & $\,\,$2$\,\,$ & $\,\,$8$\,\,$ & $\,\,$6 $\,\,$ \\
$\,\,$ 1/2$\,\,$ & $\,\,$ 1 $\,\,$ & $\,\,$7$\,\,$ & $\,\,$2 $\,\,$ \\
$\,\,$ 1/8$\,\,$ & $\,\,$ 1/7$\,\,$ & $\,\,$ 1 $\,\,$ & $\,\,$ 1/5 $\,\,$ \\
$\,\,$ 1/6$\,\,$ & $\,\,$ 1/2$\,\,$ & $\,\,$5$\,\,$ & $\,\,$ 1  $\,\,$ \\
\end{pmatrix},
\qquad
\lambda_{\max} =
4.1512,
\qquad
CR = 0.0570
\end{equation*}

\begin{equation*}
\mathbf{w}^{EM} =
\begin{pmatrix}
0.546721\\
\color{red} 0.270632\color{black} \\
0.042830\\
0.139817
\end{pmatrix}\end{equation*}
\begin{equation*}
\left[ \frac{{w}^{EM}_i}{{w}^{EM}_j} \right] =
\begin{pmatrix}
$\,\,$ 1 $\,\,$ & $\,\,$\color{red} 2.0202\color{black} $\,\,$ & $\,\,$12.7648$\,\,$ & $\,\,$3.9103$\,\,$ \\
$\,\,$\color{red} 0.4950\color{black} $\,\,$ & $\,\,$ 1 $\,\,$ & $\,\,$\color{red} 6.3187\color{black} $\,\,$ & $\,\,$\color{red} 1.9356\color{black}   $\,\,$ \\
$\,\,$0.0783$\,\,$ & $\,\,$\color{red} 0.1583\color{black} $\,\,$ & $\,\,$ 1 $\,\,$ & $\,\,$0.3063 $\,\,$ \\
$\,\,$0.2557$\,\,$ & $\,\,$\color{red} 0.5166\color{black} $\,\,$ & $\,\,$3.2645$\,\,$ & $\,\,$ 1  $\,\,$ \\
\end{pmatrix},
\end{equation*}

\begin{equation*}
\mathbf{w}^{\prime} =
\begin{pmatrix}
0.545233\\
0.272617\\
0.042714\\
0.139437
\end{pmatrix} =
0.997279\cdot
\begin{pmatrix}
0.546721\\
\color{gr} 0.273360\color{black} \\
0.042830\\
0.139817
\end{pmatrix},
\end{equation*}
\begin{equation*}
\left[ \frac{{w}^{\prime}_i}{{w}^{\prime}_j} \right] =
\begin{pmatrix}
$\,\,$ 1 $\,\,$ & $\,\,$\color{gr} \color{blue} 2\color{black} $\,\,$ & $\,\,$12.7648$\,\,$ & $\,\,$3.9103$\,\,$ \\
$\,\,$\color{gr} \color{blue}  1/2\color{black} $\,\,$ & $\,\,$ 1 $\,\,$ & $\,\,$\color{gr} 6.3824\color{black} $\,\,$ & $\,\,$\color{gr} 1.9551\color{black}   $\,\,$ \\
$\,\,$0.0783$\,\,$ & $\,\,$\color{gr} 0.1567\color{black} $\,\,$ & $\,\,$ 1 $\,\,$ & $\,\,$0.3063 $\,\,$ \\
$\,\,$0.2557$\,\,$ & $\,\,$\color{gr} 0.5115\color{black} $\,\,$ & $\,\,$3.2645$\,\,$ & $\,\,$ 1  $\,\,$ \\
\end{pmatrix},
\end{equation*}
\end{example}
\newpage
\begin{example}
\begin{equation*}
\mathbf{A} =
\begin{pmatrix}
$\,\,$ 1 $\,\,$ & $\,\,$2$\,\,$ & $\,\,$8$\,\,$ & $\,\,$6 $\,\,$ \\
$\,\,$ 1/2$\,\,$ & $\,\,$ 1 $\,\,$ & $\,\,$7$\,\,$ & $\,\,$2 $\,\,$ \\
$\,\,$ 1/8$\,\,$ & $\,\,$ 1/7$\,\,$ & $\,\,$ 1 $\,\,$ & $\,\,$ 1/6 $\,\,$ \\
$\,\,$ 1/6$\,\,$ & $\,\,$ 1/2$\,\,$ & $\,\,$6$\,\,$ & $\,\,$ 1  $\,\,$ \\
\end{pmatrix},
\qquad
\lambda_{\max} =
4.1964,
\qquad
CR = 0.0741
\end{equation*}

\begin{equation*}
\mathbf{w}^{EM} =
\begin{pmatrix}
0.545259\\
\color{red} 0.266855\color{black} \\
0.040913\\
0.146973
\end{pmatrix}\end{equation*}
\begin{equation*}
\left[ \frac{{w}^{EM}_i}{{w}^{EM}_j} \right] =
\begin{pmatrix}
$\,\,$ 1 $\,\,$ & $\,\,$\color{red} 2.0433\color{black} $\,\,$ & $\,\,$13.3271$\,\,$ & $\,\,$3.7099$\,\,$ \\
$\,\,$\color{red} 0.4894\color{black} $\,\,$ & $\,\,$ 1 $\,\,$ & $\,\,$\color{red} 6.5224\color{black} $\,\,$ & $\,\,$\color{red} 1.8157\color{black}   $\,\,$ \\
$\,\,$0.0750$\,\,$ & $\,\,$\color{red} 0.1533\color{black} $\,\,$ & $\,\,$ 1 $\,\,$ & $\,\,$0.2784 $\,\,$ \\
$\,\,$0.2695$\,\,$ & $\,\,$\color{red} 0.5508\color{black} $\,\,$ & $\,\,$3.5923$\,\,$ & $\,\,$ 1  $\,\,$ \\
\end{pmatrix},
\end{equation*}

\begin{equation*}
\mathbf{w}^{\prime} =
\begin{pmatrix}
0.542128\\
0.271064\\
0.040678\\
0.146130
\end{pmatrix} =
0.994258\cdot
\begin{pmatrix}
0.545259\\
\color{gr} 0.272629\color{black} \\
0.040913\\
0.146973
\end{pmatrix},
\end{equation*}
\begin{equation*}
\left[ \frac{{w}^{\prime}_i}{{w}^{\prime}_j} \right] =
\begin{pmatrix}
$\,\,$ 1 $\,\,$ & $\,\,$\color{gr} \color{blue} 2\color{black} $\,\,$ & $\,\,$13.3271$\,\,$ & $\,\,$3.7099$\,\,$ \\
$\,\,$\color{gr} \color{blue}  1/2\color{black} $\,\,$ & $\,\,$ 1 $\,\,$ & $\,\,$\color{gr} 6.6636\color{black} $\,\,$ & $\,\,$\color{gr} 1.8550\color{black}   $\,\,$ \\
$\,\,$0.0750$\,\,$ & $\,\,$\color{gr} 0.1501\color{black} $\,\,$ & $\,\,$ 1 $\,\,$ & $\,\,$0.2784 $\,\,$ \\
$\,\,$0.2695$\,\,$ & $\,\,$\color{gr} 0.5391\color{black} $\,\,$ & $\,\,$3.5923$\,\,$ & $\,\,$ 1  $\,\,$ \\
\end{pmatrix},
\end{equation*}
\end{example}
\newpage
\begin{example}
\begin{equation*}
\mathbf{A} =
\begin{pmatrix}
$\,\,$ 1 $\,\,$ & $\,\,$2$\,\,$ & $\,\,$8$\,\,$ & $\,\,$6 $\,\,$ \\
$\,\,$ 1/2$\,\,$ & $\,\,$ 1 $\,\,$ & $\,\,$7$\,\,$ & $\,\,$2 $\,\,$ \\
$\,\,$ 1/8$\,\,$ & $\,\,$ 1/7$\,\,$ & $\,\,$ 1 $\,\,$ & $\,\,$ 1/7 $\,\,$ \\
$\,\,$ 1/6$\,\,$ & $\,\,$ 1/2$\,\,$ & $\,\,$7$\,\,$ & $\,\,$ 1  $\,\,$ \\
\end{pmatrix},
\qquad
\lambda_{\max} =
4.2421,
\qquad
CR = 0.0913
\end{equation*}

\begin{equation*}
\mathbf{w}^{EM} =
\begin{pmatrix}
0.543683\\
\color{red} 0.263472\color{black} \\
0.039335\\
0.153510
\end{pmatrix}\end{equation*}
\begin{equation*}
\left[ \frac{{w}^{EM}_i}{{w}^{EM}_j} \right] =
\begin{pmatrix}
$\,\,$ 1 $\,\,$ & $\,\,$\color{red} 2.0635\color{black} $\,\,$ & $\,\,$13.8218$\,\,$ & $\,\,$3.5417$\,\,$ \\
$\,\,$\color{red} 0.4846\color{black} $\,\,$ & $\,\,$ 1 $\,\,$ & $\,\,$\color{red} 6.6981\color{black} $\,\,$ & $\,\,$\color{red} 1.7163\color{black}   $\,\,$ \\
$\,\,$0.0723$\,\,$ & $\,\,$\color{red} 0.1493\color{black} $\,\,$ & $\,\,$ 1 $\,\,$ & $\,\,$0.2562 $\,\,$ \\
$\,\,$0.2824$\,\,$ & $\,\,$\color{red} 0.5826\color{black} $\,\,$ & $\,\,$3.9026$\,\,$ & $\,\,$ 1  $\,\,$ \\
\end{pmatrix},
\end{equation*}

\begin{equation*}
\mathbf{w}^{\prime} =
\begin{pmatrix}
0.539171\\
0.269585\\
0.039009\\
0.152236
\end{pmatrix} =
0.991700\cdot
\begin{pmatrix}
0.543683\\
\color{gr} 0.271842\color{black} \\
0.039335\\
0.153510
\end{pmatrix},
\end{equation*}
\begin{equation*}
\left[ \frac{{w}^{\prime}_i}{{w}^{\prime}_j} \right] =
\begin{pmatrix}
$\,\,$ 1 $\,\,$ & $\,\,$\color{gr} \color{blue} 2\color{black} $\,\,$ & $\,\,$13.8218$\,\,$ & $\,\,$3.5417$\,\,$ \\
$\,\,$\color{gr} \color{blue}  1/2\color{black} $\,\,$ & $\,\,$ 1 $\,\,$ & $\,\,$\color{gr} 6.9109\color{black} $\,\,$ & $\,\,$\color{gr} 1.7708\color{black}   $\,\,$ \\
$\,\,$0.0723$\,\,$ & $\,\,$\color{gr} 0.1447\color{black} $\,\,$ & $\,\,$ 1 $\,\,$ & $\,\,$0.2562 $\,\,$ \\
$\,\,$0.2824$\,\,$ & $\,\,$\color{gr} 0.5647\color{black} $\,\,$ & $\,\,$3.9026$\,\,$ & $\,\,$ 1  $\,\,$ \\
\end{pmatrix},
\end{equation*}
\end{example}
\newpage
\begin{example}
\begin{equation*}
\mathbf{A} =
\begin{pmatrix}
$\,\,$ 1 $\,\,$ & $\,\,$2$\,\,$ & $\,\,$8$\,\,$ & $\,\,$7 $\,\,$ \\
$\,\,$ 1/2$\,\,$ & $\,\,$ 1 $\,\,$ & $\,\,$7$\,\,$ & $\,\,$2 $\,\,$ \\
$\,\,$ 1/8$\,\,$ & $\,\,$ 1/7$\,\,$ & $\,\,$ 1 $\,\,$ & $\,\,$ 1/5 $\,\,$ \\
$\,\,$ 1/7$\,\,$ & $\,\,$ 1/2$\,\,$ & $\,\,$5$\,\,$ & $\,\,$ 1  $\,\,$ \\
\end{pmatrix},
\qquad
\lambda_{\max} =
4.1897,
\qquad
CR = 0.0715
\end{equation*}

\begin{equation*}
\mathbf{w}^{EM} =
\begin{pmatrix}
0.561755\\
\color{red} 0.263600\color{black} \\
0.042129\\
0.132517
\end{pmatrix}\end{equation*}
\begin{equation*}
\left[ \frac{{w}^{EM}_i}{{w}^{EM}_j} \right] =
\begin{pmatrix}
$\,\,$ 1 $\,\,$ & $\,\,$\color{red} 2.1311\color{black} $\,\,$ & $\,\,$13.3342$\,\,$ & $\,\,$4.2391$\,\,$ \\
$\,\,$\color{red} 0.4692\color{black} $\,\,$ & $\,\,$ 1 $\,\,$ & $\,\,$\color{red} 6.2570\color{black} $\,\,$ & $\,\,$\color{red} 1.9892\color{black}   $\,\,$ \\
$\,\,$0.0750$\,\,$ & $\,\,$\color{red} 0.1598\color{black} $\,\,$ & $\,\,$ 1 $\,\,$ & $\,\,$0.3179 $\,\,$ \\
$\,\,$0.2359$\,\,$ & $\,\,$\color{red} 0.5027\color{black} $\,\,$ & $\,\,$3.1455$\,\,$ & $\,\,$ 1  $\,\,$ \\
\end{pmatrix},
\end{equation*}

\begin{equation*}
\mathbf{w}^{\prime} =
\begin{pmatrix}
0.560950\\
0.264654\\
0.042068\\
0.132327
\end{pmatrix} =
0.998567\cdot
\begin{pmatrix}
0.561755\\
\color{gr} 0.265034\color{black} \\
0.042129\\
0.132517
\end{pmatrix},
\end{equation*}
\begin{equation*}
\left[ \frac{{w}^{\prime}_i}{{w}^{\prime}_j} \right] =
\begin{pmatrix}
$\,\,$ 1 $\,\,$ & $\,\,$\color{gr} 2.1196\color{black} $\,\,$ & $\,\,$13.3342$\,\,$ & $\,\,$4.2391$\,\,$ \\
$\,\,$\color{gr} 0.4718\color{black} $\,\,$ & $\,\,$ 1 $\,\,$ & $\,\,$\color{gr} 6.2910\color{black} $\,\,$ & $\,\,$\color{gr} \color{blue} 2\color{black}   $\,\,$ \\
$\,\,$0.0750$\,\,$ & $\,\,$\color{gr} 0.1590\color{black} $\,\,$ & $\,\,$ 1 $\,\,$ & $\,\,$0.3179 $\,\,$ \\
$\,\,$0.2359$\,\,$ & $\,\,$\color{gr} \color{blue}  1/2\color{black} $\,\,$ & $\,\,$3.1455$\,\,$ & $\,\,$ 1  $\,\,$ \\
\end{pmatrix},
\end{equation*}
\end{example}
\newpage
\begin{example}
\begin{equation*}
\mathbf{A} =
\begin{pmatrix}
$\,\,$ 1 $\,\,$ & $\,\,$2$\,\,$ & $\,\,$8$\,\,$ & $\,\,$7 $\,\,$ \\
$\,\,$ 1/2$\,\,$ & $\,\,$ 1 $\,\,$ & $\,\,$7$\,\,$ & $\,\,$2 $\,\,$ \\
$\,\,$ 1/8$\,\,$ & $\,\,$ 1/7$\,\,$ & $\,\,$ 1 $\,\,$ & $\,\,$ 1/6 $\,\,$ \\
$\,\,$ 1/7$\,\,$ & $\,\,$ 1/2$\,\,$ & $\,\,$6$\,\,$ & $\,\,$ 1  $\,\,$ \\
\end{pmatrix},
\qquad
\lambda_{\max} =
4.2395,
\qquad
CR = 0.0903
\end{equation*}

\begin{equation*}
\mathbf{w}^{EM} =
\begin{pmatrix}
0.560804\\
\color{red} 0.259582\color{black} \\
0.040256\\
0.139357
\end{pmatrix}\end{equation*}
\begin{equation*}
\left[ \frac{{w}^{EM}_i}{{w}^{EM}_j} \right] =
\begin{pmatrix}
$\,\,$ 1 $\,\,$ & $\,\,$\color{red} 2.1604\color{black} $\,\,$ & $\,\,$13.9308$\,\,$ & $\,\,$4.0242$\,\,$ \\
$\,\,$\color{red} 0.4629\color{black} $\,\,$ & $\,\,$ 1 $\,\,$ & $\,\,$\color{red} 6.4482\color{black} $\,\,$ & $\,\,$\color{red} 1.8627\color{black}   $\,\,$ \\
$\,\,$0.0718$\,\,$ & $\,\,$\color{red} 0.1551\color{black} $\,\,$ & $\,\,$ 1 $\,\,$ & $\,\,$0.2889 $\,\,$ \\
$\,\,$0.2485$\,\,$ & $\,\,$\color{red} 0.5369\color{black} $\,\,$ & $\,\,$3.4617$\,\,$ & $\,\,$ 1  $\,\,$ \\
\end{pmatrix},
\end{equation*}

\begin{equation*}
\mathbf{w}^{\prime} =
\begin{pmatrix}
0.550277\\
0.273482\\
0.039501\\
0.136741
\end{pmatrix} =
0.981227\cdot
\begin{pmatrix}
0.560804\\
\color{gr} 0.278714\color{black} \\
0.040256\\
0.139357
\end{pmatrix},
\end{equation*}
\begin{equation*}
\left[ \frac{{w}^{\prime}_i}{{w}^{\prime}_j} \right] =
\begin{pmatrix}
$\,\,$ 1 $\,\,$ & $\,\,$\color{gr} 2.0121\color{black} $\,\,$ & $\,\,$13.9308$\,\,$ & $\,\,$4.0242$\,\,$ \\
$\,\,$\color{gr} 0.4970\color{black} $\,\,$ & $\,\,$ 1 $\,\,$ & $\,\,$\color{gr} 6.9235\color{black} $\,\,$ & $\,\,$\color{gr} \color{blue} 2\color{black}   $\,\,$ \\
$\,\,$0.0718$\,\,$ & $\,\,$\color{gr} 0.1444\color{black} $\,\,$ & $\,\,$ 1 $\,\,$ & $\,\,$0.2889 $\,\,$ \\
$\,\,$0.2485$\,\,$ & $\,\,$\color{gr} \color{blue}  1/2\color{black} $\,\,$ & $\,\,$3.4617$\,\,$ & $\,\,$ 1  $\,\,$ \\
\end{pmatrix},
\end{equation*}
\end{example}
\newpage
\begin{example}
\begin{equation*}
\mathbf{A} =
\begin{pmatrix}
$\,\,$ 1 $\,\,$ & $\,\,$2$\,\,$ & $\,\,$8$\,\,$ & $\,\,$7 $\,\,$ \\
$\,\,$ 1/2$\,\,$ & $\,\,$ 1 $\,\,$ & $\,\,$8$\,\,$ & $\,\,$2 $\,\,$ \\
$\,\,$ 1/8$\,\,$ & $\,\,$ 1/8$\,\,$ & $\,\,$ 1 $\,\,$ & $\,\,$ 1/6 $\,\,$ \\
$\,\,$ 1/7$\,\,$ & $\,\,$ 1/2$\,\,$ & $\,\,$6$\,\,$ & $\,\,$ 1  $\,\,$ \\
\end{pmatrix},
\qquad
\lambda_{\max} =
4.2421,
\qquad
CR = 0.0913
\end{equation*}

\begin{equation*}
\mathbf{w}^{EM} =
\begin{pmatrix}
0.557120\\
\color{red} 0.266547\color{black} \\
0.038825\\
0.137507
\end{pmatrix}\end{equation*}
\begin{equation*}
\left[ \frac{{w}^{EM}_i}{{w}^{EM}_j} \right] =
\begin{pmatrix}
$\,\,$ 1 $\,\,$ & $\,\,$\color{red} 2.0901\color{black} $\,\,$ & $\,\,$14.3494$\,\,$ & $\,\,$4.0516$\,\,$ \\
$\,\,$\color{red} 0.4784\color{black} $\,\,$ & $\,\,$ 1 $\,\,$ & $\,\,$\color{red} 6.8653\color{black} $\,\,$ & $\,\,$\color{red} 1.9384\color{black}   $\,\,$ \\
$\,\,$0.0697$\,\,$ & $\,\,$\color{red} 0.1457\color{black} $\,\,$ & $\,\,$ 1 $\,\,$ & $\,\,$0.2824 $\,\,$ \\
$\,\,$0.2468$\,\,$ & $\,\,$\color{red} 0.5159\color{black} $\,\,$ & $\,\,$3.5417$\,\,$ & $\,\,$ 1  $\,\,$ \\
\end{pmatrix},
\end{equation*}

\begin{equation*}
\mathbf{w}^{\prime} =
\begin{pmatrix}
0.552443\\
0.272705\\
0.038499\\
0.136353
\end{pmatrix} =
0.991604\cdot
\begin{pmatrix}
0.557120\\
\color{gr} 0.275014\color{black} \\
0.038825\\
0.137507
\end{pmatrix},
\end{equation*}
\begin{equation*}
\left[ \frac{{w}^{\prime}_i}{{w}^{\prime}_j} \right] =
\begin{pmatrix}
$\,\,$ 1 $\,\,$ & $\,\,$\color{gr} 2.0258\color{black} $\,\,$ & $\,\,$14.3494$\,\,$ & $\,\,$4.0516$\,\,$ \\
$\,\,$\color{gr} 0.4936\color{black} $\,\,$ & $\,\,$ 1 $\,\,$ & $\,\,$\color{gr} 7.0834\color{black} $\,\,$ & $\,\,$\color{gr} \color{blue} 2\color{black}   $\,\,$ \\
$\,\,$0.0697$\,\,$ & $\,\,$\color{gr} 0.1412\color{black} $\,\,$ & $\,\,$ 1 $\,\,$ & $\,\,$0.2824 $\,\,$ \\
$\,\,$0.2468$\,\,$ & $\,\,$\color{gr} \color{blue}  1/2\color{black} $\,\,$ & $\,\,$3.5417$\,\,$ & $\,\,$ 1  $\,\,$ \\
\end{pmatrix},
\end{equation*}
\end{example}
\newpage
\begin{example}
\begin{equation*}
\mathbf{A} =
\begin{pmatrix}
$\,\,$ 1 $\,\,$ & $\,\,$2$\,\,$ & $\,\,$8$\,\,$ & $\,\,$8 $\,\,$ \\
$\,\,$ 1/2$\,\,$ & $\,\,$ 1 $\,\,$ & $\,\,$6$\,\,$ & $\,\,$3 $\,\,$ \\
$\,\,$ 1/8$\,\,$ & $\,\,$ 1/6$\,\,$ & $\,\,$ 1 $\,\,$ & $\,\,$ 1/3 $\,\,$ \\
$\,\,$ 1/8$\,\,$ & $\,\,$ 1/3$\,\,$ & $\,\,$3$\,\,$ & $\,\,$ 1  $\,\,$ \\
\end{pmatrix},
\qquad
\lambda_{\max} =
4.1031,
\qquad
CR = 0.0389
\end{equation*}

\begin{equation*}
\mathbf{w}^{EM} =
\begin{pmatrix}
0.567538\\
\color{red} 0.283091\color{black} \\
0.048862\\
0.100509
\end{pmatrix}\end{equation*}
\begin{equation*}
\left[ \frac{{w}^{EM}_i}{{w}^{EM}_j} \right] =
\begin{pmatrix}
$\,\,$ 1 $\,\,$ & $\,\,$\color{red} 2.0048\color{black} $\,\,$ & $\,\,$11.6150$\,\,$ & $\,\,$5.6467$\,\,$ \\
$\,\,$\color{red} 0.4988\color{black} $\,\,$ & $\,\,$ 1 $\,\,$ & $\,\,$\color{red} 5.7936\color{black} $\,\,$ & $\,\,$\color{red} 2.8166\color{black}   $\,\,$ \\
$\,\,$0.0861$\,\,$ & $\,\,$\color{red} 0.1726\color{black} $\,\,$ & $\,\,$ 1 $\,\,$ & $\,\,$0.4862 $\,\,$ \\
$\,\,$0.1771$\,\,$ & $\,\,$\color{red} 0.3550\color{black} $\,\,$ & $\,\,$2.0570$\,\,$ & $\,\,$ 1  $\,\,$ \\
\end{pmatrix},
\end{equation*}

\begin{equation*}
\mathbf{w}^{\prime} =
\begin{pmatrix}
0.567153\\
0.283577\\
0.048829\\
0.100441
\end{pmatrix} =
0.999322\cdot
\begin{pmatrix}
0.567538\\
\color{gr} 0.283769\color{black} \\
0.048862\\
0.100509
\end{pmatrix},
\end{equation*}
\begin{equation*}
\left[ \frac{{w}^{\prime}_i}{{w}^{\prime}_j} \right] =
\begin{pmatrix}
$\,\,$ 1 $\,\,$ & $\,\,$\color{gr} \color{blue} 2\color{black} $\,\,$ & $\,\,$11.6150$\,\,$ & $\,\,$5.6467$\,\,$ \\
$\,\,$\color{gr} \color{blue}  1/2\color{black} $\,\,$ & $\,\,$ 1 $\,\,$ & $\,\,$\color{gr} 5.8075\color{black} $\,\,$ & $\,\,$\color{gr} 2.8233\color{black}   $\,\,$ \\
$\,\,$0.0861$\,\,$ & $\,\,$\color{gr} 0.1722\color{black} $\,\,$ & $\,\,$ 1 $\,\,$ & $\,\,$0.4862 $\,\,$ \\
$\,\,$0.1771$\,\,$ & $\,\,$\color{gr} 0.3542\color{black} $\,\,$ & $\,\,$2.0570$\,\,$ & $\,\,$ 1  $\,\,$ \\
\end{pmatrix},
\end{equation*}
\end{example}
\newpage
\begin{example}
\begin{equation*}
\mathbf{A} =
\begin{pmatrix}
$\,\,$ 1 $\,\,$ & $\,\,$2$\,\,$ & $\,\,$9$\,\,$ & $\,\,$2 $\,\,$ \\
$\,\,$ 1/2$\,\,$ & $\,\,$ 1 $\,\,$ & $\,\,$3$\,\,$ & $\,\,$2 $\,\,$ \\
$\,\,$ 1/9$\,\,$ & $\,\,$ 1/3$\,\,$ & $\,\,$ 1 $\,\,$ & $\,\,$ 1/7 $\,\,$ \\
$\,\,$ 1/2$\,\,$ & $\,\,$ 1/2$\,\,$ & $\,\,$7$\,\,$ & $\,\,$ 1  $\,\,$ \\
\end{pmatrix},
\qquad
\lambda_{\max} =
4.2086,
\qquad
CR = 0.0786
\end{equation*}

\begin{equation*}
\mathbf{w}^{EM} =
\begin{pmatrix}
\color{red} 0.455559\color{black} \\
0.263013\\
0.053259\\
0.228169
\end{pmatrix}\end{equation*}
\begin{equation*}
\left[ \frac{{w}^{EM}_i}{{w}^{EM}_j} \right] =
\begin{pmatrix}
$\,\,$ 1 $\,\,$ & $\,\,$\color{red} 1.7321\color{black} $\,\,$ & $\,\,$\color{red} 8.5537\color{black} $\,\,$ & $\,\,$\color{red} 1.9966\color{black} $\,\,$ \\
$\,\,$\color{red} 0.5773\color{black} $\,\,$ & $\,\,$ 1 $\,\,$ & $\,\,$4.9384$\,\,$ & $\,\,$1.1527  $\,\,$ \\
$\,\,$\color{red} 0.1169\color{black} $\,\,$ & $\,\,$0.2025$\,\,$ & $\,\,$ 1 $\,\,$ & $\,\,$0.2334 $\,\,$ \\
$\,\,$\color{red} 0.5009\color{black} $\,\,$ & $\,\,$0.8675$\,\,$ & $\,\,$4.2842$\,\,$ & $\,\,$ 1  $\,\,$ \\
\end{pmatrix},
\end{equation*}

\begin{equation*}
\mathbf{w}^{\prime} =
\begin{pmatrix}
0.455983\\
0.262808\\
0.053217\\
0.227992
\end{pmatrix} =
0.999221\cdot
\begin{pmatrix}
\color{gr} 0.456339\color{black} \\
0.263013\\
0.053259\\
0.228169
\end{pmatrix},
\end{equation*}
\begin{equation*}
\left[ \frac{{w}^{\prime}_i}{{w}^{\prime}_j} \right] =
\begin{pmatrix}
$\,\,$ 1 $\,\,$ & $\,\,$\color{gr} 1.7350\color{black} $\,\,$ & $\,\,$\color{gr} 8.5683\color{black} $\,\,$ & $\,\,$\color{gr} \color{blue} 2\color{black} $\,\,$ \\
$\,\,$\color{gr} 0.5764\color{black} $\,\,$ & $\,\,$ 1 $\,\,$ & $\,\,$4.9384$\,\,$ & $\,\,$1.1527  $\,\,$ \\
$\,\,$\color{gr} 0.1167\color{black} $\,\,$ & $\,\,$0.2025$\,\,$ & $\,\,$ 1 $\,\,$ & $\,\,$0.2334 $\,\,$ \\
$\,\,$\color{gr} \color{blue}  1/2\color{black} $\,\,$ & $\,\,$0.8675$\,\,$ & $\,\,$4.2842$\,\,$ & $\,\,$ 1  $\,\,$ \\
\end{pmatrix},
\end{equation*}
\end{example}
\newpage
\begin{example}
\begin{equation*}
\mathbf{A} =
\begin{pmatrix}
$\,\,$ 1 $\,\,$ & $\,\,$2$\,\,$ & $\,\,$9$\,\,$ & $\,\,$2 $\,\,$ \\
$\,\,$ 1/2$\,\,$ & $\,\,$ 1 $\,\,$ & $\,\,$3$\,\,$ & $\,\,$2 $\,\,$ \\
$\,\,$ 1/9$\,\,$ & $\,\,$ 1/3$\,\,$ & $\,\,$ 1 $\,\,$ & $\,\,$ 1/8 $\,\,$ \\
$\,\,$ 1/2$\,\,$ & $\,\,$ 1/2$\,\,$ & $\,\,$8$\,\,$ & $\,\,$ 1  $\,\,$ \\
\end{pmatrix},
\qquad
\lambda_{\max} =
4.2469,
\qquad
CR = 0.0931
\end{equation*}

\begin{equation*}
\mathbf{w}^{EM} =
\begin{pmatrix}
\color{red} 0.449790\color{black} \\
0.262391\\
0.051430\\
0.236388
\end{pmatrix}\end{equation*}
\begin{equation*}
\left[ \frac{{w}^{EM}_i}{{w}^{EM}_j} \right] =
\begin{pmatrix}
$\,\,$ 1 $\,\,$ & $\,\,$\color{red} 1.7142\color{black} $\,\,$ & $\,\,$\color{red} 8.7457\color{black} $\,\,$ & $\,\,$\color{red} 1.9028\color{black} $\,\,$ \\
$\,\,$\color{red} 0.5834\color{black} $\,\,$ & $\,\,$ 1 $\,\,$ & $\,\,$5.1019$\,\,$ & $\,\,$1.1100  $\,\,$ \\
$\,\,$\color{red} 0.1143\color{black} $\,\,$ & $\,\,$0.1960$\,\,$ & $\,\,$ 1 $\,\,$ & $\,\,$0.2176 $\,\,$ \\
$\,\,$\color{red} 0.5256\color{black} $\,\,$ & $\,\,$0.9009$\,\,$ & $\,\,$4.5963$\,\,$ & $\,\,$ 1  $\,\,$ \\
\end{pmatrix},
\end{equation*}

\begin{equation*}
\mathbf{w}^{\prime} =
\begin{pmatrix}
0.456894\\
0.259003\\
0.050766\\
0.233336
\end{pmatrix} =
0.987088\cdot
\begin{pmatrix}
\color{gr} 0.462871\color{black} \\
0.262391\\
0.051430\\
0.236388
\end{pmatrix},
\end{equation*}
\begin{equation*}
\left[ \frac{{w}^{\prime}_i}{{w}^{\prime}_j} \right] =
\begin{pmatrix}
$\,\,$ 1 $\,\,$ & $\,\,$\color{gr} 1.7640\color{black} $\,\,$ & $\,\,$\color{gr} \color{blue} 9\color{black} $\,\,$ & $\,\,$\color{gr} 1.9581\color{black} $\,\,$ \\
$\,\,$\color{gr} 0.5669\color{black} $\,\,$ & $\,\,$ 1 $\,\,$ & $\,\,$5.1019$\,\,$ & $\,\,$1.1100  $\,\,$ \\
$\,\,$\color{gr} \color{blue}  1/9\color{black} $\,\,$ & $\,\,$0.1960$\,\,$ & $\,\,$ 1 $\,\,$ & $\,\,$0.2176 $\,\,$ \\
$\,\,$\color{gr} 0.5107\color{black} $\,\,$ & $\,\,$0.9009$\,\,$ & $\,\,$4.5963$\,\,$ & $\,\,$ 1  $\,\,$ \\
\end{pmatrix},
\end{equation*}
\end{example}
\newpage
\begin{example}
\begin{equation*}
\mathbf{A} =
\begin{pmatrix}
$\,\,$ 1 $\,\,$ & $\,\,$2$\,\,$ & $\,\,$9$\,\,$ & $\,\,$2 $\,\,$ \\
$\,\,$ 1/2$\,\,$ & $\,\,$ 1 $\,\,$ & $\,\,$6$\,\,$ & $\,\,$3 $\,\,$ \\
$\,\,$ 1/9$\,\,$ & $\,\,$ 1/6$\,\,$ & $\,\,$ 1 $\,\,$ & $\,\,$ 1/3 $\,\,$ \\
$\,\,$ 1/2$\,\,$ & $\,\,$ 1/3$\,\,$ & $\,\,$3$\,\,$ & $\,\,$ 1  $\,\,$ \\
\end{pmatrix},
\qquad
\lambda_{\max} =
4.1031,
\qquad
CR = 0.0389
\end{equation*}

\begin{equation*}
\mathbf{w}^{EM} =
\begin{pmatrix}
0.462190\\
0.327408\\
\color{red} 0.051232\color{black} \\
0.159170
\end{pmatrix}\end{equation*}
\begin{equation*}
\left[ \frac{{w}^{EM}_i}{{w}^{EM}_j} \right] =
\begin{pmatrix}
$\,\,$ 1 $\,\,$ & $\,\,$1.4117$\,\,$ & $\,\,$\color{red} 9.0216\color{black} $\,\,$ & $\,\,$2.9038$\,\,$ \\
$\,\,$0.7084$\,\,$ & $\,\,$ 1 $\,\,$ & $\,\,$\color{red} 6.3907\color{black} $\,\,$ & $\,\,$2.0570  $\,\,$ \\
$\,\,$\color{red} 0.1108\color{black} $\,\,$ & $\,\,$\color{red} 0.1565\color{black} $\,\,$ & $\,\,$ 1 $\,\,$ & $\,\,$\color{red} 0.3219\color{black}  $\,\,$ \\
$\,\,$0.3444$\,\,$ & $\,\,$0.4862$\,\,$ & $\,\,$\color{red} 3.1069\color{black} $\,\,$ & $\,\,$ 1  $\,\,$ \\
\end{pmatrix},
\end{equation*}

\begin{equation*}
\mathbf{w}^{\prime} =
\begin{pmatrix}
0.462133\\
0.327368\\
0.051348\\
0.159150
\end{pmatrix} =
0.999877\cdot
\begin{pmatrix}
0.462190\\
0.327408\\
\color{gr} 0.051354\color{black} \\
0.159170
\end{pmatrix},
\end{equation*}
\begin{equation*}
\left[ \frac{{w}^{\prime}_i}{{w}^{\prime}_j} \right] =
\begin{pmatrix}
$\,\,$ 1 $\,\,$ & $\,\,$1.4117$\,\,$ & $\,\,$\color{gr} \color{blue} 9\color{black} $\,\,$ & $\,\,$2.9038$\,\,$ \\
$\,\,$0.7084$\,\,$ & $\,\,$ 1 $\,\,$ & $\,\,$\color{gr} 6.3755\color{black} $\,\,$ & $\,\,$2.0570  $\,\,$ \\
$\,\,$\color{gr} \color{blue}  1/9\color{black} $\,\,$ & $\,\,$\color{gr} 0.1569\color{black} $\,\,$ & $\,\,$ 1 $\,\,$ & $\,\,$\color{gr} 0.3226\color{black}  $\,\,$ \\
$\,\,$0.3444$\,\,$ & $\,\,$0.4862$\,\,$ & $\,\,$\color{gr} 3.0994\color{black} $\,\,$ & $\,\,$ 1  $\,\,$ \\
\end{pmatrix},
\end{equation*}
\end{example}
\newpage
\begin{example}
\begin{equation*}
\mathbf{A} =
\begin{pmatrix}
$\,\,$ 1 $\,\,$ & $\,\,$2$\,\,$ & $\,\,$9$\,\,$ & $\,\,$2 $\,\,$ \\
$\,\,$ 1/2$\,\,$ & $\,\,$ 1 $\,\,$ & $\,\,$7$\,\,$ & $\,\,$4 $\,\,$ \\
$\,\,$ 1/9$\,\,$ & $\,\,$ 1/7$\,\,$ & $\,\,$ 1 $\,\,$ & $\,\,$ 1/3 $\,\,$ \\
$\,\,$ 1/2$\,\,$ & $\,\,$ 1/4$\,\,$ & $\,\,$3$\,\,$ & $\,\,$ 1  $\,\,$ \\
\end{pmatrix},
\qquad
\lambda_{\max} =
4.1658,
\qquad
CR = 0.0625
\end{equation*}

\begin{equation*}
\mathbf{w}^{EM} =
\begin{pmatrix}
0.451052\\
0.357639\\
\color{red} 0.047148\color{black} \\
0.144161
\end{pmatrix}\end{equation*}
\begin{equation*}
\left[ \frac{{w}^{EM}_i}{{w}^{EM}_j} \right] =
\begin{pmatrix}
$\,\,$ 1 $\,\,$ & $\,\,$1.2612$\,\,$ & $\,\,$\color{red} 9.5666\color{black} $\,\,$ & $\,\,$3.1288$\,\,$ \\
$\,\,$0.7929$\,\,$ & $\,\,$ 1 $\,\,$ & $\,\,$\color{red} 7.5854\color{black} $\,\,$ & $\,\,$2.4808  $\,\,$ \\
$\,\,$\color{red} 0.1045\color{black} $\,\,$ & $\,\,$\color{red} 0.1318\color{black} $\,\,$ & $\,\,$ 1 $\,\,$ & $\,\,$\color{red} 0.3271\color{black}  $\,\,$ \\
$\,\,$0.3196$\,\,$ & $\,\,$0.4031$\,\,$ & $\,\,$\color{red} 3.0576\color{black} $\,\,$ & $\,\,$ 1  $\,\,$ \\
\end{pmatrix},
\end{equation*}

\begin{equation*}
\mathbf{w}^{\prime} =
\begin{pmatrix}
0.450644\\
0.357315\\
0.048010\\
0.144030
\end{pmatrix} =
0.999096\cdot
\begin{pmatrix}
0.451052\\
0.357639\\
\color{gr} 0.048054\color{black} \\
0.144161
\end{pmatrix},
\end{equation*}
\begin{equation*}
\left[ \frac{{w}^{\prime}_i}{{w}^{\prime}_j} \right] =
\begin{pmatrix}
$\,\,$ 1 $\,\,$ & $\,\,$1.2612$\,\,$ & $\,\,$\color{gr} 9.3865\color{black} $\,\,$ & $\,\,$3.1288$\,\,$ \\
$\,\,$0.7929$\,\,$ & $\,\,$ 1 $\,\,$ & $\,\,$\color{gr} 7.4425\color{black} $\,\,$ & $\,\,$2.4808  $\,\,$ \\
$\,\,$\color{gr} 0.1065\color{black} $\,\,$ & $\,\,$\color{gr} 0.1344\color{black} $\,\,$ & $\,\,$ 1 $\,\,$ & $\,\,$\color{gr} \color{blue}  1/3\color{black}  $\,\,$ \\
$\,\,$0.3196$\,\,$ & $\,\,$0.4031$\,\,$ & $\,\,$\color{gr} \color{blue} 3\color{black} $\,\,$ & $\,\,$ 1  $\,\,$ \\
\end{pmatrix},
\end{equation*}
\end{example}
\newpage
\begin{example}
\begin{equation*}
\mathbf{A} =
\begin{pmatrix}
$\,\,$ 1 $\,\,$ & $\,\,$2$\,\,$ & $\,\,$9$\,\,$ & $\,\,$2 $\,\,$ \\
$\,\,$ 1/2$\,\,$ & $\,\,$ 1 $\,\,$ & $\,\,$8$\,\,$ & $\,\,$4 $\,\,$ \\
$\,\,$ 1/9$\,\,$ & $\,\,$ 1/8$\,\,$ & $\,\,$ 1 $\,\,$ & $\,\,$ 1/3 $\,\,$ \\
$\,\,$ 1/2$\,\,$ & $\,\,$ 1/4$\,\,$ & $\,\,$3$\,\,$ & $\,\,$ 1  $\,\,$ \\
\end{pmatrix},
\qquad
\lambda_{\max} =
4.1664,
\qquad
CR = 0.0627
\end{equation*}

\begin{equation*}
\mathbf{w}^{EM} =
\begin{pmatrix}
0.448211\\
0.364426\\
\color{red} 0.045092\color{black} \\
0.142271
\end{pmatrix}\end{equation*}
\begin{equation*}
\left[ \frac{{w}^{EM}_i}{{w}^{EM}_j} \right] =
\begin{pmatrix}
$\,\,$ 1 $\,\,$ & $\,\,$1.2299$\,\,$ & $\,\,$\color{red} 9.9400\color{black} $\,\,$ & $\,\,$3.1504$\,\,$ \\
$\,\,$0.8131$\,\,$ & $\,\,$ 1 $\,\,$ & $\,\,$\color{red} 8.0819\color{black} $\,\,$ & $\,\,$2.5615  $\,\,$ \\
$\,\,$\color{red} 0.1006\color{black} $\,\,$ & $\,\,$\color{red} 0.1237\color{black} $\,\,$ & $\,\,$ 1 $\,\,$ & $\,\,$\color{red} 0.3169\color{black}  $\,\,$ \\
$\,\,$0.3174$\,\,$ & $\,\,$0.3904$\,\,$ & $\,\,$\color{red} 3.1552\color{black} $\,\,$ & $\,\,$ 1  $\,\,$ \\
\end{pmatrix},
\end{equation*}

\begin{equation*}
\mathbf{w}^{\prime} =
\begin{pmatrix}
0.448004\\
0.364258\\
0.045532\\
0.142205
\end{pmatrix} =
0.999538\cdot
\begin{pmatrix}
0.448211\\
0.364426\\
\color{gr} 0.045553\color{black} \\
0.142271
\end{pmatrix},
\end{equation*}
\begin{equation*}
\left[ \frac{{w}^{\prime}_i}{{w}^{\prime}_j} \right] =
\begin{pmatrix}
$\,\,$ 1 $\,\,$ & $\,\,$1.2299$\,\,$ & $\,\,$\color{gr} 9.8393\color{black} $\,\,$ & $\,\,$3.1504$\,\,$ \\
$\,\,$0.8131$\,\,$ & $\,\,$ 1 $\,\,$ & $\,\,$\color{gr} \color{blue} 8\color{black} $\,\,$ & $\,\,$2.5615  $\,\,$ \\
$\,\,$\color{gr} 0.1016\color{black} $\,\,$ & $\,\,$\color{gr} \color{blue}  1/8\color{black} $\,\,$ & $\,\,$ 1 $\,\,$ & $\,\,$\color{gr} 0.3202\color{black}  $\,\,$ \\
$\,\,$0.3174$\,\,$ & $\,\,$0.3904$\,\,$ & $\,\,$\color{gr} 3.1232\color{black} $\,\,$ & $\,\,$ 1  $\,\,$ \\
\end{pmatrix},
\end{equation*}
\end{example}
\newpage
\begin{example}
\begin{equation*}
\mathbf{A} =
\begin{pmatrix}
$\,\,$ 1 $\,\,$ & $\,\,$2$\,\,$ & $\,\,$9$\,\,$ & $\,\,$2 $\,\,$ \\
$\,\,$ 1/2$\,\,$ & $\,\,$ 1 $\,\,$ & $\,\,$8$\,\,$ & $\,\,$5 $\,\,$ \\
$\,\,$ 1/9$\,\,$ & $\,\,$ 1/8$\,\,$ & $\,\,$ 1 $\,\,$ & $\,\,$ 1/3 $\,\,$ \\
$\,\,$ 1/2$\,\,$ & $\,\,$ 1/5$\,\,$ & $\,\,$3$\,\,$ & $\,\,$ 1  $\,\,$ \\
\end{pmatrix},
\qquad
\lambda_{\max} =
4.2267,
\qquad
CR = 0.0855
\end{equation*}

\begin{equation*}
\mathbf{w}^{EM} =
\begin{pmatrix}
0.441193\\
0.382384\\
\color{red} 0.043715\color{black} \\
0.132709
\end{pmatrix}\end{equation*}
\begin{equation*}
\left[ \frac{{w}^{EM}_i}{{w}^{EM}_j} \right] =
\begin{pmatrix}
$\,\,$ 1 $\,\,$ & $\,\,$1.1538$\,\,$ & $\,\,$\color{red} 10.0926\color{black} $\,\,$ & $\,\,$3.3245$\,\,$ \\
$\,\,$0.8667$\,\,$ & $\,\,$ 1 $\,\,$ & $\,\,$\color{red} 8.7473\color{black} $\,\,$ & $\,\,$2.8814  $\,\,$ \\
$\,\,$\color{red} 0.0991\color{black} $\,\,$ & $\,\,$\color{red} 0.1143\color{black} $\,\,$ & $\,\,$ 1 $\,\,$ & $\,\,$\color{red} 0.3294\color{black}  $\,\,$ \\
$\,\,$0.3008$\,\,$ & $\,\,$0.3471$\,\,$ & $\,\,$\color{red} 3.0358\color{black} $\,\,$ & $\,\,$ 1  $\,\,$ \\
\end{pmatrix},
\end{equation*}

\begin{equation*}
\mathbf{w}^{\prime} =
\begin{pmatrix}
0.440963\\
0.382185\\
0.044213\\
0.132639
\end{pmatrix} =
0.999479\cdot
\begin{pmatrix}
0.441193\\
0.382384\\
\color{gr} 0.044236\color{black} \\
0.132709
\end{pmatrix},
\end{equation*}
\begin{equation*}
\left[ \frac{{w}^{\prime}_i}{{w}^{\prime}_j} \right] =
\begin{pmatrix}
$\,\,$ 1 $\,\,$ & $\,\,$1.1538$\,\,$ & $\,\,$\color{gr} 9.9736\color{black} $\,\,$ & $\,\,$3.3245$\,\,$ \\
$\,\,$0.8667$\,\,$ & $\,\,$ 1 $\,\,$ & $\,\,$\color{gr} 8.6441\color{black} $\,\,$ & $\,\,$2.8814  $\,\,$ \\
$\,\,$\color{gr} 0.1003\color{black} $\,\,$ & $\,\,$\color{gr} 0.1157\color{black} $\,\,$ & $\,\,$ 1 $\,\,$ & $\,\,$\color{gr} \color{blue}  1/3\color{black}  $\,\,$ \\
$\,\,$0.3008$\,\,$ & $\,\,$0.3471$\,\,$ & $\,\,$\color{gr} \color{blue} 3\color{black} $\,\,$ & $\,\,$ 1  $\,\,$ \\
\end{pmatrix},
\end{equation*}
\end{example}
\newpage
\begin{example}
\begin{equation*}
\mathbf{A} =
\begin{pmatrix}
$\,\,$ 1 $\,\,$ & $\,\,$2$\,\,$ & $\,\,$9$\,\,$ & $\,\,$2 $\,\,$ \\
$\,\,$ 1/2$\,\,$ & $\,\,$ 1 $\,\,$ & $\,\,$9$\,\,$ & $\,\,$5 $\,\,$ \\
$\,\,$ 1/9$\,\,$ & $\,\,$ 1/9$\,\,$ & $\,\,$ 1 $\,\,$ & $\,\,$ 1/3 $\,\,$ \\
$\,\,$ 1/2$\,\,$ & $\,\,$ 1/5$\,\,$ & $\,\,$3$\,\,$ & $\,\,$ 1  $\,\,$ \\
\end{pmatrix},
\qquad
\lambda_{\max} =
4.2277,
\qquad
CR = 0.0859
\end{equation*}

\begin{equation*}
\mathbf{w}^{EM} =
\begin{pmatrix}
0.438816\\
0.388119\\
\color{red} 0.042002\color{black} \\
0.131063
\end{pmatrix}\end{equation*}
\begin{equation*}
\left[ \frac{{w}^{EM}_i}{{w}^{EM}_j} \right] =
\begin{pmatrix}
$\,\,$ 1 $\,\,$ & $\,\,$1.1306$\,\,$ & $\,\,$\color{red} 10.4476\color{black} $\,\,$ & $\,\,$3.3481$\,\,$ \\
$\,\,$0.8845$\,\,$ & $\,\,$ 1 $\,\,$ & $\,\,$\color{red} 9.2406\color{black} $\,\,$ & $\,\,$2.9613  $\,\,$ \\
$\,\,$\color{red} 0.0957\color{black} $\,\,$ & $\,\,$\color{red} 0.1082\color{black} $\,\,$ & $\,\,$ 1 $\,\,$ & $\,\,$\color{red} 0.3205\color{black}  $\,\,$ \\
$\,\,$0.2987$\,\,$ & $\,\,$0.3377$\,\,$ & $\,\,$\color{red} 3.1204\color{black} $\,\,$ & $\,\,$ 1  $\,\,$ \\
\end{pmatrix},
\end{equation*}

\begin{equation*}
\mathbf{w}^{\prime} =
\begin{pmatrix}
0.438324\\
0.387683\\
0.043076\\
0.130916
\end{pmatrix} =
0.998879\cdot
\begin{pmatrix}
0.438816\\
0.388119\\
\color{gr} 0.043124\color{black} \\
0.131063
\end{pmatrix},
\end{equation*}
\begin{equation*}
\left[ \frac{{w}^{\prime}_i}{{w}^{\prime}_j} \right] =
\begin{pmatrix}
$\,\,$ 1 $\,\,$ & $\,\,$1.1306$\,\,$ & $\,\,$\color{gr} 10.1756\color{black} $\,\,$ & $\,\,$3.3481$\,\,$ \\
$\,\,$0.8845$\,\,$ & $\,\,$ 1 $\,\,$ & $\,\,$\color{gr} \color{blue} 9\color{black} $\,\,$ & $\,\,$2.9613  $\,\,$ \\
$\,\,$\color{gr} 0.0983\color{black} $\,\,$ & $\,\,$\color{gr} \color{blue}  1/9\color{black} $\,\,$ & $\,\,$ 1 $\,\,$ & $\,\,$\color{gr} 0.3290\color{black}  $\,\,$ \\
$\,\,$0.2987$\,\,$ & $\,\,$0.3377$\,\,$ & $\,\,$\color{gr} 3.0392\color{black} $\,\,$ & $\,\,$ 1  $\,\,$ \\
\end{pmatrix},
\end{equation*}
\end{example}
\newpage
\begin{example}
\begin{equation*}
\mathbf{A} =
\begin{pmatrix}
$\,\,$ 1 $\,\,$ & $\,\,$2$\,\,$ & $\,\,$9$\,\,$ & $\,\,$3 $\,\,$ \\
$\,\,$ 1/2$\,\,$ & $\,\,$ 1 $\,\,$ & $\,\,$3$\,\,$ & $\,\,$2 $\,\,$ \\
$\,\,$ 1/9$\,\,$ & $\,\,$ 1/3$\,\,$ & $\,\,$ 1 $\,\,$ & $\,\,$ 1/5 $\,\,$ \\
$\,\,$ 1/3$\,\,$ & $\,\,$ 1/2$\,\,$ & $\,\,$5$\,\,$ & $\,\,$ 1  $\,\,$ \\
\end{pmatrix},
\qquad
\lambda_{\max} =
4.1252,
\qquad
CR = 0.0472
\end{equation*}

\begin{equation*}
\mathbf{w}^{EM} =
\begin{pmatrix}
\color{red} 0.504029\color{black} \\
0.253804\\
0.056850\\
0.185317
\end{pmatrix}\end{equation*}
\begin{equation*}
\left[ \frac{{w}^{EM}_i}{{w}^{EM}_j} \right] =
\begin{pmatrix}
$\,\,$ 1 $\,\,$ & $\,\,$\color{red} 1.9859\color{black} $\,\,$ & $\,\,$\color{red} 8.8660\color{black} $\,\,$ & $\,\,$\color{red} 2.7198\color{black} $\,\,$ \\
$\,\,$\color{red} 0.5036\color{black} $\,\,$ & $\,\,$ 1 $\,\,$ & $\,\,$4.4645$\,\,$ & $\,\,$1.3696  $\,\,$ \\
$\,\,$\color{red} 0.1128\color{black} $\,\,$ & $\,\,$0.2240$\,\,$ & $\,\,$ 1 $\,\,$ & $\,\,$0.3068 $\,\,$ \\
$\,\,$\color{red} 0.3677\color{black} $\,\,$ & $\,\,$0.7302$\,\,$ & $\,\,$3.2598$\,\,$ & $\,\,$ 1  $\,\,$ \\
\end{pmatrix},
\end{equation*}

\begin{equation*}
\mathbf{w}^{\prime} =
\begin{pmatrix}
0.505798\\
0.252899\\
0.056647\\
0.184656
\end{pmatrix} =
0.996433\cdot
\begin{pmatrix}
\color{gr} 0.507609\color{black} \\
0.253804\\
0.056850\\
0.185317
\end{pmatrix},
\end{equation*}
\begin{equation*}
\left[ \frac{{w}^{\prime}_i}{{w}^{\prime}_j} \right] =
\begin{pmatrix}
$\,\,$ 1 $\,\,$ & $\,\,$\color{gr} \color{blue} 2\color{black} $\,\,$ & $\,\,$\color{gr} 8.9290\color{black} $\,\,$ & $\,\,$\color{gr} 2.7391\color{black} $\,\,$ \\
$\,\,$\color{gr} \color{blue}  1/2\color{black} $\,\,$ & $\,\,$ 1 $\,\,$ & $\,\,$4.4645$\,\,$ & $\,\,$1.3696  $\,\,$ \\
$\,\,$\color{gr} 0.1120\color{black} $\,\,$ & $\,\,$0.2240$\,\,$ & $\,\,$ 1 $\,\,$ & $\,\,$0.3068 $\,\,$ \\
$\,\,$\color{gr} 0.3651\color{black} $\,\,$ & $\,\,$0.7302$\,\,$ & $\,\,$3.2598$\,\,$ & $\,\,$ 1  $\,\,$ \\
\end{pmatrix},
\end{equation*}
\end{example}
\newpage
\begin{example}
\begin{equation*}
\mathbf{A} =
\begin{pmatrix}
$\,\,$ 1 $\,\,$ & $\,\,$2$\,\,$ & $\,\,$9$\,\,$ & $\,\,$3 $\,\,$ \\
$\,\,$ 1/2$\,\,$ & $\,\,$ 1 $\,\,$ & $\,\,$3$\,\,$ & $\,\,$3 $\,\,$ \\
$\,\,$ 1/9$\,\,$ & $\,\,$ 1/3$\,\,$ & $\,\,$ 1 $\,\,$ & $\,\,$ 1/5 $\,\,$ \\
$\,\,$ 1/3$\,\,$ & $\,\,$ 1/3$\,\,$ & $\,\,$5$\,\,$ & $\,\,$ 1  $\,\,$ \\
\end{pmatrix},
\qquad
\lambda_{\max} =
4.2277,
\qquad
CR = 0.0859
\end{equation*}

\begin{equation*}
\mathbf{w}^{EM} =
\begin{pmatrix}
\color{red} 0.490620\color{black} \\
0.284767\\
0.056702\\
0.167911
\end{pmatrix}\end{equation*}
\begin{equation*}
\left[ \frac{{w}^{EM}_i}{{w}^{EM}_j} \right] =
\begin{pmatrix}
$\,\,$ 1 $\,\,$ & $\,\,$\color{red} 1.7229\color{black} $\,\,$ & $\,\,$\color{red} 8.6526\color{black} $\,\,$ & $\,\,$\color{red} 2.9219\color{black} $\,\,$ \\
$\,\,$\color{red} 0.5804\color{black} $\,\,$ & $\,\,$ 1 $\,\,$ & $\,\,$5.0222$\,\,$ & $\,\,$1.6959  $\,\,$ \\
$\,\,$\color{red} 0.1156\color{black} $\,\,$ & $\,\,$0.1991$\,\,$ & $\,\,$ 1 $\,\,$ & $\,\,$0.3377 $\,\,$ \\
$\,\,$\color{red} 0.3422\color{black} $\,\,$ & $\,\,$0.5896$\,\,$ & $\,\,$2.9613$\,\,$ & $\,\,$ 1  $\,\,$ \\
\end{pmatrix},
\end{equation*}

\begin{equation*}
\mathbf{w}^{\prime} =
\begin{pmatrix}
0.497213\\
0.281081\\
0.055968\\
0.165738
\end{pmatrix} =
0.987055\cdot
\begin{pmatrix}
\color{gr} 0.503734\color{black} \\
0.284767\\
0.056702\\
0.167911
\end{pmatrix},
\end{equation*}
\begin{equation*}
\left[ \frac{{w}^{\prime}_i}{{w}^{\prime}_j} \right] =
\begin{pmatrix}
$\,\,$ 1 $\,\,$ & $\,\,$\color{gr} 1.7689\color{black} $\,\,$ & $\,\,$\color{gr} 8.8839\color{black} $\,\,$ & $\,\,$\color{gr} \color{blue} 3\color{black} $\,\,$ \\
$\,\,$\color{gr} 0.5653\color{black} $\,\,$ & $\,\,$ 1 $\,\,$ & $\,\,$5.0222$\,\,$ & $\,\,$1.6959  $\,\,$ \\
$\,\,$\color{gr} 0.1126\color{black} $\,\,$ & $\,\,$0.1991$\,\,$ & $\,\,$ 1 $\,\,$ & $\,\,$0.3377 $\,\,$ \\
$\,\,$\color{gr} \color{blue}  1/3\color{black} $\,\,$ & $\,\,$0.5896$\,\,$ & $\,\,$2.9613$\,\,$ & $\,\,$ 1  $\,\,$ \\
\end{pmatrix},
\end{equation*}
\end{example}
\newpage
\begin{example}
\begin{equation*}
\mathbf{A} =
\begin{pmatrix}
$\,\,$ 1 $\,\,$ & $\,\,$2$\,\,$ & $\,\,$9$\,\,$ & $\,\,$3 $\,\,$ \\
$\,\,$ 1/2$\,\,$ & $\,\,$ 1 $\,\,$ & $\,\,$6$\,\,$ & $\,\,$5 $\,\,$ \\
$\,\,$ 1/9$\,\,$ & $\,\,$ 1/6$\,\,$ & $\,\,$ 1 $\,\,$ & $\,\,$ 1/2 $\,\,$ \\
$\,\,$ 1/3$\,\,$ & $\,\,$ 1/5$\,\,$ & $\,\,$2$\,\,$ & $\,\,$ 1  $\,\,$ \\
\end{pmatrix},
\qquad
\lambda_{\max} =
4.1252,
\qquad
CR = 0.0472
\end{equation*}

\begin{equation*}
\mathbf{w}^{EM} =
\begin{pmatrix}
0.484385\\
0.353677\\
\color{red} 0.053441\color{black} \\
0.108497
\end{pmatrix}\end{equation*}
\begin{equation*}
\left[ \frac{{w}^{EM}_i}{{w}^{EM}_j} \right] =
\begin{pmatrix}
$\,\,$ 1 $\,\,$ & $\,\,$1.3696$\,\,$ & $\,\,$\color{red} 9.0639\color{black} $\,\,$ & $\,\,$4.4645$\,\,$ \\
$\,\,$0.7302$\,\,$ & $\,\,$ 1 $\,\,$ & $\,\,$\color{red} 6.6181\color{black} $\,\,$ & $\,\,$3.2598  $\,\,$ \\
$\,\,$\color{red} 0.1103\color{black} $\,\,$ & $\,\,$\color{red} 0.1511\color{black} $\,\,$ & $\,\,$ 1 $\,\,$ & $\,\,$\color{red} 0.4926\color{black}  $\,\,$ \\
$\,\,$0.2240$\,\,$ & $\,\,$0.3068$\,\,$ & $\,\,$\color{red} 2.0302\color{black} $\,\,$ & $\,\,$ 1  $\,\,$ \\
\end{pmatrix},
\end{equation*}

\begin{equation*}
\mathbf{w}^{\prime} =
\begin{pmatrix}
0.484201\\
0.353543\\
0.053800\\
0.108456
\end{pmatrix} =
0.999621\cdot
\begin{pmatrix}
0.484385\\
0.353677\\
\color{gr} 0.053821\color{black} \\
0.108497
\end{pmatrix},
\end{equation*}
\begin{equation*}
\left[ \frac{{w}^{\prime}_i}{{w}^{\prime}_j} \right] =
\begin{pmatrix}
$\,\,$ 1 $\,\,$ & $\,\,$1.3696$\,\,$ & $\,\,$\color{gr} \color{blue} 9\color{black} $\,\,$ & $\,\,$4.4645$\,\,$ \\
$\,\,$0.7302$\,\,$ & $\,\,$ 1 $\,\,$ & $\,\,$\color{gr} 6.5714\color{black} $\,\,$ & $\,\,$3.2598  $\,\,$ \\
$\,\,$\color{gr} \color{blue}  1/9\color{black} $\,\,$ & $\,\,$\color{gr} 0.1522\color{black} $\,\,$ & $\,\,$ 1 $\,\,$ & $\,\,$\color{gr} 0.4961\color{black}  $\,\,$ \\
$\,\,$0.2240$\,\,$ & $\,\,$0.3068$\,\,$ & $\,\,$\color{gr} 2.0159\color{black} $\,\,$ & $\,\,$ 1  $\,\,$ \\
\end{pmatrix},
\end{equation*}
\end{example}
\newpage
\begin{example}
\begin{equation*}
\mathbf{A} =
\begin{pmatrix}
$\,\,$ 1 $\,\,$ & $\,\,$2$\,\,$ & $\,\,$9$\,\,$ & $\,\,$3 $\,\,$ \\
$\,\,$ 1/2$\,\,$ & $\,\,$ 1 $\,\,$ & $\,\,$7$\,\,$ & $\,\,$5 $\,\,$ \\
$\,\,$ 1/9$\,\,$ & $\,\,$ 1/7$\,\,$ & $\,\,$ 1 $\,\,$ & $\,\,$ 1/2 $\,\,$ \\
$\,\,$ 1/3$\,\,$ & $\,\,$ 1/5$\,\,$ & $\,\,$2$\,\,$ & $\,\,$ 1  $\,\,$ \\
\end{pmatrix},
\qquad
\lambda_{\max} =
4.1239,
\qquad
CR = 0.0467
\end{equation*}

\begin{equation*}
\mathbf{w}^{EM} =
\begin{pmatrix}
0.480532\\
0.361786\\
\color{red} 0.050751\color{black} \\
0.106930
\end{pmatrix}\end{equation*}
\begin{equation*}
\left[ \frac{{w}^{EM}_i}{{w}^{EM}_j} \right] =
\begin{pmatrix}
$\,\,$ 1 $\,\,$ & $\,\,$1.3282$\,\,$ & $\,\,$\color{red} 9.4684\color{black} $\,\,$ & $\,\,$4.4939$\,\,$ \\
$\,\,$0.7529$\,\,$ & $\,\,$ 1 $\,\,$ & $\,\,$\color{red} 7.1286\color{black} $\,\,$ & $\,\,$3.3834  $\,\,$ \\
$\,\,$\color{red} 0.1056\color{black} $\,\,$ & $\,\,$\color{red} 0.1403\color{black} $\,\,$ & $\,\,$ 1 $\,\,$ & $\,\,$\color{red} 0.4746\color{black}  $\,\,$ \\
$\,\,$0.2225$\,\,$ & $\,\,$0.2956$\,\,$ & $\,\,$\color{red} 2.1069\color{black} $\,\,$ & $\,\,$ 1  $\,\,$ \\
\end{pmatrix},
\end{equation*}

\begin{equation*}
\mathbf{w}^{\prime} =
\begin{pmatrix}
0.480085\\
0.361449\\
0.051636\\
0.106831
\end{pmatrix} =
0.999069\cdot
\begin{pmatrix}
0.480532\\
0.361786\\
\color{gr} 0.051684\color{black} \\
0.106930
\end{pmatrix},
\end{equation*}
\begin{equation*}
\left[ \frac{{w}^{\prime}_i}{{w}^{\prime}_j} \right] =
\begin{pmatrix}
$\,\,$ 1 $\,\,$ & $\,\,$1.3282$\,\,$ & $\,\,$\color{gr} 9.2976\color{black} $\,\,$ & $\,\,$4.4939$\,\,$ \\
$\,\,$0.7529$\,\,$ & $\,\,$ 1 $\,\,$ & $\,\,$\color{gr} \color{blue} 7\color{black} $\,\,$ & $\,\,$3.3834  $\,\,$ \\
$\,\,$\color{gr} 0.1076\color{black} $\,\,$ & $\,\,$\color{gr} \color{blue}  1/7\color{black} $\,\,$ & $\,\,$ 1 $\,\,$ & $\,\,$\color{gr} 0.4833\color{black}  $\,\,$ \\
$\,\,$0.2225$\,\,$ & $\,\,$0.2956$\,\,$ & $\,\,$\color{gr} 2.0689\color{black} $\,\,$ & $\,\,$ 1  $\,\,$ \\
\end{pmatrix},
\end{equation*}
\end{example}
\newpage
\begin{example}
\begin{equation*}
\mathbf{A} =
\begin{pmatrix}
$\,\,$ 1 $\,\,$ & $\,\,$2$\,\,$ & $\,\,$9$\,\,$ & $\,\,$3 $\,\,$ \\
$\,\,$ 1/2$\,\,$ & $\,\,$ 1 $\,\,$ & $\,\,$7$\,\,$ & $\,\,$6 $\,\,$ \\
$\,\,$ 1/9$\,\,$ & $\,\,$ 1/7$\,\,$ & $\,\,$ 1 $\,\,$ & $\,\,$ 1/2 $\,\,$ \\
$\,\,$ 1/3$\,\,$ & $\,\,$ 1/6$\,\,$ & $\,\,$2$\,\,$ & $\,\,$ 1  $\,\,$ \\
\end{pmatrix},
\qquad
\lambda_{\max} =
4.1658,
\qquad
CR = 0.0625
\end{equation*}

\begin{equation*}
\mathbf{w}^{EM} =
\begin{pmatrix}
0.473821\\
0.375692\\
\color{red} 0.049528\color{black} \\
0.100958
\end{pmatrix}\end{equation*}
\begin{equation*}
\left[ \frac{{w}^{EM}_i}{{w}^{EM}_j} \right] =
\begin{pmatrix}
$\,\,$ 1 $\,\,$ & $\,\,$1.2612$\,\,$ & $\,\,$\color{red} 9.5666\color{black} $\,\,$ & $\,\,$4.6932$\,\,$ \\
$\,\,$0.7929$\,\,$ & $\,\,$ 1 $\,\,$ & $\,\,$\color{red} 7.5854\color{black} $\,\,$ & $\,\,$3.7213  $\,\,$ \\
$\,\,$\color{red} 0.1045\color{black} $\,\,$ & $\,\,$\color{red} 0.1318\color{black} $\,\,$ & $\,\,$ 1 $\,\,$ & $\,\,$\color{red} 0.4906\color{black}  $\,\,$ \\
$\,\,$0.2131$\,\,$ & $\,\,$0.2687$\,\,$ & $\,\,$\color{red} 2.0384\color{black} $\,\,$ & $\,\,$ 1  $\,\,$ \\
\end{pmatrix},
\end{equation*}

\begin{equation*}
\mathbf{w}^{\prime} =
\begin{pmatrix}
0.473371\\
0.375335\\
0.050431\\
0.100863
\end{pmatrix} =
0.999050\cdot
\begin{pmatrix}
0.473821\\
0.375692\\
\color{gr} 0.050479\color{black} \\
0.100958
\end{pmatrix},
\end{equation*}
\begin{equation*}
\left[ \frac{{w}^{\prime}_i}{{w}^{\prime}_j} \right] =
\begin{pmatrix}
$\,\,$ 1 $\,\,$ & $\,\,$1.2612$\,\,$ & $\,\,$\color{gr} 9.3865\color{black} $\,\,$ & $\,\,$4.6932$\,\,$ \\
$\,\,$0.7929$\,\,$ & $\,\,$ 1 $\,\,$ & $\,\,$\color{gr} 7.4425\color{black} $\,\,$ & $\,\,$3.7213  $\,\,$ \\
$\,\,$\color{gr} 0.1065\color{black} $\,\,$ & $\,\,$\color{gr} 0.1344\color{black} $\,\,$ & $\,\,$ 1 $\,\,$ & $\,\,$\color{gr} \color{blue}  1/2\color{black}  $\,\,$ \\
$\,\,$0.2131$\,\,$ & $\,\,$0.2687$\,\,$ & $\,\,$\color{gr} \color{blue} 2\color{black} $\,\,$ & $\,\,$ 1  $\,\,$ \\
\end{pmatrix},
\end{equation*}
\end{example}
\newpage
\begin{example}
\begin{equation*}
\mathbf{A} =
\begin{pmatrix}
$\,\,$ 1 $\,\,$ & $\,\,$2$\,\,$ & $\,\,$9$\,\,$ & $\,\,$3 $\,\,$ \\
$\,\,$ 1/2$\,\,$ & $\,\,$ 1 $\,\,$ & $\,\,$8$\,\,$ & $\,\,$6 $\,\,$ \\
$\,\,$ 1/9$\,\,$ & $\,\,$ 1/8$\,\,$ & $\,\,$ 1 $\,\,$ & $\,\,$ 1/2 $\,\,$ \\
$\,\,$ 1/3$\,\,$ & $\,\,$ 1/6$\,\,$ & $\,\,$2$\,\,$ & $\,\,$ 1  $\,\,$ \\
\end{pmatrix},
\qquad
\lambda_{\max} =
4.1664,
\qquad
CR = 0.0627
\end{equation*}

\begin{equation*}
\mathbf{w}^{EM} =
\begin{pmatrix}
0.470525\\
0.382569\\
\color{red} 0.047336\color{black} \\
0.099569
\end{pmatrix}\end{equation*}
\begin{equation*}
\left[ \frac{{w}^{EM}_i}{{w}^{EM}_j} \right] =
\begin{pmatrix}
$\,\,$ 1 $\,\,$ & $\,\,$1.2299$\,\,$ & $\,\,$\color{red} 9.9400\color{black} $\,\,$ & $\,\,$4.7256$\,\,$ \\
$\,\,$0.8131$\,\,$ & $\,\,$ 1 $\,\,$ & $\,\,$\color{red} 8.0819\color{black} $\,\,$ & $\,\,$3.8422  $\,\,$ \\
$\,\,$\color{red} 0.1006\color{black} $\,\,$ & $\,\,$\color{red} 0.1237\color{black} $\,\,$ & $\,\,$ 1 $\,\,$ & $\,\,$\color{red} 0.4754\color{black}  $\,\,$ \\
$\,\,$0.2116$\,\,$ & $\,\,$0.2603$\,\,$ & $\,\,$\color{red} 2.1034\color{black} $\,\,$ & $\,\,$ 1  $\,\,$ \\
\end{pmatrix},
\end{equation*}

\begin{equation*}
\mathbf{w}^{\prime} =
\begin{pmatrix}
0.470297\\
0.382384\\
0.047798\\
0.099521
\end{pmatrix} =
0.999516\cdot
\begin{pmatrix}
0.470525\\
0.382569\\
\color{gr} 0.047821\color{black} \\
0.099569
\end{pmatrix},
\end{equation*}
\begin{equation*}
\left[ \frac{{w}^{\prime}_i}{{w}^{\prime}_j} \right] =
\begin{pmatrix}
$\,\,$ 1 $\,\,$ & $\,\,$1.2299$\,\,$ & $\,\,$\color{gr} 9.8393\color{black} $\,\,$ & $\,\,$4.7256$\,\,$ \\
$\,\,$0.8131$\,\,$ & $\,\,$ 1 $\,\,$ & $\,\,$\color{gr} \color{blue} 8\color{black} $\,\,$ & $\,\,$3.8422  $\,\,$ \\
$\,\,$\color{gr} 0.1016\color{black} $\,\,$ & $\,\,$\color{gr} \color{blue}  1/8\color{black} $\,\,$ & $\,\,$ 1 $\,\,$ & $\,\,$\color{gr} 0.4803\color{black}  $\,\,$ \\
$\,\,$0.2116$\,\,$ & $\,\,$0.2603$\,\,$ & $\,\,$\color{gr} 2.0821\color{black} $\,\,$ & $\,\,$ 1  $\,\,$ \\
\end{pmatrix},
\end{equation*}
\end{example}
\newpage
\begin{example}
\begin{equation*}
\mathbf{A} =
\begin{pmatrix}
$\,\,$ 1 $\,\,$ & $\,\,$2$\,\,$ & $\,\,$9$\,\,$ & $\,\,$3 $\,\,$ \\
$\,\,$ 1/2$\,\,$ & $\,\,$ 1 $\,\,$ & $\,\,$8$\,\,$ & $\,\,$7 $\,\,$ \\
$\,\,$ 1/9$\,\,$ & $\,\,$ 1/8$\,\,$ & $\,\,$ 1 $\,\,$ & $\,\,$ 1/2 $\,\,$ \\
$\,\,$ 1/3$\,\,$ & $\,\,$ 1/7$\,\,$ & $\,\,$2$\,\,$ & $\,\,$ 1  $\,\,$ \\
\end{pmatrix},
\qquad
\lambda_{\max} =
4.2065,
\qquad
CR = 0.0779
\end{equation*}

\begin{equation*}
\mathbf{w}^{EM} =
\begin{pmatrix}
0.464495\\
0.394553\\
\color{red} 0.046244\color{black} \\
0.094708
\end{pmatrix}\end{equation*}
\begin{equation*}
\left[ \frac{{w}^{EM}_i}{{w}^{EM}_j} \right] =
\begin{pmatrix}
$\,\,$ 1 $\,\,$ & $\,\,$1.1773$\,\,$ & $\,\,$\color{red} 10.0444\color{black} $\,\,$ & $\,\,$4.9045$\,\,$ \\
$\,\,$0.8494$\,\,$ & $\,\,$ 1 $\,\,$ & $\,\,$\color{red} 8.5320\color{black} $\,\,$ & $\,\,$4.1660  $\,\,$ \\
$\,\,$\color{red} 0.0996\color{black} $\,\,$ & $\,\,$\color{red} 0.1172\color{black} $\,\,$ & $\,\,$ 1 $\,\,$ & $\,\,$\color{red} 0.4883\color{black}  $\,\,$ \\
$\,\,$0.2039$\,\,$ & $\,\,$0.2400$\,\,$ & $\,\,$\color{red} 2.0480\color{black} $\,\,$ & $\,\,$ 1  $\,\,$ \\
\end{pmatrix},
\end{equation*}

\begin{equation*}
\mathbf{w}^{\prime} =
\begin{pmatrix}
0.463980\\
0.394116\\
0.047301\\
0.094603
\end{pmatrix} =
0.998891\cdot
\begin{pmatrix}
0.464495\\
0.394553\\
\color{gr} 0.047354\color{black} \\
0.094708
\end{pmatrix},
\end{equation*}
\begin{equation*}
\left[ \frac{{w}^{\prime}_i}{{w}^{\prime}_j} \right] =
\begin{pmatrix}
$\,\,$ 1 $\,\,$ & $\,\,$1.1773$\,\,$ & $\,\,$\color{gr} 9.8090\color{black} $\,\,$ & $\,\,$4.9045$\,\,$ \\
$\,\,$0.8494$\,\,$ & $\,\,$ 1 $\,\,$ & $\,\,$\color{gr} 8.3320\color{black} $\,\,$ & $\,\,$4.1660  $\,\,$ \\
$\,\,$\color{gr} 0.1019\color{black} $\,\,$ & $\,\,$\color{gr} 0.1200\color{black} $\,\,$ & $\,\,$ 1 $\,\,$ & $\,\,$\color{gr} \color{blue}  1/2\color{black}  $\,\,$ \\
$\,\,$0.2039$\,\,$ & $\,\,$0.2400$\,\,$ & $\,\,$\color{gr} \color{blue} 2\color{black} $\,\,$ & $\,\,$ 1  $\,\,$ \\
\end{pmatrix},
\end{equation*}
\end{example}
\newpage
\begin{example}
\begin{equation*}
\mathbf{A} =
\begin{pmatrix}
$\,\,$ 1 $\,\,$ & $\,\,$2$\,\,$ & $\,\,$9$\,\,$ & $\,\,$3 $\,\,$ \\
$\,\,$ 1/2$\,\,$ & $\,\,$ 1 $\,\,$ & $\,\,$8$\,\,$ & $\,\,$8 $\,\,$ \\
$\,\,$ 1/9$\,\,$ & $\,\,$ 1/8$\,\,$ & $\,\,$ 1 $\,\,$ & $\,\,$ 1/2 $\,\,$ \\
$\,\,$ 1/3$\,\,$ & $\,\,$ 1/8$\,\,$ & $\,\,$2$\,\,$ & $\,\,$ 1  $\,\,$ \\
\end{pmatrix},
\qquad
\lambda_{\max} =
4.2469,
\qquad
CR = 0.0931
\end{equation*}

\begin{equation*}
\mathbf{w}^{EM} =
\begin{pmatrix}
0.458816\\
0.405345\\
\color{red} 0.045255\color{black} \\
0.090584
\end{pmatrix}\end{equation*}
\begin{equation*}
\left[ \frac{{w}^{EM}_i}{{w}^{EM}_j} \right] =
\begin{pmatrix}
$\,\,$ 1 $\,\,$ & $\,\,$1.1319$\,\,$ & $\,\,$\color{red} 10.1384\color{black} $\,\,$ & $\,\,$5.0651$\,\,$ \\
$\,\,$0.8835$\,\,$ & $\,\,$ 1 $\,\,$ & $\,\,$\color{red} 8.9569\color{black} $\,\,$ & $\,\,$4.4748  $\,\,$ \\
$\,\,$\color{red} 0.0986\color{black} $\,\,$ & $\,\,$\color{red} 0.1116\color{black} $\,\,$ & $\,\,$ 1 $\,\,$ & $\,\,$\color{red} 0.4996\color{black}  $\,\,$ \\
$\,\,$0.1974$\,\,$ & $\,\,$0.2235$\,\,$ & $\,\,$\color{red} 2.0016\color{black} $\,\,$ & $\,\,$ 1  $\,\,$ \\
\end{pmatrix},
\end{equation*}

\begin{equation*}
\mathbf{w}^{\prime} =
\begin{pmatrix}
0.458800\\
0.405330\\
0.045290\\
0.090580
\end{pmatrix} =
0.999963\cdot
\begin{pmatrix}
0.458816\\
0.405345\\
\color{gr} 0.045292\color{black} \\
0.090584
\end{pmatrix},
\end{equation*}
\begin{equation*}
\left[ \frac{{w}^{\prime}_i}{{w}^{\prime}_j} \right] =
\begin{pmatrix}
$\,\,$ 1 $\,\,$ & $\,\,$1.1319$\,\,$ & $\,\,$\color{gr} 10.1302\color{black} $\,\,$ & $\,\,$5.0651$\,\,$ \\
$\,\,$0.8835$\,\,$ & $\,\,$ 1 $\,\,$ & $\,\,$\color{gr} 8.9496\color{black} $\,\,$ & $\,\,$4.4748  $\,\,$ \\
$\,\,$\color{gr} 0.0987\color{black} $\,\,$ & $\,\,$\color{gr} 0.1117\color{black} $\,\,$ & $\,\,$ 1 $\,\,$ & $\,\,$\color{gr} \color{blue}  1/2\color{black}  $\,\,$ \\
$\,\,$0.1974$\,\,$ & $\,\,$0.2235$\,\,$ & $\,\,$\color{gr} \color{blue} 2\color{black} $\,\,$ & $\,\,$ 1  $\,\,$ \\
\end{pmatrix},
\end{equation*}
\end{example}
\newpage
\begin{example}
\begin{equation*}
\mathbf{A} =
\begin{pmatrix}
$\,\,$ 1 $\,\,$ & $\,\,$2$\,\,$ & $\,\,$9$\,\,$ & $\,\,$3 $\,\,$ \\
$\,\,$ 1/2$\,\,$ & $\,\,$ 1 $\,\,$ & $\,\,$9$\,\,$ & $\,\,$7 $\,\,$ \\
$\,\,$ 1/9$\,\,$ & $\,\,$ 1/9$\,\,$ & $\,\,$ 1 $\,\,$ & $\,\,$ 1/2 $\,\,$ \\
$\,\,$ 1/3$\,\,$ & $\,\,$ 1/7$\,\,$ & $\,\,$2$\,\,$ & $\,\,$ 1  $\,\,$ \\
\end{pmatrix},
\qquad
\lambda_{\max} =
4.2086,
\qquad
CR = 0.0786
\end{equation*}

\begin{equation*}
\mathbf{w}^{EM} =
\begin{pmatrix}
0.461629\\
0.400473\\
\color{red} 0.044421\color{black} \\
0.093477
\end{pmatrix}\end{equation*}
\begin{equation*}
\left[ \frac{{w}^{EM}_i}{{w}^{EM}_j} \right] =
\begin{pmatrix}
$\,\,$ 1 $\,\,$ & $\,\,$1.1527$\,\,$ & $\,\,$\color{red} 10.3921\color{black} $\,\,$ & $\,\,$4.9384$\,\,$ \\
$\,\,$0.8675$\,\,$ & $\,\,$ 1 $\,\,$ & $\,\,$\color{red} 9.0154\color{black} $\,\,$ & $\,\,$4.2842  $\,\,$ \\
$\,\,$\color{red} 0.0962\color{black} $\,\,$ & $\,\,$\color{red} 0.1109\color{black} $\,\,$ & $\,\,$ 1 $\,\,$ & $\,\,$\color{red} 0.4752\color{black}  $\,\,$ \\
$\,\,$0.2025$\,\,$ & $\,\,$0.2334$\,\,$ & $\,\,$\color{red} 2.1044\color{black} $\,\,$ & $\,\,$ 1  $\,\,$ \\
\end{pmatrix},
\end{equation*}

\begin{equation*}
\mathbf{w}^{\prime} =
\begin{pmatrix}
0.461594\\
0.400443\\
0.044494\\
0.093470
\end{pmatrix} =
0.999924\cdot
\begin{pmatrix}
0.461629\\
0.400473\\
\color{gr} 0.044497\color{black} \\
0.093477
\end{pmatrix},
\end{equation*}
\begin{equation*}
\left[ \frac{{w}^{\prime}_i}{{w}^{\prime}_j} \right] =
\begin{pmatrix}
$\,\,$ 1 $\,\,$ & $\,\,$1.1527$\,\,$ & $\,\,$\color{gr} 10.3744\color{black} $\,\,$ & $\,\,$4.9384$\,\,$ \\
$\,\,$0.8675$\,\,$ & $\,\,$ 1 $\,\,$ & $\,\,$\color{gr} \color{blue} 9\color{black} $\,\,$ & $\,\,$4.2842  $\,\,$ \\
$\,\,$\color{gr} 0.0964\color{black} $\,\,$ & $\,\,$\color{gr} \color{blue}  1/9\color{black} $\,\,$ & $\,\,$ 1 $\,\,$ & $\,\,$\color{gr} 0.4760\color{black}  $\,\,$ \\
$\,\,$0.2025$\,\,$ & $\,\,$0.2334$\,\,$ & $\,\,$\color{gr} 2.1008\color{black} $\,\,$ & $\,\,$ 1  $\,\,$ \\
\end{pmatrix},
\end{equation*}
\end{example}
\newpage
\begin{example}
\begin{equation*}
\mathbf{A} =
\begin{pmatrix}
$\,\,$ 1 $\,\,$ & $\,\,$2$\,\,$ & $\,\,$9$\,\,$ & $\,\,$3 $\,\,$ \\
$\,\,$ 1/2$\,\,$ & $\,\,$ 1 $\,\,$ & $\,\,$9$\,\,$ & $\,\,$8 $\,\,$ \\
$\,\,$ 1/9$\,\,$ & $\,\,$ 1/9$\,\,$ & $\,\,$ 1 $\,\,$ & $\,\,$ 1/2 $\,\,$ \\
$\,\,$ 1/3$\,\,$ & $\,\,$ 1/8$\,\,$ & $\,\,$2$\,\,$ & $\,\,$ 1  $\,\,$ \\
\end{pmatrix},
\qquad
\lambda_{\max} =
4.2469,
\qquad
CR = 0.0931
\end{equation*}

\begin{equation*}
\mathbf{w}^{EM} =
\begin{pmatrix}
0.456176\\
0.410968\\
\color{red} 0.043443\color{black} \\
0.089413
\end{pmatrix}\end{equation*}
\begin{equation*}
\left[ \frac{{w}^{EM}_i}{{w}^{EM}_j} \right] =
\begin{pmatrix}
$\,\,$ 1 $\,\,$ & $\,\,$1.1100$\,\,$ & $\,\,$\color{red} 10.5005\color{black} $\,\,$ & $\,\,$5.1019$\,\,$ \\
$\,\,$0.9009$\,\,$ & $\,\,$ 1 $\,\,$ & $\,\,$\color{red} 9.4599\color{black} $\,\,$ & $\,\,$4.5963  $\,\,$ \\
$\,\,$\color{red} 0.0952\color{black} $\,\,$ & $\,\,$\color{red} 0.1057\color{black} $\,\,$ & $\,\,$ 1 $\,\,$ & $\,\,$\color{red} 0.4859\color{black}  $\,\,$ \\
$\,\,$0.1960$\,\,$ & $\,\,$0.2176$\,\,$ & $\,\,$\color{red} 2.0582\color{black} $\,\,$ & $\,\,$ 1  $\,\,$ \\
\end{pmatrix},
\end{equation*}

\begin{equation*}
\mathbf{w}^{\prime} =
\begin{pmatrix}
0.455600\\
0.410450\\
0.044650\\
0.089300
\end{pmatrix} =
0.998738\cdot
\begin{pmatrix}
0.456176\\
0.410968\\
\color{gr} 0.044706\color{black} \\
0.089413
\end{pmatrix},
\end{equation*}
\begin{equation*}
\left[ \frac{{w}^{\prime}_i}{{w}^{\prime}_j} \right] =
\begin{pmatrix}
$\,\,$ 1 $\,\,$ & $\,\,$1.1100$\,\,$ & $\,\,$\color{gr} 10.2038\color{black} $\,\,$ & $\,\,$5.1019$\,\,$ \\
$\,\,$0.9009$\,\,$ & $\,\,$ 1 $\,\,$ & $\,\,$\color{gr} 9.1926\color{black} $\,\,$ & $\,\,$4.5963  $\,\,$ \\
$\,\,$\color{gr} 0.0980\color{black} $\,\,$ & $\,\,$\color{gr} 0.1088\color{black} $\,\,$ & $\,\,$ 1 $\,\,$ & $\,\,$\color{gr} \color{blue}  1/2\color{black}  $\,\,$ \\
$\,\,$0.1960$\,\,$ & $\,\,$0.2176$\,\,$ & $\,\,$\color{gr} \color{blue} 2\color{black} $\,\,$ & $\,\,$ 1  $\,\,$ \\
\end{pmatrix},
\end{equation*}
\end{example}
\newpage
\begin{example}
\begin{equation*}
\mathbf{A} =
\begin{pmatrix}
$\,\,$ 1 $\,\,$ & $\,\,$2$\,\,$ & $\,\,$9$\,\,$ & $\,\,$4 $\,\,$ \\
$\,\,$ 1/2$\,\,$ & $\,\,$ 1 $\,\,$ & $\,\,$3$\,\,$ & $\,\,$3 $\,\,$ \\
$\,\,$ 1/9$\,\,$ & $\,\,$ 1/3$\,\,$ & $\,\,$ 1 $\,\,$ & $\,\,$ 1/3 $\,\,$ \\
$\,\,$ 1/4$\,\,$ & $\,\,$ 1/3$\,\,$ & $\,\,$3$\,\,$ & $\,\,$ 1  $\,\,$ \\
\end{pmatrix},
\qquad
\lambda_{\max} =
4.1031,
\qquad
CR = 0.0389
\end{equation*}

\begin{equation*}
\mathbf{w}^{EM} =
\begin{pmatrix}
\color{red} 0.530010\color{black} \\
0.274445\\
0.062725\\
0.132820
\end{pmatrix}\end{equation*}
\begin{equation*}
\left[ \frac{{w}^{EM}_i}{{w}^{EM}_j} \right] =
\begin{pmatrix}
$\,\,$ 1 $\,\,$ & $\,\,$\color{red} 1.9312\color{black} $\,\,$ & $\,\,$\color{red} 8.4497\color{black} $\,\,$ & $\,\,$\color{red} 3.9904\color{black} $\,\,$ \\
$\,\,$\color{red} 0.5178\color{black} $\,\,$ & $\,\,$ 1 $\,\,$ & $\,\,$4.3754$\,\,$ & $\,\,$2.0663  $\,\,$ \\
$\,\,$\color{red} 0.1183\color{black} $\,\,$ & $\,\,$0.2286$\,\,$ & $\,\,$ 1 $\,\,$ & $\,\,$0.4723 $\,\,$ \\
$\,\,$\color{red} 0.2506\color{black} $\,\,$ & $\,\,$0.4840$\,\,$ & $\,\,$2.1175$\,\,$ & $\,\,$ 1  $\,\,$ \\
\end{pmatrix},
\end{equation*}

\begin{equation*}
\mathbf{w}^{\prime} =
\begin{pmatrix}
0.530606\\
0.274097\\
0.062646\\
0.132652
\end{pmatrix} =
0.998731\cdot
\begin{pmatrix}
\color{gr} 0.531280\color{black} \\
0.274445\\
0.062725\\
0.132820
\end{pmatrix},
\end{equation*}
\begin{equation*}
\left[ \frac{{w}^{\prime}_i}{{w}^{\prime}_j} \right] =
\begin{pmatrix}
$\,\,$ 1 $\,\,$ & $\,\,$\color{gr} 1.9358\color{black} $\,\,$ & $\,\,$\color{gr} 8.4700\color{black} $\,\,$ & $\,\,$\color{gr} \color{blue} 4\color{black} $\,\,$ \\
$\,\,$\color{gr} 0.5166\color{black} $\,\,$ & $\,\,$ 1 $\,\,$ & $\,\,$4.3754$\,\,$ & $\,\,$2.0663  $\,\,$ \\
$\,\,$\color{gr} 0.1181\color{black} $\,\,$ & $\,\,$0.2286$\,\,$ & $\,\,$ 1 $\,\,$ & $\,\,$0.4723 $\,\,$ \\
$\,\,$\color{gr} \color{blue}  1/4\color{black} $\,\,$ & $\,\,$0.4840$\,\,$ & $\,\,$2.1175$\,\,$ & $\,\,$ 1  $\,\,$ \\
\end{pmatrix},
\end{equation*}
\end{example}
\newpage
\begin{example}
\begin{equation*}
\mathbf{A} =
\begin{pmatrix}
$\,\,$ 1 $\,\,$ & $\,\,$2$\,\,$ & $\,\,$9$\,\,$ & $\,\,$4 $\,\,$ \\
$\,\,$ 1/2$\,\,$ & $\,\,$ 1 $\,\,$ & $\,\,$3$\,\,$ & $\,\,$3 $\,\,$ \\
$\,\,$ 1/9$\,\,$ & $\,\,$ 1/3$\,\,$ & $\,\,$ 1 $\,\,$ & $\,\,$ 1/4 $\,\,$ \\
$\,\,$ 1/4$\,\,$ & $\,\,$ 1/3$\,\,$ & $\,\,$4$\,\,$ & $\,\,$ 1  $\,\,$ \\
\end{pmatrix},
\qquad
\lambda_{\max} =
4.1664,
\qquad
CR = 0.0627
\end{equation*}

\begin{equation*}
\mathbf{w}^{EM} =
\begin{pmatrix}
\color{red} 0.522406\color{black} \\
0.274712\\
0.058639\\
0.144242
\end{pmatrix}\end{equation*}
\begin{equation*}
\left[ \frac{{w}^{EM}_i}{{w}^{EM}_j} \right] =
\begin{pmatrix}
$\,\,$ 1 $\,\,$ & $\,\,$\color{red} 1.9017\color{black} $\,\,$ & $\,\,$\color{red} 8.9088\color{black} $\,\,$ & $\,\,$\color{red} 3.6217\color{black} $\,\,$ \\
$\,\,$\color{red} 0.5259\color{black} $\,\,$ & $\,\,$ 1 $\,\,$ & $\,\,$4.6848$\,\,$ & $\,\,$1.9045  $\,\,$ \\
$\,\,$\color{red} 0.1122\color{black} $\,\,$ & $\,\,$0.2135$\,\,$ & $\,\,$ 1 $\,\,$ & $\,\,$0.4065 $\,\,$ \\
$\,\,$\color{red} 0.2761\color{black} $\,\,$ & $\,\,$0.5251$\,\,$ & $\,\,$2.4598$\,\,$ & $\,\,$ 1  $\,\,$ \\
\end{pmatrix},
\end{equation*}

\begin{equation*}
\mathbf{w}^{\prime} =
\begin{pmatrix}
0.524947\\
0.273250\\
0.058327\\
0.143475
\end{pmatrix} =
0.994679\cdot
\begin{pmatrix}
\color{gr} 0.527755\color{black} \\
0.274712\\
0.058639\\
0.144242
\end{pmatrix},
\end{equation*}
\begin{equation*}
\left[ \frac{{w}^{\prime}_i}{{w}^{\prime}_j} \right] =
\begin{pmatrix}
$\,\,$ 1 $\,\,$ & $\,\,$\color{gr} 1.9211\color{black} $\,\,$ & $\,\,$\color{gr} \color{blue} 9\color{black} $\,\,$ & $\,\,$\color{gr} 3.6588\color{black} $\,\,$ \\
$\,\,$\color{gr} 0.5205\color{black} $\,\,$ & $\,\,$ 1 $\,\,$ & $\,\,$4.6848$\,\,$ & $\,\,$1.9045  $\,\,$ \\
$\,\,$\color{gr} \color{blue}  1/9\color{black} $\,\,$ & $\,\,$0.2135$\,\,$ & $\,\,$ 1 $\,\,$ & $\,\,$0.4065 $\,\,$ \\
$\,\,$\color{gr} 0.2733\color{black} $\,\,$ & $\,\,$0.5251$\,\,$ & $\,\,$2.4598$\,\,$ & $\,\,$ 1  $\,\,$ \\
\end{pmatrix},
\end{equation*}
\end{example}
\newpage
\begin{example}
\begin{equation*}
\mathbf{A} =
\begin{pmatrix}
$\,\,$ 1 $\,\,$ & $\,\,$2$\,\,$ & $\,\,$9$\,\,$ & $\,\,$4 $\,\,$ \\
$\,\,$ 1/2$\,\,$ & $\,\,$ 1 $\,\,$ & $\,\,$3$\,\,$ & $\,\,$4 $\,\,$ \\
$\,\,$ 1/9$\,\,$ & $\,\,$ 1/3$\,\,$ & $\,\,$ 1 $\,\,$ & $\,\,$ 1/4 $\,\,$ \\
$\,\,$ 1/4$\,\,$ & $\,\,$ 1/4$\,\,$ & $\,\,$4$\,\,$ & $\,\,$ 1  $\,\,$ \\
\end{pmatrix},
\qquad
\lambda_{\max} =
4.2469,
\qquad
CR = 0.0931
\end{equation*}

\begin{equation*}
\mathbf{w}^{EM} =
\begin{pmatrix}
\color{red} 0.510079\color{black} \\
0.297561\\
0.058324\\
0.134036
\end{pmatrix}\end{equation*}
\begin{equation*}
\left[ \frac{{w}^{EM}_i}{{w}^{EM}_j} \right] =
\begin{pmatrix}
$\,\,$ 1 $\,\,$ & $\,\,$\color{red} 1.7142\color{black} $\,\,$ & $\,\,$\color{red} 8.7457\color{black} $\,\,$ & $\,\,$\color{red} 3.8055\color{black} $\,\,$ \\
$\,\,$\color{red} 0.5834\color{black} $\,\,$ & $\,\,$ 1 $\,\,$ & $\,\,$5.1019$\,\,$ & $\,\,$2.2200  $\,\,$ \\
$\,\,$\color{red} 0.1143\color{black} $\,\,$ & $\,\,$0.1960$\,\,$ & $\,\,$ 1 $\,\,$ & $\,\,$0.4351 $\,\,$ \\
$\,\,$\color{red} 0.2628\color{black} $\,\,$ & $\,\,$0.4504$\,\,$ & $\,\,$2.2982$\,\,$ & $\,\,$ 1  $\,\,$ \\
\end{pmatrix},
\end{equation*}

\begin{equation*}
\mathbf{w}^{\prime} =
\begin{pmatrix}
0.517240\\
0.293212\\
0.057471\\
0.132077
\end{pmatrix} =
0.985383\cdot
\begin{pmatrix}
\color{gr} 0.524912\color{black} \\
0.297561\\
0.058324\\
0.134036
\end{pmatrix},
\end{equation*}
\begin{equation*}
\left[ \frac{{w}^{\prime}_i}{{w}^{\prime}_j} \right] =
\begin{pmatrix}
$\,\,$ 1 $\,\,$ & $\,\,$\color{gr} 1.7640\color{black} $\,\,$ & $\,\,$\color{gr} \color{blue} 9\color{black} $\,\,$ & $\,\,$\color{gr} 3.9162\color{black} $\,\,$ \\
$\,\,$\color{gr} 0.5669\color{black} $\,\,$ & $\,\,$ 1 $\,\,$ & $\,\,$5.1019$\,\,$ & $\,\,$2.2200  $\,\,$ \\
$\,\,$\color{gr} \color{blue}  1/9\color{black} $\,\,$ & $\,\,$0.1960$\,\,$ & $\,\,$ 1 $\,\,$ & $\,\,$0.4351 $\,\,$ \\
$\,\,$\color{gr} 0.2554\color{black} $\,\,$ & $\,\,$0.4504$\,\,$ & $\,\,$2.2982$\,\,$ & $\,\,$ 1  $\,\,$ \\
\end{pmatrix},
\end{equation*}
\end{example}
\newpage
\begin{example}
\begin{equation*}
\mathbf{A} =
\begin{pmatrix}
$\,\,$ 1 $\,\,$ & $\,\,$2$\,\,$ & $\,\,$9$\,\,$ & $\,\,$5 $\,\,$ \\
$\,\,$ 1/2$\,\,$ & $\,\,$ 1 $\,\,$ & $\,\,$3$\,\,$ & $\,\,$4 $\,\,$ \\
$\,\,$ 1/9$\,\,$ & $\,\,$ 1/3$\,\,$ & $\,\,$ 1 $\,\,$ & $\,\,$ 1/3 $\,\,$ \\
$\,\,$ 1/5$\,\,$ & $\,\,$ 1/4$\,\,$ & $\,\,$3$\,\,$ & $\,\,$ 1  $\,\,$ \\
\end{pmatrix},
\qquad
\lambda_{\max} =
4.1655,
\qquad
CR = 0.0624
\end{equation*}

\begin{equation*}
\mathbf{w}^{EM} =
\begin{pmatrix}
\color{red} 0.536652\color{black} \\
0.287561\\
0.061185\\
0.114601
\end{pmatrix}\end{equation*}
\begin{equation*}
\left[ \frac{{w}^{EM}_i}{{w}^{EM}_j} \right] =
\begin{pmatrix}
$\,\,$ 1 $\,\,$ & $\,\,$\color{red} 1.8662\color{black} $\,\,$ & $\,\,$\color{red} 8.7710\color{black} $\,\,$ & $\,\,$\color{red} 4.6828\color{black} $\,\,$ \\
$\,\,$\color{red} 0.5358\color{black} $\,\,$ & $\,\,$ 1 $\,\,$ & $\,\,$4.6999$\,\,$ & $\,\,$2.5092  $\,\,$ \\
$\,\,$\color{red} 0.1140\color{black} $\,\,$ & $\,\,$0.2128$\,\,$ & $\,\,$ 1 $\,\,$ & $\,\,$0.5339 $\,\,$ \\
$\,\,$\color{red} 0.2135\color{black} $\,\,$ & $\,\,$0.3985$\,\,$ & $\,\,$1.8730$\,\,$ & $\,\,$ 1  $\,\,$ \\
\end{pmatrix},
\end{equation*}

\begin{equation*}
\mathbf{w}^{\prime} =
\begin{pmatrix}
0.543054\\
0.283588\\
0.060339\\
0.113018
\end{pmatrix} =
0.986184\cdot
\begin{pmatrix}
\color{gr} 0.550662\color{black} \\
0.287561\\
0.061185\\
0.114601
\end{pmatrix},
\end{equation*}
\begin{equation*}
\left[ \frac{{w}^{\prime}_i}{{w}^{\prime}_j} \right] =
\begin{pmatrix}
$\,\,$ 1 $\,\,$ & $\,\,$\color{gr} 1.9149\color{black} $\,\,$ & $\,\,$\color{gr} \color{blue} 9\color{black} $\,\,$ & $\,\,$\color{gr} 4.8050\color{black} $\,\,$ \\
$\,\,$\color{gr} 0.5222\color{black} $\,\,$ & $\,\,$ 1 $\,\,$ & $\,\,$4.6999$\,\,$ & $\,\,$2.5092  $\,\,$ \\
$\,\,$\color{gr} \color{blue}  1/9\color{black} $\,\,$ & $\,\,$0.2128$\,\,$ & $\,\,$ 1 $\,\,$ & $\,\,$0.5339 $\,\,$ \\
$\,\,$\color{gr} 0.2081\color{black} $\,\,$ & $\,\,$0.3985$\,\,$ & $\,\,$1.8730$\,\,$ & $\,\,$ 1  $\,\,$ \\
\end{pmatrix},
\end{equation*}
\end{example}
\newpage
\begin{example}
\begin{equation*}
\mathbf{A} =
\begin{pmatrix}
$\,\,$ 1 $\,\,$ & $\,\,$2$\,\,$ & $\,\,$9$\,\,$ & $\,\,$5 $\,\,$ \\
$\,\,$ 1/2$\,\,$ & $\,\,$ 1 $\,\,$ & $\,\,$3$\,\,$ & $\,\,$5 $\,\,$ \\
$\,\,$ 1/9$\,\,$ & $\,\,$ 1/3$\,\,$ & $\,\,$ 1 $\,\,$ & $\,\,$ 1/3 $\,\,$ \\
$\,\,$ 1/5$\,\,$ & $\,\,$ 1/5$\,\,$ & $\,\,$3$\,\,$ & $\,\,$ 1  $\,\,$ \\
\end{pmatrix},
\qquad
\lambda_{\max} =
4.2277,
\qquad
CR = 0.0859
\end{equation*}

\begin{equation*}
\mathbf{w}^{EM} =
\begin{pmatrix}
\color{red} 0.525945\color{black} \\
0.305270\\
0.060784\\
0.108001
\end{pmatrix}\end{equation*}
\begin{equation*}
\left[ \frac{{w}^{EM}_i}{{w}^{EM}_j} \right] =
\begin{pmatrix}
$\,\,$ 1 $\,\,$ & $\,\,$\color{red} 1.7229\color{black} $\,\,$ & $\,\,$\color{red} 8.6526\color{black} $\,\,$ & $\,\,$\color{red} 4.8698\color{black} $\,\,$ \\
$\,\,$\color{red} 0.5804\color{black} $\,\,$ & $\,\,$ 1 $\,\,$ & $\,\,$5.0222$\,\,$ & $\,\,$2.8266  $\,\,$ \\
$\,\,$\color{red} 0.1156\color{black} $\,\,$ & $\,\,$0.1991$\,\,$ & $\,\,$ 1 $\,\,$ & $\,\,$0.5628 $\,\,$ \\
$\,\,$\color{red} 0.2053\color{black} $\,\,$ & $\,\,$0.3538$\,\,$ & $\,\,$1.7768$\,\,$ & $\,\,$ 1  $\,\,$ \\
\end{pmatrix},
\end{equation*}

\begin{equation*}
\mathbf{w}^{\prime} =
\begin{pmatrix}
0.532517\\
0.301038\\
0.059942\\
0.106503
\end{pmatrix} =
0.986136\cdot
\begin{pmatrix}
\color{gr} 0.540003\color{black} \\
0.305270\\
0.060784\\
0.108001
\end{pmatrix},
\end{equation*}
\begin{equation*}
\left[ \frac{{w}^{\prime}_i}{{w}^{\prime}_j} \right] =
\begin{pmatrix}
$\,\,$ 1 $\,\,$ & $\,\,$\color{gr} 1.7689\color{black} $\,\,$ & $\,\,$\color{gr} 8.8839\color{black} $\,\,$ & $\,\,$\color{gr} \color{blue} 5\color{black} $\,\,$ \\
$\,\,$\color{gr} 0.5653\color{black} $\,\,$ & $\,\,$ 1 $\,\,$ & $\,\,$5.0222$\,\,$ & $\,\,$2.8266  $\,\,$ \\
$\,\,$\color{gr} 0.1126\color{black} $\,\,$ & $\,\,$0.1991$\,\,$ & $\,\,$ 1 $\,\,$ & $\,\,$0.5628 $\,\,$ \\
$\,\,$\color{gr} \color{blue}  1/5\color{black} $\,\,$ & $\,\,$0.3538$\,\,$ & $\,\,$1.7768$\,\,$ & $\,\,$ 1  $\,\,$ \\
\end{pmatrix},
\end{equation*}
\end{example}
\newpage
\begin{example}
\begin{equation*}
\mathbf{A} =
\begin{pmatrix}
$\,\,$ 1 $\,\,$ & $\,\,$2$\,\,$ & $\,\,$9$\,\,$ & $\,\,$5 $\,\,$ \\
$\,\,$ 1/2$\,\,$ & $\,\,$ 1 $\,\,$ & $\,\,$6$\,\,$ & $\,\,$2 $\,\,$ \\
$\,\,$ 1/9$\,\,$ & $\,\,$ 1/6$\,\,$ & $\,\,$ 1 $\,\,$ & $\,\,$ 1/4 $\,\,$ \\
$\,\,$ 1/5$\,\,$ & $\,\,$ 1/2$\,\,$ & $\,\,$4$\,\,$ & $\,\,$ 1  $\,\,$ \\
\end{pmatrix},
\qquad
\lambda_{\max} =
4.0539,
\qquad
CR = 0.0203
\end{equation*}

\begin{equation*}
\mathbf{w}^{EM} =
\begin{pmatrix}
0.542698\\
\color{red} 0.271084\color{black} \\
0.046017\\
0.140200
\end{pmatrix}\end{equation*}
\begin{equation*}
\left[ \frac{{w}^{EM}_i}{{w}^{EM}_j} \right] =
\begin{pmatrix}
$\,\,$ 1 $\,\,$ & $\,\,$\color{red} 2.0020\color{black} $\,\,$ & $\,\,$11.7933$\,\,$ & $\,\,$3.8709$\,\,$ \\
$\,\,$\color{red} 0.4995\color{black} $\,\,$ & $\,\,$ 1 $\,\,$ & $\,\,$\color{red} 5.8909\color{black} $\,\,$ & $\,\,$\color{red} 1.9336\color{black}   $\,\,$ \\
$\,\,$0.0848$\,\,$ & $\,\,$\color{red} 0.1698\color{black} $\,\,$ & $\,\,$ 1 $\,\,$ & $\,\,$0.3282 $\,\,$ \\
$\,\,$0.2583$\,\,$ & $\,\,$\color{red} 0.5172\color{black} $\,\,$ & $\,\,$3.0467$\,\,$ & $\,\,$ 1  $\,\,$ \\
\end{pmatrix},
\end{equation*}

\begin{equation*}
\mathbf{w}^{\prime} =
\begin{pmatrix}
0.542555\\
0.271277\\
0.046005\\
0.140163
\end{pmatrix} =
0.999735\cdot
\begin{pmatrix}
0.542698\\
\color{gr} 0.271349\color{black} \\
0.046017\\
0.140200
\end{pmatrix},
\end{equation*}
\begin{equation*}
\left[ \frac{{w}^{\prime}_i}{{w}^{\prime}_j} \right] =
\begin{pmatrix}
$\,\,$ 1 $\,\,$ & $\,\,$\color{gr} \color{blue} 2\color{black} $\,\,$ & $\,\,$11.7933$\,\,$ & $\,\,$3.8709$\,\,$ \\
$\,\,$\color{gr} \color{blue}  1/2\color{black} $\,\,$ & $\,\,$ 1 $\,\,$ & $\,\,$\color{gr} 5.8967\color{black} $\,\,$ & $\,\,$\color{gr} 1.9354\color{black}   $\,\,$ \\
$\,\,$0.0848$\,\,$ & $\,\,$\color{gr} 0.1696\color{black} $\,\,$ & $\,\,$ 1 $\,\,$ & $\,\,$0.3282 $\,\,$ \\
$\,\,$0.2583$\,\,$ & $\,\,$\color{gr} 0.5167\color{black} $\,\,$ & $\,\,$3.0467$\,\,$ & $\,\,$ 1  $\,\,$ \\
\end{pmatrix},
\end{equation*}
\end{example}
\newpage
\begin{example}
\begin{equation*}
\mathbf{A} =
\begin{pmatrix}
$\,\,$ 1 $\,\,$ & $\,\,$2$\,\,$ & $\,\,$9$\,\,$ & $\,\,$6 $\,\,$ \\
$\,\,$ 1/2$\,\,$ & $\,\,$ 1 $\,\,$ & $\,\,$7$\,\,$ & $\,\,$2 $\,\,$ \\
$\,\,$ 1/9$\,\,$ & $\,\,$ 1/7$\,\,$ & $\,\,$ 1 $\,\,$ & $\,\,$ 1/5 $\,\,$ \\
$\,\,$ 1/6$\,\,$ & $\,\,$ 1/2$\,\,$ & $\,\,$5$\,\,$ & $\,\,$ 1  $\,\,$ \\
\end{pmatrix},
\qquad
\lambda_{\max} =
4.1239,
\qquad
CR = 0.0467
\end{equation*}

\begin{equation*}
\mathbf{w}^{EM} =
\begin{pmatrix}
0.553492\\
\color{red} 0.268071\color{black} \\
0.040760\\
0.137677
\end{pmatrix}\end{equation*}
\begin{equation*}
\left[ \frac{{w}^{EM}_i}{{w}^{EM}_j} \right] =
\begin{pmatrix}
$\,\,$ 1 $\,\,$ & $\,\,$\color{red} 2.0647\color{black} $\,\,$ & $\,\,$13.5792$\,\,$ & $\,\,$4.0202$\,\,$ \\
$\,\,$\color{red} 0.4843\color{black} $\,\,$ & $\,\,$ 1 $\,\,$ & $\,\,$\color{red} 6.5768\color{black} $\,\,$ & $\,\,$\color{red} 1.9471\color{black}   $\,\,$ \\
$\,\,$0.0736$\,\,$ & $\,\,$\color{red} 0.1521\color{black} $\,\,$ & $\,\,$ 1 $\,\,$ & $\,\,$0.2961 $\,\,$ \\
$\,\,$0.2487$\,\,$ & $\,\,$\color{red} 0.5136\color{black} $\,\,$ & $\,\,$3.3777$\,\,$ & $\,\,$ 1  $\,\,$ \\
\end{pmatrix},
\end{equation*}

\begin{equation*}
\mathbf{w}^{\prime} =
\begin{pmatrix}
0.549491\\
0.273363\\
0.040466\\
0.136681
\end{pmatrix} =
0.992770\cdot
\begin{pmatrix}
0.553492\\
\color{gr} 0.275353\color{black} \\
0.040760\\
0.137677
\end{pmatrix},
\end{equation*}
\begin{equation*}
\left[ \frac{{w}^{\prime}_i}{{w}^{\prime}_j} \right] =
\begin{pmatrix}
$\,\,$ 1 $\,\,$ & $\,\,$\color{gr} 2.0101\color{black} $\,\,$ & $\,\,$13.5792$\,\,$ & $\,\,$4.0202$\,\,$ \\
$\,\,$\color{gr} 0.4975\color{black} $\,\,$ & $\,\,$ 1 $\,\,$ & $\,\,$\color{gr} 6.7554\color{black} $\,\,$ & $\,\,$\color{gr} \color{blue} 2\color{black}   $\,\,$ \\
$\,\,$0.0736$\,\,$ & $\,\,$\color{gr} 0.1480\color{black} $\,\,$ & $\,\,$ 1 $\,\,$ & $\,\,$0.2961 $\,\,$ \\
$\,\,$0.2487$\,\,$ & $\,\,$\color{gr} \color{blue}  1/2\color{black} $\,\,$ & $\,\,$3.3777$\,\,$ & $\,\,$ 1  $\,\,$ \\
\end{pmatrix},
\end{equation*}
\end{example}
\newpage
\begin{example}
\begin{equation*}
\mathbf{A} =
\begin{pmatrix}
$\,\,$ 1 $\,\,$ & $\,\,$2$\,\,$ & $\,\,$9$\,\,$ & $\,\,$6 $\,\,$ \\
$\,\,$ 1/2$\,\,$ & $\,\,$ 1 $\,\,$ & $\,\,$7$\,\,$ & $\,\,$2 $\,\,$ \\
$\,\,$ 1/9$\,\,$ & $\,\,$ 1/7$\,\,$ & $\,\,$ 1 $\,\,$ & $\,\,$ 1/6 $\,\,$ \\
$\,\,$ 1/6$\,\,$ & $\,\,$ 1/2$\,\,$ & $\,\,$6$\,\,$ & $\,\,$ 1  $\,\,$ \\
\end{pmatrix},
\qquad
\lambda_{\max} =
4.1658,
\qquad
CR = 0.0625
\end{equation*}

\begin{equation*}
\mathbf{w}^{EM} =
\begin{pmatrix}
0.551881\\
\color{red} 0.264585\color{black} \\
0.038922\\
0.144611
\end{pmatrix}\end{equation*}
\begin{equation*}
\left[ \frac{{w}^{EM}_i}{{w}^{EM}_j} \right] =
\begin{pmatrix}
$\,\,$ 1 $\,\,$ & $\,\,$\color{red} 2.0858\color{black} $\,\,$ & $\,\,$14.1790$\,\,$ & $\,\,$3.8163$\,\,$ \\
$\,\,$\color{red} 0.4794\color{black} $\,\,$ & $\,\,$ 1 $\,\,$ & $\,\,$\color{red} 6.7978\color{black} $\,\,$ & $\,\,$\color{red} 1.8296\color{black}   $\,\,$ \\
$\,\,$0.0705$\,\,$ & $\,\,$\color{red} 0.1471\color{black} $\,\,$ & $\,\,$ 1 $\,\,$ & $\,\,$0.2692 $\,\,$ \\
$\,\,$0.2620$\,\,$ & $\,\,$\color{red} 0.5466\color{black} $\,\,$ & $\,\,$3.7154$\,\,$ & $\,\,$ 1  $\,\,$ \\
\end{pmatrix},
\end{equation*}

\begin{equation*}
\mathbf{w}^{\prime} =
\begin{pmatrix}
0.547571\\
0.270329\\
0.038618\\
0.143482
\end{pmatrix} =
0.992190\cdot
\begin{pmatrix}
0.551881\\
\color{gr} 0.272457\color{black} \\
0.038922\\
0.144611
\end{pmatrix},
\end{equation*}
\begin{equation*}
\left[ \frac{{w}^{\prime}_i}{{w}^{\prime}_j} \right] =
\begin{pmatrix}
$\,\,$ 1 $\,\,$ & $\,\,$\color{gr} 2.0256\color{black} $\,\,$ & $\,\,$14.1790$\,\,$ & $\,\,$3.8163$\,\,$ \\
$\,\,$\color{gr} 0.4937\color{black} $\,\,$ & $\,\,$ 1 $\,\,$ & $\,\,$\color{gr} \color{blue} 7\color{black} $\,\,$ & $\,\,$\color{gr} 1.8841\color{black}   $\,\,$ \\
$\,\,$0.0705$\,\,$ & $\,\,$\color{gr} \color{blue}  1/7\color{black} $\,\,$ & $\,\,$ 1 $\,\,$ & $\,\,$0.2692 $\,\,$ \\
$\,\,$0.2620$\,\,$ & $\,\,$\color{gr} 0.5308\color{black} $\,\,$ & $\,\,$3.7154$\,\,$ & $\,\,$ 1  $\,\,$ \\
\end{pmatrix},
\end{equation*}
\end{example}
\newpage
\begin{example}
\begin{equation*}
\mathbf{A} =
\begin{pmatrix}
$\,\,$ 1 $\,\,$ & $\,\,$2$\,\,$ & $\,\,$9$\,\,$ & $\,\,$6 $\,\,$ \\
$\,\,$ 1/2$\,\,$ & $\,\,$ 1 $\,\,$ & $\,\,$7$\,\,$ & $\,\,$2 $\,\,$ \\
$\,\,$ 1/9$\,\,$ & $\,\,$ 1/7$\,\,$ & $\,\,$ 1 $\,\,$ & $\,\,$ 1/7 $\,\,$ \\
$\,\,$ 1/6$\,\,$ & $\,\,$ 1/2$\,\,$ & $\,\,$7$\,\,$ & $\,\,$ 1  $\,\,$ \\
\end{pmatrix},
\qquad
\lambda_{\max} =
4.2086,
\qquad
CR = 0.0786
\end{equation*}

\begin{equation*}
\mathbf{w}^{EM} =
\begin{pmatrix}
0.550187\\
\color{red} 0.261452\color{black} \\
0.037414\\
0.150947
\end{pmatrix}\end{equation*}
\begin{equation*}
\left[ \frac{{w}^{EM}_i}{{w}^{EM}_j} \right] =
\begin{pmatrix}
$\,\,$ 1 $\,\,$ & $\,\,$\color{red} 2.1044\color{black} $\,\,$ & $\,\,$14.7053$\,\,$ & $\,\,$3.6449$\,\,$ \\
$\,\,$\color{red} 0.4752\color{black} $\,\,$ & $\,\,$ 1 $\,\,$ & $\,\,$\color{red} 6.9880\color{black} $\,\,$ & $\,\,$\color{red} 1.7321\color{black}   $\,\,$ \\
$\,\,$0.0680$\,\,$ & $\,\,$\color{red} 0.1431\color{black} $\,\,$ & $\,\,$ 1 $\,\,$ & $\,\,$0.2479 $\,\,$ \\
$\,\,$0.2744$\,\,$ & $\,\,$\color{red} 0.5773\color{black} $\,\,$ & $\,\,$4.0345$\,\,$ & $\,\,$ 1  $\,\,$ \\
\end{pmatrix},
\end{equation*}

\begin{equation*}
\mathbf{w}^{\prime} =
\begin{pmatrix}
0.549941\\
0.261782\\
0.037397\\
0.150879
\end{pmatrix} =
0.999553\cdot
\begin{pmatrix}
0.550187\\
\color{gr} 0.261899\color{black} \\
0.037414\\
0.150947
\end{pmatrix},
\end{equation*}
\begin{equation*}
\left[ \frac{{w}^{\prime}_i}{{w}^{\prime}_j} \right] =
\begin{pmatrix}
$\,\,$ 1 $\,\,$ & $\,\,$\color{gr} 2.1008\color{black} $\,\,$ & $\,\,$14.7053$\,\,$ & $\,\,$3.6449$\,\,$ \\
$\,\,$\color{gr} 0.4760\color{black} $\,\,$ & $\,\,$ 1 $\,\,$ & $\,\,$\color{gr} \color{blue} 7\color{black} $\,\,$ & $\,\,$\color{gr} 1.7350\color{black}   $\,\,$ \\
$\,\,$0.0680$\,\,$ & $\,\,$\color{gr} \color{blue}  1/7\color{black} $\,\,$ & $\,\,$ 1 $\,\,$ & $\,\,$0.2479 $\,\,$ \\
$\,\,$0.2744$\,\,$ & $\,\,$\color{gr} 0.5764\color{black} $\,\,$ & $\,\,$4.0345$\,\,$ & $\,\,$ 1  $\,\,$ \\
\end{pmatrix},
\end{equation*}
\end{example}
\newpage
\begin{example}
\begin{equation*}
\mathbf{A} =
\begin{pmatrix}
$\,\,$ 1 $\,\,$ & $\,\,$2$\,\,$ & $\,\,$9$\,\,$ & $\,\,$6 $\,\,$ \\
$\,\,$ 1/2$\,\,$ & $\,\,$ 1 $\,\,$ & $\,\,$8$\,\,$ & $\,\,$2 $\,\,$ \\
$\,\,$ 1/9$\,\,$ & $\,\,$ 1/8$\,\,$ & $\,\,$ 1 $\,\,$ & $\,\,$ 1/6 $\,\,$ \\
$\,\,$ 1/6$\,\,$ & $\,\,$ 1/2$\,\,$ & $\,\,$6$\,\,$ & $\,\,$ 1  $\,\,$ \\
\end{pmatrix},
\qquad
\lambda_{\max} =
4.1664,
\qquad
CR = 0.0627
\end{equation*}

\begin{equation*}
\mathbf{w}^{EM} =
\begin{pmatrix}
0.548389\\
\color{red} 0.271415\color{black} \\
0.037470\\
0.142726
\end{pmatrix}\end{equation*}
\begin{equation*}
\left[ \frac{{w}^{EM}_i}{{w}^{EM}_j} \right] =
\begin{pmatrix}
$\,\,$ 1 $\,\,$ & $\,\,$\color{red} 2.0205\color{black} $\,\,$ & $\,\,$14.6352$\,\,$ & $\,\,$3.8422$\,\,$ \\
$\,\,$\color{red} 0.4949\color{black} $\,\,$ & $\,\,$ 1 $\,\,$ & $\,\,$\color{red} 7.2434\color{black} $\,\,$ & $\,\,$\color{red} 1.9017\color{black}   $\,\,$ \\
$\,\,$0.0683$\,\,$ & $\,\,$\color{red} 0.1381\color{black} $\,\,$ & $\,\,$ 1 $\,\,$ & $\,\,$0.2625 $\,\,$ \\
$\,\,$0.2603$\,\,$ & $\,\,$\color{red} 0.5259\color{black} $\,\,$ & $\,\,$3.8090$\,\,$ & $\,\,$ 1  $\,\,$ \\
\end{pmatrix},
\end{equation*}

\begin{equation*}
\mathbf{w}^{\prime} =
\begin{pmatrix}
0.546869\\
0.273434\\
0.037367\\
0.142330
\end{pmatrix} =
0.997228\cdot
\begin{pmatrix}
0.548389\\
\color{gr} 0.274194\color{black} \\
0.037470\\
0.142726
\end{pmatrix},
\end{equation*}
\begin{equation*}
\left[ \frac{{w}^{\prime}_i}{{w}^{\prime}_j} \right] =
\begin{pmatrix}
$\,\,$ 1 $\,\,$ & $\,\,$\color{gr} \color{blue} 2\color{black} $\,\,$ & $\,\,$14.6352$\,\,$ & $\,\,$3.8422$\,\,$ \\
$\,\,$\color{gr} \color{blue}  1/2\color{black} $\,\,$ & $\,\,$ 1 $\,\,$ & $\,\,$\color{gr} 7.3176\color{black} $\,\,$ & $\,\,$\color{gr} 1.9211\color{black}   $\,\,$ \\
$\,\,$0.0683$\,\,$ & $\,\,$\color{gr} 0.1367\color{black} $\,\,$ & $\,\,$ 1 $\,\,$ & $\,\,$0.2625 $\,\,$ \\
$\,\,$0.2603$\,\,$ & $\,\,$\color{gr} 0.5205\color{black} $\,\,$ & $\,\,$3.8090$\,\,$ & $\,\,$ 1  $\,\,$ \\
\end{pmatrix},
\end{equation*}
\end{example}
\newpage
\begin{example}
\begin{equation*}
\mathbf{A} =
\begin{pmatrix}
$\,\,$ 1 $\,\,$ & $\,\,$2$\,\,$ & $\,\,$9$\,\,$ & $\,\,$6 $\,\,$ \\
$\,\,$ 1/2$\,\,$ & $\,\,$ 1 $\,\,$ & $\,\,$8$\,\,$ & $\,\,$2 $\,\,$ \\
$\,\,$ 1/9$\,\,$ & $\,\,$ 1/8$\,\,$ & $\,\,$ 1 $\,\,$ & $\,\,$ 1/7 $\,\,$ \\
$\,\,$ 1/6$\,\,$ & $\,\,$ 1/2$\,\,$ & $\,\,$7$\,\,$ & $\,\,$ 1  $\,\,$ \\
\end{pmatrix},
\qquad
\lambda_{\max} =
4.2065,
\qquad
CR = 0.0779
\end{equation*}

\begin{equation*}
\mathbf{w}^{EM} =
\begin{pmatrix}
0.546981\\
\color{red} 0.268077\color{black} \\
0.036038\\
0.148904
\end{pmatrix}\end{equation*}
\begin{equation*}
\left[ \frac{{w}^{EM}_i}{{w}^{EM}_j} \right] =
\begin{pmatrix}
$\,\,$ 1 $\,\,$ & $\,\,$\color{red} 2.0404\color{black} $\,\,$ & $\,\,$15.1779$\,\,$ & $\,\,$3.6734$\,\,$ \\
$\,\,$\color{red} 0.4901\color{black} $\,\,$ & $\,\,$ 1 $\,\,$ & $\,\,$\color{red} 7.4388\color{black} $\,\,$ & $\,\,$\color{red} 1.8003\color{black}   $\,\,$ \\
$\,\,$0.0659$\,\,$ & $\,\,$\color{red} 0.1344\color{black} $\,\,$ & $\,\,$ 1 $\,\,$ & $\,\,$0.2420 $\,\,$ \\
$\,\,$0.2722$\,\,$ & $\,\,$\color{red} 0.5555\color{black} $\,\,$ & $\,\,$4.1319$\,\,$ & $\,\,$ 1  $\,\,$ \\
\end{pmatrix},
\end{equation*}

\begin{equation*}
\mathbf{w}^{\prime} =
\begin{pmatrix}
0.544036\\
0.272018\\
0.035844\\
0.148102
\end{pmatrix} =
0.994616\cdot
\begin{pmatrix}
0.546981\\
\color{gr} 0.273490\color{black} \\
0.036038\\
0.148904
\end{pmatrix},
\end{equation*}
\begin{equation*}
\left[ \frac{{w}^{\prime}_i}{{w}^{\prime}_j} \right] =
\begin{pmatrix}
$\,\,$ 1 $\,\,$ & $\,\,$\color{gr} \color{blue} 2\color{black} $\,\,$ & $\,\,$15.1779$\,\,$ & $\,\,$3.6734$\,\,$ \\
$\,\,$\color{gr} \color{blue}  1/2\color{black} $\,\,$ & $\,\,$ 1 $\,\,$ & $\,\,$\color{gr} 7.5890\color{black} $\,\,$ & $\,\,$\color{gr} 1.8367\color{black}   $\,\,$ \\
$\,\,$0.0659$\,\,$ & $\,\,$\color{gr} 0.1318\color{black} $\,\,$ & $\,\,$ 1 $\,\,$ & $\,\,$0.2420 $\,\,$ \\
$\,\,$0.2722$\,\,$ & $\,\,$\color{gr} 0.5445\color{black} $\,\,$ & $\,\,$4.1319$\,\,$ & $\,\,$ 1  $\,\,$ \\
\end{pmatrix},
\end{equation*}
\end{example}
\newpage
\begin{example}
\begin{equation*}
\mathbf{A} =
\begin{pmatrix}
$\,\,$ 1 $\,\,$ & $\,\,$2$\,\,$ & $\,\,$9$\,\,$ & $\,\,$6 $\,\,$ \\
$\,\,$ 1/2$\,\,$ & $\,\,$ 1 $\,\,$ & $\,\,$8$\,\,$ & $\,\,$2 $\,\,$ \\
$\,\,$ 1/9$\,\,$ & $\,\,$ 1/8$\,\,$ & $\,\,$ 1 $\,\,$ & $\,\,$ 1/8 $\,\,$ \\
$\,\,$ 1/6$\,\,$ & $\,\,$ 1/2$\,\,$ & $\,\,$8$\,\,$ & $\,\,$ 1  $\,\,$ \\
\end{pmatrix},
\qquad
\lambda_{\max} =
4.2469,
\qquad
CR = 0.0931
\end{equation*}

\begin{equation*}
\mathbf{w}^{EM} =
\begin{pmatrix}
0.545510\\
\color{red} 0.265047\color{black} \\
0.034824\\
0.154619
\end{pmatrix}\end{equation*}
\begin{equation*}
\left[ \frac{{w}^{EM}_i}{{w}^{EM}_j} \right] =
\begin{pmatrix}
$\,\,$ 1 $\,\,$ & $\,\,$\color{red} 2.0582\color{black} $\,\,$ & $\,\,$15.6648$\,\,$ & $\,\,$3.5281$\,\,$ \\
$\,\,$\color{red} 0.4859\color{black} $\,\,$ & $\,\,$ 1 $\,\,$ & $\,\,$\color{red} 7.6110\color{black} $\,\,$ & $\,\,$\color{red} 1.7142\color{black}   $\,\,$ \\
$\,\,$0.0638$\,\,$ & $\,\,$\color{red} 0.1314\color{black} $\,\,$ & $\,\,$ 1 $\,\,$ & $\,\,$0.2252 $\,\,$ \\
$\,\,$0.2834$\,\,$ & $\,\,$\color{red} 0.5834\color{black} $\,\,$ & $\,\,$4.4400$\,\,$ & $\,\,$ 1  $\,\,$ \\
\end{pmatrix},
\end{equation*}

\begin{equation*}
\mathbf{w}^{\prime} =
\begin{pmatrix}
0.541337\\
0.270669\\
0.034558\\
0.153436
\end{pmatrix} =
0.992351\cdot
\begin{pmatrix}
0.545510\\
\color{gr} 0.272755\color{black} \\
0.034824\\
0.154619
\end{pmatrix},
\end{equation*}
\begin{equation*}
\left[ \frac{{w}^{\prime}_i}{{w}^{\prime}_j} \right] =
\begin{pmatrix}
$\,\,$ 1 $\,\,$ & $\,\,$\color{gr} \color{blue} 2\color{black} $\,\,$ & $\,\,$15.6648$\,\,$ & $\,\,$3.5281$\,\,$ \\
$\,\,$\color{gr} \color{blue}  1/2\color{black} $\,\,$ & $\,\,$ 1 $\,\,$ & $\,\,$\color{gr} 7.8324\color{black} $\,\,$ & $\,\,$\color{gr} 1.7640\color{black}   $\,\,$ \\
$\,\,$0.0638$\,\,$ & $\,\,$\color{gr} 0.1277\color{black} $\,\,$ & $\,\,$ 1 $\,\,$ & $\,\,$0.2252 $\,\,$ \\
$\,\,$0.2834$\,\,$ & $\,\,$\color{gr} 0.5669\color{black} $\,\,$ & $\,\,$4.4400$\,\,$ & $\,\,$ 1  $\,\,$ \\
\end{pmatrix},
\end{equation*}
\end{example}
\newpage
\begin{example}
\begin{equation*}
\mathbf{A} =
\begin{pmatrix}
$\,\,$ 1 $\,\,$ & $\,\,$2$\,\,$ & $\,\,$9$\,\,$ & $\,\,$6 $\,\,$ \\
$\,\,$ 1/2$\,\,$ & $\,\,$ 1 $\,\,$ & $\,\,$9$\,\,$ & $\,\,$2 $\,\,$ \\
$\,\,$ 1/9$\,\,$ & $\,\,$ 1/9$\,\,$ & $\,\,$ 1 $\,\,$ & $\,\,$ 1/8 $\,\,$ \\
$\,\,$ 1/6$\,\,$ & $\,\,$ 1/2$\,\,$ & $\,\,$8$\,\,$ & $\,\,$ 1  $\,\,$ \\
\end{pmatrix},
\qquad
\lambda_{\max} =
4.2469,
\qquad
CR = 0.0931
\end{equation*}

\begin{equation*}
\mathbf{w}^{EM} =
\begin{pmatrix}
0.542552\\
\color{red} 0.271056\color{black} \\
0.033720\\
0.152672
\end{pmatrix}\end{equation*}
\begin{equation*}
\left[ \frac{{w}^{EM}_i}{{w}^{EM}_j} \right] =
\begin{pmatrix}
$\,\,$ 1 $\,\,$ & $\,\,$\color{red} 2.0016\color{black} $\,\,$ & $\,\,$16.0901$\,\,$ & $\,\,$3.5537$\,\,$ \\
$\,\,$\color{red} 0.4996\color{black} $\,\,$ & $\,\,$ 1 $\,\,$ & $\,\,$\color{red} 8.0385\color{black} $\,\,$ & $\,\,$\color{red} 1.7754\color{black}   $\,\,$ \\
$\,\,$0.0622$\,\,$ & $\,\,$\color{red} 0.1244\color{black} $\,\,$ & $\,\,$ 1 $\,\,$ & $\,\,$0.2209 $\,\,$ \\
$\,\,$0.2814$\,\,$ & $\,\,$\color{red} 0.5632\color{black} $\,\,$ & $\,\,$4.5277$\,\,$ & $\,\,$ 1  $\,\,$ \\
\end{pmatrix},
\end{equation*}

\begin{equation*}
\mathbf{w}^{\prime} =
\begin{pmatrix}
0.542433\\
0.271217\\
0.033712\\
0.152638
\end{pmatrix} =
0.999780\cdot
\begin{pmatrix}
0.542552\\
\color{gr} 0.271276\color{black} \\
0.033720\\
0.152672
\end{pmatrix},
\end{equation*}
\begin{equation*}
\left[ \frac{{w}^{\prime}_i}{{w}^{\prime}_j} \right] =
\begin{pmatrix}
$\,\,$ 1 $\,\,$ & $\,\,$\color{gr} \color{blue} 2\color{black} $\,\,$ & $\,\,$16.0901$\,\,$ & $\,\,$3.5537$\,\,$ \\
$\,\,$\color{gr} \color{blue}  1/2\color{black} $\,\,$ & $\,\,$ 1 $\,\,$ & $\,\,$\color{gr} 8.0450\color{black} $\,\,$ & $\,\,$\color{gr} 1.7769\color{black}   $\,\,$ \\
$\,\,$0.0622$\,\,$ & $\,\,$\color{gr} 0.1243\color{black} $\,\,$ & $\,\,$ 1 $\,\,$ & $\,\,$0.2209 $\,\,$ \\
$\,\,$0.2814$\,\,$ & $\,\,$\color{gr} 0.5628\color{black} $\,\,$ & $\,\,$4.5277$\,\,$ & $\,\,$ 1  $\,\,$ \\
\end{pmatrix},
\end{equation*}
\end{example}
\newpage
\begin{example}
\begin{equation*}
\mathbf{A} =
\begin{pmatrix}
$\,\,$ 1 $\,\,$ & $\,\,$2$\,\,$ & $\,\,$9$\,\,$ & $\,\,$7 $\,\,$ \\
$\,\,$ 1/2$\,\,$ & $\,\,$ 1 $\,\,$ & $\,\,$3$\,\,$ & $\,\,$5 $\,\,$ \\
$\,\,$ 1/9$\,\,$ & $\,\,$ 1/3$\,\,$ & $\,\,$ 1 $\,\,$ & $\,\,$ 1/2 $\,\,$ \\
$\,\,$ 1/7$\,\,$ & $\,\,$ 1/5$\,\,$ & $\,\,$2$\,\,$ & $\,\,$ 1  $\,\,$ \\
\end{pmatrix},
\qquad
\lambda_{\max} =
4.1239,
\qquad
CR = 0.0467
\end{equation*}

\begin{equation*}
\mathbf{w}^{EM} =
\begin{pmatrix}
\color{red} 0.561690\color{black} \\
0.288475\\
0.064430\\
0.085405
\end{pmatrix}\end{equation*}
\begin{equation*}
\left[ \frac{{w}^{EM}_i}{{w}^{EM}_j} \right] =
\begin{pmatrix}
$\,\,$ 1 $\,\,$ & $\,\,$\color{red} 1.9471\color{black} $\,\,$ & $\,\,$\color{red} 8.7179\color{black} $\,\,$ & $\,\,$\color{red} 6.5768\color{black} $\,\,$ \\
$\,\,$\color{red} 0.5136\color{black} $\,\,$ & $\,\,$ 1 $\,\,$ & $\,\,$4.4774$\,\,$ & $\,\,$3.3777  $\,\,$ \\
$\,\,$\color{red} 0.1147\color{black} $\,\,$ & $\,\,$0.2233$\,\,$ & $\,\,$ 1 $\,\,$ & $\,\,$0.7544 $\,\,$ \\
$\,\,$\color{red} 0.1521\color{black} $\,\,$ & $\,\,$0.2961$\,\,$ & $\,\,$1.3256$\,\,$ & $\,\,$ 1  $\,\,$ \\
\end{pmatrix},
\end{equation*}

\begin{equation*}
\mathbf{w}^{\prime} =
\begin{pmatrix}
0.568278\\
0.284139\\
0.063461\\
0.084122
\end{pmatrix} =
0.984970\cdot
\begin{pmatrix}
\color{gr} 0.576949\color{black} \\
0.288475\\
0.064430\\
0.085405
\end{pmatrix},
\end{equation*}
\begin{equation*}
\left[ \frac{{w}^{\prime}_i}{{w}^{\prime}_j} \right] =
\begin{pmatrix}
$\,\,$ 1 $\,\,$ & $\,\,$\color{gr} \color{blue} 2\color{black} $\,\,$ & $\,\,$\color{gr} 8.9547\color{black} $\,\,$ & $\,\,$\color{gr} 6.7554\color{black} $\,\,$ \\
$\,\,$\color{gr} \color{blue}  1/2\color{black} $\,\,$ & $\,\,$ 1 $\,\,$ & $\,\,$4.4774$\,\,$ & $\,\,$3.3777  $\,\,$ \\
$\,\,$\color{gr} 0.1117\color{black} $\,\,$ & $\,\,$0.2233$\,\,$ & $\,\,$ 1 $\,\,$ & $\,\,$0.7544 $\,\,$ \\
$\,\,$\color{gr} 0.1480\color{black} $\,\,$ & $\,\,$0.2961$\,\,$ & $\,\,$1.3256$\,\,$ & $\,\,$ 1  $\,\,$ \\
\end{pmatrix},
\end{equation*}
\end{example}
\newpage
\begin{example}
\begin{equation*}
\mathbf{A} =
\begin{pmatrix}
$\,\,$ 1 $\,\,$ & $\,\,$2$\,\,$ & $\,\,$9$\,\,$ & $\,\,$7 $\,\,$ \\
$\,\,$ 1/2$\,\,$ & $\,\,$ 1 $\,\,$ & $\,\,$3$\,\,$ & $\,\,$6 $\,\,$ \\
$\,\,$ 1/9$\,\,$ & $\,\,$ 1/3$\,\,$ & $\,\,$ 1 $\,\,$ & $\,\,$ 1/2 $\,\,$ \\
$\,\,$ 1/7$\,\,$ & $\,\,$ 1/6$\,\,$ & $\,\,$2$\,\,$ & $\,\,$ 1  $\,\,$ \\
\end{pmatrix},
\qquad
\lambda_{\max} =
4.1658,
\qquad
CR = 0.0625
\end{equation*}

\begin{equation*}
\mathbf{w}^{EM} =
\begin{pmatrix}
\color{red} 0.552625\color{black} \\
0.302041\\
0.064038\\
0.081295
\end{pmatrix}\end{equation*}
\begin{equation*}
\left[ \frac{{w}^{EM}_i}{{w}^{EM}_j} \right] =
\begin{pmatrix}
$\,\,$ 1 $\,\,$ & $\,\,$\color{red} 1.8296\color{black} $\,\,$ & $\,\,$\color{red} 8.6296\color{black} $\,\,$ & $\,\,$\color{red} 6.7978\color{black} $\,\,$ \\
$\,\,$\color{red} 0.5466\color{black} $\,\,$ & $\,\,$ 1 $\,\,$ & $\,\,$4.7166$\,\,$ & $\,\,$3.7154  $\,\,$ \\
$\,\,$\color{red} 0.1159\color{black} $\,\,$ & $\,\,$0.2120$\,\,$ & $\,\,$ 1 $\,\,$ & $\,\,$0.7877 $\,\,$ \\
$\,\,$\color{red} 0.1471\color{black} $\,\,$ & $\,\,$0.2692$\,\,$ & $\,\,$1.2695$\,\,$ & $\,\,$ 1  $\,\,$ \\
\end{pmatrix},
\end{equation*}

\begin{equation*}
\mathbf{w}^{\prime} =
\begin{pmatrix}
0.559862\\
0.297156\\
0.063002\\
0.079980
\end{pmatrix} =
0.983825\cdot
\begin{pmatrix}
\color{gr} 0.569066\color{black} \\
0.302041\\
0.064038\\
0.081295
\end{pmatrix},
\end{equation*}
\begin{equation*}
\left[ \frac{{w}^{\prime}_i}{{w}^{\prime}_j} \right] =
\begin{pmatrix}
$\,\,$ 1 $\,\,$ & $\,\,$\color{gr} 1.8841\color{black} $\,\,$ & $\,\,$\color{gr} 8.8864\color{black} $\,\,$ & $\,\,$\color{gr} \color{blue} 7\color{black} $\,\,$ \\
$\,\,$\color{gr} 0.5308\color{black} $\,\,$ & $\,\,$ 1 $\,\,$ & $\,\,$4.7166$\,\,$ & $\,\,$3.7154  $\,\,$ \\
$\,\,$\color{gr} 0.1125\color{black} $\,\,$ & $\,\,$0.2120$\,\,$ & $\,\,$ 1 $\,\,$ & $\,\,$0.7877 $\,\,$ \\
$\,\,$\color{gr} \color{blue}  1/7\color{black} $\,\,$ & $\,\,$0.2692$\,\,$ & $\,\,$1.2695$\,\,$ & $\,\,$ 1  $\,\,$ \\
\end{pmatrix},
\end{equation*}
\end{example}
\newpage
\begin{example}
\begin{equation*}
\mathbf{A} =
\begin{pmatrix}
$\,\,$ 1 $\,\,$ & $\,\,$2$\,\,$ & $\,\,$9$\,\,$ & $\,\,$7 $\,\,$ \\
$\,\,$ 1/2$\,\,$ & $\,\,$ 1 $\,\,$ & $\,\,$3$\,\,$ & $\,\,$7 $\,\,$ \\
$\,\,$ 1/9$\,\,$ & $\,\,$ 1/3$\,\,$ & $\,\,$ 1 $\,\,$ & $\,\,$ 1/2 $\,\,$ \\
$\,\,$ 1/7$\,\,$ & $\,\,$ 1/7$\,\,$ & $\,\,$2$\,\,$ & $\,\,$ 1  $\,\,$ \\
\end{pmatrix},
\qquad
\lambda_{\max} =
4.2086,
\qquad
CR = 0.0786
\end{equation*}

\begin{equation*}
\mathbf{w}^{EM} =
\begin{pmatrix}
\color{red} 0.544262\color{black} \\
0.314224\\
0.063629\\
0.077885
\end{pmatrix}\end{equation*}
\begin{equation*}
\left[ \frac{{w}^{EM}_i}{{w}^{EM}_j} \right] =
\begin{pmatrix}
$\,\,$ 1 $\,\,$ & $\,\,$\color{red} 1.7321\color{black} $\,\,$ & $\,\,$\color{red} 8.5537\color{black} $\,\,$ & $\,\,$\color{red} 6.9880\color{black} $\,\,$ \\
$\,\,$\color{red} 0.5773\color{black} $\,\,$ & $\,\,$ 1 $\,\,$ & $\,\,$4.9384$\,\,$ & $\,\,$4.0345  $\,\,$ \\
$\,\,$\color{red} 0.1169\color{black} $\,\,$ & $\,\,$0.2025$\,\,$ & $\,\,$ 1 $\,\,$ & $\,\,$0.8170 $\,\,$ \\
$\,\,$\color{red} 0.1431\color{black} $\,\,$ & $\,\,$0.2479$\,\,$ & $\,\,$1.2240$\,\,$ & $\,\,$ 1  $\,\,$ \\
\end{pmatrix},
\end{equation*}

\begin{equation*}
\mathbf{w}^{\prime} =
\begin{pmatrix}
0.544686\\
0.313932\\
0.063570\\
0.077812
\end{pmatrix} =
0.999070\cdot
\begin{pmatrix}
\color{gr} 0.545193\color{black} \\
0.314224\\
0.063629\\
0.077885
\end{pmatrix},
\end{equation*}
\begin{equation*}
\left[ \frac{{w}^{\prime}_i}{{w}^{\prime}_j} \right] =
\begin{pmatrix}
$\,\,$ 1 $\,\,$ & $\,\,$\color{gr} 1.7350\color{black} $\,\,$ & $\,\,$\color{gr} 8.5683\color{black} $\,\,$ & $\,\,$\color{gr} \color{blue} 7\color{black} $\,\,$ \\
$\,\,$\color{gr} 0.5764\color{black} $\,\,$ & $\,\,$ 1 $\,\,$ & $\,\,$4.9384$\,\,$ & $\,\,$4.0345  $\,\,$ \\
$\,\,$\color{gr} 0.1167\color{black} $\,\,$ & $\,\,$0.2025$\,\,$ & $\,\,$ 1 $\,\,$ & $\,\,$0.8170 $\,\,$ \\
$\,\,$\color{gr} \color{blue}  1/7\color{black} $\,\,$ & $\,\,$0.2479$\,\,$ & $\,\,$1.2240$\,\,$ & $\,\,$ 1  $\,\,$ \\
\end{pmatrix},
\end{equation*}
\end{example}
\newpage
\begin{example}
\begin{equation*}
\mathbf{A} =
\begin{pmatrix}
$\,\,$ 1 $\,\,$ & $\,\,$2$\,\,$ & $\,\,$9$\,\,$ & $\,\,$7 $\,\,$ \\
$\,\,$ 1/2$\,\,$ & $\,\,$ 1 $\,\,$ & $\,\,$7$\,\,$ & $\,\,$2 $\,\,$ \\
$\,\,$ 1/9$\,\,$ & $\,\,$ 1/7$\,\,$ & $\,\,$ 1 $\,\,$ & $\,\,$ 1/6 $\,\,$ \\
$\,\,$ 1/7$\,\,$ & $\,\,$ 1/2$\,\,$ & $\,\,$6$\,\,$ & $\,\,$ 1  $\,\,$ \\
\end{pmatrix},
\qquad
\lambda_{\max} =
4.2059,
\qquad
CR = 0.0776
\end{equation*}

\begin{equation*}
\mathbf{w}^{EM} =
\begin{pmatrix}
0.567229\\
\color{red} 0.257483\color{black} \\
0.038256\\
0.137031
\end{pmatrix}\end{equation*}
\begin{equation*}
\left[ \frac{{w}^{EM}_i}{{w}^{EM}_j} \right] =
\begin{pmatrix}
$\,\,$ 1 $\,\,$ & $\,\,$\color{red} 2.2030\color{black} $\,\,$ & $\,\,$14.8270$\,\,$ & $\,\,$4.1394$\,\,$ \\
$\,\,$\color{red} 0.4539\color{black} $\,\,$ & $\,\,$ 1 $\,\,$ & $\,\,$\color{red} 6.7305\color{black} $\,\,$ & $\,\,$\color{red} 1.8790\color{black}   $\,\,$ \\
$\,\,$0.0674$\,\,$ & $\,\,$\color{red} 0.1486\color{black} $\,\,$ & $\,\,$ 1 $\,\,$ & $\,\,$0.2792 $\,\,$ \\
$\,\,$0.2416$\,\,$ & $\,\,$\color{red} 0.5322\color{black} $\,\,$ & $\,\,$3.5819$\,\,$ & $\,\,$ 1  $\,\,$ \\
\end{pmatrix},
\end{equation*}

\begin{equation*}
\mathbf{w}^{\prime} =
\begin{pmatrix}
0.561439\\
0.265062\\
0.037866\\
0.135633
\end{pmatrix} =
0.989793\cdot
\begin{pmatrix}
0.567229\\
\color{gr} 0.267795\color{black} \\
0.038256\\
0.137031
\end{pmatrix},
\end{equation*}
\begin{equation*}
\left[ \frac{{w}^{\prime}_i}{{w}^{\prime}_j} \right] =
\begin{pmatrix}
$\,\,$ 1 $\,\,$ & $\,\,$\color{gr} 2.1181\color{black} $\,\,$ & $\,\,$14.8270$\,\,$ & $\,\,$4.1394$\,\,$ \\
$\,\,$\color{gr} 0.4721\color{black} $\,\,$ & $\,\,$ 1 $\,\,$ & $\,\,$\color{gr} \color{blue} 7\color{black} $\,\,$ & $\,\,$\color{gr} 1.9543\color{black}   $\,\,$ \\
$\,\,$0.0674$\,\,$ & $\,\,$\color{gr} \color{blue}  1/7\color{black} $\,\,$ & $\,\,$ 1 $\,\,$ & $\,\,$0.2792 $\,\,$ \\
$\,\,$0.2416$\,\,$ & $\,\,$\color{gr} 0.5117\color{black} $\,\,$ & $\,\,$3.5819$\,\,$ & $\,\,$ 1  $\,\,$ \\
\end{pmatrix},
\end{equation*}
\end{example}
\newpage
\begin{example}
\begin{equation*}
\mathbf{A} =
\begin{pmatrix}
$\,\,$ 1 $\,\,$ & $\,\,$2$\,\,$ & $\,\,$9$\,\,$ & $\,\,$7 $\,\,$ \\
$\,\,$ 1/2$\,\,$ & $\,\,$ 1 $\,\,$ & $\,\,$7$\,\,$ & $\,\,$2 $\,\,$ \\
$\,\,$ 1/9$\,\,$ & $\,\,$ 1/7$\,\,$ & $\,\,$ 1 $\,\,$ & $\,\,$ 1/7 $\,\,$ \\
$\,\,$ 1/7$\,\,$ & $\,\,$ 1/2$\,\,$ & $\,\,$7$\,\,$ & $\,\,$ 1  $\,\,$ \\
\end{pmatrix},
\qquad
\lambda_{\max} =
4.2526,
\qquad
CR = 0.0952
\end{equation*}

\begin{equation*}
\mathbf{w}^{EM} =
\begin{pmatrix}
0.565987\\
\color{red} 0.254146\color{black} \\
0.036781\\
0.143085
\end{pmatrix}\end{equation*}
\begin{equation*}
\left[ \frac{{w}^{EM}_i}{{w}^{EM}_j} \right] =
\begin{pmatrix}
$\,\,$ 1 $\,\,$ & $\,\,$\color{red} 2.2270\color{black} $\,\,$ & $\,\,$15.3879$\,\,$ & $\,\,$3.9556$\,\,$ \\
$\,\,$\color{red} 0.4490\color{black} $\,\,$ & $\,\,$ 1 $\,\,$ & $\,\,$\color{red} 6.9096\color{black} $\,\,$ & $\,\,$\color{red} 1.7762\color{black}   $\,\,$ \\
$\,\,$0.0650$\,\,$ & $\,\,$\color{red} 0.1447\color{black} $\,\,$ & $\,\,$ 1 $\,\,$ & $\,\,$0.2571 $\,\,$ \\
$\,\,$0.2528$\,\,$ & $\,\,$\color{red} 0.5630\color{black} $\,\,$ & $\,\,$3.8902$\,\,$ & $\,\,$ 1  $\,\,$ \\
\end{pmatrix},
\end{equation*}

\begin{equation*}
\mathbf{w}^{\prime} =
\begin{pmatrix}
0.564112\\
0.256617\\
0.036660\\
0.142611
\end{pmatrix} =
0.996688\cdot
\begin{pmatrix}
0.565987\\
\color{gr} 0.257470\color{black} \\
0.036781\\
0.143085
\end{pmatrix},
\end{equation*}
\begin{equation*}
\left[ \frac{{w}^{\prime}_i}{{w}^{\prime}_j} \right] =
\begin{pmatrix}
$\,\,$ 1 $\,\,$ & $\,\,$\color{gr} 2.1983\color{black} $\,\,$ & $\,\,$15.3879$\,\,$ & $\,\,$3.9556$\,\,$ \\
$\,\,$\color{gr} 0.4549\color{black} $\,\,$ & $\,\,$ 1 $\,\,$ & $\,\,$\color{gr} \color{blue} 7\color{black} $\,\,$ & $\,\,$\color{gr} 1.7994\color{black}   $\,\,$ \\
$\,\,$0.0650$\,\,$ & $\,\,$\color{gr} \color{blue}  1/7\color{black} $\,\,$ & $\,\,$ 1 $\,\,$ & $\,\,$0.2571 $\,\,$ \\
$\,\,$0.2528$\,\,$ & $\,\,$\color{gr} 0.5557\color{black} $\,\,$ & $\,\,$3.8902$\,\,$ & $\,\,$ 1  $\,\,$ \\
\end{pmatrix},
\end{equation*}
\end{example}
\newpage
\begin{example}
\begin{equation*}
\mathbf{A} =
\begin{pmatrix}
$\,\,$ 1 $\,\,$ & $\,\,$2$\,\,$ & $\,\,$9$\,\,$ & $\,\,$7 $\,\,$ \\
$\,\,$ 1/2$\,\,$ & $\,\,$ 1 $\,\,$ & $\,\,$8$\,\,$ & $\,\,$2 $\,\,$ \\
$\,\,$ 1/9$\,\,$ & $\,\,$ 1/8$\,\,$ & $\,\,$ 1 $\,\,$ & $\,\,$ 1/6 $\,\,$ \\
$\,\,$ 1/7$\,\,$ & $\,\,$ 1/2$\,\,$ & $\,\,$6$\,\,$ & $\,\,$ 1  $\,\,$ \\
\end{pmatrix},
\qquad
\lambda_{\max} =
4.2065,
\qquad
CR = 0.0779
\end{equation*}

\begin{equation*}
\mathbf{w}^{EM} =
\begin{pmatrix}
0.563613\\
\color{red} 0.264236\color{black} \\
0.036862\\
0.135288
\end{pmatrix}\end{equation*}
\begin{equation*}
\left[ \frac{{w}^{EM}_i}{{w}^{EM}_j} \right] =
\begin{pmatrix}
$\,\,$ 1 $\,\,$ & $\,\,$\color{red} 2.1330\color{black} $\,\,$ & $\,\,$15.2896$\,\,$ & $\,\,$4.1660$\,\,$ \\
$\,\,$\color{red} 0.4688\color{black} $\,\,$ & $\,\,$ 1 $\,\,$ & $\,\,$\color{red} 7.1681\color{black} $\,\,$ & $\,\,$\color{red} 1.9531\color{black}   $\,\,$ \\
$\,\,$0.0654$\,\,$ & $\,\,$\color{red} 0.1395\color{black} $\,\,$ & $\,\,$ 1 $\,\,$ & $\,\,$0.2725 $\,\,$ \\
$\,\,$0.2400$\,\,$ & $\,\,$\color{red} 0.5120\color{black} $\,\,$ & $\,\,$3.6701$\,\,$ & $\,\,$ 1  $\,\,$ \\
\end{pmatrix},
\end{equation*}

\begin{equation*}
\mathbf{w}^{\prime} =
\begin{pmatrix}
0.560062\\
0.268872\\
0.036630\\
0.134436
\end{pmatrix} =
0.993699\cdot
\begin{pmatrix}
0.563613\\
\color{gr} 0.270577\color{black} \\
0.036862\\
0.135288
\end{pmatrix},
\end{equation*}
\begin{equation*}
\left[ \frac{{w}^{\prime}_i}{{w}^{\prime}_j} \right] =
\begin{pmatrix}
$\,\,$ 1 $\,\,$ & $\,\,$\color{gr} 2.0830\color{black} $\,\,$ & $\,\,$15.2896$\,\,$ & $\,\,$4.1660$\,\,$ \\
$\,\,$\color{gr} 0.4801\color{black} $\,\,$ & $\,\,$ 1 $\,\,$ & $\,\,$\color{gr} 7.3402\color{black} $\,\,$ & $\,\,$\color{gr} \color{blue} 2\color{black}   $\,\,$ \\
$\,\,$0.0654$\,\,$ & $\,\,$\color{gr} 0.1362\color{black} $\,\,$ & $\,\,$ 1 $\,\,$ & $\,\,$0.2725 $\,\,$ \\
$\,\,$0.2400$\,\,$ & $\,\,$\color{gr} \color{blue}  1/2\color{black} $\,\,$ & $\,\,$3.6701$\,\,$ & $\,\,$ 1  $\,\,$ \\
\end{pmatrix},
\end{equation*}
\end{example}
\newpage
\begin{example}
\begin{equation*}
\mathbf{A} =
\begin{pmatrix}
$\,\,$ 1 $\,\,$ & $\,\,$2$\,\,$ & $\,\,$9$\,\,$ & $\,\,$7 $\,\,$ \\
$\,\,$ 1/2$\,\,$ & $\,\,$ 1 $\,\,$ & $\,\,$8$\,\,$ & $\,\,$2 $\,\,$ \\
$\,\,$ 1/9$\,\,$ & $\,\,$ 1/8$\,\,$ & $\,\,$ 1 $\,\,$ & $\,\,$ 1/7 $\,\,$ \\
$\,\,$ 1/7$\,\,$ & $\,\,$ 1/2$\,\,$ & $\,\,$7$\,\,$ & $\,\,$ 1  $\,\,$ \\
\end{pmatrix},
\qquad
\lambda_{\max} =
4.2506,
\qquad
CR = 0.0945
\end{equation*}

\begin{equation*}
\mathbf{w}^{EM} =
\begin{pmatrix}
0.562646\\
\color{red} 0.260697\color{black} \\
0.035463\\
0.141195
\end{pmatrix}\end{equation*}
\begin{equation*}
\left[ \frac{{w}^{EM}_i}{{w}^{EM}_j} \right] =
\begin{pmatrix}
$\,\,$ 1 $\,\,$ & $\,\,$\color{red} 2.1582\color{black} $\,\,$ & $\,\,$15.8659$\,\,$ & $\,\,$3.9849$\,\,$ \\
$\,\,$\color{red} 0.4633\color{black} $\,\,$ & $\,\,$ 1 $\,\,$ & $\,\,$\color{red} 7.3513\color{black} $\,\,$ & $\,\,$\color{red} 1.8464\color{black}   $\,\,$ \\
$\,\,$0.0630$\,\,$ & $\,\,$\color{red} 0.1360\color{black} $\,\,$ & $\,\,$ 1 $\,\,$ & $\,\,$0.2512 $\,\,$ \\
$\,\,$0.2509$\,\,$ & $\,\,$\color{red} 0.5416\color{black} $\,\,$ & $\,\,$3.9815$\,\,$ & $\,\,$ 1  $\,\,$ \\
\end{pmatrix},
\end{equation*}

\begin{equation*}
\mathbf{w}^{\prime} =
\begin{pmatrix}
0.551275\\
0.275638\\
0.034746\\
0.138342
\end{pmatrix} =
0.979791\cdot
\begin{pmatrix}
0.562646\\
\color{gr} 0.281323\color{black} \\
0.035463\\
0.141195
\end{pmatrix},
\end{equation*}
\begin{equation*}
\left[ \frac{{w}^{\prime}_i}{{w}^{\prime}_j} \right] =
\begin{pmatrix}
$\,\,$ 1 $\,\,$ & $\,\,$\color{gr} \color{blue} 2\color{black} $\,\,$ & $\,\,$15.8659$\,\,$ & $\,\,$3.9849$\,\,$ \\
$\,\,$\color{gr} \color{blue}  1/2\color{black} $\,\,$ & $\,\,$ 1 $\,\,$ & $\,\,$\color{gr} 7.9329\color{black} $\,\,$ & $\,\,$\color{gr} 1.9924\color{black}   $\,\,$ \\
$\,\,$0.0630$\,\,$ & $\,\,$\color{gr} 0.1261\color{black} $\,\,$ & $\,\,$ 1 $\,\,$ & $\,\,$0.2512 $\,\,$ \\
$\,\,$0.2509$\,\,$ & $\,\,$\color{gr} 0.5019\color{black} $\,\,$ & $\,\,$3.9815$\,\,$ & $\,\,$ 1  $\,\,$ \\
\end{pmatrix},
\end{equation*}
\end{example}
\newpage
\begin{example}
\begin{equation*}
\mathbf{A} =
\begin{pmatrix}
$\,\,$ 1 $\,\,$ & $\,\,$2$\,\,$ & $\,\,$9$\,\,$ & $\,\,$7 $\,\,$ \\
$\,\,$ 1/2$\,\,$ & $\,\,$ 1 $\,\,$ & $\,\,$9$\,\,$ & $\,\,$2 $\,\,$ \\
$\,\,$ 1/9$\,\,$ & $\,\,$ 1/9$\,\,$ & $\,\,$ 1 $\,\,$ & $\,\,$ 1/7 $\,\,$ \\
$\,\,$ 1/7$\,\,$ & $\,\,$ 1/2$\,\,$ & $\,\,$7$\,\,$ & $\,\,$ 1  $\,\,$ \\
\end{pmatrix},
\qquad
\lambda_{\max} =
4.2526,
\qquad
CR = 0.0952
\end{equation*}

\begin{equation*}
\mathbf{w}^{EM} =
\begin{pmatrix}
0.559337\\
\color{red} 0.266809\color{black} \\
0.034349\\
0.139505
\end{pmatrix}\end{equation*}
\begin{equation*}
\left[ \frac{{w}^{EM}_i}{{w}^{EM}_j} \right] =
\begin{pmatrix}
$\,\,$ 1 $\,\,$ & $\,\,$\color{red} 2.0964\color{black} $\,\,$ & $\,\,$16.2839$\,\,$ & $\,\,$4.0094$\,\,$ \\
$\,\,$\color{red} 0.4770\color{black} $\,\,$ & $\,\,$ 1 $\,\,$ & $\,\,$\color{red} 7.7676\color{black} $\,\,$ & $\,\,$\color{red} 1.9125\color{black}   $\,\,$ \\
$\,\,$0.0614$\,\,$ & $\,\,$\color{red} 0.1287\color{black} $\,\,$ & $\,\,$ 1 $\,\,$ & $\,\,$0.2462 $\,\,$ \\
$\,\,$0.2494$\,\,$ & $\,\,$\color{red} 0.5229\color{black} $\,\,$ & $\,\,$4.0614$\,\,$ & $\,\,$ 1  $\,\,$ \\
\end{pmatrix},
\end{equation*}

\begin{equation*}
\mathbf{w}^{\prime} =
\begin{pmatrix}
0.552595\\
0.275647\\
0.033935\\
0.137823
\end{pmatrix} =
0.987946\cdot
\begin{pmatrix}
0.559337\\
\color{gr} 0.279010\color{black} \\
0.034349\\
0.139505
\end{pmatrix},
\end{equation*}
\begin{equation*}
\left[ \frac{{w}^{\prime}_i}{{w}^{\prime}_j} \right] =
\begin{pmatrix}
$\,\,$ 1 $\,\,$ & $\,\,$\color{gr} 2.0047\color{black} $\,\,$ & $\,\,$16.2839$\,\,$ & $\,\,$4.0094$\,\,$ \\
$\,\,$\color{gr} 0.4988\color{black} $\,\,$ & $\,\,$ 1 $\,\,$ & $\,\,$\color{gr} 8.1228\color{black} $\,\,$ & $\,\,$\color{gr} \color{blue} 2\color{black}   $\,\,$ \\
$\,\,$0.0614$\,\,$ & $\,\,$\color{gr} 0.1231\color{black} $\,\,$ & $\,\,$ 1 $\,\,$ & $\,\,$0.2462 $\,\,$ \\
$\,\,$0.2494$\,\,$ & $\,\,$\color{gr} \color{blue}  1/2\color{black} $\,\,$ & $\,\,$4.0614$\,\,$ & $\,\,$ 1  $\,\,$ \\
\end{pmatrix},
\end{equation*}
\end{example}
\newpage
\begin{example}
\begin{equation*}
\mathbf{A} =
\begin{pmatrix}
$\,\,$ 1 $\,\,$ & $\,\,$2$\,\,$ & $\,\,$9$\,\,$ & $\,\,$8 $\,\,$ \\
$\,\,$ 1/2$\,\,$ & $\,\,$ 1 $\,\,$ & $\,\,$3$\,\,$ & $\,\,$6 $\,\,$ \\
$\,\,$ 1/9$\,\,$ & $\,\,$ 1/3$\,\,$ & $\,\,$ 1 $\,\,$ & $\,\,$ 1/2 $\,\,$ \\
$\,\,$ 1/8$\,\,$ & $\,\,$ 1/6$\,\,$ & $\,\,$2$\,\,$ & $\,\,$ 1  $\,\,$ \\
\end{pmatrix},
\qquad
\lambda_{\max} =
4.1664,
\qquad
CR = 0.0627
\end{equation*}

\begin{equation*}
\mathbf{w}^{EM} =
\begin{pmatrix}
\color{red} 0.563011\color{black} \\
0.296064\\
0.063197\\
0.077727
\end{pmatrix}\end{equation*}
\begin{equation*}
\left[ \frac{{w}^{EM}_i}{{w}^{EM}_j} \right] =
\begin{pmatrix}
$\,\,$ 1 $\,\,$ & $\,\,$\color{red} 1.9017\color{black} $\,\,$ & $\,\,$\color{red} 8.9088\color{black} $\,\,$ & $\,\,$\color{red} 7.2434\color{black} $\,\,$ \\
$\,\,$\color{red} 0.5259\color{black} $\,\,$ & $\,\,$ 1 $\,\,$ & $\,\,$4.6848$\,\,$ & $\,\,$3.8090  $\,\,$ \\
$\,\,$\color{red} 0.1122\color{black} $\,\,$ & $\,\,$0.2135$\,\,$ & $\,\,$ 1 $\,\,$ & $\,\,$0.8131 $\,\,$ \\
$\,\,$\color{red} 0.1381\color{black} $\,\,$ & $\,\,$0.2625$\,\,$ & $\,\,$1.2299$\,\,$ & $\,\,$ 1  $\,\,$ \\
\end{pmatrix},
\end{equation*}

\begin{equation*}
\mathbf{w}^{\prime} =
\begin{pmatrix}
0.565516\\
0.294367\\
0.062835\\
0.077281
\end{pmatrix} =
0.994268\cdot
\begin{pmatrix}
\color{gr} 0.568776\color{black} \\
0.296064\\
0.063197\\
0.077727
\end{pmatrix},
\end{equation*}
\begin{equation*}
\left[ \frac{{w}^{\prime}_i}{{w}^{\prime}_j} \right] =
\begin{pmatrix}
$\,\,$ 1 $\,\,$ & $\,\,$\color{gr} 1.9211\color{black} $\,\,$ & $\,\,$\color{gr} \color{blue} 9\color{black} $\,\,$ & $\,\,$\color{gr} 7.3176\color{black} $\,\,$ \\
$\,\,$\color{gr} 0.5205\color{black} $\,\,$ & $\,\,$ 1 $\,\,$ & $\,\,$4.6848$\,\,$ & $\,\,$3.8090  $\,\,$ \\
$\,\,$\color{gr} \color{blue}  1/9\color{black} $\,\,$ & $\,\,$0.2135$\,\,$ & $\,\,$ 1 $\,\,$ & $\,\,$0.8131 $\,\,$ \\
$\,\,$\color{gr} 0.1367\color{black} $\,\,$ & $\,\,$0.2625$\,\,$ & $\,\,$1.2299$\,\,$ & $\,\,$ 1  $\,\,$ \\
\end{pmatrix},
\end{equation*}
\end{example}
\newpage
\begin{example}
\begin{equation*}
\mathbf{A} =
\begin{pmatrix}
$\,\,$ 1 $\,\,$ & $\,\,$2$\,\,$ & $\,\,$9$\,\,$ & $\,\,$8 $\,\,$ \\
$\,\,$ 1/2$\,\,$ & $\,\,$ 1 $\,\,$ & $\,\,$3$\,\,$ & $\,\,$7 $\,\,$ \\
$\,\,$ 1/9$\,\,$ & $\,\,$ 1/3$\,\,$ & $\,\,$ 1 $\,\,$ & $\,\,$ 1/2 $\,\,$ \\
$\,\,$ 1/8$\,\,$ & $\,\,$ 1/7$\,\,$ & $\,\,$2$\,\,$ & $\,\,$ 1  $\,\,$ \\
\end{pmatrix},
\qquad
\lambda_{\max} =
4.2065,
\qquad
CR = 0.0779
\end{equation*}

\begin{equation*}
\mathbf{w}^{EM} =
\begin{pmatrix}
\color{red} 0.554558\color{black} \\
0.308030\\
0.062862\\
0.074550
\end{pmatrix}\end{equation*}
\begin{equation*}
\left[ \frac{{w}^{EM}_i}{{w}^{EM}_j} \right] =
\begin{pmatrix}
$\,\,$ 1 $\,\,$ & $\,\,$\color{red} 1.8003\color{black} $\,\,$ & $\,\,$\color{red} 8.8219\color{black} $\,\,$ & $\,\,$\color{red} 7.4388\color{black} $\,\,$ \\
$\,\,$\color{red} 0.5555\color{black} $\,\,$ & $\,\,$ 1 $\,\,$ & $\,\,$4.9001$\,\,$ & $\,\,$4.1319  $\,\,$ \\
$\,\,$\color{red} 0.1134\color{black} $\,\,$ & $\,\,$0.2041$\,\,$ & $\,\,$ 1 $\,\,$ & $\,\,$0.8432 $\,\,$ \\
$\,\,$\color{red} 0.1344\color{black} $\,\,$ & $\,\,$0.2420$\,\,$ & $\,\,$1.1859$\,\,$ & $\,\,$ 1  $\,\,$ \\
\end{pmatrix},
\end{equation*}

\begin{equation*}
\mathbf{w}^{\prime} =
\begin{pmatrix}
0.559491\\
0.304619\\
0.062166\\
0.073724
\end{pmatrix} =
0.988926\cdot
\begin{pmatrix}
\color{gr} 0.565756\color{black} \\
0.308030\\
0.062862\\
0.074550
\end{pmatrix},
\end{equation*}
\begin{equation*}
\left[ \frac{{w}^{\prime}_i}{{w}^{\prime}_j} \right] =
\begin{pmatrix}
$\,\,$ 1 $\,\,$ & $\,\,$\color{gr} 1.8367\color{black} $\,\,$ & $\,\,$\color{gr} \color{blue} 9\color{black} $\,\,$ & $\,\,$\color{gr} 7.5890\color{black} $\,\,$ \\
$\,\,$\color{gr} 0.5445\color{black} $\,\,$ & $\,\,$ 1 $\,\,$ & $\,\,$4.9001$\,\,$ & $\,\,$4.1319  $\,\,$ \\
$\,\,$\color{gr} \color{blue}  1/9\color{black} $\,\,$ & $\,\,$0.2041$\,\,$ & $\,\,$ 1 $\,\,$ & $\,\,$0.8432 $\,\,$ \\
$\,\,$\color{gr} 0.1318\color{black} $\,\,$ & $\,\,$0.2420$\,\,$ & $\,\,$1.1859$\,\,$ & $\,\,$ 1  $\,\,$ \\
\end{pmatrix},
\end{equation*}
\end{example}
\newpage
\begin{example}
\begin{equation*}
\mathbf{A} =
\begin{pmatrix}
$\,\,$ 1 $\,\,$ & $\,\,$2$\,\,$ & $\,\,$9$\,\,$ & $\,\,$8 $\,\,$ \\
$\,\,$ 1/2$\,\,$ & $\,\,$ 1 $\,\,$ & $\,\,$3$\,\,$ & $\,\,$8 $\,\,$ \\
$\,\,$ 1/9$\,\,$ & $\,\,$ 1/3$\,\,$ & $\,\,$ 1 $\,\,$ & $\,\,$ 1/2 $\,\,$ \\
$\,\,$ 1/8$\,\,$ & $\,\,$ 1/8$\,\,$ & $\,\,$2$\,\,$ & $\,\,$ 1  $\,\,$ \\
\end{pmatrix},
\qquad
\lambda_{\max} =
4.2469,
\qquad
CR = 0.0931
\end{equation*}

\begin{equation*}
\mathbf{w}^{EM} =
\begin{pmatrix}
\color{red} 0.546719\color{black} \\
0.318936\\
0.062513\\
0.071832
\end{pmatrix}\end{equation*}
\begin{equation*}
\left[ \frac{{w}^{EM}_i}{{w}^{EM}_j} \right] =
\begin{pmatrix}
$\,\,$ 1 $\,\,$ & $\,\,$\color{red} 1.7142\color{black} $\,\,$ & $\,\,$\color{red} 8.7457\color{black} $\,\,$ & $\,\,$\color{red} 7.6110\color{black} $\,\,$ \\
$\,\,$\color{red} 0.5834\color{black} $\,\,$ & $\,\,$ 1 $\,\,$ & $\,\,$5.1019$\,\,$ & $\,\,$4.4400  $\,\,$ \\
$\,\,$\color{red} 0.1143\color{black} $\,\,$ & $\,\,$0.1960$\,\,$ & $\,\,$ 1 $\,\,$ & $\,\,$0.8703 $\,\,$ \\
$\,\,$\color{red} 0.1314\color{black} $\,\,$ & $\,\,$0.2252$\,\,$ & $\,\,$1.1491$\,\,$ & $\,\,$ 1  $\,\,$ \\
\end{pmatrix},
\end{equation*}

\begin{equation*}
\mathbf{w}^{\prime} =
\begin{pmatrix}
0.553813\\
0.313944\\
0.061535\\
0.070708
\end{pmatrix} =
0.984350\cdot
\begin{pmatrix}
\color{gr} 0.562618\color{black} \\
0.318936\\
0.062513\\
0.071832
\end{pmatrix},
\end{equation*}
\begin{equation*}
\left[ \frac{{w}^{\prime}_i}{{w}^{\prime}_j} \right] =
\begin{pmatrix}
$\,\,$ 1 $\,\,$ & $\,\,$\color{gr} 1.7640\color{black} $\,\,$ & $\,\,$\color{gr} \color{blue} 9\color{black} $\,\,$ & $\,\,$\color{gr} 7.8324\color{black} $\,\,$ \\
$\,\,$\color{gr} 0.5669\color{black} $\,\,$ & $\,\,$ 1 $\,\,$ & $\,\,$5.1019$\,\,$ & $\,\,$4.4400  $\,\,$ \\
$\,\,$\color{gr} \color{blue}  1/9\color{black} $\,\,$ & $\,\,$0.1960$\,\,$ & $\,\,$ 1 $\,\,$ & $\,\,$0.8703 $\,\,$ \\
$\,\,$\color{gr} 0.1277\color{black} $\,\,$ & $\,\,$0.2252$\,\,$ & $\,\,$1.1491$\,\,$ & $\,\,$ 1  $\,\,$ \\
\end{pmatrix},
\end{equation*}
\end{example}
\newpage
\begin{example}
\begin{equation*}
\mathbf{A} =
\begin{pmatrix}
$\,\,$ 1 $\,\,$ & $\,\,$2$\,\,$ & $\,\,$9$\,\,$ & $\,\,$8 $\,\,$ \\
$\,\,$ 1/2$\,\,$ & $\,\,$ 1 $\,\,$ & $\,\,$7$\,\,$ & $\,\,$3 $\,\,$ \\
$\,\,$ 1/9$\,\,$ & $\,\,$ 1/7$\,\,$ & $\,\,$ 1 $\,\,$ & $\,\,$ 1/4 $\,\,$ \\
$\,\,$ 1/8$\,\,$ & $\,\,$ 1/3$\,\,$ & $\,\,$4$\,\,$ & $\,\,$ 1  $\,\,$ \\
\end{pmatrix},
\qquad
\lambda_{\max} =
4.1397,
\qquad
CR = 0.0527
\end{equation*}

\begin{equation*}
\mathbf{w}^{EM} =
\begin{pmatrix}
0.568919\\
\color{red} 0.283985\color{black} \\
0.041466\\
0.105630
\end{pmatrix}\end{equation*}
\begin{equation*}
\left[ \frac{{w}^{EM}_i}{{w}^{EM}_j} \right] =
\begin{pmatrix}
$\,\,$ 1 $\,\,$ & $\,\,$\color{red} 2.0033\color{black} $\,\,$ & $\,\,$13.7200$\,\,$ & $\,\,$5.3860$\,\,$ \\
$\,\,$\color{red} 0.4992\color{black} $\,\,$ & $\,\,$ 1 $\,\,$ & $\,\,$\color{red} 6.8486\color{black} $\,\,$ & $\,\,$\color{red} 2.6885\color{black}   $\,\,$ \\
$\,\,$0.0729$\,\,$ & $\,\,$\color{red} 0.1460\color{black} $\,\,$ & $\,\,$ 1 $\,\,$ & $\,\,$0.3926 $\,\,$ \\
$\,\,$0.1857$\,\,$ & $\,\,$\color{red} 0.3720\color{black} $\,\,$ & $\,\,$2.5474$\,\,$ & $\,\,$ 1  $\,\,$ \\
\end{pmatrix},
\end{equation*}

\begin{equation*}
\mathbf{w}^{\prime} =
\begin{pmatrix}
0.568649\\
0.284324\\
0.041447\\
0.105580
\end{pmatrix} =
0.999526\cdot
\begin{pmatrix}
0.568919\\
\color{gr} 0.284459\color{black} \\
0.041466\\
0.105630
\end{pmatrix},
\end{equation*}
\begin{equation*}
\left[ \frac{{w}^{\prime}_i}{{w}^{\prime}_j} \right] =
\begin{pmatrix}
$\,\,$ 1 $\,\,$ & $\,\,$\color{gr} \color{blue} 2\color{black} $\,\,$ & $\,\,$13.7200$\,\,$ & $\,\,$5.3860$\,\,$ \\
$\,\,$\color{gr} \color{blue}  1/2\color{black} $\,\,$ & $\,\,$ 1 $\,\,$ & $\,\,$\color{gr} 6.8600\color{black} $\,\,$ & $\,\,$\color{gr} 2.6930\color{black}   $\,\,$ \\
$\,\,$0.0729$\,\,$ & $\,\,$\color{gr} 0.1458\color{black} $\,\,$ & $\,\,$ 1 $\,\,$ & $\,\,$0.3926 $\,\,$ \\
$\,\,$0.1857$\,\,$ & $\,\,$\color{gr} 0.3713\color{black} $\,\,$ & $\,\,$2.5474$\,\,$ & $\,\,$ 1  $\,\,$ \\
\end{pmatrix},
\end{equation*}
\end{example}
\newpage
\begin{example}
\begin{equation*}
\mathbf{A} =
\begin{pmatrix}
$\,\,$ 1 $\,\,$ & $\,\,$2$\,\,$ & $\,\,$9$\,\,$ & $\,\,$8 $\,\,$ \\
$\,\,$ 1/2$\,\,$ & $\,\,$ 1 $\,\,$ & $\,\,$8$\,\,$ & $\,\,$2 $\,\,$ \\
$\,\,$ 1/9$\,\,$ & $\,\,$ 1/8$\,\,$ & $\,\,$ 1 $\,\,$ & $\,\,$ 1/6 $\,\,$ \\
$\,\,$ 1/8$\,\,$ & $\,\,$ 1/2$\,\,$ & $\,\,$6$\,\,$ & $\,\,$ 1  $\,\,$ \\
\end{pmatrix},
\qquad
\lambda_{\max} =
4.2469,
\qquad
CR = 0.0931
\end{equation*}

\begin{equation*}
\mathbf{w}^{EM} =
\begin{pmatrix}
0.577053\\
\color{red} 0.257703\color{black} \\
0.036288\\
0.128956
\end{pmatrix}\end{equation*}
\begin{equation*}
\left[ \frac{{w}^{EM}_i}{{w}^{EM}_j} \right] =
\begin{pmatrix}
$\,\,$ 1 $\,\,$ & $\,\,$\color{red} 2.2392\color{black} $\,\,$ & $\,\,$15.9022$\,\,$ & $\,\,$4.4748$\,\,$ \\
$\,\,$\color{red} 0.4466\color{black} $\,\,$ & $\,\,$ 1 $\,\,$ & $\,\,$\color{red} 7.1017\color{black} $\,\,$ & $\,\,$\color{red} 1.9984\color{black}   $\,\,$ \\
$\,\,$0.0629$\,\,$ & $\,\,$\color{red} 0.1408\color{black} $\,\,$ & $\,\,$ 1 $\,\,$ & $\,\,$0.2814 $\,\,$ \\
$\,\,$0.2235$\,\,$ & $\,\,$\color{red} 0.5004\color{black} $\,\,$ & $\,\,$3.5537$\,\,$ & $\,\,$ 1  $\,\,$ \\
\end{pmatrix},
\end{equation*}

\begin{equation*}
\mathbf{w}^{\prime} =
\begin{pmatrix}
0.576933\\
0.257858\\
0.036280\\
0.128929
\end{pmatrix} =
0.999791\cdot
\begin{pmatrix}
0.577053\\
\color{gr} 0.257912\color{black} \\
0.036288\\
0.128956
\end{pmatrix},
\end{equation*}
\begin{equation*}
\left[ \frac{{w}^{\prime}_i}{{w}^{\prime}_j} \right] =
\begin{pmatrix}
$\,\,$ 1 $\,\,$ & $\,\,$\color{gr} 2.2374\color{black} $\,\,$ & $\,\,$15.9022$\,\,$ & $\,\,$4.4748$\,\,$ \\
$\,\,$\color{gr} 0.4469\color{black} $\,\,$ & $\,\,$ 1 $\,\,$ & $\,\,$\color{gr} 7.1074\color{black} $\,\,$ & $\,\,$\color{gr} \color{blue} 2\color{black}   $\,\,$ \\
$\,\,$0.0629$\,\,$ & $\,\,$\color{gr} 0.1407\color{black} $\,\,$ & $\,\,$ 1 $\,\,$ & $\,\,$0.2814 $\,\,$ \\
$\,\,$0.2235$\,\,$ & $\,\,$\color{gr} \color{blue}  1/2\color{black} $\,\,$ & $\,\,$3.5537$\,\,$ & $\,\,$ 1  $\,\,$ \\
\end{pmatrix},
\end{equation*}
\end{example}
\newpage
\begin{example}
\begin{equation*}
\mathbf{A} =
\begin{pmatrix}
$\,\,$ 1 $\,\,$ & $\,\,$2$\,\,$ & $\,\,$9$\,\,$ & $\,\,$9 $\,\,$ \\
$\,\,$ 1/2$\,\,$ & $\,\,$ 1 $\,\,$ & $\,\,$6$\,\,$ & $\,\,$3 $\,\,$ \\
$\,\,$ 1/9$\,\,$ & $\,\,$ 1/6$\,\,$ & $\,\,$ 1 $\,\,$ & $\,\,$ 1/3 $\,\,$ \\
$\,\,$ 1/9$\,\,$ & $\,\,$ 1/3$\,\,$ & $\,\,$3$\,\,$ & $\,\,$ 1  $\,\,$ \\
\end{pmatrix},
\qquad
\lambda_{\max} =
4.1031,
\qquad
CR = 0.0389
\end{equation*}

\begin{equation*}
\mathbf{w}^{EM} =
\begin{pmatrix}
0.584828\\
\color{red} 0.274535\color{black} \\
0.045866\\
0.094771
\end{pmatrix}\end{equation*}
\begin{equation*}
\left[ \frac{{w}^{EM}_i}{{w}^{EM}_j} \right] =
\begin{pmatrix}
$\,\,$ 1 $\,\,$ & $\,\,$\color{red} 2.1302\color{black} $\,\,$ & $\,\,$12.7509$\,\,$ & $\,\,$6.1709$\,\,$ \\
$\,\,$\color{red} 0.4694\color{black} $\,\,$ & $\,\,$ 1 $\,\,$ & $\,\,$\color{red} 5.9857\color{black} $\,\,$ & $\,\,$\color{red} 2.8968\color{black}   $\,\,$ \\
$\,\,$0.0784$\,\,$ & $\,\,$\color{red} 0.1671\color{black} $\,\,$ & $\,\,$ 1 $\,\,$ & $\,\,$0.4840 $\,\,$ \\
$\,\,$0.1621$\,\,$ & $\,\,$\color{red} 0.3452\color{black} $\,\,$ & $\,\,$2.0663$\,\,$ & $\,\,$ 1  $\,\,$ \\
\end{pmatrix},
\end{equation*}

\begin{equation*}
\mathbf{w}^{\prime} =
\begin{pmatrix}
0.584443\\
0.275012\\
0.045835\\
0.094709
\end{pmatrix} =
0.999342\cdot
\begin{pmatrix}
0.584828\\
\color{gr} 0.275193\color{black} \\
0.045866\\
0.094771
\end{pmatrix},
\end{equation*}
\begin{equation*}
\left[ \frac{{w}^{\prime}_i}{{w}^{\prime}_j} \right] =
\begin{pmatrix}
$\,\,$ 1 $\,\,$ & $\,\,$\color{gr} 2.1252\color{black} $\,\,$ & $\,\,$12.7509$\,\,$ & $\,\,$6.1709$\,\,$ \\
$\,\,$\color{gr} 0.4706\color{black} $\,\,$ & $\,\,$ 1 $\,\,$ & $\,\,$\color{gr} \color{blue} 6\color{black} $\,\,$ & $\,\,$\color{gr} 2.9038\color{black}   $\,\,$ \\
$\,\,$0.0784$\,\,$ & $\,\,$\color{gr} \color{blue}  1/6\color{black} $\,\,$ & $\,\,$ 1 $\,\,$ & $\,\,$0.4840 $\,\,$ \\
$\,\,$0.1621$\,\,$ & $\,\,$\color{gr} 0.3444\color{black} $\,\,$ & $\,\,$2.0663$\,\,$ & $\,\,$ 1  $\,\,$ \\
\end{pmatrix},
\end{equation*}
\end{example}
\newpage
\begin{example}
\begin{equation*}
\mathbf{A} =
\begin{pmatrix}
$\,\,$ 1 $\,\,$ & $\,\,$2$\,\,$ & $\,\,$9$\,\,$ & $\,\,$9 $\,\,$ \\
$\,\,$ 1/2$\,\,$ & $\,\,$ 1 $\,\,$ & $\,\,$7$\,\,$ & $\,\,$3 $\,\,$ \\
$\,\,$ 1/9$\,\,$ & $\,\,$ 1/7$\,\,$ & $\,\,$ 1 $\,\,$ & $\,\,$ 1/4 $\,\,$ \\
$\,\,$ 1/9$\,\,$ & $\,\,$ 1/3$\,\,$ & $\,\,$4$\,\,$ & $\,\,$ 1  $\,\,$ \\
\end{pmatrix},
\qquad
\lambda_{\max} =
4.1658,
\qquad
CR = 0.0625
\end{equation*}

\begin{equation*}
\mathbf{w}^{EM} =
\begin{pmatrix}
0.579831\\
\color{red} 0.277985\color{black} \\
0.040894\\
0.101290
\end{pmatrix}\end{equation*}
\begin{equation*}
\left[ \frac{{w}^{EM}_i}{{w}^{EM}_j} \right] =
\begin{pmatrix}
$\,\,$ 1 $\,\,$ & $\,\,$\color{red} 2.0858\color{black} $\,\,$ & $\,\,$14.1790$\,\,$ & $\,\,$5.7245$\,\,$ \\
$\,\,$\color{red} 0.4794\color{black} $\,\,$ & $\,\,$ 1 $\,\,$ & $\,\,$\color{red} 6.7978\color{black} $\,\,$ & $\,\,$\color{red} 2.7445\color{black}   $\,\,$ \\
$\,\,$0.0705$\,\,$ & $\,\,$\color{red} 0.1471\color{black} $\,\,$ & $\,\,$ 1 $\,\,$ & $\,\,$0.4037 $\,\,$ \\
$\,\,$0.1747$\,\,$ & $\,\,$\color{red} 0.3644\color{black} $\,\,$ & $\,\,$2.4769$\,\,$ & $\,\,$ 1  $\,\,$ \\
\end{pmatrix},
\end{equation*}

\begin{equation*}
\mathbf{w}^{\prime} =
\begin{pmatrix}
0.575075\\
0.283907\\
0.040558\\
0.100459
\end{pmatrix} =
0.991798\cdot
\begin{pmatrix}
0.579831\\
\color{gr} 0.286255\color{black} \\
0.040894\\
0.101290
\end{pmatrix},
\end{equation*}
\begin{equation*}
\left[ \frac{{w}^{\prime}_i}{{w}^{\prime}_j} \right] =
\begin{pmatrix}
$\,\,$ 1 $\,\,$ & $\,\,$\color{gr} 2.0256\color{black} $\,\,$ & $\,\,$14.1790$\,\,$ & $\,\,$5.7245$\,\,$ \\
$\,\,$\color{gr} 0.4937\color{black} $\,\,$ & $\,\,$ 1 $\,\,$ & $\,\,$\color{gr} \color{blue} 7\color{black} $\,\,$ & $\,\,$\color{gr} 2.8261\color{black}   $\,\,$ \\
$\,\,$0.0705$\,\,$ & $\,\,$\color{gr} \color{blue}  1/7\color{black} $\,\,$ & $\,\,$ 1 $\,\,$ & $\,\,$0.4037 $\,\,$ \\
$\,\,$0.1747$\,\,$ & $\,\,$\color{gr} 0.3538\color{black} $\,\,$ & $\,\,$2.4769$\,\,$ & $\,\,$ 1  $\,\,$ \\
\end{pmatrix},
\end{equation*}
\end{example}
\newpage
\begin{example}
\begin{equation*}
\mathbf{A} =
\begin{pmatrix}
$\,\,$ 1 $\,\,$ & $\,\,$2$\,\,$ & $\,\,$9$\,\,$ & $\,\,$9 $\,\,$ \\
$\,\,$ 1/2$\,\,$ & $\,\,$ 1 $\,\,$ & $\,\,$8$\,\,$ & $\,\,$3 $\,\,$ \\
$\,\,$ 1/9$\,\,$ & $\,\,$ 1/8$\,\,$ & $\,\,$ 1 $\,\,$ & $\,\,$ 1/4 $\,\,$ \\
$\,\,$ 1/9$\,\,$ & $\,\,$ 1/3$\,\,$ & $\,\,$4$\,\,$ & $\,\,$ 1  $\,\,$ \\
\end{pmatrix},
\qquad
\lambda_{\max} =
4.1664,
\qquad
CR = 0.0627
\end{equation*}

\begin{equation*}
\mathbf{w}^{EM} =
\begin{pmatrix}
0.575782\\
\color{red} 0.284973\color{black} \\
0.039342\\
0.099904
\end{pmatrix}\end{equation*}
\begin{equation*}
\left[ \frac{{w}^{EM}_i}{{w}^{EM}_j} \right] =
\begin{pmatrix}
$\,\,$ 1 $\,\,$ & $\,\,$\color{red} 2.0205\color{black} $\,\,$ & $\,\,$14.6352$\,\,$ & $\,\,$5.7634$\,\,$ \\
$\,\,$\color{red} 0.4949\color{black} $\,\,$ & $\,\,$ 1 $\,\,$ & $\,\,$\color{red} 7.2434\color{black} $\,\,$ & $\,\,$\color{red} 2.8525\color{black}   $\,\,$ \\
$\,\,$0.0683$\,\,$ & $\,\,$\color{red} 0.1381\color{black} $\,\,$ & $\,\,$ 1 $\,\,$ & $\,\,$0.3938 $\,\,$ \\
$\,\,$0.1735$\,\,$ & $\,\,$\color{red} 0.3506\color{black} $\,\,$ & $\,\,$2.5394$\,\,$ & $\,\,$ 1  $\,\,$ \\
\end{pmatrix},
\end{equation*}

\begin{equation*}
\mathbf{w}^{\prime} =
\begin{pmatrix}
0.574106\\
0.287053\\
0.039228\\
0.099613
\end{pmatrix} =
0.997090\cdot
\begin{pmatrix}
0.575782\\
\color{gr} 0.287891\color{black} \\
0.039342\\
0.099904
\end{pmatrix},
\end{equation*}
\begin{equation*}
\left[ \frac{{w}^{\prime}_i}{{w}^{\prime}_j} \right] =
\begin{pmatrix}
$\,\,$ 1 $\,\,$ & $\,\,$\color{gr} \color{blue} 2\color{black} $\,\,$ & $\,\,$14.6352$\,\,$ & $\,\,$5.7634$\,\,$ \\
$\,\,$\color{gr} \color{blue}  1/2\color{black} $\,\,$ & $\,\,$ 1 $\,\,$ & $\,\,$\color{gr} 7.3176\color{black} $\,\,$ & $\,\,$\color{gr} 2.8817\color{black}   $\,\,$ \\
$\,\,$0.0683$\,\,$ & $\,\,$\color{gr} 0.1367\color{black} $\,\,$ & $\,\,$ 1 $\,\,$ & $\,\,$0.3938 $\,\,$ \\
$\,\,$0.1735$\,\,$ & $\,\,$\color{gr} 0.3470\color{black} $\,\,$ & $\,\,$2.5394$\,\,$ & $\,\,$ 1  $\,\,$ \\
\end{pmatrix},
\end{equation*}
\end{example}
\newpage
\begin{example}
\begin{equation*}
\mathbf{A} =
\begin{pmatrix}
$\,\,$ 1 $\,\,$ & $\,\,$2$\,\,$ & $\,\,$9$\,\,$ & $\,\,$9 $\,\,$ \\
$\,\,$ 1/2$\,\,$ & $\,\,$ 1 $\,\,$ & $\,\,$8$\,\,$ & $\,\,$3 $\,\,$ \\
$\,\,$ 1/9$\,\,$ & $\,\,$ 1/8$\,\,$ & $\,\,$ 1 $\,\,$ & $\,\,$ 1/5 $\,\,$ \\
$\,\,$ 1/9$\,\,$ & $\,\,$ 1/3$\,\,$ & $\,\,$5$\,\,$ & $\,\,$ 1  $\,\,$ \\
\end{pmatrix},
\qquad
\lambda_{\max} =
4.2267,
\qquad
CR = 0.0855
\end{equation*}

\begin{equation*}
\mathbf{w}^{EM} =
\begin{pmatrix}
0.575367\\
\color{red} 0.280734\color{black} \\
0.037295\\
0.106604
\end{pmatrix}\end{equation*}
\begin{equation*}
\left[ \frac{{w}^{EM}_i}{{w}^{EM}_j} \right] =
\begin{pmatrix}
$\,\,$ 1 $\,\,$ & $\,\,$\color{red} 2.0495\color{black} $\,\,$ & $\,\,$15.4274$\,\,$ & $\,\,$5.3972$\,\,$ \\
$\,\,$\color{red} 0.4879\color{black} $\,\,$ & $\,\,$ 1 $\,\,$ & $\,\,$\color{red} 7.5273\color{black} $\,\,$ & $\,\,$\color{red} 2.6334\color{black}   $\,\,$ \\
$\,\,$0.0648$\,\,$ & $\,\,$\color{red} 0.1328\color{black} $\,\,$ & $\,\,$ 1 $\,\,$ & $\,\,$0.3498 $\,\,$ \\
$\,\,$0.1853$\,\,$ & $\,\,$\color{red} 0.3797\color{black} $\,\,$ & $\,\,$2.8584$\,\,$ & $\,\,$ 1  $\,\,$ \\
\end{pmatrix},
\end{equation*}

\begin{equation*}
\mathbf{w}^{\prime} =
\begin{pmatrix}
0.571396\\
0.285698\\
0.037038\\
0.105868
\end{pmatrix} =
0.993099\cdot
\begin{pmatrix}
0.575367\\
\color{gr} 0.287683\color{black} \\
0.037295\\
0.106604
\end{pmatrix},
\end{equation*}
\begin{equation*}
\left[ \frac{{w}^{\prime}_i}{{w}^{\prime}_j} \right] =
\begin{pmatrix}
$\,\,$ 1 $\,\,$ & $\,\,$\color{gr} \color{blue} 2\color{black} $\,\,$ & $\,\,$15.4274$\,\,$ & $\,\,$5.3972$\,\,$ \\
$\,\,$\color{gr} \color{blue}  1/2\color{black} $\,\,$ & $\,\,$ 1 $\,\,$ & $\,\,$\color{gr} 7.7137\color{black} $\,\,$ & $\,\,$\color{gr} 2.6986\color{black}   $\,\,$ \\
$\,\,$0.0648$\,\,$ & $\,\,$\color{gr} 0.1296\color{black} $\,\,$ & $\,\,$ 1 $\,\,$ & $\,\,$0.3498 $\,\,$ \\
$\,\,$0.1853$\,\,$ & $\,\,$\color{gr} 0.3706\color{black} $\,\,$ & $\,\,$2.8584$\,\,$ & $\,\,$ 1  $\,\,$ \\
\end{pmatrix},
\end{equation*}
\end{example}
\newpage
\begin{example}
\begin{equation*}
\mathbf{A} =
\begin{pmatrix}
$\,\,$ 1 $\,\,$ & $\,\,$3$\,\,$ & $\,\,$4$\,\,$ & $\,\,$1 $\,\,$ \\
$\,\,$ 1/3$\,\,$ & $\,\,$ 1 $\,\,$ & $\,\,$2$\,\,$ & $\,\,$1 $\,\,$ \\
$\,\,$ 1/4$\,\,$ & $\,\,$ 1/2$\,\,$ & $\,\,$ 1 $\,\,$ & $\,\,$ 1/3 $\,\,$ \\
$\,\,$ 1 $\,\,$ & $\,\,$ 1 $\,\,$ & $\,\,$3$\,\,$ & $\,\,$ 1  $\,\,$ \\
\end{pmatrix},
\qquad
\lambda_{\max} =
4.1031,
\qquad
CR = 0.0389
\end{equation*}

\begin{equation*}
\mathbf{w}^{EM} =
\begin{pmatrix}
0.412083\\
0.200334\\
\color{red} 0.096722\color{black} \\
0.290861
\end{pmatrix}\end{equation*}
\begin{equation*}
\left[ \frac{{w}^{EM}_i}{{w}^{EM}_j} \right] =
\begin{pmatrix}
$\,\,$ 1 $\,\,$ & $\,\,$2.0570$\,\,$ & $\,\,$\color{red} 4.2605\color{black} $\,\,$ & $\,\,$1.4168$\,\,$ \\
$\,\,$0.4862$\,\,$ & $\,\,$ 1 $\,\,$ & $\,\,$\color{red} 2.0712\color{black} $\,\,$ & $\,\,$0.6888  $\,\,$ \\
$\,\,$\color{red} 0.2347\color{black} $\,\,$ & $\,\,$\color{red} 0.4828\color{black} $\,\,$ & $\,\,$ 1 $\,\,$ & $\,\,$\color{red} 0.3325\color{black}  $\,\,$ \\
$\,\,$0.7058$\,\,$ & $\,\,$1.4519$\,\,$ & $\,\,$\color{red} 3.0072\color{black} $\,\,$ & $\,\,$ 1  $\,\,$ \\
\end{pmatrix},
\end{equation*}

\begin{equation*}
\mathbf{w}^{\prime} =
\begin{pmatrix}
0.411987\\
0.200288\\
0.096931\\
0.290794
\end{pmatrix} =
0.999768\cdot
\begin{pmatrix}
0.412083\\
0.200334\\
\color{gr} 0.096954\color{black} \\
0.290861
\end{pmatrix},
\end{equation*}
\begin{equation*}
\left[ \frac{{w}^{\prime}_i}{{w}^{\prime}_j} \right] =
\begin{pmatrix}
$\,\,$ 1 $\,\,$ & $\,\,$2.0570$\,\,$ & $\,\,$\color{gr} 4.2503\color{black} $\,\,$ & $\,\,$1.4168$\,\,$ \\
$\,\,$0.4862$\,\,$ & $\,\,$ 1 $\,\,$ & $\,\,$\color{gr} 2.0663\color{black} $\,\,$ & $\,\,$0.6888  $\,\,$ \\
$\,\,$\color{gr} 0.2353\color{black} $\,\,$ & $\,\,$\color{gr} 0.4840\color{black} $\,\,$ & $\,\,$ 1 $\,\,$ & $\,\,$\color{gr} \color{blue}  1/3\color{black}  $\,\,$ \\
$\,\,$0.7058$\,\,$ & $\,\,$1.4519$\,\,$ & $\,\,$\color{gr} \color{blue} 3\color{black} $\,\,$ & $\,\,$ 1  $\,\,$ \\
\end{pmatrix},
\end{equation*}
\end{example}
\newpage
\begin{example}
\begin{equation*}
\mathbf{A} =
\begin{pmatrix}
$\,\,$ 1 $\,\,$ & $\,\,$3$\,\,$ & $\,\,$4$\,\,$ & $\,\,$2 $\,\,$ \\
$\,\,$ 1/3$\,\,$ & $\,\,$ 1 $\,\,$ & $\,\,$1$\,\,$ & $\,\,$1 $\,\,$ \\
$\,\,$ 1/4$\,\,$ & $\,\,$ 1 $\,\,$ & $\,\,$ 1 $\,\,$ & $\,\,$ 1/3 $\,\,$ \\
$\,\,$ 1/2$\,\,$ & $\,\,$ 1 $\,\,$ & $\,\,$3$\,\,$ & $\,\,$ 1  $\,\,$ \\
\end{pmatrix},
\qquad
\lambda_{\max} =
4.1031,
\qquad
CR = 0.0389
\end{equation*}

\begin{equation*}
\mathbf{w}^{EM} =
\begin{pmatrix}
\color{red} 0.470932\color{black} \\
0.167200\\
0.118015\\
0.243853
\end{pmatrix}\end{equation*}
\begin{equation*}
\left[ \frac{{w}^{EM}_i}{{w}^{EM}_j} \right] =
\begin{pmatrix}
$\,\,$ 1 $\,\,$ & $\,\,$\color{red} 2.8166\color{black} $\,\,$ & $\,\,$\color{red} 3.9904\color{black} $\,\,$ & $\,\,$\color{red} 1.9312\color{black} $\,\,$ \\
$\,\,$\color{red} 0.3550\color{black} $\,\,$ & $\,\,$ 1 $\,\,$ & $\,\,$1.4168$\,\,$ & $\,\,$0.6857  $\,\,$ \\
$\,\,$\color{red} 0.2506\color{black} $\,\,$ & $\,\,$0.7058$\,\,$ & $\,\,$ 1 $\,\,$ & $\,\,$0.4840 $\,\,$ \\
$\,\,$\color{red} 0.5178\color{black} $\,\,$ & $\,\,$1.4585$\,\,$ & $\,\,$2.0663$\,\,$ & $\,\,$ 1  $\,\,$ \\
\end{pmatrix},
\end{equation*}

\begin{equation*}
\mathbf{w}^{\prime} =
\begin{pmatrix}
0.471528\\
0.167012\\
0.117882\\
0.243578
\end{pmatrix} =
0.998873\cdot
\begin{pmatrix}
\color{gr} 0.472060\color{black} \\
0.167200\\
0.118015\\
0.243853
\end{pmatrix},
\end{equation*}
\begin{equation*}
\left[ \frac{{w}^{\prime}_i}{{w}^{\prime}_j} \right] =
\begin{pmatrix}
$\,\,$ 1 $\,\,$ & $\,\,$\color{gr} 2.8233\color{black} $\,\,$ & $\,\,$\color{gr} \color{blue} 4\color{black} $\,\,$ & $\,\,$\color{gr} 1.9358\color{black} $\,\,$ \\
$\,\,$\color{gr} 0.3542\color{black} $\,\,$ & $\,\,$ 1 $\,\,$ & $\,\,$1.4168$\,\,$ & $\,\,$0.6857  $\,\,$ \\
$\,\,$\color{gr} \color{blue}  1/4\color{black} $\,\,$ & $\,\,$0.7058$\,\,$ & $\,\,$ 1 $\,\,$ & $\,\,$0.4840 $\,\,$ \\
$\,\,$\color{gr} 0.5166\color{black} $\,\,$ & $\,\,$1.4585$\,\,$ & $\,\,$2.0663$\,\,$ & $\,\,$ 1  $\,\,$ \\
\end{pmatrix},
\end{equation*}
\end{example}
\newpage
\begin{example}
\begin{equation*}
\mathbf{A} =
\begin{pmatrix}
$\,\,$ 1 $\,\,$ & $\,\,$3$\,\,$ & $\,\,$5$\,\,$ & $\,\,$2 $\,\,$ \\
$\,\,$ 1/3$\,\,$ & $\,\,$ 1 $\,\,$ & $\,\,$1$\,\,$ & $\,\,$1 $\,\,$ \\
$\,\,$ 1/5$\,\,$ & $\,\,$ 1 $\,\,$ & $\,\,$ 1 $\,\,$ & $\,\,$ 1/4 $\,\,$ \\
$\,\,$ 1/2$\,\,$ & $\,\,$ 1 $\,\,$ & $\,\,$4$\,\,$ & $\,\,$ 1  $\,\,$ \\
\end{pmatrix},
\qquad
\lambda_{\max} =
4.1655,
\qquad
CR = 0.0624
\end{equation*}

\begin{equation*}
\mathbf{w}^{EM} =
\begin{pmatrix}
\color{red} 0.478142\color{black} \\
0.163542\\
0.102107\\
0.256209
\end{pmatrix}\end{equation*}
\begin{equation*}
\left[ \frac{{w}^{EM}_i}{{w}^{EM}_j} \right] =
\begin{pmatrix}
$\,\,$ 1 $\,\,$ & $\,\,$\color{red} 2.9237\color{black} $\,\,$ & $\,\,$\color{red} 4.6828\color{black} $\,\,$ & $\,\,$\color{red} 1.8662\color{black} $\,\,$ \\
$\,\,$\color{red} 0.3420\color{black} $\,\,$ & $\,\,$ 1 $\,\,$ & $\,\,$1.6017$\,\,$ & $\,\,$0.6383  $\,\,$ \\
$\,\,$\color{red} 0.2135\color{black} $\,\,$ & $\,\,$0.6243$\,\,$ & $\,\,$ 1 $\,\,$ & $\,\,$0.3985 $\,\,$ \\
$\,\,$\color{red} 0.5358\color{black} $\,\,$ & $\,\,$1.5666$\,\,$ & $\,\,$2.5092$\,\,$ & $\,\,$ 1  $\,\,$ \\
\end{pmatrix},
\end{equation*}

\begin{equation*}
\mathbf{w}^{\prime} =
\begin{pmatrix}
0.484576\\
0.161525\\
0.100848\\
0.253051
\end{pmatrix} =
0.987671\cdot
\begin{pmatrix}
\color{gr} 0.490625\color{black} \\
0.163542\\
0.102107\\
0.256209
\end{pmatrix},
\end{equation*}
\begin{equation*}
\left[ \frac{{w}^{\prime}_i}{{w}^{\prime}_j} \right] =
\begin{pmatrix}
$\,\,$ 1 $\,\,$ & $\,\,$\color{gr} \color{blue} 3\color{black} $\,\,$ & $\,\,$\color{gr} 4.8050\color{black} $\,\,$ & $\,\,$\color{gr} 1.9149\color{black} $\,\,$ \\
$\,\,$\color{gr} \color{blue}  1/3\color{black} $\,\,$ & $\,\,$ 1 $\,\,$ & $\,\,$1.6017$\,\,$ & $\,\,$0.6383  $\,\,$ \\
$\,\,$\color{gr} 0.2081\color{black} $\,\,$ & $\,\,$0.6243$\,\,$ & $\,\,$ 1 $\,\,$ & $\,\,$0.3985 $\,\,$ \\
$\,\,$\color{gr} 0.5222\color{black} $\,\,$ & $\,\,$1.5666$\,\,$ & $\,\,$2.5092$\,\,$ & $\,\,$ 1  $\,\,$ \\
\end{pmatrix},
\end{equation*}
\end{example}
\newpage
\begin{example}
\begin{equation*}
\mathbf{A} =
\begin{pmatrix}
$\,\,$ 1 $\,\,$ & $\,\,$3$\,\,$ & $\,\,$5$\,\,$ & $\,\,$2 $\,\,$ \\
$\,\,$ 1/3$\,\,$ & $\,\,$ 1 $\,\,$ & $\,\,$1$\,\,$ & $\,\,$1 $\,\,$ \\
$\,\,$ 1/5$\,\,$ & $\,\,$ 1 $\,\,$ & $\,\,$ 1 $\,\,$ & $\,\,$ 1/5 $\,\,$ \\
$\,\,$ 1/2$\,\,$ & $\,\,$ 1 $\,\,$ & $\,\,$5$\,\,$ & $\,\,$ 1  $\,\,$ \\
\end{pmatrix},
\qquad
\lambda_{\max} =
4.2277,
\qquad
CR = 0.0859
\end{equation*}

\begin{equation*}
\mathbf{w}^{EM} =
\begin{pmatrix}
\color{red} 0.468936\color{black} \\
0.162588\\
0.096294\\
0.272182
\end{pmatrix}\end{equation*}
\begin{equation*}
\left[ \frac{{w}^{EM}_i}{{w}^{EM}_j} \right] =
\begin{pmatrix}
$\,\,$ 1 $\,\,$ & $\,\,$\color{red} 2.8842\color{black} $\,\,$ & $\,\,$\color{red} 4.8698\color{black} $\,\,$ & $\,\,$\color{red} 1.7229\color{black} $\,\,$ \\
$\,\,$\color{red} 0.3467\color{black} $\,\,$ & $\,\,$ 1 $\,\,$ & $\,\,$1.6884$\,\,$ & $\,\,$0.5973  $\,\,$ \\
$\,\,$\color{red} 0.2053\color{black} $\,\,$ & $\,\,$0.5923$\,\,$ & $\,\,$ 1 $\,\,$ & $\,\,$0.3538 $\,\,$ \\
$\,\,$\color{red} 0.5804\color{black} $\,\,$ & $\,\,$1.6741$\,\,$ & $\,\,$2.8266$\,\,$ & $\,\,$ 1  $\,\,$ \\
\end{pmatrix},
\end{equation*}

\begin{equation*}
\mathbf{w}^{\prime} =
\begin{pmatrix}
0.475511\\
0.160575\\
0.095102\\
0.268812
\end{pmatrix} =
0.987620\cdot
\begin{pmatrix}
\color{gr} 0.481471\color{black} \\
0.162588\\
0.096294\\
0.272182
\end{pmatrix},
\end{equation*}
\begin{equation*}
\left[ \frac{{w}^{\prime}_i}{{w}^{\prime}_j} \right] =
\begin{pmatrix}
$\,\,$ 1 $\,\,$ & $\,\,$\color{gr} 2.9613\color{black} $\,\,$ & $\,\,$\color{gr} \color{blue} 5\color{black} $\,\,$ & $\,\,$\color{gr} 1.7689\color{black} $\,\,$ \\
$\,\,$\color{gr} 0.3377\color{black} $\,\,$ & $\,\,$ 1 $\,\,$ & $\,\,$1.6884$\,\,$ & $\,\,$0.5973  $\,\,$ \\
$\,\,$\color{gr} \color{blue}  1/5\color{black} $\,\,$ & $\,\,$0.5923$\,\,$ & $\,\,$ 1 $\,\,$ & $\,\,$0.3538 $\,\,$ \\
$\,\,$\color{gr} 0.5653\color{black} $\,\,$ & $\,\,$1.6741$\,\,$ & $\,\,$2.8266$\,\,$ & $\,\,$ 1  $\,\,$ \\
\end{pmatrix},
\end{equation*}
\end{example}
\newpage
\begin{example}
\begin{equation*}
\mathbf{A} =
\begin{pmatrix}
$\,\,$ 1 $\,\,$ & $\,\,$3$\,\,$ & $\,\,$5$\,\,$ & $\,\,$4 $\,\,$ \\
$\,\,$ 1/3$\,\,$ & $\,\,$ 1 $\,\,$ & $\,\,$1$\,\,$ & $\,\,$2 $\,\,$ \\
$\,\,$ 1/5$\,\,$ & $\,\,$ 1 $\,\,$ & $\,\,$ 1 $\,\,$ & $\,\,$ 1/2 $\,\,$ \\
$\,\,$ 1/4$\,\,$ & $\,\,$ 1/2$\,\,$ & $\,\,$2$\,\,$ & $\,\,$ 1  $\,\,$ \\
\end{pmatrix},
\qquad
\lambda_{\max} =
4.1655,
\qquad
CR = 0.0624
\end{equation*}

\begin{equation*}
\mathbf{w}^{EM} =
\begin{pmatrix}
\color{red} 0.548394\color{black} \\
0.187570\\
0.117109\\
0.146927
\end{pmatrix}\end{equation*}
\begin{equation*}
\left[ \frac{{w}^{EM}_i}{{w}^{EM}_j} \right] =
\begin{pmatrix}
$\,\,$ 1 $\,\,$ & $\,\,$\color{red} 2.9237\color{black} $\,\,$ & $\,\,$\color{red} 4.6828\color{black} $\,\,$ & $\,\,$\color{red} 3.7324\color{black} $\,\,$ \\
$\,\,$\color{red} 0.3420\color{black} $\,\,$ & $\,\,$ 1 $\,\,$ & $\,\,$1.6017$\,\,$ & $\,\,$1.2766  $\,\,$ \\
$\,\,$\color{red} 0.2135\color{black} $\,\,$ & $\,\,$0.6243$\,\,$ & $\,\,$ 1 $\,\,$ & $\,\,$0.7971 $\,\,$ \\
$\,\,$\color{red} 0.2679\color{black} $\,\,$ & $\,\,$0.7833$\,\,$ & $\,\,$1.2546$\,\,$ & $\,\,$ 1  $\,\,$ \\
\end{pmatrix},
\end{equation*}

\begin{equation*}
\mathbf{w}^{\prime} =
\begin{pmatrix}
0.554768\\
0.184923\\
0.115456\\
0.144853
\end{pmatrix} =
0.985886\cdot
\begin{pmatrix}
\color{gr} 0.562711\color{black} \\
0.187570\\
0.117109\\
0.146927
\end{pmatrix},
\end{equation*}
\begin{equation*}
\left[ \frac{{w}^{\prime}_i}{{w}^{\prime}_j} \right] =
\begin{pmatrix}
$\,\,$ 1 $\,\,$ & $\,\,$\color{gr} \color{blue} 3\color{black} $\,\,$ & $\,\,$\color{gr} 4.8050\color{black} $\,\,$ & $\,\,$\color{gr} 3.8299\color{black} $\,\,$ \\
$\,\,$\color{gr} \color{blue}  1/3\color{black} $\,\,$ & $\,\,$ 1 $\,\,$ & $\,\,$1.6017$\,\,$ & $\,\,$1.2766  $\,\,$ \\
$\,\,$\color{gr} 0.2081\color{black} $\,\,$ & $\,\,$0.6243$\,\,$ & $\,\,$ 1 $\,\,$ & $\,\,$0.7971 $\,\,$ \\
$\,\,$\color{gr} 0.2611\color{black} $\,\,$ & $\,\,$0.7833$\,\,$ & $\,\,$1.2546$\,\,$ & $\,\,$ 1  $\,\,$ \\
\end{pmatrix},
\end{equation*}
\end{example}
\newpage
\begin{example}
\begin{equation*}
\mathbf{A} =
\begin{pmatrix}
$\,\,$ 1 $\,\,$ & $\,\,$3$\,\,$ & $\,\,$5$\,\,$ & $\,\,$5 $\,\,$ \\
$\,\,$ 1/3$\,\,$ & $\,\,$ 1 $\,\,$ & $\,\,$9$\,\,$ & $\,\,$3 $\,\,$ \\
$\,\,$ 1/5$\,\,$ & $\,\,$ 1/9$\,\,$ & $\,\,$ 1 $\,\,$ & $\,\,$ 1/2 $\,\,$ \\
$\,\,$ 1/5$\,\,$ & $\,\,$ 1/3$\,\,$ & $\,\,$2$\,\,$ & $\,\,$ 1  $\,\,$ \\
\end{pmatrix},
\qquad
\lambda_{\max} =
4.2507,
\qquad
CR = 0.0946
\end{equation*}

\begin{equation*}
\mathbf{w}^{EM} =
\begin{pmatrix}
0.530906\\
0.309814\\
0.058720\\
\color{red} 0.100559\color{black}
\end{pmatrix}\end{equation*}
\begin{equation*}
\left[ \frac{{w}^{EM}_i}{{w}^{EM}_j} \right] =
\begin{pmatrix}
$\,\,$ 1 $\,\,$ & $\,\,$1.7136$\,\,$ & $\,\,$9.0413$\,\,$ & $\,\,$\color{red} 5.2795\color{black} $\,\,$ \\
$\,\,$0.5836$\,\,$ & $\,\,$ 1 $\,\,$ & $\,\,$5.2761$\,\,$ & $\,\,$\color{red} 3.0809\color{black}   $\,\,$ \\
$\,\,$0.1106$\,\,$ & $\,\,$0.1895$\,\,$ & $\,\,$ 1 $\,\,$ & $\,\,$\color{red} 0.5839\color{black}  $\,\,$ \\
$\,\,$\color{red} 0.1894\color{black} $\,\,$ & $\,\,$\color{red} 0.3246\color{black} $\,\,$ & $\,\,$\color{red} 1.7125\color{black} $\,\,$ & $\,\,$ 1  $\,\,$ \\
\end{pmatrix},
\end{equation*}

\begin{equation*}
\mathbf{w}^{\prime} =
\begin{pmatrix}
0.529470\\
0.308976\\
0.058561\\
0.102992
\end{pmatrix} =
0.997295\cdot
\begin{pmatrix}
0.530906\\
0.309814\\
0.058720\\
\color{gr} 0.103271\color{black}
\end{pmatrix},
\end{equation*}
\begin{equation*}
\left[ \frac{{w}^{\prime}_i}{{w}^{\prime}_j} \right] =
\begin{pmatrix}
$\,\,$ 1 $\,\,$ & $\,\,$1.7136$\,\,$ & $\,\,$9.0413$\,\,$ & $\,\,$\color{gr} 5.1409\color{black} $\,\,$ \\
$\,\,$0.5836$\,\,$ & $\,\,$ 1 $\,\,$ & $\,\,$5.2761$\,\,$ & $\,\,$\color{gr} \color{blue} 3\color{black}   $\,\,$ \\
$\,\,$0.1106$\,\,$ & $\,\,$0.1895$\,\,$ & $\,\,$ 1 $\,\,$ & $\,\,$\color{gr} 0.5686\color{black}  $\,\,$ \\
$\,\,$\color{gr} 0.1945\color{black} $\,\,$ & $\,\,$\color{gr} \color{blue}  1/3\color{black} $\,\,$ & $\,\,$\color{gr} 1.7587\color{black} $\,\,$ & $\,\,$ 1  $\,\,$ \\
\end{pmatrix},
\end{equation*}
\end{example}
\newpage
\begin{example}
\begin{equation*}
\mathbf{A} =
\begin{pmatrix}
$\,\,$ 1 $\,\,$ & $\,\,$3$\,\,$ & $\,\,$6$\,\,$ & $\,\,$1 $\,\,$ \\
$\,\,$ 1/3$\,\,$ & $\,\,$ 1 $\,\,$ & $\,\,$3$\,\,$ & $\,\,$1 $\,\,$ \\
$\,\,$ 1/6$\,\,$ & $\,\,$ 1/3$\,\,$ & $\,\,$ 1 $\,\,$ & $\,\,$ 1/4 $\,\,$ \\
$\,\,$ 1 $\,\,$ & $\,\,$ 1 $\,\,$ & $\,\,$4$\,\,$ & $\,\,$ 1  $\,\,$ \\
\end{pmatrix},
\qquad
\lambda_{\max} =
4.1031,
\qquad
CR = 0.0389
\end{equation*}

\begin{equation*}
\mathbf{w}^{EM} =
\begin{pmatrix}
0.429083\\
0.207659\\
\color{red} 0.069054\color{black} \\
0.294204
\end{pmatrix}\end{equation*}
\begin{equation*}
\left[ \frac{{w}^{EM}_i}{{w}^{EM}_j} \right] =
\begin{pmatrix}
$\,\,$ 1 $\,\,$ & $\,\,$2.0663$\,\,$ & $\,\,$\color{red} 6.2137\color{black} $\,\,$ & $\,\,$1.4585$\,\,$ \\
$\,\,$0.4840$\,\,$ & $\,\,$ 1 $\,\,$ & $\,\,$\color{red} 3.0072\color{black} $\,\,$ & $\,\,$0.7058  $\,\,$ \\
$\,\,$\color{red} 0.1609\color{black} $\,\,$ & $\,\,$\color{red} 0.3325\color{black} $\,\,$ & $\,\,$ 1 $\,\,$ & $\,\,$\color{red} 0.2347\color{black}  $\,\,$ \\
$\,\,$0.6857$\,\,$ & $\,\,$1.4168$\,\,$ & $\,\,$\color{red} 4.2605\color{black} $\,\,$ & $\,\,$ 1  $\,\,$ \\
\end{pmatrix},
\end{equation*}

\begin{equation*}
\mathbf{w}^{\prime} =
\begin{pmatrix}
0.429012\\
0.207624\\
0.069208\\
0.294156
\end{pmatrix} =
0.999835\cdot
\begin{pmatrix}
0.429083\\
0.207659\\
\color{gr} 0.069220\color{black} \\
0.294204
\end{pmatrix},
\end{equation*}
\begin{equation*}
\left[ \frac{{w}^{\prime}_i}{{w}^{\prime}_j} \right] =
\begin{pmatrix}
$\,\,$ 1 $\,\,$ & $\,\,$2.0663$\,\,$ & $\,\,$\color{gr} 6.1989\color{black} $\,\,$ & $\,\,$1.4585$\,\,$ \\
$\,\,$0.4840$\,\,$ & $\,\,$ 1 $\,\,$ & $\,\,$\color{gr} \color{blue} 3\color{black} $\,\,$ & $\,\,$0.7058  $\,\,$ \\
$\,\,$\color{gr} 0.1613\color{black} $\,\,$ & $\,\,$\color{gr} \color{blue}  1/3\color{black} $\,\,$ & $\,\,$ 1 $\,\,$ & $\,\,$\color{gr} 0.2353\color{black}  $\,\,$ \\
$\,\,$0.6857$\,\,$ & $\,\,$1.4168$\,\,$ & $\,\,$\color{gr} 4.2503\color{black} $\,\,$ & $\,\,$ 1  $\,\,$ \\
\end{pmatrix},
\end{equation*}
\end{example}
\newpage
\begin{example}
\begin{equation*}
\mathbf{A} =
\begin{pmatrix}
$\,\,$ 1 $\,\,$ & $\,\,$3$\,\,$ & $\,\,$6$\,\,$ & $\,\,$5 $\,\,$ \\
$\,\,$ 1/3$\,\,$ & $\,\,$ 1 $\,\,$ & $\,\,$9$\,\,$ & $\,\,$3 $\,\,$ \\
$\,\,$ 1/6$\,\,$ & $\,\,$ 1/9$\,\,$ & $\,\,$ 1 $\,\,$ & $\,\,$ 1/2 $\,\,$ \\
$\,\,$ 1/5$\,\,$ & $\,\,$ 1/3$\,\,$ & $\,\,$2$\,\,$ & $\,\,$ 1  $\,\,$ \\
\end{pmatrix},
\qquad
\lambda_{\max} =
4.1966,
\qquad
CR = 0.0741
\end{equation*}

\begin{equation*}
\mathbf{w}^{EM} =
\begin{pmatrix}
0.542642\\
0.303309\\
0.054420\\
\color{red} 0.099629\color{black}
\end{pmatrix}\end{equation*}
\begin{equation*}
\left[ \frac{{w}^{EM}_i}{{w}^{EM}_j} \right] =
\begin{pmatrix}
$\,\,$ 1 $\,\,$ & $\,\,$1.7891$\,\,$ & $\,\,$9.9714$\,\,$ & $\,\,$\color{red} 5.4466\color{black} $\,\,$ \\
$\,\,$0.5589$\,\,$ & $\,\,$ 1 $\,\,$ & $\,\,$5.5735$\,\,$ & $\,\,$\color{red} 3.0444\color{black}   $\,\,$ \\
$\,\,$0.1003$\,\,$ & $\,\,$0.1794$\,\,$ & $\,\,$ 1 $\,\,$ & $\,\,$\color{red} 0.5462\color{black}  $\,\,$ \\
$\,\,$\color{red} 0.1836\color{black} $\,\,$ & $\,\,$\color{red} 0.3285\color{black} $\,\,$ & $\,\,$\color{red} 1.8308\color{black} $\,\,$ & $\,\,$ 1  $\,\,$ \\
\end{pmatrix},
\end{equation*}

\begin{equation*}
\mathbf{w}^{\prime} =
\begin{pmatrix}
0.541844\\
0.302862\\
0.054340\\
0.100954
\end{pmatrix} =
0.998528\cdot
\begin{pmatrix}
0.542642\\
0.303309\\
0.054420\\
\color{gr} 0.101103\color{black}
\end{pmatrix},
\end{equation*}
\begin{equation*}
\left[ \frac{{w}^{\prime}_i}{{w}^{\prime}_j} \right] =
\begin{pmatrix}
$\,\,$ 1 $\,\,$ & $\,\,$1.7891$\,\,$ & $\,\,$9.9714$\,\,$ & $\,\,$\color{gr} 5.3672\color{black} $\,\,$ \\
$\,\,$0.5589$\,\,$ & $\,\,$ 1 $\,\,$ & $\,\,$5.5735$\,\,$ & $\,\,$\color{gr} \color{blue} 3\color{black}   $\,\,$ \\
$\,\,$0.1003$\,\,$ & $\,\,$0.1794$\,\,$ & $\,\,$ 1 $\,\,$ & $\,\,$\color{gr} 0.5383\color{black}  $\,\,$ \\
$\,\,$\color{gr} 0.1863\color{black} $\,\,$ & $\,\,$\color{gr} \color{blue}  1/3\color{black} $\,\,$ & $\,\,$\color{gr} 1.8578\color{black} $\,\,$ & $\,\,$ 1  $\,\,$ \\
\end{pmatrix},
\end{equation*}
\end{example}
\newpage
\begin{example}
\begin{equation*}
\mathbf{A} =
\begin{pmatrix}
$\,\,$ 1 $\,\,$ & $\,\,$3$\,\,$ & $\,\,$7$\,\,$ & $\,\,$5 $\,\,$ \\
$\,\,$ 1/3$\,\,$ & $\,\,$ 1 $\,\,$ & $\,\,$9$\,\,$ & $\,\,$3 $\,\,$ \\
$\,\,$ 1/7$\,\,$ & $\,\,$ 1/9$\,\,$ & $\,\,$ 1 $\,\,$ & $\,\,$ 1/2 $\,\,$ \\
$\,\,$ 1/5$\,\,$ & $\,\,$ 1/3$\,\,$ & $\,\,$2$\,\,$ & $\,\,$ 1  $\,\,$ \\
\end{pmatrix},
\qquad
\lambda_{\max} =
4.1583,
\qquad
CR = 0.0597
\end{equation*}

\begin{equation*}
\mathbf{w}^{EM} =
\begin{pmatrix}
0.552414\\
0.297726\\
0.051097\\
\color{red} 0.098763\color{black}
\end{pmatrix}\end{equation*}
\begin{equation*}
\left[ \frac{{w}^{EM}_i}{{w}^{EM}_j} \right] =
\begin{pmatrix}
$\,\,$ 1 $\,\,$ & $\,\,$1.8554$\,\,$ & $\,\,$10.8111$\,\,$ & $\,\,$\color{red} 5.5933\color{black} $\,\,$ \\
$\,\,$0.5390$\,\,$ & $\,\,$ 1 $\,\,$ & $\,\,$5.8267$\,\,$ & $\,\,$\color{red} 3.0146\color{black}   $\,\,$ \\
$\,\,$0.0925$\,\,$ & $\,\,$0.1716$\,\,$ & $\,\,$ 1 $\,\,$ & $\,\,$\color{red} 0.5174\color{black}  $\,\,$ \\
$\,\,$\color{red} 0.1788\color{black} $\,\,$ & $\,\,$\color{red} 0.3317\color{black} $\,\,$ & $\,\,$\color{red} 1.9328\color{black} $\,\,$ & $\,\,$ 1  $\,\,$ \\
\end{pmatrix},
\end{equation*}

\begin{equation*}
\mathbf{w}^{\prime} =
\begin{pmatrix}
0.552149\\
0.297584\\
0.051073\\
0.099195
\end{pmatrix} =
0.999521\cdot
\begin{pmatrix}
0.552414\\
0.297726\\
0.051097\\
\color{gr} 0.099242\color{black}
\end{pmatrix},
\end{equation*}
\begin{equation*}
\left[ \frac{{w}^{\prime}_i}{{w}^{\prime}_j} \right] =
\begin{pmatrix}
$\,\,$ 1 $\,\,$ & $\,\,$1.8554$\,\,$ & $\,\,$10.8111$\,\,$ & $\,\,$\color{gr} 5.5663\color{black} $\,\,$ \\
$\,\,$0.5390$\,\,$ & $\,\,$ 1 $\,\,$ & $\,\,$5.8267$\,\,$ & $\,\,$\color{gr} \color{blue} 3\color{black}   $\,\,$ \\
$\,\,$0.0925$\,\,$ & $\,\,$0.1716$\,\,$ & $\,\,$ 1 $\,\,$ & $\,\,$\color{gr} 0.5149\color{black}  $\,\,$ \\
$\,\,$\color{gr} 0.1797\color{black} $\,\,$ & $\,\,$\color{gr} \color{blue}  1/3\color{black} $\,\,$ & $\,\,$\color{gr} 1.9422\color{black} $\,\,$ & $\,\,$ 1  $\,\,$ \\
\end{pmatrix},
\end{equation*}
\end{example}
\newpage
\begin{example}
\begin{equation*}
\mathbf{A} =
\begin{pmatrix}
$\,\,$ 1 $\,\,$ & $\,\,$3$\,\,$ & $\,\,$7$\,\,$ & $\,\,$6 $\,\,$ \\
$\,\,$ 1/3$\,\,$ & $\,\,$ 1 $\,\,$ & $\,\,$8$\,\,$ & $\,\,$3 $\,\,$ \\
$\,\,$ 1/7$\,\,$ & $\,\,$ 1/8$\,\,$ & $\,\,$ 1 $\,\,$ & $\,\,$ 1/2 $\,\,$ \\
$\,\,$ 1/6$\,\,$ & $\,\,$ 1/3$\,\,$ & $\,\,$2$\,\,$ & $\,\,$ 1  $\,\,$ \\
\end{pmatrix},
\qquad
\lambda_{\max} =
4.1317,
\qquad
CR = 0.0496
\end{equation*}

\begin{equation*}
\mathbf{w}^{EM} =
\begin{pmatrix}
0.569382\\
0.284309\\
0.052327\\
\color{red} 0.093982\color{black}
\end{pmatrix}\end{equation*}
\begin{equation*}
\left[ \frac{{w}^{EM}_i}{{w}^{EM}_j} \right] =
\begin{pmatrix}
$\,\,$ 1 $\,\,$ & $\,\,$2.0027$\,\,$ & $\,\,$10.8812$\,\,$ & $\,\,$\color{red} 6.0584\color{black} $\,\,$ \\
$\,\,$0.4993$\,\,$ & $\,\,$ 1 $\,\,$ & $\,\,$5.4333$\,\,$ & $\,\,$\color{red} 3.0251\color{black}   $\,\,$ \\
$\,\,$0.0919$\,\,$ & $\,\,$0.1840$\,\,$ & $\,\,$ 1 $\,\,$ & $\,\,$\color{red} 0.5568\color{black}  $\,\,$ \\
$\,\,$\color{red} 0.1651\color{black} $\,\,$ & $\,\,$\color{red} 0.3306\color{black} $\,\,$ & $\,\,$\color{red} 1.7961\color{black} $\,\,$ & $\,\,$ 1  $\,\,$ \\
\end{pmatrix},
\end{equation*}

\begin{equation*}
\mathbf{w}^{\prime} =
\begin{pmatrix}
0.568934\\
0.284085\\
0.052286\\
0.094695
\end{pmatrix} =
0.999213\cdot
\begin{pmatrix}
0.569382\\
0.284309\\
0.052327\\
\color{gr} 0.094770\color{black}
\end{pmatrix},
\end{equation*}
\begin{equation*}
\left[ \frac{{w}^{\prime}_i}{{w}^{\prime}_j} \right] =
\begin{pmatrix}
$\,\,$ 1 $\,\,$ & $\,\,$2.0027$\,\,$ & $\,\,$10.8812$\,\,$ & $\,\,$\color{gr} 6.0081\color{black} $\,\,$ \\
$\,\,$0.4993$\,\,$ & $\,\,$ 1 $\,\,$ & $\,\,$5.4333$\,\,$ & $\,\,$\color{gr} \color{blue} 3\color{black}   $\,\,$ \\
$\,\,$0.0919$\,\,$ & $\,\,$0.1840$\,\,$ & $\,\,$ 1 $\,\,$ & $\,\,$\color{gr} 0.5521\color{black}  $\,\,$ \\
$\,\,$\color{gr} 0.1664\color{black} $\,\,$ & $\,\,$\color{gr} \color{blue}  1/3\color{black} $\,\,$ & $\,\,$\color{gr} 1.8111\color{black} $\,\,$ & $\,\,$ 1  $\,\,$ \\
\end{pmatrix},
\end{equation*}
\end{example}
\newpage
\begin{example}
\begin{equation*}
\mathbf{A} =
\begin{pmatrix}
$\,\,$ 1 $\,\,$ & $\,\,$3$\,\,$ & $\,\,$7$\,\,$ & $\,\,$6 $\,\,$ \\
$\,\,$ 1/3$\,\,$ & $\,\,$ 1 $\,\,$ & $\,\,$9$\,\,$ & $\,\,$3 $\,\,$ \\
$\,\,$ 1/7$\,\,$ & $\,\,$ 1/9$\,\,$ & $\,\,$ 1 $\,\,$ & $\,\,$ 1/2 $\,\,$ \\
$\,\,$ 1/6$\,\,$ & $\,\,$ 1/3$\,\,$ & $\,\,$2$\,\,$ & $\,\,$ 1  $\,\,$ \\
\end{pmatrix},
\qquad
\lambda_{\max} =
4.1571,
\qquad
CR = 0.0593
\end{equation*}

\begin{equation*}
\mathbf{w}^{EM} =
\begin{pmatrix}
0.565185\\
0.291677\\
0.050509\\
\color{red} 0.092629\color{black}
\end{pmatrix}\end{equation*}
\begin{equation*}
\left[ \frac{{w}^{EM}_i}{{w}^{EM}_j} \right] =
\begin{pmatrix}
$\,\,$ 1 $\,\,$ & $\,\,$1.9377$\,\,$ & $\,\,$11.1898$\,\,$ & $\,\,$\color{red} 6.1016\color{black} $\,\,$ \\
$\,\,$0.5161$\,\,$ & $\,\,$ 1 $\,\,$ & $\,\,$5.7747$\,\,$ & $\,\,$\color{red} 3.1489\color{black}   $\,\,$ \\
$\,\,$0.0894$\,\,$ & $\,\,$0.1732$\,\,$ & $\,\,$ 1 $\,\,$ & $\,\,$\color{red} 0.5453\color{black}  $\,\,$ \\
$\,\,$\color{red} 0.1639\color{black} $\,\,$ & $\,\,$\color{red} 0.3176\color{black} $\,\,$ & $\,\,$\color{red} 1.8339\color{black} $\,\,$ & $\,\,$ 1  $\,\,$ \\
\end{pmatrix},
\end{equation*}

\begin{equation*}
\mathbf{w}^{\prime} =
\begin{pmatrix}
0.564300\\
0.291220\\
0.050430\\
0.094050
\end{pmatrix} =
0.998434\cdot
\begin{pmatrix}
0.565185\\
0.291677\\
0.050509\\
\color{gr} 0.094198\color{black}
\end{pmatrix},
\end{equation*}
\begin{equation*}
\left[ \frac{{w}^{\prime}_i}{{w}^{\prime}_j} \right] =
\begin{pmatrix}
$\,\,$ 1 $\,\,$ & $\,\,$1.9377$\,\,$ & $\,\,$11.1898$\,\,$ & $\,\,$\color{gr} \color{blue} 6\color{black} $\,\,$ \\
$\,\,$0.5161$\,\,$ & $\,\,$ 1 $\,\,$ & $\,\,$5.7747$\,\,$ & $\,\,$\color{gr} 3.0964\color{black}   $\,\,$ \\
$\,\,$0.0894$\,\,$ & $\,\,$0.1732$\,\,$ & $\,\,$ 1 $\,\,$ & $\,\,$\color{gr} 0.5362\color{black}  $\,\,$ \\
$\,\,$\color{gr} \color{blue}  1/6\color{black} $\,\,$ & $\,\,$\color{gr} 0.3230\color{black} $\,\,$ & $\,\,$\color{gr} 1.8650\color{black} $\,\,$ & $\,\,$ 1  $\,\,$ \\
\end{pmatrix},
\end{equation*}
\end{example}
\newpage
\begin{example}
\begin{equation*}
\mathbf{A} =
\begin{pmatrix}
$\,\,$ 1 $\,\,$ & $\,\,$3$\,\,$ & $\,\,$8$\,\,$ & $\,\,$1 $\,\,$ \\
$\,\,$ 1/3$\,\,$ & $\,\,$ 1 $\,\,$ & $\,\,$4$\,\,$ & $\,\,$1 $\,\,$ \\
$\,\,$ 1/8$\,\,$ & $\,\,$ 1/4$\,\,$ & $\,\,$ 1 $\,\,$ & $\,\,$ 1/6 $\,\,$ \\
$\,\,$ 1 $\,\,$ & $\,\,$ 1 $\,\,$ & $\,\,$6$\,\,$ & $\,\,$ 1  $\,\,$ \\
\end{pmatrix},
\qquad
\lambda_{\max} =
4.1031,
\qquad
CR = 0.0389
\end{equation*}

\begin{equation*}
\mathbf{w}^{EM} =
\begin{pmatrix}
0.433024\\
0.210515\\
\color{red} 0.050819\color{black} \\
0.305642
\end{pmatrix}\end{equation*}
\begin{equation*}
\left[ \frac{{w}^{EM}_i}{{w}^{EM}_j} \right] =
\begin{pmatrix}
$\,\,$ 1 $\,\,$ & $\,\,$2.0570$\,\,$ & $\,\,$\color{red} 8.5210\color{black} $\,\,$ & $\,\,$1.4168$\,\,$ \\
$\,\,$0.4862$\,\,$ & $\,\,$ 1 $\,\,$ & $\,\,$\color{red} 4.1425\color{black} $\,\,$ & $\,\,$0.6888  $\,\,$ \\
$\,\,$\color{red} 0.1174\color{black} $\,\,$ & $\,\,$\color{red} 0.2414\color{black} $\,\,$ & $\,\,$ 1 $\,\,$ & $\,\,$\color{red} 0.1663\color{black}  $\,\,$ \\
$\,\,$0.7058$\,\,$ & $\,\,$1.4519$\,\,$ & $\,\,$\color{red} 6.0144\color{black} $\,\,$ & $\,\,$ 1  $\,\,$ \\
\end{pmatrix},
\end{equation*}

\begin{equation*}
\mathbf{w}^{\prime} =
\begin{pmatrix}
0.432972\\
0.210489\\
0.050934\\
0.305605
\end{pmatrix} =
0.999878\cdot
\begin{pmatrix}
0.433024\\
0.210515\\
\color{gr} 0.050940\color{black} \\
0.305642
\end{pmatrix},
\end{equation*}
\begin{equation*}
\left[ \frac{{w}^{\prime}_i}{{w}^{\prime}_j} \right] =
\begin{pmatrix}
$\,\,$ 1 $\,\,$ & $\,\,$2.0570$\,\,$ & $\,\,$\color{gr} 8.5006\color{black} $\,\,$ & $\,\,$1.4168$\,\,$ \\
$\,\,$0.4862$\,\,$ & $\,\,$ 1 $\,\,$ & $\,\,$\color{gr} 4.1326\color{black} $\,\,$ & $\,\,$0.6888  $\,\,$ \\
$\,\,$\color{gr} 0.1176\color{black} $\,\,$ & $\,\,$\color{gr} 0.2420\color{black} $\,\,$ & $\,\,$ 1 $\,\,$ & $\,\,$\color{gr} \color{blue}  1/6\color{black}  $\,\,$ \\
$\,\,$0.7058$\,\,$ & $\,\,$1.4519$\,\,$ & $\,\,$\color{gr} \color{blue} 6\color{black} $\,\,$ & $\,\,$ 1  $\,\,$ \\
\end{pmatrix},
\end{equation*}
\end{example}
\newpage
\begin{example}
\begin{equation*}
\mathbf{A} =
\begin{pmatrix}
$\,\,$ 1 $\,\,$ & $\,\,$3$\,\,$ & $\,\,$8$\,\,$ & $\,\,$3 $\,\,$ \\
$\,\,$ 1/3$\,\,$ & $\,\,$ 1 $\,\,$ & $\,\,$4$\,\,$ & $\,\,$3 $\,\,$ \\
$\,\,$ 1/8$\,\,$ & $\,\,$ 1/4$\,\,$ & $\,\,$ 1 $\,\,$ & $\,\,$ 1/2 $\,\,$ \\
$\,\,$ 1/3$\,\,$ & $\,\,$ 1/3$\,\,$ & $\,\,$2$\,\,$ & $\,\,$ 1  $\,\,$ \\
\end{pmatrix},
\qquad
\lambda_{\max} =
4.1031,
\qquad
CR = 0.0389
\end{equation*}

\begin{equation*}
\mathbf{w}^{EM} =
\begin{pmatrix}
0.543837\\
0.264387\\
\color{red} 0.063823\color{black} \\
0.127952
\end{pmatrix}\end{equation*}
\begin{equation*}
\left[ \frac{{w}^{EM}_i}{{w}^{EM}_j} \right] =
\begin{pmatrix}
$\,\,$ 1 $\,\,$ & $\,\,$2.0570$\,\,$ & $\,\,$\color{red} 8.5210\color{black} $\,\,$ & $\,\,$4.2503$\,\,$ \\
$\,\,$0.4862$\,\,$ & $\,\,$ 1 $\,\,$ & $\,\,$\color{red} 4.1425\color{black} $\,\,$ & $\,\,$2.0663  $\,\,$ \\
$\,\,$\color{red} 0.1174\color{black} $\,\,$ & $\,\,$\color{red} 0.2414\color{black} $\,\,$ & $\,\,$ 1 $\,\,$ & $\,\,$\color{red} 0.4988\color{black}  $\,\,$ \\
$\,\,$0.2353$\,\,$ & $\,\,$0.4840$\,\,$ & $\,\,$\color{red} 2.0048\color{black} $\,\,$ & $\,\,$ 1  $\,\,$ \\
\end{pmatrix},
\end{equation*}

\begin{equation*}
\mathbf{w}^{\prime} =
\begin{pmatrix}
0.543754\\
0.264346\\
0.063966\\
0.127933
\end{pmatrix} =
0.999847\cdot
\begin{pmatrix}
0.543837\\
0.264387\\
\color{gr} 0.063976\color{black} \\
0.127952
\end{pmatrix},
\end{equation*}
\begin{equation*}
\left[ \frac{{w}^{\prime}_i}{{w}^{\prime}_j} \right] =
\begin{pmatrix}
$\,\,$ 1 $\,\,$ & $\,\,$2.0570$\,\,$ & $\,\,$\color{gr} 8.5006\color{black} $\,\,$ & $\,\,$4.2503$\,\,$ \\
$\,\,$0.4862$\,\,$ & $\,\,$ 1 $\,\,$ & $\,\,$\color{gr} 4.1326\color{black} $\,\,$ & $\,\,$2.0663  $\,\,$ \\
$\,\,$\color{gr} 0.1176\color{black} $\,\,$ & $\,\,$\color{gr} 0.2420\color{black} $\,\,$ & $\,\,$ 1 $\,\,$ & $\,\,$\color{gr} \color{blue}  1/2\color{black}  $\,\,$ \\
$\,\,$0.2353$\,\,$ & $\,\,$0.4840$\,\,$ & $\,\,$\color{gr} \color{blue} 2\color{black} $\,\,$ & $\,\,$ 1  $\,\,$ \\
\end{pmatrix},
\end{equation*}
\end{example}
\newpage
\begin{example}
\begin{equation*}
\mathbf{A} =
\begin{pmatrix}
$\,\,$ 1 $\,\,$ & $\,\,$3$\,\,$ & $\,\,$8$\,\,$ & $\,\,$6 $\,\,$ \\
$\,\,$ 1/3$\,\,$ & $\,\,$ 1 $\,\,$ & $\,\,$8$\,\,$ & $\,\,$3 $\,\,$ \\
$\,\,$ 1/8$\,\,$ & $\,\,$ 1/8$\,\,$ & $\,\,$ 1 $\,\,$ & $\,\,$ 1/2 $\,\,$ \\
$\,\,$ 1/6$\,\,$ & $\,\,$ 1/3$\,\,$ & $\,\,$2$\,\,$ & $\,\,$ 1  $\,\,$ \\
\end{pmatrix},
\qquad
\lambda_{\max} =
4.1031,
\qquad
CR = 0.0389
\end{equation*}

\begin{equation*}
\mathbf{w}^{EM} =
\begin{pmatrix}
0.577834\\
0.279648\\
0.049525\\
\color{red} 0.092993\color{black}
\end{pmatrix}\end{equation*}
\begin{equation*}
\left[ \frac{{w}^{EM}_i}{{w}^{EM}_j} \right] =
\begin{pmatrix}
$\,\,$ 1 $\,\,$ & $\,\,$2.0663$\,\,$ & $\,\,$11.6676$\,\,$ & $\,\,$\color{red} 6.2137\color{black} $\,\,$ \\
$\,\,$0.4840$\,\,$ & $\,\,$ 1 $\,\,$ & $\,\,$5.6467$\,\,$ & $\,\,$\color{red} 3.0072\color{black}   $\,\,$ \\
$\,\,$0.0857$\,\,$ & $\,\,$0.1771$\,\,$ & $\,\,$ 1 $\,\,$ & $\,\,$\color{red} 0.5326\color{black}  $\,\,$ \\
$\,\,$\color{red} 0.1609\color{black} $\,\,$ & $\,\,$\color{red} 0.3325\color{black} $\,\,$ & $\,\,$\color{red} 1.8777\color{black} $\,\,$ & $\,\,$ 1  $\,\,$ \\
\end{pmatrix},
\end{equation*}

\begin{equation*}
\mathbf{w}^{\prime} =
\begin{pmatrix}
0.577705\\
0.279586\\
0.049514\\
0.093195
\end{pmatrix} =
0.999777\cdot
\begin{pmatrix}
0.577834\\
0.279648\\
0.049525\\
\color{gr} 0.093216\color{black}
\end{pmatrix},
\end{equation*}
\begin{equation*}
\left[ \frac{{w}^{\prime}_i}{{w}^{\prime}_j} \right] =
\begin{pmatrix}
$\,\,$ 1 $\,\,$ & $\,\,$2.0663$\,\,$ & $\,\,$11.6676$\,\,$ & $\,\,$\color{gr} 6.1989\color{black} $\,\,$ \\
$\,\,$0.4840$\,\,$ & $\,\,$ 1 $\,\,$ & $\,\,$5.6467$\,\,$ & $\,\,$\color{gr} \color{blue} 3\color{black}   $\,\,$ \\
$\,\,$0.0857$\,\,$ & $\,\,$0.1771$\,\,$ & $\,\,$ 1 $\,\,$ & $\,\,$\color{gr} 0.5313\color{black}  $\,\,$ \\
$\,\,$\color{gr} 0.1613\color{black} $\,\,$ & $\,\,$\color{gr} \color{blue}  1/3\color{black} $\,\,$ & $\,\,$\color{gr} 1.8822\color{black} $\,\,$ & $\,\,$ 1  $\,\,$ \\
\end{pmatrix},
\end{equation*}
\end{example}
\newpage
\begin{example}
\begin{equation*}
\mathbf{A} =
\begin{pmatrix}
$\,\,$ 1 $\,\,$ & $\,\,$3$\,\,$ & $\,\,$8$\,\,$ & $\,\,$6 $\,\,$ \\
$\,\,$ 1/3$\,\,$ & $\,\,$ 1 $\,\,$ & $\,\,$9$\,\,$ & $\,\,$3 $\,\,$ \\
$\,\,$ 1/8$\,\,$ & $\,\,$ 1/9$\,\,$ & $\,\,$ 1 $\,\,$ & $\,\,$ 1/2 $\,\,$ \\
$\,\,$ 1/6$\,\,$ & $\,\,$ 1/3$\,\,$ & $\,\,$2$\,\,$ & $\,\,$ 1  $\,\,$ \\
\end{pmatrix},
\qquad
\lambda_{\max} =
4.1263,
\qquad
CR = 0.0476
\end{equation*}

\begin{equation*}
\mathbf{w}^{EM} =
\begin{pmatrix}
0.573635\\
0.286818\\
0.047803\\
\color{red} 0.091744\color{black}
\end{pmatrix}\end{equation*}
\begin{equation*}
\left[ \frac{{w}^{EM}_i}{{w}^{EM}_j} \right] =
\begin{pmatrix}
$\,\,$ 1 $\,\,$ & $\,\,$2$\,\,$ & $\,\,$12$\,\,$ & $\,\,$\color{red} 6.2526\color{black} $\,\,$ \\
$\,\,$1/2$\,\,$ & $\,\,$ 1 $\,\,$ & $\,\,$6$\,\,$ & $\,\,$\color{red} 3.1263\color{black}   $\,\,$ \\
$\,\,$1/12$\,\,$ & $\,\,$1/6$\,\,$ & $\,\,$ 1 $\,\,$ & $\,\,$\color{red} 0.5210\color{black}  $\,\,$ \\
$\,\,$\color{red} 0.1599\color{black} $\,\,$ & $\,\,$\color{red} 0.3199\color{black} $\,\,$ & $\,\,$\color{red} 1.9192\color{black} $\,\,$ & $\,\,$ 1  $\,\,$ \\
\end{pmatrix},
\end{equation*}

\begin{equation*}
\mathbf{w}^{\prime} =
\begin{pmatrix}
0.571429\\
0.285714\\
0.047619\\
0.095238
\end{pmatrix} =
0.996153\cdot
\begin{pmatrix}
0.573635\\
0.286818\\
0.047803\\
\color{gr} 0.095606\color{black}
\end{pmatrix},
\end{equation*}
\begin{equation*}
\left[ \frac{{w}^{\prime}_i}{{w}^{\prime}_j} \right] =
\begin{pmatrix}
$\,\,$ 1 $\,\,$ & $\,\,$2$\,\,$ & $\,\,$12$\,\,$ & $\,\,$\color{blue} 6\color{black} $\,\,$ \\
$\,\,$1/2$\,\,$ & $\,\,$ 1 $\,\,$ & $\,\,$6$\,\,$ & $\,\,$\color{blue} 3\color{black}   $\,\,$ \\
$\,\,$1/12$\,\,$ & $\,\,$1/6$\,\,$ & $\,\,$ 1 $\,\,$ & $\,\,$\color{gr} \color{blue}  1/2\color{black}  $\,\,$ \\
$\,\,$\color{blue} 1/6\color{black} $\,\,$ & $\,\,$\color{blue} 1/3\color{black} $\,\,$ & $\,\,$\color{gr} \color{blue} 2\color{black} $\,\,$ & $\,\,$ 1  $\,\,$ \\
\end{pmatrix},
\end{equation*}
\end{example}
\newpage
\begin{example}
\begin{equation*}
\mathbf{A} =
\begin{pmatrix}
$\,\,$ 1 $\,\,$ & $\,\,$3$\,\,$ & $\,\,$9$\,\,$ & $\,\,$1 $\,\,$ \\
$\,\,$ 1/3$\,\,$ & $\,\,$ 1 $\,\,$ & $\,\,$4$\,\,$ & $\,\,$1 $\,\,$ \\
$\,\,$ 1/9$\,\,$ & $\,\,$ 1/4$\,\,$ & $\,\,$ 1 $\,\,$ & $\,\,$ 1/6 $\,\,$ \\
$\,\,$ 1 $\,\,$ & $\,\,$ 1 $\,\,$ & $\,\,$6$\,\,$ & $\,\,$ 1  $\,\,$ \\
\end{pmatrix},
\qquad
\lambda_{\max} =
4.1031,
\qquad
CR = 0.0389
\end{equation*}

\begin{equation*}
\mathbf{w}^{EM} =
\begin{pmatrix}
0.440167\\
0.207872\\
\color{red} 0.048791\color{black} \\
0.303171
\end{pmatrix}\end{equation*}
\begin{equation*}
\left[ \frac{{w}^{EM}_i}{{w}^{EM}_j} \right] =
\begin{pmatrix}
$\,\,$ 1 $\,\,$ & $\,\,$2.1175$\,\,$ & $\,\,$\color{red} 9.0216\color{black} $\,\,$ & $\,\,$1.4519$\,\,$ \\
$\,\,$0.4723$\,\,$ & $\,\,$ 1 $\,\,$ & $\,\,$\color{red} 4.2605\color{black} $\,\,$ & $\,\,$0.6857  $\,\,$ \\
$\,\,$\color{red} 0.1108\color{black} $\,\,$ & $\,\,$\color{red} 0.2347\color{black} $\,\,$ & $\,\,$ 1 $\,\,$ & $\,\,$\color{red} 0.1609\color{black}  $\,\,$ \\
$\,\,$0.6888$\,\,$ & $\,\,$1.4585$\,\,$ & $\,\,$\color{red} 6.2137\color{black} $\,\,$ & $\,\,$ 1  $\,\,$ \\
\end{pmatrix},
\end{equation*}

\begin{equation*}
\mathbf{w}^{\prime} =
\begin{pmatrix}
0.440116\\
0.207847\\
0.048902\\
0.303135
\end{pmatrix} =
0.999883\cdot
\begin{pmatrix}
0.440167\\
0.207872\\
\color{gr} 0.048907\color{black} \\
0.303171
\end{pmatrix},
\end{equation*}
\begin{equation*}
\left[ \frac{{w}^{\prime}_i}{{w}^{\prime}_j} \right] =
\begin{pmatrix}
$\,\,$ 1 $\,\,$ & $\,\,$2.1175$\,\,$ & $\,\,$\color{gr} \color{blue} 9\color{black} $\,\,$ & $\,\,$1.4519$\,\,$ \\
$\,\,$0.4723$\,\,$ & $\,\,$ 1 $\,\,$ & $\,\,$\color{gr} 4.2503\color{black} $\,\,$ & $\,\,$0.6857  $\,\,$ \\
$\,\,$\color{gr} \color{blue}  1/9\color{black} $\,\,$ & $\,\,$\color{gr} 0.2353\color{black} $\,\,$ & $\,\,$ 1 $\,\,$ & $\,\,$\color{gr} 0.1613\color{black}  $\,\,$ \\
$\,\,$0.6888$\,\,$ & $\,\,$1.4585$\,\,$ & $\,\,$\color{gr} 6.1989\color{black} $\,\,$ & $\,\,$ 1  $\,\,$ \\
\end{pmatrix},
\end{equation*}
\end{example}
\newpage
\begin{example}
\begin{equation*}
\mathbf{A} =
\begin{pmatrix}
$\,\,$ 1 $\,\,$ & $\,\,$3$\,\,$ & $\,\,$9$\,\,$ & $\,\,$3 $\,\,$ \\
$\,\,$ 1/3$\,\,$ & $\,\,$ 1 $\,\,$ & $\,\,$4$\,\,$ & $\,\,$3 $\,\,$ \\
$\,\,$ 1/9$\,\,$ & $\,\,$ 1/4$\,\,$ & $\,\,$ 1 $\,\,$ & $\,\,$ 1/2 $\,\,$ \\
$\,\,$ 1/3$\,\,$ & $\,\,$ 1/3$\,\,$ & $\,\,$2$\,\,$ & $\,\,$ 1  $\,\,$ \\
\end{pmatrix},
\qquad
\lambda_{\max} =
4.1031,
\qquad
CR = 0.0389
\end{equation*}

\begin{equation*}
\mathbf{w}^{EM} =
\begin{pmatrix}
0.551666\\
0.260528\\
\color{red} 0.061150\color{black} \\
0.126656
\end{pmatrix}\end{equation*}
\begin{equation*}
\left[ \frac{{w}^{EM}_i}{{w}^{EM}_j} \right] =
\begin{pmatrix}
$\,\,$ 1 $\,\,$ & $\,\,$2.1175$\,\,$ & $\,\,$\color{red} 9.0216\color{black} $\,\,$ & $\,\,$4.3556$\,\,$ \\
$\,\,$0.4723$\,\,$ & $\,\,$ 1 $\,\,$ & $\,\,$\color{red} 4.2605\color{black} $\,\,$ & $\,\,$2.0570  $\,\,$ \\
$\,\,$\color{red} 0.1108\color{black} $\,\,$ & $\,\,$\color{red} 0.2347\color{black} $\,\,$ & $\,\,$ 1 $\,\,$ & $\,\,$\color{red} 0.4828\color{black}  $\,\,$ \\
$\,\,$0.2296$\,\,$ & $\,\,$0.4862$\,\,$ & $\,\,$\color{red} 2.0712\color{black} $\,\,$ & $\,\,$ 1  $\,\,$ \\
\end{pmatrix},
\end{equation*}

\begin{equation*}
\mathbf{w}^{\prime} =
\begin{pmatrix}
0.551586\\
0.260490\\
0.061287\\
0.126637
\end{pmatrix} =
0.999853\cdot
\begin{pmatrix}
0.551666\\
0.260528\\
\color{gr} 0.061296\color{black} \\
0.126656
\end{pmatrix},
\end{equation*}
\begin{equation*}
\left[ \frac{{w}^{\prime}_i}{{w}^{\prime}_j} \right] =
\begin{pmatrix}
$\,\,$ 1 $\,\,$ & $\,\,$2.1175$\,\,$ & $\,\,$\color{gr} \color{blue} 9\color{black} $\,\,$ & $\,\,$4.3556$\,\,$ \\
$\,\,$0.4723$\,\,$ & $\,\,$ 1 $\,\,$ & $\,\,$\color{gr} 4.2503\color{black} $\,\,$ & $\,\,$2.0570  $\,\,$ \\
$\,\,$\color{gr} \color{blue}  1/9\color{black} $\,\,$ & $\,\,$\color{gr} 0.2353\color{black} $\,\,$ & $\,\,$ 1 $\,\,$ & $\,\,$\color{gr} 0.4840\color{black}  $\,\,$ \\
$\,\,$0.2296$\,\,$ & $\,\,$0.4862$\,\,$ & $\,\,$\color{gr} 2.0663\color{black} $\,\,$ & $\,\,$ 1  $\,\,$ \\
\end{pmatrix},
\end{equation*}
\end{example}
\newpage
\begin{example}
\begin{equation*}
\mathbf{A} =
\begin{pmatrix}
$\,\,$ 1 $\,\,$ & $\,\,$3$\,\,$ & $\,\,$9$\,\,$ & $\,\,$6 $\,\,$ \\
$\,\,$ 1/3$\,\,$ & $\,\,$ 1 $\,\,$ & $\,\,$9$\,\,$ & $\,\,$3 $\,\,$ \\
$\,\,$ 1/9$\,\,$ & $\,\,$ 1/9$\,\,$ & $\,\,$ 1 $\,\,$ & $\,\,$ 1/2 $\,\,$ \\
$\,\,$ 1/6$\,\,$ & $\,\,$ 1/3$\,\,$ & $\,\,$2$\,\,$ & $\,\,$ 1  $\,\,$ \\
\end{pmatrix},
\qquad
\lambda_{\max} =
4.1031,
\qquad
CR = 0.0389
\end{equation*}

\begin{equation*}
\mathbf{w}^{EM} =
\begin{pmatrix}
0.581040\\
0.282473\\
0.045568\\
\color{red} 0.090919\color{black}
\end{pmatrix}\end{equation*}
\begin{equation*}
\left[ \frac{{w}^{EM}_i}{{w}^{EM}_j} \right] =
\begin{pmatrix}
$\,\,$ 1 $\,\,$ & $\,\,$2.0570$\,\,$ & $\,\,$12.7509$\,\,$ & $\,\,$\color{red} 6.3907\color{black} $\,\,$ \\
$\,\,$0.4862$\,\,$ & $\,\,$ 1 $\,\,$ & $\,\,$6.1989$\,\,$ & $\,\,$\color{red} 3.1069\color{black}   $\,\,$ \\
$\,\,$0.0784$\,\,$ & $\,\,$0.1613$\,\,$ & $\,\,$ 1 $\,\,$ & $\,\,$\color{red} 0.5012\color{black}  $\,\,$ \\
$\,\,$\color{red} 0.1565\color{black} $\,\,$ & $\,\,$\color{red} 0.3219\color{black} $\,\,$ & $\,\,$\color{red} 1.9952\color{black} $\,\,$ & $\,\,$ 1  $\,\,$ \\
\end{pmatrix},
\end{equation*}

\begin{equation*}
\mathbf{w}^{\prime} =
\begin{pmatrix}
0.580913\\
0.282411\\
0.045559\\
0.091117
\end{pmatrix} =
0.999782\cdot
\begin{pmatrix}
0.581040\\
0.282473\\
0.045568\\
\color{gr} 0.091137\color{black}
\end{pmatrix},
\end{equation*}
\begin{equation*}
\left[ \frac{{w}^{\prime}_i}{{w}^{\prime}_j} \right] =
\begin{pmatrix}
$\,\,$ 1 $\,\,$ & $\,\,$2.0570$\,\,$ & $\,\,$12.7509$\,\,$ & $\,\,$\color{gr} 6.3755\color{black} $\,\,$ \\
$\,\,$0.4862$\,\,$ & $\,\,$ 1 $\,\,$ & $\,\,$6.1989$\,\,$ & $\,\,$\color{gr} 3.0994\color{black}   $\,\,$ \\
$\,\,$0.0784$\,\,$ & $\,\,$0.1613$\,\,$ & $\,\,$ 1 $\,\,$ & $\,\,$\color{gr} \color{blue}  1/2\color{black}  $\,\,$ \\
$\,\,$\color{gr} 0.1569\color{black} $\,\,$ & $\,\,$\color{gr} 0.3226\color{black} $\,\,$ & $\,\,$\color{gr} \color{blue} 2\color{black} $\,\,$ & $\,\,$ 1  $\,\,$ \\
\end{pmatrix},
\end{equation*}
\end{example}
\newpage
\begin{example}
\begin{equation*}
\mathbf{A} =
\begin{pmatrix}
$\,\,$ 1 $\,\,$ & $\,\,$4$\,\,$ & $\,\,$5$\,\,$ & $\,\,$1 $\,\,$ \\
$\,\,$ 1/4$\,\,$ & $\,\,$ 1 $\,\,$ & $\,\,$2$\,\,$ & $\,\,$1 $\,\,$ \\
$\,\,$ 1/5$\,\,$ & $\,\,$ 1/2$\,\,$ & $\,\,$ 1 $\,\,$ & $\,\,$ 1/3 $\,\,$ \\
$\,\,$ 1 $\,\,$ & $\,\,$ 1 $\,\,$ & $\,\,$3$\,\,$ & $\,\,$ 1  $\,\,$ \\
\end{pmatrix},
\qquad
\lambda_{\max} =
4.1655,
\qquad
CR = 0.0624
\end{equation*}

\begin{equation*}
\mathbf{w}^{EM} =
\begin{pmatrix}
0.452399\\
0.179395\\
\color{red} 0.086574\color{black} \\
0.281632
\end{pmatrix}\end{equation*}
\begin{equation*}
\left[ \frac{{w}^{EM}_i}{{w}^{EM}_j} \right] =
\begin{pmatrix}
$\,\,$ 1 $\,\,$ & $\,\,$2.5218$\,\,$ & $\,\,$\color{red} 5.2255\color{black} $\,\,$ & $\,\,$1.6063$\,\,$ \\
$\,\,$0.3965$\,\,$ & $\,\,$ 1 $\,\,$ & $\,\,$\color{red} 2.0721\color{black} $\,\,$ & $\,\,$0.6370  $\,\,$ \\
$\,\,$\color{red} 0.1914\color{black} $\,\,$ & $\,\,$\color{red} 0.4826\color{black} $\,\,$ & $\,\,$ 1 $\,\,$ & $\,\,$\color{red} 0.3074\color{black}  $\,\,$ \\
$\,\,$0.6225$\,\,$ & $\,\,$1.5699$\,\,$ & $\,\,$\color{red} 3.2531\color{black} $\,\,$ & $\,\,$ 1  $\,\,$ \\
\end{pmatrix},
\end{equation*}

\begin{equation*}
\mathbf{w}^{\prime} =
\begin{pmatrix}
0.450990\\
0.178836\\
0.089418\\
0.280755
\end{pmatrix} =
0.996887\cdot
\begin{pmatrix}
0.452399\\
0.179395\\
\color{gr} 0.089697\color{black} \\
0.281632
\end{pmatrix},
\end{equation*}
\begin{equation*}
\left[ \frac{{w}^{\prime}_i}{{w}^{\prime}_j} \right] =
\begin{pmatrix}
$\,\,$ 1 $\,\,$ & $\,\,$2.5218$\,\,$ & $\,\,$\color{gr} 5.0436\color{black} $\,\,$ & $\,\,$1.6063$\,\,$ \\
$\,\,$0.3965$\,\,$ & $\,\,$ 1 $\,\,$ & $\,\,$\color{gr} \color{blue} 2\color{black} $\,\,$ & $\,\,$0.6370  $\,\,$ \\
$\,\,$\color{gr} 0.1983\color{black} $\,\,$ & $\,\,$\color{gr} \color{blue}  1/2\color{black} $\,\,$ & $\,\,$ 1 $\,\,$ & $\,\,$\color{gr} 0.3185\color{black}  $\,\,$ \\
$\,\,$0.6225$\,\,$ & $\,\,$1.5699$\,\,$ & $\,\,$\color{gr} 3.1398\color{black} $\,\,$ & $\,\,$ 1  $\,\,$ \\
\end{pmatrix},
\end{equation*}
\end{example}
\newpage
\begin{example}
\begin{equation*}
\mathbf{A} =
\begin{pmatrix}
$\,\,$ 1 $\,\,$ & $\,\,$4$\,\,$ & $\,\,$6$\,\,$ & $\,\,$2 $\,\,$ \\
$\,\,$ 1/4$\,\,$ & $\,\,$ 1 $\,\,$ & $\,\,$1$\,\,$ & $\,\,$1 $\,\,$ \\
$\,\,$ 1/6$\,\,$ & $\,\,$ 1 $\,\,$ & $\,\,$ 1 $\,\,$ & $\,\,$ 1/5 $\,\,$ \\
$\,\,$ 1/2$\,\,$ & $\,\,$ 1 $\,\,$ & $\,\,$5$\,\,$ & $\,\,$ 1  $\,\,$ \\
\end{pmatrix},
\qquad
\lambda_{\max} =
4.2277,
\qquad
CR = 0.0859
\end{equation*}

\begin{equation*}
\mathbf{w}^{EM} =
\begin{pmatrix}
\color{red} 0.505834\color{black} \\
0.146799\\
0.087690\\
0.259677
\end{pmatrix}\end{equation*}
\begin{equation*}
\left[ \frac{{w}^{EM}_i}{{w}^{EM}_j} \right] =
\begin{pmatrix}
$\,\,$ 1 $\,\,$ & $\,\,$\color{red} 3.4458\color{black} $\,\,$ & $\,\,$\color{red} 5.7684\color{black} $\,\,$ & $\,\,$\color{red} 1.9479\color{black} $\,\,$ \\
$\,\,$\color{red} 0.2902\color{black} $\,\,$ & $\,\,$ 1 $\,\,$ & $\,\,$1.6741$\,\,$ & $\,\,$0.5653  $\,\,$ \\
$\,\,$\color{red} 0.1734\color{black} $\,\,$ & $\,\,$0.5973$\,\,$ & $\,\,$ 1 $\,\,$ & $\,\,$0.3377 $\,\,$ \\
$\,\,$\color{red} 0.5134\color{black} $\,\,$ & $\,\,$1.7689$\,\,$ & $\,\,$2.9613$\,\,$ & $\,\,$ 1  $\,\,$ \\
\end{pmatrix},
\end{equation*}

\begin{equation*}
\mathbf{w}^{\prime} =
\begin{pmatrix}
0.512426\\
0.144840\\
0.086520\\
0.256213
\end{pmatrix} =
0.986659\cdot
\begin{pmatrix}
\color{gr} 0.519355\color{black} \\
0.146799\\
0.087690\\
0.259677
\end{pmatrix},
\end{equation*}
\begin{equation*}
\left[ \frac{{w}^{\prime}_i}{{w}^{\prime}_j} \right] =
\begin{pmatrix}
$\,\,$ 1 $\,\,$ & $\,\,$\color{gr} 3.5379\color{black} $\,\,$ & $\,\,$\color{gr} 5.9226\color{black} $\,\,$ & $\,\,$\color{gr} \color{blue} 2\color{black} $\,\,$ \\
$\,\,$\color{gr} 0.2827\color{black} $\,\,$ & $\,\,$ 1 $\,\,$ & $\,\,$1.6741$\,\,$ & $\,\,$0.5653  $\,\,$ \\
$\,\,$\color{gr} 0.1688\color{black} $\,\,$ & $\,\,$0.5973$\,\,$ & $\,\,$ 1 $\,\,$ & $\,\,$0.3377 $\,\,$ \\
$\,\,$\color{gr} \color{blue}  1/2\color{black} $\,\,$ & $\,\,$1.7689$\,\,$ & $\,\,$2.9613$\,\,$ & $\,\,$ 1  $\,\,$ \\
\end{pmatrix},
\end{equation*}
\end{example}
\newpage
\begin{example}
\begin{equation*}
\mathbf{A} =
\begin{pmatrix}
$\,\,$ 1 $\,\,$ & $\,\,$4$\,\,$ & $\,\,$6$\,\,$ & $\,\,$3 $\,\,$ \\
$\,\,$ 1/4$\,\,$ & $\,\,$ 1 $\,\,$ & $\,\,$1$\,\,$ & $\,\,$1 $\,\,$ \\
$\,\,$ 1/6$\,\,$ & $\,\,$ 1 $\,\,$ & $\,\,$ 1 $\,\,$ & $\,\,$ 1/3 $\,\,$ \\
$\,\,$ 1/3$\,\,$ & $\,\,$ 1 $\,\,$ & $\,\,$3$\,\,$ & $\,\,$ 1  $\,\,$ \\
\end{pmatrix},
\qquad
\lambda_{\max} =
4.1031,
\qquad
CR = 0.0389
\end{equation*}

\begin{equation*}
\mathbf{w}^{EM} =
\begin{pmatrix}
\color{red} 0.562353\color{black} \\
0.140925\\
0.097064\\
0.199658
\end{pmatrix}\end{equation*}
\begin{equation*}
\left[ \frac{{w}^{EM}_i}{{w}^{EM}_j} \right] =
\begin{pmatrix}
$\,\,$ 1 $\,\,$ & $\,\,$\color{red} 3.9904\color{black} $\,\,$ & $\,\,$\color{red} 5.7936\color{black} $\,\,$ & $\,\,$\color{red} 2.8166\color{black} $\,\,$ \\
$\,\,$\color{red} 0.2506\color{black} $\,\,$ & $\,\,$ 1 $\,\,$ & $\,\,$1.4519$\,\,$ & $\,\,$0.7058  $\,\,$ \\
$\,\,$\color{red} 0.1726\color{black} $\,\,$ & $\,\,$0.6888$\,\,$ & $\,\,$ 1 $\,\,$ & $\,\,$0.4862 $\,\,$ \\
$\,\,$\color{red} 0.3550\color{black} $\,\,$ & $\,\,$1.4168$\,\,$ & $\,\,$2.0570$\,\,$ & $\,\,$ 1  $\,\,$ \\
\end{pmatrix},
\end{equation*}

\begin{equation*}
\mathbf{w}^{\prime} =
\begin{pmatrix}
0.562942\\
0.140735\\
0.096933\\
0.199390
\end{pmatrix} =
0.998654\cdot
\begin{pmatrix}
\color{gr} 0.563700\color{black} \\
0.140925\\
0.097064\\
0.199658
\end{pmatrix},
\end{equation*}
\begin{equation*}
\left[ \frac{{w}^{\prime}_i}{{w}^{\prime}_j} \right] =
\begin{pmatrix}
$\,\,$ 1 $\,\,$ & $\,\,$\color{gr} \color{blue} 4\color{black} $\,\,$ & $\,\,$\color{gr} 5.8075\color{black} $\,\,$ & $\,\,$\color{gr} 2.8233\color{black} $\,\,$ \\
$\,\,$\color{gr} \color{blue}  1/4\color{black} $\,\,$ & $\,\,$ 1 $\,\,$ & $\,\,$1.4519$\,\,$ & $\,\,$0.7058  $\,\,$ \\
$\,\,$\color{gr} 0.1722\color{black} $\,\,$ & $\,\,$0.6888$\,\,$ & $\,\,$ 1 $\,\,$ & $\,\,$0.4862 $\,\,$ \\
$\,\,$\color{gr} 0.3542\color{black} $\,\,$ & $\,\,$1.4168$\,\,$ & $\,\,$2.0570$\,\,$ & $\,\,$ 1  $\,\,$ \\
\end{pmatrix},
\end{equation*}
\end{example}
\newpage
\begin{example}
\begin{equation*}
\mathbf{A} =
\begin{pmatrix}
$\,\,$ 1 $\,\,$ & $\,\,$4$\,\,$ & $\,\,$6$\,\,$ & $\,\,$5 $\,\,$ \\
$\,\,$ 1/4$\,\,$ & $\,\,$ 1 $\,\,$ & $\,\,$1$\,\,$ & $\,\,$2 $\,\,$ \\
$\,\,$ 1/6$\,\,$ & $\,\,$ 1 $\,\,$ & $\,\,$ 1 $\,\,$ & $\,\,$ 1/2 $\,\,$ \\
$\,\,$ 1/5$\,\,$ & $\,\,$ 1/2$\,\,$ & $\,\,$2$\,\,$ & $\,\,$ 1  $\,\,$ \\
\end{pmatrix},
\qquad
\lambda_{\max} =
4.1655,
\qquad
CR = 0.0624
\end{equation*}

\begin{equation*}
\mathbf{w}^{EM} =
\begin{pmatrix}
\color{red} 0.605148\color{black} \\
0.162132\\
0.103491\\
0.129229
\end{pmatrix}\end{equation*}
\begin{equation*}
\left[ \frac{{w}^{EM}_i}{{w}^{EM}_j} \right] =
\begin{pmatrix}
$\,\,$ 1 $\,\,$ & $\,\,$\color{red} 3.7324\color{black} $\,\,$ & $\,\,$\color{red} 5.8473\color{black} $\,\,$ & $\,\,$\color{red} 4.6828\color{black} $\,\,$ \\
$\,\,$\color{red} 0.2679\color{black} $\,\,$ & $\,\,$ 1 $\,\,$ & $\,\,$1.5666$\,\,$ & $\,\,$1.2546  $\,\,$ \\
$\,\,$\color{red} 0.1710\color{black} $\,\,$ & $\,\,$0.6383$\,\,$ & $\,\,$ 1 $\,\,$ & $\,\,$0.8008 $\,\,$ \\
$\,\,$\color{red} 0.2135\color{black} $\,\,$ & $\,\,$0.7971$\,\,$ & $\,\,$1.2487$\,\,$ & $\,\,$ 1  $\,\,$ \\
\end{pmatrix},
\end{equation*}

\begin{equation*}
\mathbf{w}^{\prime} =
\begin{pmatrix}
0.611289\\
0.159611\\
0.101882\\
0.127219
\end{pmatrix} =
0.984448\cdot
\begin{pmatrix}
\color{gr} 0.620946\color{black} \\
0.162132\\
0.103491\\
0.129229
\end{pmatrix},
\end{equation*}
\begin{equation*}
\left[ \frac{{w}^{\prime}_i}{{w}^{\prime}_j} \right] =
\begin{pmatrix}
$\,\,$ 1 $\,\,$ & $\,\,$\color{gr} 3.8299\color{black} $\,\,$ & $\,\,$\color{gr} \color{blue} 6\color{black} $\,\,$ & $\,\,$\color{gr} 4.8050\color{black} $\,\,$ \\
$\,\,$\color{gr} 0.2611\color{black} $\,\,$ & $\,\,$ 1 $\,\,$ & $\,\,$1.5666$\,\,$ & $\,\,$1.2546  $\,\,$ \\
$\,\,$\color{gr} \color{blue}  1/6\color{black} $\,\,$ & $\,\,$0.6383$\,\,$ & $\,\,$ 1 $\,\,$ & $\,\,$0.8008 $\,\,$ \\
$\,\,$\color{gr} 0.2081\color{black} $\,\,$ & $\,\,$0.7971$\,\,$ & $\,\,$1.2487$\,\,$ & $\,\,$ 1  $\,\,$ \\
\end{pmatrix},
\end{equation*}
\end{example}
\newpage
\begin{example}
\begin{equation*}
\mathbf{A} =
\begin{pmatrix}
$\,\,$ 1 $\,\,$ & $\,\,$4$\,\,$ & $\,\,$7$\,\,$ & $\,\,$1 $\,\,$ \\
$\,\,$ 1/4$\,\,$ & $\,\,$ 1 $\,\,$ & $\,\,$3$\,\,$ & $\,\,$1 $\,\,$ \\
$\,\,$ 1/7$\,\,$ & $\,\,$ 1/3$\,\,$ & $\,\,$ 1 $\,\,$ & $\,\,$ 1/5 $\,\,$ \\
$\,\,$ 1 $\,\,$ & $\,\,$ 1 $\,\,$ & $\,\,$5$\,\,$ & $\,\,$ 1  $\,\,$ \\
\end{pmatrix},
\qquad
\lambda_{\max} =
4.1667,
\qquad
CR = 0.0629
\end{equation*}

\begin{equation*}
\mathbf{w}^{EM} =
\begin{pmatrix}
0.458661\\
0.185748\\
\color{red} 0.058977\color{black} \\
0.296615
\end{pmatrix}\end{equation*}
\begin{equation*}
\left[ \frac{{w}^{EM}_i}{{w}^{EM}_j} \right] =
\begin{pmatrix}
$\,\,$ 1 $\,\,$ & $\,\,$2.4693$\,\,$ & $\,\,$\color{red} 7.7770\color{black} $\,\,$ & $\,\,$1.5463$\,\,$ \\
$\,\,$0.4050$\,\,$ & $\,\,$ 1 $\,\,$ & $\,\,$\color{red} 3.1495\color{black} $\,\,$ & $\,\,$0.6262  $\,\,$ \\
$\,\,$\color{red} 0.1286\color{black} $\,\,$ & $\,\,$\color{red} 0.3175\color{black} $\,\,$ & $\,\,$ 1 $\,\,$ & $\,\,$\color{red} 0.1988\color{black}  $\,\,$ \\
$\,\,$0.6467$\,\,$ & $\,\,$1.5969$\,\,$ & $\,\,$\color{red} 5.0294\color{black} $\,\,$ & $\,\,$ 1  $\,\,$ \\
\end{pmatrix},
\end{equation*}

\begin{equation*}
\mathbf{w}^{\prime} =
\begin{pmatrix}
0.458502\\
0.185684\\
0.059302\\
0.296512
\end{pmatrix} =
0.999654\cdot
\begin{pmatrix}
0.458661\\
0.185748\\
\color{gr} 0.059323\color{black} \\
0.296615
\end{pmatrix},
\end{equation*}
\begin{equation*}
\left[ \frac{{w}^{\prime}_i}{{w}^{\prime}_j} \right] =
\begin{pmatrix}
$\,\,$ 1 $\,\,$ & $\,\,$2.4693$\,\,$ & $\,\,$\color{gr} 7.7316\color{black} $\,\,$ & $\,\,$1.5463$\,\,$ \\
$\,\,$0.4050$\,\,$ & $\,\,$ 1 $\,\,$ & $\,\,$\color{gr} 3.1311\color{black} $\,\,$ & $\,\,$0.6262  $\,\,$ \\
$\,\,$\color{gr} 0.1293\color{black} $\,\,$ & $\,\,$\color{gr} 0.3194\color{black} $\,\,$ & $\,\,$ 1 $\,\,$ & $\,\,$\color{gr} \color{blue}  1/5\color{black}  $\,\,$ \\
$\,\,$0.6467$\,\,$ & $\,\,$1.5969$\,\,$ & $\,\,$\color{gr} \color{blue} 5\color{black} $\,\,$ & $\,\,$ 1  $\,\,$ \\
\end{pmatrix},
\end{equation*}
\end{example}
\newpage
\begin{example}
\begin{equation*}
\mathbf{A} =
\begin{pmatrix}
$\,\,$ 1 $\,\,$ & $\,\,$4$\,\,$ & $\,\,$7$\,\,$ & $\,\,$5 $\,\,$ \\
$\,\,$ 1/4$\,\,$ & $\,\,$ 1 $\,\,$ & $\,\,$1$\,\,$ & $\,\,$2 $\,\,$ \\
$\,\,$ 1/7$\,\,$ & $\,\,$ 1 $\,\,$ & $\,\,$ 1 $\,\,$ & $\,\,$ 1/2 $\,\,$ \\
$\,\,$ 1/5$\,\,$ & $\,\,$ 1/2$\,\,$ & $\,\,$2$\,\,$ & $\,\,$ 1  $\,\,$ \\
\end{pmatrix},
\qquad
\lambda_{\max} =
4.1665,
\qquad
CR = 0.0628
\end{equation*}

\begin{equation*}
\mathbf{w}^{EM} =
\begin{pmatrix}
\color{red} 0.616762\color{black} \\
0.159230\\
0.098007\\
0.126001
\end{pmatrix}\end{equation*}
\begin{equation*}
\left[ \frac{{w}^{EM}_i}{{w}^{EM}_j} \right] =
\begin{pmatrix}
$\,\,$ 1 $\,\,$ & $\,\,$\color{red} 3.8734\color{black} $\,\,$ & $\,\,$\color{red} 6.2930\color{black} $\,\,$ & $\,\,$\color{red} 4.8949\color{black} $\,\,$ \\
$\,\,$\color{red} 0.2582\color{black} $\,\,$ & $\,\,$ 1 $\,\,$ & $\,\,$1.6247$\,\,$ & $\,\,$1.2637  $\,\,$ \\
$\,\,$\color{red} 0.1589\color{black} $\,\,$ & $\,\,$0.6155$\,\,$ & $\,\,$ 1 $\,\,$ & $\,\,$0.7778 $\,\,$ \\
$\,\,$\color{red} 0.2043\color{black} $\,\,$ & $\,\,$0.7913$\,\,$ & $\,\,$1.2856$\,\,$ & $\,\,$ 1  $\,\,$ \\
\end{pmatrix},
\end{equation*}

\begin{equation*}
\mathbf{w}^{\prime} =
\begin{pmatrix}
0.621771\\
0.157149\\
0.096726\\
0.124354
\end{pmatrix} =
0.986930\cdot
\begin{pmatrix}
\color{gr} 0.630005\color{black} \\
0.159230\\
0.098007\\
0.126001
\end{pmatrix},
\end{equation*}
\begin{equation*}
\left[ \frac{{w}^{\prime}_i}{{w}^{\prime}_j} \right] =
\begin{pmatrix}
$\,\,$ 1 $\,\,$ & $\,\,$\color{gr} 3.9566\color{black} $\,\,$ & $\,\,$\color{gr} 6.4282\color{black} $\,\,$ & $\,\,$\color{gr} \color{blue} 5\color{black} $\,\,$ \\
$\,\,$\color{gr} 0.2527\color{black} $\,\,$ & $\,\,$ 1 $\,\,$ & $\,\,$1.6247$\,\,$ & $\,\,$1.2637  $\,\,$ \\
$\,\,$\color{gr} 0.1556\color{black} $\,\,$ & $\,\,$0.6155$\,\,$ & $\,\,$ 1 $\,\,$ & $\,\,$0.7778 $\,\,$ \\
$\,\,$\color{gr} \color{blue}  1/5\color{black} $\,\,$ & $\,\,$0.7913$\,\,$ & $\,\,$1.2856$\,\,$ & $\,\,$ 1  $\,\,$ \\
\end{pmatrix},
\end{equation*}
\end{example}
\newpage
\begin{example}
\begin{equation*}
\mathbf{A} =
\begin{pmatrix}
$\,\,$ 1 $\,\,$ & $\,\,$4$\,\,$ & $\,\,$7$\,\,$ & $\,\,$7 $\,\,$ \\
$\,\,$ 1/4$\,\,$ & $\,\,$ 1 $\,\,$ & $\,\,$9$\,\,$ & $\,\,$3 $\,\,$ \\
$\,\,$ 1/7$\,\,$ & $\,\,$ 1/9$\,\,$ & $\,\,$ 1 $\,\,$ & $\,\,$ 1/2 $\,\,$ \\
$\,\,$ 1/7$\,\,$ & $\,\,$ 1/3$\,\,$ & $\,\,$2$\,\,$ & $\,\,$ 1  $\,\,$ \\
\end{pmatrix},
\qquad
\lambda_{\max} =
4.2359,
\qquad
CR = 0.0890
\end{equation*}

\begin{equation*}
\mathbf{w}^{EM} =
\begin{pmatrix}
0.607679\\
0.259943\\
0.048685\\
\color{red} 0.083694\color{black}
\end{pmatrix}\end{equation*}
\begin{equation*}
\left[ \frac{{w}^{EM}_i}{{w}^{EM}_j} \right] =
\begin{pmatrix}
$\,\,$ 1 $\,\,$ & $\,\,$2.3377$\,\,$ & $\,\,$12.4820$\,\,$ & $\,\,$\color{red} 7.2608\color{black} $\,\,$ \\
$\,\,$0.4278$\,\,$ & $\,\,$ 1 $\,\,$ & $\,\,$5.3393$\,\,$ & $\,\,$\color{red} 3.1059\color{black}   $\,\,$ \\
$\,\,$0.0801$\,\,$ & $\,\,$0.1873$\,\,$ & $\,\,$ 1 $\,\,$ & $\,\,$\color{red} 0.5817\color{black}  $\,\,$ \\
$\,\,$\color{red} 0.1377\color{black} $\,\,$ & $\,\,$\color{red} 0.3220\color{black} $\,\,$ & $\,\,$\color{red} 1.7191\color{black} $\,\,$ & $\,\,$ 1  $\,\,$ \\
\end{pmatrix},
\end{equation*}

\begin{equation*}
\mathbf{w}^{\prime} =
\begin{pmatrix}
0.605889\\
0.259177\\
0.048541\\
0.086392
\end{pmatrix} =
0.997055\cdot
\begin{pmatrix}
0.607679\\
0.259943\\
0.048685\\
\color{gr} 0.086648\color{black}
\end{pmatrix},
\end{equation*}
\begin{equation*}
\left[ \frac{{w}^{\prime}_i}{{w}^{\prime}_j} \right] =
\begin{pmatrix}
$\,\,$ 1 $\,\,$ & $\,\,$2.3377$\,\,$ & $\,\,$12.4820$\,\,$ & $\,\,$\color{gr} 7.0132\color{black} $\,\,$ \\
$\,\,$0.4278$\,\,$ & $\,\,$ 1 $\,\,$ & $\,\,$5.3393$\,\,$ & $\,\,$\color{gr} \color{blue} 3\color{black}   $\,\,$ \\
$\,\,$0.0801$\,\,$ & $\,\,$0.1873$\,\,$ & $\,\,$ 1 $\,\,$ & $\,\,$\color{gr} 0.5619\color{black}  $\,\,$ \\
$\,\,$\color{gr} 0.1426\color{black} $\,\,$ & $\,\,$\color{gr} \color{blue}  1/3\color{black} $\,\,$ & $\,\,$\color{gr} 1.7798\color{black} $\,\,$ & $\,\,$ 1  $\,\,$ \\
\end{pmatrix},
\end{equation*}
\end{example}
\newpage
\begin{example}
\begin{equation*}
\mathbf{A} =
\begin{pmatrix}
$\,\,$ 1 $\,\,$ & $\,\,$4$\,\,$ & $\,\,$8$\,\,$ & $\,\,$1 $\,\,$ \\
$\,\,$ 1/4$\,\,$ & $\,\,$ 1 $\,\,$ & $\,\,$3$\,\,$ & $\,\,$1 $\,\,$ \\
$\,\,$ 1/8$\,\,$ & $\,\,$ 1/3$\,\,$ & $\,\,$ 1 $\,\,$ & $\,\,$ 1/5 $\,\,$ \\
$\,\,$ 1 $\,\,$ & $\,\,$ 1 $\,\,$ & $\,\,$5$\,\,$ & $\,\,$ 1  $\,\,$ \\
\end{pmatrix},
\qquad
\lambda_{\max} =
4.1655,
\qquad
CR = 0.0624
\end{equation*}

\begin{equation*}
\mathbf{w}^{EM} =
\begin{pmatrix}
0.466511\\
0.183094\\
\color{red} 0.056284\color{black} \\
0.294112
\end{pmatrix}\end{equation*}
\begin{equation*}
\left[ \frac{{w}^{EM}_i}{{w}^{EM}_j} \right] =
\begin{pmatrix}
$\,\,$ 1 $\,\,$ & $\,\,$2.5479$\,\,$ & $\,\,$\color{red} 8.2886\color{black} $\,\,$ & $\,\,$1.5862$\,\,$ \\
$\,\,$0.3925$\,\,$ & $\,\,$ 1 $\,\,$ & $\,\,$\color{red} 3.2531\color{black} $\,\,$ & $\,\,$0.6225  $\,\,$ \\
$\,\,$\color{red} 0.1206\color{black} $\,\,$ & $\,\,$\color{red} 0.3074\color{black} $\,\,$ & $\,\,$ 1 $\,\,$ & $\,\,$\color{red} 0.1914\color{black}  $\,\,$ \\
$\,\,$0.6305$\,\,$ & $\,\,$1.6063$\,\,$ & $\,\,$\color{red} 5.2255\color{black} $\,\,$ & $\,\,$ 1  $\,\,$ \\
\end{pmatrix},
\end{equation*}

\begin{equation*}
\mathbf{w}^{\prime} =
\begin{pmatrix}
0.465565\\
0.182723\\
0.058196\\
0.293516
\end{pmatrix} =
0.997974\cdot
\begin{pmatrix}
0.466511\\
0.183094\\
\color{gr} 0.058314\color{black} \\
0.294112
\end{pmatrix},
\end{equation*}
\begin{equation*}
\left[ \frac{{w}^{\prime}_i}{{w}^{\prime}_j} \right] =
\begin{pmatrix}
$\,\,$ 1 $\,\,$ & $\,\,$2.5479$\,\,$ & $\,\,$\color{gr} \color{blue} 8\color{black} $\,\,$ & $\,\,$1.5862$\,\,$ \\
$\,\,$0.3925$\,\,$ & $\,\,$ 1 $\,\,$ & $\,\,$\color{gr} 3.1398\color{black} $\,\,$ & $\,\,$0.6225  $\,\,$ \\
$\,\,$\color{gr} \color{blue}  1/8\color{black} $\,\,$ & $\,\,$\color{gr} 0.3185\color{black} $\,\,$ & $\,\,$ 1 $\,\,$ & $\,\,$\color{gr} 0.1983\color{black}  $\,\,$ \\
$\,\,$0.6305$\,\,$ & $\,\,$1.6063$\,\,$ & $\,\,$\color{gr} 5.0436\color{black} $\,\,$ & $\,\,$ 1  $\,\,$ \\
\end{pmatrix},
\end{equation*}
\end{example}
\newpage
\begin{example}
\begin{equation*}
\mathbf{A} =
\begin{pmatrix}
$\,\,$ 1 $\,\,$ & $\,\,$4$\,\,$ & $\,\,$8$\,\,$ & $\,\,$7 $\,\,$ \\
$\,\,$ 1/4$\,\,$ & $\,\,$ 1 $\,\,$ & $\,\,$9$\,\,$ & $\,\,$3 $\,\,$ \\
$\,\,$ 1/8$\,\,$ & $\,\,$ 1/9$\,\,$ & $\,\,$ 1 $\,\,$ & $\,\,$ 1/2 $\,\,$ \\
$\,\,$ 1/7$\,\,$ & $\,\,$ 1/3$\,\,$ & $\,\,$2$\,\,$ & $\,\,$ 1  $\,\,$ \\
\end{pmatrix},
\qquad
\lambda_{\max} =
4.1964,
\qquad
CR = 0.0741
\end{equation*}

\begin{equation*}
\mathbf{w}^{EM} =
\begin{pmatrix}
0.615914\\
0.255276\\
0.045925\\
\color{red} 0.082884\color{black}
\end{pmatrix}\end{equation*}
\begin{equation*}
\left[ \frac{{w}^{EM}_i}{{w}^{EM}_j} \right] =
\begin{pmatrix}
$\,\,$ 1 $\,\,$ & $\,\,$2.4127$\,\,$ & $\,\,$13.4112$\,\,$ & $\,\,$\color{red} 7.4310\color{black} $\,\,$ \\
$\,\,$0.4145$\,\,$ & $\,\,$ 1 $\,\,$ & $\,\,$5.5585$\,\,$ & $\,\,$\color{red} 3.0799\color{black}   $\,\,$ \\
$\,\,$0.0746$\,\,$ & $\,\,$0.1799$\,\,$ & $\,\,$ 1 $\,\,$ & $\,\,$\color{red} 0.5541\color{black}  $\,\,$ \\
$\,\,$\color{red} 0.1346\color{black} $\,\,$ & $\,\,$\color{red} 0.3247\color{black} $\,\,$ & $\,\,$\color{red} 1.8048\color{black} $\,\,$ & $\,\,$ 1  $\,\,$ \\
\end{pmatrix},
\end{equation*}

\begin{equation*}
\mathbf{w}^{\prime} =
\begin{pmatrix}
0.614557\\
0.254714\\
0.045824\\
0.084905
\end{pmatrix} =
0.997797\cdot
\begin{pmatrix}
0.615914\\
0.255276\\
0.045925\\
\color{gr} 0.085092\color{black}
\end{pmatrix},
\end{equation*}
\begin{equation*}
\left[ \frac{{w}^{\prime}_i}{{w}^{\prime}_j} \right] =
\begin{pmatrix}
$\,\,$ 1 $\,\,$ & $\,\,$2.4127$\,\,$ & $\,\,$13.4112$\,\,$ & $\,\,$\color{gr} 7.2382\color{black} $\,\,$ \\
$\,\,$0.4145$\,\,$ & $\,\,$ 1 $\,\,$ & $\,\,$5.5585$\,\,$ & $\,\,$\color{gr} \color{blue} 3\color{black}   $\,\,$ \\
$\,\,$0.0746$\,\,$ & $\,\,$0.1799$\,\,$ & $\,\,$ 1 $\,\,$ & $\,\,$\color{gr} 0.5397\color{black}  $\,\,$ \\
$\,\,$\color{gr} 0.1382\color{black} $\,\,$ & $\,\,$\color{gr} \color{blue}  1/3\color{black} $\,\,$ & $\,\,$\color{gr} 1.8528\color{black} $\,\,$ & $\,\,$ 1  $\,\,$ \\
\end{pmatrix},
\end{equation*}
\end{example}
\newpage
\begin{example}
\begin{equation*}
\mathbf{A} =
\begin{pmatrix}
$\,\,$ 1 $\,\,$ & $\,\,$4$\,\,$ & $\,\,$9$\,\,$ & $\,\,$1 $\,\,$ \\
$\,\,$ 1/4$\,\,$ & $\,\,$ 1 $\,\,$ & $\,\,$4$\,\,$ & $\,\,$1 $\,\,$ \\
$\,\,$ 1/9$\,\,$ & $\,\,$ 1/4$\,\,$ & $\,\,$ 1 $\,\,$ & $\,\,$ 1/6 $\,\,$ \\
$\,\,$ 1 $\,\,$ & $\,\,$ 1 $\,\,$ & $\,\,$6$\,\,$ & $\,\,$ 1  $\,\,$ \\
\end{pmatrix},
\qquad
\lambda_{\max} =
4.1664,
\qquad
CR = 0.0627
\end{equation*}

\begin{equation*}
\mathbf{w}^{EM} =
\begin{pmatrix}
0.466859\\
0.189794\\
\color{red} 0.046968\color{black} \\
0.296380
\end{pmatrix}\end{equation*}
\begin{equation*}
\left[ \frac{{w}^{EM}_i}{{w}^{EM}_j} \right] =
\begin{pmatrix}
$\,\,$ 1 $\,\,$ & $\,\,$2.4598$\,\,$ & $\,\,$\color{red} 9.9400\color{black} $\,\,$ & $\,\,$1.5752$\,\,$ \\
$\,\,$0.4065$\,\,$ & $\,\,$ 1 $\,\,$ & $\,\,$\color{red} 4.0410\color{black} $\,\,$ & $\,\,$0.6404  $\,\,$ \\
$\,\,$\color{red} 0.1006\color{black} $\,\,$ & $\,\,$\color{red} 0.2475\color{black} $\,\,$ & $\,\,$ 1 $\,\,$ & $\,\,$\color{red} 0.1585\color{black}  $\,\,$ \\
$\,\,$0.6348$\,\,$ & $\,\,$1.5616$\,\,$ & $\,\,$\color{red} 6.3103\color{black} $\,\,$ & $\,\,$ 1  $\,\,$ \\
\end{pmatrix},
\end{equation*}

\begin{equation*}
\mathbf{w}^{\prime} =
\begin{pmatrix}
0.466634\\
0.189703\\
0.047426\\
0.296237
\end{pmatrix} =
0.999519\cdot
\begin{pmatrix}
0.466859\\
0.189794\\
\color{gr} 0.047448\color{black} \\
0.296380
\end{pmatrix},
\end{equation*}
\begin{equation*}
\left[ \frac{{w}^{\prime}_i}{{w}^{\prime}_j} \right] =
\begin{pmatrix}
$\,\,$ 1 $\,\,$ & $\,\,$2.4598$\,\,$ & $\,\,$\color{gr} 9.8393\color{black} $\,\,$ & $\,\,$1.5752$\,\,$ \\
$\,\,$0.4065$\,\,$ & $\,\,$ 1 $\,\,$ & $\,\,$\color{gr} \color{blue} 4\color{black} $\,\,$ & $\,\,$0.6404  $\,\,$ \\
$\,\,$\color{gr} 0.1016\color{black} $\,\,$ & $\,\,$\color{gr} \color{blue}  1/4\color{black} $\,\,$ & $\,\,$ 1 $\,\,$ & $\,\,$\color{gr} 0.1601\color{black}  $\,\,$ \\
$\,\,$0.6348$\,\,$ & $\,\,$1.5616$\,\,$ & $\,\,$\color{gr} 6.2463\color{black} $\,\,$ & $\,\,$ 1  $\,\,$ \\
\end{pmatrix},
\end{equation*}
\end{example}
\newpage
\begin{example}
\begin{equation*}
\mathbf{A} =
\begin{pmatrix}
$\,\,$ 1 $\,\,$ & $\,\,$4$\,\,$ & $\,\,$9$\,\,$ & $\,\,$3 $\,\,$ \\
$\,\,$ 1/4$\,\,$ & $\,\,$ 1 $\,\,$ & $\,\,$4$\,\,$ & $\,\,$3 $\,\,$ \\
$\,\,$ 1/9$\,\,$ & $\,\,$ 1/4$\,\,$ & $\,\,$ 1 $\,\,$ & $\,\,$ 1/2 $\,\,$ \\
$\,\,$ 1/3$\,\,$ & $\,\,$ 1/3$\,\,$ & $\,\,$2$\,\,$ & $\,\,$ 1  $\,\,$ \\
\end{pmatrix},
\qquad
\lambda_{\max} =
4.1664,
\qquad
CR = 0.0627
\end{equation*}

\begin{equation*}
\mathbf{w}^{EM} =
\begin{pmatrix}
0.581818\\
0.236529\\
\color{red} 0.058533\color{black} \\
0.123120
\end{pmatrix}\end{equation*}
\begin{equation*}
\left[ \frac{{w}^{EM}_i}{{w}^{EM}_j} \right] =
\begin{pmatrix}
$\,\,$ 1 $\,\,$ & $\,\,$2.4598$\,\,$ & $\,\,$\color{red} 9.9400\color{black} $\,\,$ & $\,\,$4.7256$\,\,$ \\
$\,\,$0.4065$\,\,$ & $\,\,$ 1 $\,\,$ & $\,\,$\color{red} 4.0410\color{black} $\,\,$ & $\,\,$1.9211  $\,\,$ \\
$\,\,$\color{red} 0.1006\color{black} $\,\,$ & $\,\,$\color{red} 0.2475\color{black} $\,\,$ & $\,\,$ 1 $\,\,$ & $\,\,$\color{red} 0.4754\color{black}  $\,\,$ \\
$\,\,$0.2116$\,\,$ & $\,\,$0.5205$\,\,$ & $\,\,$\color{red} 2.1034\color{black} $\,\,$ & $\,\,$ 1  $\,\,$ \\
\end{pmatrix},
\end{equation*}

\begin{equation*}
\mathbf{w}^{\prime} =
\begin{pmatrix}
0.581470\\
0.236387\\
0.059097\\
0.123046
\end{pmatrix} =
0.999401\cdot
\begin{pmatrix}
0.581818\\
0.236529\\
\color{gr} 0.059132\color{black} \\
0.123120
\end{pmatrix},
\end{equation*}
\begin{equation*}
\left[ \frac{{w}^{\prime}_i}{{w}^{\prime}_j} \right] =
\begin{pmatrix}
$\,\,$ 1 $\,\,$ & $\,\,$2.4598$\,\,$ & $\,\,$\color{gr} 9.8393\color{black} $\,\,$ & $\,\,$4.7256$\,\,$ \\
$\,\,$0.4065$\,\,$ & $\,\,$ 1 $\,\,$ & $\,\,$\color{gr} \color{blue} 4\color{black} $\,\,$ & $\,\,$1.9211  $\,\,$ \\
$\,\,$\color{gr} 0.1016\color{black} $\,\,$ & $\,\,$\color{gr} \color{blue}  1/4\color{black} $\,\,$ & $\,\,$ 1 $\,\,$ & $\,\,$\color{gr} 0.4803\color{black}  $\,\,$ \\
$\,\,$0.2116$\,\,$ & $\,\,$0.5205$\,\,$ & $\,\,$\color{gr} 2.0821\color{black} $\,\,$ & $\,\,$ 1  $\,\,$ \\
\end{pmatrix},
\end{equation*}
\end{example}
\newpage
\begin{example}
\begin{equation*}
\mathbf{A} =
\begin{pmatrix}
$\,\,$ 1 $\,\,$ & $\,\,$4$\,\,$ & $\,\,$9$\,\,$ & $\,\,$7 $\,\,$ \\
$\,\,$ 1/4$\,\,$ & $\,\,$ 1 $\,\,$ & $\,\,$9$\,\,$ & $\,\,$3 $\,\,$ \\
$\,\,$ 1/9$\,\,$ & $\,\,$ 1/9$\,\,$ & $\,\,$ 1 $\,\,$ & $\,\,$ 1/2 $\,\,$ \\
$\,\,$ 1/7$\,\,$ & $\,\,$ 1/3$\,\,$ & $\,\,$2$\,\,$ & $\,\,$ 1  $\,\,$ \\
\end{pmatrix},
\qquad
\lambda_{\max} =
4.1658,
\qquad
CR = 0.0625
\end{equation*}

\begin{equation*}
\mathbf{w}^{EM} =
\begin{pmatrix}
0.623057\\
0.251148\\
0.043655\\
\color{red} 0.082139\color{black}
\end{pmatrix}\end{equation*}
\begin{equation*}
\left[ \frac{{w}^{EM}_i}{{w}^{EM}_j} \right] =
\begin{pmatrix}
$\,\,$ 1 $\,\,$ & $\,\,$2.4808$\,\,$ & $\,\,$14.2722$\,\,$ & $\,\,$\color{red} 7.5854\color{black} $\,\,$ \\
$\,\,$0.4031$\,\,$ & $\,\,$ 1 $\,\,$ & $\,\,$5.7530$\,\,$ & $\,\,$\color{red} 3.0576\color{black}   $\,\,$ \\
$\,\,$0.0701$\,\,$ & $\,\,$0.1738$\,\,$ & $\,\,$ 1 $\,\,$ & $\,\,$\color{red} 0.5315\color{black}  $\,\,$ \\
$\,\,$\color{red} 0.1318\color{black} $\,\,$ & $\,\,$\color{red} 0.3271\color{black} $\,\,$ & $\,\,$\color{red} 1.8815\color{black} $\,\,$ & $\,\,$ 1  $\,\,$ \\
\end{pmatrix},
\end{equation*}

\begin{equation*}
\mathbf{w}^{\prime} =
\begin{pmatrix}
0.622076\\
0.250753\\
0.043587\\
0.083584
\end{pmatrix} =
0.998426\cdot
\begin{pmatrix}
0.623057\\
0.251148\\
0.043655\\
\color{gr} 0.083716\color{black}
\end{pmatrix},
\end{equation*}
\begin{equation*}
\left[ \frac{{w}^{\prime}_i}{{w}^{\prime}_j} \right] =
\begin{pmatrix}
$\,\,$ 1 $\,\,$ & $\,\,$2.4808$\,\,$ & $\,\,$14.2722$\,\,$ & $\,\,$\color{gr} 7.4425\color{black} $\,\,$ \\
$\,\,$0.4031$\,\,$ & $\,\,$ 1 $\,\,$ & $\,\,$5.7530$\,\,$ & $\,\,$\color{gr} \color{blue} 3\color{black}   $\,\,$ \\
$\,\,$0.0701$\,\,$ & $\,\,$0.1738$\,\,$ & $\,\,$ 1 $\,\,$ & $\,\,$\color{gr} 0.5215\color{black}  $\,\,$ \\
$\,\,$\color{gr} 0.1344\color{black} $\,\,$ & $\,\,$\color{gr} \color{blue}  1/3\color{black} $\,\,$ & $\,\,$\color{gr} 1.9177\color{black} $\,\,$ & $\,\,$ 1  $\,\,$ \\
\end{pmatrix},
\end{equation*}
\end{example}
\newpage
\begin{example}
\begin{equation*}
\mathbf{A} =
\begin{pmatrix}
$\,\,$ 1 $\,\,$ & $\,\,$4$\,\,$ & $\,\,$9$\,\,$ & $\,\,$8 $\,\,$ \\
$\,\,$ 1/4$\,\,$ & $\,\,$ 1 $\,\,$ & $\,\,$8$\,\,$ & $\,\,$3 $\,\,$ \\
$\,\,$ 1/9$\,\,$ & $\,\,$ 1/8$\,\,$ & $\,\,$ 1 $\,\,$ & $\,\,$ 1/2 $\,\,$ \\
$\,\,$ 1/8$\,\,$ & $\,\,$ 1/3$\,\,$ & $\,\,$2$\,\,$ & $\,\,$ 1  $\,\,$ \\
\end{pmatrix},
\qquad
\lambda_{\max} =
4.1403,
\qquad
CR = 0.0529
\end{equation*}

\begin{equation*}
\mathbf{w}^{EM} =
\begin{pmatrix}
0.635892\\
0.240168\\
0.044680\\
\color{red} 0.079261\color{black}
\end{pmatrix}\end{equation*}
\begin{equation*}
\left[ \frac{{w}^{EM}_i}{{w}^{EM}_j} \right] =
\begin{pmatrix}
$\,\,$ 1 $\,\,$ & $\,\,$2.6477$\,\,$ & $\,\,$14.2323$\,\,$ & $\,\,$\color{red} 8.0228\color{black} $\,\,$ \\
$\,\,$0.3777$\,\,$ & $\,\,$ 1 $\,\,$ & $\,\,$5.3753$\,\,$ & $\,\,$\color{red} 3.0301\color{black}   $\,\,$ \\
$\,\,$0.0703$\,\,$ & $\,\,$0.1860$\,\,$ & $\,\,$ 1 $\,\,$ & $\,\,$\color{red} 0.5637\color{black}  $\,\,$ \\
$\,\,$\color{red} 0.1246\color{black} $\,\,$ & $\,\,$\color{red} 0.3300\color{black} $\,\,$ & $\,\,$\color{red} 1.7740\color{black} $\,\,$ & $\,\,$ 1  $\,\,$ \\
\end{pmatrix},
\end{equation*}

\begin{equation*}
\mathbf{w}^{\prime} =
\begin{pmatrix}
0.635748\\
0.240114\\
0.044669\\
0.079469
\end{pmatrix} =
0.999775\cdot
\begin{pmatrix}
0.635892\\
0.240168\\
0.044680\\
\color{gr} 0.079486\color{black}
\end{pmatrix},
\end{equation*}
\begin{equation*}
\left[ \frac{{w}^{\prime}_i}{{w}^{\prime}_j} \right] =
\begin{pmatrix}
$\,\,$ 1 $\,\,$ & $\,\,$2.6477$\,\,$ & $\,\,$14.2323$\,\,$ & $\,\,$\color{gr} \color{blue} 8\color{black} $\,\,$ \\
$\,\,$0.3777$\,\,$ & $\,\,$ 1 $\,\,$ & $\,\,$5.3753$\,\,$ & $\,\,$\color{gr} 3.0215\color{black}   $\,\,$ \\
$\,\,$0.0703$\,\,$ & $\,\,$0.1860$\,\,$ & $\,\,$ 1 $\,\,$ & $\,\,$\color{gr} 0.5621\color{black}  $\,\,$ \\
$\,\,$\color{gr} \color{blue}  1/8\color{black} $\,\,$ & $\,\,$\color{gr} 0.3310\color{black} $\,\,$ & $\,\,$\color{gr} 1.7790\color{black} $\,\,$ & $\,\,$ 1  $\,\,$ \\
\end{pmatrix},
\end{equation*}
\end{example}
\newpage
\begin{example}
\begin{equation*}
\mathbf{A} =
\begin{pmatrix}
$\,\,$ 1 $\,\,$ & $\,\,$4$\,\,$ & $\,\,$9$\,\,$ & $\,\,$8 $\,\,$ \\
$\,\,$ 1/4$\,\,$ & $\,\,$ 1 $\,\,$ & $\,\,$9$\,\,$ & $\,\,$3 $\,\,$ \\
$\,\,$ 1/9$\,\,$ & $\,\,$ 1/9$\,\,$ & $\,\,$ 1 $\,\,$ & $\,\,$ 1/2 $\,\,$ \\
$\,\,$ 1/8$\,\,$ & $\,\,$ 1/3$\,\,$ & $\,\,$2$\,\,$ & $\,\,$ 1  $\,\,$ \\
\end{pmatrix},
\qquad
\lambda_{\max} =
4.1664,
\qquad
CR = 0.0627
\end{equation*}

\begin{equation*}
\mathbf{w}^{EM} =
\begin{pmatrix}
0.631928\\
0.246703\\
0.043179\\
\color{red} 0.078190\color{black}
\end{pmatrix}\end{equation*}
\begin{equation*}
\left[ \frac{{w}^{EM}_i}{{w}^{EM}_j} \right] =
\begin{pmatrix}
$\,\,$ 1 $\,\,$ & $\,\,$2.5615$\,\,$ & $\,\,$14.6352$\,\,$ & $\,\,$\color{red} 8.0819\color{black} $\,\,$ \\
$\,\,$0.3904$\,\,$ & $\,\,$ 1 $\,\,$ & $\,\,$5.7135$\,\,$ & $\,\,$\color{red} 3.1552\color{black}   $\,\,$ \\
$\,\,$0.0683$\,\,$ & $\,\,$0.1750$\,\,$ & $\,\,$ 1 $\,\,$ & $\,\,$\color{red} 0.5522\color{black}  $\,\,$ \\
$\,\,$\color{red} 0.1237\color{black} $\,\,$ & $\,\,$\color{red} 0.3169\color{black} $\,\,$ & $\,\,$\color{red} 1.8109\color{black} $\,\,$ & $\,\,$ 1  $\,\,$ \\
\end{pmatrix},
\end{equation*}

\begin{equation*}
\mathbf{w}^{\prime} =
\begin{pmatrix}
0.631423\\
0.246505\\
0.043144\\
0.078928
\end{pmatrix} =
0.999200\cdot
\begin{pmatrix}
0.631928\\
0.246703\\
0.043179\\
\color{gr} 0.078991\color{black}
\end{pmatrix},
\end{equation*}
\begin{equation*}
\left[ \frac{{w}^{\prime}_i}{{w}^{\prime}_j} \right] =
\begin{pmatrix}
$\,\,$ 1 $\,\,$ & $\,\,$2.5615$\,\,$ & $\,\,$14.6352$\,\,$ & $\,\,$\color{gr} \color{blue} 8\color{black} $\,\,$ \\
$\,\,$0.3904$\,\,$ & $\,\,$ 1 $\,\,$ & $\,\,$5.7135$\,\,$ & $\,\,$\color{gr} 3.1232\color{black}   $\,\,$ \\
$\,\,$0.0683$\,\,$ & $\,\,$0.1750$\,\,$ & $\,\,$ 1 $\,\,$ & $\,\,$\color{gr} 0.5466\color{black}  $\,\,$ \\
$\,\,$\color{gr} \color{blue}  1/8\color{black} $\,\,$ & $\,\,$\color{gr} 0.3202\color{black} $\,\,$ & $\,\,$\color{gr} 1.8294\color{black} $\,\,$ & $\,\,$ 1  $\,\,$ \\
\end{pmatrix},
\end{equation*}
\end{example}
\newpage
\begin{example}
\begin{equation*}
\mathbf{A} =
\begin{pmatrix}
$\,\,$ 1 $\,\,$ & $\,\,$5$\,\,$ & $\,\,$3$\,\,$ & $\,\,$7 $\,\,$ \\
$\,\,$ 1/5$\,\,$ & $\,\,$ 1 $\,\,$ & $\,\,$1$\,\,$ & $\,\,$1 $\,\,$ \\
$\,\,$ 1/3$\,\,$ & $\,\,$ 1 $\,\,$ & $\,\,$ 1 $\,\,$ & $\,\,$ 1/2 $\,\,$ \\
$\,\,$ 1/7$\,\,$ & $\,\,$ 1 $\,\,$ & $\,\,$2$\,\,$ & $\,\,$ 1  $\,\,$ \\
\end{pmatrix},
\qquad
\lambda_{\max} =
4.2095,
\qquad
CR = 0.0790
\end{equation*}

\begin{equation*}
\mathbf{w}^{EM} =
\begin{pmatrix}
0.613545\\
\color{red} 0.120956\color{black} \\
0.123527\\
0.141972
\end{pmatrix}\end{equation*}
\begin{equation*}
\left[ \frac{{w}^{EM}_i}{{w}^{EM}_j} \right] =
\begin{pmatrix}
$\,\,$ 1 $\,\,$ & $\,\,$\color{red} 5.0725\color{black} $\,\,$ & $\,\,$4.9669$\,\,$ & $\,\,$4.3216$\,\,$ \\
$\,\,$\color{red} 0.1971\color{black} $\,\,$ & $\,\,$ 1 $\,\,$ & $\,\,$\color{red} 0.9792\color{black} $\,\,$ & $\,\,$\color{red} 0.8520\color{black}   $\,\,$ \\
$\,\,$0.2013$\,\,$ & $\,\,$\color{red} 1.0213\color{black} $\,\,$ & $\,\,$ 1 $\,\,$ & $\,\,$0.8701 $\,\,$ \\
$\,\,$0.2314$\,\,$ & $\,\,$\color{red} 1.1737\color{black} $\,\,$ & $\,\,$1.1493$\,\,$ & $\,\,$ 1  $\,\,$ \\
\end{pmatrix},
\end{equation*}

\begin{equation*}
\mathbf{w}^{\prime} =
\begin{pmatrix}
0.612471\\
0.122494\\
0.123310\\
0.141724
\end{pmatrix} =
0.998250\cdot
\begin{pmatrix}
0.613545\\
\color{gr} 0.122709\color{black} \\
0.123527\\
0.141972
\end{pmatrix},
\end{equation*}
\begin{equation*}
\left[ \frac{{w}^{\prime}_i}{{w}^{\prime}_j} \right] =
\begin{pmatrix}
$\,\,$ 1 $\,\,$ & $\,\,$\color{gr} \color{blue} 5\color{black} $\,\,$ & $\,\,$4.9669$\,\,$ & $\,\,$4.3216$\,\,$ \\
$\,\,$\color{gr} \color{blue}  1/5\color{black} $\,\,$ & $\,\,$ 1 $\,\,$ & $\,\,$\color{gr} 0.9934\color{black} $\,\,$ & $\,\,$\color{gr} 0.8643\color{black}   $\,\,$ \\
$\,\,$0.2013$\,\,$ & $\,\,$\color{gr} 1.0067\color{black} $\,\,$ & $\,\,$ 1 $\,\,$ & $\,\,$0.8701 $\,\,$ \\
$\,\,$0.2314$\,\,$ & $\,\,$\color{gr} 1.1570\color{black} $\,\,$ & $\,\,$1.1493$\,\,$ & $\,\,$ 1  $\,\,$ \\
\end{pmatrix},
\end{equation*}
\end{example}
\newpage
\begin{example}
\begin{equation*}
\mathbf{A} =
\begin{pmatrix}
$\,\,$ 1 $\,\,$ & $\,\,$5$\,\,$ & $\,\,$3$\,\,$ & $\,\,$8 $\,\,$ \\
$\,\,$ 1/5$\,\,$ & $\,\,$ 1 $\,\,$ & $\,\,$1$\,\,$ & $\,\,$1 $\,\,$ \\
$\,\,$ 1/3$\,\,$ & $\,\,$ 1 $\,\,$ & $\,\,$ 1 $\,\,$ & $\,\,$ 1/2 $\,\,$ \\
$\,\,$ 1/8$\,\,$ & $\,\,$ 1 $\,\,$ & $\,\,$2$\,\,$ & $\,\,$ 1  $\,\,$ \\
\end{pmatrix},
\qquad
\lambda_{\max} =
4.2460,
\qquad
CR = 0.0928
\end{equation*}

\begin{equation*}
\mathbf{w}^{EM} =
\begin{pmatrix}
0.626067\\
\color{red} 0.117555\color{black} \\
0.121310\\
0.135067
\end{pmatrix}\end{equation*}
\begin{equation*}
\left[ \frac{{w}^{EM}_i}{{w}^{EM}_j} \right] =
\begin{pmatrix}
$\,\,$ 1 $\,\,$ & $\,\,$\color{red} 5.3257\color{black} $\,\,$ & $\,\,$5.1609$\,\,$ & $\,\,$4.6352$\,\,$ \\
$\,\,$\color{red} 0.1878\color{black} $\,\,$ & $\,\,$ 1 $\,\,$ & $\,\,$\color{red} 0.9690\color{black} $\,\,$ & $\,\,$\color{red} 0.8703\color{black}   $\,\,$ \\
$\,\,$0.1938$\,\,$ & $\,\,$\color{red} 1.0319\color{black} $\,\,$ & $\,\,$ 1 $\,\,$ & $\,\,$0.8981 $\,\,$ \\
$\,\,$0.2157$\,\,$ & $\,\,$\color{red} 1.1490\color{black} $\,\,$ & $\,\,$1.1134$\,\,$ & $\,\,$ 1  $\,\,$ \\
\end{pmatrix},
\end{equation*}

\begin{equation*}
\mathbf{w}^{\prime} =
\begin{pmatrix}
0.623726\\
0.120856\\
0.120856\\
0.134562
\end{pmatrix} =
0.996259\cdot
\begin{pmatrix}
0.626067\\
\color{gr} 0.121310\color{black} \\
0.121310\\
0.135067
\end{pmatrix},
\end{equation*}
\begin{equation*}
\left[ \frac{{w}^{\prime}_i}{{w}^{\prime}_j} \right] =
\begin{pmatrix}
$\,\,$ 1 $\,\,$ & $\,\,$\color{gr} 5.1609\color{black} $\,\,$ & $\,\,$5.1609$\,\,$ & $\,\,$4.6352$\,\,$ \\
$\,\,$\color{gr} 0.1938\color{black} $\,\,$ & $\,\,$ 1 $\,\,$ & $\,\,$\color{gr} \color{blue} 1\color{black} $\,\,$ & $\,\,$\color{gr} 0.8981\color{black}   $\,\,$ \\
$\,\,$0.1938$\,\,$ & $\,\,$\color{gr} \color{blue} 1\color{black} $\,\,$ & $\,\,$ 1 $\,\,$ & $\,\,$0.8981 $\,\,$ \\
$\,\,$0.2157$\,\,$ & $\,\,$\color{gr} 1.1134\color{black} $\,\,$ & $\,\,$1.1134$\,\,$ & $\,\,$ 1  $\,\,$ \\
\end{pmatrix},
\end{equation*}
\end{example}
\newpage
\begin{example}
\begin{equation*}
\mathbf{A} =
\begin{pmatrix}
$\,\,$ 1 $\,\,$ & $\,\,$5$\,\,$ & $\,\,$6$\,\,$ & $\,\,$1 $\,\,$ \\
$\,\,$ 1/5$\,\,$ & $\,\,$ 1 $\,\,$ & $\,\,$2$\,\,$ & $\,\,$1 $\,\,$ \\
$\,\,$ 1/6$\,\,$ & $\,\,$ 1/2$\,\,$ & $\,\,$ 1 $\,\,$ & $\,\,$ 1/3 $\,\,$ \\
$\,\,$ 1 $\,\,$ & $\,\,$ 1 $\,\,$ & $\,\,$3$\,\,$ & $\,\,$ 1  $\,\,$ \\
\end{pmatrix},
\qquad
\lambda_{\max} =
4.2277,
\qquad
CR = 0.0859
\end{equation*}

\begin{equation*}
\mathbf{w}^{EM} =
\begin{pmatrix}
0.484184\\
0.163504\\
\color{red} 0.078596\color{black} \\
0.273715
\end{pmatrix}\end{equation*}
\begin{equation*}
\left[ \frac{{w}^{EM}_i}{{w}^{EM}_j} \right] =
\begin{pmatrix}
$\,\,$ 1 $\,\,$ & $\,\,$2.9613$\,\,$ & $\,\,$\color{red} 6.1604\color{black} $\,\,$ & $\,\,$1.7689$\,\,$ \\
$\,\,$0.3377$\,\,$ & $\,\,$ 1 $\,\,$ & $\,\,$\color{red} 2.0803\color{black} $\,\,$ & $\,\,$0.5973  $\,\,$ \\
$\,\,$\color{red} 0.1623\color{black} $\,\,$ & $\,\,$\color{red} 0.4807\color{black} $\,\,$ & $\,\,$ 1 $\,\,$ & $\,\,$\color{red} 0.2871\color{black}  $\,\,$ \\
$\,\,$0.5653$\,\,$ & $\,\,$1.6741$\,\,$ & $\,\,$\color{red} 3.4825\color{black} $\,\,$ & $\,\,$ 1  $\,\,$ \\
\end{pmatrix},
\end{equation*}

\begin{equation*}
\mathbf{w}^{\prime} =
\begin{pmatrix}
0.483169\\
0.163161\\
0.080528\\
0.273142
\end{pmatrix} =
0.997903\cdot
\begin{pmatrix}
0.484184\\
0.163504\\
\color{gr} 0.080697\color{black} \\
0.273715
\end{pmatrix},
\end{equation*}
\begin{equation*}
\left[ \frac{{w}^{\prime}_i}{{w}^{\prime}_j} \right] =
\begin{pmatrix}
$\,\,$ 1 $\,\,$ & $\,\,$2.9613$\,\,$ & $\,\,$\color{gr} \color{blue} 6\color{black} $\,\,$ & $\,\,$1.7689$\,\,$ \\
$\,\,$0.3377$\,\,$ & $\,\,$ 1 $\,\,$ & $\,\,$\color{gr} 2.0261\color{black} $\,\,$ & $\,\,$0.5973  $\,\,$ \\
$\,\,$\color{gr} \color{blue}  1/6\color{black} $\,\,$ & $\,\,$\color{gr} 0.4936\color{black} $\,\,$ & $\,\,$ 1 $\,\,$ & $\,\,$\color{gr} 0.2948\color{black}  $\,\,$ \\
$\,\,$0.5653$\,\,$ & $\,\,$1.6741$\,\,$ & $\,\,$\color{gr} 3.3919\color{black} $\,\,$ & $\,\,$ 1  $\,\,$ \\
\end{pmatrix},
\end{equation*}
\end{example}
\newpage
\begin{example}
\begin{equation*}
\mathbf{A} =
\begin{pmatrix}
$\,\,$ 1 $\,\,$ & $\,\,$5$\,\,$ & $\,\,$7$\,\,$ & $\,\,$1 $\,\,$ \\
$\,\,$ 1/5$\,\,$ & $\,\,$ 1 $\,\,$ & $\,\,$2$\,\,$ & $\,\,$1 $\,\,$ \\
$\,\,$ 1/7$\,\,$ & $\,\,$ 1/2$\,\,$ & $\,\,$ 1 $\,\,$ & $\,\,$ 1/4 $\,\,$ \\
$\,\,$ 1 $\,\,$ & $\,\,$ 1 $\,\,$ & $\,\,$4$\,\,$ & $\,\,$ 1  $\,\,$ \\
\end{pmatrix},
\qquad
\lambda_{\max} =
4.2287,
\qquad
CR = 0.0862
\end{equation*}

\begin{equation*}
\mathbf{w}^{EM} =
\begin{pmatrix}
0.485707\\
0.160782\\
\color{red} 0.068461\color{black} \\
0.285050
\end{pmatrix}\end{equation*}
\begin{equation*}
\left[ \frac{{w}^{EM}_i}{{w}^{EM}_j} \right] =
\begin{pmatrix}
$\,\,$ 1 $\,\,$ & $\,\,$3.0209$\,\,$ & $\,\,$\color{red} 7.0946\color{black} $\,\,$ & $\,\,$1.7039$\,\,$ \\
$\,\,$0.3310$\,\,$ & $\,\,$ 1 $\,\,$ & $\,\,$\color{red} 2.3485\color{black} $\,\,$ & $\,\,$0.5640  $\,\,$ \\
$\,\,$\color{red} 0.1410\color{black} $\,\,$ & $\,\,$\color{red} 0.4258\color{black} $\,\,$ & $\,\,$ 1 $\,\,$ & $\,\,$\color{red} 0.2402\color{black}  $\,\,$ \\
$\,\,$0.5869$\,\,$ & $\,\,$1.7729$\,\,$ & $\,\,$\color{red} 4.1637\color{black} $\,\,$ & $\,\,$ 1  $\,\,$ \\
\end{pmatrix},
\end{equation*}

\begin{equation*}
\mathbf{w}^{\prime} =
\begin{pmatrix}
0.485258\\
0.160634\\
0.069323\\
0.284786
\end{pmatrix} =
0.999076\cdot
\begin{pmatrix}
0.485707\\
0.160782\\
\color{gr} 0.069387\color{black} \\
0.285050
\end{pmatrix},
\end{equation*}
\begin{equation*}
\left[ \frac{{w}^{\prime}_i}{{w}^{\prime}_j} \right] =
\begin{pmatrix}
$\,\,$ 1 $\,\,$ & $\,\,$3.0209$\,\,$ & $\,\,$\color{gr} \color{blue} 7\color{black} $\,\,$ & $\,\,$1.7039$\,\,$ \\
$\,\,$0.3310$\,\,$ & $\,\,$ 1 $\,\,$ & $\,\,$\color{gr} 2.3172\color{black} $\,\,$ & $\,\,$0.5640  $\,\,$ \\
$\,\,$\color{gr} \color{blue}  1/7\color{black} $\,\,$ & $\,\,$\color{gr} 0.4316\color{black} $\,\,$ & $\,\,$ 1 $\,\,$ & $\,\,$\color{gr} 0.2434\color{black}  $\,\,$ \\
$\,\,$0.5869$\,\,$ & $\,\,$1.7729$\,\,$ & $\,\,$\color{gr} 4.1081\color{black} $\,\,$ & $\,\,$ 1  $\,\,$ \\
\end{pmatrix},
\end{equation*}
\end{example}
\newpage
\begin{example}
\begin{equation*}
\mathbf{A} =
\begin{pmatrix}
$\,\,$ 1 $\,\,$ & $\,\,$5$\,\,$ & $\,\,$7$\,\,$ & $\,\,$1 $\,\,$ \\
$\,\,$ 1/5$\,\,$ & $\,\,$ 1 $\,\,$ & $\,\,$3$\,\,$ & $\,\,$1 $\,\,$ \\
$\,\,$ 1/7$\,\,$ & $\,\,$ 1/3$\,\,$ & $\,\,$ 1 $\,\,$ & $\,\,$ 1/5 $\,\,$ \\
$\,\,$ 1 $\,\,$ & $\,\,$ 1 $\,\,$ & $\,\,$5$\,\,$ & $\,\,$ 1  $\,\,$ \\
\end{pmatrix},
\qquad
\lambda_{\max} =
4.2309,
\qquad
CR = 0.0871
\end{equation*}

\begin{equation*}
\mathbf{w}^{EM} =
\begin{pmatrix}
0.480266\\
0.172490\\
\color{red} 0.056998\color{black} \\
0.290245
\end{pmatrix}\end{equation*}
\begin{equation*}
\left[ \frac{{w}^{EM}_i}{{w}^{EM}_j} \right] =
\begin{pmatrix}
$\,\,$ 1 $\,\,$ & $\,\,$2.7843$\,\,$ & $\,\,$\color{red} 8.4260\color{black} $\,\,$ & $\,\,$1.6547$\,\,$ \\
$\,\,$0.3592$\,\,$ & $\,\,$ 1 $\,\,$ & $\,\,$\color{red} 3.0262\color{black} $\,\,$ & $\,\,$0.5943  $\,\,$ \\
$\,\,$\color{red} 0.1187\color{black} $\,\,$ & $\,\,$\color{red} 0.3304\color{black} $\,\,$ & $\,\,$ 1 $\,\,$ & $\,\,$\color{red} 0.1964\color{black}  $\,\,$ \\
$\,\,$0.6043$\,\,$ & $\,\,$1.6827$\,\,$ & $\,\,$\color{red} 5.0922\color{black} $\,\,$ & $\,\,$ 1  $\,\,$ \\
\end{pmatrix},
\end{equation*}

\begin{equation*}
\mathbf{w}^{\prime} =
\begin{pmatrix}
0.480027\\
0.172404\\
0.057468\\
0.290101
\end{pmatrix} =
0.999502\cdot
\begin{pmatrix}
0.480266\\
0.172490\\
\color{gr} 0.057497\color{black} \\
0.290245
\end{pmatrix},
\end{equation*}
\begin{equation*}
\left[ \frac{{w}^{\prime}_i}{{w}^{\prime}_j} \right] =
\begin{pmatrix}
$\,\,$ 1 $\,\,$ & $\,\,$2.7843$\,\,$ & $\,\,$\color{gr} 8.3530\color{black} $\,\,$ & $\,\,$1.6547$\,\,$ \\
$\,\,$0.3592$\,\,$ & $\,\,$ 1 $\,\,$ & $\,\,$\color{gr} \color{blue} 3\color{black} $\,\,$ & $\,\,$0.5943  $\,\,$ \\
$\,\,$\color{gr} 0.1197\color{black} $\,\,$ & $\,\,$\color{gr} \color{blue}  1/3\color{black} $\,\,$ & $\,\,$ 1 $\,\,$ & $\,\,$\color{gr} 0.1981\color{black}  $\,\,$ \\
$\,\,$0.6043$\,\,$ & $\,\,$1.6827$\,\,$ & $\,\,$\color{gr} 5.0480\color{black} $\,\,$ & $\,\,$ 1  $\,\,$ \\
\end{pmatrix},
\end{equation*}
\end{example}
\newpage
\begin{example}
\begin{equation*}
\mathbf{A} =
\begin{pmatrix}
$\,\,$ 1 $\,\,$ & $\,\,$5$\,\,$ & $\,\,$8$\,\,$ & $\,\,$1 $\,\,$ \\
$\,\,$ 1/5$\,\,$ & $\,\,$ 1 $\,\,$ & $\,\,$3$\,\,$ & $\,\,$1 $\,\,$ \\
$\,\,$ 1/8$\,\,$ & $\,\,$ 1/3$\,\,$ & $\,\,$ 1 $\,\,$ & $\,\,$ 1/5 $\,\,$ \\
$\,\,$ 1 $\,\,$ & $\,\,$ 1 $\,\,$ & $\,\,$5$\,\,$ & $\,\,$ 1  $\,\,$ \\
\end{pmatrix},
\qquad
\lambda_{\max} =
4.2259,
\qquad
CR = 0.0852
\end{equation*}

\begin{equation*}
\mathbf{w}^{EM} =
\begin{pmatrix}
0.487576\\
0.170044\\
\color{red} 0.054323\color{black} \\
0.288057
\end{pmatrix}\end{equation*}
\begin{equation*}
\left[ \frac{{w}^{EM}_i}{{w}^{EM}_j} \right] =
\begin{pmatrix}
$\,\,$ 1 $\,\,$ & $\,\,$2.8674$\,\,$ & $\,\,$\color{red} 8.9755\color{black} $\,\,$ & $\,\,$1.6926$\,\,$ \\
$\,\,$0.3488$\,\,$ & $\,\,$ 1 $\,\,$ & $\,\,$\color{red} 3.1302\color{black} $\,\,$ & $\,\,$0.5903  $\,\,$ \\
$\,\,$\color{red} 0.1114\color{black} $\,\,$ & $\,\,$\color{red} 0.3195\color{black} $\,\,$ & $\,\,$ 1 $\,\,$ & $\,\,$\color{red} 0.1886\color{black}  $\,\,$ \\
$\,\,$0.5908$\,\,$ & $\,\,$1.6940$\,\,$ & $\,\,$\color{red} 5.3027\color{black} $\,\,$ & $\,\,$ 1  $\,\,$ \\
\end{pmatrix},
\end{equation*}

\begin{equation*}
\mathbf{w}^{\prime} =
\begin{pmatrix}
0.486429\\
0.169644\\
0.056548\\
0.287379
\end{pmatrix} =
0.997647\cdot
\begin{pmatrix}
0.487576\\
0.170044\\
\color{gr} 0.056681\color{black} \\
0.288057
\end{pmatrix},
\end{equation*}
\begin{equation*}
\left[ \frac{{w}^{\prime}_i}{{w}^{\prime}_j} \right] =
\begin{pmatrix}
$\,\,$ 1 $\,\,$ & $\,\,$2.8674$\,\,$ & $\,\,$\color{gr} 8.6021\color{black} $\,\,$ & $\,\,$1.6926$\,\,$ \\
$\,\,$0.3488$\,\,$ & $\,\,$ 1 $\,\,$ & $\,\,$\color{gr} \color{blue} 3\color{black} $\,\,$ & $\,\,$0.5903  $\,\,$ \\
$\,\,$\color{gr} 0.1163\color{black} $\,\,$ & $\,\,$\color{gr} \color{blue}  1/3\color{black} $\,\,$ & $\,\,$ 1 $\,\,$ & $\,\,$\color{gr} 0.1968\color{black}  $\,\,$ \\
$\,\,$0.5908$\,\,$ & $\,\,$1.6940$\,\,$ & $\,\,$\color{gr} 5.0820\color{black} $\,\,$ & $\,\,$ 1  $\,\,$ \\
\end{pmatrix},
\end{equation*}
\end{example}
\newpage
\begin{example}
\begin{equation*}
\mathbf{A} =
\begin{pmatrix}
$\,\,$ 1 $\,\,$ & $\,\,$5$\,\,$ & $\,\,$8$\,\,$ & $\,\,$3 $\,\,$ \\
$\,\,$ 1/5$\,\,$ & $\,\,$ 1 $\,\,$ & $\,\,$1$\,\,$ & $\,\,$1 $\,\,$ \\
$\,\,$ 1/8$\,\,$ & $\,\,$ 1 $\,\,$ & $\,\,$ 1 $\,\,$ & $\,\,$ 1/4 $\,\,$ \\
$\,\,$ 1/3$\,\,$ & $\,\,$ 1 $\,\,$ & $\,\,$4$\,\,$ & $\,\,$ 1  $\,\,$ \\
\end{pmatrix},
\qquad
\lambda_{\max} =
4.1655,
\qquad
CR = 0.0624
\end{equation*}

\begin{equation*}
\mathbf{w}^{EM} =
\begin{pmatrix}
\color{red} 0.591875\color{black} \\
0.126394\\
0.079288\\
0.202442
\end{pmatrix}\end{equation*}
\begin{equation*}
\left[ \frac{{w}^{EM}_i}{{w}^{EM}_j} \right] =
\begin{pmatrix}
$\,\,$ 1 $\,\,$ & $\,\,$\color{red} 4.6828\color{black} $\,\,$ & $\,\,$\color{red} 7.4649\color{black} $\,\,$ & $\,\,$\color{red} 2.9237\color{black} $\,\,$ \\
$\,\,$\color{red} 0.2135\color{black} $\,\,$ & $\,\,$ 1 $\,\,$ & $\,\,$1.5941$\,\,$ & $\,\,$0.6243  $\,\,$ \\
$\,\,$\color{red} 0.1340\color{black} $\,\,$ & $\,\,$0.6273$\,\,$ & $\,\,$ 1 $\,\,$ & $\,\,$0.3917 $\,\,$ \\
$\,\,$\color{red} 0.3420\color{black} $\,\,$ & $\,\,$1.6017$\,\,$ & $\,\,$2.5533$\,\,$ & $\,\,$ 1  $\,\,$ \\
\end{pmatrix},
\end{equation*}

\begin{equation*}
\mathbf{w}^{\prime} =
\begin{pmatrix}
0.598086\\
0.124471\\
0.078082\\
0.199362
\end{pmatrix} =
0.984784\cdot
\begin{pmatrix}
\color{gr} 0.607327\color{black} \\
0.126394\\
0.079288\\
0.202442
\end{pmatrix},
\end{equation*}
\begin{equation*}
\left[ \frac{{w}^{\prime}_i}{{w}^{\prime}_j} \right] =
\begin{pmatrix}
$\,\,$ 1 $\,\,$ & $\,\,$\color{gr} 4.8050\color{black} $\,\,$ & $\,\,$\color{gr} 7.6598\color{black} $\,\,$ & $\,\,$\color{gr} \color{blue} 3\color{black} $\,\,$ \\
$\,\,$\color{gr} 0.2081\color{black} $\,\,$ & $\,\,$ 1 $\,\,$ & $\,\,$1.5941$\,\,$ & $\,\,$0.6243  $\,\,$ \\
$\,\,$\color{gr} 0.1306\color{black} $\,\,$ & $\,\,$0.6273$\,\,$ & $\,\,$ 1 $\,\,$ & $\,\,$0.3917 $\,\,$ \\
$\,\,$\color{gr} \color{blue}  1/3\color{black} $\,\,$ & $\,\,$1.6017$\,\,$ & $\,\,$2.5533$\,\,$ & $\,\,$ 1  $\,\,$ \\
\end{pmatrix},
\end{equation*}
\end{example}
\newpage
\begin{example}
\begin{equation*}
\mathbf{A} =
\begin{pmatrix}
$\,\,$ 1 $\,\,$ & $\,\,$5$\,\,$ & $\,\,$8$\,\,$ & $\,\,$3 $\,\,$ \\
$\,\,$ 1/5$\,\,$ & $\,\,$ 1 $\,\,$ & $\,\,$1$\,\,$ & $\,\,$1 $\,\,$ \\
$\,\,$ 1/8$\,\,$ & $\,\,$ 1 $\,\,$ & $\,\,$ 1 $\,\,$ & $\,\,$ 1/5 $\,\,$ \\
$\,\,$ 1/3$\,\,$ & $\,\,$ 1 $\,\,$ & $\,\,$5$\,\,$ & $\,\,$ 1  $\,\,$ \\
\end{pmatrix},
\qquad
\lambda_{\max} =
4.2259,
\qquad
CR = 0.0852
\end{equation*}

\begin{equation*}
\mathbf{w}^{EM} =
\begin{pmatrix}
\color{red} 0.582764\color{black} \\
0.126314\\
0.075118\\
0.215804
\end{pmatrix}\end{equation*}
\begin{equation*}
\left[ \frac{{w}^{EM}_i}{{w}^{EM}_j} \right] =
\begin{pmatrix}
$\,\,$ 1 $\,\,$ & $\,\,$\color{red} 4.6136\color{black} $\,\,$ & $\,\,$\color{red} 7.7580\color{black} $\,\,$ & $\,\,$\color{red} 2.7004\color{black} $\,\,$ \\
$\,\,$\color{red} 0.2168\color{black} $\,\,$ & $\,\,$ 1 $\,\,$ & $\,\,$1.6816$\,\,$ & $\,\,$0.5853  $\,\,$ \\
$\,\,$\color{red} 0.1289\color{black} $\,\,$ & $\,\,$0.5947$\,\,$ & $\,\,$ 1 $\,\,$ & $\,\,$0.3481 $\,\,$ \\
$\,\,$\color{red} 0.3703\color{black} $\,\,$ & $\,\,$1.7085$\,\,$ & $\,\,$2.8729$\,\,$ & $\,\,$ 1  $\,\,$ \\
\end{pmatrix},
\end{equation*}

\begin{equation*}
\mathbf{w}^{\prime} =
\begin{pmatrix}
0.590213\\
0.124059\\
0.073777\\
0.211951
\end{pmatrix} =
0.982147\cdot
\begin{pmatrix}
\color{gr} 0.600942\color{black} \\
0.126314\\
0.075118\\
0.215804
\end{pmatrix},
\end{equation*}
\begin{equation*}
\left[ \frac{{w}^{\prime}_i}{{w}^{\prime}_j} \right] =
\begin{pmatrix}
$\,\,$ 1 $\,\,$ & $\,\,$\color{gr} 4.7575\color{black} $\,\,$ & $\,\,$\color{gr} \color{blue} 8\color{black} $\,\,$ & $\,\,$\color{gr} 2.7847\color{black} $\,\,$ \\
$\,\,$\color{gr} 0.2102\color{black} $\,\,$ & $\,\,$ 1 $\,\,$ & $\,\,$1.6816$\,\,$ & $\,\,$0.5853  $\,\,$ \\
$\,\,$\color{gr} \color{blue}  1/8\color{black} $\,\,$ & $\,\,$0.5947$\,\,$ & $\,\,$ 1 $\,\,$ & $\,\,$0.3481 $\,\,$ \\
$\,\,$\color{gr} 0.3591\color{black} $\,\,$ & $\,\,$1.7085$\,\,$ & $\,\,$2.8729$\,\,$ & $\,\,$ 1  $\,\,$ \\
\end{pmatrix},
\end{equation*}
\end{example}
\newpage
\begin{example}
\begin{equation*}
\mathbf{A} =
\begin{pmatrix}
$\,\,$ 1 $\,\,$ & $\,\,$5$\,\,$ & $\,\,$8$\,\,$ & $\,\,$6 $\,\,$ \\
$\,\,$ 1/5$\,\,$ & $\,\,$ 1 $\,\,$ & $\,\,$1$\,\,$ & $\,\,$2 $\,\,$ \\
$\,\,$ 1/8$\,\,$ & $\,\,$ 1 $\,\,$ & $\,\,$ 1 $\,\,$ & $\,\,$ 1/2 $\,\,$ \\
$\,\,$ 1/6$\,\,$ & $\,\,$ 1/2$\,\,$ & $\,\,$2$\,\,$ & $\,\,$ 1  $\,\,$ \\
\end{pmatrix},
\qquad
\lambda_{\max} =
4.1655,
\qquad
CR = 0.0624
\end{equation*}

\begin{equation*}
\mathbf{w}^{EM} =
\begin{pmatrix}
\color{red} 0.658533\color{black} \\
0.140629\\
0.088218\\
0.112621
\end{pmatrix}\end{equation*}
\begin{equation*}
\left[ \frac{{w}^{EM}_i}{{w}^{EM}_j} \right] =
\begin{pmatrix}
$\,\,$ 1 $\,\,$ & $\,\,$\color{red} 4.6828\color{black} $\,\,$ & $\,\,$\color{red} 7.4649\color{black} $\,\,$ & $\,\,$\color{red} 5.8473\color{black} $\,\,$ \\
$\,\,$\color{red} 0.2135\color{black} $\,\,$ & $\,\,$ 1 $\,\,$ & $\,\,$1.5941$\,\,$ & $\,\,$1.2487  $\,\,$ \\
$\,\,$\color{red} 0.1340\color{black} $\,\,$ & $\,\,$0.6273$\,\,$ & $\,\,$ 1 $\,\,$ & $\,\,$0.7833 $\,\,$ \\
$\,\,$\color{red} 0.1710\color{black} $\,\,$ & $\,\,$0.8008$\,\,$ & $\,\,$1.2766$\,\,$ & $\,\,$ 1  $\,\,$ \\
\end{pmatrix},
\end{equation*}

\begin{equation*}
\mathbf{w}^{\prime} =
\begin{pmatrix}
0.664304\\
0.138252\\
0.086727\\
0.110717
\end{pmatrix} =
0.983099\cdot
\begin{pmatrix}
\color{gr} 0.675725\color{black} \\
0.140629\\
0.088218\\
0.112621
\end{pmatrix},
\end{equation*}
\begin{equation*}
\left[ \frac{{w}^{\prime}_i}{{w}^{\prime}_j} \right] =
\begin{pmatrix}
$\,\,$ 1 $\,\,$ & $\,\,$\color{gr} 4.8050\color{black} $\,\,$ & $\,\,$\color{gr} 7.6598\color{black} $\,\,$ & $\,\,$\color{gr} \color{blue} 6\color{black} $\,\,$ \\
$\,\,$\color{gr} 0.2081\color{black} $\,\,$ & $\,\,$ 1 $\,\,$ & $\,\,$1.5941$\,\,$ & $\,\,$1.2487  $\,\,$ \\
$\,\,$\color{gr} 0.1306\color{black} $\,\,$ & $\,\,$0.6273$\,\,$ & $\,\,$ 1 $\,\,$ & $\,\,$0.7833 $\,\,$ \\
$\,\,$\color{gr} \color{blue}  1/6\color{black} $\,\,$ & $\,\,$0.8008$\,\,$ & $\,\,$1.2766$\,\,$ & $\,\,$ 1  $\,\,$ \\
\end{pmatrix},
\end{equation*}
\end{example}
\newpage
\begin{example}
\begin{equation*}
\mathbf{A} =
\begin{pmatrix}
$\,\,$ 1 $\,\,$ & $\,\,$5$\,\,$ & $\,\,$8$\,\,$ & $\,\,$7 $\,\,$ \\
$\,\,$ 1/5$\,\,$ & $\,\,$ 1 $\,\,$ & $\,\,$1$\,\,$ & $\,\,$2 $\,\,$ \\
$\,\,$ 1/8$\,\,$ & $\,\,$ 1 $\,\,$ & $\,\,$ 1 $\,\,$ & $\,\,$ 1/2 $\,\,$ \\
$\,\,$ 1/7$\,\,$ & $\,\,$ 1/2$\,\,$ & $\,\,$2$\,\,$ & $\,\,$ 1  $\,\,$ \\
\end{pmatrix},
\qquad
\lambda_{\max} =
4.1665,
\qquad
CR = 0.0628
\end{equation*}

\begin{equation*}
\mathbf{w}^{EM} =
\begin{pmatrix}
\color{red} 0.670113\color{black} \\
0.136900\\
0.086502\\
0.106485
\end{pmatrix}\end{equation*}
\begin{equation*}
\left[ \frac{{w}^{EM}_i}{{w}^{EM}_j} \right] =
\begin{pmatrix}
$\,\,$ 1 $\,\,$ & $\,\,$\color{red} 4.8949\color{black} $\,\,$ & $\,\,$\color{red} 7.7468\color{black} $\,\,$ & $\,\,$\color{red} 6.2930\color{black} $\,\,$ \\
$\,\,$\color{red} 0.2043\color{black} $\,\,$ & $\,\,$ 1 $\,\,$ & $\,\,$1.5826$\,\,$ & $\,\,$1.2856  $\,\,$ \\
$\,\,$\color{red} 0.1291\color{black} $\,\,$ & $\,\,$0.6319$\,\,$ & $\,\,$ 1 $\,\,$ & $\,\,$0.8123 $\,\,$ \\
$\,\,$\color{red} 0.1589\color{black} $\,\,$ & $\,\,$0.7778$\,\,$ & $\,\,$1.2310$\,\,$ & $\,\,$ 1  $\,\,$ \\
\end{pmatrix},
\end{equation*}

\begin{equation*}
\mathbf{w}^{\prime} =
\begin{pmatrix}
0.674792\\
0.134958\\
0.085275\\
0.104974
\end{pmatrix} =
0.985816\cdot
\begin{pmatrix}
\color{gr} 0.684502\color{black} \\
0.136900\\
0.086502\\
0.106485
\end{pmatrix},
\end{equation*}
\begin{equation*}
\left[ \frac{{w}^{\prime}_i}{{w}^{\prime}_j} \right] =
\begin{pmatrix}
$\,\,$ 1 $\,\,$ & $\,\,$\color{gr} \color{blue} 5\color{black} $\,\,$ & $\,\,$\color{gr} 7.9132\color{black} $\,\,$ & $\,\,$\color{gr} 6.4282\color{black} $\,\,$ \\
$\,\,$\color{gr} \color{blue}  1/5\color{black} $\,\,$ & $\,\,$ 1 $\,\,$ & $\,\,$1.5826$\,\,$ & $\,\,$1.2856  $\,\,$ \\
$\,\,$\color{gr} 0.1264\color{black} $\,\,$ & $\,\,$0.6319$\,\,$ & $\,\,$ 1 $\,\,$ & $\,\,$0.8123 $\,\,$ \\
$\,\,$\color{gr} 0.1556\color{black} $\,\,$ & $\,\,$0.7778$\,\,$ & $\,\,$1.2310$\,\,$ & $\,\,$ 1  $\,\,$ \\
\end{pmatrix},
\end{equation*}
\end{example}
\newpage
\begin{example}
\begin{equation*}
\mathbf{A} =
\begin{pmatrix}
$\,\,$ 1 $\,\,$ & $\,\,$5$\,\,$ & $\,\,$8$\,\,$ & $\,\,$8 $\,\,$ \\
$\,\,$ 1/5$\,\,$ & $\,\,$ 1 $\,\,$ & $\,\,$9$\,\,$ & $\,\,$3 $\,\,$ \\
$\,\,$ 1/8$\,\,$ & $\,\,$ 1/9$\,\,$ & $\,\,$ 1 $\,\,$ & $\,\,$ 1/2 $\,\,$ \\
$\,\,$ 1/8$\,\,$ & $\,\,$ 1/3$\,\,$ & $\,\,$2$\,\,$ & $\,\,$ 1  $\,\,$ \\
\end{pmatrix},
\qquad
\lambda_{\max} =
4.2637,
\qquad
CR = 0.0994
\end{equation*}

\begin{equation*}
\mathbf{w}^{EM} =
\begin{pmatrix}
0.648601\\
0.231457\\
0.044309\\
\color{red} 0.075634\color{black}
\end{pmatrix}\end{equation*}
\begin{equation*}
\left[ \frac{{w}^{EM}_i}{{w}^{EM}_j} \right] =
\begin{pmatrix}
$\,\,$ 1 $\,\,$ & $\,\,$2.8023$\,\,$ & $\,\,$14.6382$\,\,$ & $\,\,$\color{red} 8.5755\color{black} $\,\,$ \\
$\,\,$0.3569$\,\,$ & $\,\,$ 1 $\,\,$ & $\,\,$5.2237$\,\,$ & $\,\,$\color{red} 3.0602\color{black}   $\,\,$ \\
$\,\,$0.0683$\,\,$ & $\,\,$0.1914$\,\,$ & $\,\,$ 1 $\,\,$ & $\,\,$\color{red} 0.5858\color{black}  $\,\,$ \\
$\,\,$\color{red} 0.1166\color{black} $\,\,$ & $\,\,$\color{red} 0.3268\color{black} $\,\,$ & $\,\,$\color{red} 1.7070\color{black} $\,\,$ & $\,\,$ 1  $\,\,$ \\
\end{pmatrix},
\end{equation*}

\begin{equation*}
\mathbf{w}^{\prime} =
\begin{pmatrix}
0.647617\\
0.231106\\
0.044242\\
0.077035
\end{pmatrix} =
0.998484\cdot
\begin{pmatrix}
0.648601\\
0.231457\\
0.044309\\
\color{gr} 0.077152\color{black}
\end{pmatrix},
\end{equation*}
\begin{equation*}
\left[ \frac{{w}^{\prime}_i}{{w}^{\prime}_j} \right] =
\begin{pmatrix}
$\,\,$ 1 $\,\,$ & $\,\,$2.8023$\,\,$ & $\,\,$14.6382$\,\,$ & $\,\,$\color{gr} 8.4068\color{black} $\,\,$ \\
$\,\,$0.3569$\,\,$ & $\,\,$ 1 $\,\,$ & $\,\,$5.2237$\,\,$ & $\,\,$\color{gr} \color{blue} 3\color{black}   $\,\,$ \\
$\,\,$0.0683$\,\,$ & $\,\,$0.1914$\,\,$ & $\,\,$ 1 $\,\,$ & $\,\,$\color{gr} 0.5743\color{black}  $\,\,$ \\
$\,\,$\color{gr} 0.1190\color{black} $\,\,$ & $\,\,$\color{gr} \color{blue}  1/3\color{black} $\,\,$ & $\,\,$\color{gr} 1.7412\color{black} $\,\,$ & $\,\,$ 1  $\,\,$ \\
\end{pmatrix},
\end{equation*}
\end{example}
\newpage
\begin{example}
\begin{equation*}
\mathbf{A} =
\begin{pmatrix}
$\,\,$ 1 $\,\,$ & $\,\,$5$\,\,$ & $\,\,$9$\,\,$ & $\,\,$1 $\,\,$ \\
$\,\,$ 1/5$\,\,$ & $\,\,$ 1 $\,\,$ & $\,\,$3$\,\,$ & $\,\,$1 $\,\,$ \\
$\,\,$ 1/9$\,\,$ & $\,\,$ 1/3$\,\,$ & $\,\,$ 1 $\,\,$ & $\,\,$ 1/5 $\,\,$ \\
$\,\,$ 1 $\,\,$ & $\,\,$ 1 $\,\,$ & $\,\,$5$\,\,$ & $\,\,$ 1  $\,\,$ \\
\end{pmatrix},
\qquad
\lambda_{\max} =
4.2253,
\qquad
CR = 0.0849
\end{equation*}

\begin{equation*}
\mathbf{w}^{EM} =
\begin{pmatrix}
0.494139\\
0.167773\\
\color{red} 0.052097\color{black} \\
0.285991
\end{pmatrix}\end{equation*}
\begin{equation*}
\left[ \frac{{w}^{EM}_i}{{w}^{EM}_j} \right] =
\begin{pmatrix}
$\,\,$ 1 $\,\,$ & $\,\,$2.9453$\,\,$ & $\,\,$\color{red} 9.4850\color{black} $\,\,$ & $\,\,$1.7278$\,\,$ \\
$\,\,$0.3395$\,\,$ & $\,\,$ 1 $\,\,$ & $\,\,$\color{red} 3.2204\color{black} $\,\,$ & $\,\,$0.5866  $\,\,$ \\
$\,\,$\color{red} 0.1054\color{black} $\,\,$ & $\,\,$\color{red} 0.3105\color{black} $\,\,$ & $\,\,$ 1 $\,\,$ & $\,\,$\color{red} 0.1822\color{black}  $\,\,$ \\
$\,\,$0.5788$\,\,$ & $\,\,$1.7046$\,\,$ & $\,\,$\color{red} 5.4896\color{black} $\,\,$ & $\,\,$ 1  $\,\,$ \\
\end{pmatrix},
\end{equation*}

\begin{equation*}
\mathbf{w}^{\prime} =
\begin{pmatrix}
0.492756\\
0.167303\\
0.054751\\
0.285191
\end{pmatrix} =
0.997201\cdot
\begin{pmatrix}
0.494139\\
0.167773\\
\color{gr} 0.054904\color{black} \\
0.285991
\end{pmatrix},
\end{equation*}
\begin{equation*}
\left[ \frac{{w}^{\prime}_i}{{w}^{\prime}_j} \right] =
\begin{pmatrix}
$\,\,$ 1 $\,\,$ & $\,\,$2.9453$\,\,$ & $\,\,$\color{gr} \color{blue} 9\color{black} $\,\,$ & $\,\,$1.7278$\,\,$ \\
$\,\,$0.3395$\,\,$ & $\,\,$ 1 $\,\,$ & $\,\,$\color{gr} 3.0557\color{black} $\,\,$ & $\,\,$0.5866  $\,\,$ \\
$\,\,$\color{gr} \color{blue}  1/9\color{black} $\,\,$ & $\,\,$\color{gr} 0.3273\color{black} $\,\,$ & $\,\,$ 1 $\,\,$ & $\,\,$\color{gr} 0.1920\color{black}  $\,\,$ \\
$\,\,$0.5788$\,\,$ & $\,\,$1.7046$\,\,$ & $\,\,$\color{gr} 5.2089\color{black} $\,\,$ & $\,\,$ 1  $\,\,$ \\
\end{pmatrix},
\end{equation*}
\end{example}
\newpage
\begin{example}
\begin{equation*}
\mathbf{A} =
\begin{pmatrix}
$\,\,$ 1 $\,\,$ & $\,\,$5$\,\,$ & $\,\,$9$\,\,$ & $\,\,$3 $\,\,$ \\
$\,\,$ 1/5$\,\,$ & $\,\,$ 1 $\,\,$ & $\,\,$1$\,\,$ & $\,\,$1 $\,\,$ \\
$\,\,$ 1/9$\,\,$ & $\,\,$ 1 $\,\,$ & $\,\,$ 1 $\,\,$ & $\,\,$ 1/5 $\,\,$ \\
$\,\,$ 1/3$\,\,$ & $\,\,$ 1 $\,\,$ & $\,\,$5$\,\,$ & $\,\,$ 1  $\,\,$ \\
\end{pmatrix},
\qquad
\lambda_{\max} =
4.2253,
\qquad
CR = 0.0849
\end{equation*}

\begin{equation*}
\mathbf{w}^{EM} =
\begin{pmatrix}
\color{red} 0.591510\color{black} \\
0.124677\\
0.072159\\
0.211654
\end{pmatrix}\end{equation*}
\begin{equation*}
\left[ \frac{{w}^{EM}_i}{{w}^{EM}_j} \right] =
\begin{pmatrix}
$\,\,$ 1 $\,\,$ & $\,\,$\color{red} 4.7444\color{black} $\,\,$ & $\,\,$\color{red} 8.1973\color{black} $\,\,$ & $\,\,$\color{red} 2.7947\color{black} $\,\,$ \\
$\,\,$\color{red} 0.2108\color{black} $\,\,$ & $\,\,$ 1 $\,\,$ & $\,\,$1.7278$\,\,$ & $\,\,$0.5891  $\,\,$ \\
$\,\,$\color{red} 0.1220\color{black} $\,\,$ & $\,\,$0.5788$\,\,$ & $\,\,$ 1 $\,\,$ & $\,\,$0.3409 $\,\,$ \\
$\,\,$\color{red} 0.3578\color{black} $\,\,$ & $\,\,$1.6976$\,\,$ & $\,\,$2.9332$\,\,$ & $\,\,$ 1  $\,\,$ \\
\end{pmatrix},
\end{equation*}

\begin{equation*}
\mathbf{w}^{\prime} =
\begin{pmatrix}
0.604128\\
0.120826\\
0.069930\\
0.205117
\end{pmatrix} =
0.969111\cdot
\begin{pmatrix}
\color{gr} 0.623384\color{black} \\
0.124677\\
0.072159\\
0.211654
\end{pmatrix},
\end{equation*}
\begin{equation*}
\left[ \frac{{w}^{\prime}_i}{{w}^{\prime}_j} \right] =
\begin{pmatrix}
$\,\,$ 1 $\,\,$ & $\,\,$\color{gr} \color{blue} 5\color{black} $\,\,$ & $\,\,$\color{gr} 8.6391\color{black} $\,\,$ & $\,\,$\color{gr} 2.9453\color{black} $\,\,$ \\
$\,\,$\color{gr} \color{blue}  1/5\color{black} $\,\,$ & $\,\,$ 1 $\,\,$ & $\,\,$1.7278$\,\,$ & $\,\,$0.5891  $\,\,$ \\
$\,\,$\color{gr} 0.1158\color{black} $\,\,$ & $\,\,$0.5788$\,\,$ & $\,\,$ 1 $\,\,$ & $\,\,$0.3409 $\,\,$ \\
$\,\,$\color{gr} 0.3395\color{black} $\,\,$ & $\,\,$1.6976$\,\,$ & $\,\,$2.9332$\,\,$ & $\,\,$ 1  $\,\,$ \\
\end{pmatrix},
\end{equation*}
\end{example}
\newpage
\begin{example}
\begin{equation*}
\mathbf{A} =
\begin{pmatrix}
$\,\,$ 1 $\,\,$ & $\,\,$5$\,\,$ & $\,\,$9$\,\,$ & $\,\,$8 $\,\,$ \\
$\,\,$ 1/5$\,\,$ & $\,\,$ 1 $\,\,$ & $\,\,$9$\,\,$ & $\,\,$3 $\,\,$ \\
$\,\,$ 1/9$\,\,$ & $\,\,$ 1/9$\,\,$ & $\,\,$ 1 $\,\,$ & $\,\,$ 1/2 $\,\,$ \\
$\,\,$ 1/8$\,\,$ & $\,\,$ 1/3$\,\,$ & $\,\,$2$\,\,$ & $\,\,$ 1  $\,\,$ \\
\end{pmatrix},
\qquad
\lambda_{\max} =
4.2267,
\qquad
CR = 0.0855
\end{equation*}

\begin{equation*}
\mathbf{w}^{EM} =
\begin{pmatrix}
0.655533\\
0.227506\\
0.042019\\
\color{red} 0.074941\color{black}
\end{pmatrix}\end{equation*}
\begin{equation*}
\left[ \frac{{w}^{EM}_i}{{w}^{EM}_j} \right] =
\begin{pmatrix}
$\,\,$ 1 $\,\,$ & $\,\,$2.8814$\,\,$ & $\,\,$15.6007$\,\,$ & $\,\,$\color{red} 8.7473\color{black} $\,\,$ \\
$\,\,$0.3471$\,\,$ & $\,\,$ 1 $\,\,$ & $\,\,$5.4143$\,\,$ & $\,\,$\color{red} 3.0358\color{black}   $\,\,$ \\
$\,\,$0.0641$\,\,$ & $\,\,$0.1847$\,\,$ & $\,\,$ 1 $\,\,$ & $\,\,$\color{red} 0.5607\color{black}  $\,\,$ \\
$\,\,$\color{red} 0.1143\color{black} $\,\,$ & $\,\,$\color{red} 0.3294\color{black} $\,\,$ & $\,\,$\color{red} 1.7835\color{black} $\,\,$ & $\,\,$ 1  $\,\,$ \\
\end{pmatrix},
\end{equation*}

\begin{equation*}
\mathbf{w}^{\prime} =
\begin{pmatrix}
0.654947\\
0.227303\\
0.041982\\
0.075768
\end{pmatrix} =
0.999106\cdot
\begin{pmatrix}
0.655533\\
0.227506\\
0.042019\\
\color{gr} 0.075835\color{black}
\end{pmatrix},
\end{equation*}
\begin{equation*}
\left[ \frac{{w}^{\prime}_i}{{w}^{\prime}_j} \right] =
\begin{pmatrix}
$\,\,$ 1 $\,\,$ & $\,\,$2.8814$\,\,$ & $\,\,$15.6007$\,\,$ & $\,\,$\color{gr} 8.6441\color{black} $\,\,$ \\
$\,\,$0.3471$\,\,$ & $\,\,$ 1 $\,\,$ & $\,\,$5.4143$\,\,$ & $\,\,$\color{gr} \color{blue} 3\color{black}   $\,\,$ \\
$\,\,$0.0641$\,\,$ & $\,\,$0.1847$\,\,$ & $\,\,$ 1 $\,\,$ & $\,\,$\color{gr} 0.5541\color{black}  $\,\,$ \\
$\,\,$\color{gr} 0.1157\color{black} $\,\,$ & $\,\,$\color{gr} \color{blue}  1/3\color{black} $\,\,$ & $\,\,$\color{gr} 1.8048\color{black} $\,\,$ & $\,\,$ 1  $\,\,$ \\
\end{pmatrix},
\end{equation*}
\end{example}
\newpage
\begin{example}
\begin{equation*}
\mathbf{A} =
\begin{pmatrix}
$\,\,$ 1 $\,\,$ & $\,\,$5$\,\,$ & $\,\,$9$\,\,$ & $\,\,$9 $\,\,$ \\
$\,\,$ 1/5$\,\,$ & $\,\,$ 1 $\,\,$ & $\,\,$9$\,\,$ & $\,\,$3 $\,\,$ \\
$\,\,$ 1/9$\,\,$ & $\,\,$ 1/9$\,\,$ & $\,\,$ 1 $\,\,$ & $\,\,$ 1/2 $\,\,$ \\
$\,\,$ 1/9$\,\,$ & $\,\,$ 1/3$\,\,$ & $\,\,$2$\,\,$ & $\,\,$ 1  $\,\,$ \\
\end{pmatrix},
\qquad
\lambda_{\max} =
4.2277,
\qquad
CR = 0.0859
\end{equation*}

\begin{equation*}
\mathbf{w}^{EM} =
\begin{pmatrix}
0.662813\\
0.223825\\
0.041633\\
\color{red} 0.071729\color{black}
\end{pmatrix}\end{equation*}
\begin{equation*}
\left[ \frac{{w}^{EM}_i}{{w}^{EM}_j} \right] =
\begin{pmatrix}
$\,\,$ 1 $\,\,$ & $\,\,$2.9613$\,\,$ & $\,\,$15.9204$\,\,$ & $\,\,$\color{red} 9.2406\color{black} $\,\,$ \\
$\,\,$0.3377$\,\,$ & $\,\,$ 1 $\,\,$ & $\,\,$5.3761$\,\,$ & $\,\,$\color{red} 3.1204\color{black}   $\,\,$ \\
$\,\,$0.0628$\,\,$ & $\,\,$0.1860$\,\,$ & $\,\,$ 1 $\,\,$ & $\,\,$\color{red} 0.5804\color{black}  $\,\,$ \\
$\,\,$\color{red} 0.1082\color{black} $\,\,$ & $\,\,$\color{red} 0.3205\color{black} $\,\,$ & $\,\,$\color{red} 1.7229\color{black} $\,\,$ & $\,\,$ 1  $\,\,$ \\
\end{pmatrix},
\end{equation*}

\begin{equation*}
\mathbf{w}^{\prime} =
\begin{pmatrix}
0.661545\\
0.223397\\
0.041553\\
0.073505
\end{pmatrix} =
0.998086\cdot
\begin{pmatrix}
0.662813\\
0.223825\\
0.041633\\
\color{gr} 0.073646\color{black}
\end{pmatrix},
\end{equation*}
\begin{equation*}
\left[ \frac{{w}^{\prime}_i}{{w}^{\prime}_j} \right] =
\begin{pmatrix}
$\,\,$ 1 $\,\,$ & $\,\,$2.9613$\,\,$ & $\,\,$15.9204$\,\,$ & $\,\,$\color{gr} \color{blue} 9\color{black} $\,\,$ \\
$\,\,$0.3377$\,\,$ & $\,\,$ 1 $\,\,$ & $\,\,$5.3761$\,\,$ & $\,\,$\color{gr} 3.0392\color{black}   $\,\,$ \\
$\,\,$0.0628$\,\,$ & $\,\,$0.1860$\,\,$ & $\,\,$ 1 $\,\,$ & $\,\,$\color{gr} 0.5653\color{black}  $\,\,$ \\
$\,\,$\color{gr} \color{blue}  1/9\color{black} $\,\,$ & $\,\,$\color{gr} 0.3290\color{black} $\,\,$ & $\,\,$\color{gr} 1.7689\color{black} $\,\,$ & $\,\,$ 1  $\,\,$ \\
\end{pmatrix},
\end{equation*}
\end{example}
\newpage
\begin{example}
\begin{equation*}
\mathbf{A} =
\begin{pmatrix}
$\,\,$ 1 $\,\,$ & $\,\,$6$\,\,$ & $\,\,$8$\,\,$ & $\,\,$4 $\,\,$ \\
$\,\,$ 1/6$\,\,$ & $\,\,$ 1 $\,\,$ & $\,\,$1$\,\,$ & $\,\,$1 $\,\,$ \\
$\,\,$ 1/8$\,\,$ & $\,\,$ 1 $\,\,$ & $\,\,$ 1 $\,\,$ & $\,\,$ 1/3 $\,\,$ \\
$\,\,$ 1/4$\,\,$ & $\,\,$ 1 $\,\,$ & $\,\,$3$\,\,$ & $\,\,$ 1  $\,\,$ \\
\end{pmatrix},
\qquad
\lambda_{\max} =
4.1031,
\qquad
CR = 0.0389
\end{equation*}

\begin{equation*}
\mathbf{w}^{EM} =
\begin{pmatrix}
\color{red} 0.640317\color{black} \\
0.113670\\
0.080232\\
0.165782
\end{pmatrix}\end{equation*}
\begin{equation*}
\left[ \frac{{w}^{EM}_i}{{w}^{EM}_j} \right] =
\begin{pmatrix}
$\,\,$ 1 $\,\,$ & $\,\,$\color{red} 5.6331\color{black} $\,\,$ & $\,\,$\color{red} 7.9809\color{black} $\,\,$ & $\,\,$\color{red} 3.8624\color{black} $\,\,$ \\
$\,\,$\color{red} 0.1775\color{black} $\,\,$ & $\,\,$ 1 $\,\,$ & $\,\,$1.4168$\,\,$ & $\,\,$0.6857  $\,\,$ \\
$\,\,$\color{red} 0.1253\color{black} $\,\,$ & $\,\,$0.7058$\,\,$ & $\,\,$ 1 $\,\,$ & $\,\,$0.4840 $\,\,$ \\
$\,\,$\color{red} 0.2589\color{black} $\,\,$ & $\,\,$1.4585$\,\,$ & $\,\,$2.0663$\,\,$ & $\,\,$ 1  $\,\,$ \\
\end{pmatrix},
\end{equation*}

\begin{equation*}
\mathbf{w}^{\prime} =
\begin{pmatrix}
0.640869\\
0.113495\\
0.080109\\
0.165527
\end{pmatrix} =
0.998468\cdot
\begin{pmatrix}
\color{gr} 0.641852\color{black} \\
0.113670\\
0.080232\\
0.165782
\end{pmatrix},
\end{equation*}
\begin{equation*}
\left[ \frac{{w}^{\prime}_i}{{w}^{\prime}_j} \right] =
\begin{pmatrix}
$\,\,$ 1 $\,\,$ & $\,\,$\color{gr} 5.6467\color{black} $\,\,$ & $\,\,$\color{gr} \color{blue} 8\color{black} $\,\,$ & $\,\,$\color{gr} 3.8717\color{black} $\,\,$ \\
$\,\,$\color{gr} 0.1771\color{black} $\,\,$ & $\,\,$ 1 $\,\,$ & $\,\,$1.4168$\,\,$ & $\,\,$0.6857  $\,\,$ \\
$\,\,$\color{gr} \color{blue}  1/8\color{black} $\,\,$ & $\,\,$0.7058$\,\,$ & $\,\,$ 1 $\,\,$ & $\,\,$0.4840 $\,\,$ \\
$\,\,$\color{gr} 0.2583\color{black} $\,\,$ & $\,\,$1.4585$\,\,$ & $\,\,$2.0663$\,\,$ & $\,\,$ 1  $\,\,$ \\
\end{pmatrix},
\end{equation*}
\end{example}
\newpage
\begin{example}
\begin{equation*}
\mathbf{A} =
\begin{pmatrix}
$\,\,$ 1 $\,\,$ & $\,\,$6$\,\,$ & $\,\,$9$\,\,$ & $\,\,$3 $\,\,$ \\
$\,\,$ 1/6$\,\,$ & $\,\,$ 1 $\,\,$ & $\,\,$1$\,\,$ & $\,\,$1 $\,\,$ \\
$\,\,$ 1/9$\,\,$ & $\,\,$ 1 $\,\,$ & $\,\,$ 1 $\,\,$ & $\,\,$ 1/5 $\,\,$ \\
$\,\,$ 1/3$\,\,$ & $\,\,$ 1 $\,\,$ & $\,\,$5$\,\,$ & $\,\,$ 1  $\,\,$ \\
\end{pmatrix},
\qquad
\lambda_{\max} =
4.2277,
\qquad
CR = 0.0859
\end{equation*}

\begin{equation*}
\mathbf{w}^{EM} =
\begin{pmatrix}
\color{red} 0.605587\color{black} \\
0.117166\\
0.069989\\
0.207258
\end{pmatrix}\end{equation*}
\begin{equation*}
\left[ \frac{{w}^{EM}_i}{{w}^{EM}_j} \right] =
\begin{pmatrix}
$\,\,$ 1 $\,\,$ & $\,\,$\color{red} 5.1686\color{black} $\,\,$ & $\,\,$\color{red} 8.6526\color{black} $\,\,$ & $\,\,$\color{red} 2.9219\color{black} $\,\,$ \\
$\,\,$\color{red} 0.1935\color{black} $\,\,$ & $\,\,$ 1 $\,\,$ & $\,\,$1.6741$\,\,$ & $\,\,$0.5653  $\,\,$ \\
$\,\,$\color{red} 0.1156\color{black} $\,\,$ & $\,\,$0.5973$\,\,$ & $\,\,$ 1 $\,\,$ & $\,\,$0.3377 $\,\,$ \\
$\,\,$\color{red} 0.3422\color{black} $\,\,$ & $\,\,$1.7689$\,\,$ & $\,\,$2.9613$\,\,$ & $\,\,$ 1  $\,\,$ \\
\end{pmatrix},
\end{equation*}

\begin{equation*}
\mathbf{w}^{\prime} =
\begin{pmatrix}
0.611870\\
0.115299\\
0.068874\\
0.203957
\end{pmatrix} =
0.984070\cdot
\begin{pmatrix}
\color{gr} 0.621775\color{black} \\
0.117166\\
0.069989\\
0.207258
\end{pmatrix},
\end{equation*}
\begin{equation*}
\left[ \frac{{w}^{\prime}_i}{{w}^{\prime}_j} \right] =
\begin{pmatrix}
$\,\,$ 1 $\,\,$ & $\,\,$\color{gr} 5.3068\color{black} $\,\,$ & $\,\,$\color{gr} 8.8839\color{black} $\,\,$ & $\,\,$\color{gr} \color{blue} 3\color{black} $\,\,$ \\
$\,\,$\color{gr} 0.1884\color{black} $\,\,$ & $\,\,$ 1 $\,\,$ & $\,\,$1.6741$\,\,$ & $\,\,$0.5653  $\,\,$ \\
$\,\,$\color{gr} 0.1126\color{black} $\,\,$ & $\,\,$0.5973$\,\,$ & $\,\,$ 1 $\,\,$ & $\,\,$0.3377 $\,\,$ \\
$\,\,$\color{gr} \color{blue}  1/3\color{black} $\,\,$ & $\,\,$1.7689$\,\,$ & $\,\,$2.9613$\,\,$ & $\,\,$ 1  $\,\,$ \\
\end{pmatrix},
\end{equation*}
\end{example}
\newpage
\begin{example}
\begin{equation*}
\mathbf{A} =
\begin{pmatrix}
$\,\,$ 1 $\,\,$ & $\,\,$6$\,\,$ & $\,\,$9$\,\,$ & $\,\,$4 $\,\,$ \\
$\,\,$ 1/6$\,\,$ & $\,\,$ 1 $\,\,$ & $\,\,$1$\,\,$ & $\,\,$1 $\,\,$ \\
$\,\,$ 1/9$\,\,$ & $\,\,$ 1 $\,\,$ & $\,\,$ 1 $\,\,$ & $\,\,$ 1/3 $\,\,$ \\
$\,\,$ 1/4$\,\,$ & $\,\,$ 1 $\,\,$ & $\,\,$3$\,\,$ & $\,\,$ 1  $\,\,$ \\
\end{pmatrix},
\qquad
\lambda_{\max} =
4.1031,
\qquad
CR = 0.0389
\end{equation*}

\begin{equation*}
\mathbf{w}^{EM} =
\begin{pmatrix}
\color{red} 0.648698\color{black} \\
0.111967\\
0.076771\\
0.162563
\end{pmatrix}\end{equation*}
\begin{equation*}
\left[ \frac{{w}^{EM}_i}{{w}^{EM}_j} \right] =
\begin{pmatrix}
$\,\,$ 1 $\,\,$ & $\,\,$\color{red} 5.7936\color{black} $\,\,$ & $\,\,$\color{red} 8.4497\color{black} $\,\,$ & $\,\,$\color{red} 3.9904\color{black} $\,\,$ \\
$\,\,$\color{red} 0.1726\color{black} $\,\,$ & $\,\,$ 1 $\,\,$ & $\,\,$1.4585$\,\,$ & $\,\,$0.6888  $\,\,$ \\
$\,\,$\color{red} 0.1183\color{black} $\,\,$ & $\,\,$0.6857$\,\,$ & $\,\,$ 1 $\,\,$ & $\,\,$0.4723 $\,\,$ \\
$\,\,$\color{red} 0.2506\color{black} $\,\,$ & $\,\,$1.4519$\,\,$ & $\,\,$2.1175$\,\,$ & $\,\,$ 1  $\,\,$ \\
\end{pmatrix},
\end{equation*}

\begin{equation*}
\mathbf{w}^{\prime} =
\begin{pmatrix}
0.649243\\
0.111794\\
0.076652\\
0.162311
\end{pmatrix} =
0.998448\cdot
\begin{pmatrix}
\color{gr} 0.650253\color{black} \\
0.111967\\
0.076771\\
0.162563
\end{pmatrix},
\end{equation*}
\begin{equation*}
\left[ \frac{{w}^{\prime}_i}{{w}^{\prime}_j} \right] =
\begin{pmatrix}
$\,\,$ 1 $\,\,$ & $\,\,$\color{gr} 5.8075\color{black} $\,\,$ & $\,\,$\color{gr} 8.4700\color{black} $\,\,$ & $\,\,$\color{gr} \color{blue} 4\color{black} $\,\,$ \\
$\,\,$\color{gr} 0.1722\color{black} $\,\,$ & $\,\,$ 1 $\,\,$ & $\,\,$1.4585$\,\,$ & $\,\,$0.6888  $\,\,$ \\
$\,\,$\color{gr} 0.1181\color{black} $\,\,$ & $\,\,$0.6857$\,\,$ & $\,\,$ 1 $\,\,$ & $\,\,$0.4723 $\,\,$ \\
$\,\,$\color{gr} \color{blue}  1/4\color{black} $\,\,$ & $\,\,$1.4519$\,\,$ & $\,\,$2.1175$\,\,$ & $\,\,$ 1  $\,\,$ \\
\end{pmatrix},
\end{equation*}
\end{example}
\newpage
\begin{example}
\begin{equation*}
\mathbf{A} =
\begin{pmatrix}
$\,\,$ 1 $\,\,$ & $\,\,$6$\,\,$ & $\,\,$9$\,\,$ & $\,\,$4 $\,\,$ \\
$\,\,$ 1/6$\,\,$ & $\,\,$ 1 $\,\,$ & $\,\,$1$\,\,$ & $\,\,$1 $\,\,$ \\
$\,\,$ 1/9$\,\,$ & $\,\,$ 1 $\,\,$ & $\,\,$ 1 $\,\,$ & $\,\,$ 1/4 $\,\,$ \\
$\,\,$ 1/4$\,\,$ & $\,\,$ 1 $\,\,$ & $\,\,$4$\,\,$ & $\,\,$ 1  $\,\,$ \\
\end{pmatrix},
\qquad
\lambda_{\max} =
4.1664,
\qquad
CR = 0.0627
\end{equation*}

\begin{equation*}
\mathbf{w}^{EM} =
\begin{pmatrix}
\color{red} 0.639531\color{black} \\
0.112101\\
0.071787\\
0.176582
\end{pmatrix}\end{equation*}
\begin{equation*}
\left[ \frac{{w}^{EM}_i}{{w}^{EM}_j} \right] =
\begin{pmatrix}
$\,\,$ 1 $\,\,$ & $\,\,$\color{red} 5.7050\color{black} $\,\,$ & $\,\,$\color{red} 8.9088\color{black} $\,\,$ & $\,\,$\color{red} 3.6217\color{black} $\,\,$ \\
$\,\,$\color{red} 0.1753\color{black} $\,\,$ & $\,\,$ 1 $\,\,$ & $\,\,$1.5616$\,\,$ & $\,\,$0.6348  $\,\,$ \\
$\,\,$\color{red} 0.1122\color{black} $\,\,$ & $\,\,$0.6404$\,\,$ & $\,\,$ 1 $\,\,$ & $\,\,$0.4065 $\,\,$ \\
$\,\,$\color{red} 0.2761\color{black} $\,\,$ & $\,\,$1.5752$\,\,$ & $\,\,$2.4598$\,\,$ & $\,\,$ 1  $\,\,$ \\
\end{pmatrix},
\end{equation*}

\begin{equation*}
\mathbf{w}^{\prime} =
\begin{pmatrix}
0.641876\\
0.111372\\
0.071320\\
0.175433
\end{pmatrix} =
0.993494\cdot
\begin{pmatrix}
\color{gr} 0.646079\color{black} \\
0.112101\\
0.071787\\
0.176582
\end{pmatrix},
\end{equation*}
\begin{equation*}
\left[ \frac{{w}^{\prime}_i}{{w}^{\prime}_j} \right] =
\begin{pmatrix}
$\,\,$ 1 $\,\,$ & $\,\,$\color{gr} 5.7634\color{black} $\,\,$ & $\,\,$\color{gr} \color{blue} 9\color{black} $\,\,$ & $\,\,$\color{gr} 3.6588\color{black} $\,\,$ \\
$\,\,$\color{gr} 0.1735\color{black} $\,\,$ & $\,\,$ 1 $\,\,$ & $\,\,$1.5616$\,\,$ & $\,\,$0.6348  $\,\,$ \\
$\,\,$\color{gr} \color{blue}  1/9\color{black} $\,\,$ & $\,\,$0.6404$\,\,$ & $\,\,$ 1 $\,\,$ & $\,\,$0.4065 $\,\,$ \\
$\,\,$\color{gr} 0.2733\color{black} $\,\,$ & $\,\,$1.5752$\,\,$ & $\,\,$2.4598$\,\,$ & $\,\,$ 1  $\,\,$ \\
\end{pmatrix},
\end{equation*}
\end{example}
\newpage
\begin{example}
\begin{equation*}
\mathbf{A} =
\begin{pmatrix}
$\,\,$ 1 $\,\,$ & $\,\,$6$\,\,$ & $\,\,$9$\,\,$ & $\,\,$7 $\,\,$ \\
$\,\,$ 1/6$\,\,$ & $\,\,$ 1 $\,\,$ & $\,\,$1$\,\,$ & $\,\,$2 $\,\,$ \\
$\,\,$ 1/9$\,\,$ & $\,\,$ 1 $\,\,$ & $\,\,$ 1 $\,\,$ & $\,\,$ 1/2 $\,\,$ \\
$\,\,$ 1/7$\,\,$ & $\,\,$ 1/2$\,\,$ & $\,\,$2$\,\,$ & $\,\,$ 1  $\,\,$ \\
\end{pmatrix},
\qquad
\lambda_{\max} =
4.1658,
\qquad
CR = 0.0625
\end{equation*}

\begin{equation*}
\mathbf{w}^{EM} =
\begin{pmatrix}
\color{red} 0.691959\color{black} \\
0.126065\\
0.080184\\
0.101792
\end{pmatrix}\end{equation*}
\begin{equation*}
\left[ \frac{{w}^{EM}_i}{{w}^{EM}_j} \right] =
\begin{pmatrix}
$\,\,$ 1 $\,\,$ & $\,\,$\color{red} 5.4889\color{black} $\,\,$ & $\,\,$\color{red} 8.6296\color{black} $\,\,$ & $\,\,$\color{red} 6.7978\color{black} $\,\,$ \\
$\,\,$\color{red} 0.1822\color{black} $\,\,$ & $\,\,$ 1 $\,\,$ & $\,\,$1.5722$\,\,$ & $\,\,$1.2385  $\,\,$ \\
$\,\,$\color{red} 0.1159\color{black} $\,\,$ & $\,\,$0.6361$\,\,$ & $\,\,$ 1 $\,\,$ & $\,\,$0.7877 $\,\,$ \\
$\,\,$\color{red} 0.1471\color{black} $\,\,$ & $\,\,$0.8075$\,\,$ & $\,\,$1.2695$\,\,$ & $\,\,$ 1  $\,\,$ \\
\end{pmatrix},
\end{equation*}

\begin{equation*}
\mathbf{w}^{\prime} =
\begin{pmatrix}
0.698172\\
0.123522\\
0.078567\\
0.099739
\end{pmatrix} =
0.979829\cdot
\begin{pmatrix}
\color{gr} 0.712545\color{black} \\
0.126065\\
0.080184\\
0.101792
\end{pmatrix},
\end{equation*}
\begin{equation*}
\left[ \frac{{w}^{\prime}_i}{{w}^{\prime}_j} \right] =
\begin{pmatrix}
$\,\,$ 1 $\,\,$ & $\,\,$\color{gr} 5.6522\color{black} $\,\,$ & $\,\,$\color{gr} 8.8864\color{black} $\,\,$ & $\,\,$\color{gr} \color{blue} 7\color{black} $\,\,$ \\
$\,\,$\color{gr} 0.1769\color{black} $\,\,$ & $\,\,$ 1 $\,\,$ & $\,\,$1.5722$\,\,$ & $\,\,$1.2385  $\,\,$ \\
$\,\,$\color{gr} 0.1125\color{black} $\,\,$ & $\,\,$0.6361$\,\,$ & $\,\,$ 1 $\,\,$ & $\,\,$0.7877 $\,\,$ \\
$\,\,$\color{gr} \color{blue}  1/7\color{black} $\,\,$ & $\,\,$0.8075$\,\,$ & $\,\,$1.2695$\,\,$ & $\,\,$ 1  $\,\,$ \\
\end{pmatrix},
\end{equation*}
\end{example}
\newpage
\begin{example}
\begin{equation*}
\mathbf{A} =
\begin{pmatrix}
$\,\,$ 1 $\,\,$ & $\,\,$6$\,\,$ & $\,\,$9$\,\,$ & $\,\,$8 $\,\,$ \\
$\,\,$ 1/6$\,\,$ & $\,\,$ 1 $\,\,$ & $\,\,$1$\,\,$ & $\,\,$2 $\,\,$ \\
$\,\,$ 1/9$\,\,$ & $\,\,$ 1 $\,\,$ & $\,\,$ 1 $\,\,$ & $\,\,$ 1/2 $\,\,$ \\
$\,\,$ 1/8$\,\,$ & $\,\,$ 1/2$\,\,$ & $\,\,$2$\,\,$ & $\,\,$ 1  $\,\,$ \\
\end{pmatrix},
\qquad
\lambda_{\max} =
4.1664,
\qquad
CR = 0.0627
\end{equation*}

\begin{equation*}
\mathbf{w}^{EM} =
\begin{pmatrix}
\color{red} 0.701463\color{black} \\
0.122957\\
0.078738\\
0.096841
\end{pmatrix}\end{equation*}
\begin{equation*}
\left[ \frac{{w}^{EM}_i}{{w}^{EM}_j} \right] =
\begin{pmatrix}
$\,\,$ 1 $\,\,$ & $\,\,$\color{red} 5.7050\color{black} $\,\,$ & $\,\,$\color{red} 8.9088\color{black} $\,\,$ & $\,\,$\color{red} 7.2434\color{black} $\,\,$ \\
$\,\,$\color{red} 0.1753\color{black} $\,\,$ & $\,\,$ 1 $\,\,$ & $\,\,$1.5616$\,\,$ & $\,\,$1.2697  $\,\,$ \\
$\,\,$\color{red} 0.1122\color{black} $\,\,$ & $\,\,$0.6404$\,\,$ & $\,\,$ 1 $\,\,$ & $\,\,$0.8131 $\,\,$ \\
$\,\,$\color{red} 0.1381\color{black} $\,\,$ & $\,\,$0.7876$\,\,$ & $\,\,$1.2299$\,\,$ & $\,\,$ 1  $\,\,$ \\
\end{pmatrix},
\end{equation*}

\begin{equation*}
\mathbf{w}^{\prime} =
\begin{pmatrix}
0.703592\\
0.122080\\
0.078177\\
0.096151
\end{pmatrix} =
0.992868\cdot
\begin{pmatrix}
\color{gr} 0.708646\color{black} \\
0.122957\\
0.078738\\
0.096841
\end{pmatrix},
\end{equation*}
\begin{equation*}
\left[ \frac{{w}^{\prime}_i}{{w}^{\prime}_j} \right] =
\begin{pmatrix}
$\,\,$ 1 $\,\,$ & $\,\,$\color{gr} 5.7634\color{black} $\,\,$ & $\,\,$\color{gr} \color{blue} 9\color{black} $\,\,$ & $\,\,$\color{gr} 7.3176\color{black} $\,\,$ \\
$\,\,$\color{gr} 0.1735\color{black} $\,\,$ & $\,\,$ 1 $\,\,$ & $\,\,$1.5616$\,\,$ & $\,\,$1.2697  $\,\,$ \\
$\,\,$\color{gr} \color{blue}  1/9\color{black} $\,\,$ & $\,\,$0.6404$\,\,$ & $\,\,$ 1 $\,\,$ & $\,\,$0.8131 $\,\,$ \\
$\,\,$\color{gr} 0.1367\color{black} $\,\,$ & $\,\,$0.7876$\,\,$ & $\,\,$1.2299$\,\,$ & $\,\,$ 1  $\,\,$ \\
\end{pmatrix},
\end{equation*}
\end{example}

\newpage
\section{Inef{\kern0pt}f{\kern0pt}icient $AMAST$ (spanning trees, arithmetic mean)
weight vector}
\begin{example}
\begin{equation*}
\mathbf{A} =
\begin{pmatrix}
$\,\,$ 1 $\,\,$ & $\,\,$1$\,\,$ & $\,\,$6$\,\,$ & $\,\,$2 $\,\,$ \\
$\,\,$ 1 $\,\,$ & $\,\,$ 1 $\,\,$ & $\,\,$4$\,\,$ & $\,\,$3 $\,\,$ \\
$\,\,$ 1/6$\,\,$ & $\,\,$ 1/4$\,\,$ & $\,\,$ 1 $\,\,$ & $\,\,$1 $\,\,$ \\
$\,\,$ 1/2$\,\,$ & $\,\,$ 1/3$\,\,$ & $\,\,$ 1 $\,\,$ & $\,\,$ 1  $\,\,$ \\
\end{pmatrix},
\qquad
\lambda_{\max} =
4.1031,
\qquad
CR = 0.0389
\end{equation*}

\begin{equation*}
\mathbf{w}^{AMAST} =
\begin{pmatrix}
0.385684\\
\color{red} 0.382840\color{black} \\
0.096144\\
0.135332
\end{pmatrix}\end{equation*}
\begin{equation*}
\left[ \frac{{w}^{AMAST}_i}{{w}^{AMAST}_j} \right] =
\begin{pmatrix}
$\,\,$ 1 $\,\,$ & $\,\,$\color{red} 1.0074\color{black} $\,\,$ & $\,\,$4.0115$\,\,$ & $\,\,$2.8499$\,\,$ \\
$\,\,$\color{red} 0.9926\color{black} $\,\,$ & $\,\,$ 1 $\,\,$ & $\,\,$\color{red} 3.9820\color{black} $\,\,$ & $\,\,$\color{red} 2.8289\color{black}   $\,\,$ \\
$\,\,$0.2493$\,\,$ & $\,\,$\color{red} 0.2511\color{black} $\,\,$ & $\,\,$ 1 $\,\,$ & $\,\,$0.7104 $\,\,$ \\
$\,\,$0.3509$\,\,$ & $\,\,$\color{red} 0.3535\color{black} $\,\,$ & $\,\,$1.4076$\,\,$ & $\,\,$ 1  $\,\,$ \\
\end{pmatrix},
\end{equation*}

\begin{equation*}
\mathbf{w}^{\prime} =
\begin{pmatrix}
0.385017\\
0.383908\\
0.095977\\
0.135098
\end{pmatrix} =
0.998269\cdot
\begin{pmatrix}
0.385684\\
\color{gr} 0.384574\color{black} \\
0.096144\\
0.135332
\end{pmatrix},
\end{equation*}
\begin{equation*}
\left[ \frac{{w}^{\prime}_i}{{w}^{\prime}_j} \right] =
\begin{pmatrix}
$\,\,$ 1 $\,\,$ & $\,\,$\color{gr} 1.0029\color{black} $\,\,$ & $\,\,$4.0115$\,\,$ & $\,\,$2.8499$\,\,$ \\
$\,\,$\color{gr} 0.9971\color{black} $\,\,$ & $\,\,$ 1 $\,\,$ & $\,\,$\color{gr} \color{blue} 4\color{black} $\,\,$ & $\,\,$\color{gr} 2.8417\color{black}   $\,\,$ \\
$\,\,$0.2493$\,\,$ & $\,\,$\color{gr} \color{blue}  1/4\color{black} $\,\,$ & $\,\,$ 1 $\,\,$ & $\,\,$0.7104 $\,\,$ \\
$\,\,$0.3509$\,\,$ & $\,\,$\color{gr} 0.3519\color{black} $\,\,$ & $\,\,$1.4076$\,\,$ & $\,\,$ 1  $\,\,$ \\
\end{pmatrix},
\end{equation*}
\end{example}
\newpage
\begin{example}
\begin{equation*}
\mathbf{A} =
\begin{pmatrix}
$\,\,$ 1 $\,\,$ & $\,\,$1$\,\,$ & $\,\,$7$\,\,$ & $\,\,$3 $\,\,$ \\
$\,\,$ 1 $\,\,$ & $\,\,$ 1 $\,\,$ & $\,\,$4$\,\,$ & $\,\,$5 $\,\,$ \\
$\,\,$ 1/7$\,\,$ & $\,\,$ 1/4$\,\,$ & $\,\,$ 1 $\,\,$ & $\,\,$2 $\,\,$ \\
$\,\,$ 1/3$\,\,$ & $\,\,$ 1/5$\,\,$ & $\,\,$ 1/2$\,\,$ & $\,\,$ 1  $\,\,$ \\
\end{pmatrix},
\qquad
\lambda_{\max} =
4.2057,
\qquad
CR = 0.0776
\end{equation*}

\begin{equation*}
\mathbf{w}^{AMAST} =
\begin{pmatrix}
0.409127\\
\color{red} 0.399967\color{black} \\
0.104638\\
0.086267
\end{pmatrix}\end{equation*}
\begin{equation*}
\left[ \frac{{w}^{AMAST}_i}{{w}^{AMAST}_j} \right] =
\begin{pmatrix}
$\,\,$ 1 $\,\,$ & $\,\,$\color{red} 1.0229\color{black} $\,\,$ & $\,\,$3.9099$\,\,$ & $\,\,$4.7426$\,\,$ \\
$\,\,$\color{red} 0.9776\color{black} $\,\,$ & $\,\,$ 1 $\,\,$ & $\,\,$\color{red} 3.8224\color{black} $\,\,$ & $\,\,$\color{red} 4.6364\color{black}   $\,\,$ \\
$\,\,$0.2558$\,\,$ & $\,\,$\color{red} 0.2616\color{black} $\,\,$ & $\,\,$ 1 $\,\,$ & $\,\,$1.2130 $\,\,$ \\
$\,\,$0.2109$\,\,$ & $\,\,$\color{red} 0.2157\color{black} $\,\,$ & $\,\,$0.8244$\,\,$ & $\,\,$ 1  $\,\,$ \\
\end{pmatrix},
\end{equation*}

\begin{equation*}
\mathbf{w}^{\prime} =
\begin{pmatrix}
0.405414\\
0.405414\\
0.103689\\
0.085484
\end{pmatrix} =
0.990923\cdot
\begin{pmatrix}
0.409127\\
\color{gr} 0.409127\color{black} \\
0.104638\\
0.086267
\end{pmatrix},
\end{equation*}
\begin{equation*}
\left[ \frac{{w}^{\prime}_i}{{w}^{\prime}_j} \right] =
\begin{pmatrix}
$\,\,$ 1 $\,\,$ & $\,\,$\color{gr} \color{blue} 1\color{black} $\,\,$ & $\,\,$3.9099$\,\,$ & $\,\,$4.7426$\,\,$ \\
$\,\,$\color{gr} \color{blue} 1\color{black} $\,\,$ & $\,\,$ 1 $\,\,$ & $\,\,$\color{gr} 3.9099\color{black} $\,\,$ & $\,\,$\color{gr} 4.7426\color{black}   $\,\,$ \\
$\,\,$0.2558$\,\,$ & $\,\,$\color{gr} 0.2558\color{black} $\,\,$ & $\,\,$ 1 $\,\,$ & $\,\,$1.2130 $\,\,$ \\
$\,\,$0.2109$\,\,$ & $\,\,$\color{gr} 0.2109\color{black} $\,\,$ & $\,\,$0.8244$\,\,$ & $\,\,$ 1  $\,\,$ \\
\end{pmatrix},
\end{equation*}
\end{example}
\newpage
\begin{example}
\begin{equation*}
\mathbf{A} =
\begin{pmatrix}
$\,\,$ 1 $\,\,$ & $\,\,$1$\,\,$ & $\,\,$8$\,\,$ & $\,\,$2 $\,\,$ \\
$\,\,$ 1 $\,\,$ & $\,\,$ 1 $\,\,$ & $\,\,$5$\,\,$ & $\,\,$3 $\,\,$ \\
$\,\,$ 1/8$\,\,$ & $\,\,$ 1/5$\,\,$ & $\,\,$ 1 $\,\,$ & $\,\,$1 $\,\,$ \\
$\,\,$ 1/2$\,\,$ & $\,\,$ 1/3$\,\,$ & $\,\,$ 1 $\,\,$ & $\,\,$ 1  $\,\,$ \\
\end{pmatrix},
\qquad
\lambda_{\max} =
4.1655,
\qquad
CR = 0.0624
\end{equation*}

\begin{equation*}
\mathbf{w}^{AMAST} =
\begin{pmatrix}
0.397316\\
\color{red} 0.387624\color{black} \\
0.083118\\
0.131942
\end{pmatrix}\end{equation*}
\begin{equation*}
\left[ \frac{{w}^{AMAST}_i}{{w}^{AMAST}_j} \right] =
\begin{pmatrix}
$\,\,$ 1 $\,\,$ & $\,\,$\color{red} 1.0250\color{black} $\,\,$ & $\,\,$4.7801$\,\,$ & $\,\,$3.0113$\,\,$ \\
$\,\,$\color{red} 0.9756\color{black} $\,\,$ & $\,\,$ 1 $\,\,$ & $\,\,$\color{red} 4.6635\color{black} $\,\,$ & $\,\,$\color{red} 2.9378\color{black}   $\,\,$ \\
$\,\,$0.2092$\,\,$ & $\,\,$\color{red} 0.2144\color{black} $\,\,$ & $\,\,$ 1 $\,\,$ & $\,\,$0.6300 $\,\,$ \\
$\,\,$0.3321$\,\,$ & $\,\,$\color{red} 0.3404\color{black} $\,\,$ & $\,\,$1.5874$\,\,$ & $\,\,$ 1  $\,\,$ \\
\end{pmatrix},
\end{equation*}

\begin{equation*}
\mathbf{w}^{\prime} =
\begin{pmatrix}
0.394084\\
0.392605\\
0.082442\\
0.130868
\end{pmatrix} =
0.991866\cdot
\begin{pmatrix}
0.397316\\
\color{gr} 0.395825\color{black} \\
0.083118\\
0.131942
\end{pmatrix},
\end{equation*}
\begin{equation*}
\left[ \frac{{w}^{\prime}_i}{{w}^{\prime}_j} \right] =
\begin{pmatrix}
$\,\,$ 1 $\,\,$ & $\,\,$\color{gr} 1.0038\color{black} $\,\,$ & $\,\,$4.7801$\,\,$ & $\,\,$3.0113$\,\,$ \\
$\,\,$\color{gr} 0.9962\color{black} $\,\,$ & $\,\,$ 1 $\,\,$ & $\,\,$\color{gr} 4.7622\color{black} $\,\,$ & $\,\,$\color{gr} \color{blue} 3\color{black}   $\,\,$ \\
$\,\,$0.2092$\,\,$ & $\,\,$\color{gr} 0.2100\color{black} $\,\,$ & $\,\,$ 1 $\,\,$ & $\,\,$0.6300 $\,\,$ \\
$\,\,$0.3321$\,\,$ & $\,\,$\color{gr} \color{blue}  1/3\color{black} $\,\,$ & $\,\,$1.5874$\,\,$ & $\,\,$ 1  $\,\,$ \\
\end{pmatrix},
\end{equation*}
\end{example}
\newpage
\begin{example}
\begin{equation*}
\mathbf{A} =
\begin{pmatrix}
$\,\,$ 1 $\,\,$ & $\,\,$1$\,\,$ & $\,\,$8$\,\,$ & $\,\,$3 $\,\,$ \\
$\,\,$ 1 $\,\,$ & $\,\,$ 1 $\,\,$ & $\,\,$4$\,\,$ & $\,\,$5 $\,\,$ \\
$\,\,$ 1/8$\,\,$ & $\,\,$ 1/4$\,\,$ & $\,\,$ 1 $\,\,$ & $\,\,$2 $\,\,$ \\
$\,\,$ 1/3$\,\,$ & $\,\,$ 1/5$\,\,$ & $\,\,$ 1/2$\,\,$ & $\,\,$ 1  $\,\,$ \\
\end{pmatrix},
\qquad
\lambda_{\max} =
4.2460,
\qquad
CR = 0.0928
\end{equation*}

\begin{equation*}
\mathbf{w}^{AMAST} =
\begin{pmatrix}
0.417350\\
\color{red} 0.395325\color{black} \\
0.101389\\
0.085936
\end{pmatrix}\end{equation*}
\begin{equation*}
\left[ \frac{{w}^{AMAST}_i}{{w}^{AMAST}_j} \right] =
\begin{pmatrix}
$\,\,$ 1 $\,\,$ & $\,\,$\color{red} 1.0557\color{black} $\,\,$ & $\,\,$4.1163$\,\,$ & $\,\,$4.8565$\,\,$ \\
$\,\,$\color{red} 0.9472\color{black} $\,\,$ & $\,\,$ 1 $\,\,$ & $\,\,$\color{red} 3.8991\color{black} $\,\,$ & $\,\,$\color{red} 4.6003\color{black}   $\,\,$ \\
$\,\,$0.2429$\,\,$ & $\,\,$\color{red} 0.2565\color{black} $\,\,$ & $\,\,$ 1 $\,\,$ & $\,\,$1.1798 $\,\,$ \\
$\,\,$0.2059$\,\,$ & $\,\,$\color{red} 0.2174\color{black} $\,\,$ & $\,\,$0.8476$\,\,$ & $\,\,$ 1  $\,\,$ \\
\end{pmatrix},
\end{equation*}

\begin{equation*}
\mathbf{w}^{\prime} =
\begin{pmatrix}
0.413123\\
0.401450\\
0.100362\\
0.085065
\end{pmatrix} =
0.989872\cdot
\begin{pmatrix}
0.417350\\
\color{gr} 0.405557\color{black} \\
0.101389\\
0.085936
\end{pmatrix},
\end{equation*}
\begin{equation*}
\left[ \frac{{w}^{\prime}_i}{{w}^{\prime}_j} \right] =
\begin{pmatrix}
$\,\,$ 1 $\,\,$ & $\,\,$\color{gr} 1.0291\color{black} $\,\,$ & $\,\,$4.1163$\,\,$ & $\,\,$4.8565$\,\,$ \\
$\,\,$\color{gr} 0.9717\color{black} $\,\,$ & $\,\,$ 1 $\,\,$ & $\,\,$\color{gr} \color{blue} 4\color{black} $\,\,$ & $\,\,$\color{gr} 4.7193\color{black}   $\,\,$ \\
$\,\,$0.2429$\,\,$ & $\,\,$\color{gr} \color{blue}  1/4\color{black} $\,\,$ & $\,\,$ 1 $\,\,$ & $\,\,$1.1798 $\,\,$ \\
$\,\,$0.2059$\,\,$ & $\,\,$\color{gr} 0.2119\color{black} $\,\,$ & $\,\,$0.8476$\,\,$ & $\,\,$ 1  $\,\,$ \\
\end{pmatrix},
\end{equation*}
\end{example}
\newpage
\begin{example}
\begin{equation*}
\mathbf{A} =
\begin{pmatrix}
$\,\,$ 1 $\,\,$ & $\,\,$1$\,\,$ & $\,\,$8$\,\,$ & $\,\,$3 $\,\,$ \\
$\,\,$ 1 $\,\,$ & $\,\,$ 1 $\,\,$ & $\,\,$5$\,\,$ & $\,\,$5 $\,\,$ \\
$\,\,$ 1/8$\,\,$ & $\,\,$ 1/5$\,\,$ & $\,\,$ 1 $\,\,$ & $\,\,$2 $\,\,$ \\
$\,\,$ 1/3$\,\,$ & $\,\,$ 1/5$\,\,$ & $\,\,$ 1/2$\,\,$ & $\,\,$ 1  $\,\,$ \\
\end{pmatrix},
\qquad
\lambda_{\max} =
4.2460,
\qquad
CR = 0.0928
\end{equation*}

\begin{equation*}
\mathbf{w}^{AMAST} =
\begin{pmatrix}
0.410632\\
\color{red} 0.409846\color{black} \\
0.094553\\
0.084969
\end{pmatrix}\end{equation*}
\begin{equation*}
\left[ \frac{{w}^{AMAST}_i}{{w}^{AMAST}_j} \right] =
\begin{pmatrix}
$\,\,$ 1 $\,\,$ & $\,\,$\color{red} 1.0019\color{black} $\,\,$ & $\,\,$4.3429$\,\,$ & $\,\,$4.8327$\,\,$ \\
$\,\,$\color{red} 0.9981\color{black} $\,\,$ & $\,\,$ 1 $\,\,$ & $\,\,$\color{red} 4.3346\color{black} $\,\,$ & $\,\,$\color{red} 4.8235\color{black}   $\,\,$ \\
$\,\,$0.2303$\,\,$ & $\,\,$\color{red} 0.2307\color{black} $\,\,$ & $\,\,$ 1 $\,\,$ & $\,\,$1.1128 $\,\,$ \\
$\,\,$0.2069$\,\,$ & $\,\,$\color{red} 0.2073\color{black} $\,\,$ & $\,\,$0.8986$\,\,$ & $\,\,$ 1  $\,\,$ \\
\end{pmatrix},
\end{equation*}

\begin{equation*}
\mathbf{w}^{\prime} =
\begin{pmatrix}
0.410310\\
0.410310\\
0.094478\\
0.084902
\end{pmatrix} =
0.999214\cdot
\begin{pmatrix}
0.410632\\
\color{gr} 0.410632\color{black} \\
0.094553\\
0.084969
\end{pmatrix},
\end{equation*}
\begin{equation*}
\left[ \frac{{w}^{\prime}_i}{{w}^{\prime}_j} \right] =
\begin{pmatrix}
$\,\,$ 1 $\,\,$ & $\,\,$\color{gr} \color{blue} 1\color{black} $\,\,$ & $\,\,$4.3429$\,\,$ & $\,\,$4.8327$\,\,$ \\
$\,\,$\color{gr} \color{blue} 1\color{black} $\,\,$ & $\,\,$ 1 $\,\,$ & $\,\,$\color{gr} 4.3429\color{black} $\,\,$ & $\,\,$\color{gr} 4.8327\color{black}   $\,\,$ \\
$\,\,$0.2303$\,\,$ & $\,\,$\color{gr} 0.2303\color{black} $\,\,$ & $\,\,$ 1 $\,\,$ & $\,\,$1.1128 $\,\,$ \\
$\,\,$0.2069$\,\,$ & $\,\,$\color{gr} 0.2069\color{black} $\,\,$ & $\,\,$0.8986$\,\,$ & $\,\,$ 1  $\,\,$ \\
\end{pmatrix},
\end{equation*}
\end{example}
\newpage
\begin{example}
\begin{equation*}
\mathbf{A} =
\begin{pmatrix}
$\,\,$ 1 $\,\,$ & $\,\,$1$\,\,$ & $\,\,$8$\,\,$ & $\,\,$4 $\,\,$ \\
$\,\,$ 1 $\,\,$ & $\,\,$ 1 $\,\,$ & $\,\,$5$\,\,$ & $\,\,$6 $\,\,$ \\
$\,\,$ 1/8$\,\,$ & $\,\,$ 1/5$\,\,$ & $\,\,$ 1 $\,\,$ & $\,\,$2 $\,\,$ \\
$\,\,$ 1/4$\,\,$ & $\,\,$ 1/6$\,\,$ & $\,\,$ 1/2$\,\,$ & $\,\,$ 1  $\,\,$ \\
\end{pmatrix},
\qquad
\lambda_{\max} =
4.1655,
\qquad
CR = 0.0624
\end{equation*}

\begin{equation*}
\mathbf{w}^{AMAST} =
\begin{pmatrix}
0.424588\\
\color{red} 0.415087\color{black} \\
0.089049\\
0.071276
\end{pmatrix}\end{equation*}
\begin{equation*}
\left[ \frac{{w}^{AMAST}_i}{{w}^{AMAST}_j} \right] =
\begin{pmatrix}
$\,\,$ 1 $\,\,$ & $\,\,$\color{red} 1.0229\color{black} $\,\,$ & $\,\,$4.7680$\,\,$ & $\,\,$5.9570$\,\,$ \\
$\,\,$\color{red} 0.9776\color{black} $\,\,$ & $\,\,$ 1 $\,\,$ & $\,\,$\color{red} 4.6613\color{black} $\,\,$ & $\,\,$\color{red} 5.8237\color{black}   $\,\,$ \\
$\,\,$0.2097$\,\,$ & $\,\,$\color{red} 0.2145\color{black} $\,\,$ & $\,\,$ 1 $\,\,$ & $\,\,$1.2494 $\,\,$ \\
$\,\,$0.1679$\,\,$ & $\,\,$\color{red} 0.1717\color{black} $\,\,$ & $\,\,$0.8004$\,\,$ & $\,\,$ 1  $\,\,$ \\
\end{pmatrix},
\end{equation*}

\begin{equation*}
\mathbf{w}^{\prime} =
\begin{pmatrix}
0.420592\\
0.420592\\
0.088211\\
0.070605
\end{pmatrix} =
0.990589\cdot
\begin{pmatrix}
0.424588\\
\color{gr} 0.424588\color{black} \\
0.089049\\
0.071276
\end{pmatrix},
\end{equation*}
\begin{equation*}
\left[ \frac{{w}^{\prime}_i}{{w}^{\prime}_j} \right] =
\begin{pmatrix}
$\,\,$ 1 $\,\,$ & $\,\,$\color{gr} \color{blue} 1\color{black} $\,\,$ & $\,\,$4.7680$\,\,$ & $\,\,$5.9570$\,\,$ \\
$\,\,$\color{gr} \color{blue} 1\color{black} $\,\,$ & $\,\,$ 1 $\,\,$ & $\,\,$\color{gr} 4.7680\color{black} $\,\,$ & $\,\,$\color{gr} 5.9570\color{black}   $\,\,$ \\
$\,\,$0.2097$\,\,$ & $\,\,$\color{gr} 0.2097\color{black} $\,\,$ & $\,\,$ 1 $\,\,$ & $\,\,$1.2494 $\,\,$ \\
$\,\,$0.1679$\,\,$ & $\,\,$\color{gr} 0.1679\color{black} $\,\,$ & $\,\,$0.8004$\,\,$ & $\,\,$ 1  $\,\,$ \\
\end{pmatrix},
\end{equation*}
\end{example}
\newpage
\begin{example}
\begin{equation*}
\mathbf{A} =
\begin{pmatrix}
$\,\,$ 1 $\,\,$ & $\,\,$1$\,\,$ & $\,\,$9$\,\,$ & $\,\,$2 $\,\,$ \\
$\,\,$ 1 $\,\,$ & $\,\,$ 1 $\,\,$ & $\,\,$5$\,\,$ & $\,\,$3 $\,\,$ \\
$\,\,$ 1/9$\,\,$ & $\,\,$ 1/5$\,\,$ & $\,\,$ 1 $\,\,$ & $\,\,$1 $\,\,$ \\
$\,\,$ 1/2$\,\,$ & $\,\,$ 1/3$\,\,$ & $\,\,$ 1 $\,\,$ & $\,\,$ 1  $\,\,$ \\
\end{pmatrix},
\qquad
\lambda_{\max} =
4.1966,
\qquad
CR = 0.0741
\end{equation*}

\begin{equation*}
\mathbf{w}^{AMAST} =
\begin{pmatrix}
0.404467\\
\color{red} 0.383512\color{black} \\
0.080739\\
0.131282
\end{pmatrix}\end{equation*}
\begin{equation*}
\left[ \frac{{w}^{AMAST}_i}{{w}^{AMAST}_j} \right] =
\begin{pmatrix}
$\,\,$ 1 $\,\,$ & $\,\,$\color{red} 1.0546\color{black} $\,\,$ & $\,\,$5.0095$\,\,$ & $\,\,$3.0809$\,\,$ \\
$\,\,$\color{red} 0.9482\color{black} $\,\,$ & $\,\,$ 1 $\,\,$ & $\,\,$\color{red} 4.7500\color{black} $\,\,$ & $\,\,$\color{red} 2.9213\color{black}   $\,\,$ \\
$\,\,$0.1996$\,\,$ & $\,\,$\color{red} 0.2105\color{black} $\,\,$ & $\,\,$ 1 $\,\,$ & $\,\,$0.6150 $\,\,$ \\
$\,\,$0.3246$\,\,$ & $\,\,$\color{red} 0.3423\color{black} $\,\,$ & $\,\,$1.6260$\,\,$ & $\,\,$ 1  $\,\,$ \\
\end{pmatrix},
\end{equation*}

\begin{equation*}
\mathbf{w}^{\prime} =
\begin{pmatrix}
0.400330\\
0.389818\\
0.079914\\
0.129939
\end{pmatrix} =
0.989771\cdot
\begin{pmatrix}
0.404467\\
\color{gr} 0.393846\color{black} \\
0.080739\\
0.131282
\end{pmatrix},
\end{equation*}
\begin{equation*}
\left[ \frac{{w}^{\prime}_i}{{w}^{\prime}_j} \right] =
\begin{pmatrix}
$\,\,$ 1 $\,\,$ & $\,\,$\color{gr} 1.0270\color{black} $\,\,$ & $\,\,$5.0095$\,\,$ & $\,\,$3.0809$\,\,$ \\
$\,\,$\color{gr} 0.9737\color{black} $\,\,$ & $\,\,$ 1 $\,\,$ & $\,\,$\color{gr} 4.8780\color{black} $\,\,$ & $\,\,$\color{gr} \color{blue} 3\color{black}   $\,\,$ \\
$\,\,$0.1996$\,\,$ & $\,\,$\color{gr} 0.2050\color{black} $\,\,$ & $\,\,$ 1 $\,\,$ & $\,\,$0.6150 $\,\,$ \\
$\,\,$0.3246$\,\,$ & $\,\,$\color{gr} \color{blue}  1/3\color{black} $\,\,$ & $\,\,$1.6260$\,\,$ & $\,\,$ 1  $\,\,$ \\
\end{pmatrix},
\end{equation*}
\end{example}
\newpage
\begin{example}
\begin{equation*}
\mathbf{A} =
\begin{pmatrix}
$\,\,$ 1 $\,\,$ & $\,\,$1$\,\,$ & $\,\,$9$\,\,$ & $\,\,$3 $\,\,$ \\
$\,\,$ 1 $\,\,$ & $\,\,$ 1 $\,\,$ & $\,\,$6$\,\,$ & $\,\,$4 $\,\,$ \\
$\,\,$ 1/9$\,\,$ & $\,\,$ 1/6$\,\,$ & $\,\,$ 1 $\,\,$ & $\,\,$1 $\,\,$ \\
$\,\,$ 1/3$\,\,$ & $\,\,$ 1/4$\,\,$ & $\,\,$ 1 $\,\,$ & $\,\,$ 1  $\,\,$ \\
\end{pmatrix},
\qquad
\lambda_{\max} =
4.1031,
\qquad
CR = 0.0389
\end{equation*}

\begin{equation*}
\mathbf{w}^{AMAST} =
\begin{pmatrix}
0.420599\\
\color{red} 0.406855\color{black} \\
0.070549\\
0.101998
\end{pmatrix}\end{equation*}
\begin{equation*}
\left[ \frac{{w}^{AMAST}_i}{{w}^{AMAST}_j} \right] =
\begin{pmatrix}
$\,\,$ 1 $\,\,$ & $\,\,$\color{red} 1.0338\color{black} $\,\,$ & $\,\,$5.9618$\,\,$ & $\,\,$4.1236$\,\,$ \\
$\,\,$\color{red} 0.9673\color{black} $\,\,$ & $\,\,$ 1 $\,\,$ & $\,\,$\color{red} 5.7670\color{black} $\,\,$ & $\,\,$\color{red} 3.9889\color{black}   $\,\,$ \\
$\,\,$0.1677$\,\,$ & $\,\,$\color{red} 0.1734\color{black} $\,\,$ & $\,\,$ 1 $\,\,$ & $\,\,$0.6917 $\,\,$ \\
$\,\,$0.2425$\,\,$ & $\,\,$\color{red} 0.2507\color{black} $\,\,$ & $\,\,$1.4458$\,\,$ & $\,\,$ 1  $\,\,$ \\
\end{pmatrix},
\end{equation*}

\begin{equation*}
\mathbf{w}^{\prime} =
\begin{pmatrix}
0.420122\\
0.407527\\
0.070469\\
0.101882
\end{pmatrix} =
0.998866\cdot
\begin{pmatrix}
0.420599\\
\color{gr} 0.407990\color{black} \\
0.070549\\
0.101998
\end{pmatrix},
\end{equation*}
\begin{equation*}
\left[ \frac{{w}^{\prime}_i}{{w}^{\prime}_j} \right] =
\begin{pmatrix}
$\,\,$ 1 $\,\,$ & $\,\,$\color{gr} 1.0309\color{black} $\,\,$ & $\,\,$5.9618$\,\,$ & $\,\,$4.1236$\,\,$ \\
$\,\,$\color{gr} 0.9700\color{black} $\,\,$ & $\,\,$ 1 $\,\,$ & $\,\,$\color{gr} 5.7831\color{black} $\,\,$ & $\,\,$\color{gr} \color{blue} 4\color{black}   $\,\,$ \\
$\,\,$0.1677$\,\,$ & $\,\,$\color{gr} 0.1729\color{black} $\,\,$ & $\,\,$ 1 $\,\,$ & $\,\,$0.6917 $\,\,$ \\
$\,\,$0.2425$\,\,$ & $\,\,$\color{gr} \color{blue}  1/4\color{black} $\,\,$ & $\,\,$1.4458$\,\,$ & $\,\,$ 1  $\,\,$ \\
\end{pmatrix},
\end{equation*}
\end{example}
\newpage
\begin{example}
\begin{equation*}
\mathbf{A} =
\begin{pmatrix}
$\,\,$ 1 $\,\,$ & $\,\,$1$\,\,$ & $\,\,$9$\,\,$ & $\,\,$4 $\,\,$ \\
$\,\,$ 1 $\,\,$ & $\,\,$ 1 $\,\,$ & $\,\,$5$\,\,$ & $\,\,$6 $\,\,$ \\
$\,\,$ 1/9$\,\,$ & $\,\,$ 1/5$\,\,$ & $\,\,$ 1 $\,\,$ & $\,\,$2 $\,\,$ \\
$\,\,$ 1/4$\,\,$ & $\,\,$ 1/6$\,\,$ & $\,\,$ 1/2$\,\,$ & $\,\,$ 1  $\,\,$ \\
\end{pmatrix},
\qquad
\lambda_{\max} =
4.1966,
\qquad
CR = 0.0741
\end{equation*}

\begin{equation*}
\mathbf{w}^{AMAST} =
\begin{pmatrix}
0.431941\\
\color{red} 0.410594\color{black} \\
0.086486\\
0.070979
\end{pmatrix}\end{equation*}
\begin{equation*}
\left[ \frac{{w}^{AMAST}_i}{{w}^{AMAST}_j} \right] =
\begin{pmatrix}
$\,\,$ 1 $\,\,$ & $\,\,$\color{red} 1.0520\color{black} $\,\,$ & $\,\,$4.9943$\,\,$ & $\,\,$6.0855$\,\,$ \\
$\,\,$\color{red} 0.9506\color{black} $\,\,$ & $\,\,$ 1 $\,\,$ & $\,\,$\color{red} 4.7475\color{black} $\,\,$ & $\,\,$\color{red} 5.7847\color{black}   $\,\,$ \\
$\,\,$0.2002$\,\,$ & $\,\,$\color{red} 0.2106\color{black} $\,\,$ & $\,\,$ 1 $\,\,$ & $\,\,$1.2185 $\,\,$ \\
$\,\,$0.1643$\,\,$ & $\,\,$\color{red} 0.1729\color{black} $\,\,$ & $\,\,$0.8207$\,\,$ & $\,\,$ 1  $\,\,$ \\
\end{pmatrix},
\end{equation*}

\begin{equation*}
\mathbf{w}^{\prime} =
\begin{pmatrix}
0.425440\\
0.419465\\
0.085184\\
0.069911
\end{pmatrix} =
0.984950\cdot
\begin{pmatrix}
0.431941\\
\color{gr} 0.425875\color{black} \\
0.086486\\
0.070979
\end{pmatrix},
\end{equation*}
\begin{equation*}
\left[ \frac{{w}^{\prime}_i}{{w}^{\prime}_j} \right] =
\begin{pmatrix}
$\,\,$ 1 $\,\,$ & $\,\,$\color{gr} 1.0142\color{black} $\,\,$ & $\,\,$4.9943$\,\,$ & $\,\,$6.0855$\,\,$ \\
$\,\,$\color{gr} 0.9860\color{black} $\,\,$ & $\,\,$ 1 $\,\,$ & $\,\,$\color{gr} 4.9242\color{black} $\,\,$ & $\,\,$\color{gr} \color{blue} 6\color{black}   $\,\,$ \\
$\,\,$0.2002$\,\,$ & $\,\,$\color{gr} 0.2031\color{black} $\,\,$ & $\,\,$ 1 $\,\,$ & $\,\,$1.2185 $\,\,$ \\
$\,\,$0.1643$\,\,$ & $\,\,$\color{gr} \color{blue}  1/6\color{black} $\,\,$ & $\,\,$0.8207$\,\,$ & $\,\,$ 1  $\,\,$ \\
\end{pmatrix},
\end{equation*}
\end{example}
\newpage
\begin{example}
\begin{equation*}
\mathbf{A} =
\begin{pmatrix}
$\,\,$ 1 $\,\,$ & $\,\,$1$\,\,$ & $\,\,$9$\,\,$ & $\,\,$4 $\,\,$ \\
$\,\,$ 1 $\,\,$ & $\,\,$ 1 $\,\,$ & $\,\,$5$\,\,$ & $\,\,$7 $\,\,$ \\
$\,\,$ 1/9$\,\,$ & $\,\,$ 1/5$\,\,$ & $\,\,$ 1 $\,\,$ & $\,\,$2 $\,\,$ \\
$\,\,$ 1/4$\,\,$ & $\,\,$ 1/7$\,\,$ & $\,\,$ 1/2$\,\,$ & $\,\,$ 1  $\,\,$ \\
\end{pmatrix},
\qquad
\lambda_{\max} =
4.1975,
\qquad
CR = 0.0745
\end{equation*}

\begin{equation*}
\mathbf{w}^{AMAST} =
\begin{pmatrix}
0.427051\\
\color{red} 0.420478\color{black} \\
0.084957\\
0.067514
\end{pmatrix}\end{equation*}
\begin{equation*}
\left[ \frac{{w}^{AMAST}_i}{{w}^{AMAST}_j} \right] =
\begin{pmatrix}
$\,\,$ 1 $\,\,$ & $\,\,$\color{red} 1.0156\color{black} $\,\,$ & $\,\,$5.0267$\,\,$ & $\,\,$6.3253$\,\,$ \\
$\,\,$\color{red} 0.9846\color{black} $\,\,$ & $\,\,$ 1 $\,\,$ & $\,\,$\color{red} 4.9493\color{black} $\,\,$ & $\,\,$\color{red} 6.2280\color{black}   $\,\,$ \\
$\,\,$0.1989$\,\,$ & $\,\,$\color{red} 0.2020\color{black} $\,\,$ & $\,\,$ 1 $\,\,$ & $\,\,$1.2583 $\,\,$ \\
$\,\,$0.1581$\,\,$ & $\,\,$\color{red} 0.1606\color{black} $\,\,$ & $\,\,$0.7947$\,\,$ & $\,\,$ 1  $\,\,$ \\
\end{pmatrix},
\end{equation*}

\begin{equation*}
\mathbf{w}^{\prime} =
\begin{pmatrix}
0.425221\\
0.422962\\
0.084592\\
0.067225
\end{pmatrix} =
0.995714\cdot
\begin{pmatrix}
0.427051\\
\color{gr} 0.424783\color{black} \\
0.084957\\
0.067514
\end{pmatrix},
\end{equation*}
\begin{equation*}
\left[ \frac{{w}^{\prime}_i}{{w}^{\prime}_j} \right] =
\begin{pmatrix}
$\,\,$ 1 $\,\,$ & $\,\,$\color{gr} 1.0053\color{black} $\,\,$ & $\,\,$5.0267$\,\,$ & $\,\,$6.3253$\,\,$ \\
$\,\,$\color{gr} 0.9947\color{black} $\,\,$ & $\,\,$ 1 $\,\,$ & $\,\,$\color{gr} \color{blue} 5\color{black} $\,\,$ & $\,\,$\color{gr} 6.2917\color{black}   $\,\,$ \\
$\,\,$0.1989$\,\,$ & $\,\,$\color{gr} \color{blue}  1/5\color{black} $\,\,$ & $\,\,$ 1 $\,\,$ & $\,\,$1.2583 $\,\,$ \\
$\,\,$0.1581$\,\,$ & $\,\,$\color{gr} 0.1589\color{black} $\,\,$ & $\,\,$0.7947$\,\,$ & $\,\,$ 1  $\,\,$ \\
\end{pmatrix},
\end{equation*}
\end{example}
\newpage
\begin{example}
\begin{equation*}
\mathbf{A} =
\begin{pmatrix}
$\,\,$ 1 $\,\,$ & $\,\,$2$\,\,$ & $\,\,$1$\,\,$ & $\,\,$6 $\,\,$ \\
$\,\,$ 1/2$\,\,$ & $\,\,$ 1 $\,\,$ & $\,\,$2$\,\,$ & $\,\,$5 $\,\,$ \\
$\,\,$ 1 $\,\,$ & $\,\,$ 1/2$\,\,$ & $\,\,$ 1 $\,\,$ & $\,\,$4 $\,\,$ \\
$\,\,$ 1/6$\,\,$ & $\,\,$ 1/5$\,\,$ & $\,\,$ 1/4$\,\,$ & $\,\,$ 1  $\,\,$ \\
\end{pmatrix},
\qquad
\lambda_{\max} =
4.1655,
\qquad
CR = 0.0624
\end{equation*}

\begin{equation*}
\mathbf{w}^{AMAST} =
\begin{pmatrix}
0.380752\\
0.309459\\
0.247939\\
\color{red} 0.061850\color{black}
\end{pmatrix}\end{equation*}
\begin{equation*}
\left[ \frac{{w}^{AMAST}_i}{{w}^{AMAST}_j} \right] =
\begin{pmatrix}
$\,\,$ 1 $\,\,$ & $\,\,$1.2304$\,\,$ & $\,\,$1.5357$\,\,$ & $\,\,$\color{red} 6.1560\color{black} $\,\,$ \\
$\,\,$0.8128$\,\,$ & $\,\,$ 1 $\,\,$ & $\,\,$1.2481$\,\,$ & $\,\,$\color{red} 5.0034\color{black}   $\,\,$ \\
$\,\,$0.6512$\,\,$ & $\,\,$0.8012$\,\,$ & $\,\,$ 1 $\,\,$ & $\,\,$\color{red} 4.0087\color{black}  $\,\,$ \\
$\,\,$\color{red} 0.1624\color{black} $\,\,$ & $\,\,$\color{red} 0.1999\color{black} $\,\,$ & $\,\,$\color{red} 0.2495\color{black} $\,\,$ & $\,\,$ 1  $\,\,$ \\
\end{pmatrix},
\end{equation*}

\begin{equation*}
\mathbf{w}^{\prime} =
\begin{pmatrix}
0.380736\\
0.309446\\
0.247929\\
0.061889
\end{pmatrix} =
0.999959\cdot
\begin{pmatrix}
0.380752\\
0.309459\\
0.247939\\
\color{gr} 0.061892\color{black}
\end{pmatrix},
\end{equation*}
\begin{equation*}
\left[ \frac{{w}^{\prime}_i}{{w}^{\prime}_j} \right] =
\begin{pmatrix}
$\,\,$ 1 $\,\,$ & $\,\,$1.2304$\,\,$ & $\,\,$1.5357$\,\,$ & $\,\,$\color{gr} 6.1519\color{black} $\,\,$ \\
$\,\,$0.8128$\,\,$ & $\,\,$ 1 $\,\,$ & $\,\,$1.2481$\,\,$ & $\,\,$\color{gr} \color{blue} 5\color{black}   $\,\,$ \\
$\,\,$0.6512$\,\,$ & $\,\,$0.8012$\,\,$ & $\,\,$ 1 $\,\,$ & $\,\,$\color{gr} 4.0060\color{black}  $\,\,$ \\
$\,\,$\color{gr} 0.1626\color{black} $\,\,$ & $\,\,$\color{gr} \color{blue}  1/5\color{black} $\,\,$ & $\,\,$\color{gr} 0.2496\color{black} $\,\,$ & $\,\,$ 1  $\,\,$ \\
\end{pmatrix},
\end{equation*}
\end{example}
\newpage
\begin{example}
\begin{equation*}
\mathbf{A} =
\begin{pmatrix}
$\,\,$ 1 $\,\,$ & $\,\,$2$\,\,$ & $\,\,$3$\,\,$ & $\,\,$8 $\,\,$ \\
$\,\,$ 1/2$\,\,$ & $\,\,$ 1 $\,\,$ & $\,\,$1$\,\,$ & $\,\,$6 $\,\,$ \\
$\,\,$ 1/3$\,\,$ & $\,\,$ 1 $\,\,$ & $\,\,$ 1 $\,\,$ & $\,\,$2 $\,\,$ \\
$\,\,$ 1/8$\,\,$ & $\,\,$ 1/6$\,\,$ & $\,\,$ 1/2$\,\,$ & $\,\,$ 1  $\,\,$ \\
\end{pmatrix},
\qquad
\lambda_{\max} =
4.1031,
\qquad
CR = 0.0389
\end{equation*}

\begin{equation*}
\mathbf{w}^{AMAST} =
\begin{pmatrix}
\color{red} 0.502260\color{black} \\
0.256851\\
0.177486\\
0.063402
\end{pmatrix}\end{equation*}
\begin{equation*}
\left[ \frac{{w}^{AMAST}_i}{{w}^{AMAST}_j} \right] =
\begin{pmatrix}
$\,\,$ 1 $\,\,$ & $\,\,$\color{red} 1.9555\color{black} $\,\,$ & $\,\,$\color{red} 2.8299\color{black} $\,\,$ & $\,\,$\color{red} 7.9218\color{black} $\,\,$ \\
$\,\,$\color{red} 0.5114\color{black} $\,\,$ & $\,\,$ 1 $\,\,$ & $\,\,$1.4472$\,\,$ & $\,\,$4.0512  $\,\,$ \\
$\,\,$\color{red} 0.3534\color{black} $\,\,$ & $\,\,$0.6910$\,\,$ & $\,\,$ 1 $\,\,$ & $\,\,$2.7994 $\,\,$ \\
$\,\,$\color{red} 0.1262\color{black} $\,\,$ & $\,\,$0.2468$\,\,$ & $\,\,$0.3572$\,\,$ & $\,\,$ 1  $\,\,$ \\
\end{pmatrix},
\end{equation*}

\begin{equation*}
\mathbf{w}^{\prime} =
\begin{pmatrix}
0.504715\\
0.255585\\
0.176611\\
0.063089
\end{pmatrix} =
0.995069\cdot
\begin{pmatrix}
\color{gr} 0.507216\color{black} \\
0.256851\\
0.177486\\
0.063402
\end{pmatrix},
\end{equation*}
\begin{equation*}
\left[ \frac{{w}^{\prime}_i}{{w}^{\prime}_j} \right] =
\begin{pmatrix}
$\,\,$ 1 $\,\,$ & $\,\,$\color{gr} 1.9747\color{black} $\,\,$ & $\,\,$\color{gr} 2.8578\color{black} $\,\,$ & $\,\,$\color{gr} \color{blue} 8\color{black} $\,\,$ \\
$\,\,$\color{gr} 0.5064\color{black} $\,\,$ & $\,\,$ 1 $\,\,$ & $\,\,$1.4472$\,\,$ & $\,\,$4.0512  $\,\,$ \\
$\,\,$\color{gr} 0.3499\color{black} $\,\,$ & $\,\,$0.6910$\,\,$ & $\,\,$ 1 $\,\,$ & $\,\,$2.7994 $\,\,$ \\
$\,\,$\color{gr} \color{blue}  1/8\color{black} $\,\,$ & $\,\,$0.2468$\,\,$ & $\,\,$0.3572$\,\,$ & $\,\,$ 1  $\,\,$ \\
\end{pmatrix},
\end{equation*}
\end{example}
\newpage
\begin{example}
\begin{equation*}
\mathbf{A} =
\begin{pmatrix}
$\,\,$ 1 $\,\,$ & $\,\,$2$\,\,$ & $\,\,$3$\,\,$ & $\,\,$9 $\,\,$ \\
$\,\,$ 1/2$\,\,$ & $\,\,$ 1 $\,\,$ & $\,\,$1$\,\,$ & $\,\,$7 $\,\,$ \\
$\,\,$ 1/3$\,\,$ & $\,\,$ 1 $\,\,$ & $\,\,$ 1 $\,\,$ & $\,\,$2 $\,\,$ \\
$\,\,$ 1/9$\,\,$ & $\,\,$ 1/7$\,\,$ & $\,\,$ 1/2$\,\,$ & $\,\,$ 1  $\,\,$ \\
\end{pmatrix},
\qquad
\lambda_{\max} =
4.1342,
\qquad
CR = 0.0506
\end{equation*}

\begin{equation*}
\mathbf{w}^{AMAST} =
\begin{pmatrix}
\color{red} 0.504799\color{black} \\
0.261793\\
0.174790\\
0.058618
\end{pmatrix}\end{equation*}
\begin{equation*}
\left[ \frac{{w}^{AMAST}_i}{{w}^{AMAST}_j} \right] =
\begin{pmatrix}
$\,\,$ 1 $\,\,$ & $\,\,$\color{red} 1.9282\color{black} $\,\,$ & $\,\,$\color{red} 2.8880\color{black} $\,\,$ & $\,\,$\color{red} 8.6117\color{black} $\,\,$ \\
$\,\,$\color{red} 0.5186\color{black} $\,\,$ & $\,\,$ 1 $\,\,$ & $\,\,$1.4978$\,\,$ & $\,\,$4.4661  $\,\,$ \\
$\,\,$\color{red} 0.3463\color{black} $\,\,$ & $\,\,$0.6677$\,\,$ & $\,\,$ 1 $\,\,$ & $\,\,$2.9818 $\,\,$ \\
$\,\,$\color{red} 0.1161\color{black} $\,\,$ & $\,\,$0.2239$\,\,$ & $\,\,$0.3354$\,\,$ & $\,\,$ 1  $\,\,$ \\
\end{pmatrix},
\end{equation*}

\begin{equation*}
\mathbf{w}^{\prime} =
\begin{pmatrix}
0.513931\\
0.256965\\
0.171566\\
0.057537
\end{pmatrix} =
0.981560\cdot
\begin{pmatrix}
\color{gr} 0.523586\color{black} \\
0.261793\\
0.174790\\
0.058618
\end{pmatrix},
\end{equation*}
\begin{equation*}
\left[ \frac{{w}^{\prime}_i}{{w}^{\prime}_j} \right] =
\begin{pmatrix}
$\,\,$ 1 $\,\,$ & $\,\,$\color{gr} \color{blue} 2\color{black} $\,\,$ & $\,\,$\color{gr} 2.9955\color{black} $\,\,$ & $\,\,$\color{gr} 8.9322\color{black} $\,\,$ \\
$\,\,$\color{gr} \color{blue}  1/2\color{black} $\,\,$ & $\,\,$ 1 $\,\,$ & $\,\,$1.4978$\,\,$ & $\,\,$4.4661  $\,\,$ \\
$\,\,$\color{gr} 0.3338\color{black} $\,\,$ & $\,\,$0.6677$\,\,$ & $\,\,$ 1 $\,\,$ & $\,\,$2.9818 $\,\,$ \\
$\,\,$\color{gr} 0.1120\color{black} $\,\,$ & $\,\,$0.2239$\,\,$ & $\,\,$0.3354$\,\,$ & $\,\,$ 1  $\,\,$ \\
\end{pmatrix},
\end{equation*}
\end{example}
\newpage
\begin{example}
\begin{equation*}
\mathbf{A} =
\begin{pmatrix}
$\,\,$ 1 $\,\,$ & $\,\,$2$\,\,$ & $\,\,$3$\,\,$ & $\,\,$9 $\,\,$ \\
$\,\,$ 1/2$\,\,$ & $\,\,$ 1 $\,\,$ & $\,\,$1$\,\,$ & $\,\,$8 $\,\,$ \\
$\,\,$ 1/3$\,\,$ & $\,\,$ 1 $\,\,$ & $\,\,$ 1 $\,\,$ & $\,\,$2 $\,\,$ \\
$\,\,$ 1/9$\,\,$ & $\,\,$ 1/8$\,\,$ & $\,\,$ 1/2$\,\,$ & $\,\,$ 1  $\,\,$ \\
\end{pmatrix},
\qquad
\lambda_{\max} =
4.1664,
\qquad
CR = 0.0627
\end{equation*}

\begin{equation*}
\mathbf{w}^{AMAST} =
\begin{pmatrix}
\color{red} 0.499789\color{black} \\
0.269245\\
0.174137\\
0.056829
\end{pmatrix}\end{equation*}
\begin{equation*}
\left[ \frac{{w}^{AMAST}_i}{{w}^{AMAST}_j} \right] =
\begin{pmatrix}
$\,\,$ 1 $\,\,$ & $\,\,$\color{red} 1.8563\color{black} $\,\,$ & $\,\,$\color{red} 2.8701\color{black} $\,\,$ & $\,\,$\color{red} 8.7946\color{black} $\,\,$ \\
$\,\,$\color{red} 0.5387\color{black} $\,\,$ & $\,\,$ 1 $\,\,$ & $\,\,$1.5462$\,\,$ & $\,\,$4.7378  $\,\,$ \\
$\,\,$\color{red} 0.3484\color{black} $\,\,$ & $\,\,$0.6468$\,\,$ & $\,\,$ 1 $\,\,$ & $\,\,$3.0642 $\,\,$ \\
$\,\,$\color{red} 0.1137\color{black} $\,\,$ & $\,\,$0.2111$\,\,$ & $\,\,$0.3263$\,\,$ & $\,\,$ 1  $\,\,$ \\
\end{pmatrix},
\end{equation*}

\begin{equation*}
\mathbf{w}^{\prime} =
\begin{pmatrix}
0.505560\\
0.266139\\
0.172128\\
0.056173
\end{pmatrix} =
0.988463\cdot
\begin{pmatrix}
\color{gr} 0.511461\color{black} \\
0.269245\\
0.174137\\
0.056829
\end{pmatrix},
\end{equation*}
\begin{equation*}
\left[ \frac{{w}^{\prime}_i}{{w}^{\prime}_j} \right] =
\begin{pmatrix}
$\,\,$ 1 $\,\,$ & $\,\,$\color{gr} 1.8996\color{black} $\,\,$ & $\,\,$\color{gr} 2.9371\color{black} $\,\,$ & $\,\,$\color{gr} \color{blue} 9\color{black} $\,\,$ \\
$\,\,$\color{gr} 0.5264\color{black} $\,\,$ & $\,\,$ 1 $\,\,$ & $\,\,$1.5462$\,\,$ & $\,\,$4.7378  $\,\,$ \\
$\,\,$\color{gr} 0.3405\color{black} $\,\,$ & $\,\,$0.6468$\,\,$ & $\,\,$ 1 $\,\,$ & $\,\,$3.0642 $\,\,$ \\
$\,\,$\color{gr} \color{blue}  1/9\color{black} $\,\,$ & $\,\,$0.2111$\,\,$ & $\,\,$0.3263$\,\,$ & $\,\,$ 1  $\,\,$ \\
\end{pmatrix},
\end{equation*}
\end{example}
\newpage
\begin{example}
\begin{equation*}
\mathbf{A} =
\begin{pmatrix}
$\,\,$ 1 $\,\,$ & $\,\,$2$\,\,$ & $\,\,$3$\,\,$ & $\,\,$9 $\,\,$ \\
$\,\,$ 1/2$\,\,$ & $\,\,$ 1 $\,\,$ & $\,\,$1$\,\,$ & $\,\,$9 $\,\,$ \\
$\,\,$ 1/3$\,\,$ & $\,\,$ 1 $\,\,$ & $\,\,$ 1 $\,\,$ & $\,\,$2 $\,\,$ \\
$\,\,$ 1/9$\,\,$ & $\,\,$ 1/9$\,\,$ & $\,\,$ 1/2$\,\,$ & $\,\,$ 1  $\,\,$ \\
\end{pmatrix},
\qquad
\lambda_{\max} =
4.1990,
\qquad
CR = 0.0750
\end{equation*}

\begin{equation*}
\mathbf{w}^{AMAST} =
\begin{pmatrix}
\color{red} 0.495247\color{black} \\
0.275863\\
0.173545\\
0.055345
\end{pmatrix}\end{equation*}
\begin{equation*}
\left[ \frac{{w}^{AMAST}_i}{{w}^{AMAST}_j} \right] =
\begin{pmatrix}
$\,\,$ 1 $\,\,$ & $\,\,$\color{red} 1.7953\color{black} $\,\,$ & $\,\,$\color{red} 2.8537\color{black} $\,\,$ & $\,\,$\color{red} 8.9484\color{black} $\,\,$ \\
$\,\,$\color{red} 0.5570\color{black} $\,\,$ & $\,\,$ 1 $\,\,$ & $\,\,$1.5896$\,\,$ & $\,\,$4.9844  $\,\,$ \\
$\,\,$\color{red} 0.3504\color{black} $\,\,$ & $\,\,$0.6291$\,\,$ & $\,\,$ 1 $\,\,$ & $\,\,$3.1357 $\,\,$ \\
$\,\,$\color{red} 0.1118\color{black} $\,\,$ & $\,\,$0.2006$\,\,$ & $\,\,$0.3189$\,\,$ & $\,\,$ 1  $\,\,$ \\
\end{pmatrix},
\end{equation*}

\begin{equation*}
\mathbf{w}^{\prime} =
\begin{pmatrix}
0.496685\\
0.275077\\
0.173051\\
0.055187
\end{pmatrix} =
0.997152\cdot
\begin{pmatrix}
\color{gr} 0.498103\color{black} \\
0.275863\\
0.173545\\
0.055345
\end{pmatrix},
\end{equation*}
\begin{equation*}
\left[ \frac{{w}^{\prime}_i}{{w}^{\prime}_j} \right] =
\begin{pmatrix}
$\,\,$ 1 $\,\,$ & $\,\,$\color{gr} 1.8056\color{black} $\,\,$ & $\,\,$\color{gr} 2.8702\color{black} $\,\,$ & $\,\,$\color{gr} \color{blue} 9\color{black} $\,\,$ \\
$\,\,$\color{gr} 0.5538\color{black} $\,\,$ & $\,\,$ 1 $\,\,$ & $\,\,$1.5896$\,\,$ & $\,\,$4.9844  $\,\,$ \\
$\,\,$\color{gr} 0.3484\color{black} $\,\,$ & $\,\,$0.6291$\,\,$ & $\,\,$ 1 $\,\,$ & $\,\,$3.1357 $\,\,$ \\
$\,\,$\color{gr} \color{blue}  1/9\color{black} $\,\,$ & $\,\,$0.2006$\,\,$ & $\,\,$0.3189$\,\,$ & $\,\,$ 1  $\,\,$ \\
\end{pmatrix},
\end{equation*}
\end{example}
\newpage
\begin{example}
\begin{equation*}
\mathbf{A} =
\begin{pmatrix}
$\,\,$ 1 $\,\,$ & $\,\,$2$\,\,$ & $\,\,$4$\,\,$ & $\,\,$6 $\,\,$ \\
$\,\,$ 1/2$\,\,$ & $\,\,$ 1 $\,\,$ & $\,\,$3$\,\,$ & $\,\,$2 $\,\,$ \\
$\,\,$ 1/4$\,\,$ & $\,\,$ 1/3$\,\,$ & $\,\,$ 1 $\,\,$ & $\,\,$2 $\,\,$ \\
$\,\,$ 1/6$\,\,$ & $\,\,$ 1/2$\,\,$ & $\,\,$ 1/2$\,\,$ & $\,\,$ 1  $\,\,$ \\
\end{pmatrix},
\qquad
\lambda_{\max} =
4.1031,
\qquad
CR = 0.0389
\end{equation*}

\begin{equation*}
\mathbf{w}^{AMAST} =
\begin{pmatrix}
\color{red} 0.515267\color{black} \\
0.263562\\
0.129367\\
0.091804
\end{pmatrix}\end{equation*}
\begin{equation*}
\left[ \frac{{w}^{AMAST}_i}{{w}^{AMAST}_j} \right] =
\begin{pmatrix}
$\,\,$ 1 $\,\,$ & $\,\,$\color{red} 1.9550\color{black} $\,\,$ & $\,\,$\color{red} 3.9830\color{black} $\,\,$ & $\,\,$\color{red} 5.6127\color{black} $\,\,$ \\
$\,\,$\color{red} 0.5115\color{black} $\,\,$ & $\,\,$ 1 $\,\,$ & $\,\,$2.0373$\,\,$ & $\,\,$2.8709  $\,\,$ \\
$\,\,$\color{red} 0.2511\color{black} $\,\,$ & $\,\,$0.4908$\,\,$ & $\,\,$ 1 $\,\,$ & $\,\,$1.4092 $\,\,$ \\
$\,\,$\color{red} 0.1782\color{black} $\,\,$ & $\,\,$0.3483$\,\,$ & $\,\,$0.7096$\,\,$ & $\,\,$ 1  $\,\,$ \\
\end{pmatrix},
\end{equation*}

\begin{equation*}
\mathbf{w}^{\prime} =
\begin{pmatrix}
0.516332\\
0.262983\\
0.129083\\
0.091602
\end{pmatrix} =
0.997803\cdot
\begin{pmatrix}
\color{gr} 0.517469\color{black} \\
0.263562\\
0.129367\\
0.091804
\end{pmatrix},
\end{equation*}
\begin{equation*}
\left[ \frac{{w}^{\prime}_i}{{w}^{\prime}_j} \right] =
\begin{pmatrix}
$\,\,$ 1 $\,\,$ & $\,\,$\color{gr} 1.9634\color{black} $\,\,$ & $\,\,$\color{gr} \color{blue} 4\color{black} $\,\,$ & $\,\,$\color{gr} 5.6367\color{black} $\,\,$ \\
$\,\,$\color{gr} 0.5093\color{black} $\,\,$ & $\,\,$ 1 $\,\,$ & $\,\,$2.0373$\,\,$ & $\,\,$2.8709  $\,\,$ \\
$\,\,$\color{gr} \color{blue}  1/4\color{black} $\,\,$ & $\,\,$0.4908$\,\,$ & $\,\,$ 1 $\,\,$ & $\,\,$1.4092 $\,\,$ \\
$\,\,$\color{gr} 0.1774\color{black} $\,\,$ & $\,\,$0.3483$\,\,$ & $\,\,$0.7096$\,\,$ & $\,\,$ 1  $\,\,$ \\
\end{pmatrix},
\end{equation*}
\end{example}
\newpage
\begin{example}
\begin{equation*}
\mathbf{A} =
\begin{pmatrix}
$\,\,$ 1 $\,\,$ & $\,\,$2$\,\,$ & $\,\,$4$\,\,$ & $\,\,$6 $\,\,$ \\
$\,\,$ 1/2$\,\,$ & $\,\,$ 1 $\,\,$ & $\,\,$3$\,\,$ & $\,\,$2 $\,\,$ \\
$\,\,$ 1/4$\,\,$ & $\,\,$ 1/3$\,\,$ & $\,\,$ 1 $\,\,$ & $\,\,$3 $\,\,$ \\
$\,\,$ 1/6$\,\,$ & $\,\,$ 1/2$\,\,$ & $\,\,$ 1/3$\,\,$ & $\,\,$ 1  $\,\,$ \\
\end{pmatrix},
\qquad
\lambda_{\max} =
4.1990,
\qquad
CR = 0.0750
\end{equation*}

\begin{equation*}
\mathbf{w}^{AMAST} =
\begin{pmatrix}
\color{red} 0.507109\color{black} \\
0.263066\\
0.145265\\
0.084560
\end{pmatrix}\end{equation*}
\begin{equation*}
\left[ \frac{{w}^{AMAST}_i}{{w}^{AMAST}_j} \right] =
\begin{pmatrix}
$\,\,$ 1 $\,\,$ & $\,\,$\color{red} 1.9277\color{black} $\,\,$ & $\,\,$\color{red} 3.4909\color{black} $\,\,$ & $\,\,$\color{red} 5.9970\color{black} $\,\,$ \\
$\,\,$\color{red} 0.5188\color{black} $\,\,$ & $\,\,$ 1 $\,\,$ & $\,\,$1.8109$\,\,$ & $\,\,$3.1110  $\,\,$ \\
$\,\,$\color{red} 0.2865\color{black} $\,\,$ & $\,\,$0.5522$\,\,$ & $\,\,$ 1 $\,\,$ & $\,\,$1.7179 $\,\,$ \\
$\,\,$\color{red} 0.1667\color{black} $\,\,$ & $\,\,$0.3214$\,\,$ & $\,\,$0.5821$\,\,$ & $\,\,$ 1  $\,\,$ \\
\end{pmatrix},
\end{equation*}

\begin{equation*}
\mathbf{w}^{\prime} =
\begin{pmatrix}
0.507233\\
0.263000\\
0.145228\\
0.084539
\end{pmatrix} =
0.999748\cdot
\begin{pmatrix}
\color{gr} 0.507360\color{black} \\
0.263066\\
0.145265\\
0.084560
\end{pmatrix},
\end{equation*}
\begin{equation*}
\left[ \frac{{w}^{\prime}_i}{{w}^{\prime}_j} \right] =
\begin{pmatrix}
$\,\,$ 1 $\,\,$ & $\,\,$\color{gr} 1.9286\color{black} $\,\,$ & $\,\,$\color{gr} 3.4927\color{black} $\,\,$ & $\,\,$\color{gr} \color{blue} 6\color{black} $\,\,$ \\
$\,\,$\color{gr} 0.5185\color{black} $\,\,$ & $\,\,$ 1 $\,\,$ & $\,\,$1.8109$\,\,$ & $\,\,$3.1110  $\,\,$ \\
$\,\,$\color{gr} 0.2863\color{black} $\,\,$ & $\,\,$0.5522$\,\,$ & $\,\,$ 1 $\,\,$ & $\,\,$1.7179 $\,\,$ \\
$\,\,$\color{gr} \color{blue}  1/6\color{black} $\,\,$ & $\,\,$0.3214$\,\,$ & $\,\,$0.5821$\,\,$ & $\,\,$ 1  $\,\,$ \\
\end{pmatrix},
\end{equation*}
\end{example}
\newpage
\begin{example}
\begin{equation*}
\mathbf{A} =
\begin{pmatrix}
$\,\,$ 1 $\,\,$ & $\,\,$2$\,\,$ & $\,\,$4$\,\,$ & $\,\,$7 $\,\,$ \\
$\,\,$ 1/2$\,\,$ & $\,\,$ 1 $\,\,$ & $\,\,$3$\,\,$ & $\,\,$2 $\,\,$ \\
$\,\,$ 1/4$\,\,$ & $\,\,$ 1/3$\,\,$ & $\,\,$ 1 $\,\,$ & $\,\,$3 $\,\,$ \\
$\,\,$ 1/7$\,\,$ & $\,\,$ 1/2$\,\,$ & $\,\,$ 1/3$\,\,$ & $\,\,$ 1  $\,\,$ \\
\end{pmatrix},
\qquad
\lambda_{\max} =
4.1964,
\qquad
CR = 0.0741
\end{equation*}

\begin{equation*}
\mathbf{w}^{AMAST} =
\begin{pmatrix}
\color{red} 0.517898\color{black} \\
0.259639\\
0.142311\\
0.080152
\end{pmatrix}\end{equation*}
\begin{equation*}
\left[ \frac{{w}^{AMAST}_i}{{w}^{AMAST}_j} \right] =
\begin{pmatrix}
$\,\,$ 1 $\,\,$ & $\,\,$\color{red} 1.9947\color{black} $\,\,$ & $\,\,$\color{red} 3.6392\color{black} $\,\,$ & $\,\,$\color{red} 6.4615\color{black} $\,\,$ \\
$\,\,$\color{red} 0.5013\color{black} $\,\,$ & $\,\,$ 1 $\,\,$ & $\,\,$1.8245$\,\,$ & $\,\,$3.2393  $\,\,$ \\
$\,\,$\color{red} 0.2748\color{black} $\,\,$ & $\,\,$0.5481$\,\,$ & $\,\,$ 1 $\,\,$ & $\,\,$1.7755 $\,\,$ \\
$\,\,$\color{red} 0.1548\color{black} $\,\,$ & $\,\,$0.3087$\,\,$ & $\,\,$0.5632$\,\,$ & $\,\,$ 1  $\,\,$ \\
\end{pmatrix},
\end{equation*}

\begin{equation*}
\mathbf{w}^{\prime} =
\begin{pmatrix}
0.518563\\
0.259281\\
0.142115\\
0.080041
\end{pmatrix} =
0.998622\cdot
\begin{pmatrix}
\color{gr} 0.519278\color{black} \\
0.259639\\
0.142311\\
0.080152
\end{pmatrix},
\end{equation*}
\begin{equation*}
\left[ \frac{{w}^{\prime}_i}{{w}^{\prime}_j} \right] =
\begin{pmatrix}
$\,\,$ 1 $\,\,$ & $\,\,$\color{gr} \color{blue} 2\color{black} $\,\,$ & $\,\,$\color{gr} 3.6489\color{black} $\,\,$ & $\,\,$\color{gr} 6.4787\color{black} $\,\,$ \\
$\,\,$\color{gr} \color{blue}  1/2\color{black} $\,\,$ & $\,\,$ 1 $\,\,$ & $\,\,$1.8245$\,\,$ & $\,\,$3.2393  $\,\,$ \\
$\,\,$\color{gr} 0.2741\color{black} $\,\,$ & $\,\,$0.5481$\,\,$ & $\,\,$ 1 $\,\,$ & $\,\,$1.7755 $\,\,$ \\
$\,\,$\color{gr} 0.1544\color{black} $\,\,$ & $\,\,$0.3087$\,\,$ & $\,\,$0.5632$\,\,$ & $\,\,$ 1  $\,\,$ \\
\end{pmatrix},
\end{equation*}
\end{example}
\newpage
\begin{example}
\begin{equation*}
\mathbf{A} =
\begin{pmatrix}
$\,\,$ 1 $\,\,$ & $\,\,$2$\,\,$ & $\,\,$5$\,\,$ & $\,\,$6 $\,\,$ \\
$\,\,$ 1/2$\,\,$ & $\,\,$ 1 $\,\,$ & $\,\,$4$\,\,$ & $\,\,$2 $\,\,$ \\
$\,\,$ 1/5$\,\,$ & $\,\,$ 1/4$\,\,$ & $\,\,$ 1 $\,\,$ & $\,\,$2 $\,\,$ \\
$\,\,$ 1/6$\,\,$ & $\,\,$ 1/2$\,\,$ & $\,\,$ 1/2$\,\,$ & $\,\,$ 1  $\,\,$ \\
\end{pmatrix},
\qquad
\lambda_{\max} =
4.1655,
\qquad
CR = 0.0624
\end{equation*}

\begin{equation*}
\mathbf{w}^{AMAST} =
\begin{pmatrix}
\color{red} 0.523242\color{black} \\
0.274510\\
0.112136\\
0.090112
\end{pmatrix}\end{equation*}
\begin{equation*}
\left[ \frac{{w}^{AMAST}_i}{{w}^{AMAST}_j} \right] =
\begin{pmatrix}
$\,\,$ 1 $\,\,$ & $\,\,$\color{red} 1.9061\color{black} $\,\,$ & $\,\,$\color{red} 4.6661\color{black} $\,\,$ & $\,\,$\color{red} 5.8066\color{black} $\,\,$ \\
$\,\,$\color{red} 0.5246\color{black} $\,\,$ & $\,\,$ 1 $\,\,$ & $\,\,$2.4480$\,\,$ & $\,\,$3.0463  $\,\,$ \\
$\,\,$\color{red} 0.2143\color{black} $\,\,$ & $\,\,$0.4085$\,\,$ & $\,\,$ 1 $\,\,$ & $\,\,$1.2444 $\,\,$ \\
$\,\,$\color{red} 0.1722\color{black} $\,\,$ & $\,\,$0.3283$\,\,$ & $\,\,$0.8036$\,\,$ & $\,\,$ 1  $\,\,$ \\
\end{pmatrix},
\end{equation*}

\begin{equation*}
\mathbf{w}^{\prime} =
\begin{pmatrix}
0.531410\\
0.269807\\
0.110215\\
0.088568
\end{pmatrix} =
0.982868\cdot
\begin{pmatrix}
\color{gr} 0.540673\color{black} \\
0.274510\\
0.112136\\
0.090112
\end{pmatrix},
\end{equation*}
\begin{equation*}
\left[ \frac{{w}^{\prime}_i}{{w}^{\prime}_j} \right] =
\begin{pmatrix}
$\,\,$ 1 $\,\,$ & $\,\,$\color{gr} 1.9696\color{black} $\,\,$ & $\,\,$\color{gr} 4.8216\color{black} $\,\,$ & $\,\,$\color{gr} \color{blue} 6\color{black} $\,\,$ \\
$\,\,$\color{gr} 0.5077\color{black} $\,\,$ & $\,\,$ 1 $\,\,$ & $\,\,$2.4480$\,\,$ & $\,\,$3.0463  $\,\,$ \\
$\,\,$\color{gr} 0.2074\color{black} $\,\,$ & $\,\,$0.4085$\,\,$ & $\,\,$ 1 $\,\,$ & $\,\,$1.2444 $\,\,$ \\
$\,\,$\color{gr} \color{blue}  1/6\color{black} $\,\,$ & $\,\,$0.3283$\,\,$ & $\,\,$0.8036$\,\,$ & $\,\,$ 1  $\,\,$ \\
\end{pmatrix},
\end{equation*}
\end{example}
\newpage
\begin{example}
\begin{equation*}
\mathbf{A} =
\begin{pmatrix}
$\,\,$ 1 $\,\,$ & $\,\,$2$\,\,$ & $\,\,$5$\,\,$ & $\,\,$7 $\,\,$ \\
$\,\,$ 1/2$\,\,$ & $\,\,$ 1 $\,\,$ & $\,\,$4$\,\,$ & $\,\,$2 $\,\,$ \\
$\,\,$ 1/5$\,\,$ & $\,\,$ 1/4$\,\,$ & $\,\,$ 1 $\,\,$ & $\,\,$2 $\,\,$ \\
$\,\,$ 1/7$\,\,$ & $\,\,$ 1/2$\,\,$ & $\,\,$ 1/2$\,\,$ & $\,\,$ 1  $\,\,$ \\
\end{pmatrix},
\qquad
\lambda_{\max} =
4.1665,
\qquad
CR = 0.0628
\end{equation*}

\begin{equation*}
\mathbf{w}^{AMAST} =
\begin{pmatrix}
\color{red} 0.534042\color{black} \\
0.270672\\
0.109841\\
0.085444
\end{pmatrix}\end{equation*}
\begin{equation*}
\left[ \frac{{w}^{AMAST}_i}{{w}^{AMAST}_j} \right] =
\begin{pmatrix}
$\,\,$ 1 $\,\,$ & $\,\,$\color{red} 1.9730\color{black} $\,\,$ & $\,\,$\color{red} 4.8619\color{black} $\,\,$ & $\,\,$\color{red} 6.2502\color{black} $\,\,$ \\
$\,\,$\color{red} 0.5068\color{black} $\,\,$ & $\,\,$ 1 $\,\,$ & $\,\,$2.4642$\,\,$ & $\,\,$3.1678  $\,\,$ \\
$\,\,$\color{red} 0.2057\color{black} $\,\,$ & $\,\,$0.4058$\,\,$ & $\,\,$ 1 $\,\,$ & $\,\,$1.2855 $\,\,$ \\
$\,\,$\color{red} 0.1600\color{black} $\,\,$ & $\,\,$0.3157$\,\,$ & $\,\,$0.7779$\,\,$ & $\,\,$ 1  $\,\,$ \\
\end{pmatrix},
\end{equation*}

\begin{equation*}
\mathbf{w}^{\prime} =
\begin{pmatrix}
0.537420\\
0.268710\\
0.109045\\
0.084825
\end{pmatrix} =
0.992750\cdot
\begin{pmatrix}
\color{gr} 0.541345\color{black} \\
0.270672\\
0.109841\\
0.085444
\end{pmatrix},
\end{equation*}
\begin{equation*}
\left[ \frac{{w}^{\prime}_i}{{w}^{\prime}_j} \right] =
\begin{pmatrix}
$\,\,$ 1 $\,\,$ & $\,\,$\color{gr} \color{blue} 2\color{black} $\,\,$ & $\,\,$\color{gr} 4.9284\color{black} $\,\,$ & $\,\,$\color{gr} 6.3356\color{black} $\,\,$ \\
$\,\,$\color{gr} \color{blue}  1/2\color{black} $\,\,$ & $\,\,$ 1 $\,\,$ & $\,\,$2.4642$\,\,$ & $\,\,$3.1678  $\,\,$ \\
$\,\,$\color{gr} 0.2029\color{black} $\,\,$ & $\,\,$0.4058$\,\,$ & $\,\,$ 1 $\,\,$ & $\,\,$1.2855 $\,\,$ \\
$\,\,$\color{gr} 0.1578\color{black} $\,\,$ & $\,\,$0.3157$\,\,$ & $\,\,$0.7779$\,\,$ & $\,\,$ 1  $\,\,$ \\
\end{pmatrix},
\end{equation*}
\end{example}
\newpage
\begin{example}
\begin{equation*}
\mathbf{A} =
\begin{pmatrix}
$\,\,$ 1 $\,\,$ & $\,\,$2$\,\,$ & $\,\,$6$\,\,$ & $\,\,$4 $\,\,$ \\
$\,\,$ 1/2$\,\,$ & $\,\,$ 1 $\,\,$ & $\,\,$5$\,\,$ & $\,\,$9 $\,\,$ \\
$\,\,$ 1/6$\,\,$ & $\,\,$ 1/5$\,\,$ & $\,\,$ 1 $\,\,$ & $\,\,$1 $\,\,$ \\
$\,\,$ 1/4$\,\,$ & $\,\,$ 1/9$\,\,$ & $\,\,$ 1 $\,\,$ & $\,\,$ 1  $\,\,$ \\
\end{pmatrix},
\qquad
\lambda_{\max} =
4.1966,
\qquad
CR = 0.0741
\end{equation*}

\begin{equation*}
\mathbf{w}^{AMAST} =
\begin{pmatrix}
0.462145\\
0.385693\\
\color{red} 0.075864\color{black} \\
0.076297
\end{pmatrix}\end{equation*}
\begin{equation*}
\left[ \frac{{w}^{AMAST}_i}{{w}^{AMAST}_j} \right] =
\begin{pmatrix}
$\,\,$ 1 $\,\,$ & $\,\,$1.1982$\,\,$ & $\,\,$\color{red} 6.0918\color{black} $\,\,$ & $\,\,$6.0572$\,\,$ \\
$\,\,$0.8346$\,\,$ & $\,\,$ 1 $\,\,$ & $\,\,$\color{red} 5.0840\color{black} $\,\,$ & $\,\,$5.0551  $\,\,$ \\
$\,\,$\color{red} 0.1642\color{black} $\,\,$ & $\,\,$\color{red} 0.1967\color{black} $\,\,$ & $\,\,$ 1 $\,\,$ & $\,\,$\color{red} 0.9943\color{black}  $\,\,$ \\
$\,\,$0.1651$\,\,$ & $\,\,$0.1978$\,\,$ & $\,\,$\color{red} 1.0057\color{black} $\,\,$ & $\,\,$ 1  $\,\,$ \\
\end{pmatrix},
\end{equation*}

\begin{equation*}
\mathbf{w}^{\prime} =
\begin{pmatrix}
0.461945\\
0.385526\\
0.076264\\
0.076264
\end{pmatrix} =
0.999567\cdot
\begin{pmatrix}
0.462145\\
0.385693\\
\color{gr} 0.076297\color{black} \\
0.076297
\end{pmatrix},
\end{equation*}
\begin{equation*}
\left[ \frac{{w}^{\prime}_i}{{w}^{\prime}_j} \right] =
\begin{pmatrix}
$\,\,$ 1 $\,\,$ & $\,\,$1.1982$\,\,$ & $\,\,$\color{gr} 6.0572\color{black} $\,\,$ & $\,\,$6.0572$\,\,$ \\
$\,\,$0.8346$\,\,$ & $\,\,$ 1 $\,\,$ & $\,\,$\color{gr} 5.0551\color{black} $\,\,$ & $\,\,$5.0551  $\,\,$ \\
$\,\,$\color{gr} 0.1651\color{black} $\,\,$ & $\,\,$\color{gr} 0.1978\color{black} $\,\,$ & $\,\,$ 1 $\,\,$ & $\,\,$\color{gr} \color{blue} 1\color{black}  $\,\,$ \\
$\,\,$0.1651$\,\,$ & $\,\,$0.1978$\,\,$ & $\,\,$\color{gr} \color{blue} 1\color{black} $\,\,$ & $\,\,$ 1  $\,\,$ \\
\end{pmatrix},
\end{equation*}
\end{example}
\newpage
\begin{example}
\begin{equation*}
\mathbf{A} =
\begin{pmatrix}
$\,\,$ 1 $\,\,$ & $\,\,$2$\,\,$ & $\,\,$6$\,\,$ & $\,\,$6 $\,\,$ \\
$\,\,$ 1/2$\,\,$ & $\,\,$ 1 $\,\,$ & $\,\,$2$\,\,$ & $\,\,$4 $\,\,$ \\
$\,\,$ 1/6$\,\,$ & $\,\,$ 1/2$\,\,$ & $\,\,$ 1 $\,\,$ & $\,\,$3 $\,\,$ \\
$\,\,$ 1/6$\,\,$ & $\,\,$ 1/4$\,\,$ & $\,\,$ 1/3$\,\,$ & $\,\,$ 1  $\,\,$ \\
\end{pmatrix},
\qquad
\lambda_{\max} =
4.1031,
\qquad
CR = 0.0389
\end{equation*}

\begin{equation*}
\mathbf{w}^{AMAST} =
\begin{pmatrix}
0.536135\\
\color{red} 0.262722\color{black} \\
0.135366\\
0.065777
\end{pmatrix}\end{equation*}
\begin{equation*}
\left[ \frac{{w}^{AMAST}_i}{{w}^{AMAST}_j} \right] =
\begin{pmatrix}
$\,\,$ 1 $\,\,$ & $\,\,$\color{red} 2.0407\color{black} $\,\,$ & $\,\,$3.9606$\,\,$ & $\,\,$8.1508$\,\,$ \\
$\,\,$\color{red} 0.4900\color{black} $\,\,$ & $\,\,$ 1 $\,\,$ & $\,\,$\color{red} 1.9408\color{black} $\,\,$ & $\,\,$\color{red} 3.9941\color{black}   $\,\,$ \\
$\,\,$0.2525$\,\,$ & $\,\,$\color{red} 0.5152\color{black} $\,\,$ & $\,\,$ 1 $\,\,$ & $\,\,$2.0580 $\,\,$ \\
$\,\,$0.1227$\,\,$ & $\,\,$\color{red} 0.2504\color{black} $\,\,$ & $\,\,$0.4859$\,\,$ & $\,\,$ 1  $\,\,$ \\
\end{pmatrix},
\end{equation*}

\begin{equation*}
\mathbf{w}^{\prime} =
\begin{pmatrix}
0.535929\\
0.263006\\
0.135314\\
0.065751
\end{pmatrix} =
0.999615\cdot
\begin{pmatrix}
0.536135\\
\color{gr} 0.263107\color{black} \\
0.135366\\
0.065777
\end{pmatrix},
\end{equation*}
\begin{equation*}
\left[ \frac{{w}^{\prime}_i}{{w}^{\prime}_j} \right] =
\begin{pmatrix}
$\,\,$ 1 $\,\,$ & $\,\,$\color{gr} 2.0377\color{black} $\,\,$ & $\,\,$3.9606$\,\,$ & $\,\,$8.1508$\,\,$ \\
$\,\,$\color{gr} 0.4907\color{black} $\,\,$ & $\,\,$ 1 $\,\,$ & $\,\,$\color{gr} 1.9437\color{black} $\,\,$ & $\,\,$\color{gr} \color{blue} 4\color{black}   $\,\,$ \\
$\,\,$0.2525$\,\,$ & $\,\,$\color{gr} 0.5145\color{black} $\,\,$ & $\,\,$ 1 $\,\,$ & $\,\,$2.0580 $\,\,$ \\
$\,\,$0.1227$\,\,$ & $\,\,$\color{gr} \color{blue}  1/4\color{black} $\,\,$ & $\,\,$0.4859$\,\,$ & $\,\,$ 1  $\,\,$ \\
\end{pmatrix},
\end{equation*}
\end{example}
\newpage
\begin{example}
\begin{equation*}
\mathbf{A} =
\begin{pmatrix}
$\,\,$ 1 $\,\,$ & $\,\,$2$\,\,$ & $\,\,$6$\,\,$ & $\,\,$6 $\,\,$ \\
$\,\,$ 1/2$\,\,$ & $\,\,$ 1 $\,\,$ & $\,\,$5$\,\,$ & $\,\,$2 $\,\,$ \\
$\,\,$ 1/6$\,\,$ & $\,\,$ 1/5$\,\,$ & $\,\,$ 1 $\,\,$ & $\,\,$2 $\,\,$ \\
$\,\,$ 1/6$\,\,$ & $\,\,$ 1/2$\,\,$ & $\,\,$ 1/2$\,\,$ & $\,\,$ 1  $\,\,$ \\
\end{pmatrix},
\qquad
\lambda_{\max} =
4.2277,
\qquad
CR = 0.0859
\end{equation*}

\begin{equation*}
\mathbf{w}^{AMAST} =
\begin{pmatrix}
\color{red} 0.528529\color{black} \\
0.282240\\
0.100402\\
0.088829
\end{pmatrix}\end{equation*}
\begin{equation*}
\left[ \frac{{w}^{AMAST}_i}{{w}^{AMAST}_j} \right] =
\begin{pmatrix}
$\,\,$ 1 $\,\,$ & $\,\,$\color{red} 1.8726\color{black} $\,\,$ & $\,\,$\color{red} 5.2641\color{black} $\,\,$ & $\,\,$\color{red} 5.9499\color{black} $\,\,$ \\
$\,\,$\color{red} 0.5340\color{black} $\,\,$ & $\,\,$ 1 $\,\,$ & $\,\,$2.8111$\,\,$ & $\,\,$3.1773  $\,\,$ \\
$\,\,$\color{red} 0.1900\color{black} $\,\,$ & $\,\,$0.3557$\,\,$ & $\,\,$ 1 $\,\,$ & $\,\,$1.1303 $\,\,$ \\
$\,\,$\color{red} 0.1681\color{black} $\,\,$ & $\,\,$0.3147$\,\,$ & $\,\,$0.8847$\,\,$ & $\,\,$ 1  $\,\,$ \\
\end{pmatrix},
\end{equation*}

\begin{equation*}
\mathbf{w}^{\prime} =
\begin{pmatrix}
0.530616\\
0.280991\\
0.099957\\
0.088436
\end{pmatrix} =
0.995573\cdot
\begin{pmatrix}
\color{gr} 0.532976\color{black} \\
0.282240\\
0.100402\\
0.088829
\end{pmatrix},
\end{equation*}
\begin{equation*}
\left[ \frac{{w}^{\prime}_i}{{w}^{\prime}_j} \right] =
\begin{pmatrix}
$\,\,$ 1 $\,\,$ & $\,\,$\color{gr} 1.8884\color{black} $\,\,$ & $\,\,$\color{gr} 5.3084\color{black} $\,\,$ & $\,\,$\color{gr} \color{blue} 6\color{black} $\,\,$ \\
$\,\,$\color{gr} 0.5296\color{black} $\,\,$ & $\,\,$ 1 $\,\,$ & $\,\,$2.8111$\,\,$ & $\,\,$3.1773  $\,\,$ \\
$\,\,$\color{gr} 0.1884\color{black} $\,\,$ & $\,\,$0.3557$\,\,$ & $\,\,$ 1 $\,\,$ & $\,\,$1.1303 $\,\,$ \\
$\,\,$\color{gr} \color{blue}  1/6\color{black} $\,\,$ & $\,\,$0.3147$\,\,$ & $\,\,$0.8847$\,\,$ & $\,\,$ 1  $\,\,$ \\
\end{pmatrix},
\end{equation*}
\end{example}
\newpage
\begin{example}
\begin{equation*}
\mathbf{A} =
\begin{pmatrix}
$\,\,$ 1 $\,\,$ & $\,\,$2$\,\,$ & $\,\,$6$\,\,$ & $\,\,$7 $\,\,$ \\
$\,\,$ 1/2$\,\,$ & $\,\,$ 1 $\,\,$ & $\,\,$2$\,\,$ & $\,\,$5 $\,\,$ \\
$\,\,$ 1/6$\,\,$ & $\,\,$ 1/2$\,\,$ & $\,\,$ 1 $\,\,$ & $\,\,$4 $\,\,$ \\
$\,\,$ 1/7$\,\,$ & $\,\,$ 1/5$\,\,$ & $\,\,$ 1/4$\,\,$ & $\,\,$ 1  $\,\,$ \\
\end{pmatrix},
\qquad
\lambda_{\max} =
4.1301,
\qquad
CR = 0.0490
\end{equation*}

\begin{equation*}
\mathbf{w}^{AMAST} =
\begin{pmatrix}
0.536577\\
\color{red} 0.267918\color{black} \\
0.141234\\
0.054271
\end{pmatrix}\end{equation*}
\begin{equation*}
\left[ \frac{{w}^{AMAST}_i}{{w}^{AMAST}_j} \right] =
\begin{pmatrix}
$\,\,$ 1 $\,\,$ & $\,\,$\color{red} 2.0028\color{black} $\,\,$ & $\,\,$3.7992$\,\,$ & $\,\,$9.8869$\,\,$ \\
$\,\,$\color{red} 0.4993\color{black} $\,\,$ & $\,\,$ 1 $\,\,$ & $\,\,$\color{red} 1.8970\color{black} $\,\,$ & $\,\,$\color{red} 4.9366\color{black}   $\,\,$ \\
$\,\,$0.2632$\,\,$ & $\,\,$\color{red} 0.5272\color{black} $\,\,$ & $\,\,$ 1 $\,\,$ & $\,\,$2.6024 $\,\,$ \\
$\,\,$0.1011$\,\,$ & $\,\,$\color{red} 0.2026\color{black} $\,\,$ & $\,\,$0.3843$\,\,$ & $\,\,$ 1  $\,\,$ \\
\end{pmatrix},
\end{equation*}

\begin{equation*}
\mathbf{w}^{\prime} =
\begin{pmatrix}
0.536378\\
0.268189\\
0.141181\\
0.054251
\end{pmatrix} =
0.999629\cdot
\begin{pmatrix}
0.536577\\
\color{gr} 0.268289\color{black} \\
0.141234\\
0.054271
\end{pmatrix},
\end{equation*}
\begin{equation*}
\left[ \frac{{w}^{\prime}_i}{{w}^{\prime}_j} \right] =
\begin{pmatrix}
$\,\,$ 1 $\,\,$ & $\,\,$\color{gr} \color{blue} 2\color{black} $\,\,$ & $\,\,$3.7992$\,\,$ & $\,\,$9.8869$\,\,$ \\
$\,\,$\color{gr} \color{blue}  1/2\color{black} $\,\,$ & $\,\,$ 1 $\,\,$ & $\,\,$\color{gr} 1.8996\color{black} $\,\,$ & $\,\,$\color{gr} 4.9435\color{black}   $\,\,$ \\
$\,\,$0.2632$\,\,$ & $\,\,$\color{gr} 0.5264\color{black} $\,\,$ & $\,\,$ 1 $\,\,$ & $\,\,$2.6024 $\,\,$ \\
$\,\,$0.1011$\,\,$ & $\,\,$\color{gr} 0.2023\color{black} $\,\,$ & $\,\,$0.3843$\,\,$ & $\,\,$ 1  $\,\,$ \\
\end{pmatrix},
\end{equation*}
\end{example}
\newpage
\begin{example}
\begin{equation*}
\mathbf{A} =
\begin{pmatrix}
$\,\,$ 1 $\,\,$ & $\,\,$2$\,\,$ & $\,\,$6$\,\,$ & $\,\,$7 $\,\,$ \\
$\,\,$ 1/2$\,\,$ & $\,\,$ 1 $\,\,$ & $\,\,$5$\,\,$ & $\,\,$2 $\,\,$ \\
$\,\,$ 1/6$\,\,$ & $\,\,$ 1/5$\,\,$ & $\,\,$ 1 $\,\,$ & $\,\,$2 $\,\,$ \\
$\,\,$ 1/7$\,\,$ & $\,\,$ 1/2$\,\,$ & $\,\,$ 1/2$\,\,$ & $\,\,$ 1  $\,\,$ \\
\end{pmatrix},
\qquad
\lambda_{\max} =
4.2251,
\qquad
CR = 0.0849
\end{equation*}

\begin{equation*}
\mathbf{w}^{AMAST} =
\begin{pmatrix}
\color{red} 0.539297\color{black} \\
0.278376\\
0.098169\\
0.084158
\end{pmatrix}\end{equation*}
\begin{equation*}
\left[ \frac{{w}^{AMAST}_i}{{w}^{AMAST}_j} \right] =
\begin{pmatrix}
$\,\,$ 1 $\,\,$ & $\,\,$\color{red} 1.9373\color{black} $\,\,$ & $\,\,$\color{red} 5.4936\color{black} $\,\,$ & $\,\,$\color{red} 6.4081\color{black} $\,\,$ \\
$\,\,$\color{red} 0.5162\color{black} $\,\,$ & $\,\,$ 1 $\,\,$ & $\,\,$2.8357$\,\,$ & $\,\,$3.3078  $\,\,$ \\
$\,\,$\color{red} 0.1820\color{black} $\,\,$ & $\,\,$0.3526$\,\,$ & $\,\,$ 1 $\,\,$ & $\,\,$1.1665 $\,\,$ \\
$\,\,$\color{red} 0.1561\color{black} $\,\,$ & $\,\,$0.3023$\,\,$ & $\,\,$0.8573$\,\,$ & $\,\,$ 1  $\,\,$ \\
\end{pmatrix},
\end{equation*}

\begin{equation*}
\mathbf{w}^{\prime} =
\begin{pmatrix}
0.547200\\
0.273600\\
0.096485\\
0.082715
\end{pmatrix} =
0.982845\cdot
\begin{pmatrix}
\color{gr} 0.556751\color{black} \\
0.278376\\
0.098169\\
0.084158
\end{pmatrix},
\end{equation*}
\begin{equation*}
\left[ \frac{{w}^{\prime}_i}{{w}^{\prime}_j} \right] =
\begin{pmatrix}
$\,\,$ 1 $\,\,$ & $\,\,$\color{gr} \color{blue} 2\color{black} $\,\,$ & $\,\,$\color{gr} 5.6714\color{black} $\,\,$ & $\,\,$\color{gr} 6.6155\color{black} $\,\,$ \\
$\,\,$\color{gr} \color{blue}  1/2\color{black} $\,\,$ & $\,\,$ 1 $\,\,$ & $\,\,$2.8357$\,\,$ & $\,\,$3.3078  $\,\,$ \\
$\,\,$\color{gr} 0.1763\color{black} $\,\,$ & $\,\,$0.3526$\,\,$ & $\,\,$ 1 $\,\,$ & $\,\,$1.1665 $\,\,$ \\
$\,\,$\color{gr} 0.1512\color{black} $\,\,$ & $\,\,$0.3023$\,\,$ & $\,\,$0.8573$\,\,$ & $\,\,$ 1  $\,\,$ \\
\end{pmatrix},
\end{equation*}
\end{example}
\newpage
\begin{example}
\begin{equation*}
\mathbf{A} =
\begin{pmatrix}
$\,\,$ 1 $\,\,$ & $\,\,$2$\,\,$ & $\,\,$6$\,\,$ & $\,\,$8 $\,\,$ \\
$\,\,$ 1/2$\,\,$ & $\,\,$ 1 $\,\,$ & $\,\,$5$\,\,$ & $\,\,$2 $\,\,$ \\
$\,\,$ 1/6$\,\,$ & $\,\,$ 1/5$\,\,$ & $\,\,$ 1 $\,\,$ & $\,\,$2 $\,\,$ \\
$\,\,$ 1/8$\,\,$ & $\,\,$ 1/2$\,\,$ & $\,\,$ 1/2$\,\,$ & $\,\,$ 1  $\,\,$ \\
\end{pmatrix},
\qquad
\lambda_{\max} =
4.2277,
\qquad
CR = 0.0859
\end{equation*}

\begin{equation*}
\mathbf{w}^{AMAST} =
\begin{pmatrix}
\color{red} 0.548275\color{black} \\
0.274998\\
0.096296\\
0.080432
\end{pmatrix}\end{equation*}
\begin{equation*}
\left[ \frac{{w}^{AMAST}_i}{{w}^{AMAST}_j} \right] =
\begin{pmatrix}
$\,\,$ 1 $\,\,$ & $\,\,$\color{red} 1.9937\color{black} $\,\,$ & $\,\,$\color{red} 5.6937\color{black} $\,\,$ & $\,\,$\color{red} 6.8167\color{black} $\,\,$ \\
$\,\,$\color{red} 0.5016\color{black} $\,\,$ & $\,\,$ 1 $\,\,$ & $\,\,$2.8558$\,\,$ & $\,\,$3.4190  $\,\,$ \\
$\,\,$\color{red} 0.1756\color{black} $\,\,$ & $\,\,$0.3502$\,\,$ & $\,\,$ 1 $\,\,$ & $\,\,$1.1972 $\,\,$ \\
$\,\,$\color{red} 0.1467\color{black} $\,\,$ & $\,\,$0.2925$\,\,$ & $\,\,$0.8353$\,\,$ & $\,\,$ 1  $\,\,$ \\
\end{pmatrix},
\end{equation*}

\begin{equation*}
\mathbf{w}^{\prime} =
\begin{pmatrix}
0.549051\\
0.274526\\
0.096130\\
0.080293
\end{pmatrix} =
0.998282\cdot
\begin{pmatrix}
\color{gr} 0.549996\color{black} \\
0.274998\\
0.096296\\
0.080432
\end{pmatrix},
\end{equation*}
\begin{equation*}
\left[ \frac{{w}^{\prime}_i}{{w}^{\prime}_j} \right] =
\begin{pmatrix}
$\,\,$ 1 $\,\,$ & $\,\,$\color{gr} \color{blue} 2\color{black} $\,\,$ & $\,\,$\color{gr} 5.7115\color{black} $\,\,$ & $\,\,$\color{gr} 6.8381\color{black} $\,\,$ \\
$\,\,$\color{gr} \color{blue}  1/2\color{black} $\,\,$ & $\,\,$ 1 $\,\,$ & $\,\,$2.8558$\,\,$ & $\,\,$3.4190  $\,\,$ \\
$\,\,$\color{gr} 0.1751\color{black} $\,\,$ & $\,\,$0.3502$\,\,$ & $\,\,$ 1 $\,\,$ & $\,\,$1.1972 $\,\,$ \\
$\,\,$\color{gr} 0.1462\color{black} $\,\,$ & $\,\,$0.2925$\,\,$ & $\,\,$0.8353$\,\,$ & $\,\,$ 1  $\,\,$ \\
\end{pmatrix},
\end{equation*}
\end{example}
\newpage
\begin{example}
\begin{equation*}
\mathbf{A} =
\begin{pmatrix}
$\,\,$ 1 $\,\,$ & $\,\,$2$\,\,$ & $\,\,$7$\,\,$ & $\,\,$5 $\,\,$ \\
$\,\,$ 1/2$\,\,$ & $\,\,$ 1 $\,\,$ & $\,\,$2$\,\,$ & $\,\,$4 $\,\,$ \\
$\,\,$ 1/7$\,\,$ & $\,\,$ 1/2$\,\,$ & $\,\,$ 1 $\,\,$ & $\,\,$3 $\,\,$ \\
$\,\,$ 1/5$\,\,$ & $\,\,$ 1/4$\,\,$ & $\,\,$ 1/3$\,\,$ & $\,\,$ 1  $\,\,$ \\
\end{pmatrix},
\qquad
\lambda_{\max} =
4.1782,
\qquad
CR = 0.0672
\end{equation*}

\begin{equation*}
\mathbf{w}^{AMAST} =
\begin{pmatrix}
0.531861\\
\color{red} 0.264054\color{black} \\
0.133524\\
0.070561
\end{pmatrix}\end{equation*}
\begin{equation*}
\left[ \frac{{w}^{AMAST}_i}{{w}^{AMAST}_j} \right] =
\begin{pmatrix}
$\,\,$ 1 $\,\,$ & $\,\,$\color{red} 2.0142\color{black} $\,\,$ & $\,\,$3.9833$\,\,$ & $\,\,$7.5377$\,\,$ \\
$\,\,$\color{red} 0.4965\color{black} $\,\,$ & $\,\,$ 1 $\,\,$ & $\,\,$\color{red} 1.9776\color{black} $\,\,$ & $\,\,$\color{red} 3.7422\color{black}   $\,\,$ \\
$\,\,$0.2511$\,\,$ & $\,\,$\color{red} 0.5057\color{black} $\,\,$ & $\,\,$ 1 $\,\,$ & $\,\,$1.8923 $\,\,$ \\
$\,\,$0.1327$\,\,$ & $\,\,$\color{red} 0.2672\color{black} $\,\,$ & $\,\,$0.5284$\,\,$ & $\,\,$ 1  $\,\,$ \\
\end{pmatrix},
\end{equation*}

\begin{equation*}
\mathbf{w}^{\prime} =
\begin{pmatrix}
0.530865\\
0.265432\\
0.133274\\
0.070428
\end{pmatrix} =
0.998127\cdot
\begin{pmatrix}
0.531861\\
\color{gr} 0.265930\color{black} \\
0.133524\\
0.070561
\end{pmatrix},
\end{equation*}
\begin{equation*}
\left[ \frac{{w}^{\prime}_i}{{w}^{\prime}_j} \right] =
\begin{pmatrix}
$\,\,$ 1 $\,\,$ & $\,\,$\color{gr} \color{blue} 2\color{black} $\,\,$ & $\,\,$3.9833$\,\,$ & $\,\,$7.5377$\,\,$ \\
$\,\,$\color{gr} \color{blue}  1/2\color{black} $\,\,$ & $\,\,$ 1 $\,\,$ & $\,\,$\color{gr} 1.9916\color{black} $\,\,$ & $\,\,$\color{gr} 3.7688\color{black}   $\,\,$ \\
$\,\,$0.2511$\,\,$ & $\,\,$\color{gr} 0.5021\color{black} $\,\,$ & $\,\,$ 1 $\,\,$ & $\,\,$1.8923 $\,\,$ \\
$\,\,$0.1327$\,\,$ & $\,\,$\color{gr} 0.2653\color{black} $\,\,$ & $\,\,$0.5284$\,\,$ & $\,\,$ 1  $\,\,$ \\
\end{pmatrix},
\end{equation*}
\end{example}
\newpage
\begin{example}
\begin{equation*}
\mathbf{A} =
\begin{pmatrix}
$\,\,$ 1 $\,\,$ & $\,\,$2$\,\,$ & $\,\,$7$\,\,$ & $\,\,$5 $\,\,$ \\
$\,\,$ 1/2$\,\,$ & $\,\,$ 1 $\,\,$ & $\,\,$2$\,\,$ & $\,\,$4 $\,\,$ \\
$\,\,$ 1/7$\,\,$ & $\,\,$ 1/2$\,\,$ & $\,\,$ 1 $\,\,$ & $\,\,$4 $\,\,$ \\
$\,\,$ 1/5$\,\,$ & $\,\,$ 1/4$\,\,$ & $\,\,$ 1/4$\,\,$ & $\,\,$ 1  $\,\,$ \\
\end{pmatrix},
\qquad
\lambda_{\max} =
4.2610,
\qquad
CR = 0.0984
\end{equation*}

\begin{equation*}
\mathbf{w}^{AMAST} =
\begin{pmatrix}
0.525696\\
\color{red} 0.262014\color{black} \\
0.145506\\
0.066784
\end{pmatrix}\end{equation*}
\begin{equation*}
\left[ \frac{{w}^{AMAST}_i}{{w}^{AMAST}_j} \right] =
\begin{pmatrix}
$\,\,$ 1 $\,\,$ & $\,\,$\color{red} 2.0064\color{black} $\,\,$ & $\,\,$3.6129$\,\,$ & $\,\,$7.8716$\,\,$ \\
$\,\,$\color{red} 0.4984\color{black} $\,\,$ & $\,\,$ 1 $\,\,$ & $\,\,$\color{red} 1.8007\color{black} $\,\,$ & $\,\,$\color{red} 3.9233\color{black}   $\,\,$ \\
$\,\,$0.2768$\,\,$ & $\,\,$\color{red} 0.5553\color{black} $\,\,$ & $\,\,$ 1 $\,\,$ & $\,\,$2.1788 $\,\,$ \\
$\,\,$0.1270$\,\,$ & $\,\,$\color{red} 0.2549\color{black} $\,\,$ & $\,\,$0.4590$\,\,$ & $\,\,$ 1  $\,\,$ \\
\end{pmatrix},
\end{equation*}

\begin{equation*}
\mathbf{w}^{\prime} =
\begin{pmatrix}
0.525258\\
0.262629\\
0.145385\\
0.066728
\end{pmatrix} =
0.999166\cdot
\begin{pmatrix}
0.525696\\
\color{gr} 0.262848\color{black} \\
0.145506\\
0.066784
\end{pmatrix},
\end{equation*}
\begin{equation*}
\left[ \frac{{w}^{\prime}_i}{{w}^{\prime}_j} \right] =
\begin{pmatrix}
$\,\,$ 1 $\,\,$ & $\,\,$\color{gr} \color{blue} 2\color{black} $\,\,$ & $\,\,$3.6129$\,\,$ & $\,\,$7.8716$\,\,$ \\
$\,\,$\color{gr} \color{blue}  1/2\color{black} $\,\,$ & $\,\,$ 1 $\,\,$ & $\,\,$\color{gr} 1.8064\color{black} $\,\,$ & $\,\,$\color{gr} 3.9358\color{black}   $\,\,$ \\
$\,\,$0.2768$\,\,$ & $\,\,$\color{gr} 0.5536\color{black} $\,\,$ & $\,\,$ 1 $\,\,$ & $\,\,$2.1788 $\,\,$ \\
$\,\,$0.1270$\,\,$ & $\,\,$\color{gr} 0.2541\color{black} $\,\,$ & $\,\,$0.4590$\,\,$ & $\,\,$ 1  $\,\,$ \\
\end{pmatrix},
\end{equation*}
\end{example}
\newpage
\begin{example}
\begin{equation*}
\mathbf{A} =
\begin{pmatrix}
$\,\,$ 1 $\,\,$ & $\,\,$2$\,\,$ & $\,\,$7$\,\,$ & $\,\,$6 $\,\,$ \\
$\,\,$ 1/2$\,\,$ & $\,\,$ 1 $\,\,$ & $\,\,$2$\,\,$ & $\,\,$4 $\,\,$ \\
$\,\,$ 1/7$\,\,$ & $\,\,$ 1/2$\,\,$ & $\,\,$ 1 $\,\,$ & $\,\,$3 $\,\,$ \\
$\,\,$ 1/6$\,\,$ & $\,\,$ 1/4$\,\,$ & $\,\,$ 1/3$\,\,$ & $\,\,$ 1  $\,\,$ \\
\end{pmatrix},
\qquad
\lambda_{\max} =
4.1365,
\qquad
CR = 0.0515
\end{equation*}

\begin{equation*}
\mathbf{w}^{AMAST} =
\begin{pmatrix}
0.546133\\
\color{red} 0.258923\color{black} \\
0.129725\\
0.065219
\end{pmatrix}\end{equation*}
\begin{equation*}
\left[ \frac{{w}^{AMAST}_i}{{w}^{AMAST}_j} \right] =
\begin{pmatrix}
$\,\,$ 1 $\,\,$ & $\,\,$\color{red} 2.1092\color{black} $\,\,$ & $\,\,$4.2099$\,\,$ & $\,\,$8.3738$\,\,$ \\
$\,\,$\color{red} 0.4741\color{black} $\,\,$ & $\,\,$ 1 $\,\,$ & $\,\,$\color{red} 1.9959\color{black} $\,\,$ & $\,\,$\color{red} 3.9700\color{black}   $\,\,$ \\
$\,\,$0.2375$\,\,$ & $\,\,$\color{red} 0.5010\color{black} $\,\,$ & $\,\,$ 1 $\,\,$ & $\,\,$1.9891 $\,\,$ \\
$\,\,$0.1194$\,\,$ & $\,\,$\color{red} 0.2519\color{black} $\,\,$ & $\,\,$0.5028$\,\,$ & $\,\,$ 1  $\,\,$ \\
\end{pmatrix},
\end{equation*}

\begin{equation*}
\mathbf{w}^{\prime} =
\begin{pmatrix}
0.545845\\
0.259313\\
0.129657\\
0.065185
\end{pmatrix} =
0.999474\cdot
\begin{pmatrix}
0.546133\\
\color{gr} 0.259450\color{black} \\
0.129725\\
0.065219
\end{pmatrix},
\end{equation*}
\begin{equation*}
\left[ \frac{{w}^{\prime}_i}{{w}^{\prime}_j} \right] =
\begin{pmatrix}
$\,\,$ 1 $\,\,$ & $\,\,$\color{gr} 2.1050\color{black} $\,\,$ & $\,\,$4.2099$\,\,$ & $\,\,$8.3738$\,\,$ \\
$\,\,$\color{gr} 0.4751\color{black} $\,\,$ & $\,\,$ 1 $\,\,$ & $\,\,$\color{gr} \color{blue} 2\color{black} $\,\,$ & $\,\,$\color{gr} 3.9781\color{black}   $\,\,$ \\
$\,\,$0.2375$\,\,$ & $\,\,$\color{gr} \color{blue}  1/2\color{black} $\,\,$ & $\,\,$ 1 $\,\,$ & $\,\,$1.9891 $\,\,$ \\
$\,\,$0.1194$\,\,$ & $\,\,$\color{gr} 0.2514\color{black} $\,\,$ & $\,\,$0.5028$\,\,$ & $\,\,$ 1  $\,\,$ \\
\end{pmatrix},
\end{equation*}
\end{example}
\newpage
\begin{example}
\begin{equation*}
\mathbf{A} =
\begin{pmatrix}
$\,\,$ 1 $\,\,$ & $\,\,$2$\,\,$ & $\,\,$7$\,\,$ & $\,\,$7 $\,\,$ \\
$\,\,$ 1/2$\,\,$ & $\,\,$ 1 $\,\,$ & $\,\,$2$\,\,$ & $\,\,$5 $\,\,$ \\
$\,\,$ 1/7$\,\,$ & $\,\,$ 1/2$\,\,$ & $\,\,$ 1 $\,\,$ & $\,\,$4 $\,\,$ \\
$\,\,$ 1/7$\,\,$ & $\,\,$ 1/5$\,\,$ & $\,\,$ 1/4$\,\,$ & $\,\,$ 1  $\,\,$ \\
\end{pmatrix},
\qquad
\lambda_{\max} =
4.1665,
\qquad
CR = 0.0628
\end{equation*}

\begin{equation*}
\mathbf{w}^{AMAST} =
\begin{pmatrix}
0.546518\\
\color{red} 0.264095\color{black} \\
0.135511\\
0.053876
\end{pmatrix}\end{equation*}
\begin{equation*}
\left[ \frac{{w}^{AMAST}_i}{{w}^{AMAST}_j} \right] =
\begin{pmatrix}
$\,\,$ 1 $\,\,$ & $\,\,$\color{red} 2.0694\color{black} $\,\,$ & $\,\,$4.0330$\,\,$ & $\,\,$10.1439$\,\,$ \\
$\,\,$\color{red} 0.4832\color{black} $\,\,$ & $\,\,$ 1 $\,\,$ & $\,\,$\color{red} 1.9489\color{black} $\,\,$ & $\,\,$\color{red} 4.9019\color{black}   $\,\,$ \\
$\,\,$0.2480$\,\,$ & $\,\,$\color{red} 0.5131\color{black} $\,\,$ & $\,\,$ 1 $\,\,$ & $\,\,$2.5152 $\,\,$ \\
$\,\,$0.0986$\,\,$ & $\,\,$\color{red} 0.2040\color{black} $\,\,$ & $\,\,$0.3976$\,\,$ & $\,\,$ 1  $\,\,$ \\
\end{pmatrix},
\end{equation*}

\begin{equation*}
\mathbf{w}^{\prime} =
\begin{pmatrix}
0.543643\\
0.267965\\
0.134798\\
0.053593
\end{pmatrix} =
0.994740\cdot
\begin{pmatrix}
0.546518\\
\color{gr} 0.269382\color{black} \\
0.135511\\
0.053876
\end{pmatrix},
\end{equation*}
\begin{equation*}
\left[ \frac{{w}^{\prime}_i}{{w}^{\prime}_j} \right] =
\begin{pmatrix}
$\,\,$ 1 $\,\,$ & $\,\,$\color{gr} 2.0288\color{black} $\,\,$ & $\,\,$4.0330$\,\,$ & $\,\,$10.1439$\,\,$ \\
$\,\,$\color{gr} 0.4929\color{black} $\,\,$ & $\,\,$ 1 $\,\,$ & $\,\,$\color{gr} 1.9879\color{black} $\,\,$ & $\,\,$\color{gr} \color{blue} 5\color{black}   $\,\,$ \\
$\,\,$0.2480$\,\,$ & $\,\,$\color{gr} 0.5030\color{black} $\,\,$ & $\,\,$ 1 $\,\,$ & $\,\,$2.5152 $\,\,$ \\
$\,\,$0.0986$\,\,$ & $\,\,$\color{gr} \color{blue}  1/5\color{black} $\,\,$ & $\,\,$0.3976$\,\,$ & $\,\,$ 1  $\,\,$ \\
\end{pmatrix},
\end{equation*}
\end{example}
\newpage
\begin{example}
\begin{equation*}
\mathbf{A} =
\begin{pmatrix}
$\,\,$ 1 $\,\,$ & $\,\,$2$\,\,$ & $\,\,$7$\,\,$ & $\,\,$8 $\,\,$ \\
$\,\,$ 1/2$\,\,$ & $\,\,$ 1 $\,\,$ & $\,\,$2$\,\,$ & $\,\,$6 $\,\,$ \\
$\,\,$ 1/7$\,\,$ & $\,\,$ 1/2$\,\,$ & $\,\,$ 1 $\,\,$ & $\,\,$5 $\,\,$ \\
$\,\,$ 1/8$\,\,$ & $\,\,$ 1/6$\,\,$ & $\,\,$ 1/5$\,\,$ & $\,\,$ 1  $\,\,$ \\
\end{pmatrix},
\qquad
\lambda_{\max} =
4.1888,
\qquad
CR = 0.0712
\end{equation*}

\begin{equation*}
\mathbf{w}^{AMAST} =
\begin{pmatrix}
0.546716\\
\color{red} 0.267744\color{black} \\
0.139543\\
0.045998
\end{pmatrix}\end{equation*}
\begin{equation*}
\left[ \frac{{w}^{AMAST}_i}{{w}^{AMAST}_j} \right] =
\begin{pmatrix}
$\,\,$ 1 $\,\,$ & $\,\,$\color{red} 2.0419\color{black} $\,\,$ & $\,\,$3.9179$\,\,$ & $\,\,$11.8856$\,\,$ \\
$\,\,$\color{red} 0.4897\color{black} $\,\,$ & $\,\,$ 1 $\,\,$ & $\,\,$\color{red} 1.9187\color{black} $\,\,$ & $\,\,$\color{red} 5.8208\color{black}   $\,\,$ \\
$\,\,$0.2552$\,\,$ & $\,\,$\color{red} 0.5212\color{black} $\,\,$ & $\,\,$ 1 $\,\,$ & $\,\,$3.0337 $\,\,$ \\
$\,\,$0.0841$\,\,$ & $\,\,$\color{red} 0.1718\color{black} $\,\,$ & $\,\,$0.3296$\,\,$ & $\,\,$ 1  $\,\,$ \\
\end{pmatrix},
\end{equation*}

\begin{equation*}
\mathbf{w}^{\prime} =
\begin{pmatrix}
0.543663\\
0.271832\\
0.138763\\
0.045741
\end{pmatrix} =
0.994417\cdot
\begin{pmatrix}
0.546716\\
\color{gr} 0.273358\color{black} \\
0.139543\\
0.045998
\end{pmatrix},
\end{equation*}
\begin{equation*}
\left[ \frac{{w}^{\prime}_i}{{w}^{\prime}_j} \right] =
\begin{pmatrix}
$\,\,$ 1 $\,\,$ & $\,\,$\color{gr} \color{blue} 2\color{black} $\,\,$ & $\,\,$3.9179$\,\,$ & $\,\,$11.8856$\,\,$ \\
$\,\,$\color{gr} \color{blue}  1/2\color{black} $\,\,$ & $\,\,$ 1 $\,\,$ & $\,\,$\color{gr} 1.9590\color{black} $\,\,$ & $\,\,$\color{gr} 5.9428\color{black}   $\,\,$ \\
$\,\,$0.2552$\,\,$ & $\,\,$\color{gr} 0.5105\color{black} $\,\,$ & $\,\,$ 1 $\,\,$ & $\,\,$3.0337 $\,\,$ \\
$\,\,$0.0841$\,\,$ & $\,\,$\color{gr} 0.1683\color{black} $\,\,$ & $\,\,$0.3296$\,\,$ & $\,\,$ 1  $\,\,$ \\
\end{pmatrix},
\end{equation*}
\end{example}
\newpage
\begin{example}
\begin{equation*}
\mathbf{A} =
\begin{pmatrix}
$\,\,$ 1 $\,\,$ & $\,\,$2$\,\,$ & $\,\,$7$\,\,$ & $\,\,$8 $\,\,$ \\
$\,\,$ 1/2$\,\,$ & $\,\,$ 1 $\,\,$ & $\,\,$2$\,\,$ & $\,\,$6 $\,\,$ \\
$\,\,$ 1/7$\,\,$ & $\,\,$ 1/2$\,\,$ & $\,\,$ 1 $\,\,$ & $\,\,$6 $\,\,$ \\
$\,\,$ 1/8$\,\,$ & $\,\,$ 1/6$\,\,$ & $\,\,$ 1/6$\,\,$ & $\,\,$ 1  $\,\,$ \\
\end{pmatrix},
\qquad
\lambda_{\max} =
4.2421,
\qquad
CR = 0.0913
\end{equation*}

\begin{equation*}
\mathbf{w}^{AMAST} =
\begin{pmatrix}
0.542092\\
\color{red} 0.266245\color{black} \\
0.147243\\
0.044421
\end{pmatrix}\end{equation*}
\begin{equation*}
\left[ \frac{{w}^{AMAST}_i}{{w}^{AMAST}_j} \right] =
\begin{pmatrix}
$\,\,$ 1 $\,\,$ & $\,\,$\color{red} 2.0361\color{black} $\,\,$ & $\,\,$3.6816$\,\,$ & $\,\,$12.2036$\,\,$ \\
$\,\,$\color{red} 0.4911\color{black} $\,\,$ & $\,\,$ 1 $\,\,$ & $\,\,$\color{red} 1.8082\color{black} $\,\,$ & $\,\,$\color{red} 5.9937\color{black}   $\,\,$ \\
$\,\,$0.2716$\,\,$ & $\,\,$\color{red} 0.5530\color{black} $\,\,$ & $\,\,$ 1 $\,\,$ & $\,\,$3.3147 $\,\,$ \\
$\,\,$0.0819$\,\,$ & $\,\,$\color{red} 0.1668\color{black} $\,\,$ & $\,\,$0.3017$\,\,$ & $\,\,$ 1  $\,\,$ \\
\end{pmatrix},
\end{equation*}

\begin{equation*}
\mathbf{w}^{\prime} =
\begin{pmatrix}
0.541941\\
0.266449\\
0.147202\\
0.044408
\end{pmatrix} =
0.999721\cdot
\begin{pmatrix}
0.542092\\
\color{gr} 0.266523\color{black} \\
0.147243\\
0.044421
\end{pmatrix},
\end{equation*}
\begin{equation*}
\left[ \frac{{w}^{\prime}_i}{{w}^{\prime}_j} \right] =
\begin{pmatrix}
$\,\,$ 1 $\,\,$ & $\,\,$\color{gr} 2.0339\color{black} $\,\,$ & $\,\,$3.6816$\,\,$ & $\,\,$12.2036$\,\,$ \\
$\,\,$\color{gr} 0.4917\color{black} $\,\,$ & $\,\,$ 1 $\,\,$ & $\,\,$\color{gr} 1.8101\color{black} $\,\,$ & $\,\,$\color{gr} \color{blue} 6\color{black}   $\,\,$ \\
$\,\,$0.2716$\,\,$ & $\,\,$\color{gr} 0.5525\color{black} $\,\,$ & $\,\,$ 1 $\,\,$ & $\,\,$3.3147 $\,\,$ \\
$\,\,$0.0819$\,\,$ & $\,\,$\color{gr} \color{blue}  1/6\color{black} $\,\,$ & $\,\,$0.3017$\,\,$ & $\,\,$ 1  $\,\,$ \\
\end{pmatrix},
\end{equation*}
\end{example}
\newpage
\begin{example}
\begin{equation*}
\mathbf{A} =
\begin{pmatrix}
$\,\,$ 1 $\,\,$ & $\,\,$2$\,\,$ & $\,\,$7$\,\,$ & $\,\,$9 $\,\,$ \\
$\,\,$ 1/2$\,\,$ & $\,\,$ 1 $\,\,$ & $\,\,$8$\,\,$ & $\,\,$6 $\,\,$ \\
$\,\,$ 1/7$\,\,$ & $\,\,$ 1/8$\,\,$ & $\,\,$ 1 $\,\,$ & $\,\,$1 $\,\,$ \\
$\,\,$ 1/9$\,\,$ & $\,\,$ 1/6$\,\,$ & $\,\,$ 1 $\,\,$ & $\,\,$ 1  $\,\,$ \\
\end{pmatrix},
\qquad
\lambda_{\max} =
4.0576,
\qquad
CR = 0.0217
\end{equation*}

\begin{equation*}
\mathbf{w}^{AMAST} =
\begin{pmatrix}
0.529447\\
0.352722\\
0.059119\\
\color{red} 0.058713\color{black}
\end{pmatrix}\end{equation*}
\begin{equation*}
\left[ \frac{{w}^{AMAST}_i}{{w}^{AMAST}_j} \right] =
\begin{pmatrix}
$\,\,$ 1 $\,\,$ & $\,\,$1.5010$\,\,$ & $\,\,$8.9556$\,\,$ & $\,\,$\color{red} 9.0176\color{black} $\,\,$ \\
$\,\,$0.6662$\,\,$ & $\,\,$ 1 $\,\,$ & $\,\,$5.9663$\,\,$ & $\,\,$\color{red} 6.0076\color{black}   $\,\,$ \\
$\,\,$0.1117$\,\,$ & $\,\,$0.1676$\,\,$ & $\,\,$ 1 $\,\,$ & $\,\,$\color{red} 1.0069\color{black}  $\,\,$ \\
$\,\,$\color{red} 0.1109\color{black} $\,\,$ & $\,\,$\color{red} 0.1665\color{black} $\,\,$ & $\,\,$\color{red} 0.9931\color{black} $\,\,$ & $\,\,$ 1  $\,\,$ \\
\end{pmatrix},
\end{equation*}

\begin{equation*}
\mathbf{w}^{\prime} =
\begin{pmatrix}
0.529407\\
0.352696\\
0.059114\\
0.058783
\end{pmatrix} =
0.999926\cdot
\begin{pmatrix}
0.529447\\
0.352722\\
0.059119\\
\color{gr} 0.058787\color{black}
\end{pmatrix},
\end{equation*}
\begin{equation*}
\left[ \frac{{w}^{\prime}_i}{{w}^{\prime}_j} \right] =
\begin{pmatrix}
$\,\,$ 1 $\,\,$ & $\,\,$1.5010$\,\,$ & $\,\,$8.9556$\,\,$ & $\,\,$\color{gr} 9.0062\color{black} $\,\,$ \\
$\,\,$0.6662$\,\,$ & $\,\,$ 1 $\,\,$ & $\,\,$5.9663$\,\,$ & $\,\,$\color{gr} \color{blue} 6\color{black}   $\,\,$ \\
$\,\,$0.1117$\,\,$ & $\,\,$0.1676$\,\,$ & $\,\,$ 1 $\,\,$ & $\,\,$\color{gr} 1.0056\color{black}  $\,\,$ \\
$\,\,$\color{gr} 0.1110\color{black} $\,\,$ & $\,\,$\color{gr} \color{blue}  1/6\color{black} $\,\,$ & $\,\,$\color{gr} 0.9944\color{black} $\,\,$ & $\,\,$ 1  $\,\,$ \\
\end{pmatrix},
\end{equation*}
\end{example}
\newpage
\begin{example}
\begin{equation*}
\mathbf{A} =
\begin{pmatrix}
$\,\,$ 1 $\,\,$ & $\,\,$2$\,\,$ & $\,\,$8$\,\,$ & $\,\,$6 $\,\,$ \\
$\,\,$ 1/2$\,\,$ & $\,\,$ 1 $\,\,$ & $\,\,$2$\,\,$ & $\,\,$5 $\,\,$ \\
$\,\,$ 1/8$\,\,$ & $\,\,$ 1/2$\,\,$ & $\,\,$ 1 $\,\,$ & $\,\,$4 $\,\,$ \\
$\,\,$ 1/6$\,\,$ & $\,\,$ 1/5$\,\,$ & $\,\,$ 1/4$\,\,$ & $\,\,$ 1  $\,\,$ \\
\end{pmatrix},
\qquad
\lambda_{\max} =
4.2460,
\qquad
CR = 0.0928
\end{equation*}

\begin{equation*}
\mathbf{w}^{AMAST} =
\begin{pmatrix}
0.542674\\
\color{red} 0.265441\color{black} \\
0.134476\\
0.057409
\end{pmatrix}\end{equation*}
\begin{equation*}
\left[ \frac{{w}^{AMAST}_i}{{w}^{AMAST}_j} \right] =
\begin{pmatrix}
$\,\,$ 1 $\,\,$ & $\,\,$\color{red} 2.0444\color{black} $\,\,$ & $\,\,$4.0355$\,\,$ & $\,\,$9.4528$\,\,$ \\
$\,\,$\color{red} 0.4891\color{black} $\,\,$ & $\,\,$ 1 $\,\,$ & $\,\,$\color{red} 1.9739\color{black} $\,\,$ & $\,\,$\color{red} 4.6237\color{black}   $\,\,$ \\
$\,\,$0.2478$\,\,$ & $\,\,$\color{red} 0.5066\color{black} $\,\,$ & $\,\,$ 1 $\,\,$ & $\,\,$2.3424 $\,\,$ \\
$\,\,$0.1058$\,\,$ & $\,\,$\color{red} 0.2163\color{black} $\,\,$ & $\,\,$0.4269$\,\,$ & $\,\,$ 1  $\,\,$ \\
\end{pmatrix},
\end{equation*}

\begin{equation*}
\mathbf{w}^{\prime} =
\begin{pmatrix}
0.540776\\
0.268010\\
0.134005\\
0.057208
\end{pmatrix} =
0.996503\cdot
\begin{pmatrix}
0.542674\\
\color{gr} 0.268951\color{black} \\
0.134476\\
0.057409
\end{pmatrix},
\end{equation*}
\begin{equation*}
\left[ \frac{{w}^{\prime}_i}{{w}^{\prime}_j} \right] =
\begin{pmatrix}
$\,\,$ 1 $\,\,$ & $\,\,$\color{gr} 2.0177\color{black} $\,\,$ & $\,\,$4.0355$\,\,$ & $\,\,$9.4528$\,\,$ \\
$\,\,$\color{gr} 0.4956\color{black} $\,\,$ & $\,\,$ 1 $\,\,$ & $\,\,$\color{gr} \color{blue} 2\color{black} $\,\,$ & $\,\,$\color{gr} 4.6848\color{black}   $\,\,$ \\
$\,\,$0.2478$\,\,$ & $\,\,$\color{gr} \color{blue}  1/2\color{black} $\,\,$ & $\,\,$ 1 $\,\,$ & $\,\,$2.3424 $\,\,$ \\
$\,\,$0.1058$\,\,$ & $\,\,$\color{gr} 0.2135\color{black} $\,\,$ & $\,\,$0.4269$\,\,$ & $\,\,$ 1  $\,\,$ \\
\end{pmatrix},
\end{equation*}
\end{example}
\newpage
\begin{example}
\begin{equation*}
\mathbf{A} =
\begin{pmatrix}
$\,\,$ 1 $\,\,$ & $\,\,$2$\,\,$ & $\,\,$8$\,\,$ & $\,\,$6 $\,\,$ \\
$\,\,$ 1/2$\,\,$ & $\,\,$ 1 $\,\,$ & $\,\,$6$\,\,$ & $\,\,$2 $\,\,$ \\
$\,\,$ 1/8$\,\,$ & $\,\,$ 1/6$\,\,$ & $\,\,$ 1 $\,\,$ & $\,\,$1 $\,\,$ \\
$\,\,$ 1/6$\,\,$ & $\,\,$ 1/2$\,\,$ & $\,\,$ 1 $\,\,$ & $\,\,$ 1  $\,\,$ \\
\end{pmatrix},
\qquad
\lambda_{\max} =
4.1031,
\qquad
CR = 0.0389
\end{equation*}

\begin{equation*}
\mathbf{w}^{AMAST} =
\begin{pmatrix}
\color{red} 0.550810\color{black} \\
0.281434\\
0.069595\\
0.098160
\end{pmatrix}\end{equation*}
\begin{equation*}
\left[ \frac{{w}^{AMAST}_i}{{w}^{AMAST}_j} \right] =
\begin{pmatrix}
$\,\,$ 1 $\,\,$ & $\,\,$\color{red} 1.9572\color{black} $\,\,$ & $\,\,$\color{red} 7.9145\color{black} $\,\,$ & $\,\,$\color{red} 5.6113\color{black} $\,\,$ \\
$\,\,$\color{red} 0.5109\color{black} $\,\,$ & $\,\,$ 1 $\,\,$ & $\,\,$4.0439$\,\,$ & $\,\,$2.8671  $\,\,$ \\
$\,\,$\color{red} 0.1264\color{black} $\,\,$ & $\,\,$0.2473$\,\,$ & $\,\,$ 1 $\,\,$ & $\,\,$0.7090 $\,\,$ \\
$\,\,$\color{red} 0.1782\color{black} $\,\,$ & $\,\,$0.3488$\,\,$ & $\,\,$1.4104$\,\,$ & $\,\,$ 1  $\,\,$ \\
\end{pmatrix},
\end{equation*}

\begin{equation*}
\mathbf{w}^{\prime} =
\begin{pmatrix}
0.553468\\
0.279769\\
0.069183\\
0.097579
\end{pmatrix} =
0.994083\cdot
\begin{pmatrix}
\color{gr} 0.556762\color{black} \\
0.281434\\
0.069595\\
0.098160
\end{pmatrix},
\end{equation*}
\begin{equation*}
\left[ \frac{{w}^{\prime}_i}{{w}^{\prime}_j} \right] =
\begin{pmatrix}
$\,\,$ 1 $\,\,$ & $\,\,$\color{gr} 1.9783\color{black} $\,\,$ & $\,\,$\color{gr} \color{blue} 8\color{black} $\,\,$ & $\,\,$\color{gr} 5.6720\color{black} $\,\,$ \\
$\,\,$\color{gr} 0.5055\color{black} $\,\,$ & $\,\,$ 1 $\,\,$ & $\,\,$4.0439$\,\,$ & $\,\,$2.8671  $\,\,$ \\
$\,\,$\color{gr} \color{blue}  1/8\color{black} $\,\,$ & $\,\,$0.2473$\,\,$ & $\,\,$ 1 $\,\,$ & $\,\,$0.7090 $\,\,$ \\
$\,\,$\color{gr} 0.1763\color{black} $\,\,$ & $\,\,$0.3488$\,\,$ & $\,\,$1.4104$\,\,$ & $\,\,$ 1  $\,\,$ \\
\end{pmatrix},
\end{equation*}
\end{example}
\newpage
\begin{example}
\begin{equation*}
\mathbf{A} =
\begin{pmatrix}
$\,\,$ 1 $\,\,$ & $\,\,$2$\,\,$ & $\,\,$8$\,\,$ & $\,\,$7 $\,\,$ \\
$\,\,$ 1/2$\,\,$ & $\,\,$ 1 $\,\,$ & $\,\,$2$\,\,$ & $\,\,$5 $\,\,$ \\
$\,\,$ 1/8$\,\,$ & $\,\,$ 1/2$\,\,$ & $\,\,$ 1 $\,\,$ & $\,\,$4 $\,\,$ \\
$\,\,$ 1/7$\,\,$ & $\,\,$ 1/5$\,\,$ & $\,\,$ 1/4$\,\,$ & $\,\,$ 1  $\,\,$ \\
\end{pmatrix},
\qquad
\lambda_{\max} =
4.2035,
\qquad
CR = 0.0767
\end{equation*}

\begin{equation*}
\mathbf{w}^{AMAST} =
\begin{pmatrix}
0.554550\\
\color{red} 0.260918\color{black} \\
0.130981\\
0.053551
\end{pmatrix}\end{equation*}
\begin{equation*}
\left[ \frac{{w}^{AMAST}_i}{{w}^{AMAST}_j} \right] =
\begin{pmatrix}
$\,\,$ 1 $\,\,$ & $\,\,$\color{red} 2.1254\color{black} $\,\,$ & $\,\,$4.2338$\,\,$ & $\,\,$10.3555$\,\,$ \\
$\,\,$\color{red} 0.4705\color{black} $\,\,$ & $\,\,$ 1 $\,\,$ & $\,\,$\color{red} 1.9920\color{black} $\,\,$ & $\,\,$\color{red} 4.8723\color{black}   $\,\,$ \\
$\,\,$0.2362$\,\,$ & $\,\,$\color{red} 0.5020\color{black} $\,\,$ & $\,\,$ 1 $\,\,$ & $\,\,$2.4459 $\,\,$ \\
$\,\,$0.0966$\,\,$ & $\,\,$\color{red} 0.2052\color{black} $\,\,$ & $\,\,$0.4088$\,\,$ & $\,\,$ 1  $\,\,$ \\
\end{pmatrix},
\end{equation*}

\begin{equation*}
\mathbf{w}^{\prime} =
\begin{pmatrix}
0.553971\\
0.261689\\
0.130844\\
0.053495
\end{pmatrix} =
0.998957\cdot
\begin{pmatrix}
0.554550\\
\color{gr} 0.261962\color{black} \\
0.130981\\
0.053551
\end{pmatrix},
\end{equation*}
\begin{equation*}
\left[ \frac{{w}^{\prime}_i}{{w}^{\prime}_j} \right] =
\begin{pmatrix}
$\,\,$ 1 $\,\,$ & $\,\,$\color{gr} 2.1169\color{black} $\,\,$ & $\,\,$4.2338$\,\,$ & $\,\,$10.3555$\,\,$ \\
$\,\,$\color{gr} 0.4724\color{black} $\,\,$ & $\,\,$ 1 $\,\,$ & $\,\,$\color{gr} \color{blue} 2\color{black} $\,\,$ & $\,\,$\color{gr} 4.8918\color{black}   $\,\,$ \\
$\,\,$0.2362$\,\,$ & $\,\,$\color{gr} \color{blue}  1/2\color{black} $\,\,$ & $\,\,$ 1 $\,\,$ & $\,\,$2.4459 $\,\,$ \\
$\,\,$0.0966$\,\,$ & $\,\,$\color{gr} 0.2044\color{black} $\,\,$ & $\,\,$0.4088$\,\,$ & $\,\,$ 1  $\,\,$ \\
\end{pmatrix},
\end{equation*}
\end{example}
\newpage
\begin{example}
\begin{equation*}
\mathbf{A} =
\begin{pmatrix}
$\,\,$ 1 $\,\,$ & $\,\,$2$\,\,$ & $\,\,$8$\,\,$ & $\,\,$8 $\,\,$ \\
$\,\,$ 1/2$\,\,$ & $\,\,$ 1 $\,\,$ & $\,\,$2$\,\,$ & $\,\,$6 $\,\,$ \\
$\,\,$ 1/8$\,\,$ & $\,\,$ 1/2$\,\,$ & $\,\,$ 1 $\,\,$ & $\,\,$5 $\,\,$ \\
$\,\,$ 1/8$\,\,$ & $\,\,$ 1/6$\,\,$ & $\,\,$ 1/5$\,\,$ & $\,\,$ 1  $\,\,$ \\
\end{pmatrix},
\qquad
\lambda_{\max} =
4.2277,
\qquad
CR = 0.0859
\end{equation*}

\begin{equation*}
\mathbf{w}^{AMAST} =
\begin{pmatrix}
0.554718\\
\color{red} 0.264557\color{black} \\
0.134973\\
0.045752
\end{pmatrix}\end{equation*}
\begin{equation*}
\left[ \frac{{w}^{AMAST}_i}{{w}^{AMAST}_j} \right] =
\begin{pmatrix}
$\,\,$ 1 $\,\,$ & $\,\,$\color{red} 2.0968\color{black} $\,\,$ & $\,\,$4.1099$\,\,$ & $\,\,$12.1244$\,\,$ \\
$\,\,$\color{red} 0.4769\color{black} $\,\,$ & $\,\,$ 1 $\,\,$ & $\,\,$\color{red} 1.9601\color{black} $\,\,$ & $\,\,$\color{red} 5.7824\color{black}   $\,\,$ \\
$\,\,$0.2433$\,\,$ & $\,\,$\color{red} 0.5102\color{black} $\,\,$ & $\,\,$ 1 $\,\,$ & $\,\,$2.9501 $\,\,$ \\
$\,\,$0.0825$\,\,$ & $\,\,$\color{red} 0.1729\color{black} $\,\,$ & $\,\,$0.3390$\,\,$ & $\,\,$ 1  $\,\,$ \\
\end{pmatrix},
\end{equation*}

\begin{equation*}
\mathbf{w}^{\prime} =
\begin{pmatrix}
0.551745\\
0.268499\\
0.134249\\
0.045507
\end{pmatrix} =
0.994641\cdot
\begin{pmatrix}
0.554718\\
\color{gr} 0.269945\color{black} \\
0.134973\\
0.045752
\end{pmatrix},
\end{equation*}
\begin{equation*}
\left[ \frac{{w}^{\prime}_i}{{w}^{\prime}_j} \right] =
\begin{pmatrix}
$\,\,$ 1 $\,\,$ & $\,\,$\color{gr} 2.0549\color{black} $\,\,$ & $\,\,$4.1099$\,\,$ & $\,\,$12.1244$\,\,$ \\
$\,\,$\color{gr} 0.4866\color{black} $\,\,$ & $\,\,$ 1 $\,\,$ & $\,\,$\color{gr} \color{blue} 2\color{black} $\,\,$ & $\,\,$\color{gr} 5.9001\color{black}   $\,\,$ \\
$\,\,$0.2433$\,\,$ & $\,\,$\color{gr} \color{blue}  1/2\color{black} $\,\,$ & $\,\,$ 1 $\,\,$ & $\,\,$2.9501 $\,\,$ \\
$\,\,$0.0825$\,\,$ & $\,\,$\color{gr} 0.1695\color{black} $\,\,$ & $\,\,$0.3390$\,\,$ & $\,\,$ 1  $\,\,$ \\
\end{pmatrix},
\end{equation*}
\end{example}
\newpage
\begin{example}
\begin{equation*}
\mathbf{A} =
\begin{pmatrix}
$\,\,$ 1 $\,\,$ & $\,\,$2$\,\,$ & $\,\,$8$\,\,$ & $\,\,$9 $\,\,$ \\
$\,\,$ 1/2$\,\,$ & $\,\,$ 1 $\,\,$ & $\,\,$2$\,\,$ & $\,\,$7 $\,\,$ \\
$\,\,$ 1/8$\,\,$ & $\,\,$ 1/2$\,\,$ & $\,\,$ 1 $\,\,$ & $\,\,$6 $\,\,$ \\
$\,\,$ 1/9$\,\,$ & $\,\,$ 1/7$\,\,$ & $\,\,$ 1/6$\,\,$ & $\,\,$ 1  $\,\,$ \\
\end{pmatrix},
\qquad
\lambda_{\max} =
4.2463,
\qquad
CR = 0.0929
\end{equation*}

\begin{equation*}
\mathbf{w}^{AMAST} =
\begin{pmatrix}
0.554828\\
\color{red} 0.267270\color{black} \\
0.137920\\
0.039982
\end{pmatrix}\end{equation*}
\begin{equation*}
\left[ \frac{{w}^{AMAST}_i}{{w}^{AMAST}_j} \right] =
\begin{pmatrix}
$\,\,$ 1 $\,\,$ & $\,\,$\color{red} 2.0759\color{black} $\,\,$ & $\,\,$4.0228$\,\,$ & $\,\,$13.8769$\,\,$ \\
$\,\,$\color{red} 0.4817\color{black} $\,\,$ & $\,\,$ 1 $\,\,$ & $\,\,$\color{red} 1.9379\color{black} $\,\,$ & $\,\,$\color{red} 6.6847\color{black}   $\,\,$ \\
$\,\,$0.2486$\,\,$ & $\,\,$\color{red} 0.5160\color{black} $\,\,$ & $\,\,$ 1 $\,\,$ & $\,\,$3.4495 $\,\,$ \\
$\,\,$0.0721$\,\,$ & $\,\,$\color{red} 0.1496\color{black} $\,\,$ & $\,\,$0.2899$\,\,$ & $\,\,$ 1  $\,\,$ \\
\end{pmatrix},
\end{equation*}

\begin{equation*}
\mathbf{w}^{\prime} =
\begin{pmatrix}
0.550113\\
0.273496\\
0.136748\\
0.039642
\end{pmatrix} =
0.991502\cdot
\begin{pmatrix}
0.554828\\
\color{gr} 0.275840\color{black} \\
0.137920\\
0.039982
\end{pmatrix},
\end{equation*}
\begin{equation*}
\left[ \frac{{w}^{\prime}_i}{{w}^{\prime}_j} \right] =
\begin{pmatrix}
$\,\,$ 1 $\,\,$ & $\,\,$\color{gr} 2.0114\color{black} $\,\,$ & $\,\,$4.0228$\,\,$ & $\,\,$13.8769$\,\,$ \\
$\,\,$\color{gr} 0.4972\color{black} $\,\,$ & $\,\,$ 1 $\,\,$ & $\,\,$\color{gr} \color{blue} 2\color{black} $\,\,$ & $\,\,$\color{gr} 6.8991\color{black}   $\,\,$ \\
$\,\,$0.2486$\,\,$ & $\,\,$\color{gr} \color{blue}  1/2\color{black} $\,\,$ & $\,\,$ 1 $\,\,$ & $\,\,$3.4495 $\,\,$ \\
$\,\,$0.0721$\,\,$ & $\,\,$\color{gr} 0.1449\color{black} $\,\,$ & $\,\,$0.2899$\,\,$ & $\,\,$ 1  $\,\,$ \\
\end{pmatrix},
\end{equation*}
\end{example}
\newpage
\begin{example}
\begin{equation*}
\mathbf{A} =
\begin{pmatrix}
$\,\,$ 1 $\,\,$ & $\,\,$2$\,\,$ & $\,\,$9$\,\,$ & $\,\,$6 $\,\,$ \\
$\,\,$ 1/2$\,\,$ & $\,\,$ 1 $\,\,$ & $\,\,$3$\,\,$ & $\,\,$4 $\,\,$ \\
$\,\,$ 1/9$\,\,$ & $\,\,$ 1/3$\,\,$ & $\,\,$ 1 $\,\,$ & $\,\,$2 $\,\,$ \\
$\,\,$ 1/6$\,\,$ & $\,\,$ 1/4$\,\,$ & $\,\,$ 1/2$\,\,$ & $\,\,$ 1  $\,\,$ \\
\end{pmatrix},
\qquad
\lambda_{\max} =
4.1031,
\qquad
CR = 0.0389
\end{equation*}

\begin{equation*}
\mathbf{w}^{AMAST} =
\begin{pmatrix}
0.560692\\
\color{red} 0.275350\color{black} \\
0.095033\\
0.068925
\end{pmatrix}\end{equation*}
\begin{equation*}
\left[ \frac{{w}^{AMAST}_i}{{w}^{AMAST}_j} \right] =
\begin{pmatrix}
$\,\,$ 1 $\,\,$ & $\,\,$\color{red} 2.0363\color{black} $\,\,$ & $\,\,$5.8999$\,\,$ & $\,\,$8.1348$\,\,$ \\
$\,\,$\color{red} 0.4911\color{black} $\,\,$ & $\,\,$ 1 $\,\,$ & $\,\,$\color{red} 2.8974\color{black} $\,\,$ & $\,\,$\color{red} 3.9949\color{black}   $\,\,$ \\
$\,\,$0.1695$\,\,$ & $\,\,$\color{red} 0.3451\color{black} $\,\,$ & $\,\,$ 1 $\,\,$ & $\,\,$1.3788 $\,\,$ \\
$\,\,$0.1229$\,\,$ & $\,\,$\color{red} 0.2503\color{black} $\,\,$ & $\,\,$0.7253$\,\,$ & $\,\,$ 1  $\,\,$ \\
\end{pmatrix},
\end{equation*}

\begin{equation*}
\mathbf{w}^{\prime} =
\begin{pmatrix}
0.560496\\
0.275603\\
0.095000\\
0.068901
\end{pmatrix} =
0.999652\cdot
\begin{pmatrix}
0.560692\\
\color{gr} 0.275699\color{black} \\
0.095033\\
0.068925
\end{pmatrix},
\end{equation*}
\begin{equation*}
\left[ \frac{{w}^{\prime}_i}{{w}^{\prime}_j} \right] =
\begin{pmatrix}
$\,\,$ 1 $\,\,$ & $\,\,$\color{gr} 2.0337\color{black} $\,\,$ & $\,\,$5.8999$\,\,$ & $\,\,$8.1348$\,\,$ \\
$\,\,$\color{gr} 0.4917\color{black} $\,\,$ & $\,\,$ 1 $\,\,$ & $\,\,$\color{gr} 2.9011\color{black} $\,\,$ & $\,\,$\color{gr} \color{blue} 4\color{black}   $\,\,$ \\
$\,\,$0.1695$\,\,$ & $\,\,$\color{gr} 0.3447\color{black} $\,\,$ & $\,\,$ 1 $\,\,$ & $\,\,$1.3788 $\,\,$ \\
$\,\,$0.1229$\,\,$ & $\,\,$\color{gr} \color{blue}  1/4\color{black} $\,\,$ & $\,\,$0.7253$\,\,$ & $\,\,$ 1  $\,\,$ \\
\end{pmatrix},
\end{equation*}
\end{example}
\newpage
\begin{example}
\begin{equation*}
\mathbf{A} =
\begin{pmatrix}
$\,\,$ 1 $\,\,$ & $\,\,$2$\,\,$ & $\,\,$9$\,\,$ & $\,\,$6 $\,\,$ \\
$\,\,$ 1/2$\,\,$ & $\,\,$ 1 $\,\,$ & $\,\,$7$\,\,$ & $\,\,$2 $\,\,$ \\
$\,\,$ 1/9$\,\,$ & $\,\,$ 1/7$\,\,$ & $\,\,$ 1 $\,\,$ & $\,\,$1 $\,\,$ \\
$\,\,$ 1/6$\,\,$ & $\,\,$ 1/2$\,\,$ & $\,\,$ 1 $\,\,$ & $\,\,$ 1  $\,\,$ \\
\end{pmatrix},
\qquad
\lambda_{\max} =
4.1342,
\qquad
CR = 0.0506
\end{equation*}

\begin{equation*}
\mathbf{w}^{AMAST} =
\begin{pmatrix}
\color{red} 0.552727\color{black} \\
0.286313\\
0.064258\\
0.096702
\end{pmatrix}\end{equation*}
\begin{equation*}
\left[ \frac{{w}^{AMAST}_i}{{w}^{AMAST}_j} \right] =
\begin{pmatrix}
$\,\,$ 1 $\,\,$ & $\,\,$\color{red} 1.9305\color{black} $\,\,$ & $\,\,$\color{red} 8.6017\color{black} $\,\,$ & $\,\,$\color{red} 5.7158\color{black} $\,\,$ \\
$\,\,$\color{red} 0.5180\color{black} $\,\,$ & $\,\,$ 1 $\,\,$ & $\,\,$4.4557$\,\,$ & $\,\,$2.9608  $\,\,$ \\
$\,\,$\color{red} 0.1163\color{black} $\,\,$ & $\,\,$0.2244$\,\,$ & $\,\,$ 1 $\,\,$ & $\,\,$0.6645 $\,\,$ \\
$\,\,$\color{red} 0.1750\color{black} $\,\,$ & $\,\,$0.3377$\,\,$ & $\,\,$1.5049$\,\,$ & $\,\,$ 1  $\,\,$ \\
\end{pmatrix},
\end{equation*}

\begin{equation*}
\mathbf{w}^{\prime} =
\begin{pmatrix}
0.561454\\
0.280727\\
0.063004\\
0.094815
\end{pmatrix} =
0.980489\cdot
\begin{pmatrix}
\color{gr} 0.572626\color{black} \\
0.286313\\
0.064258\\
0.096702
\end{pmatrix},
\end{equation*}
\begin{equation*}
\left[ \frac{{w}^{\prime}_i}{{w}^{\prime}_j} \right] =
\begin{pmatrix}
$\,\,$ 1 $\,\,$ & $\,\,$\color{gr} \color{blue} 2\color{black} $\,\,$ & $\,\,$\color{gr} 8.9113\color{black} $\,\,$ & $\,\,$\color{gr} 5.9216\color{black} $\,\,$ \\
$\,\,$\color{gr} \color{blue}  1/2\color{black} $\,\,$ & $\,\,$ 1 $\,\,$ & $\,\,$4.4557$\,\,$ & $\,\,$2.9608  $\,\,$ \\
$\,\,$\color{gr} 0.1122\color{black} $\,\,$ & $\,\,$0.2244$\,\,$ & $\,\,$ 1 $\,\,$ & $\,\,$0.6645 $\,\,$ \\
$\,\,$\color{gr} 0.1689\color{black} $\,\,$ & $\,\,$0.3377$\,\,$ & $\,\,$1.5049$\,\,$ & $\,\,$ 1  $\,\,$ \\
\end{pmatrix},
\end{equation*}
\end{example}
\newpage
\begin{example}
\begin{equation*}
\mathbf{A} =
\begin{pmatrix}
$\,\,$ 1 $\,\,$ & $\,\,$2$\,\,$ & $\,\,$9$\,\,$ & $\,\,$6 $\,\,$ \\
$\,\,$ 1/2$\,\,$ & $\,\,$ 1 $\,\,$ & $\,\,$8$\,\,$ & $\,\,$2 $\,\,$ \\
$\,\,$ 1/9$\,\,$ & $\,\,$ 1/8$\,\,$ & $\,\,$ 1 $\,\,$ & $\,\,$1 $\,\,$ \\
$\,\,$ 1/6$\,\,$ & $\,\,$ 1/2$\,\,$ & $\,\,$ 1 $\,\,$ & $\,\,$ 1  $\,\,$ \\
\end{pmatrix},
\qquad
\lambda_{\max} =
4.1664,
\qquad
CR = 0.0627
\end{equation*}

\begin{equation*}
\mathbf{w}^{AMAST} =
\begin{pmatrix}
\color{red} 0.547018\color{black} \\
0.294223\\
0.062281\\
0.096477
\end{pmatrix}\end{equation*}
\begin{equation*}
\left[ \frac{{w}^{AMAST}_i}{{w}^{AMAST}_j} \right] =
\begin{pmatrix}
$\,\,$ 1 $\,\,$ & $\,\,$\color{red} 1.8592\color{black} $\,\,$ & $\,\,$\color{red} 8.7830\color{black} $\,\,$ & $\,\,$\color{red} 5.6699\color{black} $\,\,$ \\
$\,\,$\color{red} 0.5379\color{black} $\,\,$ & $\,\,$ 1 $\,\,$ & $\,\,$4.7241$\,\,$ & $\,\,$3.0497  $\,\,$ \\
$\,\,$\color{red} 0.1139\color{black} $\,\,$ & $\,\,$0.2117$\,\,$ & $\,\,$ 1 $\,\,$ & $\,\,$0.6456 $\,\,$ \\
$\,\,$\color{red} 0.1764\color{black} $\,\,$ & $\,\,$0.3279$\,\,$ & $\,\,$1.5491$\,\,$ & $\,\,$ 1  $\,\,$ \\
\end{pmatrix},
\end{equation*}

\begin{equation*}
\mathbf{w}^{\prime} =
\begin{pmatrix}
0.553058\\
0.290301\\
0.061451\\
0.095191
\end{pmatrix} =
0.986668\cdot
\begin{pmatrix}
\color{gr} 0.560531\color{black} \\
0.294223\\
0.062281\\
0.096477
\end{pmatrix},
\end{equation*}
\begin{equation*}
\left[ \frac{{w}^{\prime}_i}{{w}^{\prime}_j} \right] =
\begin{pmatrix}
$\,\,$ 1 $\,\,$ & $\,\,$\color{gr} 1.9051\color{black} $\,\,$ & $\,\,$\color{gr} \color{blue} 9\color{black} $\,\,$ & $\,\,$\color{gr} 5.8100\color{black} $\,\,$ \\
$\,\,$\color{gr} 0.5249\color{black} $\,\,$ & $\,\,$ 1 $\,\,$ & $\,\,$4.7241$\,\,$ & $\,\,$3.0497  $\,\,$ \\
$\,\,$\color{gr} \color{blue}  1/9\color{black} $\,\,$ & $\,\,$0.2117$\,\,$ & $\,\,$ 1 $\,\,$ & $\,\,$0.6456 $\,\,$ \\
$\,\,$\color{gr} 0.1721\color{black} $\,\,$ & $\,\,$0.3279$\,\,$ & $\,\,$1.5491$\,\,$ & $\,\,$ 1  $\,\,$ \\
\end{pmatrix},
\end{equation*}
\end{example}
\newpage
\begin{example}
\begin{equation*}
\mathbf{A} =
\begin{pmatrix}
$\,\,$ 1 $\,\,$ & $\,\,$2$\,\,$ & $\,\,$9$\,\,$ & $\,\,$6 $\,\,$ \\
$\,\,$ 1/2$\,\,$ & $\,\,$ 1 $\,\,$ & $\,\,$9$\,\,$ & $\,\,$2 $\,\,$ \\
$\,\,$ 1/9$\,\,$ & $\,\,$ 1/9$\,\,$ & $\,\,$ 1 $\,\,$ & $\,\,$1 $\,\,$ \\
$\,\,$ 1/6$\,\,$ & $\,\,$ 1/2$\,\,$ & $\,\,$ 1 $\,\,$ & $\,\,$ 1  $\,\,$ \\
\end{pmatrix},
\qquad
\lambda_{\max} =
4.1990,
\qquad
CR = 0.0750
\end{equation*}

\begin{equation*}
\mathbf{w}^{AMAST} =
\begin{pmatrix}
\color{red} 0.541852\color{black} \\
0.301223\\
0.060642\\
0.096282
\end{pmatrix}\end{equation*}
\begin{equation*}
\left[ \frac{{w}^{AMAST}_i}{{w}^{AMAST}_j} \right] =
\begin{pmatrix}
$\,\,$ 1 $\,\,$ & $\,\,$\color{red} 1.7988\color{black} $\,\,$ & $\,\,$\color{red} 8.9352\color{black} $\,\,$ & $\,\,$\color{red} 5.6278\color{black} $\,\,$ \\
$\,\,$\color{red} 0.5559\color{black} $\,\,$ & $\,\,$ 1 $\,\,$ & $\,\,$4.9672$\,\,$ & $\,\,$3.1286  $\,\,$ \\
$\,\,$\color{red} 0.1119\color{black} $\,\,$ & $\,\,$0.2013$\,\,$ & $\,\,$ 1 $\,\,$ & $\,\,$0.6298 $\,\,$ \\
$\,\,$\color{red} 0.1777\color{black} $\,\,$ & $\,\,$0.3196$\,\,$ & $\,\,$1.5877$\,\,$ & $\,\,$ 1  $\,\,$ \\
\end{pmatrix},
\end{equation*}

\begin{equation*}
\mathbf{w}^{\prime} =
\begin{pmatrix}
0.543646\\
0.300044\\
0.060405\\
0.095905
\end{pmatrix} =
0.996086\cdot
\begin{pmatrix}
\color{gr} 0.545782\color{black} \\
0.301223\\
0.060642\\
0.096282
\end{pmatrix},
\end{equation*}
\begin{equation*}
\left[ \frac{{w}^{\prime}_i}{{w}^{\prime}_j} \right] =
\begin{pmatrix}
$\,\,$ 1 $\,\,$ & $\,\,$\color{gr} 1.8119\color{black} $\,\,$ & $\,\,$\color{gr} \color{blue} 9\color{black} $\,\,$ & $\,\,$\color{gr} 5.6686\color{black} $\,\,$ \\
$\,\,$\color{gr} 0.5519\color{black} $\,\,$ & $\,\,$ 1 $\,\,$ & $\,\,$4.9672$\,\,$ & $\,\,$3.1286  $\,\,$ \\
$\,\,$\color{gr} \color{blue}  1/9\color{black} $\,\,$ & $\,\,$0.2013$\,\,$ & $\,\,$ 1 $\,\,$ & $\,\,$0.6298 $\,\,$ \\
$\,\,$\color{gr} 0.1764\color{black} $\,\,$ & $\,\,$0.3196$\,\,$ & $\,\,$1.5877$\,\,$ & $\,\,$ 1  $\,\,$ \\
\end{pmatrix},
\end{equation*}
\end{example}
\newpage
\begin{example}
\begin{equation*}
\mathbf{A} =
\begin{pmatrix}
$\,\,$ 1 $\,\,$ & $\,\,$2$\,\,$ & $\,\,$9$\,\,$ & $\,\,$7 $\,\,$ \\
$\,\,$ 1/2$\,\,$ & $\,\,$ 1 $\,\,$ & $\,\,$7$\,\,$ & $\,\,$2 $\,\,$ \\
$\,\,$ 1/9$\,\,$ & $\,\,$ 1/7$\,\,$ & $\,\,$ 1 $\,\,$ & $\,\,$1 $\,\,$ \\
$\,\,$ 1/7$\,\,$ & $\,\,$ 1/2$\,\,$ & $\,\,$ 1 $\,\,$ & $\,\,$ 1  $\,\,$ \\
\end{pmatrix},
\qquad
\lambda_{\max} =
4.1372,
\qquad
CR = 0.0517
\end{equation*}

\begin{equation*}
\mathbf{w}^{AMAST} =
\begin{pmatrix}
\color{red} 0.563535\color{black} \\
0.281932\\
0.062889\\
0.091644
\end{pmatrix}\end{equation*}
\begin{equation*}
\left[ \frac{{w}^{AMAST}_i}{{w}^{AMAST}_j} \right] =
\begin{pmatrix}
$\,\,$ 1 $\,\,$ & $\,\,$\color{red} 1.9988\color{black} $\,\,$ & $\,\,$\color{red} 8.9608\color{black} $\,\,$ & $\,\,$\color{red} 6.1492\color{black} $\,\,$ \\
$\,\,$\color{red} 0.5003\color{black} $\,\,$ & $\,\,$ 1 $\,\,$ & $\,\,$4.4830$\,\,$ & $\,\,$3.0764  $\,\,$ \\
$\,\,$\color{red} 0.1116\color{black} $\,\,$ & $\,\,$0.2231$\,\,$ & $\,\,$ 1 $\,\,$ & $\,\,$0.6862 $\,\,$ \\
$\,\,$\color{red} 0.1626\color{black} $\,\,$ & $\,\,$0.3251$\,\,$ & $\,\,$1.4572$\,\,$ & $\,\,$ 1  $\,\,$ \\
\end{pmatrix},
\end{equation*}

\begin{equation*}
\mathbf{w}^{\prime} =
\begin{pmatrix}
0.563679\\
0.281839\\
0.062868\\
0.091613
\end{pmatrix} =
0.999671\cdot
\begin{pmatrix}
\color{gr} 0.563864\color{black} \\
0.281932\\
0.062889\\
0.091644
\end{pmatrix},
\end{equation*}
\begin{equation*}
\left[ \frac{{w}^{\prime}_i}{{w}^{\prime}_j} \right] =
\begin{pmatrix}
$\,\,$ 1 $\,\,$ & $\,\,$\color{gr} \color{blue} 2\color{black} $\,\,$ & $\,\,$\color{gr} 8.9660\color{black} $\,\,$ & $\,\,$\color{gr} 6.1528\color{black} $\,\,$ \\
$\,\,$\color{gr} \color{blue}  1/2\color{black} $\,\,$ & $\,\,$ 1 $\,\,$ & $\,\,$4.4830$\,\,$ & $\,\,$3.0764  $\,\,$ \\
$\,\,$\color{gr} 0.1115\color{black} $\,\,$ & $\,\,$0.2231$\,\,$ & $\,\,$ 1 $\,\,$ & $\,\,$0.6862 $\,\,$ \\
$\,\,$\color{gr} 0.1625\color{black} $\,\,$ & $\,\,$0.3251$\,\,$ & $\,\,$1.4572$\,\,$ & $\,\,$ 1  $\,\,$ \\
\end{pmatrix},
\end{equation*}
\end{example}
\newpage
\begin{example}
\begin{equation*}
\mathbf{A} =
\begin{pmatrix}
$\,\,$ 1 $\,\,$ & $\,\,$2$\,\,$ & $\,\,$9$\,\,$ & $\,\,$9 $\,\,$ \\
$\,\,$ 1/2$\,\,$ & $\,\,$ 1 $\,\,$ & $\,\,$7$\,\,$ & $\,\,$3 $\,\,$ \\
$\,\,$ 1/9$\,\,$ & $\,\,$ 1/7$\,\,$ & $\,\,$ 1 $\,\,$ & $\,\,$2 $\,\,$ \\
$\,\,$ 1/9$\,\,$ & $\,\,$ 1/3$\,\,$ & $\,\,$ 1/2$\,\,$ & $\,\,$ 1  $\,\,$ \\
\end{pmatrix},
\qquad
\lambda_{\max} =
4.2086,
\qquad
CR = 0.0786
\end{equation*}

\begin{equation*}
\mathbf{w}^{AMAST} =
\begin{pmatrix}
\color{red} 0.566846\color{black} \\
0.296398\\
0.073182\\
0.063575
\end{pmatrix}\end{equation*}
\begin{equation*}
\left[ \frac{{w}^{AMAST}_i}{{w}^{AMAST}_j} \right] =
\begin{pmatrix}
$\,\,$ 1 $\,\,$ & $\,\,$\color{red} 1.9125\color{black} $\,\,$ & $\,\,$\color{red} 7.7457\color{black} $\,\,$ & $\,\,$\color{red} 8.9162\color{black} $\,\,$ \\
$\,\,$\color{red} 0.5229\color{black} $\,\,$ & $\,\,$ 1 $\,\,$ & $\,\,$4.0501$\,\,$ & $\,\,$4.6622  $\,\,$ \\
$\,\,$\color{red} 0.1291\color{black} $\,\,$ & $\,\,$0.2469$\,\,$ & $\,\,$ 1 $\,\,$ & $\,\,$1.1511 $\,\,$ \\
$\,\,$\color{red} 0.1122\color{black} $\,\,$ & $\,\,$0.2145$\,\,$ & $\,\,$0.8687$\,\,$ & $\,\,$ 1  $\,\,$ \\
\end{pmatrix},
\end{equation*}

\begin{equation*}
\mathbf{w}^{\prime} =
\begin{pmatrix}
0.569141\\
0.294827\\
0.072794\\
0.063238
\end{pmatrix} =
0.994700\cdot
\begin{pmatrix}
\color{gr} 0.572174\color{black} \\
0.296398\\
0.073182\\
0.063575
\end{pmatrix},
\end{equation*}
\begin{equation*}
\left[ \frac{{w}^{\prime}_i}{{w}^{\prime}_j} \right] =
\begin{pmatrix}
$\,\,$ 1 $\,\,$ & $\,\,$\color{gr} 1.9304\color{black} $\,\,$ & $\,\,$\color{gr} 7.8185\color{black} $\,\,$ & $\,\,$\color{gr} \color{blue} 9\color{black} $\,\,$ \\
$\,\,$\color{gr} 0.5180\color{black} $\,\,$ & $\,\,$ 1 $\,\,$ & $\,\,$4.0501$\,\,$ & $\,\,$4.6622  $\,\,$ \\
$\,\,$\color{gr} 0.1279\color{black} $\,\,$ & $\,\,$0.2469$\,\,$ & $\,\,$ 1 $\,\,$ & $\,\,$1.1511 $\,\,$ \\
$\,\,$\color{gr} \color{blue}  1/9\color{black} $\,\,$ & $\,\,$0.2145$\,\,$ & $\,\,$0.8687$\,\,$ & $\,\,$ 1  $\,\,$ \\
\end{pmatrix},
\end{equation*}
\end{example}
\newpage
\begin{example}
\begin{equation*}
\mathbf{A} =
\begin{pmatrix}
$\,\,$ 1 $\,\,$ & $\,\,$2$\,\,$ & $\,\,$9$\,\,$ & $\,\,$9 $\,\,$ \\
$\,\,$ 1/2$\,\,$ & $\,\,$ 1 $\,\,$ & $\,\,$8$\,\,$ & $\,\,$3 $\,\,$ \\
$\,\,$ 1/9$\,\,$ & $\,\,$ 1/8$\,\,$ & $\,\,$ 1 $\,\,$ & $\,\,$2 $\,\,$ \\
$\,\,$ 1/9$\,\,$ & $\,\,$ 1/3$\,\,$ & $\,\,$ 1/2$\,\,$ & $\,\,$ 1  $\,\,$ \\
\end{pmatrix},
\qquad
\lambda_{\max} =
4.2469,
\qquad
CR = 0.0931
\end{equation*}

\begin{equation*}
\mathbf{w}^{AMAST} =
\begin{pmatrix}
\color{red} 0.560919\color{black} \\
0.304471\\
0.071056\\
0.063553
\end{pmatrix}\end{equation*}
\begin{equation*}
\left[ \frac{{w}^{AMAST}_i}{{w}^{AMAST}_j} \right] =
\begin{pmatrix}
$\,\,$ 1 $\,\,$ & $\,\,$\color{red} 1.8423\color{black} $\,\,$ & $\,\,$\color{red} 7.8940\color{black} $\,\,$ & $\,\,$\color{red} 8.8259\color{black} $\,\,$ \\
$\,\,$\color{red} 0.5428\color{black} $\,\,$ & $\,\,$ 1 $\,\,$ & $\,\,$4.2849$\,\,$ & $\,\,$4.7908  $\,\,$ \\
$\,\,$\color{red} 0.1267\color{black} $\,\,$ & $\,\,$0.2334$\,\,$ & $\,\,$ 1 $\,\,$ & $\,\,$1.1181 $\,\,$ \\
$\,\,$\color{red} 0.1133\color{black} $\,\,$ & $\,\,$0.2087$\,\,$ & $\,\,$0.8944$\,\,$ & $\,\,$ 1  $\,\,$ \\
\end{pmatrix},
\end{equation*}

\begin{equation*}
\mathbf{w}^{\prime} =
\begin{pmatrix}
0.565723\\
0.301140\\
0.070279\\
0.062858
\end{pmatrix} =
0.989059\cdot
\begin{pmatrix}
\color{gr} 0.571981\color{black} \\
0.304471\\
0.071056\\
0.063553
\end{pmatrix},
\end{equation*}
\begin{equation*}
\left[ \frac{{w}^{\prime}_i}{{w}^{\prime}_j} \right] =
\begin{pmatrix}
$\,\,$ 1 $\,\,$ & $\,\,$\color{gr} 1.8786\color{black} $\,\,$ & $\,\,$\color{gr} 8.0497\color{black} $\,\,$ & $\,\,$\color{gr} \color{blue} 9\color{black} $\,\,$ \\
$\,\,$\color{gr} 0.5323\color{black} $\,\,$ & $\,\,$ 1 $\,\,$ & $\,\,$4.2849$\,\,$ & $\,\,$4.7908  $\,\,$ \\
$\,\,$\color{gr} 0.1242\color{black} $\,\,$ & $\,\,$0.2334$\,\,$ & $\,\,$ 1 $\,\,$ & $\,\,$1.1181 $\,\,$ \\
$\,\,$\color{gr} \color{blue}  1/9\color{black} $\,\,$ & $\,\,$0.2087$\,\,$ & $\,\,$0.8944$\,\,$ & $\,\,$ 1  $\,\,$ \\
\end{pmatrix},
\end{equation*}
\end{example}
\newpage
\begin{example}
\begin{equation*}
\mathbf{A} =
\begin{pmatrix}
$\,\,$ 1 $\,\,$ & $\,\,$3$\,\,$ & $\,\,$2$\,\,$ & $\,\,$6 $\,\,$ \\
$\,\,$ 1/3$\,\,$ & $\,\,$ 1 $\,\,$ & $\,\,$1$\,\,$ & $\,\,$6 $\,\,$ \\
$\,\,$ 1/2$\,\,$ & $\,\,$ 1 $\,\,$ & $\,\,$ 1 $\,\,$ & $\,\,$4 $\,\,$ \\
$\,\,$ 1/6$\,\,$ & $\,\,$ 1/6$\,\,$ & $\,\,$ 1/4$\,\,$ & $\,\,$ 1  $\,\,$ \\
\end{pmatrix},
\qquad
\lambda_{\max} =
4.1031,
\qquad
CR = 0.0389
\end{equation*}

\begin{equation*}
\mathbf{w}^{AMAST} =
\begin{pmatrix}
0.474795\\
0.235762\\
\color{red} 0.231458\color{black} \\
0.057985
\end{pmatrix}\end{equation*}
\begin{equation*}
\left[ \frac{{w}^{AMAST}_i}{{w}^{AMAST}_j} \right] =
\begin{pmatrix}
$\,\,$ 1 $\,\,$ & $\,\,$2.0139$\,\,$ & $\,\,$\color{red} 2.0513\color{black} $\,\,$ & $\,\,$8.1882$\,\,$ \\
$\,\,$0.4966$\,\,$ & $\,\,$ 1 $\,\,$ & $\,\,$\color{red} 1.0186\color{black} $\,\,$ & $\,\,$4.0659  $\,\,$ \\
$\,\,$\color{red} 0.4875\color{black} $\,\,$ & $\,\,$\color{red} 0.9817\color{black} $\,\,$ & $\,\,$ 1 $\,\,$ & $\,\,$\color{red} 3.9917\color{black}  $\,\,$ \\
$\,\,$0.1221$\,\,$ & $\,\,$0.2459$\,\,$ & $\,\,$\color{red} 0.2505\color{black} $\,\,$ & $\,\,$ 1  $\,\,$ \\
\end{pmatrix},
\end{equation*}

\begin{equation*}
\mathbf{w}^{\prime} =
\begin{pmatrix}
0.474565\\
0.235648\\
0.231830\\
0.057957
\end{pmatrix} =
0.999517\cdot
\begin{pmatrix}
0.474795\\
0.235762\\
\color{gr} 0.231942\color{black} \\
0.057985
\end{pmatrix},
\end{equation*}
\begin{equation*}
\left[ \frac{{w}^{\prime}_i}{{w}^{\prime}_j} \right] =
\begin{pmatrix}
$\,\,$ 1 $\,\,$ & $\,\,$2.0139$\,\,$ & $\,\,$\color{gr} 2.0470\color{black} $\,\,$ & $\,\,$8.1882$\,\,$ \\
$\,\,$0.4966$\,\,$ & $\,\,$ 1 $\,\,$ & $\,\,$\color{gr} 1.0165\color{black} $\,\,$ & $\,\,$4.0659  $\,\,$ \\
$\,\,$\color{gr} 0.4885\color{black} $\,\,$ & $\,\,$\color{gr} 0.9838\color{black} $\,\,$ & $\,\,$ 1 $\,\,$ & $\,\,$\color{gr} \color{blue} 4\color{black}  $\,\,$ \\
$\,\,$0.1221$\,\,$ & $\,\,$0.2459$\,\,$ & $\,\,$\color{gr} \color{blue}  1/4\color{black} $\,\,$ & $\,\,$ 1  $\,\,$ \\
\end{pmatrix},
\end{equation*}
\end{example}
\newpage
\begin{example}
\begin{equation*}
\mathbf{A} =
\begin{pmatrix}
$\,\,$ 1 $\,\,$ & $\,\,$3$\,\,$ & $\,\,$2$\,\,$ & $\,\,$7 $\,\,$ \\
$\,\,$ 1/3$\,\,$ & $\,\,$ 1 $\,\,$ & $\,\,$1$\,\,$ & $\,\,$7 $\,\,$ \\
$\,\,$ 1/2$\,\,$ & $\,\,$ 1 $\,\,$ & $\,\,$ 1 $\,\,$ & $\,\,$5 $\,\,$ \\
$\,\,$ 1/7$\,\,$ & $\,\,$ 1/7$\,\,$ & $\,\,$ 1/5$\,\,$ & $\,\,$ 1  $\,\,$ \\
\end{pmatrix},
\qquad
\lambda_{\max} =
4.1027,
\qquad
CR = 0.0387
\end{equation*}

\begin{equation*}
\mathbf{w}^{AMAST} =
\begin{pmatrix}
0.477363\\
0.236801\\
\color{red} 0.236682\color{black} \\
0.049154
\end{pmatrix}\end{equation*}
\begin{equation*}
\left[ \frac{{w}^{AMAST}_i}{{w}^{AMAST}_j} \right] =
\begin{pmatrix}
$\,\,$ 1 $\,\,$ & $\,\,$2.0159$\,\,$ & $\,\,$\color{red} 2.0169\color{black} $\,\,$ & $\,\,$9.7117$\,\,$ \\
$\,\,$0.4961$\,\,$ & $\,\,$ 1 $\,\,$ & $\,\,$\color{red} 1.0005\color{black} $\,\,$ & $\,\,$4.8176  $\,\,$ \\
$\,\,$\color{red} 0.4958\color{black} $\,\,$ & $\,\,$\color{red} 0.9995\color{black} $\,\,$ & $\,\,$ 1 $\,\,$ & $\,\,$\color{red} 4.8152\color{black}  $\,\,$ \\
$\,\,$0.1030$\,\,$ & $\,\,$0.2076$\,\,$ & $\,\,$\color{red} 0.2077\color{black} $\,\,$ & $\,\,$ 1  $\,\,$ \\
\end{pmatrix},
\end{equation*}

\begin{equation*}
\mathbf{w}^{\prime} =
\begin{pmatrix}
0.477307\\
0.236773\\
0.236773\\
0.049148
\end{pmatrix} =
0.999881\cdot
\begin{pmatrix}
0.477363\\
0.236801\\
\color{gr} 0.236801\color{black} \\
0.049154
\end{pmatrix},
\end{equation*}
\begin{equation*}
\left[ \frac{{w}^{\prime}_i}{{w}^{\prime}_j} \right] =
\begin{pmatrix}
$\,\,$ 1 $\,\,$ & $\,\,$2.0159$\,\,$ & $\,\,$\color{gr} 2.0159\color{black} $\,\,$ & $\,\,$9.7117$\,\,$ \\
$\,\,$0.4961$\,\,$ & $\,\,$ 1 $\,\,$ & $\,\,$\color{gr} \color{blue} 1\color{black} $\,\,$ & $\,\,$4.8176  $\,\,$ \\
$\,\,$\color{gr} 0.4961\color{black} $\,\,$ & $\,\,$\color{gr} \color{blue} 1\color{black} $\,\,$ & $\,\,$ 1 $\,\,$ & $\,\,$\color{gr} 4.8176\color{black}  $\,\,$ \\
$\,\,$0.1030$\,\,$ & $\,\,$0.2076$\,\,$ & $\,\,$\color{gr} 0.2076\color{black} $\,\,$ & $\,\,$ 1  $\,\,$ \\
\end{pmatrix},
\end{equation*}
\end{example}
\newpage
\begin{example}
\begin{equation*}
\mathbf{A} =
\begin{pmatrix}
$\,\,$ 1 $\,\,$ & $\,\,$3$\,\,$ & $\,\,$2$\,\,$ & $\,\,$7 $\,\,$ \\
$\,\,$ 1/3$\,\,$ & $\,\,$ 1 $\,\,$ & $\,\,$1$\,\,$ & $\,\,$8 $\,\,$ \\
$\,\,$ 1/2$\,\,$ & $\,\,$ 1 $\,\,$ & $\,\,$ 1 $\,\,$ & $\,\,$5 $\,\,$ \\
$\,\,$ 1/7$\,\,$ & $\,\,$ 1/8$\,\,$ & $\,\,$ 1/5$\,\,$ & $\,\,$ 1  $\,\,$ \\
\end{pmatrix},
\qquad
\lambda_{\max} =
4.1301,
\qquad
CR = 0.0490
\end{equation*}

\begin{equation*}
\mathbf{w}^{AMAST} =
\begin{pmatrix}
0.473553\\
0.243920\\
\color{red} 0.234905\color{black} \\
0.047622
\end{pmatrix}\end{equation*}
\begin{equation*}
\left[ \frac{{w}^{AMAST}_i}{{w}^{AMAST}_j} \right] =
\begin{pmatrix}
$\,\,$ 1 $\,\,$ & $\,\,$1.9414$\,\,$ & $\,\,$\color{red} 2.0159\color{black} $\,\,$ & $\,\,$9.9441$\,\,$ \\
$\,\,$0.5151$\,\,$ & $\,\,$ 1 $\,\,$ & $\,\,$\color{red} 1.0384\color{black} $\,\,$ & $\,\,$5.1220  $\,\,$ \\
$\,\,$\color{red} 0.4960\color{black} $\,\,$ & $\,\,$\color{red} 0.9630\color{black} $\,\,$ & $\,\,$ 1 $\,\,$ & $\,\,$\color{red} 4.9327\color{black}  $\,\,$ \\
$\,\,$0.1006$\,\,$ & $\,\,$0.1952$\,\,$ & $\,\,$\color{red} 0.2027\color{black} $\,\,$ & $\,\,$ 1  $\,\,$ \\
\end{pmatrix},
\end{equation*}

\begin{equation*}
\mathbf{w}^{\prime} =
\begin{pmatrix}
0.472669\\
0.243464\\
0.236334\\
0.047533
\end{pmatrix} =
0.998132\cdot
\begin{pmatrix}
0.473553\\
0.243920\\
\color{gr} 0.236777\color{black} \\
0.047622
\end{pmatrix},
\end{equation*}
\begin{equation*}
\left[ \frac{{w}^{\prime}_i}{{w}^{\prime}_j} \right] =
\begin{pmatrix}
$\,\,$ 1 $\,\,$ & $\,\,$1.9414$\,\,$ & $\,\,$\color{gr} \color{blue} 2\color{black} $\,\,$ & $\,\,$9.9441$\,\,$ \\
$\,\,$0.5151$\,\,$ & $\,\,$ 1 $\,\,$ & $\,\,$\color{gr} 1.0302\color{black} $\,\,$ & $\,\,$5.1220  $\,\,$ \\
$\,\,$\color{gr} \color{blue}  1/2\color{black} $\,\,$ & $\,\,$\color{gr} 0.9707\color{black} $\,\,$ & $\,\,$ 1 $\,\,$ & $\,\,$\color{gr} 4.9720\color{black}  $\,\,$ \\
$\,\,$0.1006$\,\,$ & $\,\,$0.1952$\,\,$ & $\,\,$\color{gr} 0.2011\color{black} $\,\,$ & $\,\,$ 1  $\,\,$ \\
\end{pmatrix},
\end{equation*}
\end{example}
\newpage
\begin{example}
\begin{equation*}
\mathbf{A} =
\begin{pmatrix}
$\,\,$ 1 $\,\,$ & $\,\,$3$\,\,$ & $\,\,$2$\,\,$ & $\,\,$9 $\,\,$ \\
$\,\,$ 1/3$\,\,$ & $\,\,$ 1 $\,\,$ & $\,\,$1$\,\,$ & $\,\,$9 $\,\,$ \\
$\,\,$ 1/2$\,\,$ & $\,\,$ 1 $\,\,$ & $\,\,$ 1 $\,\,$ & $\,\,$6 $\,\,$ \\
$\,\,$ 1/9$\,\,$ & $\,\,$ 1/9$\,\,$ & $\,\,$ 1/6$\,\,$ & $\,\,$ 1  $\,\,$ \\
\end{pmatrix},
\qquad
\lambda_{\max} =
4.1031,
\qquad
CR = 0.0389
\end{equation*}

\begin{equation*}
\mathbf{w}^{AMAST} =
\begin{pmatrix}
0.484038\\
0.240382\\
\color{red} 0.236086\color{black} \\
0.039494
\end{pmatrix}\end{equation*}
\begin{equation*}
\left[ \frac{{w}^{AMAST}_i}{{w}^{AMAST}_j} \right] =
\begin{pmatrix}
$\,\,$ 1 $\,\,$ & $\,\,$2.0136$\,\,$ & $\,\,$\color{red} 2.0503\color{black} $\,\,$ & $\,\,$12.2560$\,\,$ \\
$\,\,$0.4966$\,\,$ & $\,\,$ 1 $\,\,$ & $\,\,$\color{red} 1.0182\color{black} $\,\,$ & $\,\,$6.0865  $\,\,$ \\
$\,\,$\color{red} 0.4877\color{black} $\,\,$ & $\,\,$\color{red} 0.9821\color{black} $\,\,$ & $\,\,$ 1 $\,\,$ & $\,\,$\color{red} 5.9778\color{black}  $\,\,$ \\
$\,\,$0.0816$\,\,$ & $\,\,$0.1643$\,\,$ & $\,\,$\color{red} 0.1673\color{black} $\,\,$ & $\,\,$ 1  $\,\,$ \\
\end{pmatrix},
\end{equation*}

\begin{equation*}
\mathbf{w}^{\prime} =
\begin{pmatrix}
0.483614\\
0.240171\\
0.236756\\
0.039459
\end{pmatrix} =
0.999124\cdot
\begin{pmatrix}
0.484038\\
0.240382\\
\color{gr} 0.236963\color{black} \\
0.039494
\end{pmatrix},
\end{equation*}
\begin{equation*}
\left[ \frac{{w}^{\prime}_i}{{w}^{\prime}_j} \right] =
\begin{pmatrix}
$\,\,$ 1 $\,\,$ & $\,\,$2.0136$\,\,$ & $\,\,$\color{gr} 2.0427\color{black} $\,\,$ & $\,\,$12.2560$\,\,$ \\
$\,\,$0.4966$\,\,$ & $\,\,$ 1 $\,\,$ & $\,\,$\color{gr} 1.0144\color{black} $\,\,$ & $\,\,$6.0865  $\,\,$ \\
$\,\,$\color{gr} 0.4896\color{black} $\,\,$ & $\,\,$\color{gr} 0.9858\color{black} $\,\,$ & $\,\,$ 1 $\,\,$ & $\,\,$\color{gr} \color{blue} 6\color{black}  $\,\,$ \\
$\,\,$0.0816$\,\,$ & $\,\,$0.1643$\,\,$ & $\,\,$\color{gr} \color{blue}  1/6\color{black} $\,\,$ & $\,\,$ 1  $\,\,$ \\
\end{pmatrix},
\end{equation*}
\end{example}
\newpage
\begin{example}
\begin{equation*}
\mathbf{A} =
\begin{pmatrix}
$\,\,$ 1 $\,\,$ & $\,\,$3$\,\,$ & $\,\,$3$\,\,$ & $\,\,$9 $\,\,$ \\
$\,\,$ 1/3$\,\,$ & $\,\,$ 1 $\,\,$ & $\,\,$2$\,\,$ & $\,\,$2 $\,\,$ \\
$\,\,$ 1/3$\,\,$ & $\,\,$ 1/2$\,\,$ & $\,\,$ 1 $\,\,$ & $\,\,$5 $\,\,$ \\
$\,\,$ 1/9$\,\,$ & $\,\,$ 1/2$\,\,$ & $\,\,$ 1/5$\,\,$ & $\,\,$ 1  $\,\,$ \\
\end{pmatrix},
\qquad
\lambda_{\max} =
4.2277,
\qquad
CR = 0.0859
\end{equation*}

\begin{equation*}
\mathbf{w}^{AMAST} =
\begin{pmatrix}
\color{red} 0.544146\color{black} \\
0.206957\\
0.184948\\
0.063950
\end{pmatrix}\end{equation*}
\begin{equation*}
\left[ \frac{{w}^{AMAST}_i}{{w}^{AMAST}_j} \right] =
\begin{pmatrix}
$\,\,$ 1 $\,\,$ & $\,\,$\color{red} 2.6293\color{black} $\,\,$ & $\,\,$\color{red} 2.9422\color{black} $\,\,$ & $\,\,$\color{red} 8.5089\color{black} $\,\,$ \\
$\,\,$\color{red} 0.3803\color{black} $\,\,$ & $\,\,$ 1 $\,\,$ & $\,\,$1.1190$\,\,$ & $\,\,$3.2362  $\,\,$ \\
$\,\,$\color{red} 0.3399\color{black} $\,\,$ & $\,\,$0.8937$\,\,$ & $\,\,$ 1 $\,\,$ & $\,\,$2.8921 $\,\,$ \\
$\,\,$\color{red} 0.1175\color{black} $\,\,$ & $\,\,$0.3090$\,\,$ & $\,\,$0.3458$\,\,$ & $\,\,$ 1  $\,\,$ \\
\end{pmatrix},
\end{equation*}

\begin{equation*}
\mathbf{w}^{\prime} =
\begin{pmatrix}
0.548971\\
0.204766\\
0.182990\\
0.063273
\end{pmatrix} =
0.989416\cdot
\begin{pmatrix}
\color{gr} 0.554843\color{black} \\
0.206957\\
0.184948\\
0.063950
\end{pmatrix},
\end{equation*}
\begin{equation*}
\left[ \frac{{w}^{\prime}_i}{{w}^{\prime}_j} \right] =
\begin{pmatrix}
$\,\,$ 1 $\,\,$ & $\,\,$\color{gr} 2.6810\color{black} $\,\,$ & $\,\,$\color{gr} \color{blue} 3\color{black} $\,\,$ & $\,\,$\color{gr} 8.6762\color{black} $\,\,$ \\
$\,\,$\color{gr} 0.3730\color{black} $\,\,$ & $\,\,$ 1 $\,\,$ & $\,\,$1.1190$\,\,$ & $\,\,$3.2362  $\,\,$ \\
$\,\,$\color{gr} \color{blue}  1/3\color{black} $\,\,$ & $\,\,$0.8937$\,\,$ & $\,\,$ 1 $\,\,$ & $\,\,$2.8921 $\,\,$ \\
$\,\,$\color{gr} 0.1153\color{black} $\,\,$ & $\,\,$0.3090$\,\,$ & $\,\,$0.3458$\,\,$ & $\,\,$ 1  $\,\,$ \\
\end{pmatrix},
\end{equation*}
\end{example}
\newpage
\begin{example}
\begin{equation*}
\mathbf{A} =
\begin{pmatrix}
$\,\,$ 1 $\,\,$ & $\,\,$3$\,\,$ & $\,\,$3$\,\,$ & $\,\,$9 $\,\,$ \\
$\,\,$ 1/3$\,\,$ & $\,\,$ 1 $\,\,$ & $\,\,$4$\,\,$ & $\,\,$5 $\,\,$ \\
$\,\,$ 1/3$\,\,$ & $\,\,$ 1/4$\,\,$ & $\,\,$ 1 $\,\,$ & $\,\,$2 $\,\,$ \\
$\,\,$ 1/9$\,\,$ & $\,\,$ 1/5$\,\,$ & $\,\,$ 1/2$\,\,$ & $\,\,$ 1  $\,\,$ \\
\end{pmatrix},
\qquad
\lambda_{\max} =
4.1655,
\qquad
CR = 0.0624
\end{equation*}

\begin{equation*}
\mathbf{w}^{AMAST} =
\begin{pmatrix}
0.529197\\
0.293272\\
0.119073\\
\color{red} 0.058459\color{black}
\end{pmatrix}\end{equation*}
\begin{equation*}
\left[ \frac{{w}^{AMAST}_i}{{w}^{AMAST}_j} \right] =
\begin{pmatrix}
$\,\,$ 1 $\,\,$ & $\,\,$1.8045$\,\,$ & $\,\,$4.4443$\,\,$ & $\,\,$\color{red} 9.0525\color{black} $\,\,$ \\
$\,\,$0.5542$\,\,$ & $\,\,$ 1 $\,\,$ & $\,\,$2.4630$\,\,$ & $\,\,$\color{red} 5.0168\color{black}   $\,\,$ \\
$\,\,$0.2250$\,\,$ & $\,\,$0.4060$\,\,$ & $\,\,$ 1 $\,\,$ & $\,\,$\color{red} 2.0369\color{black}  $\,\,$ \\
$\,\,$\color{red} 0.1105\color{black} $\,\,$ & $\,\,$\color{red} 0.1993\color{black} $\,\,$ & $\,\,$\color{red} 0.4909\color{black} $\,\,$ & $\,\,$ 1  $\,\,$ \\
\end{pmatrix},
\end{equation*}

\begin{equation*}
\mathbf{w}^{\prime} =
\begin{pmatrix}
0.529093\\
0.293215\\
0.119049\\
0.058643
\end{pmatrix} =
0.999804\cdot
\begin{pmatrix}
0.529197\\
0.293272\\
0.119073\\
\color{gr} 0.058654\color{black}
\end{pmatrix},
\end{equation*}
\begin{equation*}
\left[ \frac{{w}^{\prime}_i}{{w}^{\prime}_j} \right] =
\begin{pmatrix}
$\,\,$ 1 $\,\,$ & $\,\,$1.8045$\,\,$ & $\,\,$4.4443$\,\,$ & $\,\,$\color{gr} 9.0223\color{black} $\,\,$ \\
$\,\,$0.5542$\,\,$ & $\,\,$ 1 $\,\,$ & $\,\,$2.4630$\,\,$ & $\,\,$\color{gr} \color{blue} 5\color{black}   $\,\,$ \\
$\,\,$0.2250$\,\,$ & $\,\,$0.4060$\,\,$ & $\,\,$ 1 $\,\,$ & $\,\,$\color{gr} 2.0301\color{black}  $\,\,$ \\
$\,\,$\color{gr} 0.1108\color{black} $\,\,$ & $\,\,$\color{gr} \color{blue}  1/5\color{black} $\,\,$ & $\,\,$\color{gr} 0.4926\color{black} $\,\,$ & $\,\,$ 1  $\,\,$ \\
\end{pmatrix},
\end{equation*}
\end{example}
\newpage
\begin{example}
\begin{equation*}
\mathbf{A} =
\begin{pmatrix}
$\,\,$ 1 $\,\,$ & $\,\,$3$\,\,$ & $\,\,$4$\,\,$ & $\,\,$8 $\,\,$ \\
$\,\,$ 1/3$\,\,$ & $\,\,$ 1 $\,\,$ & $\,\,$2$\,\,$ & $\,\,$2 $\,\,$ \\
$\,\,$ 1/4$\,\,$ & $\,\,$ 1/2$\,\,$ & $\,\,$ 1 $\,\,$ & $\,\,$3 $\,\,$ \\
$\,\,$ 1/8$\,\,$ & $\,\,$ 1/2$\,\,$ & $\,\,$ 1/3$\,\,$ & $\,\,$ 1  $\,\,$ \\
\end{pmatrix},
\qquad
\lambda_{\max} =
4.1031,
\qquad
CR = 0.0389
\end{equation*}

\begin{equation*}
\mathbf{w}^{AMAST} =
\begin{pmatrix}
\color{red} 0.575242\color{black} \\
0.203169\\
0.148934\\
0.072655
\end{pmatrix}\end{equation*}
\begin{equation*}
\left[ \frac{{w}^{AMAST}_i}{{w}^{AMAST}_j} \right] =
\begin{pmatrix}
$\,\,$ 1 $\,\,$ & $\,\,$\color{red} 2.8313\color{black} $\,\,$ & $\,\,$\color{red} 3.8624\color{black} $\,\,$ & $\,\,$\color{red} 7.9175\color{black} $\,\,$ \\
$\,\,$\color{red} 0.3532\color{black} $\,\,$ & $\,\,$ 1 $\,\,$ & $\,\,$1.3642$\,\,$ & $\,\,$2.7964  $\,\,$ \\
$\,\,$\color{red} 0.2589\color{black} $\,\,$ & $\,\,$0.7331$\,\,$ & $\,\,$ 1 $\,\,$ & $\,\,$2.0499 $\,\,$ \\
$\,\,$\color{red} 0.1263\color{black} $\,\,$ & $\,\,$0.3576$\,\,$ & $\,\,$0.4878$\,\,$ & $\,\,$ 1  $\,\,$ \\
\end{pmatrix},
\end{equation*}

\begin{equation*}
\mathbf{w}^{\prime} =
\begin{pmatrix}
0.577774\\
0.201958\\
0.148046\\
0.072222
\end{pmatrix} =
0.994038\cdot
\begin{pmatrix}
\color{gr} 0.581239\color{black} \\
0.203169\\
0.148934\\
0.072655
\end{pmatrix},
\end{equation*}
\begin{equation*}
\left[ \frac{{w}^{\prime}_i}{{w}^{\prime}_j} \right] =
\begin{pmatrix}
$\,\,$ 1 $\,\,$ & $\,\,$\color{gr} 2.8609\color{black} $\,\,$ & $\,\,$\color{gr} 3.9027\color{black} $\,\,$ & $\,\,$\color{gr} \color{blue} 8\color{black} $\,\,$ \\
$\,\,$\color{gr} 0.3495\color{black} $\,\,$ & $\,\,$ 1 $\,\,$ & $\,\,$1.3642$\,\,$ & $\,\,$2.7964  $\,\,$ \\
$\,\,$\color{gr} 0.2562\color{black} $\,\,$ & $\,\,$0.7331$\,\,$ & $\,\,$ 1 $\,\,$ & $\,\,$2.0499 $\,\,$ \\
$\,\,$\color{gr} \color{blue}  1/8\color{black} $\,\,$ & $\,\,$0.3576$\,\,$ & $\,\,$0.4878$\,\,$ & $\,\,$ 1  $\,\,$ \\
\end{pmatrix},
\end{equation*}
\end{example}
\newpage
\begin{example}
\begin{equation*}
\mathbf{A} =
\begin{pmatrix}
$\,\,$ 1 $\,\,$ & $\,\,$3$\,\,$ & $\,\,$4$\,\,$ & $\,\,$9 $\,\,$ \\
$\,\,$ 1/3$\,\,$ & $\,\,$ 1 $\,\,$ & $\,\,$2$\,\,$ & $\,\,$2 $\,\,$ \\
$\,\,$ 1/4$\,\,$ & $\,\,$ 1/2$\,\,$ & $\,\,$ 1 $\,\,$ & $\,\,$3 $\,\,$ \\
$\,\,$ 1/9$\,\,$ & $\,\,$ 1/2$\,\,$ & $\,\,$ 1/3$\,\,$ & $\,\,$ 1  $\,\,$ \\
\end{pmatrix},
\qquad
\lambda_{\max} =
4.1031,
\qquad
CR = 0.0389
\end{equation*}

\begin{equation*}
\mathbf{w}^{AMAST} =
\begin{pmatrix}
\color{red} 0.583415\color{black} \\
0.200523\\
0.146473\\
0.069590
\end{pmatrix}\end{equation*}
\begin{equation*}
\left[ \frac{{w}^{AMAST}_i}{{w}^{AMAST}_j} \right] =
\begin{pmatrix}
$\,\,$ 1 $\,\,$ & $\,\,$\color{red} 2.9095\color{black} $\,\,$ & $\,\,$\color{red} 3.9831\color{black} $\,\,$ & $\,\,$\color{red} 8.3837\color{black} $\,\,$ \\
$\,\,$\color{red} 0.3437\color{black} $\,\,$ & $\,\,$ 1 $\,\,$ & $\,\,$1.3690$\,\,$ & $\,\,$2.8815  $\,\,$ \\
$\,\,$\color{red} 0.2511\color{black} $\,\,$ & $\,\,$0.7305$\,\,$ & $\,\,$ 1 $\,\,$ & $\,\,$2.1048 $\,\,$ \\
$\,\,$\color{red} 0.1193\color{black} $\,\,$ & $\,\,$0.3470$\,\,$ & $\,\,$0.4751$\,\,$ & $\,\,$ 1  $\,\,$ \\
\end{pmatrix},
\end{equation*}

\begin{equation*}
\mathbf{w}^{\prime} =
\begin{pmatrix}
0.584444\\
0.200028\\
0.146111\\
0.069418
\end{pmatrix} =
0.997531\cdot
\begin{pmatrix}
\color{gr} 0.585890\color{black} \\
0.200523\\
0.146473\\
0.069590
\end{pmatrix},
\end{equation*}
\begin{equation*}
\left[ \frac{{w}^{\prime}_i}{{w}^{\prime}_j} \right] =
\begin{pmatrix}
$\,\,$ 1 $\,\,$ & $\,\,$\color{gr} 2.9218\color{black} $\,\,$ & $\,\,$\color{gr} \color{blue} 4\color{black} $\,\,$ & $\,\,$\color{gr} 8.4192\color{black} $\,\,$ \\
$\,\,$\color{gr} 0.3423\color{black} $\,\,$ & $\,\,$ 1 $\,\,$ & $\,\,$1.3690$\,\,$ & $\,\,$2.8815  $\,\,$ \\
$\,\,$\color{gr} \color{blue}  1/4\color{black} $\,\,$ & $\,\,$0.7305$\,\,$ & $\,\,$ 1 $\,\,$ & $\,\,$2.1048 $\,\,$ \\
$\,\,$\color{gr} 0.1188\color{black} $\,\,$ & $\,\,$0.3470$\,\,$ & $\,\,$0.4751$\,\,$ & $\,\,$ 1  $\,\,$ \\
\end{pmatrix},
\end{equation*}
\end{example}
\newpage
\begin{example}
\begin{equation*}
\mathbf{A} =
\begin{pmatrix}
$\,\,$ 1 $\,\,$ & $\,\,$3$\,\,$ & $\,\,$4$\,\,$ & $\,\,$9 $\,\,$ \\
$\,\,$ 1/3$\,\,$ & $\,\,$ 1 $\,\,$ & $\,\,$2$\,\,$ & $\,\,$2 $\,\,$ \\
$\,\,$ 1/4$\,\,$ & $\,\,$ 1/2$\,\,$ & $\,\,$ 1 $\,\,$ & $\,\,$4 $\,\,$ \\
$\,\,$ 1/9$\,\,$ & $\,\,$ 1/2$\,\,$ & $\,\,$ 1/4$\,\,$ & $\,\,$ 1  $\,\,$ \\
\end{pmatrix},
\qquad
\lambda_{\max} =
4.1664,
\qquad
CR = 0.0627
\end{equation*}

\begin{equation*}
\mathbf{w}^{AMAST} =
\begin{pmatrix}
\color{red} 0.575603\color{black} \\
0.200467\\
0.158416\\
0.065515
\end{pmatrix}\end{equation*}
\begin{equation*}
\left[ \frac{{w}^{AMAST}_i}{{w}^{AMAST}_j} \right] =
\begin{pmatrix}
$\,\,$ 1 $\,\,$ & $\,\,$\color{red} 2.8713\color{black} $\,\,$ & $\,\,$\color{red} 3.6335\color{black} $\,\,$ & $\,\,$\color{red} 8.7858\color{black} $\,\,$ \\
$\,\,$\color{red} 0.3483\color{black} $\,\,$ & $\,\,$ 1 $\,\,$ & $\,\,$1.2654$\,\,$ & $\,\,$3.0599  $\,\,$ \\
$\,\,$\color{red} 0.2752\color{black} $\,\,$ & $\,\,$0.7902$\,\,$ & $\,\,$ 1 $\,\,$ & $\,\,$2.4180 $\,\,$ \\
$\,\,$\color{red} 0.1138\color{black} $\,\,$ & $\,\,$0.3268$\,\,$ & $\,\,$0.4136$\,\,$ & $\,\,$ 1  $\,\,$ \\
\end{pmatrix},
\end{equation*}

\begin{equation*}
\mathbf{w}^{\prime} =
\begin{pmatrix}
0.581476\\
0.197692\\
0.156223\\
0.064608
\end{pmatrix} =
0.986161\cdot
\begin{pmatrix}
\color{gr} 0.589636\color{black} \\
0.200467\\
0.158416\\
0.065515
\end{pmatrix},
\end{equation*}
\begin{equation*}
\left[ \frac{{w}^{\prime}_i}{{w}^{\prime}_j} \right] =
\begin{pmatrix}
$\,\,$ 1 $\,\,$ & $\,\,$\color{gr} 2.9413\color{black} $\,\,$ & $\,\,$\color{gr} 3.7221\color{black} $\,\,$ & $\,\,$\color{gr} \color{blue} 9\color{black} $\,\,$ \\
$\,\,$\color{gr} 0.3400\color{black} $\,\,$ & $\,\,$ 1 $\,\,$ & $\,\,$1.2654$\,\,$ & $\,\,$3.0599  $\,\,$ \\
$\,\,$\color{gr} 0.2687\color{black} $\,\,$ & $\,\,$0.7902$\,\,$ & $\,\,$ 1 $\,\,$ & $\,\,$2.4180 $\,\,$ \\
$\,\,$\color{gr} \color{blue}  1/9\color{black} $\,\,$ & $\,\,$0.3268$\,\,$ & $\,\,$0.4136$\,\,$ & $\,\,$ 1  $\,\,$ \\
\end{pmatrix},
\end{equation*}
\end{example}
\newpage
\begin{example}
\begin{equation*}
\mathbf{A} =
\begin{pmatrix}
$\,\,$ 1 $\,\,$ & $\,\,$3$\,\,$ & $\,\,$4$\,\,$ & $\,\,$9 $\,\,$ \\
$\,\,$ 1/3$\,\,$ & $\,\,$ 1 $\,\,$ & $\,\,$2$\,\,$ & $\,\,$9 $\,\,$ \\
$\,\,$ 1/4$\,\,$ & $\,\,$ 1/2$\,\,$ & $\,\,$ 1 $\,\,$ & $\,\,$3 $\,\,$ \\
$\,\,$ 1/9$\,\,$ & $\,\,$ 1/9$\,\,$ & $\,\,$ 1/3$\,\,$ & $\,\,$ 1  $\,\,$ \\
\end{pmatrix},
\qquad
\lambda_{\max} =
4.1031,
\qquad
CR = 0.0389
\end{equation*}

\begin{equation*}
\mathbf{w}^{AMAST} =
\begin{pmatrix}
0.547803\\
0.272668\\
\color{red} 0.134633\color{black} \\
0.044896
\end{pmatrix}\end{equation*}
\begin{equation*}
\left[ \frac{{w}^{AMAST}_i}{{w}^{AMAST}_j} \right] =
\begin{pmatrix}
$\,\,$ 1 $\,\,$ & $\,\,$2.0091$\,\,$ & $\,\,$\color{red} 4.0689\color{black} $\,\,$ & $\,\,$12.2017$\,\,$ \\
$\,\,$0.4977$\,\,$ & $\,\,$ 1 $\,\,$ & $\,\,$\color{red} 2.0253\color{black} $\,\,$ & $\,\,$6.0734  $\,\,$ \\
$\,\,$\color{red} 0.2458\color{black} $\,\,$ & $\,\,$\color{red} 0.4938\color{black} $\,\,$ & $\,\,$ 1 $\,\,$ & $\,\,$\color{red} 2.9988\color{black}  $\,\,$ \\
$\,\,$0.0820$\,\,$ & $\,\,$0.1647$\,\,$ & $\,\,$\color{red} 0.3335\color{black} $\,\,$ & $\,\,$ 1  $\,\,$ \\
\end{pmatrix},
\end{equation*}

\begin{equation*}
\mathbf{w}^{\prime} =
\begin{pmatrix}
0.547774\\
0.272653\\
0.134680\\
0.044893
\end{pmatrix} =
0.999946\cdot
\begin{pmatrix}
0.547803\\
0.272668\\
\color{gr} 0.134687\color{black} \\
0.044896
\end{pmatrix},
\end{equation*}
\begin{equation*}
\left[ \frac{{w}^{\prime}_i}{{w}^{\prime}_j} \right] =
\begin{pmatrix}
$\,\,$ 1 $\,\,$ & $\,\,$2.0091$\,\,$ & $\,\,$\color{gr} 4.0672\color{black} $\,\,$ & $\,\,$12.2017$\,\,$ \\
$\,\,$0.4977$\,\,$ & $\,\,$ 1 $\,\,$ & $\,\,$\color{gr} 2.0245\color{black} $\,\,$ & $\,\,$6.0734  $\,\,$ \\
$\,\,$\color{gr} 0.2459\color{black} $\,\,$ & $\,\,$\color{gr} 0.4940\color{black} $\,\,$ & $\,\,$ 1 $\,\,$ & $\,\,$\color{gr} \color{blue} 3\color{black}  $\,\,$ \\
$\,\,$0.0820$\,\,$ & $\,\,$0.1647$\,\,$ & $\,\,$\color{gr} \color{blue}  1/3\color{black} $\,\,$ & $\,\,$ 1  $\,\,$ \\
\end{pmatrix},
\end{equation*}
\end{example}
\newpage
\begin{example}
\begin{equation*}
\mathbf{A} =
\begin{pmatrix}
$\,\,$ 1 $\,\,$ & $\,\,$3$\,\,$ & $\,\,$5$\,\,$ & $\,\,$8 $\,\,$ \\
$\,\,$ 1/3$\,\,$ & $\,\,$ 1 $\,\,$ & $\,\,$9$\,\,$ & $\,\,$5 $\,\,$ \\
$\,\,$ 1/5$\,\,$ & $\,\,$ 1/9$\,\,$ & $\,\,$ 1 $\,\,$ & $\,\,$1 $\,\,$ \\
$\,\,$ 1/8$\,\,$ & $\,\,$ 1/5$\,\,$ & $\,\,$ 1 $\,\,$ & $\,\,$ 1  $\,\,$ \\
\end{pmatrix},
\qquad
\lambda_{\max} =
4.2489,
\qquad
CR = 0.0939
\end{equation*}

\begin{equation*}
\mathbf{w}^{AMAST} =
\begin{pmatrix}
0.534817\\
0.330817\\
0.068227\\
\color{red} 0.066138\color{black}
\end{pmatrix}\end{equation*}
\begin{equation*}
\left[ \frac{{w}^{AMAST}_i}{{w}^{AMAST}_j} \right] =
\begin{pmatrix}
$\,\,$ 1 $\,\,$ & $\,\,$1.6167$\,\,$ & $\,\,$7.8388$\,\,$ & $\,\,$\color{red} 8.0864\color{black} $\,\,$ \\
$\,\,$0.6186$\,\,$ & $\,\,$ 1 $\,\,$ & $\,\,$4.8488$\,\,$ & $\,\,$\color{red} 5.0019\color{black}   $\,\,$ \\
$\,\,$0.1276$\,\,$ & $\,\,$0.2062$\,\,$ & $\,\,$ 1 $\,\,$ & $\,\,$\color{red} 1.0316\color{black}  $\,\,$ \\
$\,\,$\color{red} 0.1237\color{black} $\,\,$ & $\,\,$\color{red} 0.1999\color{black} $\,\,$ & $\,\,$\color{red} 0.9694\color{black} $\,\,$ & $\,\,$ 1  $\,\,$ \\
\end{pmatrix},
\end{equation*}

\begin{equation*}
\mathbf{w}^{\prime} =
\begin{pmatrix}
0.534804\\
0.330809\\
0.068225\\
0.066162
\end{pmatrix} =
0.999975\cdot
\begin{pmatrix}
0.534817\\
0.330817\\
0.068227\\
\color{gr} 0.066163\color{black}
\end{pmatrix},
\end{equation*}
\begin{equation*}
\left[ \frac{{w}^{\prime}_i}{{w}^{\prime}_j} \right] =
\begin{pmatrix}
$\,\,$ 1 $\,\,$ & $\,\,$1.6167$\,\,$ & $\,\,$7.8388$\,\,$ & $\,\,$\color{gr} 8.0833\color{black} $\,\,$ \\
$\,\,$0.6186$\,\,$ & $\,\,$ 1 $\,\,$ & $\,\,$4.8488$\,\,$ & $\,\,$\color{gr} \color{blue} 5\color{black}   $\,\,$ \\
$\,\,$0.1276$\,\,$ & $\,\,$0.2062$\,\,$ & $\,\,$ 1 $\,\,$ & $\,\,$\color{gr} 1.0312\color{black}  $\,\,$ \\
$\,\,$\color{gr} 0.1237\color{black} $\,\,$ & $\,\,$\color{gr} \color{blue}  1/5\color{black} $\,\,$ & $\,\,$\color{gr} 0.9698\color{black} $\,\,$ & $\,\,$ 1  $\,\,$ \\
\end{pmatrix},
\end{equation*}
\end{example}
\newpage
\begin{example}
\begin{equation*}
\mathbf{A} =
\begin{pmatrix}
$\,\,$ 1 $\,\,$ & $\,\,$3$\,\,$ & $\,\,$5$\,\,$ & $\,\,$9 $\,\,$ \\
$\,\,$ 1/3$\,\,$ & $\,\,$ 1 $\,\,$ & $\,\,$3$\,\,$ & $\,\,$2 $\,\,$ \\
$\,\,$ 1/5$\,\,$ & $\,\,$ 1/3$\,\,$ & $\,\,$ 1 $\,\,$ & $\,\,$3 $\,\,$ \\
$\,\,$ 1/9$\,\,$ & $\,\,$ 1/2$\,\,$ & $\,\,$ 1/3$\,\,$ & $\,\,$ 1  $\,\,$ \\
\end{pmatrix},
\qquad
\lambda_{\max} =
4.1966,
\qquad
CR = 0.0741
\end{equation*}

\begin{equation*}
\mathbf{w}^{AMAST} =
\begin{pmatrix}
\color{red} 0.590227\color{black} \\
0.216815\\
0.124155\\
0.068803
\end{pmatrix}\end{equation*}
\begin{equation*}
\left[ \frac{{w}^{AMAST}_i}{{w}^{AMAST}_j} \right] =
\begin{pmatrix}
$\,\,$ 1 $\,\,$ & $\,\,$\color{red} 2.7223\color{black} $\,\,$ & $\,\,$\color{red} 4.7539\color{black} $\,\,$ & $\,\,$\color{red} 8.5785\color{black} $\,\,$ \\
$\,\,$\color{red} 0.3673\color{black} $\,\,$ & $\,\,$ 1 $\,\,$ & $\,\,$1.7463$\,\,$ & $\,\,$3.1513  $\,\,$ \\
$\,\,$\color{red} 0.2104\color{black} $\,\,$ & $\,\,$0.5726$\,\,$ & $\,\,$ 1 $\,\,$ & $\,\,$1.8045 $\,\,$ \\
$\,\,$\color{red} 0.1166\color{black} $\,\,$ & $\,\,$0.3173$\,\,$ & $\,\,$0.5542$\,\,$ & $\,\,$ 1  $\,\,$ \\
\end{pmatrix},
\end{equation*}

\begin{equation*}
\mathbf{w}^{\prime} =
\begin{pmatrix}
0.601775\\
0.210705\\
0.120656\\
0.066864
\end{pmatrix} =
0.971818\cdot
\begin{pmatrix}
\color{gr} 0.619226\color{black} \\
0.216815\\
0.124155\\
0.068803
\end{pmatrix},
\end{equation*}
\begin{equation*}
\left[ \frac{{w}^{\prime}_i}{{w}^{\prime}_j} \right] =
\begin{pmatrix}
$\,\,$ 1 $\,\,$ & $\,\,$\color{gr} 2.8560\color{black} $\,\,$ & $\,\,$\color{gr} 4.9875\color{black} $\,\,$ & $\,\,$\color{gr} \color{blue} 9\color{black} $\,\,$ \\
$\,\,$\color{gr} 0.3501\color{black} $\,\,$ & $\,\,$ 1 $\,\,$ & $\,\,$1.7463$\,\,$ & $\,\,$3.1513  $\,\,$ \\
$\,\,$\color{gr} 0.2005\color{black} $\,\,$ & $\,\,$0.5726$\,\,$ & $\,\,$ 1 $\,\,$ & $\,\,$1.8045 $\,\,$ \\
$\,\,$\color{gr} \color{blue}  1/9\color{black} $\,\,$ & $\,\,$0.3173$\,\,$ & $\,\,$0.5542$\,\,$ & $\,\,$ 1  $\,\,$ \\
\end{pmatrix},
\end{equation*}
\end{example}
\newpage
\begin{example}
\begin{equation*}
\mathbf{A} =
\begin{pmatrix}
$\,\,$ 1 $\,\,$ & $\,\,$3$\,\,$ & $\,\,$6$\,\,$ & $\,\,$8 $\,\,$ \\
$\,\,$ 1/3$\,\,$ & $\,\,$ 1 $\,\,$ & $\,\,$3$\,\,$ & $\,\,$2 $\,\,$ \\
$\,\,$ 1/6$\,\,$ & $\,\,$ 1/3$\,\,$ & $\,\,$ 1 $\,\,$ & $\,\,$2 $\,\,$ \\
$\,\,$ 1/8$\,\,$ & $\,\,$ 1/2$\,\,$ & $\,\,$ 1/2$\,\,$ & $\,\,$ 1  $\,\,$ \\
\end{pmatrix},
\qquad
\lambda_{\max} =
4.1031,
\qquad
CR = 0.0389
\end{equation*}

\begin{equation*}
\mathbf{w}^{AMAST} =
\begin{pmatrix}
\color{red} 0.604990\color{black} \\
0.213611\\
0.104965\\
0.076435
\end{pmatrix}\end{equation*}
\begin{equation*}
\left[ \frac{{w}^{AMAST}_i}{{w}^{AMAST}_j} \right] =
\begin{pmatrix}
$\,\,$ 1 $\,\,$ & $\,\,$\color{red} 2.8322\color{black} $\,\,$ & $\,\,$\color{red} 5.7638\color{black} $\,\,$ & $\,\,$\color{red} 7.9151\color{black} $\,\,$ \\
$\,\,$\color{red} 0.3531\color{black} $\,\,$ & $\,\,$ 1 $\,\,$ & $\,\,$2.0351$\,\,$ & $\,\,$2.7947  $\,\,$ \\
$\,\,$\color{red} 0.1735\color{black} $\,\,$ & $\,\,$0.4914$\,\,$ & $\,\,$ 1 $\,\,$ & $\,\,$1.3733 $\,\,$ \\
$\,\,$\color{red} 0.1263\color{black} $\,\,$ & $\,\,$0.3578$\,\,$ & $\,\,$0.7282$\,\,$ & $\,\,$ 1  $\,\,$ \\
\end{pmatrix},
\end{equation*}

\begin{equation*}
\mathbf{w}^{\prime} =
\begin{pmatrix}
0.607535\\
0.212235\\
0.104288\\
0.075942
\end{pmatrix} =
0.993555\cdot
\begin{pmatrix}
\color{gr} 0.611476\color{black} \\
0.213611\\
0.104965\\
0.076435
\end{pmatrix},
\end{equation*}
\begin{equation*}
\left[ \frac{{w}^{\prime}_i}{{w}^{\prime}_j} \right] =
\begin{pmatrix}
$\,\,$ 1 $\,\,$ & $\,\,$\color{gr} 2.8626\color{black} $\,\,$ & $\,\,$\color{gr} 5.8256\color{black} $\,\,$ & $\,\,$\color{gr} \color{blue} 8\color{black} $\,\,$ \\
$\,\,$\color{gr} 0.3493\color{black} $\,\,$ & $\,\,$ 1 $\,\,$ & $\,\,$2.0351$\,\,$ & $\,\,$2.7947  $\,\,$ \\
$\,\,$\color{gr} 0.1717\color{black} $\,\,$ & $\,\,$0.4914$\,\,$ & $\,\,$ 1 $\,\,$ & $\,\,$1.3733 $\,\,$ \\
$\,\,$\color{gr} \color{blue}  1/8\color{black} $\,\,$ & $\,\,$0.3578$\,\,$ & $\,\,$0.7282$\,\,$ & $\,\,$ 1  $\,\,$ \\
\end{pmatrix},
\end{equation*}
\end{example}
\newpage
\begin{example}
\begin{equation*}
\mathbf{A} =
\begin{pmatrix}
$\,\,$ 1 $\,\,$ & $\,\,$3$\,\,$ & $\,\,$6$\,\,$ & $\,\,$9 $\,\,$ \\
$\,\,$ 1/3$\,\,$ & $\,\,$ 1 $\,\,$ & $\,\,$3$\,\,$ & $\,\,$2 $\,\,$ \\
$\,\,$ 1/6$\,\,$ & $\,\,$ 1/3$\,\,$ & $\,\,$ 1 $\,\,$ & $\,\,$2 $\,\,$ \\
$\,\,$ 1/9$\,\,$ & $\,\,$ 1/2$\,\,$ & $\,\,$ 1/2$\,\,$ & $\,\,$ 1  $\,\,$ \\
\end{pmatrix},
\qquad
\lambda_{\max} =
4.1031,
\qquad
CR = 0.0389
\end{equation*}

\begin{equation*}
\mathbf{w}^{AMAST} =
\begin{pmatrix}
\color{red} 0.613096\color{black} \\
0.210647\\
0.103102\\
0.073154
\end{pmatrix}\end{equation*}
\begin{equation*}
\left[ \frac{{w}^{AMAST}_i}{{w}^{AMAST}_j} \right] =
\begin{pmatrix}
$\,\,$ 1 $\,\,$ & $\,\,$\color{red} 2.9105\color{black} $\,\,$ & $\,\,$\color{red} 5.9465\color{black} $\,\,$ & $\,\,$\color{red} 8.3809\color{black} $\,\,$ \\
$\,\,$\color{red} 0.3436\color{black} $\,\,$ & $\,\,$ 1 $\,\,$ & $\,\,$2.0431$\,\,$ & $\,\,$2.8795  $\,\,$ \\
$\,\,$\color{red} 0.1682\color{black} $\,\,$ & $\,\,$0.4895$\,\,$ & $\,\,$ 1 $\,\,$ & $\,\,$1.4094 $\,\,$ \\
$\,\,$\color{red} 0.1193\color{black} $\,\,$ & $\,\,$0.3473$\,\,$ & $\,\,$0.7095$\,\,$ & $\,\,$ 1  $\,\,$ \\
\end{pmatrix},
\end{equation*}

\begin{equation*}
\mathbf{w}^{\prime} =
\begin{pmatrix}
0.615219\\
0.209492\\
0.102536\\
0.072753
\end{pmatrix} =
0.994514\cdot
\begin{pmatrix}
\color{gr} 0.618613\color{black} \\
0.210647\\
0.103102\\
0.073154
\end{pmatrix},
\end{equation*}
\begin{equation*}
\left[ \frac{{w}^{\prime}_i}{{w}^{\prime}_j} \right] =
\begin{pmatrix}
$\,\,$ 1 $\,\,$ & $\,\,$\color{gr} 2.9367\color{black} $\,\,$ & $\,\,$\color{gr} \color{blue} 6\color{black} $\,\,$ & $\,\,$\color{gr} 8.4563\color{black} $\,\,$ \\
$\,\,$\color{gr} 0.3405\color{black} $\,\,$ & $\,\,$ 1 $\,\,$ & $\,\,$2.0431$\,\,$ & $\,\,$2.8795  $\,\,$ \\
$\,\,$\color{gr} \color{blue}  1/6\color{black} $\,\,$ & $\,\,$0.4895$\,\,$ & $\,\,$ 1 $\,\,$ & $\,\,$1.4094 $\,\,$ \\
$\,\,$\color{gr} 0.1183\color{black} $\,\,$ & $\,\,$0.3473$\,\,$ & $\,\,$0.7095$\,\,$ & $\,\,$ 1  $\,\,$ \\
\end{pmatrix},
\end{equation*}
\end{example}
\newpage
\begin{example}
\begin{equation*}
\mathbf{A} =
\begin{pmatrix}
$\,\,$ 1 $\,\,$ & $\,\,$3$\,\,$ & $\,\,$6$\,\,$ & $\,\,$9 $\,\,$ \\
$\,\,$ 1/3$\,\,$ & $\,\,$ 1 $\,\,$ & $\,\,$3$\,\,$ & $\,\,$2 $\,\,$ \\
$\,\,$ 1/6$\,\,$ & $\,\,$ 1/3$\,\,$ & $\,\,$ 1 $\,\,$ & $\,\,$3 $\,\,$ \\
$\,\,$ 1/9$\,\,$ & $\,\,$ 1/2$\,\,$ & $\,\,$ 1/3$\,\,$ & $\,\,$ 1  $\,\,$ \\
\end{pmatrix},
\qquad
\lambda_{\max} =
4.1990,
\qquad
CR = 0.0750
\end{equation*}

\begin{equation*}
\mathbf{w}^{AMAST} =
\begin{pmatrix}
\color{red} 0.604110\color{black} \\
0.211611\\
0.116630\\
0.067650
\end{pmatrix}\end{equation*}
\begin{equation*}
\left[ \frac{{w}^{AMAST}_i}{{w}^{AMAST}_j} \right] =
\begin{pmatrix}
$\,\,$ 1 $\,\,$ & $\,\,$\color{red} 2.8548\color{black} $\,\,$ & $\,\,$\color{red} 5.1797\color{black} $\,\,$ & $\,\,$\color{red} 8.9300\color{black} $\,\,$ \\
$\,\,$\color{red} 0.3503\color{black} $\,\,$ & $\,\,$ 1 $\,\,$ & $\,\,$1.8144$\,\,$ & $\,\,$3.1280  $\,\,$ \\
$\,\,$\color{red} 0.1931\color{black} $\,\,$ & $\,\,$0.5512$\,\,$ & $\,\,$ 1 $\,\,$ & $\,\,$1.7240 $\,\,$ \\
$\,\,$\color{red} 0.1120\color{black} $\,\,$ & $\,\,$0.3197$\,\,$ & $\,\,$0.5800$\,\,$ & $\,\,$ 1  $\,\,$ \\
\end{pmatrix},
\end{equation*}

\begin{equation*}
\mathbf{w}^{\prime} =
\begin{pmatrix}
0.605976\\
0.210614\\
0.116080\\
0.067331
\end{pmatrix} =
0.995287\cdot
\begin{pmatrix}
\color{gr} 0.608846\color{black} \\
0.211611\\
0.116630\\
0.067650
\end{pmatrix},
\end{equation*}
\begin{equation*}
\left[ \frac{{w}^{\prime}_i}{{w}^{\prime}_j} \right] =
\begin{pmatrix}
$\,\,$ 1 $\,\,$ & $\,\,$\color{gr} 2.8772\color{black} $\,\,$ & $\,\,$\color{gr} 5.2203\color{black} $\,\,$ & $\,\,$\color{gr} \color{blue} 9\color{black} $\,\,$ \\
$\,\,$\color{gr} 0.3476\color{black} $\,\,$ & $\,\,$ 1 $\,\,$ & $\,\,$1.8144$\,\,$ & $\,\,$3.1280  $\,\,$ \\
$\,\,$\color{gr} 0.1916\color{black} $\,\,$ & $\,\,$0.5512$\,\,$ & $\,\,$ 1 $\,\,$ & $\,\,$1.7240 $\,\,$ \\
$\,\,$\color{gr} \color{blue}  1/9\color{black} $\,\,$ & $\,\,$0.3197$\,\,$ & $\,\,$0.5800$\,\,$ & $\,\,$ 1  $\,\,$ \\
\end{pmatrix},
\end{equation*}
\end{example}
\newpage
\begin{example}
\begin{equation*}
\mathbf{A} =
\begin{pmatrix}
$\,\,$ 1 $\,\,$ & $\,\,$3$\,\,$ & $\,\,$6$\,\,$ & $\,\,$9 $\,\,$ \\
$\,\,$ 1/3$\,\,$ & $\,\,$ 1 $\,\,$ & $\,\,$8$\,\,$ & $\,\,$5 $\,\,$ \\
$\,\,$ 1/6$\,\,$ & $\,\,$ 1/8$\,\,$ & $\,\,$ 1 $\,\,$ & $\,\,$1 $\,\,$ \\
$\,\,$ 1/9$\,\,$ & $\,\,$ 1/5$\,\,$ & $\,\,$ 1 $\,\,$ & $\,\,$ 1  $\,\,$ \\
\end{pmatrix},
\qquad
\lambda_{\max} =
4.1655,
\qquad
CR = 0.0624
\end{equation*}

\begin{equation*}
\mathbf{w}^{AMAST} =
\begin{pmatrix}
0.562168\\
0.311612\\
0.063912\\
\color{red} 0.062308\color{black}
\end{pmatrix}\end{equation*}
\begin{equation*}
\left[ \frac{{w}^{AMAST}_i}{{w}^{AMAST}_j} \right] =
\begin{pmatrix}
$\,\,$ 1 $\,\,$ & $\,\,$1.8041$\,\,$ & $\,\,$8.7959$\,\,$ & $\,\,$\color{red} 9.0225\color{black} $\,\,$ \\
$\,\,$0.5543$\,\,$ & $\,\,$ 1 $\,\,$ & $\,\,$4.8756$\,\,$ & $\,\,$\color{red} 5.0012\color{black}   $\,\,$ \\
$\,\,$0.1137$\,\,$ & $\,\,$0.2051$\,\,$ & $\,\,$ 1 $\,\,$ & $\,\,$\color{red} 1.0258\color{black}  $\,\,$ \\
$\,\,$\color{red} 0.1108\color{black} $\,\,$ & $\,\,$\color{red} 0.2000\color{black} $\,\,$ & $\,\,$\color{red} 0.9749\color{black} $\,\,$ & $\,\,$ 1  $\,\,$ \\
\end{pmatrix},
\end{equation*}

\begin{equation*}
\mathbf{w}^{\prime} =
\begin{pmatrix}
0.562160\\
0.311607\\
0.063912\\
0.062321
\end{pmatrix} =
0.999985\cdot
\begin{pmatrix}
0.562168\\
0.311612\\
0.063912\\
\color{gr} 0.062322\color{black}
\end{pmatrix},
\end{equation*}
\begin{equation*}
\left[ \frac{{w}^{\prime}_i}{{w}^{\prime}_j} \right] =
\begin{pmatrix}
$\,\,$ 1 $\,\,$ & $\,\,$1.8041$\,\,$ & $\,\,$8.7959$\,\,$ & $\,\,$\color{gr} 9.0203\color{black} $\,\,$ \\
$\,\,$0.5543$\,\,$ & $\,\,$ 1 $\,\,$ & $\,\,$4.8756$\,\,$ & $\,\,$\color{gr} \color{blue} 5\color{black}   $\,\,$ \\
$\,\,$0.1137$\,\,$ & $\,\,$0.2051$\,\,$ & $\,\,$ 1 $\,\,$ & $\,\,$\color{gr} 1.0255\color{black}  $\,\,$ \\
$\,\,$\color{gr} 0.1109\color{black} $\,\,$ & $\,\,$\color{gr} \color{blue}  1/5\color{black} $\,\,$ & $\,\,$\color{gr} 0.9751\color{black} $\,\,$ & $\,\,$ 1  $\,\,$ \\
\end{pmatrix},
\end{equation*}
\end{example}
\newpage
\begin{example}
\begin{equation*}
\mathbf{A} =
\begin{pmatrix}
$\,\,$ 1 $\,\,$ & $\,\,$3$\,\,$ & $\,\,$7$\,\,$ & $\,\,$8 $\,\,$ \\
$\,\,$ 1/3$\,\,$ & $\,\,$ 1 $\,\,$ & $\,\,$5$\,\,$ & $\,\,$2 $\,\,$ \\
$\,\,$ 1/7$\,\,$ & $\,\,$ 1/5$\,\,$ & $\,\,$ 1 $\,\,$ & $\,\,$2 $\,\,$ \\
$\,\,$ 1/8$\,\,$ & $\,\,$ 1/2$\,\,$ & $\,\,$ 1/2$\,\,$ & $\,\,$ 1  $\,\,$ \\
\end{pmatrix},
\qquad
\lambda_{\max} =
4.2323,
\qquad
CR = 0.0876
\end{equation*}

\begin{equation*}
\mathbf{w}^{AMAST} =
\begin{pmatrix}
\color{red} 0.598144\color{black} \\
0.237254\\
0.088977\\
0.075625
\end{pmatrix}\end{equation*}
\begin{equation*}
\left[ \frac{{w}^{AMAST}_i}{{w}^{AMAST}_j} \right] =
\begin{pmatrix}
$\,\,$ 1 $\,\,$ & $\,\,$\color{red} 2.5211\color{black} $\,\,$ & $\,\,$\color{red} 6.7224\color{black} $\,\,$ & $\,\,$\color{red} 7.9094\color{black} $\,\,$ \\
$\,\,$\color{red} 0.3966\color{black} $\,\,$ & $\,\,$ 1 $\,\,$ & $\,\,$2.6664$\,\,$ & $\,\,$3.1373  $\,\,$ \\
$\,\,$\color{red} 0.1488\color{black} $\,\,$ & $\,\,$0.3750$\,\,$ & $\,\,$ 1 $\,\,$ & $\,\,$1.1766 $\,\,$ \\
$\,\,$\color{red} 0.1264\color{black} $\,\,$ & $\,\,$0.3187$\,\,$ & $\,\,$0.8499$\,\,$ & $\,\,$ 1  $\,\,$ \\
\end{pmatrix},
\end{equation*}

\begin{equation*}
\mathbf{w}^{\prime} =
\begin{pmatrix}
0.600879\\
0.235639\\
0.088372\\
0.075110
\end{pmatrix} =
0.993195\cdot
\begin{pmatrix}
\color{gr} 0.604996\color{black} \\
0.237254\\
0.088977\\
0.075625
\end{pmatrix},
\end{equation*}
\begin{equation*}
\left[ \frac{{w}^{\prime}_i}{{w}^{\prime}_j} \right] =
\begin{pmatrix}
$\,\,$ 1 $\,\,$ & $\,\,$\color{gr} 2.5500\color{black} $\,\,$ & $\,\,$\color{gr} 6.7994\color{black} $\,\,$ & $\,\,$\color{gr} \color{blue} 8\color{black} $\,\,$ \\
$\,\,$\color{gr} 0.3922\color{black} $\,\,$ & $\,\,$ 1 $\,\,$ & $\,\,$2.6664$\,\,$ & $\,\,$3.1373  $\,\,$ \\
$\,\,$\color{gr} 0.1471\color{black} $\,\,$ & $\,\,$0.3750$\,\,$ & $\,\,$ 1 $\,\,$ & $\,\,$1.1766 $\,\,$ \\
$\,\,$\color{gr} \color{blue}  1/8\color{black} $\,\,$ & $\,\,$0.3187$\,\,$ & $\,\,$0.8499$\,\,$ & $\,\,$ 1  $\,\,$ \\
\end{pmatrix},
\end{equation*}
\end{example}
\newpage
\begin{example}
\begin{equation*}
\mathbf{A} =
\begin{pmatrix}
$\,\,$ 1 $\,\,$ & $\,\,$3$\,\,$ & $\,\,$7$\,\,$ & $\,\,$9 $\,\,$ \\
$\,\,$ 1/3$\,\,$ & $\,\,$ 1 $\,\,$ & $\,\,$4$\,\,$ & $\,\,$2 $\,\,$ \\
$\,\,$ 1/7$\,\,$ & $\,\,$ 1/4$\,\,$ & $\,\,$ 1 $\,\,$ & $\,\,$2 $\,\,$ \\
$\,\,$ 1/9$\,\,$ & $\,\,$ 1/2$\,\,$ & $\,\,$ 1/2$\,\,$ & $\,\,$ 1  $\,\,$ \\
\end{pmatrix},
\qquad
\lambda_{\max} =
4.1658,
\qquad
CR = 0.0625
\end{equation*}

\begin{equation*}
\mathbf{w}^{AMAST} =
\begin{pmatrix}
\color{red} 0.614818\color{black} \\
0.221550\\
0.091457\\
0.072175
\end{pmatrix}\end{equation*}
\begin{equation*}
\left[ \frac{{w}^{AMAST}_i}{{w}^{AMAST}_j} \right] =
\begin{pmatrix}
$\,\,$ 1 $\,\,$ & $\,\,$\color{red} 2.7751\color{black} $\,\,$ & $\,\,$\color{red} 6.7224\color{black} $\,\,$ & $\,\,$\color{red} 8.5185\color{black} $\,\,$ \\
$\,\,$\color{red} 0.3604\color{black} $\,\,$ & $\,\,$ 1 $\,\,$ & $\,\,$2.4224$\,\,$ & $\,\,$3.0696  $\,\,$ \\
$\,\,$\color{red} 0.1488\color{black} $\,\,$ & $\,\,$0.4128$\,\,$ & $\,\,$ 1 $\,\,$ & $\,\,$1.2672 $\,\,$ \\
$\,\,$\color{red} 0.1174\color{black} $\,\,$ & $\,\,$0.3258$\,\,$ & $\,\,$0.7892$\,\,$ & $\,\,$ 1  $\,\,$ \\
\end{pmatrix},
\end{equation*}

\begin{equation*}
\mathbf{w}^{\prime} =
\begin{pmatrix}
0.624354\\
0.216065\\
0.089193\\
0.070388
\end{pmatrix} =
0.975244\cdot
\begin{pmatrix}
\color{gr} 0.640202\color{black} \\
0.221550\\
0.091457\\
0.072175
\end{pmatrix},
\end{equation*}
\begin{equation*}
\left[ \frac{{w}^{\prime}_i}{{w}^{\prime}_j} \right] =
\begin{pmatrix}
$\,\,$ 1 $\,\,$ & $\,\,$\color{gr} 2.8897\color{black} $\,\,$ & $\,\,$\color{gr} \color{blue} 7\color{black} $\,\,$ & $\,\,$\color{gr} 8.8702\color{black} $\,\,$ \\
$\,\,$\color{gr} 0.3461\color{black} $\,\,$ & $\,\,$ 1 $\,\,$ & $\,\,$2.4224$\,\,$ & $\,\,$3.0696  $\,\,$ \\
$\,\,$\color{gr} \color{blue}  1/7\color{black} $\,\,$ & $\,\,$0.4128$\,\,$ & $\,\,$ 1 $\,\,$ & $\,\,$1.2672 $\,\,$ \\
$\,\,$\color{gr} 0.1127\color{black} $\,\,$ & $\,\,$0.3258$\,\,$ & $\,\,$0.7892$\,\,$ & $\,\,$ 1  $\,\,$ \\
\end{pmatrix},
\end{equation*}
\end{example}
\newpage
\begin{example}
\begin{equation*}
\mathbf{A} =
\begin{pmatrix}
$\,\,$ 1 $\,\,$ & $\,\,$3$\,\,$ & $\,\,$7$\,\,$ & $\,\,$9 $\,\,$ \\
$\,\,$ 1/3$\,\,$ & $\,\,$ 1 $\,\,$ & $\,\,$5$\,\,$ & $\,\,$2 $\,\,$ \\
$\,\,$ 1/7$\,\,$ & $\,\,$ 1/5$\,\,$ & $\,\,$ 1 $\,\,$ & $\,\,$2 $\,\,$ \\
$\,\,$ 1/9$\,\,$ & $\,\,$ 1/2$\,\,$ & $\,\,$ 1/2$\,\,$ & $\,\,$ 1  $\,\,$ \\
\end{pmatrix},
\qquad
\lambda_{\max} =
4.2300,
\qquad
CR = 0.0867
\end{equation*}

\begin{equation*}
\mathbf{w}^{AMAST} =
\begin{pmatrix}
\color{red} 0.606141\color{black} \\
0.234148\\
0.087302\\
0.072410
\end{pmatrix}\end{equation*}
\begin{equation*}
\left[ \frac{{w}^{AMAST}_i}{{w}^{AMAST}_j} \right] =
\begin{pmatrix}
$\,\,$ 1 $\,\,$ & $\,\,$\color{red} 2.5887\color{black} $\,\,$ & $\,\,$\color{red} 6.9430\color{black} $\,\,$ & $\,\,$\color{red} 8.3710\color{black} $\,\,$ \\
$\,\,$\color{red} 0.3863\color{black} $\,\,$ & $\,\,$ 1 $\,\,$ & $\,\,$2.6820$\,\,$ & $\,\,$3.2337  $\,\,$ \\
$\,\,$\color{red} 0.1440\color{black} $\,\,$ & $\,\,$0.3728$\,\,$ & $\,\,$ 1 $\,\,$ & $\,\,$1.2057 $\,\,$ \\
$\,\,$\color{red} 0.1195\color{black} $\,\,$ & $\,\,$0.3092$\,\,$ & $\,\,$0.8294$\,\,$ & $\,\,$ 1  $\,\,$ \\
\end{pmatrix},
\end{equation*}

\begin{equation*}
\mathbf{w}^{\prime} =
\begin{pmatrix}
0.608090\\
0.232989\\
0.086870\\
0.072051
\end{pmatrix} =
0.995052\cdot
\begin{pmatrix}
\color{gr} 0.611113\color{black} \\
0.234148\\
0.087302\\
0.072410
\end{pmatrix},
\end{equation*}
\begin{equation*}
\left[ \frac{{w}^{\prime}_i}{{w}^{\prime}_j} \right] =
\begin{pmatrix}
$\,\,$ 1 $\,\,$ & $\,\,$\color{gr} 2.6099\color{black} $\,\,$ & $\,\,$\color{gr} \color{blue} 7\color{black} $\,\,$ & $\,\,$\color{gr} 8.4397\color{black} $\,\,$ \\
$\,\,$\color{gr} 0.3831\color{black} $\,\,$ & $\,\,$ 1 $\,\,$ & $\,\,$2.6820$\,\,$ & $\,\,$3.2337  $\,\,$ \\
$\,\,$\color{gr} \color{blue}  1/7\color{black} $\,\,$ & $\,\,$0.3728$\,\,$ & $\,\,$ 1 $\,\,$ & $\,\,$1.2057 $\,\,$ \\
$\,\,$\color{gr} 0.1185\color{black} $\,\,$ & $\,\,$0.3092$\,\,$ & $\,\,$0.8294$\,\,$ & $\,\,$ 1  $\,\,$ \\
\end{pmatrix},
\end{equation*}
\end{example}
\newpage
\begin{example}
\begin{equation*}
\mathbf{A} =
\begin{pmatrix}
$\,\,$ 1 $\,\,$ & $\,\,$3$\,\,$ & $\,\,$8$\,\,$ & $\,\,$9 $\,\,$ \\
$\,\,$ 1/3$\,\,$ & $\,\,$ 1 $\,\,$ & $\,\,$4$\,\,$ & $\,\,$2 $\,\,$ \\
$\,\,$ 1/8$\,\,$ & $\,\,$ 1/4$\,\,$ & $\,\,$ 1 $\,\,$ & $\,\,$2 $\,\,$ \\
$\,\,$ 1/9$\,\,$ & $\,\,$ 1/2$\,\,$ & $\,\,$ 1/2$\,\,$ & $\,\,$ 1  $\,\,$ \\
\end{pmatrix},
\qquad
\lambda_{\max} =
4.1664,
\qquad
CR = 0.0627
\end{equation*}

\begin{equation*}
\mathbf{w}^{AMAST} =
\begin{pmatrix}
\color{red} 0.624361\color{black} \\
0.217327\\
0.087174\\
0.071138
\end{pmatrix}\end{equation*}
\begin{equation*}
\left[ \frac{{w}^{AMAST}_i}{{w}^{AMAST}_j} \right] =
\begin{pmatrix}
$\,\,$ 1 $\,\,$ & $\,\,$\color{red} 2.8729\color{black} $\,\,$ & $\,\,$\color{red} 7.1623\color{black} $\,\,$ & $\,\,$\color{red} 8.7768\color{black} $\,\,$ \\
$\,\,$\color{red} 0.3481\color{black} $\,\,$ & $\,\,$ 1 $\,\,$ & $\,\,$2.4930$\,\,$ & $\,\,$3.0550  $\,\,$ \\
$\,\,$\color{red} 0.1396\color{black} $\,\,$ & $\,\,$0.4011$\,\,$ & $\,\,$ 1 $\,\,$ & $\,\,$1.2254 $\,\,$ \\
$\,\,$\color{red} 0.1139\color{black} $\,\,$ & $\,\,$0.3273$\,\,$ & $\,\,$0.8160$\,\,$ & $\,\,$ 1  $\,\,$ \\
\end{pmatrix},
\end{equation*}

\begin{equation*}
\mathbf{w}^{\prime} =
\begin{pmatrix}
0.630232\\
0.213930\\
0.085811\\
0.070026
\end{pmatrix} =
0.984371\cdot
\begin{pmatrix}
\color{gr} 0.640239\color{black} \\
0.217327\\
0.087174\\
0.071138
\end{pmatrix},
\end{equation*}
\begin{equation*}
\left[ \frac{{w}^{\prime}_i}{{w}^{\prime}_j} \right] =
\begin{pmatrix}
$\,\,$ 1 $\,\,$ & $\,\,$\color{gr} 2.9460\color{black} $\,\,$ & $\,\,$\color{gr} 7.3444\color{black} $\,\,$ & $\,\,$\color{gr} \color{blue} 9\color{black} $\,\,$ \\
$\,\,$\color{gr} 0.3394\color{black} $\,\,$ & $\,\,$ 1 $\,\,$ & $\,\,$2.4930$\,\,$ & $\,\,$3.0550  $\,\,$ \\
$\,\,$\color{gr} 0.1362\color{black} $\,\,$ & $\,\,$0.4011$\,\,$ & $\,\,$ 1 $\,\,$ & $\,\,$1.2254 $\,\,$ \\
$\,\,$\color{gr} \color{blue}  1/9\color{black} $\,\,$ & $\,\,$0.3273$\,\,$ & $\,\,$0.8160$\,\,$ & $\,\,$ 1  $\,\,$ \\
\end{pmatrix},
\end{equation*}
\end{example}
\newpage
\begin{example}
\begin{equation*}
\mathbf{A} =
\begin{pmatrix}
$\,\,$ 1 $\,\,$ & $\,\,$3$\,\,$ & $\,\,$8$\,\,$ & $\,\,$9 $\,\,$ \\
$\,\,$ 1/3$\,\,$ & $\,\,$ 1 $\,\,$ & $\,\,$5$\,\,$ & $\,\,$2 $\,\,$ \\
$\,\,$ 1/8$\,\,$ & $\,\,$ 1/5$\,\,$ & $\,\,$ 1 $\,\,$ & $\,\,$2 $\,\,$ \\
$\,\,$ 1/9$\,\,$ & $\,\,$ 1/2$\,\,$ & $\,\,$ 1/2$\,\,$ & $\,\,$ 1  $\,\,$ \\
\end{pmatrix},
\qquad
\lambda_{\max} =
4.2267,
\qquad
CR = 0.0855
\end{equation*}

\begin{equation*}
\mathbf{w}^{AMAST} =
\begin{pmatrix}
\color{red} 0.615784\color{black} \\
0.229651\\
0.083175\\
0.071390
\end{pmatrix}\end{equation*}
\begin{equation*}
\left[ \frac{{w}^{AMAST}_i}{{w}^{AMAST}_j} \right] =
\begin{pmatrix}
$\,\,$ 1 $\,\,$ & $\,\,$\color{red} 2.6814\color{black} $\,\,$ & $\,\,$\color{red} 7.4034\color{black} $\,\,$ & $\,\,$\color{red} 8.6256\color{black} $\,\,$ \\
$\,\,$\color{red} 0.3729\color{black} $\,\,$ & $\,\,$ 1 $\,\,$ & $\,\,$2.7610$\,\,$ & $\,\,$3.2169  $\,\,$ \\
$\,\,$\color{red} 0.1351\color{black} $\,\,$ & $\,\,$0.3622$\,\,$ & $\,\,$ 1 $\,\,$ & $\,\,$1.1651 $\,\,$ \\
$\,\,$\color{red} 0.1159\color{black} $\,\,$ & $\,\,$0.3109$\,\,$ & $\,\,$0.8583$\,\,$ & $\,\,$ 1  $\,\,$ \\
\end{pmatrix},
\end{equation*}

\begin{equation*}
\mathbf{w}^{\prime} =
\begin{pmatrix}
0.625785\\
0.223673\\
0.081010\\
0.069532
\end{pmatrix} =
0.973971\cdot
\begin{pmatrix}
\color{gr} 0.642509\color{black} \\
0.229651\\
0.083175\\
0.071390
\end{pmatrix},
\end{equation*}
\begin{equation*}
\left[ \frac{{w}^{\prime}_i}{{w}^{\prime}_j} \right] =
\begin{pmatrix}
$\,\,$ 1 $\,\,$ & $\,\,$\color{gr} 2.7978\color{black} $\,\,$ & $\,\,$\color{gr} 7.7247\color{black} $\,\,$ & $\,\,$\color{gr} \color{blue} 9\color{black} $\,\,$ \\
$\,\,$\color{gr} 0.3574\color{black} $\,\,$ & $\,\,$ 1 $\,\,$ & $\,\,$2.7610$\,\,$ & $\,\,$3.2169  $\,\,$ \\
$\,\,$\color{gr} 0.1295\color{black} $\,\,$ & $\,\,$0.3622$\,\,$ & $\,\,$ 1 $\,\,$ & $\,\,$1.1651 $\,\,$ \\
$\,\,$\color{gr} \color{blue}  1/9\color{black} $\,\,$ & $\,\,$0.3109$\,\,$ & $\,\,$0.8583$\,\,$ & $\,\,$ 1  $\,\,$ \\
\end{pmatrix},
\end{equation*}
\end{example}
\newpage
\begin{example}
\begin{equation*}
\mathbf{A} =
\begin{pmatrix}
$\,\,$ 1 $\,\,$ & $\,\,$3$\,\,$ & $\,\,$9$\,\,$ & $\,\,$9 $\,\,$ \\
$\,\,$ 1/3$\,\,$ & $\,\,$ 1 $\,\,$ & $\,\,$2$\,\,$ & $\,\,$4 $\,\,$ \\
$\,\,$ 1/9$\,\,$ & $\,\,$ 1/2$\,\,$ & $\,\,$ 1 $\,\,$ & $\,\,$3 $\,\,$ \\
$\,\,$ 1/9$\,\,$ & $\,\,$ 1/4$\,\,$ & $\,\,$ 1/3$\,\,$ & $\,\,$ 1  $\,\,$ \\
\end{pmatrix},
\qquad
\lambda_{\max} =
4.1031,
\qquad
CR = 0.0389
\end{equation*}

\begin{equation*}
\mathbf{w}^{AMAST} =
\begin{pmatrix}
0.631818\\
\color{red} 0.208420\color{black} \\
0.107641\\
0.052121
\end{pmatrix}\end{equation*}
\begin{equation*}
\left[ \frac{{w}^{AMAST}_i}{{w}^{AMAST}_j} \right] =
\begin{pmatrix}
$\,\,$ 1 $\,\,$ & $\,\,$\color{red} 3.0315\color{black} $\,\,$ & $\,\,$5.8697$\,\,$ & $\,\,$12.1222$\,\,$ \\
$\,\,$\color{red} 0.3299\color{black} $\,\,$ & $\,\,$ 1 $\,\,$ & $\,\,$\color{red} 1.9362\color{black} $\,\,$ & $\,\,$\color{red} 3.9988\color{black}   $\,\,$ \\
$\,\,$0.1704$\,\,$ & $\,\,$\color{red} 0.5165\color{black} $\,\,$ & $\,\,$ 1 $\,\,$ & $\,\,$2.0652 $\,\,$ \\
$\,\,$0.0825$\,\,$ & $\,\,$\color{red} 0.2501\color{black} $\,\,$ & $\,\,$0.4842$\,\,$ & $\,\,$ 1  $\,\,$ \\
\end{pmatrix},
\end{equation*}

\begin{equation*}
\mathbf{w}^{\prime} =
\begin{pmatrix}
0.631779\\
0.208469\\
0.107635\\
0.052117
\end{pmatrix} =
0.999938\cdot
\begin{pmatrix}
0.631818\\
\color{gr} 0.208482\color{black} \\
0.107641\\
0.052121
\end{pmatrix},
\end{equation*}
\begin{equation*}
\left[ \frac{{w}^{\prime}_i}{{w}^{\prime}_j} \right] =
\begin{pmatrix}
$\,\,$ 1 $\,\,$ & $\,\,$\color{gr} 3.0306\color{black} $\,\,$ & $\,\,$5.8697$\,\,$ & $\,\,$12.1222$\,\,$ \\
$\,\,$\color{gr} 0.3300\color{black} $\,\,$ & $\,\,$ 1 $\,\,$ & $\,\,$\color{gr} 1.9368\color{black} $\,\,$ & $\,\,$\color{gr} \color{blue} 4\color{black}   $\,\,$ \\
$\,\,$0.1704$\,\,$ & $\,\,$\color{gr} 0.5163\color{black} $\,\,$ & $\,\,$ 1 $\,\,$ & $\,\,$2.0652 $\,\,$ \\
$\,\,$0.0825$\,\,$ & $\,\,$\color{gr} \color{blue}  1/4\color{black} $\,\,$ & $\,\,$0.4842$\,\,$ & $\,\,$ 1  $\,\,$ \\
\end{pmatrix},
\end{equation*}
\end{example}
\newpage
\begin{example}
\begin{equation*}
\mathbf{A} =
\begin{pmatrix}
$\,\,$ 1 $\,\,$ & $\,\,$3$\,\,$ & $\,\,$9$\,\,$ & $\,\,$9 $\,\,$ \\
$\,\,$ 1/3$\,\,$ & $\,\,$ 1 $\,\,$ & $\,\,$5$\,\,$ & $\,\,$2 $\,\,$ \\
$\,\,$ 1/9$\,\,$ & $\,\,$ 1/5$\,\,$ & $\,\,$ 1 $\,\,$ & $\,\,$2 $\,\,$ \\
$\,\,$ 1/9$\,\,$ & $\,\,$ 1/2$\,\,$ & $\,\,$ 1/2$\,\,$ & $\,\,$ 1  $\,\,$ \\
\end{pmatrix},
\qquad
\lambda_{\max} =
4.2277,
\qquad
CR = 0.0859
\end{equation*}

\begin{equation*}
\mathbf{w}^{AMAST} =
\begin{pmatrix}
\color{red} 0.623969\color{black} \\
0.225714\\
0.079804\\
0.070513
\end{pmatrix}\end{equation*}
\begin{equation*}
\left[ \frac{{w}^{AMAST}_i}{{w}^{AMAST}_j} \right] =
\begin{pmatrix}
$\,\,$ 1 $\,\,$ & $\,\,$\color{red} 2.7644\color{black} $\,\,$ & $\,\,$\color{red} 7.8188\color{black} $\,\,$ & $\,\,$\color{red} 8.8490\color{black} $\,\,$ \\
$\,\,$\color{red} 0.3617\color{black} $\,\,$ & $\,\,$ 1 $\,\,$ & $\,\,$2.8284$\,\,$ & $\,\,$3.2010  $\,\,$ \\
$\,\,$\color{red} 0.1279\color{black} $\,\,$ & $\,\,$0.3536$\,\,$ & $\,\,$ 1 $\,\,$ & $\,\,$1.1318 $\,\,$ \\
$\,\,$\color{red} 0.1130\color{black} $\,\,$ & $\,\,$0.3124$\,\,$ & $\,\,$0.8836$\,\,$ & $\,\,$ 1  $\,\,$ \\
\end{pmatrix},
\end{equation*}

\begin{equation*}
\mathbf{w}^{\prime} =
\begin{pmatrix}
0.627930\\
0.223337\\
0.078964\\
0.069770
\end{pmatrix} =
0.989467\cdot
\begin{pmatrix}
\color{gr} 0.634614\color{black} \\
0.225714\\
0.079804\\
0.070513
\end{pmatrix},
\end{equation*}
\begin{equation*}
\left[ \frac{{w}^{\prime}_i}{{w}^{\prime}_j} \right] =
\begin{pmatrix}
$\,\,$ 1 $\,\,$ & $\,\,$\color{gr} 2.8116\color{black} $\,\,$ & $\,\,$\color{gr} 7.9522\color{black} $\,\,$ & $\,\,$\color{gr} \color{blue} 9\color{black} $\,\,$ \\
$\,\,$\color{gr} 0.3557\color{black} $\,\,$ & $\,\,$ 1 $\,\,$ & $\,\,$2.8284$\,\,$ & $\,\,$3.2010  $\,\,$ \\
$\,\,$\color{gr} 0.1258\color{black} $\,\,$ & $\,\,$0.3536$\,\,$ & $\,\,$ 1 $\,\,$ & $\,\,$1.1318 $\,\,$ \\
$\,\,$\color{gr} \color{blue}  1/9\color{black} $\,\,$ & $\,\,$0.3124$\,\,$ & $\,\,$0.8836$\,\,$ & $\,\,$ 1  $\,\,$ \\
\end{pmatrix},
\end{equation*}
\end{example}
\newpage
\begin{example}
\begin{equation*}
\mathbf{A} =
\begin{pmatrix}
$\,\,$ 1 $\,\,$ & $\,\,$4$\,\,$ & $\,\,$2$\,\,$ & $\,\,$5 $\,\,$ \\
$\,\,$ 1/4$\,\,$ & $\,\,$ 1 $\,\,$ & $\,\,$1$\,\,$ & $\,\,$7 $\,\,$ \\
$\,\,$ 1/2$\,\,$ & $\,\,$ 1 $\,\,$ & $\,\,$ 1 $\,\,$ & $\,\,$4 $\,\,$ \\
$\,\,$ 1/5$\,\,$ & $\,\,$ 1/7$\,\,$ & $\,\,$ 1/4$\,\,$ & $\,\,$ 1  $\,\,$ \\
\end{pmatrix},
\qquad
\lambda_{\max} =
4.2610,
\qquad
CR = 0.0984
\end{equation*}

\begin{equation*}
\mathbf{w}^{AMAST} =
\begin{pmatrix}
0.478858\\
0.231909\\
\color{red} 0.229045\color{black} \\
0.060188
\end{pmatrix}\end{equation*}
\begin{equation*}
\left[ \frac{{w}^{AMAST}_i}{{w}^{AMAST}_j} \right] =
\begin{pmatrix}
$\,\,$ 1 $\,\,$ & $\,\,$2.0649$\,\,$ & $\,\,$\color{red} 2.0907\color{black} $\,\,$ & $\,\,$7.9561$\,\,$ \\
$\,\,$0.4843$\,\,$ & $\,\,$ 1 $\,\,$ & $\,\,$\color{red} 1.0125\color{black} $\,\,$ & $\,\,$3.8531  $\,\,$ \\
$\,\,$\color{red} 0.4783\color{black} $\,\,$ & $\,\,$\color{red} 0.9877\color{black} $\,\,$ & $\,\,$ 1 $\,\,$ & $\,\,$\color{red} 3.8055\color{black}  $\,\,$ \\
$\,\,$0.1257$\,\,$ & $\,\,$0.2595$\,\,$ & $\,\,$\color{red} 0.2628\color{black} $\,\,$ & $\,\,$ 1  $\,\,$ \\
\end{pmatrix},
\end{equation*}

\begin{equation*}
\mathbf{w}^{\prime} =
\begin{pmatrix}
0.477491\\
0.231246\\
0.231246\\
0.060016
\end{pmatrix} =
0.997145\cdot
\begin{pmatrix}
0.478858\\
0.231909\\
\color{gr} 0.231909\color{black} \\
0.060188
\end{pmatrix},
\end{equation*}
\begin{equation*}
\left[ \frac{{w}^{\prime}_i}{{w}^{\prime}_j} \right] =
\begin{pmatrix}
$\,\,$ 1 $\,\,$ & $\,\,$2.0649$\,\,$ & $\,\,$\color{gr} 2.0649\color{black} $\,\,$ & $\,\,$7.9561$\,\,$ \\
$\,\,$0.4843$\,\,$ & $\,\,$ 1 $\,\,$ & $\,\,$\color{gr} \color{blue} 1\color{black} $\,\,$ & $\,\,$3.8531  $\,\,$ \\
$\,\,$\color{gr} 0.4843\color{black} $\,\,$ & $\,\,$\color{gr} \color{blue} 1\color{black} $\,\,$ & $\,\,$ 1 $\,\,$ & $\,\,$\color{gr} 3.8531\color{black}  $\,\,$ \\
$\,\,$0.1257$\,\,$ & $\,\,$0.2595$\,\,$ & $\,\,$\color{gr} 0.2595\color{black} $\,\,$ & $\,\,$ 1  $\,\,$ \\
\end{pmatrix},
\end{equation*}
\end{example}
\newpage
\begin{example}
\begin{equation*}
\mathbf{A} =
\begin{pmatrix}
$\,\,$ 1 $\,\,$ & $\,\,$4$\,\,$ & $\,\,$2$\,\,$ & $\,\,$7 $\,\,$ \\
$\,\,$ 1/4$\,\,$ & $\,\,$ 1 $\,\,$ & $\,\,$1$\,\,$ & $\,\,$9 $\,\,$ \\
$\,\,$ 1/2$\,\,$ & $\,\,$ 1 $\,\,$ & $\,\,$ 1 $\,\,$ & $\,\,$5 $\,\,$ \\
$\,\,$ 1/7$\,\,$ & $\,\,$ 1/9$\,\,$ & $\,\,$ 1/5$\,\,$ & $\,\,$ 1  $\,\,$ \\
\end{pmatrix},
\qquad
\lambda_{\max} =
4.2371,
\qquad
CR = 0.0894
\end{equation*}

\begin{equation*}
\mathbf{w}^{AMAST} =
\begin{pmatrix}
0.491707\\
0.232923\\
\color{red} 0.229150\color{black} \\
0.046220
\end{pmatrix}\end{equation*}
\begin{equation*}
\left[ \frac{{w}^{AMAST}_i}{{w}^{AMAST}_j} \right] =
\begin{pmatrix}
$\,\,$ 1 $\,\,$ & $\,\,$2.1110$\,\,$ & $\,\,$\color{red} 2.1458\color{black} $\,\,$ & $\,\,$10.6385$\,\,$ \\
$\,\,$0.4737$\,\,$ & $\,\,$ 1 $\,\,$ & $\,\,$\color{red} 1.0165\color{black} $\,\,$ & $\,\,$5.0395  $\,\,$ \\
$\,\,$\color{red} 0.4660\color{black} $\,\,$ & $\,\,$\color{red} 0.9838\color{black} $\,\,$ & $\,\,$ 1 $\,\,$ & $\,\,$\color{red} 4.9578\color{black}  $\,\,$ \\
$\,\,$0.0940$\,\,$ & $\,\,$0.1984$\,\,$ & $\,\,$\color{red} 0.2017\color{black} $\,\,$ & $\,\,$ 1  $\,\,$ \\
\end{pmatrix},
\end{equation*}

\begin{equation*}
\mathbf{w}^{\prime} =
\begin{pmatrix}
0.490751\\
0.232470\\
0.230649\\
0.046130
\end{pmatrix} =
0.998055\cdot
\begin{pmatrix}
0.491707\\
0.232923\\
\color{gr} 0.231099\color{black} \\
0.046220
\end{pmatrix},
\end{equation*}
\begin{equation*}
\left[ \frac{{w}^{\prime}_i}{{w}^{\prime}_j} \right] =
\begin{pmatrix}
$\,\,$ 1 $\,\,$ & $\,\,$2.1110$\,\,$ & $\,\,$\color{gr} 2.1277\color{black} $\,\,$ & $\,\,$10.6385$\,\,$ \\
$\,\,$0.4737$\,\,$ & $\,\,$ 1 $\,\,$ & $\,\,$\color{gr} 1.0079\color{black} $\,\,$ & $\,\,$5.0395  $\,\,$ \\
$\,\,$\color{gr} 0.4700\color{black} $\,\,$ & $\,\,$\color{gr} 0.9922\color{black} $\,\,$ & $\,\,$ 1 $\,\,$ & $\,\,$\color{gr} \color{blue} 5\color{black}  $\,\,$ \\
$\,\,$0.0940$\,\,$ & $\,\,$0.1984$\,\,$ & $\,\,$\color{gr} \color{blue}  1/5\color{black} $\,\,$ & $\,\,$ 1  $\,\,$ \\
\end{pmatrix},
\end{equation*}
\end{example}
\newpage
\begin{example}
\begin{equation*}
\mathbf{A} =
\begin{pmatrix}
$\,\,$ 1 $\,\,$ & $\,\,$4$\,\,$ & $\,\,$3$\,\,$ & $\,\,$7 $\,\,$ \\
$\,\,$ 1/4$\,\,$ & $\,\,$ 1 $\,\,$ & $\,\,$1$\,\,$ & $\,\,$4 $\,\,$ \\
$\,\,$ 1/3$\,\,$ & $\,\,$ 1 $\,\,$ & $\,\,$ 1 $\,\,$ & $\,\,$3 $\,\,$ \\
$\,\,$ 1/7$\,\,$ & $\,\,$ 1/4$\,\,$ & $\,\,$ 1/3$\,\,$ & $\,\,$ 1  $\,\,$ \\
\end{pmatrix},
\qquad
\lambda_{\max} =
4.0576,
\qquad
CR = 0.0217
\end{equation*}

\begin{equation*}
\mathbf{w}^{AMAST} =
\begin{pmatrix}
0.561346\\
0.189110\\
\color{red} 0.186778\color{black} \\
0.062765
\end{pmatrix}\end{equation*}
\begin{equation*}
\left[ \frac{{w}^{AMAST}_i}{{w}^{AMAST}_j} \right] =
\begin{pmatrix}
$\,\,$ 1 $\,\,$ & $\,\,$2.9684$\,\,$ & $\,\,$\color{red} 3.0054\color{black} $\,\,$ & $\,\,$8.9436$\,\,$ \\
$\,\,$0.3369$\,\,$ & $\,\,$ 1 $\,\,$ & $\,\,$\color{red} 1.0125\color{black} $\,\,$ & $\,\,$3.0130  $\,\,$ \\
$\,\,$\color{red} 0.3327\color{black} $\,\,$ & $\,\,$\color{red} 0.9877\color{black} $\,\,$ & $\,\,$ 1 $\,\,$ & $\,\,$\color{red} 2.9758\color{black}  $\,\,$ \\
$\,\,$0.1118$\,\,$ & $\,\,$0.3319$\,\,$ & $\,\,$\color{red} 0.3360\color{black} $\,\,$ & $\,\,$ 1  $\,\,$ \\
\end{pmatrix},
\end{equation*}

\begin{equation*}
\mathbf{w}^{\prime} =
\begin{pmatrix}
0.561157\\
0.189046\\
0.187052\\
0.062744
\end{pmatrix} =
0.999663\cdot
\begin{pmatrix}
0.561346\\
0.189110\\
\color{gr} 0.187115\color{black} \\
0.062765
\end{pmatrix},
\end{equation*}
\begin{equation*}
\left[ \frac{{w}^{\prime}_i}{{w}^{\prime}_j} \right] =
\begin{pmatrix}
$\,\,$ 1 $\,\,$ & $\,\,$2.9684$\,\,$ & $\,\,$\color{gr} \color{blue} 3\color{black} $\,\,$ & $\,\,$8.9436$\,\,$ \\
$\,\,$0.3369$\,\,$ & $\,\,$ 1 $\,\,$ & $\,\,$\color{gr} 1.0107\color{black} $\,\,$ & $\,\,$3.0130  $\,\,$ \\
$\,\,$\color{gr} \color{blue}  1/3\color{black} $\,\,$ & $\,\,$\color{gr} 0.9895\color{black} $\,\,$ & $\,\,$ 1 $\,\,$ & $\,\,$\color{gr} 2.9812\color{black}  $\,\,$ \\
$\,\,$0.1118$\,\,$ & $\,\,$0.3319$\,\,$ & $\,\,$\color{gr} 0.3354\color{black} $\,\,$ & $\,\,$ 1  $\,\,$ \\
\end{pmatrix},
\end{equation*}
\end{example}
\newpage
\begin{example}
\begin{equation*}
\mathbf{A} =
\begin{pmatrix}
$\,\,$ 1 $\,\,$ & $\,\,$4$\,\,$ & $\,\,$4$\,\,$ & $\,\,$5 $\,\,$ \\
$\,\,$ 1/4$\,\,$ & $\,\,$ 1 $\,\,$ & $\,\,$2$\,\,$ & $\,\,$7 $\,\,$ \\
$\,\,$ 1/4$\,\,$ & $\,\,$ 1/2$\,\,$ & $\,\,$ 1 $\,\,$ & $\,\,$2 $\,\,$ \\
$\,\,$ 1/5$\,\,$ & $\,\,$ 1/7$\,\,$ & $\,\,$ 1/2$\,\,$ & $\,\,$ 1  $\,\,$ \\
\end{pmatrix},
\qquad
\lambda_{\max} =
4.2610,
\qquad
CR = 0.0984
\end{equation*}

\begin{equation*}
\mathbf{w}^{AMAST} =
\begin{pmatrix}
0.538445\\
0.262178\\
\color{red} 0.131076\color{black} \\
0.068301
\end{pmatrix}\end{equation*}
\begin{equation*}
\left[ \frac{{w}^{AMAST}_i}{{w}^{AMAST}_j} \right] =
\begin{pmatrix}
$\,\,$ 1 $\,\,$ & $\,\,$2.0537$\,\,$ & $\,\,$\color{red} 4.1079\color{black} $\,\,$ & $\,\,$7.8834$\,\,$ \\
$\,\,$0.4869$\,\,$ & $\,\,$ 1 $\,\,$ & $\,\,$\color{red} 2.0002\color{black} $\,\,$ & $\,\,$3.8386  $\,\,$ \\
$\,\,$\color{red} 0.2434\color{black} $\,\,$ & $\,\,$\color{red} 0.5000\color{black} $\,\,$ & $\,\,$ 1 $\,\,$ & $\,\,$\color{red} 1.9191\color{black}  $\,\,$ \\
$\,\,$0.1268$\,\,$ & $\,\,$0.2605$\,\,$ & $\,\,$\color{red} 0.5211\color{black} $\,\,$ & $\,\,$ 1  $\,\,$ \\
\end{pmatrix},
\end{equation*}

\begin{equation*}
\mathbf{w}^{\prime} =
\begin{pmatrix}
0.538438\\
0.262175\\
0.131087\\
0.068300
\end{pmatrix} =
0.999987\cdot
\begin{pmatrix}
0.538445\\
0.262178\\
\color{gr} 0.131089\color{black} \\
0.068301
\end{pmatrix},
\end{equation*}
\begin{equation*}
\left[ \frac{{w}^{\prime}_i}{{w}^{\prime}_j} \right] =
\begin{pmatrix}
$\,\,$ 1 $\,\,$ & $\,\,$2.0537$\,\,$ & $\,\,$\color{gr} 4.1075\color{black} $\,\,$ & $\,\,$7.8834$\,\,$ \\
$\,\,$0.4869$\,\,$ & $\,\,$ 1 $\,\,$ & $\,\,$\color{gr} \color{blue} 2\color{black} $\,\,$ & $\,\,$3.8386  $\,\,$ \\
$\,\,$\color{gr} 0.2435\color{black} $\,\,$ & $\,\,$\color{gr} \color{blue}  1/2\color{black} $\,\,$ & $\,\,$ 1 $\,\,$ & $\,\,$\color{gr} 1.9193\color{black}  $\,\,$ \\
$\,\,$0.1268$\,\,$ & $\,\,$0.2605$\,\,$ & $\,\,$\color{gr} 0.5210\color{black} $\,\,$ & $\,\,$ 1  $\,\,$ \\
\end{pmatrix},
\end{equation*}
\end{example}
\newpage
\begin{example}
\begin{equation*}
\mathbf{A} =
\begin{pmatrix}
$\,\,$ 1 $\,\,$ & $\,\,$4$\,\,$ & $\,\,$5$\,\,$ & $\,\,$7 $\,\,$ \\
$\,\,$ 1/4$\,\,$ & $\,\,$ 1 $\,\,$ & $\,\,$2$\,\,$ & $\,\,$6 $\,\,$ \\
$\,\,$ 1/5$\,\,$ & $\,\,$ 1/2$\,\,$ & $\,\,$ 1 $\,\,$ & $\,\,$2 $\,\,$ \\
$\,\,$ 1/7$\,\,$ & $\,\,$ 1/6$\,\,$ & $\,\,$ 1/2$\,\,$ & $\,\,$ 1  $\,\,$ \\
\end{pmatrix},
\qquad
\lambda_{\max} =
4.1301,
\qquad
CR = 0.0490
\end{equation*}

\begin{equation*}
\mathbf{w}^{AMAST} =
\begin{pmatrix}
0.588679\\
0.234721\\
\color{red} 0.116871\color{black} \\
0.059729
\end{pmatrix}\end{equation*}
\begin{equation*}
\left[ \frac{{w}^{AMAST}_i}{{w}^{AMAST}_j} \right] =
\begin{pmatrix}
$\,\,$ 1 $\,\,$ & $\,\,$2.5080$\,\,$ & $\,\,$\color{red} 5.0370\color{black} $\,\,$ & $\,\,$9.8559$\,\,$ \\
$\,\,$0.3987$\,\,$ & $\,\,$ 1 $\,\,$ & $\,\,$\color{red} 2.0084\color{black} $\,\,$ & $\,\,$3.9298  $\,\,$ \\
$\,\,$\color{red} 0.1985\color{black} $\,\,$ & $\,\,$\color{red} 0.4979\color{black} $\,\,$ & $\,\,$ 1 $\,\,$ & $\,\,$\color{red} 1.9567\color{black}  $\,\,$ \\
$\,\,$0.1015$\,\,$ & $\,\,$0.2545$\,\,$ & $\,\,$\color{red} 0.5111\color{black} $\,\,$ & $\,\,$ 1  $\,\,$ \\
\end{pmatrix},
\end{equation*}

\begin{equation*}
\mathbf{w}^{\prime} =
\begin{pmatrix}
0.588391\\
0.234606\\
0.117303\\
0.059700
\end{pmatrix} =
0.999511\cdot
\begin{pmatrix}
0.588679\\
0.234721\\
\color{gr} 0.117361\color{black} \\
0.059729
\end{pmatrix},
\end{equation*}
\begin{equation*}
\left[ \frac{{w}^{\prime}_i}{{w}^{\prime}_j} \right] =
\begin{pmatrix}
$\,\,$ 1 $\,\,$ & $\,\,$2.5080$\,\,$ & $\,\,$\color{gr} 5.0160\color{black} $\,\,$ & $\,\,$9.8559$\,\,$ \\
$\,\,$0.3987$\,\,$ & $\,\,$ 1 $\,\,$ & $\,\,$\color{gr} \color{blue} 2\color{black} $\,\,$ & $\,\,$3.9298  $\,\,$ \\
$\,\,$\color{gr} 0.1994\color{black} $\,\,$ & $\,\,$\color{gr} \color{blue}  1/2\color{black} $\,\,$ & $\,\,$ 1 $\,\,$ & $\,\,$\color{gr} 1.9649\color{black}  $\,\,$ \\
$\,\,$0.1015$\,\,$ & $\,\,$0.2545$\,\,$ & $\,\,$\color{gr} 0.5089\color{black} $\,\,$ & $\,\,$ 1  $\,\,$ \\
\end{pmatrix},
\end{equation*}
\end{example}
\newpage
\begin{example}
\begin{equation*}
\mathbf{A} =
\begin{pmatrix}
$\,\,$ 1 $\,\,$ & $\,\,$4$\,\,$ & $\,\,$6$\,\,$ & $\,\,$9 $\,\,$ \\
$\,\,$ 1/4$\,\,$ & $\,\,$ 1 $\,\,$ & $\,\,$7$\,\,$ & $\,\,$4 $\,\,$ \\
$\,\,$ 1/6$\,\,$ & $\,\,$ 1/7$\,\,$ & $\,\,$ 1 $\,\,$ & $\,\,$1 $\,\,$ \\
$\,\,$ 1/9$\,\,$ & $\,\,$ 1/4$\,\,$ & $\,\,$ 1 $\,\,$ & $\,\,$ 1  $\,\,$ \\
\end{pmatrix},
\qquad
\lambda_{\max} =
4.2065,
\qquad
CR = 0.0779
\end{equation*}

\begin{equation*}
\mathbf{w}^{AMAST} =
\begin{pmatrix}
0.598214\\
0.269101\\
0.066683\\
\color{red} 0.066002\color{black}
\end{pmatrix}\end{equation*}
\begin{equation*}
\left[ \frac{{w}^{AMAST}_i}{{w}^{AMAST}_j} \right] =
\begin{pmatrix}
$\,\,$ 1 $\,\,$ & $\,\,$2.2230$\,\,$ & $\,\,$8.9710$\,\,$ & $\,\,$\color{red} 9.0636\color{black} $\,\,$ \\
$\,\,$0.4498$\,\,$ & $\,\,$ 1 $\,\,$ & $\,\,$4.0355$\,\,$ & $\,\,$\color{red} 4.0772\color{black}   $\,\,$ \\
$\,\,$0.1115$\,\,$ & $\,\,$0.2478$\,\,$ & $\,\,$ 1 $\,\,$ & $\,\,$\color{red} 1.0103\color{black}  $\,\,$ \\
$\,\,$\color{red} 0.1103\color{black} $\,\,$ & $\,\,$\color{red} 0.2453\color{black} $\,\,$ & $\,\,$\color{red} 0.9898\color{black} $\,\,$ & $\,\,$ 1  $\,\,$ \\
\end{pmatrix},
\end{equation*}

\begin{equation*}
\mathbf{w}^{\prime} =
\begin{pmatrix}
0.597935\\
0.268976\\
0.066652\\
0.066437
\end{pmatrix} =
0.999534\cdot
\begin{pmatrix}
0.598214\\
0.269101\\
0.066683\\
\color{gr} 0.066468\color{black}
\end{pmatrix},
\end{equation*}
\begin{equation*}
\left[ \frac{{w}^{\prime}_i}{{w}^{\prime}_j} \right] =
\begin{pmatrix}
$\,\,$ 1 $\,\,$ & $\,\,$2.2230$\,\,$ & $\,\,$8.9710$\,\,$ & $\,\,$\color{gr} \color{blue} 9\color{black} $\,\,$ \\
$\,\,$0.4498$\,\,$ & $\,\,$ 1 $\,\,$ & $\,\,$4.0355$\,\,$ & $\,\,$\color{gr} 4.0486\color{black}   $\,\,$ \\
$\,\,$0.1115$\,\,$ & $\,\,$0.2478$\,\,$ & $\,\,$ 1 $\,\,$ & $\,\,$\color{gr} 1.0032\color{black}  $\,\,$ \\
$\,\,$\color{gr} \color{blue}  1/9\color{black} $\,\,$ & $\,\,$\color{gr} 0.2470\color{black} $\,\,$ & $\,\,$\color{gr} 0.9968\color{black} $\,\,$ & $\,\,$ 1  $\,\,$ \\
\end{pmatrix},
\end{equation*}
\end{example}
\newpage
\begin{example}
\begin{equation*}
\mathbf{A} =
\begin{pmatrix}
$\,\,$ 1 $\,\,$ & $\,\,$5$\,\,$ & $\,\,$3$\,\,$ & $\,\,$4 $\,\,$ \\
$\,\,$ 1/5$\,\,$ & $\,\,$ 1 $\,\,$ & $\,\,$1$\,\,$ & $\,\,$3 $\,\,$ \\
$\,\,$ 1/3$\,\,$ & $\,\,$ 1 $\,\,$ & $\,\,$ 1 $\,\,$ & $\,\,$2 $\,\,$ \\
$\,\,$ 1/4$\,\,$ & $\,\,$ 1/3$\,\,$ & $\,\,$ 1/2$\,\,$ & $\,\,$ 1  $\,\,$ \\
\end{pmatrix},
\qquad
\lambda_{\max} =
4.1502,
\qquad
CR = 0.0566
\end{equation*}

\begin{equation*}
\mathbf{w}^{AMAST} =
\begin{pmatrix}
0.545124\\
0.181296\\
\color{red} 0.180099\color{black} \\
0.093481
\end{pmatrix}\end{equation*}
\begin{equation*}
\left[ \frac{{w}^{AMAST}_i}{{w}^{AMAST}_j} \right] =
\begin{pmatrix}
$\,\,$ 1 $\,\,$ & $\,\,$3.0068$\,\,$ & $\,\,$\color{red} 3.0268\color{black} $\,\,$ & $\,\,$5.8314$\,\,$ \\
$\,\,$0.3326$\,\,$ & $\,\,$ 1 $\,\,$ & $\,\,$\color{red} 1.0066\color{black} $\,\,$ & $\,\,$1.9394  $\,\,$ \\
$\,\,$\color{red} 0.3304\color{black} $\,\,$ & $\,\,$\color{red} 0.9934\color{black} $\,\,$ & $\,\,$ 1 $\,\,$ & $\,\,$\color{red} 1.9266\color{black}  $\,\,$ \\
$\,\,$0.1715$\,\,$ & $\,\,$0.5156$\,\,$ & $\,\,$\color{red} 0.5191\color{black} $\,\,$ & $\,\,$ 1  $\,\,$ \\
\end{pmatrix},
\end{equation*}

\begin{equation*}
\mathbf{w}^{\prime} =
\begin{pmatrix}
0.544472\\
0.181079\\
0.181079\\
0.093369
\end{pmatrix} =
0.998805\cdot
\begin{pmatrix}
0.545124\\
0.181296\\
\color{gr} 0.181296\color{black} \\
0.093481
\end{pmatrix},
\end{equation*}
\begin{equation*}
\left[ \frac{{w}^{\prime}_i}{{w}^{\prime}_j} \right] =
\begin{pmatrix}
$\,\,$ 1 $\,\,$ & $\,\,$3.0068$\,\,$ & $\,\,$\color{gr} 3.0068\color{black} $\,\,$ & $\,\,$5.8314$\,\,$ \\
$\,\,$0.3326$\,\,$ & $\,\,$ 1 $\,\,$ & $\,\,$\color{gr} \color{blue} 1\color{black} $\,\,$ & $\,\,$1.9394  $\,\,$ \\
$\,\,$\color{gr} 0.3326\color{black} $\,\,$ & $\,\,$\color{gr} \color{blue} 1\color{black} $\,\,$ & $\,\,$ 1 $\,\,$ & $\,\,$\color{gr} 1.9394\color{black}  $\,\,$ \\
$\,\,$0.1715$\,\,$ & $\,\,$0.5156$\,\,$ & $\,\,$\color{gr} 0.5156\color{black} $\,\,$ & $\,\,$ 1  $\,\,$ \\
\end{pmatrix},
\end{equation*}
\end{example}
\newpage
\begin{example}
\begin{equation*}
\mathbf{A} =
\begin{pmatrix}
$\,\,$ 1 $\,\,$ & $\,\,$5$\,\,$ & $\,\,$3$\,\,$ & $\,\,$6 $\,\,$ \\
$\,\,$ 1/5$\,\,$ & $\,\,$ 1 $\,\,$ & $\,\,$1$\,\,$ & $\,\,$5 $\,\,$ \\
$\,\,$ 1/3$\,\,$ & $\,\,$ 1 $\,\,$ & $\,\,$ 1 $\,\,$ & $\,\,$3 $\,\,$ \\
$\,\,$ 1/6$\,\,$ & $\,\,$ 1/5$\,\,$ & $\,\,$ 1/3$\,\,$ & $\,\,$ 1  $\,\,$ \\
\end{pmatrix},
\qquad
\lambda_{\max} =
4.1758,
\qquad
CR = 0.0663
\end{equation*}

\begin{equation*}
\mathbf{w}^{AMAST} =
\begin{pmatrix}
0.559090\\
0.192392\\
\color{red} 0.185355\color{black} \\
0.063163
\end{pmatrix}\end{equation*}
\begin{equation*}
\left[ \frac{{w}^{AMAST}_i}{{w}^{AMAST}_j} \right] =
\begin{pmatrix}
$\,\,$ 1 $\,\,$ & $\,\,$2.9060$\,\,$ & $\,\,$\color{red} 3.0163\color{black} $\,\,$ & $\,\,$8.8515$\,\,$ \\
$\,\,$0.3441$\,\,$ & $\,\,$ 1 $\,\,$ & $\,\,$\color{red} 1.0380\color{black} $\,\,$ & $\,\,$3.0459  $\,\,$ \\
$\,\,$\color{red} 0.3315\color{black} $\,\,$ & $\,\,$\color{red} 0.9634\color{black} $\,\,$ & $\,\,$ 1 $\,\,$ & $\,\,$\color{red} 2.9345\color{black}  $\,\,$ \\
$\,\,$0.1130$\,\,$ & $\,\,$0.3283$\,\,$ & $\,\,$\color{red} 0.3408\color{black} $\,\,$ & $\,\,$ 1  $\,\,$ \\
\end{pmatrix},
\end{equation*}

\begin{equation*}
\mathbf{w}^{\prime} =
\begin{pmatrix}
0.558527\\
0.192198\\
0.186176\\
0.063100
\end{pmatrix} =
0.998993\cdot
\begin{pmatrix}
0.559090\\
0.192392\\
\color{gr} 0.186363\color{black} \\
0.063163
\end{pmatrix},
\end{equation*}
\begin{equation*}
\left[ \frac{{w}^{\prime}_i}{{w}^{\prime}_j} \right] =
\begin{pmatrix}
$\,\,$ 1 $\,\,$ & $\,\,$2.9060$\,\,$ & $\,\,$\color{gr} \color{blue} 3\color{black} $\,\,$ & $\,\,$8.8515$\,\,$ \\
$\,\,$0.3441$\,\,$ & $\,\,$ 1 $\,\,$ & $\,\,$\color{gr} 1.0323\color{black} $\,\,$ & $\,\,$3.0459  $\,\,$ \\
$\,\,$\color{gr} \color{blue}  1/3\color{black} $\,\,$ & $\,\,$\color{gr} 0.9687\color{black} $\,\,$ & $\,\,$ 1 $\,\,$ & $\,\,$\color{gr} 2.9505\color{black}  $\,\,$ \\
$\,\,$0.1130$\,\,$ & $\,\,$0.3283$\,\,$ & $\,\,$\color{gr} 0.3389\color{black} $\,\,$ & $\,\,$ 1  $\,\,$ \\
\end{pmatrix},
\end{equation*}
\end{example}
\newpage
\begin{example}
\begin{equation*}
\mathbf{A} =
\begin{pmatrix}
$\,\,$ 1 $\,\,$ & $\,\,$5$\,\,$ & $\,\,$3$\,\,$ & $\,\,$8 $\,\,$ \\
$\,\,$ 1/5$\,\,$ & $\,\,$ 1 $\,\,$ & $\,\,$1$\,\,$ & $\,\,$6 $\,\,$ \\
$\,\,$ 1/3$\,\,$ & $\,\,$ 1 $\,\,$ & $\,\,$ 1 $\,\,$ & $\,\,$4 $\,\,$ \\
$\,\,$ 1/8$\,\,$ & $\,\,$ 1/6$\,\,$ & $\,\,$ 1/4$\,\,$ & $\,\,$ 1  $\,\,$ \\
\end{pmatrix},
\qquad
\lambda_{\max} =
4.1502,
\qquad
CR = 0.0566
\end{equation*}

\begin{equation*}
\mathbf{w}^{AMAST} =
\begin{pmatrix}
0.571248\\
0.190242\\
\color{red} 0.189160\color{black} \\
0.049349
\end{pmatrix}\end{equation*}
\begin{equation*}
\left[ \frac{{w}^{AMAST}_i}{{w}^{AMAST}_j} \right] =
\begin{pmatrix}
$\,\,$ 1 $\,\,$ & $\,\,$3.0027$\,\,$ & $\,\,$\color{red} 3.0199\color{black} $\,\,$ & $\,\,$11.5756$\,\,$ \\
$\,\,$0.3330$\,\,$ & $\,\,$ 1 $\,\,$ & $\,\,$\color{red} 1.0057\color{black} $\,\,$ & $\,\,$3.8550  $\,\,$ \\
$\,\,$\color{red} 0.3311\color{black} $\,\,$ & $\,\,$\color{red} 0.9943\color{black} $\,\,$ & $\,\,$ 1 $\,\,$ & $\,\,$\color{red} 3.8331\color{black}  $\,\,$ \\
$\,\,$0.0864$\,\,$ & $\,\,$0.2594$\,\,$ & $\,\,$\color{red} 0.2609\color{black} $\,\,$ & $\,\,$ 1  $\,\,$ \\
\end{pmatrix},
\end{equation*}

\begin{equation*}
\mathbf{w}^{\prime} =
\begin{pmatrix}
0.570630\\
0.190037\\
0.190037\\
0.049296
\end{pmatrix} =
0.998919\cdot
\begin{pmatrix}
0.571248\\
0.190242\\
\color{gr} 0.190242\color{black} \\
0.049349
\end{pmatrix},
\end{equation*}
\begin{equation*}
\left[ \frac{{w}^{\prime}_i}{{w}^{\prime}_j} \right] =
\begin{pmatrix}
$\,\,$ 1 $\,\,$ & $\,\,$3.0027$\,\,$ & $\,\,$\color{gr} 3.0027\color{black} $\,\,$ & $\,\,$11.5756$\,\,$ \\
$\,\,$0.3330$\,\,$ & $\,\,$ 1 $\,\,$ & $\,\,$\color{gr} \color{blue} 1\color{black} $\,\,$ & $\,\,$3.8550  $\,\,$ \\
$\,\,$\color{gr} 0.3330\color{black} $\,\,$ & $\,\,$\color{gr} \color{blue} 1\color{black} $\,\,$ & $\,\,$ 1 $\,\,$ & $\,\,$\color{gr} 3.8550\color{black}  $\,\,$ \\
$\,\,$0.0864$\,\,$ & $\,\,$0.2594$\,\,$ & $\,\,$\color{gr} 0.2594\color{black} $\,\,$ & $\,\,$ 1  $\,\,$ \\
\end{pmatrix},
\end{equation*}
\end{example}
\newpage
\begin{example}
\begin{equation*}
\mathbf{A} =
\begin{pmatrix}
$\,\,$ 1 $\,\,$ & $\,\,$5$\,\,$ & $\,\,$3$\,\,$ & $\,\,$8 $\,\,$ \\
$\,\,$ 1/5$\,\,$ & $\,\,$ 1 $\,\,$ & $\,\,$1$\,\,$ & $\,\,$7 $\,\,$ \\
$\,\,$ 1/3$\,\,$ & $\,\,$ 1 $\,\,$ & $\,\,$ 1 $\,\,$ & $\,\,$4 $\,\,$ \\
$\,\,$ 1/8$\,\,$ & $\,\,$ 1/7$\,\,$ & $\,\,$ 1/4$\,\,$ & $\,\,$ 1  $\,\,$ \\
\end{pmatrix},
\qquad
\lambda_{\max} =
4.1888,
\qquad
CR = 0.0712
\end{equation*}

\begin{equation*}
\mathbf{w}^{AMAST} =
\begin{pmatrix}
0.566233\\
0.197953\\
\color{red} 0.188056\color{black} \\
0.047759
\end{pmatrix}\end{equation*}
\begin{equation*}
\left[ \frac{{w}^{AMAST}_i}{{w}^{AMAST}_j} \right] =
\begin{pmatrix}
$\,\,$ 1 $\,\,$ & $\,\,$2.8604$\,\,$ & $\,\,$\color{red} 3.0110\color{black} $\,\,$ & $\,\,$11.8561$\,\,$ \\
$\,\,$0.3496$\,\,$ & $\,\,$ 1 $\,\,$ & $\,\,$\color{red} 1.0526\color{black} $\,\,$ & $\,\,$4.1448  $\,\,$ \\
$\,\,$\color{red} 0.3321\color{black} $\,\,$ & $\,\,$\color{red} 0.9500\color{black} $\,\,$ & $\,\,$ 1 $\,\,$ & $\,\,$\color{red} 3.9376\color{black}  $\,\,$ \\
$\,\,$0.0843$\,\,$ & $\,\,$0.2413$\,\,$ & $\,\,$\color{red} 0.2540\color{black} $\,\,$ & $\,\,$ 1  $\,\,$ \\
\end{pmatrix},
\end{equation*}

\begin{equation*}
\mathbf{w}^{\prime} =
\begin{pmatrix}
0.565843\\
0.197817\\
0.188614\\
0.047726
\end{pmatrix} =
0.999312\cdot
\begin{pmatrix}
0.566233\\
0.197953\\
\color{gr} 0.188744\color{black} \\
0.047759
\end{pmatrix},
\end{equation*}
\begin{equation*}
\left[ \frac{{w}^{\prime}_i}{{w}^{\prime}_j} \right] =
\begin{pmatrix}
$\,\,$ 1 $\,\,$ & $\,\,$2.8604$\,\,$ & $\,\,$\color{gr} \color{blue} 3\color{black} $\,\,$ & $\,\,$11.8561$\,\,$ \\
$\,\,$0.3496$\,\,$ & $\,\,$ 1 $\,\,$ & $\,\,$\color{gr} 1.0488\color{black} $\,\,$ & $\,\,$4.1448  $\,\,$ \\
$\,\,$\color{gr} \color{blue}  1/3\color{black} $\,\,$ & $\,\,$\color{gr} 0.9535\color{black} $\,\,$ & $\,\,$ 1 $\,\,$ & $\,\,$\color{gr} 3.9520\color{black}  $\,\,$ \\
$\,\,$0.0843$\,\,$ & $\,\,$0.2413$\,\,$ & $\,\,$\color{gr} 0.2530\color{black} $\,\,$ & $\,\,$ 1  $\,\,$ \\
\end{pmatrix},
\end{equation*}
\end{example}
\newpage
\begin{example}
\begin{equation*}
\mathbf{A} =
\begin{pmatrix}
$\,\,$ 1 $\,\,$ & $\,\,$5$\,\,$ & $\,\,$3$\,\,$ & $\,\,$9 $\,\,$ \\
$\,\,$ 1/5$\,\,$ & $\,\,$ 1 $\,\,$ & $\,\,$1$\,\,$ & $\,\,$6 $\,\,$ \\
$\,\,$ 1/3$\,\,$ & $\,\,$ 1 $\,\,$ & $\,\,$ 1 $\,\,$ & $\,\,$4 $\,\,$ \\
$\,\,$ 1/9$\,\,$ & $\,\,$ 1/6$\,\,$ & $\,\,$ 1/4$\,\,$ & $\,\,$ 1  $\,\,$ \\
\end{pmatrix},
\qquad
\lambda_{\max} =
4.1252,
\qquad
CR = 0.0472
\end{equation*}

\begin{equation*}
\mathbf{w}^{AMAST} =
\begin{pmatrix}
0.580280\\
0.186638\\
\color{red} 0.186280\color{black} \\
0.046802
\end{pmatrix}\end{equation*}
\begin{equation*}
\left[ \frac{{w}^{AMAST}_i}{{w}^{AMAST}_j} \right] =
\begin{pmatrix}
$\,\,$ 1 $\,\,$ & $\,\,$3.1091$\,\,$ & $\,\,$\color{red} 3.1151\color{black} $\,\,$ & $\,\,$12.3985$\,\,$ \\
$\,\,$0.3216$\,\,$ & $\,\,$ 1 $\,\,$ & $\,\,$\color{red} 1.0019\color{black} $\,\,$ & $\,\,$3.9878  $\,\,$ \\
$\,\,$\color{red} 0.3210\color{black} $\,\,$ & $\,\,$\color{red} 0.9981\color{black} $\,\,$ & $\,\,$ 1 $\,\,$ & $\,\,$\color{red} 3.9801\color{black}  $\,\,$ \\
$\,\,$0.0807$\,\,$ & $\,\,$0.2508$\,\,$ & $\,\,$\color{red} 0.2512\color{black} $\,\,$ & $\,\,$ 1  $\,\,$ \\
\end{pmatrix},
\end{equation*}

\begin{equation*}
\mathbf{w}^{\prime} =
\begin{pmatrix}
0.580072\\
0.186571\\
0.186571\\
0.046786
\end{pmatrix} =
0.999642\cdot
\begin{pmatrix}
0.580280\\
0.186638\\
\color{gr} 0.186638\color{black} \\
0.046802
\end{pmatrix},
\end{equation*}
\begin{equation*}
\left[ \frac{{w}^{\prime}_i}{{w}^{\prime}_j} \right] =
\begin{pmatrix}
$\,\,$ 1 $\,\,$ & $\,\,$3.1091$\,\,$ & $\,\,$\color{gr} 3.1091\color{black} $\,\,$ & $\,\,$12.3985$\,\,$ \\
$\,\,$0.3216$\,\,$ & $\,\,$ 1 $\,\,$ & $\,\,$\color{gr} \color{blue} 1\color{black} $\,\,$ & $\,\,$3.9878  $\,\,$ \\
$\,\,$\color{gr} 0.3216\color{black} $\,\,$ & $\,\,$\color{gr} \color{blue} 1\color{black} $\,\,$ & $\,\,$ 1 $\,\,$ & $\,\,$\color{gr} 3.9878\color{black}  $\,\,$ \\
$\,\,$0.0807$\,\,$ & $\,\,$0.2508$\,\,$ & $\,\,$\color{gr} 0.2508\color{black} $\,\,$ & $\,\,$ 1  $\,\,$ \\
\end{pmatrix},
\end{equation*}
\end{example}
\newpage
\begin{example}
\begin{equation*}
\mathbf{A} =
\begin{pmatrix}
$\,\,$ 1 $\,\,$ & $\,\,$5$\,\,$ & $\,\,$5$\,\,$ & $\,\,$7 $\,\,$ \\
$\,\,$ 1/5$\,\,$ & $\,\,$ 1 $\,\,$ & $\,\,$2$\,\,$ & $\,\,$7 $\,\,$ \\
$\,\,$ 1/5$\,\,$ & $\,\,$ 1/2$\,\,$ & $\,\,$ 1 $\,\,$ & $\,\,$2 $\,\,$ \\
$\,\,$ 1/7$\,\,$ & $\,\,$ 1/7$\,\,$ & $\,\,$ 1/2$\,\,$ & $\,\,$ 1  $\,\,$ \\
\end{pmatrix},
\qquad
\lambda_{\max} =
4.2287,
\qquad
CR = 0.0862
\end{equation*}

\begin{equation*}
\mathbf{w}^{AMAST} =
\begin{pmatrix}
0.598472\\
0.229843\\
\color{red} 0.114353\color{black} \\
0.057332
\end{pmatrix}\end{equation*}
\begin{equation*}
\left[ \frac{{w}^{AMAST}_i}{{w}^{AMAST}_j} \right] =
\begin{pmatrix}
$\,\,$ 1 $\,\,$ & $\,\,$2.6038$\,\,$ & $\,\,$\color{red} 5.2336\color{black} $\,\,$ & $\,\,$10.4387$\,\,$ \\
$\,\,$0.3841$\,\,$ & $\,\,$ 1 $\,\,$ & $\,\,$\color{red} 2.0099\color{black} $\,\,$ & $\,\,$4.0090  $\,\,$ \\
$\,\,$\color{red} 0.1911\color{black} $\,\,$ & $\,\,$\color{red} 0.4975\color{black} $\,\,$ & $\,\,$ 1 $\,\,$ & $\,\,$\color{red} 1.9946\color{black}  $\,\,$ \\
$\,\,$0.0958$\,\,$ & $\,\,$0.2494$\,\,$ & $\,\,$\color{red} 0.5014\color{black} $\,\,$ & $\,\,$ 1  $\,\,$ \\
\end{pmatrix},
\end{equation*}

\begin{equation*}
\mathbf{w}^{\prime} =
\begin{pmatrix}
0.598286\\
0.229772\\
0.114628\\
0.057314
\end{pmatrix} =
0.999689\cdot
\begin{pmatrix}
0.598472\\
0.229843\\
\color{gr} 0.114664\color{black} \\
0.057332
\end{pmatrix},
\end{equation*}
\begin{equation*}
\left[ \frac{{w}^{\prime}_i}{{w}^{\prime}_j} \right] =
\begin{pmatrix}
$\,\,$ 1 $\,\,$ & $\,\,$2.6038$\,\,$ & $\,\,$\color{gr} 5.2194\color{black} $\,\,$ & $\,\,$10.4387$\,\,$ \\
$\,\,$0.3841$\,\,$ & $\,\,$ 1 $\,\,$ & $\,\,$\color{gr} 2.0045\color{black} $\,\,$ & $\,\,$4.0090  $\,\,$ \\
$\,\,$\color{gr} 0.1916\color{black} $\,\,$ & $\,\,$\color{gr} 0.4989\color{black} $\,\,$ & $\,\,$ 1 $\,\,$ & $\,\,$\color{gr} \color{blue} 2\color{black}  $\,\,$ \\
$\,\,$0.0958$\,\,$ & $\,\,$0.2494$\,\,$ & $\,\,$\color{gr} \color{blue}  1/2\color{black} $\,\,$ & $\,\,$ 1  $\,\,$ \\
\end{pmatrix},
\end{equation*}
\end{example}
\newpage
\begin{example}
\begin{equation*}
\mathbf{A} =
\begin{pmatrix}
$\,\,$ 1 $\,\,$ & $\,\,$6$\,\,$ & $\,\,$3$\,\,$ & $\,\,$6 $\,\,$ \\
$\,\,$ 1/6$\,\,$ & $\,\,$ 1 $\,\,$ & $\,\,$1$\,\,$ & $\,\,$5 $\,\,$ \\
$\,\,$ 1/3$\,\,$ & $\,\,$ 1 $\,\,$ & $\,\,$ 1 $\,\,$ & $\,\,$3 $\,\,$ \\
$\,\,$ 1/6$\,\,$ & $\,\,$ 1/5$\,\,$ & $\,\,$ 1/3$\,\,$ & $\,\,$ 1  $\,\,$ \\
\end{pmatrix},
\qquad
\lambda_{\max} =
4.2277,
\qquad
CR = 0.0859
\end{equation*}

\begin{equation*}
\mathbf{w}^{AMAST} =
\begin{pmatrix}
0.570866\\
0.183693\\
\color{red} 0.182690\color{black} \\
0.062752
\end{pmatrix}\end{equation*}
\begin{equation*}
\left[ \frac{{w}^{AMAST}_i}{{w}^{AMAST}_j} \right] =
\begin{pmatrix}
$\,\,$ 1 $\,\,$ & $\,\,$3.1077$\,\,$ & $\,\,$\color{red} 3.1248\color{black} $\,\,$ & $\,\,$9.0972$\,\,$ \\
$\,\,$0.3218$\,\,$ & $\,\,$ 1 $\,\,$ & $\,\,$\color{red} 1.0055\color{black} $\,\,$ & $\,\,$2.9273  $\,\,$ \\
$\,\,$\color{red} 0.3200\color{black} $\,\,$ & $\,\,$\color{red} 0.9945\color{black} $\,\,$ & $\,\,$ 1 $\,\,$ & $\,\,$\color{red} 2.9113\color{black}  $\,\,$ \\
$\,\,$0.1099$\,\,$ & $\,\,$0.3416$\,\,$ & $\,\,$\color{red} 0.3435\color{black} $\,\,$ & $\,\,$ 1  $\,\,$ \\
\end{pmatrix},
\end{equation*}

\begin{equation*}
\mathbf{w}^{\prime} =
\begin{pmatrix}
0.570294\\
0.183509\\
0.183509\\
0.062689
\end{pmatrix} =
0.998998\cdot
\begin{pmatrix}
0.570866\\
0.183693\\
\color{gr} 0.183693\color{black} \\
0.062752
\end{pmatrix},
\end{equation*}
\begin{equation*}
\left[ \frac{{w}^{\prime}_i}{{w}^{\prime}_j} \right] =
\begin{pmatrix}
$\,\,$ 1 $\,\,$ & $\,\,$3.1077$\,\,$ & $\,\,$\color{gr} 3.1077\color{black} $\,\,$ & $\,\,$9.0972$\,\,$ \\
$\,\,$0.3218$\,\,$ & $\,\,$ 1 $\,\,$ & $\,\,$\color{gr} \color{blue} 1\color{black} $\,\,$ & $\,\,$2.9273  $\,\,$ \\
$\,\,$\color{gr} 0.3218\color{black} $\,\,$ & $\,\,$\color{gr} \color{blue} 1\color{black} $\,\,$ & $\,\,$ 1 $\,\,$ & $\,\,$\color{gr} 2.9273\color{black}  $\,\,$ \\
$\,\,$0.1099$\,\,$ & $\,\,$0.3416$\,\,$ & $\,\,$\color{gr} 0.3416\color{black} $\,\,$ & $\,\,$ 1  $\,\,$ \\
\end{pmatrix},
\end{equation*}
\end{example}
\newpage
\begin{example}
\begin{equation*}
\mathbf{A} =
\begin{pmatrix}
$\,\,$ 1 $\,\,$ & $\,\,$6$\,\,$ & $\,\,$3$\,\,$ & $\,\,$8 $\,\,$ \\
$\,\,$ 1/6$\,\,$ & $\,\,$ 1 $\,\,$ & $\,\,$1$\,\,$ & $\,\,$7 $\,\,$ \\
$\,\,$ 1/3$\,\,$ & $\,\,$ 1 $\,\,$ & $\,\,$ 1 $\,\,$ & $\,\,$4 $\,\,$ \\
$\,\,$ 1/8$\,\,$ & $\,\,$ 1/7$\,\,$ & $\,\,$ 1/4$\,\,$ & $\,\,$ 1  $\,\,$ \\
\end{pmatrix},
\qquad
\lambda_{\max} =
4.2421,
\qquad
CR = 0.0913
\end{equation*}

\begin{equation*}
\mathbf{w}^{AMAST} =
\begin{pmatrix}
0.578053\\
0.189090\\
\color{red} 0.185374\color{black} \\
0.047483
\end{pmatrix}\end{equation*}
\begin{equation*}
\left[ \frac{{w}^{AMAST}_i}{{w}^{AMAST}_j} \right] =
\begin{pmatrix}
$\,\,$ 1 $\,\,$ & $\,\,$3.0570$\,\,$ & $\,\,$\color{red} 3.1183\color{black} $\,\,$ & $\,\,$12.1738$\,\,$ \\
$\,\,$0.3271$\,\,$ & $\,\,$ 1 $\,\,$ & $\,\,$\color{red} 1.0200\color{black} $\,\,$ & $\,\,$3.9822  $\,\,$ \\
$\,\,$\color{red} 0.3207\color{black} $\,\,$ & $\,\,$\color{red} 0.9804\color{black} $\,\,$ & $\,\,$ 1 $\,\,$ & $\,\,$\color{red} 3.9040\color{black}  $\,\,$ \\
$\,\,$0.0821$\,\,$ & $\,\,$0.2511$\,\,$ & $\,\,$\color{red} 0.2561\color{black} $\,\,$ & $\,\,$ 1  $\,\,$ \\
\end{pmatrix},
\end{equation*}

\begin{equation*}
\mathbf{w}^{\prime} =
\begin{pmatrix}
0.575913\\
0.188390\\
0.188390\\
0.047307
\end{pmatrix} =
0.996298\cdot
\begin{pmatrix}
0.578053\\
0.189090\\
\color{gr} 0.189090\color{black} \\
0.047483
\end{pmatrix},
\end{equation*}
\begin{equation*}
\left[ \frac{{w}^{\prime}_i}{{w}^{\prime}_j} \right] =
\begin{pmatrix}
$\,\,$ 1 $\,\,$ & $\,\,$3.0570$\,\,$ & $\,\,$\color{gr} 3.0570\color{black} $\,\,$ & $\,\,$12.1738$\,\,$ \\
$\,\,$0.3271$\,\,$ & $\,\,$ 1 $\,\,$ & $\,\,$\color{gr} \color{blue} 1\color{black} $\,\,$ & $\,\,$3.9822  $\,\,$ \\
$\,\,$\color{gr} 0.3271\color{black} $\,\,$ & $\,\,$\color{gr} \color{blue} 1\color{black} $\,\,$ & $\,\,$ 1 $\,\,$ & $\,\,$\color{gr} 3.9822\color{black}  $\,\,$ \\
$\,\,$0.0821$\,\,$ & $\,\,$0.2511$\,\,$ & $\,\,$\color{gr} 0.2511\color{black} $\,\,$ & $\,\,$ 1  $\,\,$ \\
\end{pmatrix},
\end{equation*}
\end{example}
\newpage
\begin{example}
\begin{equation*}
\mathbf{A} =
\begin{pmatrix}
$\,\,$ 1 $\,\,$ & $\,\,$6$\,\,$ & $\,\,$6$\,\,$ & $\,\,$8 $\,\,$ \\
$\,\,$ 1/6$\,\,$ & $\,\,$ 1 $\,\,$ & $\,\,$2$\,\,$ & $\,\,$7 $\,\,$ \\
$\,\,$ 1/6$\,\,$ & $\,\,$ 1/2$\,\,$ & $\,\,$ 1 $\,\,$ & $\,\,$2 $\,\,$ \\
$\,\,$ 1/8$\,\,$ & $\,\,$ 1/7$\,\,$ & $\,\,$ 1/2$\,\,$ & $\,\,$ 1  $\,\,$ \\
\end{pmatrix},
\qquad
\lambda_{\max} =
4.2421,
\qquad
CR = 0.0913
\end{equation*}

\begin{equation*}
\mathbf{w}^{AMAST} =
\begin{pmatrix}
0.634965\\
0.209134\\
\color{red} 0.103324\color{black} \\
0.052577
\end{pmatrix}\end{equation*}
\begin{equation*}
\left[ \frac{{w}^{AMAST}_i}{{w}^{AMAST}_j} \right] =
\begin{pmatrix}
$\,\,$ 1 $\,\,$ & $\,\,$3.0362$\,\,$ & $\,\,$\color{red} 6.1454\color{black} $\,\,$ & $\,\,$12.0769$\,\,$ \\
$\,\,$0.3294$\,\,$ & $\,\,$ 1 $\,\,$ & $\,\,$\color{red} 2.0241\color{black} $\,\,$ & $\,\,$3.9777  $\,\,$ \\
$\,\,$\color{red} 0.1627\color{black} $\,\,$ & $\,\,$\color{red} 0.4941\color{black} $\,\,$ & $\,\,$ 1 $\,\,$ & $\,\,$\color{red} 1.9652\color{black}  $\,\,$ \\
$\,\,$0.0828$\,\,$ & $\,\,$0.2514$\,\,$ & $\,\,$\color{red} 0.5089\color{black} $\,\,$ & $\,\,$ 1  $\,\,$ \\
\end{pmatrix},
\end{equation*}

\begin{equation*}
\mathbf{w}^{\prime} =
\begin{pmatrix}
0.634177\\
0.208875\\
0.104437\\
0.052511
\end{pmatrix} =
0.998759\cdot
\begin{pmatrix}
0.634965\\
0.209134\\
\color{gr} 0.104567\color{black} \\
0.052577
\end{pmatrix},
\end{equation*}
\begin{equation*}
\left[ \frac{{w}^{\prime}_i}{{w}^{\prime}_j} \right] =
\begin{pmatrix}
$\,\,$ 1 $\,\,$ & $\,\,$3.0362$\,\,$ & $\,\,$\color{gr} 6.0723\color{black} $\,\,$ & $\,\,$12.0769$\,\,$ \\
$\,\,$0.3294$\,\,$ & $\,\,$ 1 $\,\,$ & $\,\,$\color{gr} \color{blue} 2\color{black} $\,\,$ & $\,\,$3.9777  $\,\,$ \\
$\,\,$\color{gr} 0.1647\color{black} $\,\,$ & $\,\,$\color{gr} \color{blue}  1/2\color{black} $\,\,$ & $\,\,$ 1 $\,\,$ & $\,\,$\color{gr} 1.9888\color{black}  $\,\,$ \\
$\,\,$0.0828$\,\,$ & $\,\,$0.2514$\,\,$ & $\,\,$\color{gr} 0.5028\color{black} $\,\,$ & $\,\,$ 1  $\,\,$ \\
\end{pmatrix},
\end{equation*}
\end{example}
\newpage
\begin{example}
\begin{equation*}
\mathbf{A} =
\begin{pmatrix}
$\,\,$ 1 $\,\,$ & $\,\,$7$\,\,$ & $\,\,$4$\,\,$ & $\,\,$8 $\,\,$ \\
$\,\,$ 1/7$\,\,$ & $\,\,$ 1 $\,\,$ & $\,\,$1$\,\,$ & $\,\,$5 $\,\,$ \\
$\,\,$ 1/4$\,\,$ & $\,\,$ 1 $\,\,$ & $\,\,$ 1 $\,\,$ & $\,\,$3 $\,\,$ \\
$\,\,$ 1/8$\,\,$ & $\,\,$ 1/5$\,\,$ & $\,\,$ 1/3$\,\,$ & $\,\,$ 1  $\,\,$ \\
\end{pmatrix},
\qquad
\lambda_{\max} =
4.1888,
\qquad
CR = 0.0712
\end{equation*}

\begin{equation*}
\mathbf{w}^{AMAST} =
\begin{pmatrix}
0.628261\\
0.161875\\
\color{red} 0.156500\color{black} \\
0.053365
\end{pmatrix}\end{equation*}
\begin{equation*}
\left[ \frac{{w}^{AMAST}_i}{{w}^{AMAST}_j} \right] =
\begin{pmatrix}
$\,\,$ 1 $\,\,$ & $\,\,$3.8812$\,\,$ & $\,\,$\color{red} 4.0144\color{black} $\,\,$ & $\,\,$11.7730$\,\,$ \\
$\,\,$0.2577$\,\,$ & $\,\,$ 1 $\,\,$ & $\,\,$\color{red} 1.0343\color{black} $\,\,$ & $\,\,$3.0334  $\,\,$ \\
$\,\,$\color{red} 0.2491\color{black} $\,\,$ & $\,\,$\color{red} 0.9668\color{black} $\,\,$ & $\,\,$ 1 $\,\,$ & $\,\,$\color{red} 2.9327\color{black}  $\,\,$ \\
$\,\,$0.0849$\,\,$ & $\,\,$0.3297$\,\,$ & $\,\,$\color{red} 0.3410\color{black} $\,\,$ & $\,\,$ 1  $\,\,$ \\
\end{pmatrix},
\end{equation*}

\begin{equation*}
\mathbf{w}^{\prime} =
\begin{pmatrix}
0.627906\\
0.161783\\
0.156976\\
0.053334
\end{pmatrix} =
0.999435\cdot
\begin{pmatrix}
0.628261\\
0.161875\\
\color{gr} 0.157065\color{black} \\
0.053365
\end{pmatrix},
\end{equation*}
\begin{equation*}
\left[ \frac{{w}^{\prime}_i}{{w}^{\prime}_j} \right] =
\begin{pmatrix}
$\,\,$ 1 $\,\,$ & $\,\,$3.8812$\,\,$ & $\,\,$\color{gr} \color{blue} 4\color{black} $\,\,$ & $\,\,$11.7730$\,\,$ \\
$\,\,$0.2577$\,\,$ & $\,\,$ 1 $\,\,$ & $\,\,$\color{gr} 1.0306\color{black} $\,\,$ & $\,\,$3.0334  $\,\,$ \\
$\,\,$\color{gr} \color{blue}  1/4\color{black} $\,\,$ & $\,\,$\color{gr} 0.9703\color{black} $\,\,$ & $\,\,$ 1 $\,\,$ & $\,\,$\color{gr} 2.9432\color{black}  $\,\,$ \\
$\,\,$0.0849$\,\,$ & $\,\,$0.3297$\,\,$ & $\,\,$\color{gr} 0.3398\color{black} $\,\,$ & $\,\,$ 1  $\,\,$ \\
\end{pmatrix},
\end{equation*}
\end{example}
\newpage
\begin{example}
\begin{equation*}
\mathbf{A} =
\begin{pmatrix}
$\,\,$ 1 $\,\,$ & $\,\,$7$\,\,$ & $\,\,$7$\,\,$ & $\,\,$9 $\,\,$ \\
$\,\,$ 1/7$\,\,$ & $\,\,$ 1 $\,\,$ & $\,\,$2$\,\,$ & $\,\,$7 $\,\,$ \\
$\,\,$ 1/7$\,\,$ & $\,\,$ 1/2$\,\,$ & $\,\,$ 1 $\,\,$ & $\,\,$2 $\,\,$ \\
$\,\,$ 1/9$\,\,$ & $\,\,$ 1/7$\,\,$ & $\,\,$ 1/2$\,\,$ & $\,\,$ 1  $\,\,$ \\
\end{pmatrix},
\qquad
\lambda_{\max} =
4.2526,
\qquad
CR = 0.0952
\end{equation*}

\begin{equation*}
\mathbf{w}^{AMAST} =
\begin{pmatrix}
0.664884\\
0.192185\\
\color{red} 0.094382\color{black} \\
0.048549
\end{pmatrix}\end{equation*}
\begin{equation*}
\left[ \frac{{w}^{AMAST}_i}{{w}^{AMAST}_j} \right] =
\begin{pmatrix}
$\,\,$ 1 $\,\,$ & $\,\,$3.4596$\,\,$ & $\,\,$\color{red} 7.0446\color{black} $\,\,$ & $\,\,$13.6952$\,\,$ \\
$\,\,$0.2890$\,\,$ & $\,\,$ 1 $\,\,$ & $\,\,$\color{red} 2.0362\color{black} $\,\,$ & $\,\,$3.9586  $\,\,$ \\
$\,\,$\color{red} 0.1420\color{black} $\,\,$ & $\,\,$\color{red} 0.4911\color{black} $\,\,$ & $\,\,$ 1 $\,\,$ & $\,\,$\color{red} 1.9441\color{black}  $\,\,$ \\
$\,\,$0.0730$\,\,$ & $\,\,$0.2526$\,\,$ & $\,\,$\color{red} 0.5144\color{black} $\,\,$ & $\,\,$ 1  $\,\,$ \\
\end{pmatrix},
\end{equation*}

\begin{equation*}
\mathbf{w}^{\prime} =
\begin{pmatrix}
0.664485\\
0.192069\\
0.094926\\
0.048520
\end{pmatrix} =
0.999399\cdot
\begin{pmatrix}
0.664884\\
0.192185\\
\color{gr} 0.094983\color{black} \\
0.048549
\end{pmatrix},
\end{equation*}
\begin{equation*}
\left[ \frac{{w}^{\prime}_i}{{w}^{\prime}_j} \right] =
\begin{pmatrix}
$\,\,$ 1 $\,\,$ & $\,\,$3.4596$\,\,$ & $\,\,$\color{gr} \color{blue} 7\color{black} $\,\,$ & $\,\,$13.6952$\,\,$ \\
$\,\,$0.2890$\,\,$ & $\,\,$ 1 $\,\,$ & $\,\,$\color{gr} 2.0233\color{black} $\,\,$ & $\,\,$3.9586  $\,\,$ \\
$\,\,$\color{gr} \color{blue}  1/7\color{black} $\,\,$ & $\,\,$\color{gr} 0.4942\color{black} $\,\,$ & $\,\,$ 1 $\,\,$ & $\,\,$\color{gr} 1.9565\color{black}  $\,\,$ \\
$\,\,$0.0730$\,\,$ & $\,\,$0.2526$\,\,$ & $\,\,$\color{gr} 0.5111\color{black} $\,\,$ & $\,\,$ 1  $\,\,$ \\
\end{pmatrix},
\end{equation*}
\end{example}
\newpage
\begin{example}
\begin{equation*}
\mathbf{A} =
\begin{pmatrix}
$\,\,$ 1 $\,\,$ & $\,\,$8$\,\,$ & $\,\,$4$\,\,$ & $\,\,$8 $\,\,$ \\
$\,\,$ 1/8$\,\,$ & $\,\,$ 1 $\,\,$ & $\,\,$1$\,\,$ & $\,\,$5 $\,\,$ \\
$\,\,$ 1/4$\,\,$ & $\,\,$ 1 $\,\,$ & $\,\,$ 1 $\,\,$ & $\,\,$3 $\,\,$ \\
$\,\,$ 1/8$\,\,$ & $\,\,$ 1/5$\,\,$ & $\,\,$ 1/3$\,\,$ & $\,\,$ 1  $\,\,$ \\
\end{pmatrix},
\qquad
\lambda_{\max} =
4.2277,
\qquad
CR = 0.0859
\end{equation*}

\begin{equation*}
\mathbf{w}^{AMAST} =
\begin{pmatrix}
0.635793\\
0.156495\\
\color{red} 0.154681\color{black} \\
0.053031
\end{pmatrix}\end{equation*}
\begin{equation*}
\left[ \frac{{w}^{AMAST}_i}{{w}^{AMAST}_j} \right] =
\begin{pmatrix}
$\,\,$ 1 $\,\,$ & $\,\,$4.0627$\,\,$ & $\,\,$\color{red} 4.1103\color{black} $\,\,$ & $\,\,$11.9891$\,\,$ \\
$\,\,$0.2461$\,\,$ & $\,\,$ 1 $\,\,$ & $\,\,$\color{red} 1.0117\color{black} $\,\,$ & $\,\,$2.9510  $\,\,$ \\
$\,\,$\color{red} 0.2433\color{black} $\,\,$ & $\,\,$\color{red} 0.9884\color{black} $\,\,$ & $\,\,$ 1 $\,\,$ & $\,\,$\color{red} 2.9168\color{black}  $\,\,$ \\
$\,\,$0.0834$\,\,$ & $\,\,$0.3389$\,\,$ & $\,\,$\color{red} 0.3428\color{black} $\,\,$ & $\,\,$ 1  $\,\,$ \\
\end{pmatrix},
\end{equation*}

\begin{equation*}
\mathbf{w}^{\prime} =
\begin{pmatrix}
0.634642\\
0.156212\\
0.156212\\
0.052935
\end{pmatrix} =
0.998189\cdot
\begin{pmatrix}
0.635793\\
0.156495\\
\color{gr} 0.156495\color{black} \\
0.053031
\end{pmatrix},
\end{equation*}
\begin{equation*}
\left[ \frac{{w}^{\prime}_i}{{w}^{\prime}_j} \right] =
\begin{pmatrix}
$\,\,$ 1 $\,\,$ & $\,\,$4.0627$\,\,$ & $\,\,$\color{gr} 4.0627\color{black} $\,\,$ & $\,\,$11.9891$\,\,$ \\
$\,\,$0.2461$\,\,$ & $\,\,$ 1 $\,\,$ & $\,\,$\color{gr} \color{blue} 1\color{black} $\,\,$ & $\,\,$2.9510  $\,\,$ \\
$\,\,$\color{gr} 0.2461\color{black} $\,\,$ & $\,\,$\color{gr} \color{blue} 1\color{black} $\,\,$ & $\,\,$ 1 $\,\,$ & $\,\,$\color{gr} 2.9510\color{black}  $\,\,$ \\
$\,\,$0.0834$\,\,$ & $\,\,$0.3389$\,\,$ & $\,\,$\color{gr} 0.3389\color{black} $\,\,$ & $\,\,$ 1  $\,\,$ \\
\end{pmatrix},
\end{equation*}
\end{example}
\newpage
\begin{example}
\begin{equation*}
\mathbf{A} =
\begin{pmatrix}
$\,\,$ 1 $\,\,$ & $\,\,$1$\,\,$ & $\,\,$3$\,\,$ & $\,\,$3 $\,\,$ \\
$\,\,$ 1 $\,\,$ & $\,\,$ 1 $\,\,$ & $\,\,$4$\,\,$ & $\,\,$2 $\,\,$ \\
$\,\,$ 1/3$\,\,$ & $\,\,$ 1/4$\,\,$ & $\,\,$ 1 $\,\,$ & $\,\,$ 1/3 $\,\,$ \\
$\,\,$ 1/3$\,\,$ & $\,\,$ 1/2$\,\,$ & $\,\,$3$\,\,$ & $\,\,$ 1  $\,\,$ \\
\end{pmatrix},
\qquad
\lambda_{\max} =
4.1031,
\qquad
CR = 0.0389
\end{equation*}

\begin{equation*}
\mathbf{w}^{AMAST} =
\begin{pmatrix}
0.370383\\
\color{red} 0.356888\color{black} \\
0.089487\\
0.183241
\end{pmatrix}\end{equation*}
\begin{equation*}
\left[ \frac{{w}^{AMAST}_i}{{w}^{AMAST}_j} \right] =
\begin{pmatrix}
$\,\,$ 1 $\,\,$ & $\,\,$\color{red} 1.0378\color{black} $\,\,$ & $\,\,$4.1389$\,\,$ & $\,\,$2.0213$\,\,$ \\
$\,\,$\color{red} 0.9636\color{black} $\,\,$ & $\,\,$ 1 $\,\,$ & $\,\,$\color{red} 3.9881\color{black} $\,\,$ & $\,\,$\color{red} 1.9476\color{black}   $\,\,$ \\
$\,\,$0.2416$\,\,$ & $\,\,$\color{red} 0.2507\color{black} $\,\,$ & $\,\,$ 1 $\,\,$ & $\,\,$0.4884 $\,\,$ \\
$\,\,$0.4947$\,\,$ & $\,\,$\color{red} 0.5134\color{black} $\,\,$ & $\,\,$2.0477$\,\,$ & $\,\,$ 1  $\,\,$ \\
\end{pmatrix},
\end{equation*}

\begin{equation*}
\mathbf{w}^{\prime} =
\begin{pmatrix}
0.369991\\
0.357570\\
0.089392\\
0.183047
\end{pmatrix} =
0.998940\cdot
\begin{pmatrix}
0.370383\\
\color{gr} 0.357949\color{black} \\
0.089487\\
0.183241
\end{pmatrix},
\end{equation*}
\begin{equation*}
\left[ \frac{{w}^{\prime}_i}{{w}^{\prime}_j} \right] =
\begin{pmatrix}
$\,\,$ 1 $\,\,$ & $\,\,$\color{gr} 1.0347\color{black} $\,\,$ & $\,\,$4.1389$\,\,$ & $\,\,$2.0213$\,\,$ \\
$\,\,$\color{gr} 0.9664\color{black} $\,\,$ & $\,\,$ 1 $\,\,$ & $\,\,$\color{gr} \color{blue} 4\color{black} $\,\,$ & $\,\,$\color{gr} 1.9534\color{black}   $\,\,$ \\
$\,\,$0.2416$\,\,$ & $\,\,$\color{gr} \color{blue}  1/4\color{black} $\,\,$ & $\,\,$ 1 $\,\,$ & $\,\,$0.4884 $\,\,$ \\
$\,\,$0.4947$\,\,$ & $\,\,$\color{gr} 0.5119\color{black} $\,\,$ & $\,\,$2.0477$\,\,$ & $\,\,$ 1  $\,\,$ \\
\end{pmatrix},
\end{equation*}
\end{example}
\newpage
\begin{example}
\begin{equation*}
\mathbf{A} =
\begin{pmatrix}
$\,\,$ 1 $\,\,$ & $\,\,$1$\,\,$ & $\,\,$3$\,\,$ & $\,\,$4 $\,\,$ \\
$\,\,$ 1 $\,\,$ & $\,\,$ 1 $\,\,$ & $\,\,$5$\,\,$ & $\,\,$2 $\,\,$ \\
$\,\,$ 1/3$\,\,$ & $\,\,$ 1/5$\,\,$ & $\,\,$ 1 $\,\,$ & $\,\,$ 1/4 $\,\,$ \\
$\,\,$ 1/4$\,\,$ & $\,\,$ 1/2$\,\,$ & $\,\,$4$\,\,$ & $\,\,$ 1  $\,\,$ \\
\end{pmatrix},
\qquad
\lambda_{\max} =
4.2460,
\qquad
CR = 0.0928
\end{equation*}

\begin{equation*}
\mathbf{w}^{AMAST} =
\begin{pmatrix}
0.382031\\
\color{red} 0.359292\color{black} \\
0.078159\\
0.180517
\end{pmatrix}\end{equation*}
\begin{equation*}
\left[ \frac{{w}^{AMAST}_i}{{w}^{AMAST}_j} \right] =
\begin{pmatrix}
$\,\,$ 1 $\,\,$ & $\,\,$\color{red} 1.0633\color{black} $\,\,$ & $\,\,$4.8878$\,\,$ & $\,\,$2.1163$\,\,$ \\
$\,\,$\color{red} 0.9405\color{black} $\,\,$ & $\,\,$ 1 $\,\,$ & $\,\,$\color{red} 4.5969\color{black} $\,\,$ & $\,\,$\color{red} 1.9903\color{black}   $\,\,$ \\
$\,\,$0.2046$\,\,$ & $\,\,$\color{red} 0.2175\color{black} $\,\,$ & $\,\,$ 1 $\,\,$ & $\,\,$0.4330 $\,\,$ \\
$\,\,$0.4725$\,\,$ & $\,\,$\color{red} 0.5024\color{black} $\,\,$ & $\,\,$2.3096$\,\,$ & $\,\,$ 1  $\,\,$ \\
\end{pmatrix},
\end{equation*}

\begin{equation*}
\mathbf{w}^{\prime} =
\begin{pmatrix}
0.381367\\
0.360407\\
0.078023\\
0.180203
\end{pmatrix} =
0.998261\cdot
\begin{pmatrix}
0.382031\\
\color{gr} 0.361035\color{black} \\
0.078159\\
0.180517
\end{pmatrix},
\end{equation*}
\begin{equation*}
\left[ \frac{{w}^{\prime}_i}{{w}^{\prime}_j} \right] =
\begin{pmatrix}
$\,\,$ 1 $\,\,$ & $\,\,$\color{gr} 1.0582\color{black} $\,\,$ & $\,\,$4.8878$\,\,$ & $\,\,$2.1163$\,\,$ \\
$\,\,$\color{gr} 0.9450\color{black} $\,\,$ & $\,\,$ 1 $\,\,$ & $\,\,$\color{gr} 4.6192\color{black} $\,\,$ & $\,\,$\color{gr} \color{blue} 2\color{black}   $\,\,$ \\
$\,\,$0.2046$\,\,$ & $\,\,$\color{gr} 0.2165\color{black} $\,\,$ & $\,\,$ 1 $\,\,$ & $\,\,$0.4330 $\,\,$ \\
$\,\,$0.4725$\,\,$ & $\,\,$\color{gr} \color{blue}  1/2\color{black} $\,\,$ & $\,\,$2.3096$\,\,$ & $\,\,$ 1  $\,\,$ \\
\end{pmatrix},
\end{equation*}
\end{example}
\newpage
\begin{example}
\begin{equation*}
\mathbf{A} =
\begin{pmatrix}
$\,\,$ 1 $\,\,$ & $\,\,$1$\,\,$ & $\,\,$3$\,\,$ & $\,\,$5 $\,\,$ \\
$\,\,$ 1 $\,\,$ & $\,\,$ 1 $\,\,$ & $\,\,$4$\,\,$ & $\,\,$3 $\,\,$ \\
$\,\,$ 1/3$\,\,$ & $\,\,$ 1/4$\,\,$ & $\,\,$ 1 $\,\,$ & $\,\,$ 1/2 $\,\,$ \\
$\,\,$ 1/5$\,\,$ & $\,\,$ 1/3$\,\,$ & $\,\,$2$\,\,$ & $\,\,$ 1  $\,\,$ \\
\end{pmatrix},
\qquad
\lambda_{\max} =
4.1252,
\qquad
CR = 0.0472
\end{equation*}

\begin{equation*}
\mathbf{w}^{AMAST} =
\begin{pmatrix}
0.400821\\
\color{red} 0.376904\color{black} \\
0.094934\\
0.127341
\end{pmatrix}\end{equation*}
\begin{equation*}
\left[ \frac{{w}^{AMAST}_i}{{w}^{AMAST}_j} \right] =
\begin{pmatrix}
$\,\,$ 1 $\,\,$ & $\,\,$\color{red} 1.0635\color{black} $\,\,$ & $\,\,$4.2221$\,\,$ & $\,\,$3.1476$\,\,$ \\
$\,\,$\color{red} 0.9403\color{black} $\,\,$ & $\,\,$ 1 $\,\,$ & $\,\,$\color{red} 3.9702\color{black} $\,\,$ & $\,\,$\color{red} 2.9598\color{black}   $\,\,$ \\
$\,\,$0.2368$\,\,$ & $\,\,$\color{red} 0.2519\color{black} $\,\,$ & $\,\,$ 1 $\,\,$ & $\,\,$0.7455 $\,\,$ \\
$\,\,$0.3177$\,\,$ & $\,\,$\color{red} 0.3379\color{black} $\,\,$ & $\,\,$1.3414$\,\,$ & $\,\,$ 1  $\,\,$ \\
\end{pmatrix},
\end{equation*}

\begin{equation*}
\mathbf{w}^{\prime} =
\begin{pmatrix}
0.399689\\
0.378663\\
0.094666\\
0.126982
\end{pmatrix} =
0.997176\cdot
\begin{pmatrix}
0.400821\\
\color{gr} 0.379736\color{black} \\
0.094934\\
0.127341
\end{pmatrix},
\end{equation*}
\begin{equation*}
\left[ \frac{{w}^{\prime}_i}{{w}^{\prime}_j} \right] =
\begin{pmatrix}
$\,\,$ 1 $\,\,$ & $\,\,$\color{gr} 1.0555\color{black} $\,\,$ & $\,\,$4.2221$\,\,$ & $\,\,$3.1476$\,\,$ \\
$\,\,$\color{gr} 0.9474\color{black} $\,\,$ & $\,\,$ 1 $\,\,$ & $\,\,$\color{gr} \color{blue} 4\color{black} $\,\,$ & $\,\,$\color{gr} 2.9820\color{black}   $\,\,$ \\
$\,\,$0.2368$\,\,$ & $\,\,$\color{gr} \color{blue}  1/4\color{black} $\,\,$ & $\,\,$ 1 $\,\,$ & $\,\,$0.7455 $\,\,$ \\
$\,\,$0.3177$\,\,$ & $\,\,$\color{gr} 0.3353\color{black} $\,\,$ & $\,\,$1.3414$\,\,$ & $\,\,$ 1  $\,\,$ \\
\end{pmatrix},
\end{equation*}
\end{example}
\newpage
\begin{example}
\begin{equation*}
\mathbf{A} =
\begin{pmatrix}
$\,\,$ 1 $\,\,$ & $\,\,$1$\,\,$ & $\,\,$3$\,\,$ & $\,\,$5 $\,\,$ \\
$\,\,$ 1 $\,\,$ & $\,\,$ 1 $\,\,$ & $\,\,$5$\,\,$ & $\,\,$3 $\,\,$ \\
$\,\,$ 1/3$\,\,$ & $\,\,$ 1/5$\,\,$ & $\,\,$ 1 $\,\,$ & $\,\,$ 1/3 $\,\,$ \\
$\,\,$ 1/5$\,\,$ & $\,\,$ 1/3$\,\,$ & $\,\,$3$\,\,$ & $\,\,$ 1  $\,\,$ \\
\end{pmatrix},
\qquad
\lambda_{\max} =
4.2253,
\qquad
CR = 0.0849
\end{equation*}

\begin{equation*}
\mathbf{w}^{AMAST} =
\begin{pmatrix}
0.391881\\
\color{red} 0.386301\color{black} \\
0.081565\\
0.140253
\end{pmatrix}\end{equation*}
\begin{equation*}
\left[ \frac{{w}^{AMAST}_i}{{w}^{AMAST}_j} \right] =
\begin{pmatrix}
$\,\,$ 1 $\,\,$ & $\,\,$\color{red} 1.0144\color{black} $\,\,$ & $\,\,$4.8045$\,\,$ & $\,\,$2.7941$\,\,$ \\
$\,\,$\color{red} 0.9858\color{black} $\,\,$ & $\,\,$ 1 $\,\,$ & $\,\,$\color{red} 4.7361\color{black} $\,\,$ & $\,\,$\color{red} 2.7543\color{black}   $\,\,$ \\
$\,\,$0.2081$\,\,$ & $\,\,$\color{red} 0.2111\color{black} $\,\,$ & $\,\,$ 1 $\,\,$ & $\,\,$0.5816 $\,\,$ \\
$\,\,$0.3579$\,\,$ & $\,\,$\color{red} 0.3631\color{black} $\,\,$ & $\,\,$1.7195$\,\,$ & $\,\,$ 1  $\,\,$ \\
\end{pmatrix},
\end{equation*}

\begin{equation*}
\mathbf{w}^{\prime} =
\begin{pmatrix}
0.389706\\
0.389706\\
0.081112\\
0.139475
\end{pmatrix} =
0.994451\cdot
\begin{pmatrix}
0.391881\\
\color{gr} 0.391881\color{black} \\
0.081565\\
0.140253
\end{pmatrix},
\end{equation*}
\begin{equation*}
\left[ \frac{{w}^{\prime}_i}{{w}^{\prime}_j} \right] =
\begin{pmatrix}
$\,\,$ 1 $\,\,$ & $\,\,$\color{gr} \color{blue} 1\color{black} $\,\,$ & $\,\,$4.8045$\,\,$ & $\,\,$2.7941$\,\,$ \\
$\,\,$\color{gr} \color{blue} 1\color{black} $\,\,$ & $\,\,$ 1 $\,\,$ & $\,\,$\color{gr} 4.8045\color{black} $\,\,$ & $\,\,$\color{gr} 2.7941\color{black}   $\,\,$ \\
$\,\,$0.2081$\,\,$ & $\,\,$\color{gr} 0.2081\color{black} $\,\,$ & $\,\,$ 1 $\,\,$ & $\,\,$0.5816 $\,\,$ \\
$\,\,$0.3579$\,\,$ & $\,\,$\color{gr} 0.3579\color{black} $\,\,$ & $\,\,$1.7195$\,\,$ & $\,\,$ 1  $\,\,$ \\
\end{pmatrix},
\end{equation*}
\end{example}
\newpage
\begin{example}
\begin{equation*}
\mathbf{A} =
\begin{pmatrix}
$\,\,$ 1 $\,\,$ & $\,\,$1$\,\,$ & $\,\,$4$\,\,$ & $\,\,$3 $\,\,$ \\
$\,\,$ 1 $\,\,$ & $\,\,$ 1 $\,\,$ & $\,\,$6$\,\,$ & $\,\,$2 $\,\,$ \\
$\,\,$ 1/4$\,\,$ & $\,\,$ 1/6$\,\,$ & $\,\,$ 1 $\,\,$ & $\,\,$ 1/5 $\,\,$ \\
$\,\,$ 1/3$\,\,$ & $\,\,$ 1/2$\,\,$ & $\,\,$5$\,\,$ & $\,\,$ 1  $\,\,$ \\
\end{pmatrix},
\qquad
\lambda_{\max} =
4.1502,
\qquad
CR = 0.0566
\end{equation*}

\begin{equation*}
\mathbf{w}^{AMAST} =
\begin{pmatrix}
0.372729\\
\color{red} 0.368313\color{black} \\
0.063008\\
0.195951
\end{pmatrix}\end{equation*}
\begin{equation*}
\left[ \frac{{w}^{AMAST}_i}{{w}^{AMAST}_j} \right] =
\begin{pmatrix}
$\,\,$ 1 $\,\,$ & $\,\,$\color{red} 1.0120\color{black} $\,\,$ & $\,\,$5.9156$\,\,$ & $\,\,$1.9022$\,\,$ \\
$\,\,$\color{red} 0.9882\color{black} $\,\,$ & $\,\,$ 1 $\,\,$ & $\,\,$\color{red} 5.8455\color{black} $\,\,$ & $\,\,$\color{red} 1.8796\color{black}   $\,\,$ \\
$\,\,$0.1690$\,\,$ & $\,\,$\color{red} 0.1711\color{black} $\,\,$ & $\,\,$ 1 $\,\,$ & $\,\,$0.3215 $\,\,$ \\
$\,\,$0.5257$\,\,$ & $\,\,$\color{red} 0.5320\color{black} $\,\,$ & $\,\,$3.1100$\,\,$ & $\,\,$ 1  $\,\,$ \\
\end{pmatrix},
\end{equation*}

\begin{equation*}
\mathbf{w}^{\prime} =
\begin{pmatrix}
0.371090\\
0.371090\\
0.062731\\
0.195089
\end{pmatrix} =
0.995604\cdot
\begin{pmatrix}
0.372729\\
\color{gr} 0.372729\color{black} \\
0.063008\\
0.195951
\end{pmatrix},
\end{equation*}
\begin{equation*}
\left[ \frac{{w}^{\prime}_i}{{w}^{\prime}_j} \right] =
\begin{pmatrix}
$\,\,$ 1 $\,\,$ & $\,\,$\color{gr} \color{blue} 1\color{black} $\,\,$ & $\,\,$5.9156$\,\,$ & $\,\,$1.9022$\,\,$ \\
$\,\,$\color{gr} \color{blue} 1\color{black} $\,\,$ & $\,\,$ 1 $\,\,$ & $\,\,$\color{gr} 5.9156\color{black} $\,\,$ & $\,\,$\color{gr} 1.9022\color{black}   $\,\,$ \\
$\,\,$0.1690$\,\,$ & $\,\,$\color{gr} 0.1690\color{black} $\,\,$ & $\,\,$ 1 $\,\,$ & $\,\,$0.3215 $\,\,$ \\
$\,\,$0.5257$\,\,$ & $\,\,$\color{gr} 0.5257\color{black} $\,\,$ & $\,\,$3.1100$\,\,$ & $\,\,$ 1  $\,\,$ \\
\end{pmatrix},
\end{equation*}
\end{example}
\newpage
\begin{example}
\begin{equation*}
\mathbf{A} =
\begin{pmatrix}
$\,\,$ 1 $\,\,$ & $\,\,$1$\,\,$ & $\,\,$4$\,\,$ & $\,\,$4 $\,\,$ \\
$\,\,$ 1 $\,\,$ & $\,\,$ 1 $\,\,$ & $\,\,$6$\,\,$ & $\,\,$2 $\,\,$ \\
$\,\,$ 1/4$\,\,$ & $\,\,$ 1/6$\,\,$ & $\,\,$ 1 $\,\,$ & $\,\,$ 1/5 $\,\,$ \\
$\,\,$ 1/4$\,\,$ & $\,\,$ 1/2$\,\,$ & $\,\,$5$\,\,$ & $\,\,$ 1  $\,\,$ \\
\end{pmatrix},
\qquad
\lambda_{\max} =
4.2277,
\qquad
CR = 0.0859
\end{equation*}

\begin{equation*}
\mathbf{w}^{AMAST} =
\begin{pmatrix}
0.392871\\
\color{red} 0.361403\color{black} \\
0.062894\\
0.182832
\end{pmatrix}\end{equation*}
\begin{equation*}
\left[ \frac{{w}^{AMAST}_i}{{w}^{AMAST}_j} \right] =
\begin{pmatrix}
$\,\,$ 1 $\,\,$ & $\,\,$\color{red} 1.0871\color{black} $\,\,$ & $\,\,$6.2466$\,\,$ & $\,\,$2.1488$\,\,$ \\
$\,\,$\color{red} 0.9199\color{black} $\,\,$ & $\,\,$ 1 $\,\,$ & $\,\,$\color{red} 5.7462\color{black} $\,\,$ & $\,\,$\color{red} 1.9767\color{black}   $\,\,$ \\
$\,\,$0.1601$\,\,$ & $\,\,$\color{red} 0.1740\color{black} $\,\,$ & $\,\,$ 1 $\,\,$ & $\,\,$0.3440 $\,\,$ \\
$\,\,$0.4654$\,\,$ & $\,\,$\color{red} 0.5059\color{black} $\,\,$ & $\,\,$2.9070$\,\,$ & $\,\,$ 1  $\,\,$ \\
\end{pmatrix},
\end{equation*}

\begin{equation*}
\mathbf{w}^{\prime} =
\begin{pmatrix}
0.391204\\
0.364113\\
0.062627\\
0.182056
\end{pmatrix} =
0.995757\cdot
\begin{pmatrix}
0.392871\\
\color{gr} 0.365664\color{black} \\
0.062894\\
0.182832
\end{pmatrix},
\end{equation*}
\begin{equation*}
\left[ \frac{{w}^{\prime}_i}{{w}^{\prime}_j} \right] =
\begin{pmatrix}
$\,\,$ 1 $\,\,$ & $\,\,$\color{gr} 1.0744\color{black} $\,\,$ & $\,\,$6.2466$\,\,$ & $\,\,$2.1488$\,\,$ \\
$\,\,$\color{gr} 0.9307\color{black} $\,\,$ & $\,\,$ 1 $\,\,$ & $\,\,$\color{gr} 5.8140\color{black} $\,\,$ & $\,\,$\color{gr} \color{blue} 2\color{black}   $\,\,$ \\
$\,\,$0.1601$\,\,$ & $\,\,$\color{gr} 0.1720\color{black} $\,\,$ & $\,\,$ 1 $\,\,$ & $\,\,$0.3440 $\,\,$ \\
$\,\,$0.4654$\,\,$ & $\,\,$\color{gr} \color{blue}  1/2\color{black} $\,\,$ & $\,\,$2.9070$\,\,$ & $\,\,$ 1  $\,\,$ \\
\end{pmatrix},
\end{equation*}
\end{example}
\newpage
\begin{example}
\begin{equation*}
\mathbf{A} =
\begin{pmatrix}
$\,\,$ 1 $\,\,$ & $\,\,$1$\,\,$ & $\,\,$4$\,\,$ & $\,\,$5 $\,\,$ \\
$\,\,$ 1 $\,\,$ & $\,\,$ 1 $\,\,$ & $\,\,$6$\,\,$ & $\,\,$3 $\,\,$ \\
$\,\,$ 1/4$\,\,$ & $\,\,$ 1/6$\,\,$ & $\,\,$ 1 $\,\,$ & $\,\,$ 1/3 $\,\,$ \\
$\,\,$ 1/5$\,\,$ & $\,\,$ 1/3$\,\,$ & $\,\,$3$\,\,$ & $\,\,$ 1  $\,\,$ \\
\end{pmatrix},
\qquad
\lambda_{\max} =
4.1502,
\qquad
CR = 0.0566
\end{equation*}

\begin{equation*}
\mathbf{w}^{AMAST} =
\begin{pmatrix}
0.405870\\
\color{red} 0.392143\color{black} \\
0.068676\\
0.133311
\end{pmatrix}\end{equation*}
\begin{equation*}
\left[ \frac{{w}^{AMAST}_i}{{w}^{AMAST}_j} \right] =
\begin{pmatrix}
$\,\,$ 1 $\,\,$ & $\,\,$\color{red} 1.0350\color{black} $\,\,$ & $\,\,$5.9099$\,\,$ & $\,\,$3.0445$\,\,$ \\
$\,\,$\color{red} 0.9662\color{black} $\,\,$ & $\,\,$ 1 $\,\,$ & $\,\,$\color{red} 5.7100\color{black} $\,\,$ & $\,\,$\color{red} 2.9416\color{black}   $\,\,$ \\
$\,\,$0.1692$\,\,$ & $\,\,$\color{red} 0.1751\color{black} $\,\,$ & $\,\,$ 1 $\,\,$ & $\,\,$0.5152 $\,\,$ \\
$\,\,$0.3285$\,\,$ & $\,\,$\color{red} 0.3400\color{black} $\,\,$ & $\,\,$1.9411$\,\,$ & $\,\,$ 1  $\,\,$ \\
\end{pmatrix},
\end{equation*}

\begin{equation*}
\mathbf{w}^{\prime} =
\begin{pmatrix}
0.402733\\
0.396841\\
0.068146\\
0.132280
\end{pmatrix} =
0.992272\cdot
\begin{pmatrix}
0.405870\\
\color{gr} 0.399932\color{black} \\
0.068676\\
0.133311
\end{pmatrix},
\end{equation*}
\begin{equation*}
\left[ \frac{{w}^{\prime}_i}{{w}^{\prime}_j} \right] =
\begin{pmatrix}
$\,\,$ 1 $\,\,$ & $\,\,$\color{gr} 1.0148\color{black} $\,\,$ & $\,\,$5.9099$\,\,$ & $\,\,$3.0445$\,\,$ \\
$\,\,$\color{gr} 0.9854\color{black} $\,\,$ & $\,\,$ 1 $\,\,$ & $\,\,$\color{gr} 5.8234\color{black} $\,\,$ & $\,\,$\color{gr} \color{blue} 3\color{black}   $\,\,$ \\
$\,\,$0.1692$\,\,$ & $\,\,$\color{gr} 0.1717\color{black} $\,\,$ & $\,\,$ 1 $\,\,$ & $\,\,$0.5152 $\,\,$ \\
$\,\,$0.3285$\,\,$ & $\,\,$\color{gr} \color{blue}  1/3\color{black} $\,\,$ & $\,\,$1.9411$\,\,$ & $\,\,$ 1  $\,\,$ \\
\end{pmatrix},
\end{equation*}
\end{example}
\newpage
\begin{example}
\begin{equation*}
\mathbf{A} =
\begin{pmatrix}
$\,\,$ 1 $\,\,$ & $\,\,$1$\,\,$ & $\,\,$4$\,\,$ & $\,\,$5 $\,\,$ \\
$\,\,$ 1 $\,\,$ & $\,\,$ 1 $\,\,$ & $\,\,$6$\,\,$ & $\,\,$3 $\,\,$ \\
$\,\,$ 1/4$\,\,$ & $\,\,$ 1/6$\,\,$ & $\,\,$ 1 $\,\,$ & $\,\,$ 1/4 $\,\,$ \\
$\,\,$ 1/5$\,\,$ & $\,\,$ 1/3$\,\,$ & $\,\,$4$\,\,$ & $\,\,$ 1  $\,\,$ \\
\end{pmatrix},
\qquad
\lambda_{\max} =
4.2277,
\qquad
CR = 0.0859
\end{equation*}

\begin{equation*}
\mathbf{w}^{AMAST} =
\begin{pmatrix}
0.402519\\
\color{red} 0.387778\color{black} \\
0.064886\\
0.144816
\end{pmatrix}\end{equation*}
\begin{equation*}
\left[ \frac{{w}^{AMAST}_i}{{w}^{AMAST}_j} \right] =
\begin{pmatrix}
$\,\,$ 1 $\,\,$ & $\,\,$\color{red} 1.0380\color{black} $\,\,$ & $\,\,$6.2035$\,\,$ & $\,\,$2.7795$\,\,$ \\
$\,\,$\color{red} 0.9634\color{black} $\,\,$ & $\,\,$ 1 $\,\,$ & $\,\,$\color{red} 5.9763\color{black} $\,\,$ & $\,\,$\color{red} 2.6777\color{black}   $\,\,$ \\
$\,\,$0.1612$\,\,$ & $\,\,$\color{red} 0.1673\color{black} $\,\,$ & $\,\,$ 1 $\,\,$ & $\,\,$0.4481 $\,\,$ \\
$\,\,$0.3598$\,\,$ & $\,\,$\color{red} 0.3735\color{black} $\,\,$ & $\,\,$2.2319$\,\,$ & $\,\,$ 1  $\,\,$ \\
\end{pmatrix},
\end{equation*}

\begin{equation*}
\mathbf{w}^{\prime} =
\begin{pmatrix}
0.401901\\
0.388719\\
0.064786\\
0.144594
\end{pmatrix} =
0.998464\cdot
\begin{pmatrix}
0.402519\\
\color{gr} 0.389317\color{black} \\
0.064886\\
0.144816
\end{pmatrix},
\end{equation*}
\begin{equation*}
\left[ \frac{{w}^{\prime}_i}{{w}^{\prime}_j} \right] =
\begin{pmatrix}
$\,\,$ 1 $\,\,$ & $\,\,$\color{gr} 1.0339\color{black} $\,\,$ & $\,\,$6.2035$\,\,$ & $\,\,$2.7795$\,\,$ \\
$\,\,$\color{gr} 0.9672\color{black} $\,\,$ & $\,\,$ 1 $\,\,$ & $\,\,$\color{gr} \color{blue} 6\color{black} $\,\,$ & $\,\,$\color{gr} 2.6883\color{black}   $\,\,$ \\
$\,\,$0.1612$\,\,$ & $\,\,$\color{gr} \color{blue}  1/6\color{black} $\,\,$ & $\,\,$ 1 $\,\,$ & $\,\,$0.4481 $\,\,$ \\
$\,\,$0.3598$\,\,$ & $\,\,$\color{gr} 0.3720\color{black} $\,\,$ & $\,\,$2.2319$\,\,$ & $\,\,$ 1  $\,\,$ \\
\end{pmatrix},
\end{equation*}
\end{example}
\newpage
\begin{example}
\begin{equation*}
\mathbf{A} =
\begin{pmatrix}
$\,\,$ 1 $\,\,$ & $\,\,$1$\,\,$ & $\,\,$4$\,\,$ & $\,\,$5 $\,\,$ \\
$\,\,$ 1 $\,\,$ & $\,\,$ 1 $\,\,$ & $\,\,$7$\,\,$ & $\,\,$3 $\,\,$ \\
$\,\,$ 1/4$\,\,$ & $\,\,$ 1/7$\,\,$ & $\,\,$ 1 $\,\,$ & $\,\,$ 1/4 $\,\,$ \\
$\,\,$ 1/5$\,\,$ & $\,\,$ 1/3$\,\,$ & $\,\,$4$\,\,$ & $\,\,$ 1  $\,\,$ \\
\end{pmatrix},
\qquad
\lambda_{\max} =
4.2251,
\qquad
CR = 0.0849
\end{equation*}

\begin{equation*}
\mathbf{w}^{AMAST} =
\begin{pmatrix}
0.398499\\
\color{red} 0.397400\color{black} \\
0.061715\\
0.142386
\end{pmatrix}\end{equation*}
\begin{equation*}
\left[ \frac{{w}^{AMAST}_i}{{w}^{AMAST}_j} \right] =
\begin{pmatrix}
$\,\,$ 1 $\,\,$ & $\,\,$\color{red} 1.0028\color{black} $\,\,$ & $\,\,$6.4571$\,\,$ & $\,\,$2.7987$\,\,$ \\
$\,\,$\color{red} 0.9972\color{black} $\,\,$ & $\,\,$ 1 $\,\,$ & $\,\,$\color{red} 6.4393\color{black} $\,\,$ & $\,\,$\color{red} 2.7910\color{black}   $\,\,$ \\
$\,\,$0.1549$\,\,$ & $\,\,$\color{red} 0.1553\color{black} $\,\,$ & $\,\,$ 1 $\,\,$ & $\,\,$0.4334 $\,\,$ \\
$\,\,$0.3573$\,\,$ & $\,\,$\color{red} 0.3583\color{black} $\,\,$ & $\,\,$2.3072$\,\,$ & $\,\,$ 1  $\,\,$ \\
\end{pmatrix},
\end{equation*}

\begin{equation*}
\mathbf{w}^{\prime} =
\begin{pmatrix}
0.398061\\
0.398061\\
0.061647\\
0.142230
\end{pmatrix} =
0.998902\cdot
\begin{pmatrix}
0.398499\\
\color{gr} 0.398499\color{black} \\
0.061715\\
0.142386
\end{pmatrix},
\end{equation*}
\begin{equation*}
\left[ \frac{{w}^{\prime}_i}{{w}^{\prime}_j} \right] =
\begin{pmatrix}
$\,\,$ 1 $\,\,$ & $\,\,$\color{gr} \color{blue} 1\color{black} $\,\,$ & $\,\,$6.4571$\,\,$ & $\,\,$2.7987$\,\,$ \\
$\,\,$\color{gr} \color{blue} 1\color{black} $\,\,$ & $\,\,$ 1 $\,\,$ & $\,\,$\color{gr} 6.4571\color{black} $\,\,$ & $\,\,$\color{gr} 2.7987\color{black}   $\,\,$ \\
$\,\,$0.1549$\,\,$ & $\,\,$\color{gr} 0.1549\color{black} $\,\,$ & $\,\,$ 1 $\,\,$ & $\,\,$0.4334 $\,\,$ \\
$\,\,$0.3573$\,\,$ & $\,\,$\color{gr} 0.3573\color{black} $\,\,$ & $\,\,$2.3072$\,\,$ & $\,\,$ 1  $\,\,$ \\
\end{pmatrix},
\end{equation*}
\end{example}
\newpage
\begin{example}
\begin{equation*}
\mathbf{A} =
\begin{pmatrix}
$\,\,$ 1 $\,\,$ & $\,\,$1$\,\,$ & $\,\,$4$\,\,$ & $\,\,$6 $\,\,$ \\
$\,\,$ 1 $\,\,$ & $\,\,$ 1 $\,\,$ & $\,\,$6$\,\,$ & $\,\,$4 $\,\,$ \\
$\,\,$ 1/4$\,\,$ & $\,\,$ 1/6$\,\,$ & $\,\,$ 1 $\,\,$ & $\,\,$ 1/2 $\,\,$ \\
$\,\,$ 1/6$\,\,$ & $\,\,$ 1/4$\,\,$ & $\,\,$2$\,\,$ & $\,\,$ 1  $\,\,$ \\
\end{pmatrix},
\qquad
\lambda_{\max} =
4.1031,
\qquad
CR = 0.0389
\end{equation*}

\begin{equation*}
\mathbf{w}^{AMAST} =
\begin{pmatrix}
0.413123\\
\color{red} 0.410678\color{black} \\
0.073039\\
0.103160
\end{pmatrix}\end{equation*}
\begin{equation*}
\left[ \frac{{w}^{AMAST}_i}{{w}^{AMAST}_j} \right] =
\begin{pmatrix}
$\,\,$ 1 $\,\,$ & $\,\,$\color{red} 1.0060\color{black} $\,\,$ & $\,\,$5.6562$\,\,$ & $\,\,$4.0047$\,\,$ \\
$\,\,$\color{red} 0.9941\color{black} $\,\,$ & $\,\,$ 1 $\,\,$ & $\,\,$\color{red} 5.6227\color{black} $\,\,$ & $\,\,$\color{red} 3.9810\color{black}   $\,\,$ \\
$\,\,$0.1768$\,\,$ & $\,\,$\color{red} 0.1779\color{black} $\,\,$ & $\,\,$ 1 $\,\,$ & $\,\,$0.7080 $\,\,$ \\
$\,\,$0.2497$\,\,$ & $\,\,$\color{red} 0.2512\color{black} $\,\,$ & $\,\,$1.4124$\,\,$ & $\,\,$ 1  $\,\,$ \\
\end{pmatrix},
\end{equation*}

\begin{equation*}
\mathbf{w}^{\prime} =
\begin{pmatrix}
0.412313\\
0.411832\\
0.072896\\
0.102958
\end{pmatrix} =
0.998041\cdot
\begin{pmatrix}
0.413123\\
\color{gr} 0.412641\color{black} \\
0.073039\\
0.103160
\end{pmatrix},
\end{equation*}
\begin{equation*}
\left[ \frac{{w}^{\prime}_i}{{w}^{\prime}_j} \right] =
\begin{pmatrix}
$\,\,$ 1 $\,\,$ & $\,\,$\color{gr} 1.0012\color{black} $\,\,$ & $\,\,$5.6562$\,\,$ & $\,\,$4.0047$\,\,$ \\
$\,\,$\color{gr} 0.9988\color{black} $\,\,$ & $\,\,$ 1 $\,\,$ & $\,\,$\color{gr} 5.6496\color{black} $\,\,$ & $\,\,$\color{gr} \color{blue} 4\color{black}   $\,\,$ \\
$\,\,$0.1768$\,\,$ & $\,\,$\color{gr} 0.1770\color{black} $\,\,$ & $\,\,$ 1 $\,\,$ & $\,\,$0.7080 $\,\,$ \\
$\,\,$0.2497$\,\,$ & $\,\,$\color{gr} \color{blue}  1/4\color{black} $\,\,$ & $\,\,$1.4124$\,\,$ & $\,\,$ 1  $\,\,$ \\
\end{pmatrix},
\end{equation*}
\end{example}
\newpage
\begin{example}
\begin{equation*}
\mathbf{A} =
\begin{pmatrix}
$\,\,$ 1 $\,\,$ & $\,\,$1$\,\,$ & $\,\,$4$\,\,$ & $\,\,$7 $\,\,$ \\
$\,\,$ 1 $\,\,$ & $\,\,$ 1 $\,\,$ & $\,\,$6$\,\,$ & $\,\,$4 $\,\,$ \\
$\,\,$ 1/4$\,\,$ & $\,\,$ 1/6$\,\,$ & $\,\,$ 1 $\,\,$ & $\,\,$ 1/3 $\,\,$ \\
$\,\,$ 1/7$\,\,$ & $\,\,$ 1/4$\,\,$ & $\,\,$3$\,\,$ & $\,\,$ 1  $\,\,$ \\
\end{pmatrix},
\qquad
\lambda_{\max} =
4.2421,
\qquad
CR = 0.0913
\end{equation*}

\begin{equation*}
\mathbf{w}^{AMAST} =
\begin{pmatrix}
0.419764\\
\color{red} 0.400611\color{black} \\
0.067246\\
0.112379
\end{pmatrix}\end{equation*}
\begin{equation*}
\left[ \frac{{w}^{AMAST}_i}{{w}^{AMAST}_j} \right] =
\begin{pmatrix}
$\,\,$ 1 $\,\,$ & $\,\,$\color{red} 1.0478\color{black} $\,\,$ & $\,\,$6.2422$\,\,$ & $\,\,$3.7353$\,\,$ \\
$\,\,$\color{red} 0.9544\color{black} $\,\,$ & $\,\,$ 1 $\,\,$ & $\,\,$\color{red} 5.9574\color{black} $\,\,$ & $\,\,$\color{red} 3.5648\color{black}   $\,\,$ \\
$\,\,$0.1602$\,\,$ & $\,\,$\color{red} 0.1679\color{black} $\,\,$ & $\,\,$ 1 $\,\,$ & $\,\,$0.5984 $\,\,$ \\
$\,\,$0.2677$\,\,$ & $\,\,$\color{red} 0.2805\color{black} $\,\,$ & $\,\,$1.6712$\,\,$ & $\,\,$ 1  $\,\,$ \\
\end{pmatrix},
\end{equation*}

\begin{equation*}
\mathbf{w}^{\prime} =
\begin{pmatrix}
0.418565\\
0.402323\\
0.067054\\
0.112058
\end{pmatrix} =
0.997145\cdot
\begin{pmatrix}
0.419764\\
\color{gr} 0.403475\color{black} \\
0.067246\\
0.112379
\end{pmatrix},
\end{equation*}
\begin{equation*}
\left[ \frac{{w}^{\prime}_i}{{w}^{\prime}_j} \right] =
\begin{pmatrix}
$\,\,$ 1 $\,\,$ & $\,\,$\color{gr} 1.0404\color{black} $\,\,$ & $\,\,$6.2422$\,\,$ & $\,\,$3.7353$\,\,$ \\
$\,\,$\color{gr} 0.9612\color{black} $\,\,$ & $\,\,$ 1 $\,\,$ & $\,\,$\color{gr} \color{blue} 6\color{black} $\,\,$ & $\,\,$\color{gr} 3.5903\color{black}   $\,\,$ \\
$\,\,$0.1602$\,\,$ & $\,\,$\color{gr} \color{blue}  1/6\color{black} $\,\,$ & $\,\,$ 1 $\,\,$ & $\,\,$0.5984 $\,\,$ \\
$\,\,$0.2677$\,\,$ & $\,\,$\color{gr} 0.2785\color{black} $\,\,$ & $\,\,$1.6712$\,\,$ & $\,\,$ 1  $\,\,$ \\
\end{pmatrix},
\end{equation*}
\end{example}
\newpage
\begin{example}
\begin{equation*}
\mathbf{A} =
\begin{pmatrix}
$\,\,$ 1 $\,\,$ & $\,\,$1$\,\,$ & $\,\,$4$\,\,$ & $\,\,$7 $\,\,$ \\
$\,\,$ 1 $\,\,$ & $\,\,$ 1 $\,\,$ & $\,\,$7$\,\,$ & $\,\,$4 $\,\,$ \\
$\,\,$ 1/4$\,\,$ & $\,\,$ 1/7$\,\,$ & $\,\,$ 1 $\,\,$ & $\,\,$ 1/3 $\,\,$ \\
$\,\,$ 1/7$\,\,$ & $\,\,$ 1/4$\,\,$ & $\,\,$3$\,\,$ & $\,\,$ 1  $\,\,$ \\
\end{pmatrix},
\qquad
\lambda_{\max} =
4.2395,
\qquad
CR = 0.0903
\end{equation*}

\begin{equation*}
\mathbf{w}^{AMAST} =
\begin{pmatrix}
0.415383\\
\color{red} 0.410336\color{black} \\
0.063935\\
0.110346
\end{pmatrix}\end{equation*}
\begin{equation*}
\left[ \frac{{w}^{AMAST}_i}{{w}^{AMAST}_j} \right] =
\begin{pmatrix}
$\,\,$ 1 $\,\,$ & $\,\,$\color{red} 1.0123\color{black} $\,\,$ & $\,\,$6.4969$\,\,$ & $\,\,$3.7644$\,\,$ \\
$\,\,$\color{red} 0.9879\color{black} $\,\,$ & $\,\,$ 1 $\,\,$ & $\,\,$\color{red} 6.4180\color{black} $\,\,$ & $\,\,$\color{red} 3.7186\color{black}   $\,\,$ \\
$\,\,$0.1539$\,\,$ & $\,\,$\color{red} 0.1558\color{black} $\,\,$ & $\,\,$ 1 $\,\,$ & $\,\,$0.5794 $\,\,$ \\
$\,\,$0.2656$\,\,$ & $\,\,$\color{red} 0.2689\color{black} $\,\,$ & $\,\,$1.7259$\,\,$ & $\,\,$ 1  $\,\,$ \\
\end{pmatrix},
\end{equation*}

\begin{equation*}
\mathbf{w}^{\prime} =
\begin{pmatrix}
0.413297\\
0.413297\\
0.063614\\
0.109792
\end{pmatrix} =
0.994979\cdot
\begin{pmatrix}
0.415383\\
\color{gr} 0.415383\color{black} \\
0.063935\\
0.110346
\end{pmatrix},
\end{equation*}
\begin{equation*}
\left[ \frac{{w}^{\prime}_i}{{w}^{\prime}_j} \right] =
\begin{pmatrix}
$\,\,$ 1 $\,\,$ & $\,\,$\color{gr} \color{blue} 1\color{black} $\,\,$ & $\,\,$6.4969$\,\,$ & $\,\,$3.7644$\,\,$ \\
$\,\,$\color{gr} \color{blue} 1\color{black} $\,\,$ & $\,\,$ 1 $\,\,$ & $\,\,$\color{gr} 6.4969\color{black} $\,\,$ & $\,\,$\color{gr} 3.7644\color{black}   $\,\,$ \\
$\,\,$0.1539$\,\,$ & $\,\,$\color{gr} 0.1539\color{black} $\,\,$ & $\,\,$ 1 $\,\,$ & $\,\,$0.5794 $\,\,$ \\
$\,\,$0.2656$\,\,$ & $\,\,$\color{gr} 0.2656\color{black} $\,\,$ & $\,\,$1.7259$\,\,$ & $\,\,$ 1  $\,\,$ \\
\end{pmatrix},
\end{equation*}
\end{example}
\newpage
\begin{example}
\begin{equation*}
\mathbf{A} =
\begin{pmatrix}
$\,\,$ 1 $\,\,$ & $\,\,$1$\,\,$ & $\,\,$5$\,\,$ & $\,\,$3 $\,\,$ \\
$\,\,$ 1 $\,\,$ & $\,\,$ 1 $\,\,$ & $\,\,$7$\,\,$ & $\,\,$2 $\,\,$ \\
$\,\,$ 1/5$\,\,$ & $\,\,$ 1/7$\,\,$ & $\,\,$ 1 $\,\,$ & $\,\,$ 1/5 $\,\,$ \\
$\,\,$ 1/3$\,\,$ & $\,\,$ 1/2$\,\,$ & $\,\,$5$\,\,$ & $\,\,$ 1  $\,\,$ \\
\end{pmatrix},
\qquad
\lambda_{\max} =
4.1027,
\qquad
CR = 0.0387
\end{equation*}

\begin{equation*}
\mathbf{w}^{AMAST} =
\begin{pmatrix}
0.382646\\
\color{red} 0.373219\color{black} \\
0.054988\\
0.189148
\end{pmatrix}\end{equation*}
\begin{equation*}
\left[ \frac{{w}^{AMAST}_i}{{w}^{AMAST}_j} \right] =
\begin{pmatrix}
$\,\,$ 1 $\,\,$ & $\,\,$\color{red} 1.0253\color{black} $\,\,$ & $\,\,$6.9587$\,\,$ & $\,\,$2.0230$\,\,$ \\
$\,\,$\color{red} 0.9754\color{black} $\,\,$ & $\,\,$ 1 $\,\,$ & $\,\,$\color{red} 6.7873\color{black} $\,\,$ & $\,\,$\color{red} 1.9732\color{black}   $\,\,$ \\
$\,\,$0.1437$\,\,$ & $\,\,$\color{red} 0.1473\color{black} $\,\,$ & $\,\,$ 1 $\,\,$ & $\,\,$0.2907 $\,\,$ \\
$\,\,$0.4943$\,\,$ & $\,\,$\color{red} 0.5068\color{black} $\,\,$ & $\,\,$3.4398$\,\,$ & $\,\,$ 1  $\,\,$ \\
\end{pmatrix},
\end{equation*}

\begin{equation*}
\mathbf{w}^{\prime} =
\begin{pmatrix}
0.380713\\
0.376385\\
0.054710\\
0.188192
\end{pmatrix} =
0.994949\cdot
\begin{pmatrix}
0.382646\\
\color{gr} 0.378295\color{black} \\
0.054988\\
0.189148
\end{pmatrix},
\end{equation*}
\begin{equation*}
\left[ \frac{{w}^{\prime}_i}{{w}^{\prime}_j} \right] =
\begin{pmatrix}
$\,\,$ 1 $\,\,$ & $\,\,$\color{gr} 1.0115\color{black} $\,\,$ & $\,\,$6.9587$\,\,$ & $\,\,$2.0230$\,\,$ \\
$\,\,$\color{gr} 0.9886\color{black} $\,\,$ & $\,\,$ 1 $\,\,$ & $\,\,$\color{gr} 6.8796\color{black} $\,\,$ & $\,\,$\color{gr} \color{blue} 2\color{black}   $\,\,$ \\
$\,\,$0.1437$\,\,$ & $\,\,$\color{gr} 0.1454\color{black} $\,\,$ & $\,\,$ 1 $\,\,$ & $\,\,$0.2907 $\,\,$ \\
$\,\,$0.4943$\,\,$ & $\,\,$\color{gr} \color{blue}  1/2\color{black} $\,\,$ & $\,\,$3.4398$\,\,$ & $\,\,$ 1  $\,\,$ \\
\end{pmatrix},
\end{equation*}
\end{example}
\newpage
\begin{example}
\begin{equation*}
\mathbf{A} =
\begin{pmatrix}
$\,\,$ 1 $\,\,$ & $\,\,$1$\,\,$ & $\,\,$5$\,\,$ & $\,\,$3 $\,\,$ \\
$\,\,$ 1 $\,\,$ & $\,\,$ 1 $\,\,$ & $\,\,$8$\,\,$ & $\,\,$2 $\,\,$ \\
$\,\,$ 1/5$\,\,$ & $\,\,$ 1/8$\,\,$ & $\,\,$ 1 $\,\,$ & $\,\,$ 1/8 $\,\,$ \\
$\,\,$ 1/3$\,\,$ & $\,\,$ 1/2$\,\,$ & $\,\,$8$\,\,$ & $\,\,$ 1  $\,\,$ \\
\end{pmatrix},
\qquad
\lambda_{\max} =
4.2162,
\qquad
CR = 0.0815
\end{equation*}

\begin{equation*}
\mathbf{w}^{AMAST} =
\begin{pmatrix}
0.371742\\
\color{red} 0.371502\color{black} \\
0.047496\\
0.209261
\end{pmatrix}\end{equation*}
\begin{equation*}
\left[ \frac{{w}^{AMAST}_i}{{w}^{AMAST}_j} \right] =
\begin{pmatrix}
$\,\,$ 1 $\,\,$ & $\,\,$\color{red} 1.0006\color{black} $\,\,$ & $\,\,$7.8268$\,\,$ & $\,\,$1.7765$\,\,$ \\
$\,\,$\color{red} 0.9994\color{black} $\,\,$ & $\,\,$ 1 $\,\,$ & $\,\,$\color{red} 7.8218\color{black} $\,\,$ & $\,\,$\color{red} 1.7753\color{black}   $\,\,$ \\
$\,\,$0.1278$\,\,$ & $\,\,$\color{red} 0.1278\color{black} $\,\,$ & $\,\,$ 1 $\,\,$ & $\,\,$0.2270 $\,\,$ \\
$\,\,$0.5629$\,\,$ & $\,\,$\color{red} 0.5633\color{black} $\,\,$ & $\,\,$4.4059$\,\,$ & $\,\,$ 1  $\,\,$ \\
\end{pmatrix},
\end{equation*}

\begin{equation*}
\mathbf{w}^{\prime} =
\begin{pmatrix}
0.371653\\
0.371653\\
0.047485\\
0.209210
\end{pmatrix} =
0.999760\cdot
\begin{pmatrix}
0.371742\\
\color{gr} 0.371742\color{black} \\
0.047496\\
0.209261
\end{pmatrix},
\end{equation*}
\begin{equation*}
\left[ \frac{{w}^{\prime}_i}{{w}^{\prime}_j} \right] =
\begin{pmatrix}
$\,\,$ 1 $\,\,$ & $\,\,$\color{gr} \color{blue} 1\color{black} $\,\,$ & $\,\,$7.8268$\,\,$ & $\,\,$1.7765$\,\,$ \\
$\,\,$\color{gr} \color{blue} 1\color{black} $\,\,$ & $\,\,$ 1 $\,\,$ & $\,\,$\color{gr} 7.8268\color{black} $\,\,$ & $\,\,$\color{gr} 1.7765\color{black}   $\,\,$ \\
$\,\,$0.1278$\,\,$ & $\,\,$\color{gr} 0.1278\color{black} $\,\,$ & $\,\,$ 1 $\,\,$ & $\,\,$0.2270 $\,\,$ \\
$\,\,$0.5629$\,\,$ & $\,\,$\color{gr} 0.5629\color{black} $\,\,$ & $\,\,$4.4059$\,\,$ & $\,\,$ 1  $\,\,$ \\
\end{pmatrix},
\end{equation*}
\end{example}
\newpage
\begin{example}
\begin{equation*}
\mathbf{A} =
\begin{pmatrix}
$\,\,$ 1 $\,\,$ & $\,\,$1$\,\,$ & $\,\,$5$\,\,$ & $\,\,$3 $\,\,$ \\
$\,\,$ 1 $\,\,$ & $\,\,$ 1 $\,\,$ & $\,\,$8$\,\,$ & $\,\,$2 $\,\,$ \\
$\,\,$ 1/5$\,\,$ & $\,\,$ 1/8$\,\,$ & $\,\,$ 1 $\,\,$ & $\,\,$ 1/9 $\,\,$ \\
$\,\,$ 1/3$\,\,$ & $\,\,$ 1/2$\,\,$ & $\,\,$9$\,\,$ & $\,\,$ 1  $\,\,$ \\
\end{pmatrix},
\qquad
\lambda_{\max} =
4.2541,
\qquad
CR = 0.0958
\end{equation*}

\begin{equation*}
\mathbf{w}^{AMAST} =
\begin{pmatrix}
0.369687\\
\color{red} 0.368760\color{black} \\
0.046362\\
0.215191
\end{pmatrix}\end{equation*}
\begin{equation*}
\left[ \frac{{w}^{AMAST}_i}{{w}^{AMAST}_j} \right] =
\begin{pmatrix}
$\,\,$ 1 $\,\,$ & $\,\,$\color{red} 1.0025\color{black} $\,\,$ & $\,\,$7.9739$\,\,$ & $\,\,$1.7180$\,\,$ \\
$\,\,$\color{red} 0.9975\color{black} $\,\,$ & $\,\,$ 1 $\,\,$ & $\,\,$\color{red} 7.9539\color{black} $\,\,$ & $\,\,$\color{red} 1.7136\color{black}   $\,\,$ \\
$\,\,$0.1254$\,\,$ & $\,\,$\color{red} 0.1257\color{black} $\,\,$ & $\,\,$ 1 $\,\,$ & $\,\,$0.2154 $\,\,$ \\
$\,\,$0.5821$\,\,$ & $\,\,$\color{red} 0.5836\color{black} $\,\,$ & $\,\,$4.6415$\,\,$ & $\,\,$ 1  $\,\,$ \\
\end{pmatrix},
\end{equation*}

\begin{equation*}
\mathbf{w}^{\prime} =
\begin{pmatrix}
0.369345\\
0.369345\\
0.046319\\
0.214991
\end{pmatrix} =
0.999073\cdot
\begin{pmatrix}
0.369687\\
\color{gr} 0.369687\color{black} \\
0.046362\\
0.215191
\end{pmatrix},
\end{equation*}
\begin{equation*}
\left[ \frac{{w}^{\prime}_i}{{w}^{\prime}_j} \right] =
\begin{pmatrix}
$\,\,$ 1 $\,\,$ & $\,\,$\color{gr} \color{blue} 1\color{black} $\,\,$ & $\,\,$7.9739$\,\,$ & $\,\,$1.7180$\,\,$ \\
$\,\,$\color{gr} \color{blue} 1\color{black} $\,\,$ & $\,\,$ 1 $\,\,$ & $\,\,$\color{gr} 7.9739\color{black} $\,\,$ & $\,\,$\color{gr} 1.7180\color{black}   $\,\,$ \\
$\,\,$0.1254$\,\,$ & $\,\,$\color{gr} 0.1254\color{black} $\,\,$ & $\,\,$ 1 $\,\,$ & $\,\,$0.2154 $\,\,$ \\
$\,\,$0.5821$\,\,$ & $\,\,$\color{gr} 0.5821\color{black} $\,\,$ & $\,\,$4.6415$\,\,$ & $\,\,$ 1  $\,\,$ \\
\end{pmatrix},
\end{equation*}
\end{example}
\newpage
\begin{example}
\begin{equation*}
\mathbf{A} =
\begin{pmatrix}
$\,\,$ 1 $\,\,$ & $\,\,$1$\,\,$ & $\,\,$5$\,\,$ & $\,\,$4 $\,\,$ \\
$\,\,$ 1 $\,\,$ & $\,\,$ 1 $\,\,$ & $\,\,$7$\,\,$ & $\,\,$2 $\,\,$ \\
$\,\,$ 1/5$\,\,$ & $\,\,$ 1/7$\,\,$ & $\,\,$ 1 $\,\,$ & $\,\,$ 1/6 $\,\,$ \\
$\,\,$ 1/4$\,\,$ & $\,\,$ 1/2$\,\,$ & $\,\,$6$\,\,$ & $\,\,$ 1  $\,\,$ \\
\end{pmatrix},
\qquad
\lambda_{\max} =
4.2174,
\qquad
CR = 0.0820
\end{equation*}

\begin{equation*}
\mathbf{w}^{AMAST} =
\begin{pmatrix}
0.400168\\
\color{red} 0.362739\color{black} \\
0.052673\\
0.184419
\end{pmatrix}\end{equation*}
\begin{equation*}
\left[ \frac{{w}^{AMAST}_i}{{w}^{AMAST}_j} \right] =
\begin{pmatrix}
$\,\,$ 1 $\,\,$ & $\,\,$\color{red} 1.1032\color{black} $\,\,$ & $\,\,$7.5971$\,\,$ & $\,\,$2.1699$\,\,$ \\
$\,\,$\color{red} 0.9065\color{black} $\,\,$ & $\,\,$ 1 $\,\,$ & $\,\,$\color{red} 6.8866\color{black} $\,\,$ & $\,\,$\color{red} 1.9669\color{black}   $\,\,$ \\
$\,\,$0.1316$\,\,$ & $\,\,$\color{red} 0.1452\color{black} $\,\,$ & $\,\,$ 1 $\,\,$ & $\,\,$0.2856 $\,\,$ \\
$\,\,$0.4609$\,\,$ & $\,\,$\color{red} 0.5084\color{black} $\,\,$ & $\,\,$3.5012$\,\,$ & $\,\,$ 1  $\,\,$ \\
\end{pmatrix},
\end{equation*}

\begin{equation*}
\mathbf{w}^{\prime} =
\begin{pmatrix}
0.397791\\
0.366524\\
0.052361\\
0.183324
\end{pmatrix} =
0.994060\cdot
\begin{pmatrix}
0.400168\\
\color{gr} 0.368714\color{black} \\
0.052673\\
0.184419
\end{pmatrix},
\end{equation*}
\begin{equation*}
\left[ \frac{{w}^{\prime}_i}{{w}^{\prime}_j} \right] =
\begin{pmatrix}
$\,\,$ 1 $\,\,$ & $\,\,$\color{gr} 1.0853\color{black} $\,\,$ & $\,\,$7.5971$\,\,$ & $\,\,$2.1699$\,\,$ \\
$\,\,$\color{gr} 0.9214\color{black} $\,\,$ & $\,\,$ 1 $\,\,$ & $\,\,$\color{gr} \color{blue} 7\color{black} $\,\,$ & $\,\,$\color{gr} 1.9993\color{black}   $\,\,$ \\
$\,\,$0.1316$\,\,$ & $\,\,$\color{gr} \color{blue}  1/7\color{black} $\,\,$ & $\,\,$ 1 $\,\,$ & $\,\,$0.2856 $\,\,$ \\
$\,\,$0.4609$\,\,$ & $\,\,$\color{gr} 0.5002\color{black} $\,\,$ & $\,\,$3.5012$\,\,$ & $\,\,$ 1  $\,\,$ \\
\end{pmatrix},
\end{equation*}
\end{example}
\newpage
\begin{example}
\begin{equation*}
\mathbf{A} =
\begin{pmatrix}
$\,\,$ 1 $\,\,$ & $\,\,$1$\,\,$ & $\,\,$5$\,\,$ & $\,\,$4 $\,\,$ \\
$\,\,$ 1 $\,\,$ & $\,\,$ 1 $\,\,$ & $\,\,$8$\,\,$ & $\,\,$2 $\,\,$ \\
$\,\,$ 1/5$\,\,$ & $\,\,$ 1/8$\,\,$ & $\,\,$ 1 $\,\,$ & $\,\,$ 1/7 $\,\,$ \\
$\,\,$ 1/4$\,\,$ & $\,\,$ 1/2$\,\,$ & $\,\,$7$\,\,$ & $\,\,$ 1  $\,\,$ \\
\end{pmatrix},
\qquad
\lambda_{\max} =
4.2610,
\qquad
CR = 0.0984
\end{equation*}

\begin{equation*}
\mathbf{w}^{AMAST} =
\begin{pmatrix}
0.394242\\
\color{red} 0.367540\color{black} \\
0.048884\\
0.189334
\end{pmatrix}\end{equation*}
\begin{equation*}
\left[ \frac{{w}^{AMAST}_i}{{w}^{AMAST}_j} \right] =
\begin{pmatrix}
$\,\,$ 1 $\,\,$ & $\,\,$\color{red} 1.0727\color{black} $\,\,$ & $\,\,$8.0649$\,\,$ & $\,\,$2.0823$\,\,$ \\
$\,\,$\color{red} 0.9323\color{black} $\,\,$ & $\,\,$ 1 $\,\,$ & $\,\,$\color{red} 7.5186\color{black} $\,\,$ & $\,\,$\color{red} 1.9412\color{black}   $\,\,$ \\
$\,\,$0.1240$\,\,$ & $\,\,$\color{red} 0.1330\color{black} $\,\,$ & $\,\,$ 1 $\,\,$ & $\,\,$0.2582 $\,\,$ \\
$\,\,$0.4802$\,\,$ & $\,\,$\color{red} 0.5151\color{black} $\,\,$ & $\,\,$3.8731$\,\,$ & $\,\,$ 1  $\,\,$ \\
\end{pmatrix},
\end{equation*}

\begin{equation*}
\mathbf{w}^{\prime} =
\begin{pmatrix}
0.389903\\
0.374501\\
0.048346\\
0.187250
\end{pmatrix} =
0.988993\cdot
\begin{pmatrix}
0.394242\\
\color{gr} 0.378669\color{black} \\
0.048884\\
0.189334
\end{pmatrix},
\end{equation*}
\begin{equation*}
\left[ \frac{{w}^{\prime}_i}{{w}^{\prime}_j} \right] =
\begin{pmatrix}
$\,\,$ 1 $\,\,$ & $\,\,$\color{gr} 1.0411\color{black} $\,\,$ & $\,\,$8.0649$\,\,$ & $\,\,$2.0823$\,\,$ \\
$\,\,$\color{gr} 0.9605\color{black} $\,\,$ & $\,\,$ 1 $\,\,$ & $\,\,$\color{gr} 7.7463\color{black} $\,\,$ & $\,\,$\color{gr} \color{blue} 2\color{black}   $\,\,$ \\
$\,\,$0.1240$\,\,$ & $\,\,$\color{gr} 0.1291\color{black} $\,\,$ & $\,\,$ 1 $\,\,$ & $\,\,$0.2582 $\,\,$ \\
$\,\,$0.4802$\,\,$ & $\,\,$\color{gr} \color{blue}  1/2\color{black} $\,\,$ & $\,\,$3.8731$\,\,$ & $\,\,$ 1  $\,\,$ \\
\end{pmatrix},
\end{equation*}
\end{example}
\newpage
\begin{example}
\begin{equation*}
\mathbf{A} =
\begin{pmatrix}
$\,\,$ 1 $\,\,$ & $\,\,$1$\,\,$ & $\,\,$5$\,\,$ & $\,\,$5 $\,\,$ \\
$\,\,$ 1 $\,\,$ & $\,\,$ 1 $\,\,$ & $\,\,$7$\,\,$ & $\,\,$3 $\,\,$ \\
$\,\,$ 1/5$\,\,$ & $\,\,$ 1/7$\,\,$ & $\,\,$ 1 $\,\,$ & $\,\,$ 1/4 $\,\,$ \\
$\,\,$ 1/5$\,\,$ & $\,\,$ 1/3$\,\,$ & $\,\,$4$\,\,$ & $\,\,$ 1  $\,\,$ \\
\end{pmatrix},
\qquad
\lambda_{\max} =
4.1667,
\qquad
CR = 0.0629
\end{equation*}

\begin{equation*}
\mathbf{w}^{AMAST} =
\begin{pmatrix}
0.412658\\
\color{red} 0.392317\color{black} \\
0.056350\\
0.138674
\end{pmatrix}\end{equation*}
\begin{equation*}
\left[ \frac{{w}^{AMAST}_i}{{w}^{AMAST}_j} \right] =
\begin{pmatrix}
$\,\,$ 1 $\,\,$ & $\,\,$\color{red} 1.0519\color{black} $\,\,$ & $\,\,$7.3231$\,\,$ & $\,\,$2.9757$\,\,$ \\
$\,\,$\color{red} 0.9507\color{black} $\,\,$ & $\,\,$ 1 $\,\,$ & $\,\,$\color{red} 6.9621\color{black} $\,\,$ & $\,\,$\color{red} 2.8291\color{black}   $\,\,$ \\
$\,\,$0.1366$\,\,$ & $\,\,$\color{red} 0.1436\color{black} $\,\,$ & $\,\,$ 1 $\,\,$ & $\,\,$0.4064 $\,\,$ \\
$\,\,$0.3361$\,\,$ & $\,\,$\color{red} 0.3535\color{black} $\,\,$ & $\,\,$2.4609$\,\,$ & $\,\,$ 1  $\,\,$ \\
\end{pmatrix},
\end{equation*}

\begin{equation*}
\mathbf{w}^{\prime} =
\begin{pmatrix}
0.411779\\
0.393612\\
0.056230\\
0.138379
\end{pmatrix} =
0.997868\cdot
\begin{pmatrix}
0.412658\\
\color{gr} 0.394453\color{black} \\
0.056350\\
0.138674
\end{pmatrix},
\end{equation*}
\begin{equation*}
\left[ \frac{{w}^{\prime}_i}{{w}^{\prime}_j} \right] =
\begin{pmatrix}
$\,\,$ 1 $\,\,$ & $\,\,$\color{gr} 1.0462\color{black} $\,\,$ & $\,\,$7.3231$\,\,$ & $\,\,$2.9757$\,\,$ \\
$\,\,$\color{gr} 0.9559\color{black} $\,\,$ & $\,\,$ 1 $\,\,$ & $\,\,$\color{gr} \color{blue} 7\color{black} $\,\,$ & $\,\,$\color{gr} 2.8445\color{black}   $\,\,$ \\
$\,\,$0.1366$\,\,$ & $\,\,$\color{gr} \color{blue}  1/7\color{black} $\,\,$ & $\,\,$ 1 $\,\,$ & $\,\,$0.4064 $\,\,$ \\
$\,\,$0.3361$\,\,$ & $\,\,$\color{gr} 0.3516\color{black} $\,\,$ & $\,\,$2.4609$\,\,$ & $\,\,$ 1  $\,\,$ \\
\end{pmatrix},
\end{equation*}
\end{example}
\newpage
\begin{example}
\begin{equation*}
\mathbf{A} =
\begin{pmatrix}
$\,\,$ 1 $\,\,$ & $\,\,$1$\,\,$ & $\,\,$5$\,\,$ & $\,\,$5 $\,\,$ \\
$\,\,$ 1 $\,\,$ & $\,\,$ 1 $\,\,$ & $\,\,$8$\,\,$ & $\,\,$3 $\,\,$ \\
$\,\,$ 1/5$\,\,$ & $\,\,$ 1/8$\,\,$ & $\,\,$ 1 $\,\,$ & $\,\,$ 1/4 $\,\,$ \\
$\,\,$ 1/5$\,\,$ & $\,\,$ 1/3$\,\,$ & $\,\,$4$\,\,$ & $\,\,$ 1  $\,\,$ \\
\end{pmatrix},
\qquad
\lambda_{\max} =
4.1655,
\qquad
CR = 0.0624
\end{equation*}

\begin{equation*}
\mathbf{w}^{AMAST} =
\begin{pmatrix}
0.408696\\
\color{red} 0.400675\color{black} \\
0.053931\\
0.136699
\end{pmatrix}\end{equation*}
\begin{equation*}
\left[ \frac{{w}^{AMAST}_i}{{w}^{AMAST}_j} \right] =
\begin{pmatrix}
$\,\,$ 1 $\,\,$ & $\,\,$\color{red} 1.0200\color{black} $\,\,$ & $\,\,$7.5782$\,\,$ & $\,\,$2.9897$\,\,$ \\
$\,\,$\color{red} 0.9804\color{black} $\,\,$ & $\,\,$ 1 $\,\,$ & $\,\,$\color{red} 7.4295\color{black} $\,\,$ & $\,\,$\color{red} 2.9311\color{black}   $\,\,$ \\
$\,\,$0.1320$\,\,$ & $\,\,$\color{red} 0.1346\color{black} $\,\,$ & $\,\,$ 1 $\,\,$ & $\,\,$0.3945 $\,\,$ \\
$\,\,$0.3345$\,\,$ & $\,\,$\color{red} 0.3412\color{black} $\,\,$ & $\,\,$2.5347$\,\,$ & $\,\,$ 1  $\,\,$ \\
\end{pmatrix},
\end{equation*}

\begin{equation*}
\mathbf{w}^{\prime} =
\begin{pmatrix}
0.405444\\
0.405444\\
0.053501\\
0.135611
\end{pmatrix} =
0.992043\cdot
\begin{pmatrix}
0.408696\\
\color{gr} 0.408696\color{black} \\
0.053931\\
0.136699
\end{pmatrix},
\end{equation*}
\begin{equation*}
\left[ \frac{{w}^{\prime}_i}{{w}^{\prime}_j} \right] =
\begin{pmatrix}
$\,\,$ 1 $\,\,$ & $\,\,$\color{gr} \color{blue} 1\color{black} $\,\,$ & $\,\,$7.5782$\,\,$ & $\,\,$2.9897$\,\,$ \\
$\,\,$\color{gr} \color{blue} 1\color{black} $\,\,$ & $\,\,$ 1 $\,\,$ & $\,\,$\color{gr} 7.5782\color{black} $\,\,$ & $\,\,$\color{gr} 2.9897\color{black}   $\,\,$ \\
$\,\,$0.1320$\,\,$ & $\,\,$\color{gr} 0.1320\color{black} $\,\,$ & $\,\,$ 1 $\,\,$ & $\,\,$0.3945 $\,\,$ \\
$\,\,$0.3345$\,\,$ & $\,\,$\color{gr} 0.3345\color{black} $\,\,$ & $\,\,$2.5347$\,\,$ & $\,\,$ 1  $\,\,$ \\
\end{pmatrix},
\end{equation*}
\end{example}
\newpage
\begin{example}
\begin{equation*}
\mathbf{A} =
\begin{pmatrix}
$\,\,$ 1 $\,\,$ & $\,\,$1$\,\,$ & $\,\,$5$\,\,$ & $\,\,$5 $\,\,$ \\
$\,\,$ 1 $\,\,$ & $\,\,$ 1 $\,\,$ & $\,\,$8$\,\,$ & $\,\,$3 $\,\,$ \\
$\,\,$ 1/5$\,\,$ & $\,\,$ 1/8$\,\,$ & $\,\,$ 1 $\,\,$ & $\,\,$ 1/5 $\,\,$ \\
$\,\,$ 1/5$\,\,$ & $\,\,$ 1/3$\,\,$ & $\,\,$5$\,\,$ & $\,\,$ 1  $\,\,$ \\
\end{pmatrix},
\qquad
\lambda_{\max} =
4.2259,
\qquad
CR = 0.0852
\end{equation*}

\begin{equation*}
\mathbf{w}^{AMAST} =
\begin{pmatrix}
0.405885\\
\color{red} 0.396878\color{black} \\
0.051601\\
0.145636
\end{pmatrix}\end{equation*}
\begin{equation*}
\left[ \frac{{w}^{AMAST}_i}{{w}^{AMAST}_j} \right] =
\begin{pmatrix}
$\,\,$ 1 $\,\,$ & $\,\,$\color{red} 1.0227\color{black} $\,\,$ & $\,\,$7.8658$\,\,$ & $\,\,$2.7870$\,\,$ \\
$\,\,$\color{red} 0.9778\color{black} $\,\,$ & $\,\,$ 1 $\,\,$ & $\,\,$\color{red} 7.6912\color{black} $\,\,$ & $\,\,$\color{red} 2.7251\color{black}   $\,\,$ \\
$\,\,$0.1271$\,\,$ & $\,\,$\color{red} 0.1300\color{black} $\,\,$ & $\,\,$ 1 $\,\,$ & $\,\,$0.3543 $\,\,$ \\
$\,\,$0.3588$\,\,$ & $\,\,$\color{red} 0.3670\color{black} $\,\,$ & $\,\,$2.8223$\,\,$ & $\,\,$ 1  $\,\,$ \\
\end{pmatrix},
\end{equation*}

\begin{equation*}
\mathbf{w}^{\prime} =
\begin{pmatrix}
0.402262\\
0.402262\\
0.051141\\
0.144336
\end{pmatrix} =
0.991073\cdot
\begin{pmatrix}
0.405885\\
\color{gr} 0.405885\color{black} \\
0.051601\\
0.145636
\end{pmatrix},
\end{equation*}
\begin{equation*}
\left[ \frac{{w}^{\prime}_i}{{w}^{\prime}_j} \right] =
\begin{pmatrix}
$\,\,$ 1 $\,\,$ & $\,\,$\color{gr} \color{blue} 1\color{black} $\,\,$ & $\,\,$7.8658$\,\,$ & $\,\,$2.7870$\,\,$ \\
$\,\,$\color{gr} \color{blue} 1\color{black} $\,\,$ & $\,\,$ 1 $\,\,$ & $\,\,$\color{gr} 7.8658\color{black} $\,\,$ & $\,\,$\color{gr} 2.7870\color{black}   $\,\,$ \\
$\,\,$0.1271$\,\,$ & $\,\,$\color{gr} 0.1271\color{black} $\,\,$ & $\,\,$ 1 $\,\,$ & $\,\,$0.3543 $\,\,$ \\
$\,\,$0.3588$\,\,$ & $\,\,$\color{gr} 0.3588\color{black} $\,\,$ & $\,\,$2.8223$\,\,$ & $\,\,$ 1  $\,\,$ \\
\end{pmatrix},
\end{equation*}
\end{example}
\newpage
\begin{example}
\begin{equation*}
\mathbf{A} =
\begin{pmatrix}
$\,\,$ 1 $\,\,$ & $\,\,$1$\,\,$ & $\,\,$5$\,\,$ & $\,\,$6 $\,\,$ \\
$\,\,$ 1 $\,\,$ & $\,\,$ 1 $\,\,$ & $\,\,$7$\,\,$ & $\,\,$3 $\,\,$ \\
$\,\,$ 1/5$\,\,$ & $\,\,$ 1/7$\,\,$ & $\,\,$ 1 $\,\,$ & $\,\,$ 1/4 $\,\,$ \\
$\,\,$ 1/6$\,\,$ & $\,\,$ 1/3$\,\,$ & $\,\,$4$\,\,$ & $\,\,$ 1  $\,\,$ \\
\end{pmatrix},
\qquad
\lambda_{\max} =
4.2174,
\qquad
CR = 0.0820
\end{equation*}

\begin{equation*}
\mathbf{w}^{AMAST} =
\begin{pmatrix}
0.424656\\
\color{red} 0.386557\color{black} \\
0.056115\\
0.132672
\end{pmatrix}\end{equation*}
\begin{equation*}
\left[ \frac{{w}^{AMAST}_i}{{w}^{AMAST}_j} \right] =
\begin{pmatrix}
$\,\,$ 1 $\,\,$ & $\,\,$\color{red} 1.0986\color{black} $\,\,$ & $\,\,$7.5676$\,\,$ & $\,\,$3.2008$\,\,$ \\
$\,\,$\color{red} 0.9103\color{black} $\,\,$ & $\,\,$ 1 $\,\,$ & $\,\,$\color{red} 6.8886\color{black} $\,\,$ & $\,\,$\color{red} 2.9136\color{black}   $\,\,$ \\
$\,\,$0.1321$\,\,$ & $\,\,$\color{red} 0.1452\color{black} $\,\,$ & $\,\,$ 1 $\,\,$ & $\,\,$0.4230 $\,\,$ \\
$\,\,$0.3124$\,\,$ & $\,\,$\color{red} 0.3432\color{black} $\,\,$ & $\,\,$2.3643$\,\,$ & $\,\,$ 1  $\,\,$ \\
\end{pmatrix},
\end{equation*}

\begin{equation*}
\mathbf{w}^{\prime} =
\begin{pmatrix}
0.422019\\
0.390367\\
0.055767\\
0.131848
\end{pmatrix} =
0.993789\cdot
\begin{pmatrix}
0.424656\\
\color{gr} 0.392806\color{black} \\
0.056115\\
0.132672
\end{pmatrix},
\end{equation*}
\begin{equation*}
\left[ \frac{{w}^{\prime}_i}{{w}^{\prime}_j} \right] =
\begin{pmatrix}
$\,\,$ 1 $\,\,$ & $\,\,$\color{gr} 1.0811\color{black} $\,\,$ & $\,\,$7.5676$\,\,$ & $\,\,$3.2008$\,\,$ \\
$\,\,$\color{gr} 0.9250\color{black} $\,\,$ & $\,\,$ 1 $\,\,$ & $\,\,$\color{gr} \color{blue} 7\color{black} $\,\,$ & $\,\,$\color{gr} 2.9607\color{black}   $\,\,$ \\
$\,\,$0.1321$\,\,$ & $\,\,$\color{gr} \color{blue}  1/7\color{black} $\,\,$ & $\,\,$ 1 $\,\,$ & $\,\,$0.4230 $\,\,$ \\
$\,\,$0.3124$\,\,$ & $\,\,$\color{gr} 0.3378\color{black} $\,\,$ & $\,\,$2.3643$\,\,$ & $\,\,$ 1  $\,\,$ \\
\end{pmatrix},
\end{equation*}
\end{example}
\newpage
\begin{example}
\begin{equation*}
\mathbf{A} =
\begin{pmatrix}
$\,\,$ 1 $\,\,$ & $\,\,$1$\,\,$ & $\,\,$5$\,\,$ & $\,\,$7 $\,\,$ \\
$\,\,$ 1 $\,\,$ & $\,\,$ 1 $\,\,$ & $\,\,$7$\,\,$ & $\,\,$4 $\,\,$ \\
$\,\,$ 1/5$\,\,$ & $\,\,$ 1/7$\,\,$ & $\,\,$ 1 $\,\,$ & $\,\,$ 1/3 $\,\,$ \\
$\,\,$ 1/7$\,\,$ & $\,\,$ 1/4$\,\,$ & $\,\,$3$\,\,$ & $\,\,$ 1  $\,\,$ \\
\end{pmatrix},
\qquad
\lambda_{\max} =
4.1793,
\qquad
CR = 0.0676
\end{equation*}

\begin{equation*}
\mathbf{w}^{AMAST} =
\begin{pmatrix}
0.429920\\
\color{red} 0.404701\color{black} \\
0.058292\\
0.107087
\end{pmatrix}\end{equation*}
\begin{equation*}
\left[ \frac{{w}^{AMAST}_i}{{w}^{AMAST}_j} \right] =
\begin{pmatrix}
$\,\,$ 1 $\,\,$ & $\,\,$\color{red} 1.0623\color{black} $\,\,$ & $\,\,$7.3752$\,\,$ & $\,\,$4.0147$\,\,$ \\
$\,\,$\color{red} 0.9413\color{black} $\,\,$ & $\,\,$ 1 $\,\,$ & $\,\,$\color{red} 6.9426\color{black} $\,\,$ & $\,\,$\color{red} 3.7792\color{black}   $\,\,$ \\
$\,\,$0.1356$\,\,$ & $\,\,$\color{red} 0.1440\color{black} $\,\,$ & $\,\,$ 1 $\,\,$ & $\,\,$0.5443 $\,\,$ \\
$\,\,$0.2491$\,\,$ & $\,\,$\color{red} 0.2646\color{black} $\,\,$ & $\,\,$1.8371$\,\,$ & $\,\,$ 1  $\,\,$ \\
\end{pmatrix},
\end{equation*}

\begin{equation*}
\mathbf{w}^{\prime} =
\begin{pmatrix}
0.428486\\
0.406686\\
0.058098\\
0.106730
\end{pmatrix} =
0.996665\cdot
\begin{pmatrix}
0.429920\\
\color{gr} 0.408047\color{black} \\
0.058292\\
0.107087
\end{pmatrix},
\end{equation*}
\begin{equation*}
\left[ \frac{{w}^{\prime}_i}{{w}^{\prime}_j} \right] =
\begin{pmatrix}
$\,\,$ 1 $\,\,$ & $\,\,$\color{gr} 1.0536\color{black} $\,\,$ & $\,\,$7.3752$\,\,$ & $\,\,$4.0147$\,\,$ \\
$\,\,$\color{gr} 0.9491\color{black} $\,\,$ & $\,\,$ 1 $\,\,$ & $\,\,$\color{gr} \color{blue} 7\color{black} $\,\,$ & $\,\,$\color{gr} 3.8104\color{black}   $\,\,$ \\
$\,\,$0.1356$\,\,$ & $\,\,$\color{gr} \color{blue}  1/7\color{black} $\,\,$ & $\,\,$ 1 $\,\,$ & $\,\,$0.5443 $\,\,$ \\
$\,\,$0.2491$\,\,$ & $\,\,$\color{gr} 0.2624\color{black} $\,\,$ & $\,\,$1.8371$\,\,$ & $\,\,$ 1  $\,\,$ \\
\end{pmatrix},
\end{equation*}
\end{example}
\newpage
\begin{example}
\begin{equation*}
\mathbf{A} =
\begin{pmatrix}
$\,\,$ 1 $\,\,$ & $\,\,$1$\,\,$ & $\,\,$5$\,\,$ & $\,\,$7 $\,\,$ \\
$\,\,$ 1 $\,\,$ & $\,\,$ 1 $\,\,$ & $\,\,$7$\,\,$ & $\,\,$5 $\,\,$ \\
$\,\,$ 1/5$\,\,$ & $\,\,$ 1/7$\,\,$ & $\,\,$ 1 $\,\,$ & $\,\,$ 1/2 $\,\,$ \\
$\,\,$ 1/7$\,\,$ & $\,\,$ 1/5$\,\,$ & $\,\,$2$\,\,$ & $\,\,$ 1  $\,\,$ \\
\end{pmatrix},
\qquad
\lambda_{\max} =
4.0899,
\qquad
CR = 0.0339
\end{equation*}

\begin{equation*}
\mathbf{w}^{AMAST} =
\begin{pmatrix}
0.425744\\
\color{red} 0.423950\color{black} \\
0.062266\\
0.088041
\end{pmatrix}\end{equation*}
\begin{equation*}
\left[ \frac{{w}^{AMAST}_i}{{w}^{AMAST}_j} \right] =
\begin{pmatrix}
$\,\,$ 1 $\,\,$ & $\,\,$\color{red} 1.0042\color{black} $\,\,$ & $\,\,$6.8375$\,\,$ & $\,\,$4.8357$\,\,$ \\
$\,\,$\color{red} 0.9958\color{black} $\,\,$ & $\,\,$ 1 $\,\,$ & $\,\,$\color{red} 6.8087\color{black} $\,\,$ & $\,\,$\color{red} 4.8154\color{black}   $\,\,$ \\
$\,\,$0.1463$\,\,$ & $\,\,$\color{red} 0.1469\color{black} $\,\,$ & $\,\,$ 1 $\,\,$ & $\,\,$0.7072 $\,\,$ \\
$\,\,$0.2068$\,\,$ & $\,\,$\color{red} 0.2077\color{black} $\,\,$ & $\,\,$1.4140$\,\,$ & $\,\,$ 1  $\,\,$ \\
\end{pmatrix},
\end{equation*}

\begin{equation*}
\mathbf{w}^{\prime} =
\begin{pmatrix}
0.424981\\
0.424981\\
0.062154\\
0.087884
\end{pmatrix} =
0.998209\cdot
\begin{pmatrix}
0.425744\\
\color{gr} 0.425744\color{black} \\
0.062266\\
0.088041
\end{pmatrix},
\end{equation*}
\begin{equation*}
\left[ \frac{{w}^{\prime}_i}{{w}^{\prime}_j} \right] =
\begin{pmatrix}
$\,\,$ 1 $\,\,$ & $\,\,$\color{gr} \color{blue} 1\color{black} $\,\,$ & $\,\,$6.8375$\,\,$ & $\,\,$4.8357$\,\,$ \\
$\,\,$\color{gr} \color{blue} 1\color{black} $\,\,$ & $\,\,$ 1 $\,\,$ & $\,\,$\color{gr} 6.8375\color{black} $\,\,$ & $\,\,$\color{gr} 4.8357\color{black}   $\,\,$ \\
$\,\,$0.1463$\,\,$ & $\,\,$\color{gr} 0.1463\color{black} $\,\,$ & $\,\,$ 1 $\,\,$ & $\,\,$0.7072 $\,\,$ \\
$\,\,$0.2068$\,\,$ & $\,\,$\color{gr} 0.2068\color{black} $\,\,$ & $\,\,$1.4140$\,\,$ & $\,\,$ 1  $\,\,$ \\
\end{pmatrix},
\end{equation*}
\end{example}
\newpage
\begin{example}
\begin{equation*}
\mathbf{A} =
\begin{pmatrix}
$\,\,$ 1 $\,\,$ & $\,\,$1$\,\,$ & $\,\,$5$\,\,$ & $\,\,$7 $\,\,$ \\
$\,\,$ 1 $\,\,$ & $\,\,$ 1 $\,\,$ & $\,\,$8$\,\,$ & $\,\,$4 $\,\,$ \\
$\,\,$ 1/5$\,\,$ & $\,\,$ 1/8$\,\,$ & $\,\,$ 1 $\,\,$ & $\,\,$ 1/3 $\,\,$ \\
$\,\,$ 1/7$\,\,$ & $\,\,$ 1/4$\,\,$ & $\,\,$3$\,\,$ & $\,\,$ 1  $\,\,$ \\
\end{pmatrix},
\qquad
\lambda_{\max} =
4.1782,
\qquad
CR = 0.0672
\end{equation*}

\begin{equation*}
\mathbf{w}^{AMAST} =
\begin{pmatrix}
0.425629\\
\color{red} 0.413137\color{black} \\
0.055771\\
0.105463
\end{pmatrix}\end{equation*}
\begin{equation*}
\left[ \frac{{w}^{AMAST}_i}{{w}^{AMAST}_j} \right] =
\begin{pmatrix}
$\,\,$ 1 $\,\,$ & $\,\,$\color{red} 1.0302\color{black} $\,\,$ & $\,\,$7.6317$\,\,$ & $\,\,$4.0358$\,\,$ \\
$\,\,$\color{red} 0.9706\color{black} $\,\,$ & $\,\,$ 1 $\,\,$ & $\,\,$\color{red} 7.4077\color{black} $\,\,$ & $\,\,$\color{red} 3.9174\color{black}   $\,\,$ \\
$\,\,$0.1310$\,\,$ & $\,\,$\color{red} 0.1350\color{black} $\,\,$ & $\,\,$ 1 $\,\,$ & $\,\,$0.5288 $\,\,$ \\
$\,\,$0.2478$\,\,$ & $\,\,$\color{red} 0.2553\color{black} $\,\,$ & $\,\,$1.8910$\,\,$ & $\,\,$ 1  $\,\,$ \\
\end{pmatrix},
\end{equation*}

\begin{equation*}
\mathbf{w}^{\prime} =
\begin{pmatrix}
0.421952\\
0.418207\\
0.055290\\
0.104552
\end{pmatrix} =
0.991361\cdot
\begin{pmatrix}
0.425629\\
\color{gr} 0.421852\color{black} \\
0.055771\\
0.105463
\end{pmatrix},
\end{equation*}
\begin{equation*}
\left[ \frac{{w}^{\prime}_i}{{w}^{\prime}_j} \right] =
\begin{pmatrix}
$\,\,$ 1 $\,\,$ & $\,\,$\color{gr} 1.0090\color{black} $\,\,$ & $\,\,$7.6317$\,\,$ & $\,\,$4.0358$\,\,$ \\
$\,\,$\color{gr} 0.9911\color{black} $\,\,$ & $\,\,$ 1 $\,\,$ & $\,\,$\color{gr} 7.5639\color{black} $\,\,$ & $\,\,$\color{gr} \color{blue} 4\color{black}   $\,\,$ \\
$\,\,$0.1310$\,\,$ & $\,\,$\color{gr} 0.1322\color{black} $\,\,$ & $\,\,$ 1 $\,\,$ & $\,\,$0.5288 $\,\,$ \\
$\,\,$0.2478$\,\,$ & $\,\,$\color{gr} \color{blue}  1/4\color{black} $\,\,$ & $\,\,$1.8910$\,\,$ & $\,\,$ 1  $\,\,$ \\
\end{pmatrix},
\end{equation*}
\end{example}
\newpage
\begin{example}
\begin{equation*}
\mathbf{A} =
\begin{pmatrix}
$\,\,$ 1 $\,\,$ & $\,\,$1$\,\,$ & $\,\,$5$\,\,$ & $\,\,$7 $\,\,$ \\
$\,\,$ 1 $\,\,$ & $\,\,$ 1 $\,\,$ & $\,\,$8$\,\,$ & $\,\,$4 $\,\,$ \\
$\,\,$ 1/5$\,\,$ & $\,\,$ 1/8$\,\,$ & $\,\,$ 1 $\,\,$ & $\,\,$ 1/4 $\,\,$ \\
$\,\,$ 1/7$\,\,$ & $\,\,$ 1/4$\,\,$ & $\,\,$4$\,\,$ & $\,\,$ 1  $\,\,$ \\
\end{pmatrix},
\qquad
\lambda_{\max} =
4.2610,
\qquad
CR = 0.0984
\end{equation*}

\begin{equation*}
\mathbf{w}^{AMAST} =
\begin{pmatrix}
0.422616\\
\color{red} 0.409150\color{black} \\
0.052888\\
0.115346
\end{pmatrix}\end{equation*}
\begin{equation*}
\left[ \frac{{w}^{AMAST}_i}{{w}^{AMAST}_j} \right] =
\begin{pmatrix}
$\,\,$ 1 $\,\,$ & $\,\,$\color{red} 1.0329\color{black} $\,\,$ & $\,\,$7.9908$\,\,$ & $\,\,$3.6639$\,\,$ \\
$\,\,$\color{red} 0.9681\color{black} $\,\,$ & $\,\,$ 1 $\,\,$ & $\,\,$\color{red} 7.7362\color{black} $\,\,$ & $\,\,$\color{red} 3.5472\color{black}   $\,\,$ \\
$\,\,$0.1251$\,\,$ & $\,\,$\color{red} 0.1293\color{black} $\,\,$ & $\,\,$ 1 $\,\,$ & $\,\,$0.4585 $\,\,$ \\
$\,\,$0.2729$\,\,$ & $\,\,$\color{red} 0.2819\color{black} $\,\,$ & $\,\,$2.1809$\,\,$ & $\,\,$ 1  $\,\,$ \\
\end{pmatrix},
\end{equation*}

\begin{equation*}
\mathbf{w}^{\prime} =
\begin{pmatrix}
0.417001\\
0.417001\\
0.052185\\
0.113813
\end{pmatrix} =
0.986713\cdot
\begin{pmatrix}
0.422616\\
\color{gr} 0.422616\color{black} \\
0.052888\\
0.115346
\end{pmatrix},
\end{equation*}
\begin{equation*}
\left[ \frac{{w}^{\prime}_i}{{w}^{\prime}_j} \right] =
\begin{pmatrix}
$\,\,$ 1 $\,\,$ & $\,\,$\color{gr} \color{blue} 1\color{black} $\,\,$ & $\,\,$7.9908$\,\,$ & $\,\,$3.6639$\,\,$ \\
$\,\,$\color{gr} \color{blue} 1\color{black} $\,\,$ & $\,\,$ 1 $\,\,$ & $\,\,$\color{gr} 7.9908\color{black} $\,\,$ & $\,\,$\color{gr} 3.6639\color{black}   $\,\,$ \\
$\,\,$0.1251$\,\,$ & $\,\,$\color{gr} 0.1251\color{black} $\,\,$ & $\,\,$ 1 $\,\,$ & $\,\,$0.4585 $\,\,$ \\
$\,\,$0.2729$\,\,$ & $\,\,$\color{gr} 0.2729\color{black} $\,\,$ & $\,\,$2.1809$\,\,$ & $\,\,$ 1  $\,\,$ \\
\end{pmatrix},
\end{equation*}
\end{example}
\newpage
\begin{example}
\begin{equation*}
\mathbf{A} =
\begin{pmatrix}
$\,\,$ 1 $\,\,$ & $\,\,$1$\,\,$ & $\,\,$5$\,\,$ & $\,\,$7 $\,\,$ \\
$\,\,$ 1 $\,\,$ & $\,\,$ 1 $\,\,$ & $\,\,$9$\,\,$ & $\,\,$4 $\,\,$ \\
$\,\,$ 1/5$\,\,$ & $\,\,$ 1/9$\,\,$ & $\,\,$ 1 $\,\,$ & $\,\,$ 1/4 $\,\,$ \\
$\,\,$ 1/7$\,\,$ & $\,\,$ 1/4$\,\,$ & $\,\,$4$\,\,$ & $\,\,$ 1  $\,\,$ \\
\end{pmatrix},
\qquad
\lambda_{\max} =
4.2594,
\qquad
CR = 0.0978
\end{equation*}

\begin{equation*}
\mathbf{w}^{AMAST} =
\begin{pmatrix}
0.419047\\
\color{red} 0.416410\color{black} \\
0.050877\\
0.113666
\end{pmatrix}\end{equation*}
\begin{equation*}
\left[ \frac{{w}^{AMAST}_i}{{w}^{AMAST}_j} \right] =
\begin{pmatrix}
$\,\,$ 1 $\,\,$ & $\,\,$\color{red} 1.0063\color{black} $\,\,$ & $\,\,$8.2365$\,\,$ & $\,\,$3.6867$\,\,$ \\
$\,\,$\color{red} 0.9937\color{black} $\,\,$ & $\,\,$ 1 $\,\,$ & $\,\,$\color{red} 8.1847\color{black} $\,\,$ & $\,\,$\color{red} 3.6634\color{black}   $\,\,$ \\
$\,\,$0.1214$\,\,$ & $\,\,$\color{red} 0.1222\color{black} $\,\,$ & $\,\,$ 1 $\,\,$ & $\,\,$0.4476 $\,\,$ \\
$\,\,$0.2712$\,\,$ & $\,\,$\color{red} 0.2730\color{black} $\,\,$ & $\,\,$2.2341$\,\,$ & $\,\,$ 1  $\,\,$ \\
\end{pmatrix},
\end{equation*}

\begin{equation*}
\mathbf{w}^{\prime} =
\begin{pmatrix}
0.417945\\
0.417945\\
0.050743\\
0.113367
\end{pmatrix} =
0.997370\cdot
\begin{pmatrix}
0.419047\\
\color{gr} 0.419047\color{black} \\
0.050877\\
0.113666
\end{pmatrix},
\end{equation*}
\begin{equation*}
\left[ \frac{{w}^{\prime}_i}{{w}^{\prime}_j} \right] =
\begin{pmatrix}
$\,\,$ 1 $\,\,$ & $\,\,$\color{gr} \color{blue} 1\color{black} $\,\,$ & $\,\,$8.2365$\,\,$ & $\,\,$3.6867$\,\,$ \\
$\,\,$\color{gr} \color{blue} 1\color{black} $\,\,$ & $\,\,$ 1 $\,\,$ & $\,\,$\color{gr} 8.2365\color{black} $\,\,$ & $\,\,$\color{gr} 3.6867\color{black}   $\,\,$ \\
$\,\,$0.1214$\,\,$ & $\,\,$\color{gr} 0.1214\color{black} $\,\,$ & $\,\,$ 1 $\,\,$ & $\,\,$0.4476 $\,\,$ \\
$\,\,$0.2712$\,\,$ & $\,\,$\color{gr} 0.2712\color{black} $\,\,$ & $\,\,$2.2341$\,\,$ & $\,\,$ 1  $\,\,$ \\
\end{pmatrix},
\end{equation*}
\end{example}
\newpage
\begin{example}
\begin{equation*}
\mathbf{A} =
\begin{pmatrix}
$\,\,$ 1 $\,\,$ & $\,\,$1$\,\,$ & $\,\,$5$\,\,$ & $\,\,$8 $\,\,$ \\
$\,\,$ 1 $\,\,$ & $\,\,$ 1 $\,\,$ & $\,\,$7$\,\,$ & $\,\,$4 $\,\,$ \\
$\,\,$ 1/5$\,\,$ & $\,\,$ 1/7$\,\,$ & $\,\,$ 1 $\,\,$ & $\,\,$ 1/3 $\,\,$ \\
$\,\,$ 1/8$\,\,$ & $\,\,$ 1/4$\,\,$ & $\,\,$3$\,\,$ & $\,\,$ 1  $\,\,$ \\
\end{pmatrix},
\qquad
\lambda_{\max} =
4.2174,
\qquad
CR = 0.0820
\end{equation*}

\begin{equation*}
\mathbf{w}^{AMAST} =
\begin{pmatrix}
0.438309\\
\color{red} 0.399953\color{black} \\
0.058053\\
0.103684
\end{pmatrix}\end{equation*}
\begin{equation*}
\left[ \frac{{w}^{AMAST}_i}{{w}^{AMAST}_j} \right] =
\begin{pmatrix}
$\,\,$ 1 $\,\,$ & $\,\,$\color{red} 1.0959\color{black} $\,\,$ & $\,\,$7.5501$\,\,$ & $\,\,$4.2273$\,\,$ \\
$\,\,$\color{red} 0.9125\color{black} $\,\,$ & $\,\,$ 1 $\,\,$ & $\,\,$\color{red} 6.8894\color{black} $\,\,$ & $\,\,$\color{red} 3.8574\color{black}   $\,\,$ \\
$\,\,$0.1324$\,\,$ & $\,\,$\color{red} 0.1452\color{black} $\,\,$ & $\,\,$ 1 $\,\,$ & $\,\,$0.5599 $\,\,$ \\
$\,\,$0.2366$\,\,$ & $\,\,$\color{red} 0.2592\color{black} $\,\,$ & $\,\,$1.7860$\,\,$ & $\,\,$ 1  $\,\,$ \\
\end{pmatrix},
\end{equation*}

\begin{equation*}
\mathbf{w}^{\prime} =
\begin{pmatrix}
0.435513\\
0.403781\\
0.057683\\
0.103023
\end{pmatrix} =
0.993621\cdot
\begin{pmatrix}
0.438309\\
\color{gr} 0.406373\color{black} \\
0.058053\\
0.103684
\end{pmatrix},
\end{equation*}
\begin{equation*}
\left[ \frac{{w}^{\prime}_i}{{w}^{\prime}_j} \right] =
\begin{pmatrix}
$\,\,$ 1 $\,\,$ & $\,\,$\color{gr} 1.0786\color{black} $\,\,$ & $\,\,$7.5501$\,\,$ & $\,\,$4.2273$\,\,$ \\
$\,\,$\color{gr} 0.9271\color{black} $\,\,$ & $\,\,$ 1 $\,\,$ & $\,\,$\color{gr} \color{blue} 7\color{black} $\,\,$ & $\,\,$\color{gr} 3.9193\color{black}   $\,\,$ \\
$\,\,$0.1324$\,\,$ & $\,\,$\color{gr} \color{blue}  1/7\color{black} $\,\,$ & $\,\,$ 1 $\,\,$ & $\,\,$0.5599 $\,\,$ \\
$\,\,$0.2366$\,\,$ & $\,\,$\color{gr} 0.2551\color{black} $\,\,$ & $\,\,$1.7860$\,\,$ & $\,\,$ 1  $\,\,$ \\
\end{pmatrix},
\end{equation*}
\end{example}
\newpage
\begin{example}
\begin{equation*}
\mathbf{A} =
\begin{pmatrix}
$\,\,$ 1 $\,\,$ & $\,\,$1$\,\,$ & $\,\,$5$\,\,$ & $\,\,$8 $\,\,$ \\
$\,\,$ 1 $\,\,$ & $\,\,$ 1 $\,\,$ & $\,\,$7$\,\,$ & $\,\,$5 $\,\,$ \\
$\,\,$ 1/5$\,\,$ & $\,\,$ 1/7$\,\,$ & $\,\,$ 1 $\,\,$ & $\,\,$ 1/2 $\,\,$ \\
$\,\,$ 1/8$\,\,$ & $\,\,$ 1/5$\,\,$ & $\,\,$2$\,\,$ & $\,\,$ 1  $\,\,$ \\
\end{pmatrix},
\qquad
\lambda_{\max} =
4.1159,
\qquad
CR = 0.0437
\end{equation*}

\begin{equation*}
\mathbf{w}^{AMAST} =
\begin{pmatrix}
0.434486\\
\color{red} 0.418708\color{black} \\
0.061890\\
0.084916
\end{pmatrix}\end{equation*}
\begin{equation*}
\left[ \frac{{w}^{AMAST}_i}{{w}^{AMAST}_j} \right] =
\begin{pmatrix}
$\,\,$ 1 $\,\,$ & $\,\,$\color{red} 1.0377\color{black} $\,\,$ & $\,\,$7.0204$\,\,$ & $\,\,$5.1166$\,\,$ \\
$\,\,$\color{red} 0.9637\color{black} $\,\,$ & $\,\,$ 1 $\,\,$ & $\,\,$\color{red} 6.7654\color{black} $\,\,$ & $\,\,$\color{red} 4.9308\color{black}   $\,\,$ \\
$\,\,$0.1424$\,\,$ & $\,\,$\color{red} 0.1478\color{black} $\,\,$ & $\,\,$ 1 $\,\,$ & $\,\,$0.7288 $\,\,$ \\
$\,\,$0.1954$\,\,$ & $\,\,$\color{red} 0.2028\color{black} $\,\,$ & $\,\,$1.3721$\,\,$ & $\,\,$ 1  $\,\,$ \\
\end{pmatrix},
\end{equation*}

\begin{equation*}
\mathbf{w}^{\prime} =
\begin{pmatrix}
0.431949\\
0.422103\\
0.061528\\
0.084421
\end{pmatrix} =
0.994160\cdot
\begin{pmatrix}
0.434486\\
\color{gr} 0.424582\color{black} \\
0.061890\\
0.084916
\end{pmatrix},
\end{equation*}
\begin{equation*}
\left[ \frac{{w}^{\prime}_i}{{w}^{\prime}_j} \right] =
\begin{pmatrix}
$\,\,$ 1 $\,\,$ & $\,\,$\color{gr} 1.0233\color{black} $\,\,$ & $\,\,$7.0204$\,\,$ & $\,\,$5.1166$\,\,$ \\
$\,\,$\color{gr} 0.9772\color{black} $\,\,$ & $\,\,$ 1 $\,\,$ & $\,\,$\color{gr} 6.8603\color{black} $\,\,$ & $\,\,$\color{gr} \color{blue} 5\color{black}   $\,\,$ \\
$\,\,$0.1424$\,\,$ & $\,\,$\color{gr} 0.1458\color{black} $\,\,$ & $\,\,$ 1 $\,\,$ & $\,\,$0.7288 $\,\,$ \\
$\,\,$0.1954$\,\,$ & $\,\,$\color{gr} \color{blue}  1/5\color{black} $\,\,$ & $\,\,$1.3721$\,\,$ & $\,\,$ 1  $\,\,$ \\
\end{pmatrix},
\end{equation*}
\end{example}
\newpage
\begin{example}
\begin{equation*}
\mathbf{A} =
\begin{pmatrix}
$\,\,$ 1 $\,\,$ & $\,\,$1$\,\,$ & $\,\,$5$\,\,$ & $\,\,$8 $\,\,$ \\
$\,\,$ 1 $\,\,$ & $\,\,$ 1 $\,\,$ & $\,\,$8$\,\,$ & $\,\,$4 $\,\,$ \\
$\,\,$ 1/5$\,\,$ & $\,\,$ 1/8$\,\,$ & $\,\,$ 1 $\,\,$ & $\,\,$ 1/3 $\,\,$ \\
$\,\,$ 1/8$\,\,$ & $\,\,$ 1/4$\,\,$ & $\,\,$3$\,\,$ & $\,\,$ 1  $\,\,$ \\
\end{pmatrix},
\qquad
\lambda_{\max} =
4.2162,
\qquad
CR = 0.0815
\end{equation*}

\begin{equation*}
\mathbf{w}^{AMAST} =
\begin{pmatrix}
0.434051\\
\color{red} 0.408284\color{black} \\
0.055559\\
0.102107
\end{pmatrix}\end{equation*}
\begin{equation*}
\left[ \frac{{w}^{AMAST}_i}{{w}^{AMAST}_j} \right] =
\begin{pmatrix}
$\,\,$ 1 $\,\,$ & $\,\,$\color{red} 1.0631\color{black} $\,\,$ & $\,\,$7.8125$\,\,$ & $\,\,$4.2510$\,\,$ \\
$\,\,$\color{red} 0.9406\color{black} $\,\,$ & $\,\,$ 1 $\,\,$ & $\,\,$\color{red} 7.3487\color{black} $\,\,$ & $\,\,$\color{red} 3.9986\color{black}   $\,\,$ \\
$\,\,$0.1280$\,\,$ & $\,\,$\color{red} 0.1361\color{black} $\,\,$ & $\,\,$ 1 $\,\,$ & $\,\,$0.5441 $\,\,$ \\
$\,\,$0.2352$\,\,$ & $\,\,$\color{red} 0.2501\color{black} $\,\,$ & $\,\,$1.8378$\,\,$ & $\,\,$ 1  $\,\,$ \\
\end{pmatrix},
\end{equation*}

\begin{equation*}
\mathbf{w}^{\prime} =
\begin{pmatrix}
0.433989\\
0.408368\\
0.055551\\
0.102092
\end{pmatrix} =
0.999857\cdot
\begin{pmatrix}
0.434051\\
\color{gr} 0.408427\color{black} \\
0.055559\\
0.102107
\end{pmatrix},
\end{equation*}
\begin{equation*}
\left[ \frac{{w}^{\prime}_i}{{w}^{\prime}_j} \right] =
\begin{pmatrix}
$\,\,$ 1 $\,\,$ & $\,\,$\color{gr} 1.0627\color{black} $\,\,$ & $\,\,$7.8125$\,\,$ & $\,\,$4.2510$\,\,$ \\
$\,\,$\color{gr} 0.9410\color{black} $\,\,$ & $\,\,$ 1 $\,\,$ & $\,\,$\color{gr} 7.3513\color{black} $\,\,$ & $\,\,$\color{gr} \color{blue} 4\color{black}   $\,\,$ \\
$\,\,$0.1280$\,\,$ & $\,\,$\color{gr} 0.1360\color{black} $\,\,$ & $\,\,$ 1 $\,\,$ & $\,\,$0.5441 $\,\,$ \\
$\,\,$0.2352$\,\,$ & $\,\,$\color{gr} \color{blue}  1/4\color{black} $\,\,$ & $\,\,$1.8378$\,\,$ & $\,\,$ 1  $\,\,$ \\
\end{pmatrix},
\end{equation*}
\end{example}
\newpage
\begin{example}
\begin{equation*}
\mathbf{A} =
\begin{pmatrix}
$\,\,$ 1 $\,\,$ & $\,\,$1$\,\,$ & $\,\,$5$\,\,$ & $\,\,$8 $\,\,$ \\
$\,\,$ 1 $\,\,$ & $\,\,$ 1 $\,\,$ & $\,\,$8$\,\,$ & $\,\,$5 $\,\,$ \\
$\,\,$ 1/5$\,\,$ & $\,\,$ 1/8$\,\,$ & $\,\,$ 1 $\,\,$ & $\,\,$ 1/3 $\,\,$ \\
$\,\,$ 1/8$\,\,$ & $\,\,$ 1/5$\,\,$ & $\,\,$3$\,\,$ & $\,\,$ 1  $\,\,$ \\
\end{pmatrix},
\qquad
\lambda_{\max} =
4.2144,
\qquad
CR = 0.0808
\end{equation*}

\begin{equation*}
\mathbf{w}^{AMAST} =
\begin{pmatrix}
0.426916\\
\color{red} 0.423128\color{black} \\
0.054878\\
0.095077
\end{pmatrix}\end{equation*}
\begin{equation*}
\left[ \frac{{w}^{AMAST}_i}{{w}^{AMAST}_j} \right] =
\begin{pmatrix}
$\,\,$ 1 $\,\,$ & $\,\,$\color{red} 1.0090\color{black} $\,\,$ & $\,\,$7.7794$\,\,$ & $\,\,$4.4902$\,\,$ \\
$\,\,$\color{red} 0.9911\color{black} $\,\,$ & $\,\,$ 1 $\,\,$ & $\,\,$\color{red} 7.7104\color{black} $\,\,$ & $\,\,$\color{red} 4.4504\color{black}   $\,\,$ \\
$\,\,$0.1285$\,\,$ & $\,\,$\color{red} 0.1297\color{black} $\,\,$ & $\,\,$ 1 $\,\,$ & $\,\,$0.5772 $\,\,$ \\
$\,\,$0.2227$\,\,$ & $\,\,$\color{red} 0.2247\color{black} $\,\,$ & $\,\,$1.7325$\,\,$ & $\,\,$ 1  $\,\,$ \\
\end{pmatrix},
\end{equation*}

\begin{equation*}
\mathbf{w}^{\prime} =
\begin{pmatrix}
0.425305\\
0.425305\\
0.054671\\
0.094718
\end{pmatrix} =
0.996226\cdot
\begin{pmatrix}
0.426916\\
\color{gr} 0.426916\color{black} \\
0.054878\\
0.095077
\end{pmatrix},
\end{equation*}
\begin{equation*}
\left[ \frac{{w}^{\prime}_i}{{w}^{\prime}_j} \right] =
\begin{pmatrix}
$\,\,$ 1 $\,\,$ & $\,\,$\color{gr} \color{blue} 1\color{black} $\,\,$ & $\,\,$7.7794$\,\,$ & $\,\,$4.4902$\,\,$ \\
$\,\,$\color{gr} \color{blue} 1\color{black} $\,\,$ & $\,\,$ 1 $\,\,$ & $\,\,$\color{gr} 7.7794\color{black} $\,\,$ & $\,\,$\color{gr} 4.4902\color{black}   $\,\,$ \\
$\,\,$0.1285$\,\,$ & $\,\,$\color{gr} 0.1285\color{black} $\,\,$ & $\,\,$ 1 $\,\,$ & $\,\,$0.5772 $\,\,$ \\
$\,\,$0.2227$\,\,$ & $\,\,$\color{gr} 0.2227\color{black} $\,\,$ & $\,\,$1.7325$\,\,$ & $\,\,$ 1  $\,\,$ \\
\end{pmatrix},
\end{equation*}
\end{example}
\newpage
\begin{example}
\begin{equation*}
\mathbf{A} =
\begin{pmatrix}
$\,\,$ 1 $\,\,$ & $\,\,$1$\,\,$ & $\,\,$5$\,\,$ & $\,\,$9 $\,\,$ \\
$\,\,$ 1 $\,\,$ & $\,\,$ 1 $\,\,$ & $\,\,$7$\,\,$ & $\,\,$4 $\,\,$ \\
$\,\,$ 1/5$\,\,$ & $\,\,$ 1/7$\,\,$ & $\,\,$ 1 $\,\,$ & $\,\,$ 1/3 $\,\,$ \\
$\,\,$ 1/9$\,\,$ & $\,\,$ 1/4$\,\,$ & $\,\,$3$\,\,$ & $\,\,$ 1  $\,\,$ \\
\end{pmatrix},
\qquad
\lambda_{\max} =
4.2553,
\qquad
CR = 0.0963
\end{equation*}

\begin{equation*}
\mathbf{w}^{AMAST} =
\begin{pmatrix}
0.445417\\
\color{red} 0.395838\color{black} \\
0.057849\\
0.100896
\end{pmatrix}\end{equation*}
\begin{equation*}
\left[ \frac{{w}^{AMAST}_i}{{w}^{AMAST}_j} \right] =
\begin{pmatrix}
$\,\,$ 1 $\,\,$ & $\,\,$\color{red} 1.1253\color{black} $\,\,$ & $\,\,$7.6996$\,\,$ & $\,\,$4.4146$\,\,$ \\
$\,\,$\color{red} 0.8887\color{black} $\,\,$ & $\,\,$ 1 $\,\,$ & $\,\,$\color{red} 6.8426\color{black} $\,\,$ & $\,\,$\color{red} 3.9232\color{black}   $\,\,$ \\
$\,\,$0.1299$\,\,$ & $\,\,$\color{red} 0.1461\color{black} $\,\,$ & $\,\,$ 1 $\,\,$ & $\,\,$0.5734 $\,\,$ \\
$\,\,$0.2265$\,\,$ & $\,\,$\color{red} 0.2549\color{black} $\,\,$ & $\,\,$1.7441$\,\,$ & $\,\,$ 1  $\,\,$ \\
\end{pmatrix},
\end{equation*}

\begin{equation*}
\mathbf{w}^{\prime} =
\begin{pmatrix}
0.441994\\
0.400481\\
0.057405\\
0.100120
\end{pmatrix} =
0.992315\cdot
\begin{pmatrix}
0.445417\\
\color{gr} 0.403582\color{black} \\
0.057849\\
0.100896
\end{pmatrix},
\end{equation*}
\begin{equation*}
\left[ \frac{{w}^{\prime}_i}{{w}^{\prime}_j} \right] =
\begin{pmatrix}
$\,\,$ 1 $\,\,$ & $\,\,$\color{gr} 1.1037\color{black} $\,\,$ & $\,\,$7.6996$\,\,$ & $\,\,$4.4146$\,\,$ \\
$\,\,$\color{gr} 0.9061\color{black} $\,\,$ & $\,\,$ 1 $\,\,$ & $\,\,$\color{gr} 6.9764\color{black} $\,\,$ & $\,\,$\color{gr} \color{blue} 4\color{black}   $\,\,$ \\
$\,\,$0.1299$\,\,$ & $\,\,$\color{gr} 0.1433\color{black} $\,\,$ & $\,\,$ 1 $\,\,$ & $\,\,$0.5734 $\,\,$ \\
$\,\,$0.2265$\,\,$ & $\,\,$\color{gr} \color{blue}  1/4\color{black} $\,\,$ & $\,\,$1.7441$\,\,$ & $\,\,$ 1  $\,\,$ \\
\end{pmatrix},
\end{equation*}
\end{example}
\newpage
\begin{example}
\begin{equation*}
\mathbf{A} =
\begin{pmatrix}
$\,\,$ 1 $\,\,$ & $\,\,$1$\,\,$ & $\,\,$5$\,\,$ & $\,\,$9 $\,\,$ \\
$\,\,$ 1 $\,\,$ & $\,\,$ 1 $\,\,$ & $\,\,$8$\,\,$ & $\,\,$5 $\,\,$ \\
$\,\,$ 1/5$\,\,$ & $\,\,$ 1/8$\,\,$ & $\,\,$ 1 $\,\,$ & $\,\,$ 1/3 $\,\,$ \\
$\,\,$ 1/9$\,\,$ & $\,\,$ 1/5$\,\,$ & $\,\,$3$\,\,$ & $\,\,$ 1  $\,\,$ \\
\end{pmatrix},
\qquad
\lambda_{\max} =
4.2489,
\qquad
CR = 0.0939
\end{equation*}

\begin{equation*}
\mathbf{w}^{AMAST} =
\begin{pmatrix}
0.434248\\
\color{red} 0.418586\color{black} \\
0.054715\\
0.092452
\end{pmatrix}\end{equation*}
\begin{equation*}
\left[ \frac{{w}^{AMAST}_i}{{w}^{AMAST}_j} \right] =
\begin{pmatrix}
$\,\,$ 1 $\,\,$ & $\,\,$\color{red} 1.0374\color{black} $\,\,$ & $\,\,$7.9366$\,\,$ & $\,\,$4.6970$\,\,$ \\
$\,\,$\color{red} 0.9639\color{black} $\,\,$ & $\,\,$ 1 $\,\,$ & $\,\,$\color{red} 7.6503\color{black} $\,\,$ & $\,\,$\color{red} 4.5276\color{black}   $\,\,$ \\
$\,\,$0.1260$\,\,$ & $\,\,$\color{red} 0.1307\color{black} $\,\,$ & $\,\,$ 1 $\,\,$ & $\,\,$0.5918 $\,\,$ \\
$\,\,$0.2129$\,\,$ & $\,\,$\color{red} 0.2209\color{black} $\,\,$ & $\,\,$1.6897$\,\,$ & $\,\,$ 1  $\,\,$ \\
\end{pmatrix},
\end{equation*}

\begin{equation*}
\mathbf{w}^{\prime} =
\begin{pmatrix}
0.427551\\
0.427551\\
0.053871\\
0.091026
\end{pmatrix} =
0.984579\cdot
\begin{pmatrix}
0.434248\\
\color{gr} 0.434248\color{black} \\
0.054715\\
0.092452
\end{pmatrix},
\end{equation*}
\begin{equation*}
\left[ \frac{{w}^{\prime}_i}{{w}^{\prime}_j} \right] =
\begin{pmatrix}
$\,\,$ 1 $\,\,$ & $\,\,$\color{gr} \color{blue} 1\color{black} $\,\,$ & $\,\,$7.9366$\,\,$ & $\,\,$4.6970$\,\,$ \\
$\,\,$\color{gr} \color{blue} 1\color{black} $\,\,$ & $\,\,$ 1 $\,\,$ & $\,\,$\color{gr} 7.9366\color{black} $\,\,$ & $\,\,$\color{gr} 4.6970\color{black}   $\,\,$ \\
$\,\,$0.1260$\,\,$ & $\,\,$\color{gr} 0.1260\color{black} $\,\,$ & $\,\,$ 1 $\,\,$ & $\,\,$0.5918 $\,\,$ \\
$\,\,$0.2129$\,\,$ & $\,\,$\color{gr} 0.2129\color{black} $\,\,$ & $\,\,$1.6897$\,\,$ & $\,\,$ 1  $\,\,$ \\
\end{pmatrix},
\end{equation*}
\end{example}
\newpage
\begin{example}
\begin{equation*}
\mathbf{A} =
\begin{pmatrix}
$\,\,$ 1 $\,\,$ & $\,\,$1$\,\,$ & $\,\,$5$\,\,$ & $\,\,$9 $\,\,$ \\
$\,\,$ 1 $\,\,$ & $\,\,$ 1 $\,\,$ & $\,\,$9$\,\,$ & $\,\,$5 $\,\,$ \\
$\,\,$ 1/5$\,\,$ & $\,\,$ 1/9$\,\,$ & $\,\,$ 1 $\,\,$ & $\,\,$ 1/3 $\,\,$ \\
$\,\,$ 1/9$\,\,$ & $\,\,$ 1/5$\,\,$ & $\,\,$3$\,\,$ & $\,\,$ 1  $\,\,$ \\
\end{pmatrix},
\qquad
\lambda_{\max} =
4.2483,
\qquad
CR = 0.0936
\end{equation*}

\begin{equation*}
\mathbf{w}^{AMAST} =
\begin{pmatrix}
0.430399\\
\color{red} 0.425900\color{black} \\
0.052632\\
0.091069
\end{pmatrix}\end{equation*}
\begin{equation*}
\left[ \frac{{w}^{AMAST}_i}{{w}^{AMAST}_j} \right] =
\begin{pmatrix}
$\,\,$ 1 $\,\,$ & $\,\,$\color{red} 1.0106\color{black} $\,\,$ & $\,\,$8.1775$\,\,$ & $\,\,$4.7261$\,\,$ \\
$\,\,$\color{red} 0.9895\color{black} $\,\,$ & $\,\,$ 1 $\,\,$ & $\,\,$\color{red} 8.0920\color{black} $\,\,$ & $\,\,$\color{red} 4.6767\color{black}   $\,\,$ \\
$\,\,$0.1223$\,\,$ & $\,\,$\color{red} 0.1236\color{black} $\,\,$ & $\,\,$ 1 $\,\,$ & $\,\,$0.5779 $\,\,$ \\
$\,\,$0.2116$\,\,$ & $\,\,$\color{red} 0.2138\color{black} $\,\,$ & $\,\,$1.7303$\,\,$ & $\,\,$ 1  $\,\,$ \\
\end{pmatrix},
\end{equation*}

\begin{equation*}
\mathbf{w}^{\prime} =
\begin{pmatrix}
0.428471\\
0.428471\\
0.052396\\
0.090661
\end{pmatrix} =
0.995521\cdot
\begin{pmatrix}
0.430399\\
\color{gr} 0.430399\color{black} \\
0.052632\\
0.091069
\end{pmatrix},
\end{equation*}
\begin{equation*}
\left[ \frac{{w}^{\prime}_i}{{w}^{\prime}_j} \right] =
\begin{pmatrix}
$\,\,$ 1 $\,\,$ & $\,\,$\color{gr} \color{blue} 1\color{black} $\,\,$ & $\,\,$8.1775$\,\,$ & $\,\,$4.7261$\,\,$ \\
$\,\,$\color{gr} \color{blue} 1\color{black} $\,\,$ & $\,\,$ 1 $\,\,$ & $\,\,$\color{gr} 8.1775\color{black} $\,\,$ & $\,\,$\color{gr} 4.7261\color{black}   $\,\,$ \\
$\,\,$0.1223$\,\,$ & $\,\,$\color{gr} 0.1223\color{black} $\,\,$ & $\,\,$ 1 $\,\,$ & $\,\,$0.5779 $\,\,$ \\
$\,\,$0.2116$\,\,$ & $\,\,$\color{gr} 0.2116\color{black} $\,\,$ & $\,\,$1.7303$\,\,$ & $\,\,$ 1  $\,\,$ \\
\end{pmatrix},
\end{equation*}
\end{example}
\newpage
\begin{example}
\begin{equation*}
\mathbf{A} =
\begin{pmatrix}
$\,\,$ 1 $\,\,$ & $\,\,$1$\,\,$ & $\,\,$6$\,\,$ & $\,\,$3 $\,\,$ \\
$\,\,$ 1 $\,\,$ & $\,\,$ 1 $\,\,$ & $\,\,$8$\,\,$ & $\,\,$2 $\,\,$ \\
$\,\,$ 1/6$\,\,$ & $\,\,$ 1/8$\,\,$ & $\,\,$ 1 $\,\,$ & $\,\,$ 1/6 $\,\,$ \\
$\,\,$ 1/3$\,\,$ & $\,\,$ 1/2$\,\,$ & $\,\,$6$\,\,$ & $\,\,$ 1  $\,\,$ \\
\end{pmatrix},
\qquad
\lambda_{\max} =
4.1031,
\qquad
CR = 0.0389
\end{equation*}

\begin{equation*}
\mathbf{w}^{AMAST} =
\begin{pmatrix}
0.387498\\
\color{red} 0.373693\color{black} \\
0.047037\\
0.191772
\end{pmatrix}\end{equation*}
\begin{equation*}
\left[ \frac{{w}^{AMAST}_i}{{w}^{AMAST}_j} \right] =
\begin{pmatrix}
$\,\,$ 1 $\,\,$ & $\,\,$\color{red} 1.0369\color{black} $\,\,$ & $\,\,$8.2382$\,\,$ & $\,\,$2.0206$\,\,$ \\
$\,\,$\color{red} 0.9644\color{black} $\,\,$ & $\,\,$ 1 $\,\,$ & $\,\,$\color{red} 7.9447\color{black} $\,\,$ & $\,\,$\color{red} 1.9486\color{black}   $\,\,$ \\
$\,\,$0.1214$\,\,$ & $\,\,$\color{red} 0.1259\color{black} $\,\,$ & $\,\,$ 1 $\,\,$ & $\,\,$0.2453 $\,\,$ \\
$\,\,$0.4949$\,\,$ & $\,\,$\color{red} 0.5132\color{black} $\,\,$ & $\,\,$4.0771$\,\,$ & $\,\,$ 1  $\,\,$ \\
\end{pmatrix},
\end{equation*}

\begin{equation*}
\mathbf{w}^{\prime} =
\begin{pmatrix}
0.386494\\
0.375316\\
0.046915\\
0.191275
\end{pmatrix} =
0.997407\cdot
\begin{pmatrix}
0.387498\\
\color{gr} 0.376292\color{black} \\
0.047037\\
0.191772
\end{pmatrix},
\end{equation*}
\begin{equation*}
\left[ \frac{{w}^{\prime}_i}{{w}^{\prime}_j} \right] =
\begin{pmatrix}
$\,\,$ 1 $\,\,$ & $\,\,$\color{gr} 1.0298\color{black} $\,\,$ & $\,\,$8.2382$\,\,$ & $\,\,$2.0206$\,\,$ \\
$\,\,$\color{gr} 0.9711\color{black} $\,\,$ & $\,\,$ 1 $\,\,$ & $\,\,$\color{gr} \color{blue} 8\color{black} $\,\,$ & $\,\,$\color{gr} 1.9622\color{black}   $\,\,$ \\
$\,\,$0.1214$\,\,$ & $\,\,$\color{gr} \color{blue}  1/8\color{black} $\,\,$ & $\,\,$ 1 $\,\,$ & $\,\,$0.2453 $\,\,$ \\
$\,\,$0.4949$\,\,$ & $\,\,$\color{gr} 0.5096\color{black} $\,\,$ & $\,\,$4.0771$\,\,$ & $\,\,$ 1  $\,\,$ \\
\end{pmatrix},
\end{equation*}
\end{example}
\newpage
\begin{example}
\begin{equation*}
\mathbf{A} =
\begin{pmatrix}
$\,\,$ 1 $\,\,$ & $\,\,$1$\,\,$ & $\,\,$6$\,\,$ & $\,\,$3 $\,\,$ \\
$\,\,$ 1 $\,\,$ & $\,\,$ 1 $\,\,$ & $\,\,$9$\,\,$ & $\,\,$2 $\,\,$ \\
$\,\,$ 1/6$\,\,$ & $\,\,$ 1/9$\,\,$ & $\,\,$ 1 $\,\,$ & $\,\,$ 1/7 $\,\,$ \\
$\,\,$ 1/3$\,\,$ & $\,\,$ 1/2$\,\,$ & $\,\,$7$\,\,$ & $\,\,$ 1  $\,\,$ \\
\end{pmatrix},
\qquad
\lambda_{\max} =
4.1342,
\qquad
CR = 0.0506
\end{equation*}

\begin{equation*}
\mathbf{w}^{AMAST} =
\begin{pmatrix}
0.381748\\
\color{red} 0.377749\color{black} \\
0.043697\\
0.196806
\end{pmatrix}\end{equation*}
\begin{equation*}
\left[ \frac{{w}^{AMAST}_i}{{w}^{AMAST}_j} \right] =
\begin{pmatrix}
$\,\,$ 1 $\,\,$ & $\,\,$\color{red} 1.0106\color{black} $\,\,$ & $\,\,$8.7363$\,\,$ & $\,\,$1.9397$\,\,$ \\
$\,\,$\color{red} 0.9895\color{black} $\,\,$ & $\,\,$ 1 $\,\,$ & $\,\,$\color{red} 8.6447\color{black} $\,\,$ & $\,\,$\color{red} 1.9194\color{black}   $\,\,$ \\
$\,\,$0.1145$\,\,$ & $\,\,$\color{red} 0.1157\color{black} $\,\,$ & $\,\,$ 1 $\,\,$ & $\,\,$0.2220 $\,\,$ \\
$\,\,$0.5155$\,\,$ & $\,\,$\color{red} 0.5210\color{black} $\,\,$ & $\,\,$4.5039$\,\,$ & $\,\,$ 1  $\,\,$ \\
\end{pmatrix},
\end{equation*}

\begin{equation*}
\mathbf{w}^{\prime} =
\begin{pmatrix}
0.380228\\
0.380228\\
0.043523\\
0.196022
\end{pmatrix} =
0.996017\cdot
\begin{pmatrix}
0.381748\\
\color{gr} 0.381748\color{black} \\
0.043697\\
0.196806
\end{pmatrix},
\end{equation*}
\begin{equation*}
\left[ \frac{{w}^{\prime}_i}{{w}^{\prime}_j} \right] =
\begin{pmatrix}
$\,\,$ 1 $\,\,$ & $\,\,$\color{gr} \color{blue} 1\color{black} $\,\,$ & $\,\,$8.7363$\,\,$ & $\,\,$1.9397$\,\,$ \\
$\,\,$\color{gr} \color{blue} 1\color{black} $\,\,$ & $\,\,$ 1 $\,\,$ & $\,\,$\color{gr} 8.7363\color{black} $\,\,$ & $\,\,$\color{gr} 1.9397\color{black}   $\,\,$ \\
$\,\,$0.1145$\,\,$ & $\,\,$\color{gr} 0.1145\color{black} $\,\,$ & $\,\,$ 1 $\,\,$ & $\,\,$0.2220 $\,\,$ \\
$\,\,$0.5155$\,\,$ & $\,\,$\color{gr} 0.5155\color{black} $\,\,$ & $\,\,$4.5039$\,\,$ & $\,\,$ 1  $\,\,$ \\
\end{pmatrix},
\end{equation*}
\end{example}
\newpage
\begin{example}
\begin{equation*}
\mathbf{A} =
\begin{pmatrix}
$\,\,$ 1 $\,\,$ & $\,\,$1$\,\,$ & $\,\,$6$\,\,$ & $\,\,$3 $\,\,$ \\
$\,\,$ 1 $\,\,$ & $\,\,$ 1 $\,\,$ & $\,\,$9$\,\,$ & $\,\,$2 $\,\,$ \\
$\,\,$ 1/6$\,\,$ & $\,\,$ 1/9$\,\,$ & $\,\,$ 1 $\,\,$ & $\,\,$ 1/8 $\,\,$ \\
$\,\,$ 1/3$\,\,$ & $\,\,$ 1/2$\,\,$ & $\,\,$8$\,\,$ & $\,\,$ 1  $\,\,$ \\
\end{pmatrix},
\qquad
\lambda_{\max} =
4.1664,
\qquad
CR = 0.0627
\end{equation*}

\begin{equation*}
\mathbf{w}^{AMAST} =
\begin{pmatrix}
0.379493\\
\color{red} 0.374844\color{black} \\
0.042430\\
0.203233
\end{pmatrix}\end{equation*}
\begin{equation*}
\left[ \frac{{w}^{AMAST}_i}{{w}^{AMAST}_j} \right] =
\begin{pmatrix}
$\,\,$ 1 $\,\,$ & $\,\,$\color{red} 1.0124\color{black} $\,\,$ & $\,\,$8.9440$\,\,$ & $\,\,$1.8673$\,\,$ \\
$\,\,$\color{red} 0.9878\color{black} $\,\,$ & $\,\,$ 1 $\,\,$ & $\,\,$\color{red} 8.8344\color{black} $\,\,$ & $\,\,$\color{red} 1.8444\color{black}   $\,\,$ \\
$\,\,$0.1118$\,\,$ & $\,\,$\color{red} 0.1132\color{black} $\,\,$ & $\,\,$ 1 $\,\,$ & $\,\,$0.2088 $\,\,$ \\
$\,\,$0.5355$\,\,$ & $\,\,$\color{red} 0.5422\color{black} $\,\,$ & $\,\,$4.7898$\,\,$ & $\,\,$ 1  $\,\,$ \\
\end{pmatrix},
\end{equation*}

\begin{equation*}
\mathbf{w}^{\prime} =
\begin{pmatrix}
0.377737\\
0.377737\\
0.042234\\
0.202293
\end{pmatrix} =
0.995373\cdot
\begin{pmatrix}
0.379493\\
\color{gr} 0.379493\color{black} \\
0.042430\\
0.203233
\end{pmatrix},
\end{equation*}
\begin{equation*}
\left[ \frac{{w}^{\prime}_i}{{w}^{\prime}_j} \right] =
\begin{pmatrix}
$\,\,$ 1 $\,\,$ & $\,\,$\color{gr} \color{blue} 1\color{black} $\,\,$ & $\,\,$8.9440$\,\,$ & $\,\,$1.8673$\,\,$ \\
$\,\,$\color{gr} \color{blue} 1\color{black} $\,\,$ & $\,\,$ 1 $\,\,$ & $\,\,$\color{gr} 8.9440\color{black} $\,\,$ & $\,\,$\color{gr} 1.8673\color{black}   $\,\,$ \\
$\,\,$0.1118$\,\,$ & $\,\,$\color{gr} 0.1118\color{black} $\,\,$ & $\,\,$ 1 $\,\,$ & $\,\,$0.2088 $\,\,$ \\
$\,\,$0.5355$\,\,$ & $\,\,$\color{gr} 0.5355\color{black} $\,\,$ & $\,\,$4.7898$\,\,$ & $\,\,$ 1  $\,\,$ \\
\end{pmatrix},
\end{equation*}
\end{example}
\newpage
\begin{example}
\begin{equation*}
\mathbf{A} =
\begin{pmatrix}
$\,\,$ 1 $\,\,$ & $\,\,$1$\,\,$ & $\,\,$6$\,\,$ & $\,\,$3 $\,\,$ \\
$\,\,$ 1 $\,\,$ & $\,\,$ 1 $\,\,$ & $\,\,$9$\,\,$ & $\,\,$2 $\,\,$ \\
$\,\,$ 1/6$\,\,$ & $\,\,$ 1/9$\,\,$ & $\,\,$ 1 $\,\,$ & $\,\,$ 1/9 $\,\,$ \\
$\,\,$ 1/3$\,\,$ & $\,\,$ 1/2$\,\,$ & $\,\,$9$\,\,$ & $\,\,$ 1  $\,\,$ \\
\end{pmatrix},
\qquad
\lambda_{\max} =
4.1990,
\qquad
CR = 0.0750
\end{equation*}

\begin{equation*}
\mathbf{w}^{AMAST} =
\begin{pmatrix}
0.377402\\
\color{red} 0.372180\color{black} \\
0.041376\\
0.209041
\end{pmatrix}\end{equation*}
\begin{equation*}
\left[ \frac{{w}^{AMAST}_i}{{w}^{AMAST}_j} \right] =
\begin{pmatrix}
$\,\,$ 1 $\,\,$ & $\,\,$\color{red} 1.0140\color{black} $\,\,$ & $\,\,$9.1213$\,\,$ & $\,\,$1.8054$\,\,$ \\
$\,\,$\color{red} 0.9862\color{black} $\,\,$ & $\,\,$ 1 $\,\,$ & $\,\,$\color{red} 8.9951\color{black} $\,\,$ & $\,\,$\color{red} 1.7804\color{black}   $\,\,$ \\
$\,\,$0.1096$\,\,$ & $\,\,$\color{red} 0.1112\color{black} $\,\,$ & $\,\,$ 1 $\,\,$ & $\,\,$0.1979 $\,\,$ \\
$\,\,$0.5539$\,\,$ & $\,\,$\color{red} 0.5617\color{black} $\,\,$ & $\,\,$5.0523$\,\,$ & $\,\,$ 1  $\,\,$ \\
\end{pmatrix},
\end{equation*}

\begin{equation*}
\mathbf{w}^{\prime} =
\begin{pmatrix}
0.377326\\
0.372307\\
0.041367\\
0.208999
\end{pmatrix} =
0.999798\cdot
\begin{pmatrix}
0.377402\\
\color{gr} 0.372383\color{black} \\
0.041376\\
0.209041
\end{pmatrix},
\end{equation*}
\begin{equation*}
\left[ \frac{{w}^{\prime}_i}{{w}^{\prime}_j} \right] =
\begin{pmatrix}
$\,\,$ 1 $\,\,$ & $\,\,$\color{gr} 1.0135\color{black} $\,\,$ & $\,\,$9.1213$\,\,$ & $\,\,$1.8054$\,\,$ \\
$\,\,$\color{gr} 0.9867\color{black} $\,\,$ & $\,\,$ 1 $\,\,$ & $\,\,$\color{gr} \color{blue} 9\color{black} $\,\,$ & $\,\,$\color{gr} 1.7814\color{black}   $\,\,$ \\
$\,\,$0.1096$\,\,$ & $\,\,$\color{gr} \color{blue}  1/9\color{black} $\,\,$ & $\,\,$ 1 $\,\,$ & $\,\,$0.1979 $\,\,$ \\
$\,\,$0.5539$\,\,$ & $\,\,$\color{gr} 0.5614\color{black} $\,\,$ & $\,\,$5.0523$\,\,$ & $\,\,$ 1  $\,\,$ \\
\end{pmatrix},
\end{equation*}
\end{example}
\newpage
\begin{example}
\begin{equation*}
\mathbf{A} =
\begin{pmatrix}
$\,\,$ 1 $\,\,$ & $\,\,$1$\,\,$ & $\,\,$6$\,\,$ & $\,\,$4 $\,\,$ \\
$\,\,$ 1 $\,\,$ & $\,\,$ 1 $\,\,$ & $\,\,$8$\,\,$ & $\,\,$3 $\,\,$ \\
$\,\,$ 1/6$\,\,$ & $\,\,$ 1/8$\,\,$ & $\,\,$ 1 $\,\,$ & $\,\,$ 1/4 $\,\,$ \\
$\,\,$ 1/4$\,\,$ & $\,\,$ 1/3$\,\,$ & $\,\,$4$\,\,$ & $\,\,$ 1  $\,\,$ \\
\end{pmatrix},
\qquad
\lambda_{\max} =
4.0820,
\qquad
CR = 0.0309
\end{equation*}

\begin{equation*}
\mathbf{w}^{AMAST} =
\begin{pmatrix}
0.404837\\
\color{red} 0.402880\color{black} \\
0.050429\\
0.141854
\end{pmatrix}\end{equation*}
\begin{equation*}
\left[ \frac{{w}^{AMAST}_i}{{w}^{AMAST}_j} \right] =
\begin{pmatrix}
$\,\,$ 1 $\,\,$ & $\,\,$\color{red} 1.0049\color{black} $\,\,$ & $\,\,$8.0278$\,\,$ & $\,\,$2.8539$\,\,$ \\
$\,\,$\color{red} 0.9952\color{black} $\,\,$ & $\,\,$ 1 $\,\,$ & $\,\,$\color{red} 7.9890\color{black} $\,\,$ & $\,\,$\color{red} 2.8401\color{black}   $\,\,$ \\
$\,\,$0.1246$\,\,$ & $\,\,$\color{red} 0.1252\color{black} $\,\,$ & $\,\,$ 1 $\,\,$ & $\,\,$0.3555 $\,\,$ \\
$\,\,$0.3504$\,\,$ & $\,\,$\color{red} 0.3521\color{black} $\,\,$ & $\,\,$2.8129$\,\,$ & $\,\,$ 1  $\,\,$ \\
\end{pmatrix},
\end{equation*}

\begin{equation*}
\mathbf{w}^{\prime} =
\begin{pmatrix}
0.404612\\
0.403211\\
0.050401\\
0.141776
\end{pmatrix} =
0.999446\cdot
\begin{pmatrix}
0.404837\\
\color{gr} 0.403434\color{black} \\
0.050429\\
0.141854
\end{pmatrix},
\end{equation*}
\begin{equation*}
\left[ \frac{{w}^{\prime}_i}{{w}^{\prime}_j} \right] =
\begin{pmatrix}
$\,\,$ 1 $\,\,$ & $\,\,$\color{gr} 1.0035\color{black} $\,\,$ & $\,\,$8.0278$\,\,$ & $\,\,$2.8539$\,\,$ \\
$\,\,$\color{gr} 0.9965\color{black} $\,\,$ & $\,\,$ 1 $\,\,$ & $\,\,$\color{gr} \color{blue} 8\color{black} $\,\,$ & $\,\,$\color{gr} 2.8440\color{black}   $\,\,$ \\
$\,\,$0.1246$\,\,$ & $\,\,$\color{gr} \color{blue}  1/8\color{black} $\,\,$ & $\,\,$ 1 $\,\,$ & $\,\,$0.3555 $\,\,$ \\
$\,\,$0.3504$\,\,$ & $\,\,$\color{gr} 0.3516\color{black} $\,\,$ & $\,\,$2.8129$\,\,$ & $\,\,$ 1  $\,\,$ \\
\end{pmatrix},
\end{equation*}
\end{example}
\newpage
\begin{example}
\begin{equation*}
\mathbf{A} =
\begin{pmatrix}
$\,\,$ 1 $\,\,$ & $\,\,$1$\,\,$ & $\,\,$6$\,\,$ & $\,\,$4 $\,\,$ \\
$\,\,$ 1 $\,\,$ & $\,\,$ 1 $\,\,$ & $\,\,$9$\,\,$ & $\,\,$2 $\,\,$ \\
$\,\,$ 1/6$\,\,$ & $\,\,$ 1/9$\,\,$ & $\,\,$ 1 $\,\,$ & $\,\,$ 1/8 $\,\,$ \\
$\,\,$ 1/4$\,\,$ & $\,\,$ 1/2$\,\,$ & $\,\,$8$\,\,$ & $\,\,$ 1  $\,\,$ \\
\end{pmatrix},
\qquad
\lambda_{\max} =
4.2469,
\qquad
CR = 0.0931
\end{equation*}

\begin{equation*}
\mathbf{w}^{AMAST} =
\begin{pmatrix}
0.399939\\
\color{red} 0.367868\color{black} \\
0.042434\\
0.189759
\end{pmatrix}\end{equation*}
\begin{equation*}
\left[ \frac{{w}^{AMAST}_i}{{w}^{AMAST}_j} \right] =
\begin{pmatrix}
$\,\,$ 1 $\,\,$ & $\,\,$\color{red} 1.0872\color{black} $\,\,$ & $\,\,$9.4250$\,\,$ & $\,\,$2.1076$\,\,$ \\
$\,\,$\color{red} 0.9198\color{black} $\,\,$ & $\,\,$ 1 $\,\,$ & $\,\,$\color{red} 8.6692\color{black} $\,\,$ & $\,\,$\color{red} 1.9386\color{black}   $\,\,$ \\
$\,\,$0.1061$\,\,$ & $\,\,$\color{red} 0.1154\color{black} $\,\,$ & $\,\,$ 1 $\,\,$ & $\,\,$0.2236 $\,\,$ \\
$\,\,$0.4745$\,\,$ & $\,\,$\color{red} 0.5158\color{black} $\,\,$ & $\,\,$4.4719$\,\,$ & $\,\,$ 1  $\,\,$ \\
\end{pmatrix},
\end{equation*}

\begin{equation*}
\mathbf{w}^{\prime} =
\begin{pmatrix}
0.395333\\
0.375148\\
0.041945\\
0.187574
\end{pmatrix} =
0.988483\cdot
\begin{pmatrix}
0.399939\\
\color{gr} 0.379519\color{black} \\
0.042434\\
0.189759
\end{pmatrix},
\end{equation*}
\begin{equation*}
\left[ \frac{{w}^{\prime}_i}{{w}^{\prime}_j} \right] =
\begin{pmatrix}
$\,\,$ 1 $\,\,$ & $\,\,$\color{gr} 1.0538\color{black} $\,\,$ & $\,\,$9.4250$\,\,$ & $\,\,$2.1076$\,\,$ \\
$\,\,$\color{gr} 0.9489\color{black} $\,\,$ & $\,\,$ 1 $\,\,$ & $\,\,$\color{gr} 8.9438\color{black} $\,\,$ & $\,\,$\color{gr} \color{blue} 2\color{black}   $\,\,$ \\
$\,\,$0.1061$\,\,$ & $\,\,$\color{gr} 0.1118\color{black} $\,\,$ & $\,\,$ 1 $\,\,$ & $\,\,$0.2236 $\,\,$ \\
$\,\,$0.4745$\,\,$ & $\,\,$\color{gr} \color{blue}  1/2\color{black} $\,\,$ & $\,\,$4.4719$\,\,$ & $\,\,$ 1  $\,\,$ \\
\end{pmatrix},
\end{equation*}
\end{example}
\newpage
\begin{example}
\begin{equation*}
\mathbf{A} =
\begin{pmatrix}
$\,\,$ 1 $\,\,$ & $\,\,$1$\,\,$ & $\,\,$6$\,\,$ & $\,\,$5 $\,\,$ \\
$\,\,$ 1 $\,\,$ & $\,\,$ 1 $\,\,$ & $\,\,$8$\,\,$ & $\,\,$3 $\,\,$ \\
$\,\,$ 1/6$\,\,$ & $\,\,$ 1/8$\,\,$ & $\,\,$ 1 $\,\,$ & $\,\,$ 1/4 $\,\,$ \\
$\,\,$ 1/5$\,\,$ & $\,\,$ 1/3$\,\,$ & $\,\,$4$\,\,$ & $\,\,$ 1  $\,\,$ \\
\end{pmatrix},
\qquad
\lambda_{\max} =
4.1252,
\qquad
CR = 0.0472
\end{equation*}

\begin{equation*}
\mathbf{w}^{AMAST} =
\begin{pmatrix}
0.420413\\
\color{red} 0.395802\color{black} \\
0.050085\\
0.133699
\end{pmatrix}\end{equation*}
\begin{equation*}
\left[ \frac{{w}^{AMAST}_i}{{w}^{AMAST}_j} \right] =
\begin{pmatrix}
$\,\,$ 1 $\,\,$ & $\,\,$\color{red} 1.0622\color{black} $\,\,$ & $\,\,$8.3939$\,\,$ & $\,\,$3.1445$\,\,$ \\
$\,\,$\color{red} 0.9415\color{black} $\,\,$ & $\,\,$ 1 $\,\,$ & $\,\,$\color{red} 7.9026\color{black} $\,\,$ & $\,\,$\color{red} 2.9604\color{black}   $\,\,$ \\
$\,\,$0.1191$\,\,$ & $\,\,$\color{red} 0.1265\color{black} $\,\,$ & $\,\,$ 1 $\,\,$ & $\,\,$0.3746 $\,\,$ \\
$\,\,$0.3180$\,\,$ & $\,\,$\color{red} 0.3378\color{black} $\,\,$ & $\,\,$2.6694$\,\,$ & $\,\,$ 1  $\,\,$ \\
\end{pmatrix},
\end{equation*}

\begin{equation*}
\mathbf{w}^{\prime} =
\begin{pmatrix}
0.418372\\
0.398737\\
0.049842\\
0.133050
\end{pmatrix} =
0.995143\cdot
\begin{pmatrix}
0.420413\\
\color{gr} 0.400683\color{black} \\
0.050085\\
0.133699
\end{pmatrix},
\end{equation*}
\begin{equation*}
\left[ \frac{{w}^{\prime}_i}{{w}^{\prime}_j} \right] =
\begin{pmatrix}
$\,\,$ 1 $\,\,$ & $\,\,$\color{gr} 1.0492\color{black} $\,\,$ & $\,\,$8.3939$\,\,$ & $\,\,$3.1445$\,\,$ \\
$\,\,$\color{gr} 0.9531\color{black} $\,\,$ & $\,\,$ 1 $\,\,$ & $\,\,$\color{gr} \color{blue} 8\color{black} $\,\,$ & $\,\,$\color{gr} 2.9969\color{black}   $\,\,$ \\
$\,\,$0.1191$\,\,$ & $\,\,$\color{gr} \color{blue}  1/8\color{black} $\,\,$ & $\,\,$ 1 $\,\,$ & $\,\,$0.3746 $\,\,$ \\
$\,\,$0.3180$\,\,$ & $\,\,$\color{gr} 0.3337\color{black} $\,\,$ & $\,\,$2.6694$\,\,$ & $\,\,$ 1  $\,\,$ \\
\end{pmatrix},
\end{equation*}
\end{example}
\newpage
\begin{example}
\begin{equation*}
\mathbf{A} =
\begin{pmatrix}
$\,\,$ 1 $\,\,$ & $\,\,$1$\,\,$ & $\,\,$6$\,\,$ & $\,\,$5 $\,\,$ \\
$\,\,$ 1 $\,\,$ & $\,\,$ 1 $\,\,$ & $\,\,$9$\,\,$ & $\,\,$3 $\,\,$ \\
$\,\,$ 1/6$\,\,$ & $\,\,$ 1/9$\,\,$ & $\,\,$ 1 $\,\,$ & $\,\,$ 1/5 $\,\,$ \\
$\,\,$ 1/5$\,\,$ & $\,\,$ 1/3$\,\,$ & $\,\,$5$\,\,$ & $\,\,$ 1  $\,\,$ \\
\end{pmatrix},
\qquad
\lambda_{\max} =
4.1758,
\qquad
CR = 0.0663
\end{equation*}

\begin{equation*}
\mathbf{w}^{AMAST} =
\begin{pmatrix}
0.413811\\
\color{red} 0.399651\color{black} \\
0.046007\\
0.140532
\end{pmatrix}\end{equation*}
\begin{equation*}
\left[ \frac{{w}^{AMAST}_i}{{w}^{AMAST}_j} \right] =
\begin{pmatrix}
$\,\,$ 1 $\,\,$ & $\,\,$\color{red} 1.0354\color{black} $\,\,$ & $\,\,$8.9946$\,\,$ & $\,\,$2.9446$\,\,$ \\
$\,\,$\color{red} 0.9658\color{black} $\,\,$ & $\,\,$ 1 $\,\,$ & $\,\,$\color{red} 8.6868\color{black} $\,\,$ & $\,\,$\color{red} 2.8439\color{black}   $\,\,$ \\
$\,\,$0.1112$\,\,$ & $\,\,$\color{red} 0.1151\color{black} $\,\,$ & $\,\,$ 1 $\,\,$ & $\,\,$0.3274 $\,\,$ \\
$\,\,$0.3396$\,\,$ & $\,\,$\color{red} 0.3516\color{black} $\,\,$ & $\,\,$3.0546$\,\,$ & $\,\,$ 1  $\,\,$ \\
\end{pmatrix},
\end{equation*}

\begin{equation*}
\mathbf{w}^{\prime} =
\begin{pmatrix}
0.408033\\
0.408033\\
0.045364\\
0.138569
\end{pmatrix} =
0.986038\cdot
\begin{pmatrix}
0.413811\\
\color{gr} 0.413811\color{black} \\
0.046007\\
0.140532
\end{pmatrix},
\end{equation*}
\begin{equation*}
\left[ \frac{{w}^{\prime}_i}{{w}^{\prime}_j} \right] =
\begin{pmatrix}
$\,\,$ 1 $\,\,$ & $\,\,$\color{gr} \color{blue} 1\color{black} $\,\,$ & $\,\,$8.9946$\,\,$ & $\,\,$2.9446$\,\,$ \\
$\,\,$\color{gr} \color{blue} 1\color{black} $\,\,$ & $\,\,$ 1 $\,\,$ & $\,\,$\color{gr} 8.9946\color{black} $\,\,$ & $\,\,$\color{gr} 2.9446\color{black}   $\,\,$ \\
$\,\,$0.1112$\,\,$ & $\,\,$\color{gr} 0.1112\color{black} $\,\,$ & $\,\,$ 1 $\,\,$ & $\,\,$0.3274 $\,\,$ \\
$\,\,$0.3396$\,\,$ & $\,\,$\color{gr} 0.3396\color{black} $\,\,$ & $\,\,$3.0546$\,\,$ & $\,\,$ 1  $\,\,$ \\
\end{pmatrix},
\end{equation*}
\end{example}
\newpage
\begin{example}
\begin{equation*}
\mathbf{A} =
\begin{pmatrix}
$\,\,$ 1 $\,\,$ & $\,\,$1$\,\,$ & $\,\,$6$\,\,$ & $\,\,$5 $\,\,$ \\
$\,\,$ 1 $\,\,$ & $\,\,$ 1 $\,\,$ & $\,\,$9$\,\,$ & $\,\,$3 $\,\,$ \\
$\,\,$ 1/6$\,\,$ & $\,\,$ 1/9$\,\,$ & $\,\,$ 1 $\,\,$ & $\,\,$ 1/6 $\,\,$ \\
$\,\,$ 1/5$\,\,$ & $\,\,$ 1/3$\,\,$ & $\,\,$6$\,\,$ & $\,\,$ 1  $\,\,$ \\
\end{pmatrix},
\qquad
\lambda_{\max} =
4.2277,
\qquad
CR = 0.0859
\end{equation*}

\begin{equation*}
\mathbf{w}^{AMAST} =
\begin{pmatrix}
0.411200\\
\color{red} 0.396412\color{black} \\
0.044394\\
0.147995
\end{pmatrix}\end{equation*}
\begin{equation*}
\left[ \frac{{w}^{AMAST}_i}{{w}^{AMAST}_j} \right] =
\begin{pmatrix}
$\,\,$ 1 $\,\,$ & $\,\,$\color{red} 1.0373\color{black} $\,\,$ & $\,\,$9.2626$\,\,$ & $\,\,$2.7785$\,\,$ \\
$\,\,$\color{red} 0.9640\color{black} $\,\,$ & $\,\,$ 1 $\,\,$ & $\,\,$\color{red} 8.9295\color{black} $\,\,$ & $\,\,$\color{red} 2.6785\color{black}   $\,\,$ \\
$\,\,$0.1080$\,\,$ & $\,\,$\color{red} 0.1120\color{black} $\,\,$ & $\,\,$ 1 $\,\,$ & $\,\,$0.3000 $\,\,$ \\
$\,\,$0.3599$\,\,$ & $\,\,$\color{red} 0.3733\color{black} $\,\,$ & $\,\,$3.3337$\,\,$ & $\,\,$ 1  $\,\,$ \\
\end{pmatrix},
\end{equation*}

\begin{equation*}
\mathbf{w}^{\prime} =
\begin{pmatrix}
0.409916\\
0.398296\\
0.044255\\
0.147533
\end{pmatrix} =
0.996879\cdot
\begin{pmatrix}
0.411200\\
\color{gr} 0.399543\color{black} \\
0.044394\\
0.147995
\end{pmatrix},
\end{equation*}
\begin{equation*}
\left[ \frac{{w}^{\prime}_i}{{w}^{\prime}_j} \right] =
\begin{pmatrix}
$\,\,$ 1 $\,\,$ & $\,\,$\color{gr} 1.0292\color{black} $\,\,$ & $\,\,$9.2626$\,\,$ & $\,\,$2.7785$\,\,$ \\
$\,\,$\color{gr} 0.9717\color{black} $\,\,$ & $\,\,$ 1 $\,\,$ & $\,\,$\color{gr} \color{blue} 9\color{black} $\,\,$ & $\,\,$\color{gr} 2.6997\color{black}   $\,\,$ \\
$\,\,$0.1080$\,\,$ & $\,\,$\color{gr} \color{blue}  1/9\color{black} $\,\,$ & $\,\,$ 1 $\,\,$ & $\,\,$0.3000 $\,\,$ \\
$\,\,$0.3599$\,\,$ & $\,\,$\color{gr} 0.3704\color{black} $\,\,$ & $\,\,$3.3337$\,\,$ & $\,\,$ 1  $\,\,$ \\
\end{pmatrix},
\end{equation*}
\end{example}
\newpage
\begin{example}
\begin{equation*}
\mathbf{A} =
\begin{pmatrix}
$\,\,$ 1 $\,\,$ & $\,\,$1$\,\,$ & $\,\,$6$\,\,$ & $\,\,$6 $\,\,$ \\
$\,\,$ 1 $\,\,$ & $\,\,$ 1 $\,\,$ & $\,\,$8$\,\,$ & $\,\,$4 $\,\,$ \\
$\,\,$ 1/6$\,\,$ & $\,\,$ 1/8$\,\,$ & $\,\,$ 1 $\,\,$ & $\,\,$ 1/3 $\,\,$ \\
$\,\,$ 1/6$\,\,$ & $\,\,$ 1/4$\,\,$ & $\,\,$3$\,\,$ & $\,\,$ 1  $\,\,$ \\
\end{pmatrix},
\qquad
\lambda_{\max} =
4.1031,
\qquad
CR = 0.0389
\end{equation*}

\begin{equation*}
\mathbf{w}^{AMAST} =
\begin{pmatrix}
0.427407\\
\color{red} 0.413413\color{black} \\
0.052042\\
0.107138
\end{pmatrix}\end{equation*}
\begin{equation*}
\left[ \frac{{w}^{AMAST}_i}{{w}^{AMAST}_j} \right] =
\begin{pmatrix}
$\,\,$ 1 $\,\,$ & $\,\,$\color{red} 1.0338\color{black} $\,\,$ & $\,\,$8.2127$\,\,$ & $\,\,$3.9893$\,\,$ \\
$\,\,$\color{red} 0.9673\color{black} $\,\,$ & $\,\,$ 1 $\,\,$ & $\,\,$\color{red} 7.9438\color{black} $\,\,$ & $\,\,$\color{red} 3.8587\color{black}   $\,\,$ \\
$\,\,$0.1218$\,\,$ & $\,\,$\color{red} 0.1259\color{black} $\,\,$ & $\,\,$ 1 $\,\,$ & $\,\,$0.4857 $\,\,$ \\
$\,\,$0.2507$\,\,$ & $\,\,$\color{red} 0.2592\color{black} $\,\,$ & $\,\,$2.0587$\,\,$ & $\,\,$ 1  $\,\,$ \\
\end{pmatrix},
\end{equation*}

\begin{equation*}
\mathbf{w}^{\prime} =
\begin{pmatrix}
0.426160\\
0.415124\\
0.051890\\
0.106826
\end{pmatrix} =
0.997084\cdot
\begin{pmatrix}
0.427407\\
\color{gr} 0.416338\color{black} \\
0.052042\\
0.107138
\end{pmatrix},
\end{equation*}
\begin{equation*}
\left[ \frac{{w}^{\prime}_i}{{w}^{\prime}_j} \right] =
\begin{pmatrix}
$\,\,$ 1 $\,\,$ & $\,\,$\color{gr} 1.0266\color{black} $\,\,$ & $\,\,$8.2127$\,\,$ & $\,\,$3.9893$\,\,$ \\
$\,\,$\color{gr} 0.9741\color{black} $\,\,$ & $\,\,$ 1 $\,\,$ & $\,\,$\color{gr} \color{blue} 8\color{black} $\,\,$ & $\,\,$\color{gr} 3.8860\color{black}   $\,\,$ \\
$\,\,$0.1218$\,\,$ & $\,\,$\color{gr} \color{blue}  1/8\color{black} $\,\,$ & $\,\,$ 1 $\,\,$ & $\,\,$0.4857 $\,\,$ \\
$\,\,$0.2507$\,\,$ & $\,\,$\color{gr} 0.2573\color{black} $\,\,$ & $\,\,$2.0587$\,\,$ & $\,\,$ 1  $\,\,$ \\
\end{pmatrix},
\end{equation*}
\end{example}
\newpage
\begin{example}
\begin{equation*}
\mathbf{A} =
\begin{pmatrix}
$\,\,$ 1 $\,\,$ & $\,\,$1$\,\,$ & $\,\,$6$\,\,$ & $\,\,$6 $\,\,$ \\
$\,\,$ 1 $\,\,$ & $\,\,$ 1 $\,\,$ & $\,\,$9$\,\,$ & $\,\,$3 $\,\,$ \\
$\,\,$ 1/6$\,\,$ & $\,\,$ 1/9$\,\,$ & $\,\,$ 1 $\,\,$ & $\,\,$ 1/5 $\,\,$ \\
$\,\,$ 1/6$\,\,$ & $\,\,$ 1/3$\,\,$ & $\,\,$5$\,\,$ & $\,\,$ 1  $\,\,$ \\
\end{pmatrix},
\qquad
\lambda_{\max} =
4.2277,
\qquad
CR = 0.0859
\end{equation*}

\begin{equation*}
\mathbf{w}^{AMAST} =
\begin{pmatrix}
0.425869\\
\color{red} 0.393789\color{black} \\
0.045849\\
0.134493
\end{pmatrix}\end{equation*}
\begin{equation*}
\left[ \frac{{w}^{AMAST}_i}{{w}^{AMAST}_j} \right] =
\begin{pmatrix}
$\,\,$ 1 $\,\,$ & $\,\,$\color{red} 1.0815\color{black} $\,\,$ & $\,\,$9.2884$\,\,$ & $\,\,$3.1665$\,\,$ \\
$\,\,$\color{red} 0.9247\color{black} $\,\,$ & $\,\,$ 1 $\,\,$ & $\,\,$\color{red} 8.5887\color{black} $\,\,$ & $\,\,$\color{red} 2.9280\color{black}   $\,\,$ \\
$\,\,$0.1077$\,\,$ & $\,\,$\color{red} 0.1164\color{black} $\,\,$ & $\,\,$ 1 $\,\,$ & $\,\,$0.3409 $\,\,$ \\
$\,\,$0.3158$\,\,$ & $\,\,$\color{red} 0.3415\color{black} $\,\,$ & $\,\,$2.9334$\,\,$ & $\,\,$ 1  $\,\,$ \\
\end{pmatrix},
\end{equation*}

\begin{equation*}
\mathbf{w}^{\prime} =
\begin{pmatrix}
0.421782\\
0.399607\\
0.045409\\
0.133202
\end{pmatrix} =
0.990403\cdot
\begin{pmatrix}
0.425869\\
\color{gr} 0.403479\color{black} \\
0.045849\\
0.134493
\end{pmatrix},
\end{equation*}
\begin{equation*}
\left[ \frac{{w}^{\prime}_i}{{w}^{\prime}_j} \right] =
\begin{pmatrix}
$\,\,$ 1 $\,\,$ & $\,\,$\color{gr} 1.0555\color{black} $\,\,$ & $\,\,$9.2884$\,\,$ & $\,\,$3.1665$\,\,$ \\
$\,\,$\color{gr} 0.9474\color{black} $\,\,$ & $\,\,$ 1 $\,\,$ & $\,\,$\color{gr} 8.8001\color{black} $\,\,$ & $\,\,$\color{gr} \color{blue} 3\color{black}   $\,\,$ \\
$\,\,$0.1077$\,\,$ & $\,\,$\color{gr} 0.1136\color{black} $\,\,$ & $\,\,$ 1 $\,\,$ & $\,\,$0.3409 $\,\,$ \\
$\,\,$0.3158$\,\,$ & $\,\,$\color{gr} \color{blue}  1/3\color{black} $\,\,$ & $\,\,$2.9334$\,\,$ & $\,\,$ 1  $\,\,$ \\
\end{pmatrix},
\end{equation*}
\end{example}
\newpage
\begin{example}
\begin{equation*}
\mathbf{A} =
\begin{pmatrix}
$\,\,$ 1 $\,\,$ & $\,\,$1$\,\,$ & $\,\,$6$\,\,$ & $\,\,$6 $\,\,$ \\
$\,\,$ 1 $\,\,$ & $\,\,$ 1 $\,\,$ & $\,\,$9$\,\,$ & $\,\,$4 $\,\,$ \\
$\,\,$ 1/6$\,\,$ & $\,\,$ 1/9$\,\,$ & $\,\,$ 1 $\,\,$ & $\,\,$ 1/3 $\,\,$ \\
$\,\,$ 1/6$\,\,$ & $\,\,$ 1/4$\,\,$ & $\,\,$3$\,\,$ & $\,\,$ 1  $\,\,$ \\
\end{pmatrix},
\qquad
\lambda_{\max} =
4.1031,
\qquad
CR = 0.0389
\end{equation*}

\begin{equation*}
\mathbf{w}^{AMAST} =
\begin{pmatrix}
0.423258\\
\color{red} 0.420962\color{black} \\
0.050028\\
0.105752
\end{pmatrix}\end{equation*}
\begin{equation*}
\left[ \frac{{w}^{AMAST}_i}{{w}^{AMAST}_j} \right] =
\begin{pmatrix}
$\,\,$ 1 $\,\,$ & $\,\,$\color{red} 1.0055\color{black} $\,\,$ & $\,\,$8.4605$\,\,$ & $\,\,$4.0024$\,\,$ \\
$\,\,$\color{red} 0.9946\color{black} $\,\,$ & $\,\,$ 1 $\,\,$ & $\,\,$\color{red} 8.4146\color{black} $\,\,$ & $\,\,$\color{red} 3.9807\color{black}   $\,\,$ \\
$\,\,$0.1182$\,\,$ & $\,\,$\color{red} 0.1188\color{black} $\,\,$ & $\,\,$ 1 $\,\,$ & $\,\,$0.4731 $\,\,$ \\
$\,\,$0.2499$\,\,$ & $\,\,$\color{red} 0.2512\color{black} $\,\,$ & $\,\,$2.1139$\,\,$ & $\,\,$ 1  $\,\,$ \\
\end{pmatrix},
\end{equation*}

\begin{equation*}
\mathbf{w}^{\prime} =
\begin{pmatrix}
0.422395\\
0.422144\\
0.049926\\
0.105536
\end{pmatrix} =
0.997959\cdot
\begin{pmatrix}
0.423258\\
\color{gr} 0.423007\color{black} \\
0.050028\\
0.105752
\end{pmatrix},
\end{equation*}
\begin{equation*}
\left[ \frac{{w}^{\prime}_i}{{w}^{\prime}_j} \right] =
\begin{pmatrix}
$\,\,$ 1 $\,\,$ & $\,\,$\color{gr} 1.0006\color{black} $\,\,$ & $\,\,$8.4605$\,\,$ & $\,\,$4.0024$\,\,$ \\
$\,\,$\color{gr} 0.9994\color{black} $\,\,$ & $\,\,$ 1 $\,\,$ & $\,\,$\color{gr} 8.4554\color{black} $\,\,$ & $\,\,$\color{gr} \color{blue} 4\color{black}   $\,\,$ \\
$\,\,$0.1182$\,\,$ & $\,\,$\color{gr} 0.1183\color{black} $\,\,$ & $\,\,$ 1 $\,\,$ & $\,\,$0.4731 $\,\,$ \\
$\,\,$0.2499$\,\,$ & $\,\,$\color{gr} \color{blue}  1/4\color{black} $\,\,$ & $\,\,$2.1139$\,\,$ & $\,\,$ 1  $\,\,$ \\
\end{pmatrix},
\end{equation*}
\end{example}
\newpage
\begin{example}
\begin{equation*}
\mathbf{A} =
\begin{pmatrix}
$\,\,$ 1 $\,\,$ & $\,\,$1$\,\,$ & $\,\,$6$\,\,$ & $\,\,$6 $\,\,$ \\
$\,\,$ 1 $\,\,$ & $\,\,$ 1 $\,\,$ & $\,\,$9$\,\,$ & $\,\,$4 $\,\,$ \\
$\,\,$ 1/6$\,\,$ & $\,\,$ 1/9$\,\,$ & $\,\,$ 1 $\,\,$ & $\,\,$ 1/4 $\,\,$ \\
$\,\,$ 1/6$\,\,$ & $\,\,$ 1/4$\,\,$ & $\,\,$4$\,\,$ & $\,\,$ 1  $\,\,$ \\
\end{pmatrix},
\qquad
\lambda_{\max} =
4.1664,
\qquad
CR = 0.0627
\end{equation*}

\begin{equation*}
\mathbf{w}^{AMAST} =
\begin{pmatrix}
0.420471\\
\color{red} 0.417177\color{black} \\
0.047237\\
0.115115
\end{pmatrix}\end{equation*}
\begin{equation*}
\left[ \frac{{w}^{AMAST}_i}{{w}^{AMAST}_j} \right] =
\begin{pmatrix}
$\,\,$ 1 $\,\,$ & $\,\,$\color{red} 1.0079\color{black} $\,\,$ & $\,\,$8.9013$\,\,$ & $\,\,$3.6526$\,\,$ \\
$\,\,$\color{red} 0.9922\color{black} $\,\,$ & $\,\,$ 1 $\,\,$ & $\,\,$\color{red} 8.8316\color{black} $\,\,$ & $\,\,$\color{red} 3.6240\color{black}   $\,\,$ \\
$\,\,$0.1123$\,\,$ & $\,\,$\color{red} 0.1132\color{black} $\,\,$ & $\,\,$ 1 $\,\,$ & $\,\,$0.4103 $\,\,$ \\
$\,\,$0.2738$\,\,$ & $\,\,$\color{red} 0.2759\color{black} $\,\,$ & $\,\,$2.4370$\,\,$ & $\,\,$ 1  $\,\,$ \\
\end{pmatrix},
\end{equation*}

\begin{equation*}
\mathbf{w}^{\prime} =
\begin{pmatrix}
0.419091\\
0.419091\\
0.047082\\
0.114737
\end{pmatrix} =
0.996716\cdot
\begin{pmatrix}
0.420471\\
\color{gr} 0.420471\color{black} \\
0.047237\\
0.115115
\end{pmatrix},
\end{equation*}
\begin{equation*}
\left[ \frac{{w}^{\prime}_i}{{w}^{\prime}_j} \right] =
\begin{pmatrix}
$\,\,$ 1 $\,\,$ & $\,\,$\color{gr} \color{blue} 1\color{black} $\,\,$ & $\,\,$8.9013$\,\,$ & $\,\,$3.6526$\,\,$ \\
$\,\,$\color{gr} \color{blue} 1\color{black} $\,\,$ & $\,\,$ 1 $\,\,$ & $\,\,$\color{gr} 8.9013\color{black} $\,\,$ & $\,\,$\color{gr} 3.6526\color{black}   $\,\,$ \\
$\,\,$0.1123$\,\,$ & $\,\,$\color{gr} 0.1123\color{black} $\,\,$ & $\,\,$ 1 $\,\,$ & $\,\,$0.4103 $\,\,$ \\
$\,\,$0.2738$\,\,$ & $\,\,$\color{gr} 0.2738\color{black} $\,\,$ & $\,\,$2.4370$\,\,$ & $\,\,$ 1  $\,\,$ \\
\end{pmatrix},
\end{equation*}
\end{example}
\newpage
\begin{example}
\begin{equation*}
\mathbf{A} =
\begin{pmatrix}
$\,\,$ 1 $\,\,$ & $\,\,$1$\,\,$ & $\,\,$6$\,\,$ & $\,\,$7 $\,\,$ \\
$\,\,$ 1 $\,\,$ & $\,\,$ 1 $\,\,$ & $\,\,$8$\,\,$ & $\,\,$4 $\,\,$ \\
$\,\,$ 1/6$\,\,$ & $\,\,$ 1/8$\,\,$ & $\,\,$ 1 $\,\,$ & $\,\,$ 1/3 $\,\,$ \\
$\,\,$ 1/7$\,\,$ & $\,\,$ 1/4$\,\,$ & $\,\,$3$\,\,$ & $\,\,$ 1  $\,\,$ \\
\end{pmatrix},
\qquad
\lambda_{\max} =
4.1365,
\qquad
CR = 0.0515
\end{equation*}

\begin{equation*}
\mathbf{w}^{AMAST} =
\begin{pmatrix}
0.437614\\
\color{red} 0.407776\color{black} \\
0.051729\\
0.102881
\end{pmatrix}\end{equation*}
\begin{equation*}
\left[ \frac{{w}^{AMAST}_i}{{w}^{AMAST}_j} \right] =
\begin{pmatrix}
$\,\,$ 1 $\,\,$ & $\,\,$\color{red} 1.0732\color{black} $\,\,$ & $\,\,$8.4597$\,\,$ & $\,\,$4.2536$\,\,$ \\
$\,\,$\color{red} 0.9318\color{black} $\,\,$ & $\,\,$ 1 $\,\,$ & $\,\,$\color{red} 7.8829\color{black} $\,\,$ & $\,\,$\color{red} 3.9636\color{black}   $\,\,$ \\
$\,\,$0.1182$\,\,$ & $\,\,$\color{red} 0.1269\color{black} $\,\,$ & $\,\,$ 1 $\,\,$ & $\,\,$0.5028 $\,\,$ \\
$\,\,$0.2351$\,\,$ & $\,\,$\color{red} 0.2523\color{black} $\,\,$ & $\,\,$1.9888$\,\,$ & $\,\,$ 1  $\,\,$ \\
\end{pmatrix},
\end{equation*}

\begin{equation*}
\mathbf{w}^{\prime} =
\begin{pmatrix}
0.435981\\
0.409986\\
0.051536\\
0.102497
\end{pmatrix} =
0.996267\cdot
\begin{pmatrix}
0.437614\\
\color{gr} 0.411523\color{black} \\
0.051729\\
0.102881
\end{pmatrix},
\end{equation*}
\begin{equation*}
\left[ \frac{{w}^{\prime}_i}{{w}^{\prime}_j} \right] =
\begin{pmatrix}
$\,\,$ 1 $\,\,$ & $\,\,$\color{gr} 1.0634\color{black} $\,\,$ & $\,\,$8.4597$\,\,$ & $\,\,$4.2536$\,\,$ \\
$\,\,$\color{gr} 0.9404\color{black} $\,\,$ & $\,\,$ 1 $\,\,$ & $\,\,$\color{gr} 7.9553\color{black} $\,\,$ & $\,\,$\color{gr} \color{blue} 4\color{black}   $\,\,$ \\
$\,\,$0.1182$\,\,$ & $\,\,$\color{gr} 0.1257\color{black} $\,\,$ & $\,\,$ 1 $\,\,$ & $\,\,$0.5028 $\,\,$ \\
$\,\,$0.2351$\,\,$ & $\,\,$\color{gr} \color{blue}  1/4\color{black} $\,\,$ & $\,\,$1.9888$\,\,$ & $\,\,$ 1  $\,\,$ \\
\end{pmatrix},
\end{equation*}
\end{example}
\newpage
\begin{example}
\begin{equation*}
\mathbf{A} =
\begin{pmatrix}
$\,\,$ 1 $\,\,$ & $\,\,$1$\,\,$ & $\,\,$6$\,\,$ & $\,\,$7 $\,\,$ \\
$\,\,$ 1 $\,\,$ & $\,\,$ 1 $\,\,$ & $\,\,$9$\,\,$ & $\,\,$4 $\,\,$ \\
$\,\,$ 1/6$\,\,$ & $\,\,$ 1/9$\,\,$ & $\,\,$ 1 $\,\,$ & $\,\,$ 1/4 $\,\,$ \\
$\,\,$ 1/7$\,\,$ & $\,\,$ 1/4$\,\,$ & $\,\,$4$\,\,$ & $\,\,$ 1  $\,\,$ \\
\end{pmatrix},
\qquad
\lambda_{\max} =
4.2065,
\qquad
CR = 0.0779
\end{equation*}

\begin{equation*}
\mathbf{w}^{AMAST} =
\begin{pmatrix}
0.430552\\
\color{red} 0.411585\color{black} \\
0.047056\\
0.110807
\end{pmatrix}\end{equation*}
\begin{equation*}
\left[ \frac{{w}^{AMAST}_i}{{w}^{AMAST}_j} \right] =
\begin{pmatrix}
$\,\,$ 1 $\,\,$ & $\,\,$\color{red} 1.0461\color{black} $\,\,$ & $\,\,$9.1497$\,\,$ & $\,\,$3.8856$\,\,$ \\
$\,\,$\color{red} 0.9559\color{black} $\,\,$ & $\,\,$ 1 $\,\,$ & $\,\,$\color{red} 8.7466\color{black} $\,\,$ & $\,\,$\color{red} 3.7144\color{black}   $\,\,$ \\
$\,\,$0.1093$\,\,$ & $\,\,$\color{red} 0.1143\color{black} $\,\,$ & $\,\,$ 1 $\,\,$ & $\,\,$0.4247 $\,\,$ \\
$\,\,$0.2574$\,\,$ & $\,\,$\color{red} 0.2692\color{black} $\,\,$ & $\,\,$2.3548$\,\,$ & $\,\,$ 1  $\,\,$ \\
\end{pmatrix},
\end{equation*}

\begin{equation*}
\mathbf{w}^{\prime} =
\begin{pmatrix}
0.425479\\
0.418517\\
0.046502\\
0.109502
\end{pmatrix} =
0.988218\cdot
\begin{pmatrix}
0.430552\\
\color{gr} 0.423507\color{black} \\
0.047056\\
0.110807
\end{pmatrix},
\end{equation*}
\begin{equation*}
\left[ \frac{{w}^{\prime}_i}{{w}^{\prime}_j} \right] =
\begin{pmatrix}
$\,\,$ 1 $\,\,$ & $\,\,$\color{gr} 1.0166\color{black} $\,\,$ & $\,\,$9.1497$\,\,$ & $\,\,$3.8856$\,\,$ \\
$\,\,$\color{gr} 0.9836\color{black} $\,\,$ & $\,\,$ 1 $\,\,$ & $\,\,$\color{gr} \color{blue} 9\color{black} $\,\,$ & $\,\,$\color{gr} 3.8220\color{black}   $\,\,$ \\
$\,\,$0.1093$\,\,$ & $\,\,$\color{gr} \color{blue}  1/9\color{black} $\,\,$ & $\,\,$ 1 $\,\,$ & $\,\,$0.4247 $\,\,$ \\
$\,\,$0.2574$\,\,$ & $\,\,$\color{gr} 0.2616\color{black} $\,\,$ & $\,\,$2.3548$\,\,$ & $\,\,$ 1  $\,\,$ \\
\end{pmatrix},
\end{equation*}
\end{example}
\newpage
\begin{example}
\begin{equation*}
\mathbf{A} =
\begin{pmatrix}
$\,\,$ 1 $\,\,$ & $\,\,$1$\,\,$ & $\,\,$6$\,\,$ & $\,\,$8 $\,\,$ \\
$\,\,$ 1 $\,\,$ & $\,\,$ 1 $\,\,$ & $\,\,$8$\,\,$ & $\,\,$6 $\,\,$ \\
$\,\,$ 1/6$\,\,$ & $\,\,$ 1/8$\,\,$ & $\,\,$ 1 $\,\,$ & $\,\,$ 1/2 $\,\,$ \\
$\,\,$ 1/8$\,\,$ & $\,\,$ 1/6$\,\,$ & $\,\,$2$\,\,$ & $\,\,$ 1  $\,\,$ \\
\end{pmatrix},
\qquad
\lambda_{\max} =
4.0820,
\qquad
CR = 0.0309
\end{equation*}

\begin{equation*}
\mathbf{w}^{AMAST} =
\begin{pmatrix}
0.435112\\
\color{red} 0.433739\color{black} \\
0.054304\\
0.076845
\end{pmatrix}\end{equation*}
\begin{equation*}
\left[ \frac{{w}^{AMAST}_i}{{w}^{AMAST}_j} \right] =
\begin{pmatrix}
$\,\,$ 1 $\,\,$ & $\,\,$\color{red} 1.0032\color{black} $\,\,$ & $\,\,$8.0125$\,\,$ & $\,\,$5.6622$\,\,$ \\
$\,\,$\color{red} 0.9968\color{black} $\,\,$ & $\,\,$ 1 $\,\,$ & $\,\,$\color{red} 7.9872\color{black} $\,\,$ & $\,\,$\color{red} 5.6444\color{black}   $\,\,$ \\
$\,\,$0.1248$\,\,$ & $\,\,$\color{red} 0.1252\color{black} $\,\,$ & $\,\,$ 1 $\,\,$ & $\,\,$0.7067 $\,\,$ \\
$\,\,$0.1766$\,\,$ & $\,\,$\color{red} 0.1772\color{black} $\,\,$ & $\,\,$1.4151$\,\,$ & $\,\,$ 1  $\,\,$ \\
\end{pmatrix},
\end{equation*}

\begin{equation*}
\mathbf{w}^{\prime} =
\begin{pmatrix}
0.434809\\
0.434133\\
0.054267\\
0.076791
\end{pmatrix} =
0.999305\cdot
\begin{pmatrix}
0.435112\\
\color{gr} 0.434435\color{black} \\
0.054304\\
0.076845
\end{pmatrix},
\end{equation*}
\begin{equation*}
\left[ \frac{{w}^{\prime}_i}{{w}^{\prime}_j} \right] =
\begin{pmatrix}
$\,\,$ 1 $\,\,$ & $\,\,$\color{gr} 1.0016\color{black} $\,\,$ & $\,\,$8.0125$\,\,$ & $\,\,$5.6622$\,\,$ \\
$\,\,$\color{gr} 0.9984\color{black} $\,\,$ & $\,\,$ 1 $\,\,$ & $\,\,$\color{gr} \color{blue} 8\color{black} $\,\,$ & $\,\,$\color{gr} 5.6534\color{black}   $\,\,$ \\
$\,\,$0.1248$\,\,$ & $\,\,$\color{gr} \color{blue}  1/8\color{black} $\,\,$ & $\,\,$ 1 $\,\,$ & $\,\,$0.7067 $\,\,$ \\
$\,\,$0.1766$\,\,$ & $\,\,$\color{gr} 0.1769\color{black} $\,\,$ & $\,\,$1.4151$\,\,$ & $\,\,$ 1  $\,\,$ \\
\end{pmatrix},
\end{equation*}
\end{example}
\newpage
\begin{example}
\begin{equation*}
\mathbf{A} =
\begin{pmatrix}
$\,\,$ 1 $\,\,$ & $\,\,$1$\,\,$ & $\,\,$6$\,\,$ & $\,\,$8 $\,\,$ \\
$\,\,$ 1 $\,\,$ & $\,\,$ 1 $\,\,$ & $\,\,$9$\,\,$ & $\,\,$4 $\,\,$ \\
$\,\,$ 1/6$\,\,$ & $\,\,$ 1/9$\,\,$ & $\,\,$ 1 $\,\,$ & $\,\,$ 1/4 $\,\,$ \\
$\,\,$ 1/8$\,\,$ & $\,\,$ 1/4$\,\,$ & $\,\,$4$\,\,$ & $\,\,$ 1  $\,\,$ \\
\end{pmatrix},
\qquad
\lambda_{\max} =
4.2469,
\qquad
CR = 0.0931
\end{equation*}

\begin{equation*}
\mathbf{w}^{AMAST} =
\begin{pmatrix}
0.438946\\
\color{red} 0.406783\color{black} \\
0.046905\\
0.107366
\end{pmatrix}\end{equation*}
\begin{equation*}
\left[ \frac{{w}^{AMAST}_i}{{w}^{AMAST}_j} \right] =
\begin{pmatrix}
$\,\,$ 1 $\,\,$ & $\,\,$\color{red} 1.0791\color{black} $\,\,$ & $\,\,$9.3582$\,\,$ & $\,\,$4.0883$\,\,$ \\
$\,\,$\color{red} 0.9267\color{black} $\,\,$ & $\,\,$ 1 $\,\,$ & $\,\,$\color{red} 8.6724\color{black} $\,\,$ & $\,\,$\color{red} 3.7888\color{black}   $\,\,$ \\
$\,\,$0.1069$\,\,$ & $\,\,$\color{red} 0.1153\color{black} $\,\,$ & $\,\,$ 1 $\,\,$ & $\,\,$0.4369 $\,\,$ \\
$\,\,$0.2446$\,\,$ & $\,\,$\color{red} 0.2639\color{black} $\,\,$ & $\,\,$2.2890$\,\,$ & $\,\,$ 1  $\,\,$ \\
\end{pmatrix},
\end{equation*}

\begin{equation*}
\mathbf{w}^{\prime} =
\begin{pmatrix}
0.432304\\
0.415759\\
0.046195\\
0.105741
\end{pmatrix} =
0.984868\cdot
\begin{pmatrix}
0.438946\\
\color{gr} 0.422147\color{black} \\
0.046905\\
0.107366
\end{pmatrix},
\end{equation*}
\begin{equation*}
\left[ \frac{{w}^{\prime}_i}{{w}^{\prime}_j} \right] =
\begin{pmatrix}
$\,\,$ 1 $\,\,$ & $\,\,$\color{gr} 1.0398\color{black} $\,\,$ & $\,\,$9.3582$\,\,$ & $\,\,$4.0883$\,\,$ \\
$\,\,$\color{gr} 0.9617\color{black} $\,\,$ & $\,\,$ 1 $\,\,$ & $\,\,$\color{gr} \color{blue} 9\color{black} $\,\,$ & $\,\,$\color{gr} 3.9319\color{black}   $\,\,$ \\
$\,\,$0.1069$\,\,$ & $\,\,$\color{gr} \color{blue}  1/9\color{black} $\,\,$ & $\,\,$ 1 $\,\,$ & $\,\,$0.4369 $\,\,$ \\
$\,\,$0.2446$\,\,$ & $\,\,$\color{gr} 0.2543\color{black} $\,\,$ & $\,\,$2.2890$\,\,$ & $\,\,$ 1  $\,\,$ \\
\end{pmatrix},
\end{equation*}
\end{example}
\newpage
\begin{example}
\begin{equation*}
\mathbf{A} =
\begin{pmatrix}
$\,\,$ 1 $\,\,$ & $\,\,$1$\,\,$ & $\,\,$6$\,\,$ & $\,\,$8 $\,\,$ \\
$\,\,$ 1 $\,\,$ & $\,\,$ 1 $\,\,$ & $\,\,$9$\,\,$ & $\,\,$5 $\,\,$ \\
$\,\,$ 1/6$\,\,$ & $\,\,$ 1/9$\,\,$ & $\,\,$ 1 $\,\,$ & $\,\,$ 1/3 $\,\,$ \\
$\,\,$ 1/8$\,\,$ & $\,\,$ 1/5$\,\,$ & $\,\,$3$\,\,$ & $\,\,$ 1  $\,\,$ \\
\end{pmatrix},
\qquad
\lambda_{\max} =
4.1655,
\qquad
CR = 0.0624
\end{equation*}

\begin{equation*}
\mathbf{w}^{AMAST} =
\begin{pmatrix}
0.434677\\
\color{red} 0.425244\color{black} \\
0.048840\\
0.091240
\end{pmatrix}\end{equation*}
\begin{equation*}
\left[ \frac{{w}^{AMAST}_i}{{w}^{AMAST}_j} \right] =
\begin{pmatrix}
$\,\,$ 1 $\,\,$ & $\,\,$\color{red} 1.0222\color{black} $\,\,$ & $\,\,$8.9001$\,\,$ & $\,\,$4.7641$\,\,$ \\
$\,\,$\color{red} 0.9783\color{black} $\,\,$ & $\,\,$ 1 $\,\,$ & $\,\,$\color{red} 8.7069\color{black} $\,\,$ & $\,\,$\color{red} 4.6607\color{black}   $\,\,$ \\
$\,\,$0.1124$\,\,$ & $\,\,$\color{red} 0.1149\color{black} $\,\,$ & $\,\,$ 1 $\,\,$ & $\,\,$0.5353 $\,\,$ \\
$\,\,$0.2099$\,\,$ & $\,\,$\color{red} 0.2146\color{black} $\,\,$ & $\,\,$1.8682$\,\,$ & $\,\,$ 1  $\,\,$ \\
\end{pmatrix},
\end{equation*}

\begin{equation*}
\mathbf{w}^{\prime} =
\begin{pmatrix}
0.430615\\
0.430615\\
0.048383\\
0.090388
\end{pmatrix} =
0.990655\cdot
\begin{pmatrix}
0.434677\\
\color{gr} 0.434677\color{black} \\
0.048840\\
0.091240
\end{pmatrix},
\end{equation*}
\begin{equation*}
\left[ \frac{{w}^{\prime}_i}{{w}^{\prime}_j} \right] =
\begin{pmatrix}
$\,\,$ 1 $\,\,$ & $\,\,$\color{gr} \color{blue} 1\color{black} $\,\,$ & $\,\,$8.9001$\,\,$ & $\,\,$4.7641$\,\,$ \\
$\,\,$\color{gr} \color{blue} 1\color{black} $\,\,$ & $\,\,$ 1 $\,\,$ & $\,\,$\color{gr} 8.9001\color{black} $\,\,$ & $\,\,$\color{gr} 4.7641\color{black}   $\,\,$ \\
$\,\,$0.1124$\,\,$ & $\,\,$\color{gr} 0.1124\color{black} $\,\,$ & $\,\,$ 1 $\,\,$ & $\,\,$0.5353 $\,\,$ \\
$\,\,$0.2099$\,\,$ & $\,\,$\color{gr} 0.2099\color{black} $\,\,$ & $\,\,$1.8682$\,\,$ & $\,\,$ 1  $\,\,$ \\
\end{pmatrix},
\end{equation*}
\end{example}
\newpage
\begin{example}
\begin{equation*}
\mathbf{A} =
\begin{pmatrix}
$\,\,$ 1 $\,\,$ & $\,\,$1$\,\,$ & $\,\,$6$\,\,$ & $\,\,$9 $\,\,$ \\
$\,\,$ 1 $\,\,$ & $\,\,$ 1 $\,\,$ & $\,\,$8$\,\,$ & $\,\,$6 $\,\,$ \\
$\,\,$ 1/6$\,\,$ & $\,\,$ 1/8$\,\,$ & $\,\,$ 1 $\,\,$ & $\,\,$ 1/2 $\,\,$ \\
$\,\,$ 1/9$\,\,$ & $\,\,$ 1/6$\,\,$ & $\,\,$2$\,\,$ & $\,\,$ 1  $\,\,$ \\
\end{pmatrix},
\qquad
\lambda_{\max} =
4.1031,
\qquad
CR = 0.0389
\end{equation*}

\begin{equation*}
\mathbf{w}^{AMAST} =
\begin{pmatrix}
0.442808\\
\color{red} 0.428826\color{black} \\
0.053987\\
0.074379
\end{pmatrix}\end{equation*}
\begin{equation*}
\left[ \frac{{w}^{AMAST}_i}{{w}^{AMAST}_j} \right] =
\begin{pmatrix}
$\,\,$ 1 $\,\,$ & $\,\,$\color{red} 1.0326\color{black} $\,\,$ & $\,\,$8.2021$\,\,$ & $\,\,$5.9534$\,\,$ \\
$\,\,$\color{red} 0.9684\color{black} $\,\,$ & $\,\,$ 1 $\,\,$ & $\,\,$\color{red} 7.9431\color{black} $\,\,$ & $\,\,$\color{red} 5.7654\color{black}   $\,\,$ \\
$\,\,$0.1219$\,\,$ & $\,\,$\color{red} 0.1259\color{black} $\,\,$ & $\,\,$ 1 $\,\,$ & $\,\,$0.7258 $\,\,$ \\
$\,\,$0.1680$\,\,$ & $\,\,$\color{red} 0.1734\color{black} $\,\,$ & $\,\,$1.3777$\,\,$ & $\,\,$ 1  $\,\,$ \\
\end{pmatrix},
\end{equation*}

\begin{equation*}
\mathbf{w}^{\prime} =
\begin{pmatrix}
0.441452\\
0.430575\\
0.053822\\
0.074151
\end{pmatrix} =
0.996938\cdot
\begin{pmatrix}
0.442808\\
\color{gr} 0.431897\color{black} \\
0.053987\\
0.074379
\end{pmatrix},
\end{equation*}
\begin{equation*}
\left[ \frac{{w}^{\prime}_i}{{w}^{\prime}_j} \right] =
\begin{pmatrix}
$\,\,$ 1 $\,\,$ & $\,\,$\color{gr} 1.0253\color{black} $\,\,$ & $\,\,$8.2021$\,\,$ & $\,\,$5.9534$\,\,$ \\
$\,\,$\color{gr} 0.9754\color{black} $\,\,$ & $\,\,$ 1 $\,\,$ & $\,\,$\color{gr} \color{blue} 8\color{black} $\,\,$ & $\,\,$\color{gr} 5.8067\color{black}   $\,\,$ \\
$\,\,$0.1219$\,\,$ & $\,\,$\color{gr} \color{blue}  1/8\color{black} $\,\,$ & $\,\,$ 1 $\,\,$ & $\,\,$0.7258 $\,\,$ \\
$\,\,$0.1680$\,\,$ & $\,\,$\color{gr} 0.1722\color{black} $\,\,$ & $\,\,$1.3777$\,\,$ & $\,\,$ 1  $\,\,$ \\
\end{pmatrix},
\end{equation*}
\end{example}
\newpage
\begin{example}
\begin{equation*}
\mathbf{A} =
\begin{pmatrix}
$\,\,$ 1 $\,\,$ & $\,\,$1$\,\,$ & $\,\,$6$\,\,$ & $\,\,$9 $\,\,$ \\
$\,\,$ 1 $\,\,$ & $\,\,$ 1 $\,\,$ & $\,\,$9$\,\,$ & $\,\,$5 $\,\,$ \\
$\,\,$ 1/6$\,\,$ & $\,\,$ 1/9$\,\,$ & $\,\,$ 1 $\,\,$ & $\,\,$ 1/3 $\,\,$ \\
$\,\,$ 1/9$\,\,$ & $\,\,$ 1/5$\,\,$ & $\,\,$3$\,\,$ & $\,\,$ 1  $\,\,$ \\
\end{pmatrix},
\qquad
\lambda_{\max} =
4.1966,
\qquad
CR = 0.0741
\end{equation*}

\begin{equation*}
\mathbf{w}^{AMAST} =
\begin{pmatrix}
0.442118\\
\color{red} 0.420618\color{black} \\
0.048653\\
0.088611
\end{pmatrix}\end{equation*}
\begin{equation*}
\left[ \frac{{w}^{AMAST}_i}{{w}^{AMAST}_j} \right] =
\begin{pmatrix}
$\,\,$ 1 $\,\,$ & $\,\,$\color{red} 1.0511\color{black} $\,\,$ & $\,\,$9.0871$\,\,$ & $\,\,$4.9894$\,\,$ \\
$\,\,$\color{red} 0.9514\color{black} $\,\,$ & $\,\,$ 1 $\,\,$ & $\,\,$\color{red} 8.6452\color{black} $\,\,$ & $\,\,$\color{red} 4.7468\color{black}   $\,\,$ \\
$\,\,$0.1100$\,\,$ & $\,\,$\color{red} 0.1157\color{black} $\,\,$ & $\,\,$ 1 $\,\,$ & $\,\,$0.5491 $\,\,$ \\
$\,\,$0.2004$\,\,$ & $\,\,$\color{red} 0.2107\color{black} $\,\,$ & $\,\,$1.8213$\,\,$ & $\,\,$ 1  $\,\,$ \\
\end{pmatrix},
\end{equation*}

\begin{equation*}
\mathbf{w}^{\prime} =
\begin{pmatrix}
0.434616\\
0.430449\\
0.047828\\
0.087107
\end{pmatrix} =
0.983032\cdot
\begin{pmatrix}
0.442118\\
\color{gr} 0.437879\color{black} \\
0.048653\\
0.088611
\end{pmatrix},
\end{equation*}
\begin{equation*}
\left[ \frac{{w}^{\prime}_i}{{w}^{\prime}_j} \right] =
\begin{pmatrix}
$\,\,$ 1 $\,\,$ & $\,\,$\color{gr} 1.0097\color{black} $\,\,$ & $\,\,$9.0871$\,\,$ & $\,\,$4.9894$\,\,$ \\
$\,\,$\color{gr} 0.9904\color{black} $\,\,$ & $\,\,$ 1 $\,\,$ & $\,\,$\color{gr} \color{blue} 9\color{black} $\,\,$ & $\,\,$\color{gr} 4.9416\color{black}   $\,\,$ \\
$\,\,$0.1100$\,\,$ & $\,\,$\color{gr} \color{blue}  1/9\color{black} $\,\,$ & $\,\,$ 1 $\,\,$ & $\,\,$0.5491 $\,\,$ \\
$\,\,$0.2004$\,\,$ & $\,\,$\color{gr} 0.2024\color{black} $\,\,$ & $\,\,$1.8213$\,\,$ & $\,\,$ 1  $\,\,$ \\
\end{pmatrix},
\end{equation*}
\end{example}
\newpage
\begin{example}
\begin{equation*}
\mathbf{A} =
\begin{pmatrix}
$\,\,$ 1 $\,\,$ & $\,\,$1$\,\,$ & $\,\,$6$\,\,$ & $\,\,$9 $\,\,$ \\
$\,\,$ 1 $\,\,$ & $\,\,$ 1 $\,\,$ & $\,\,$9$\,\,$ & $\,\,$6 $\,\,$ \\
$\,\,$ 1/6$\,\,$ & $\,\,$ 1/9$\,\,$ & $\,\,$ 1 $\,\,$ & $\,\,$ 1/2 $\,\,$ \\
$\,\,$ 1/9$\,\,$ & $\,\,$ 1/6$\,\,$ & $\,\,$2$\,\,$ & $\,\,$ 1  $\,\,$ \\
\end{pmatrix},
\qquad
\lambda_{\max} =
4.1031,
\qquad
CR = 0.0389
\end{equation*}

\begin{equation*}
\mathbf{w}^{AMAST} =
\begin{pmatrix}
0.438301\\
\color{red} 0.436461\color{black} \\
0.051876\\
0.073362
\end{pmatrix}\end{equation*}
\begin{equation*}
\left[ \frac{{w}^{AMAST}_i}{{w}^{AMAST}_j} \right] =
\begin{pmatrix}
$\,\,$ 1 $\,\,$ & $\,\,$\color{red} 1.0042\color{black} $\,\,$ & $\,\,$8.4491$\,\,$ & $\,\,$5.9745$\,\,$ \\
$\,\,$\color{red} 0.9958\color{black} $\,\,$ & $\,\,$ 1 $\,\,$ & $\,\,$\color{red} 8.4136\color{black} $\,\,$ & $\,\,$\color{red} 5.9494\color{black}   $\,\,$ \\
$\,\,$0.1184$\,\,$ & $\,\,$\color{red} 0.1189\color{black} $\,\,$ & $\,\,$ 1 $\,\,$ & $\,\,$0.7071 $\,\,$ \\
$\,\,$0.1674$\,\,$ & $\,\,$\color{red} 0.1681\color{black} $\,\,$ & $\,\,$1.4142$\,\,$ & $\,\,$ 1  $\,\,$ \\
\end{pmatrix},
\end{equation*}

\begin{equation*}
\mathbf{w}^{\prime} =
\begin{pmatrix}
0.437496\\
0.437496\\
0.051780\\
0.073227
\end{pmatrix} =
0.998164\cdot
\begin{pmatrix}
0.438301\\
\color{gr} 0.438301\color{black} \\
0.051876\\
0.073362
\end{pmatrix},
\end{equation*}
\begin{equation*}
\left[ \frac{{w}^{\prime}_i}{{w}^{\prime}_j} \right] =
\begin{pmatrix}
$\,\,$ 1 $\,\,$ & $\,\,$\color{gr} \color{blue} 1\color{black} $\,\,$ & $\,\,$8.4491$\,\,$ & $\,\,$5.9745$\,\,$ \\
$\,\,$\color{gr} \color{blue} 1\color{black} $\,\,$ & $\,\,$ 1 $\,\,$ & $\,\,$\color{gr} 8.4491\color{black} $\,\,$ & $\,\,$\color{gr} 5.9745\color{black}   $\,\,$ \\
$\,\,$0.1184$\,\,$ & $\,\,$\color{gr} 0.1184\color{black} $\,\,$ & $\,\,$ 1 $\,\,$ & $\,\,$0.7071 $\,\,$ \\
$\,\,$0.1674$\,\,$ & $\,\,$\color{gr} 0.1674\color{black} $\,\,$ & $\,\,$1.4142$\,\,$ & $\,\,$ 1  $\,\,$ \\
\end{pmatrix},
\end{equation*}
\end{example}
\newpage
\begin{example}
\begin{equation*}
\mathbf{A} =
\begin{pmatrix}
$\,\,$ 1 $\,\,$ & $\,\,$1$\,\,$ & $\,\,$6$\,\,$ & $\,\,$9 $\,\,$ \\
$\,\,$ 1 $\,\,$ & $\,\,$ 1 $\,\,$ & $\,\,$9$\,\,$ & $\,\,$6 $\,\,$ \\
$\,\,$ 1/6$\,\,$ & $\,\,$ 1/9$\,\,$ & $\,\,$ 1 $\,\,$ & $\,\,$ 1/3 $\,\,$ \\
$\,\,$ 1/9$\,\,$ & $\,\,$ 1/6$\,\,$ & $\,\,$3$\,\,$ & $\,\,$ 1  $\,\,$ \\
\end{pmatrix},
\qquad
\lambda_{\max} =
4.1990,
\qquad
CR = 0.0750
\end{equation*}

\begin{equation*}
\mathbf{w}^{AMAST} =
\begin{pmatrix}
0.435592\\
\color{red} 0.432641\color{black} \\
0.048137\\
0.083630
\end{pmatrix}\end{equation*}
\begin{equation*}
\left[ \frac{{w}^{AMAST}_i}{{w}^{AMAST}_j} \right] =
\begin{pmatrix}
$\,\,$ 1 $\,\,$ & $\,\,$\color{red} 1.0068\color{black} $\,\,$ & $\,\,$9.0491$\,\,$ & $\,\,$5.2085$\,\,$ \\
$\,\,$\color{red} 0.9932\color{black} $\,\,$ & $\,\,$ 1 $\,\,$ & $\,\,$\color{red} 8.9878\color{black} $\,\,$ & $\,\,$\color{red} 5.1733\color{black}   $\,\,$ \\
$\,\,$0.1105$\,\,$ & $\,\,$\color{red} 0.1113\color{black} $\,\,$ & $\,\,$ 1 $\,\,$ & $\,\,$0.5756 $\,\,$ \\
$\,\,$0.1920$\,\,$ & $\,\,$\color{red} 0.1933\color{black} $\,\,$ & $\,\,$1.7374$\,\,$ & $\,\,$ 1  $\,\,$ \\
\end{pmatrix},
\end{equation*}

\begin{equation*}
\mathbf{w}^{\prime} =
\begin{pmatrix}
0.435335\\
0.432975\\
0.048108\\
0.083581
\end{pmatrix} =
0.999411\cdot
\begin{pmatrix}
0.435592\\
\color{gr} 0.433230\color{black} \\
0.048137\\
0.083630
\end{pmatrix},
\end{equation*}
\begin{equation*}
\left[ \frac{{w}^{\prime}_i}{{w}^{\prime}_j} \right] =
\begin{pmatrix}
$\,\,$ 1 $\,\,$ & $\,\,$\color{gr} 1.0055\color{black} $\,\,$ & $\,\,$9.0491$\,\,$ & $\,\,$5.2085$\,\,$ \\
$\,\,$\color{gr} 0.9946\color{black} $\,\,$ & $\,\,$ 1 $\,\,$ & $\,\,$\color{gr} \color{blue} 9\color{black} $\,\,$ & $\,\,$\color{gr} 5.1803\color{black}   $\,\,$ \\
$\,\,$0.1105$\,\,$ & $\,\,$\color{gr} \color{blue}  1/9\color{black} $\,\,$ & $\,\,$ 1 $\,\,$ & $\,\,$0.5756 $\,\,$ \\
$\,\,$0.1920$\,\,$ & $\,\,$\color{gr} 0.1930\color{black} $\,\,$ & $\,\,$1.7374$\,\,$ & $\,\,$ 1  $\,\,$ \\
\end{pmatrix},
\end{equation*}
\end{example}
\newpage
\begin{example}
\begin{equation*}
\mathbf{A} =
\begin{pmatrix}
$\,\,$ 1 $\,\,$ & $\,\,$1$\,\,$ & $\,\,$7$\,\,$ & $\,\,$4 $\,\,$ \\
$\,\,$ 1 $\,\,$ & $\,\,$ 1 $\,\,$ & $\,\,$9$\,\,$ & $\,\,$3 $\,\,$ \\
$\,\,$ 1/7$\,\,$ & $\,\,$ 1/9$\,\,$ & $\,\,$ 1 $\,\,$ & $\,\,$ 1/4 $\,\,$ \\
$\,\,$ 1/4$\,\,$ & $\,\,$ 1/3$\,\,$ & $\,\,$4$\,\,$ & $\,\,$ 1  $\,\,$ \\
\end{pmatrix},
\qquad
\lambda_{\max} =
4.0576,
\qquad
CR = 0.0217
\end{equation*}

\begin{equation*}
\mathbf{w}^{AMAST} =
\begin{pmatrix}
0.410844\\
\color{red} 0.405783\color{black} \\
0.045653\\
0.137719
\end{pmatrix}\end{equation*}
\begin{equation*}
\left[ \frac{{w}^{AMAST}_i}{{w}^{AMAST}_j} \right] =
\begin{pmatrix}
$\,\,$ 1 $\,\,$ & $\,\,$\color{red} 1.0125\color{black} $\,\,$ & $\,\,$8.9993$\,\,$ & $\,\,$2.9832$\,\,$ \\
$\,\,$\color{red} 0.9877\color{black} $\,\,$ & $\,\,$ 1 $\,\,$ & $\,\,$\color{red} 8.8884\color{black} $\,\,$ & $\,\,$\color{red} 2.9465\color{black}   $\,\,$ \\
$\,\,$0.1111$\,\,$ & $\,\,$\color{red} 0.1125\color{black} $\,\,$ & $\,\,$ 1 $\,\,$ & $\,\,$0.3315 $\,\,$ \\
$\,\,$0.3352$\,\,$ & $\,\,$\color{red} 0.3394\color{black} $\,\,$ & $\,\,$3.0166$\,\,$ & $\,\,$ 1  $\,\,$ \\
\end{pmatrix},
\end{equation*}

\begin{equation*}
\mathbf{w}^{\prime} =
\begin{pmatrix}
0.408776\\
0.408776\\
0.045423\\
0.137026
\end{pmatrix} =
0.994965\cdot
\begin{pmatrix}
0.410844\\
\color{gr} 0.410844\color{black} \\
0.045653\\
0.137719
\end{pmatrix},
\end{equation*}
\begin{equation*}
\left[ \frac{{w}^{\prime}_i}{{w}^{\prime}_j} \right] =
\begin{pmatrix}
$\,\,$ 1 $\,\,$ & $\,\,$\color{gr} \color{blue} 1\color{black} $\,\,$ & $\,\,$8.9993$\,\,$ & $\,\,$2.9832$\,\,$ \\
$\,\,$\color{gr} \color{blue} 1\color{black} $\,\,$ & $\,\,$ 1 $\,\,$ & $\,\,$\color{gr} 8.9993\color{black} $\,\,$ & $\,\,$\color{gr} 2.9832\color{black}   $\,\,$ \\
$\,\,$0.1111$\,\,$ & $\,\,$\color{gr} 0.1111\color{black} $\,\,$ & $\,\,$ 1 $\,\,$ & $\,\,$0.3315 $\,\,$ \\
$\,\,$0.3352$\,\,$ & $\,\,$\color{gr} 0.3352\color{black} $\,\,$ & $\,\,$3.0166$\,\,$ & $\,\,$ 1  $\,\,$ \\
\end{pmatrix},
\end{equation*}
\end{example}
\newpage
\begin{example}
\begin{equation*}
\mathbf{A} =
\begin{pmatrix}
$\,\,$ 1 $\,\,$ & $\,\,$1$\,\,$ & $\,\,$7$\,\,$ & $\,\,$8 $\,\,$ \\
$\,\,$ 1 $\,\,$ & $\,\,$ 1 $\,\,$ & $\,\,$9$\,\,$ & $\,\,$6 $\,\,$ \\
$\,\,$ 1/7$\,\,$ & $\,\,$ 1/9$\,\,$ & $\,\,$ 1 $\,\,$ & $\,\,$ 1/2 $\,\,$ \\
$\,\,$ 1/8$\,\,$ & $\,\,$ 1/6$\,\,$ & $\,\,$2$\,\,$ & $\,\,$ 1  $\,\,$ \\
\end{pmatrix},
\qquad
\lambda_{\max} =
4.0576,
\qquad
CR = 0.0217
\end{equation*}

\begin{equation*}
\mathbf{w}^{AMAST} =
\begin{pmatrix}
0.440778\\
\color{red} 0.435900\color{black} \\
0.049048\\
0.074274
\end{pmatrix}\end{equation*}
\begin{equation*}
\left[ \frac{{w}^{AMAST}_i}{{w}^{AMAST}_j} \right] =
\begin{pmatrix}
$\,\,$ 1 $\,\,$ & $\,\,$\color{red} 1.0112\color{black} $\,\,$ & $\,\,$8.9866$\,\,$ & $\,\,$5.9345$\,\,$ \\
$\,\,$\color{red} 0.9889\color{black} $\,\,$ & $\,\,$ 1 $\,\,$ & $\,\,$\color{red} 8.8871\color{black} $\,\,$ & $\,\,$\color{red} 5.8688\color{black}   $\,\,$ \\
$\,\,$0.1113$\,\,$ & $\,\,$\color{red} 0.1125\color{black} $\,\,$ & $\,\,$ 1 $\,\,$ & $\,\,$0.6604 $\,\,$ \\
$\,\,$0.1685$\,\,$ & $\,\,$\color{red} 0.1704\color{black} $\,\,$ & $\,\,$1.5143$\,\,$ & $\,\,$ 1  $\,\,$ \\
\end{pmatrix},
\end{equation*}

\begin{equation*}
\mathbf{w}^{\prime} =
\begin{pmatrix}
0.438638\\
0.438638\\
0.048810\\
0.073913
\end{pmatrix} =
0.995145\cdot
\begin{pmatrix}
0.440778\\
\color{gr} 0.440778\color{black} \\
0.049048\\
0.074274
\end{pmatrix},
\end{equation*}
\begin{equation*}
\left[ \frac{{w}^{\prime}_i}{{w}^{\prime}_j} \right] =
\begin{pmatrix}
$\,\,$ 1 $\,\,$ & $\,\,$\color{gr} \color{blue} 1\color{black} $\,\,$ & $\,\,$8.9866$\,\,$ & $\,\,$5.9345$\,\,$ \\
$\,\,$\color{gr} \color{blue} 1\color{black} $\,\,$ & $\,\,$ 1 $\,\,$ & $\,\,$\color{gr} 8.9866\color{black} $\,\,$ & $\,\,$\color{gr} 5.9345\color{black}   $\,\,$ \\
$\,\,$0.1113$\,\,$ & $\,\,$\color{gr} 0.1113\color{black} $\,\,$ & $\,\,$ 1 $\,\,$ & $\,\,$0.6604 $\,\,$ \\
$\,\,$0.1685$\,\,$ & $\,\,$\color{gr} 0.1685\color{black} $\,\,$ & $\,\,$1.5143$\,\,$ & $\,\,$ 1  $\,\,$ \\
\end{pmatrix},
\end{equation*}
\end{example}
\newpage
\begin{example}
\begin{equation*}
\mathbf{A} =
\begin{pmatrix}
$\,\,$ 1 $\,\,$ & $\,\,$1$\,\,$ & $\,\,$9$\,\,$ & $\,\,$8 $\,\,$ \\
$\,\,$ 1 $\,\,$ & $\,\,$ 1 $\,\,$ & $\,\,$6$\,\,$ & $\,\,$2 $\,\,$ \\
$\,\,$ 1/9$\,\,$ & $\,\,$ 1/6$\,\,$ & $\,\,$ 1 $\,\,$ & $\,\,$ 1/2 $\,\,$ \\
$\,\,$ 1/8$\,\,$ & $\,\,$ 1/2$\,\,$ & $\,\,$2$\,\,$ & $\,\,$ 1  $\,\,$ \\
\end{pmatrix},
\qquad
\lambda_{\max} =
4.166405,
\qquad
CR = 0.062747
\end{equation*}

\begin{equation*}
\mathbf{w}^{AMAST} =
\begin{pmatrix}
0.50419133\\
0.33118710\\
\color{red} 0.05487336\color{black} \\
0.10974820
\end{pmatrix}\end{equation*}
\begin{equation*}
\left[ \frac{{w}^{AMAST}_i}{{w}^{AMAST}_j} \right] =
\begin{pmatrix}
$\,\,$ 1 $\,\,$ & $\,\,$1.522376$\,\,$ & $\,\,$\color{red} 9.188271\color{black} $\,\,$ & $\,\,$4.594074$\,\,$ \\
$\,\,$0.656868$\,\,$ & $\,\,$ 1 $\,\,$ & $\,\,$\color{red} 6.035480\color{black} $\,\,$ & $\,\,$3.017700  $\,\,$ \\
$\,\,$\color{red} 0.108834\color{black} $\,\,$ & $\,\,$\color{red} 0.165687\color{black} $\,\,$ & $\,\,$ 1 $\,\,$ & $\,\,$\color{red} 0.499993\color{black}  $\,\,$ \\
$\,\,$0.217672$\,\,$ & $\,\,$0.331378$\,\,$ & $\,\,$\color{red} 2.000027\color{black} $\,\,$ & $\,\,$ 1  $\,\,$ \\
\end{pmatrix},
\end{equation*}

\begin{equation*}
\mathbf{w}^{\prime} =
\begin{pmatrix}
0.50419096\\
0.33118686\\
0.05487406\\
0.10974812
\end{pmatrix} =
0.999999\cdot
\begin{pmatrix}
0.50419133\\
0.33118710\\
\color{gr} 0.05487410\color{black} \\
0.10974820
\end{pmatrix},
\end{equation*}
\begin{equation*}
\left[ \frac{{w}^{\prime}_i}{{w}^{\prime}_j} \right] =
\begin{pmatrix}
$\,\,$ 1 $\,\,$ & $\,\,$1.522376$\,\,$ & $\,\,$\color{gr} 9.188148\color{black} $\,\,$ & $\,\,$4.594074$\,\,$ \\
$\,\,$0.656868$\,\,$ & $\,\,$ 1 $\,\,$ & $\,\,$\color{gr} 6.035399\color{black} $\,\,$ & $\,\,$3.017700  $\,\,$ \\
$\,\,$\color{gr} 0.108836\color{black} $\,\,$ & $\,\,$\color{gr} 0.165689\color{black} $\,\,$ & $\,\,$ 1 $\,\,$ & $\,\,$\color{gr} \color{blue}  1/2\color{black}  $\,\,$ \\
$\,\,$0.217672$\,\,$ & $\,\,$0.331378$\,\,$ & $\,\,$\color{gr} \color{blue} 2\color{black} $\,\,$ & $\,\,$ 1  $\,\,$ \\
\end{pmatrix},
\end{equation*}
\end{example}
\newpage
\begin{example}
\begin{equation*}
\mathbf{A} =
\begin{pmatrix}
$\,\,$ 1 $\,\,$ & $\,\,$2$\,\,$ & $\,\,$6$\,\,$ & $\,\,$2 $\,\,$ \\
$\,\,$ 1/2$\,\,$ & $\,\,$ 1 $\,\,$ & $\,\,$2$\,\,$ & $\,\,$2 $\,\,$ \\
$\,\,$ 1/6$\,\,$ & $\,\,$ 1/2$\,\,$ & $\,\,$ 1 $\,\,$ & $\,\,$ 1/5 $\,\,$ \\
$\,\,$ 1/2$\,\,$ & $\,\,$ 1/2$\,\,$ & $\,\,$5$\,\,$ & $\,\,$ 1  $\,\,$ \\
\end{pmatrix},
\qquad
\lambda_{\max} =
4.2277,
\qquad
CR = 0.0859
\end{equation*}

\begin{equation*}
\mathbf{w}^{AMAST} =
\begin{pmatrix}
\color{red} 0.446413\color{black} \\
0.250908\\
0.078049\\
0.224630
\end{pmatrix}\end{equation*}
\begin{equation*}
\left[ \frac{{w}^{AMAST}_i}{{w}^{AMAST}_j} \right] =
\begin{pmatrix}
$\,\,$ 1 $\,\,$ & $\,\,$\color{red} 1.7792\color{black} $\,\,$ & $\,\,$\color{red} 5.7196\color{black} $\,\,$ & $\,\,$\color{red} 1.9873\color{black} $\,\,$ \\
$\,\,$\color{red} 0.5621\color{black} $\,\,$ & $\,\,$ 1 $\,\,$ & $\,\,$3.2147$\,\,$ & $\,\,$1.1170  $\,\,$ \\
$\,\,$\color{red} 0.1748\color{black} $\,\,$ & $\,\,$0.3111$\,\,$ & $\,\,$ 1 $\,\,$ & $\,\,$0.3475 $\,\,$ \\
$\,\,$\color{red} 0.5032\color{black} $\,\,$ & $\,\,$0.8953$\,\,$ & $\,\,$2.8781$\,\,$ & $\,\,$ 1  $\,\,$ \\
\end{pmatrix},
\end{equation*}

\begin{equation*}
\mathbf{w}^{\prime} =
\begin{pmatrix}
0.447984\\
0.250196\\
0.077828\\
0.223992
\end{pmatrix} =
0.997161\cdot
\begin{pmatrix}
\color{gr} 0.449260\color{black} \\
0.250908\\
0.078049\\
0.224630
\end{pmatrix},
\end{equation*}
\begin{equation*}
\left[ \frac{{w}^{\prime}_i}{{w}^{\prime}_j} \right] =
\begin{pmatrix}
$\,\,$ 1 $\,\,$ & $\,\,$\color{gr} 1.7905\color{black} $\,\,$ & $\,\,$\color{gr} 5.7561\color{black} $\,\,$ & $\,\,$\color{gr} \color{blue} 2\color{black} $\,\,$ \\
$\,\,$\color{gr} 0.5585\color{black} $\,\,$ & $\,\,$ 1 $\,\,$ & $\,\,$3.2147$\,\,$ & $\,\,$1.1170  $\,\,$ \\
$\,\,$\color{gr} 0.1737\color{black} $\,\,$ & $\,\,$0.3111$\,\,$ & $\,\,$ 1 $\,\,$ & $\,\,$0.3475 $\,\,$ \\
$\,\,$\color{gr} \color{blue}  1/2\color{black} $\,\,$ & $\,\,$0.8953$\,\,$ & $\,\,$2.8781$\,\,$ & $\,\,$ 1  $\,\,$ \\
\end{pmatrix},
\end{equation*}
\end{example}
\newpage
\begin{example}
\begin{equation*}
\mathbf{A} =
\begin{pmatrix}
$\,\,$ 1 $\,\,$ & $\,\,$2$\,\,$ & $\,\,$6$\,\,$ & $\,\,$5 $\,\,$ \\
$\,\,$ 1/2$\,\,$ & $\,\,$ 1 $\,\,$ & $\,\,$2$\,\,$ & $\,\,$5 $\,\,$ \\
$\,\,$ 1/6$\,\,$ & $\,\,$ 1/2$\,\,$ & $\,\,$ 1 $\,\,$ & $\,\,$ 1/2 $\,\,$ \\
$\,\,$ 1/5$\,\,$ & $\,\,$ 1/5$\,\,$ & $\,\,$2$\,\,$ & $\,\,$ 1  $\,\,$ \\
\end{pmatrix},
\qquad
\lambda_{\max} =
4.2277,
\qquad
CR = 0.0859
\end{equation*}

\begin{equation*}
\mathbf{w}^{AMAST} =
\begin{pmatrix}
\color{red} 0.515207\color{black} \\
0.288037\\
0.090062\\
0.106695
\end{pmatrix}\end{equation*}
\begin{equation*}
\left[ \frac{{w}^{AMAST}_i}{{w}^{AMAST}_j} \right] =
\begin{pmatrix}
$\,\,$ 1 $\,\,$ & $\,\,$\color{red} 1.7887\color{black} $\,\,$ & $\,\,$\color{red} 5.7206\color{black} $\,\,$ & $\,\,$\color{red} 4.8288\color{black} $\,\,$ \\
$\,\,$\color{red} 0.5591\color{black} $\,\,$ & $\,\,$ 1 $\,\,$ & $\,\,$3.1982$\,\,$ & $\,\,$2.6996  $\,\,$ \\
$\,\,$\color{red} 0.1748\color{black} $\,\,$ & $\,\,$0.3127$\,\,$ & $\,\,$ 1 $\,\,$ & $\,\,$0.8441 $\,\,$ \\
$\,\,$\color{red} 0.2071\color{black} $\,\,$ & $\,\,$0.3704$\,\,$ & $\,\,$1.1847$\,\,$ & $\,\,$ 1  $\,\,$ \\
\end{pmatrix},
\end{equation*}

\begin{equation*}
\mathbf{w}^{\prime} =
\begin{pmatrix}
0.523905\\
0.282869\\
0.088446\\
0.104781
\end{pmatrix} =
0.982058\cdot
\begin{pmatrix}
\color{gr} 0.533476\color{black} \\
0.288037\\
0.090062\\
0.106695
\end{pmatrix},
\end{equation*}
\begin{equation*}
\left[ \frac{{w}^{\prime}_i}{{w}^{\prime}_j} \right] =
\begin{pmatrix}
$\,\,$ 1 $\,\,$ & $\,\,$\color{gr} 1.8521\color{black} $\,\,$ & $\,\,$\color{gr} 5.9235\color{black} $\,\,$ & $\,\,$\color{gr} \color{blue} 5\color{black} $\,\,$ \\
$\,\,$\color{gr} 0.5399\color{black} $\,\,$ & $\,\,$ 1 $\,\,$ & $\,\,$3.1982$\,\,$ & $\,\,$2.6996  $\,\,$ \\
$\,\,$\color{gr} 0.1688\color{black} $\,\,$ & $\,\,$0.3127$\,\,$ & $\,\,$ 1 $\,\,$ & $\,\,$0.8441 $\,\,$ \\
$\,\,$\color{gr} \color{blue}  1/5\color{black} $\,\,$ & $\,\,$0.3704$\,\,$ & $\,\,$1.1847$\,\,$ & $\,\,$ 1  $\,\,$ \\
\end{pmatrix},
\end{equation*}
\end{example}
\newpage
\begin{example}
\begin{equation*}
\mathbf{A} =
\begin{pmatrix}
$\,\,$ 1 $\,\,$ & $\,\,$2$\,\,$ & $\,\,$7$\,\,$ & $\,\,$3 $\,\,$ \\
$\,\,$ 1/2$\,\,$ & $\,\,$ 1 $\,\,$ & $\,\,$8$\,\,$ & $\,\,$2 $\,\,$ \\
$\,\,$ 1/7$\,\,$ & $\,\,$ 1/8$\,\,$ & $\,\,$ 1 $\,\,$ & $\,\,$ 1/3 $\,\,$ \\
$\,\,$ 1/3$\,\,$ & $\,\,$ 1/2$\,\,$ & $\,\,$3$\,\,$ & $\,\,$ 1  $\,\,$ \\
\end{pmatrix},
\qquad
\lambda_{\max} =
4.0576,
\qquad
CR = 0.0217
\end{equation*}

\begin{equation*}
\mathbf{w}^{AMAST} =
\begin{pmatrix}
0.474274\\
0.315746\\
0.052838\\
\color{red} 0.157143\color{black}
\end{pmatrix}\end{equation*}
\begin{equation*}
\left[ \frac{{w}^{AMAST}_i}{{w}^{AMAST}_j} \right] =
\begin{pmatrix}
$\,\,$ 1 $\,\,$ & $\,\,$1.5021$\,\,$ & $\,\,$8.9760$\,\,$ & $\,\,$\color{red} 3.0181\color{black} $\,\,$ \\
$\,\,$0.6657$\,\,$ & $\,\,$ 1 $\,\,$ & $\,\,$5.9758$\,\,$ & $\,\,$\color{red} 2.0093\color{black}   $\,\,$ \\
$\,\,$0.1114$\,\,$ & $\,\,$0.1673$\,\,$ & $\,\,$ 1 $\,\,$ & $\,\,$\color{red} 0.3362\color{black}  $\,\,$ \\
$\,\,$\color{red} 0.3313\color{black} $\,\,$ & $\,\,$\color{red} 0.4977\color{black} $\,\,$ & $\,\,$\color{red} 2.9741\color{black} $\,\,$ & $\,\,$ 1  $\,\,$ \\
\end{pmatrix},
\end{equation*}

\begin{equation*}
\mathbf{w}^{\prime} =
\begin{pmatrix}
0.473927\\
0.315516\\
0.052799\\
0.157758
\end{pmatrix} =
0.999270\cdot
\begin{pmatrix}
0.474274\\
0.315746\\
0.052838\\
\color{gr} 0.157873\color{black}
\end{pmatrix},
\end{equation*}
\begin{equation*}
\left[ \frac{{w}^{\prime}_i}{{w}^{\prime}_j} \right] =
\begin{pmatrix}
$\,\,$ 1 $\,\,$ & $\,\,$1.5021$\,\,$ & $\,\,$8.9760$\,\,$ & $\,\,$\color{gr} 3.0041\color{black} $\,\,$ \\
$\,\,$0.6657$\,\,$ & $\,\,$ 1 $\,\,$ & $\,\,$5.9758$\,\,$ & $\,\,$\color{gr} \color{blue} 2\color{black}   $\,\,$ \\
$\,\,$0.1114$\,\,$ & $\,\,$0.1673$\,\,$ & $\,\,$ 1 $\,\,$ & $\,\,$\color{gr} 0.3347\color{black}  $\,\,$ \\
$\,\,$\color{gr} 0.3329\color{black} $\,\,$ & $\,\,$\color{gr} \color{blue}  1/2\color{black} $\,\,$ & $\,\,$\color{gr} 2.9879\color{black} $\,\,$ & $\,\,$ 1  $\,\,$ \\
\end{pmatrix},
\end{equation*}
\end{example}
\newpage
\begin{example}
\begin{equation*}
\mathbf{A} =
\begin{pmatrix}
$\,\,$ 1 $\,\,$ & $\,\,$2$\,\,$ & $\,\,$8$\,\,$ & $\,\,$2 $\,\,$ \\
$\,\,$ 1/2$\,\,$ & $\,\,$ 1 $\,\,$ & $\,\,$3$\,\,$ & $\,\,$2 $\,\,$ \\
$\,\,$ 1/8$\,\,$ & $\,\,$ 1/3$\,\,$ & $\,\,$ 1 $\,\,$ & $\,\,$ 1/7 $\,\,$ \\
$\,\,$ 1/2$\,\,$ & $\,\,$ 1/2$\,\,$ & $\,\,$7$\,\,$ & $\,\,$ 1  $\,\,$ \\
\end{pmatrix},
\qquad
\lambda_{\max} =
4.2109,
\qquad
CR = 0.0795
\end{equation*}

\begin{equation*}
\mathbf{w}^{AMAST} =
\begin{pmatrix}
\color{red} 0.452875\color{black} \\
0.260532\\
0.056766\\
0.229827
\end{pmatrix}\end{equation*}
\begin{equation*}
\left[ \frac{{w}^{AMAST}_i}{{w}^{AMAST}_j} \right] =
\begin{pmatrix}
$\,\,$ 1 $\,\,$ & $\,\,$\color{red} 1.7383\color{black} $\,\,$ & $\,\,$\color{red} 7.9779\color{black} $\,\,$ & $\,\,$\color{red} 1.9705\color{black} $\,\,$ \\
$\,\,$\color{red} 0.5753\color{black} $\,\,$ & $\,\,$ 1 $\,\,$ & $\,\,$4.5896$\,\,$ & $\,\,$1.1336  $\,\,$ \\
$\,\,$\color{red} 0.1253\color{black} $\,\,$ & $\,\,$0.2179$\,\,$ & $\,\,$ 1 $\,\,$ & $\,\,$0.2470 $\,\,$ \\
$\,\,$\color{red} 0.5075\color{black} $\,\,$ & $\,\,$0.8821$\,\,$ & $\,\,$4.0487$\,\,$ & $\,\,$ 1  $\,\,$ \\
\end{pmatrix},
\end{equation*}

\begin{equation*}
\mathbf{w}^{\prime} =
\begin{pmatrix}
0.453560\\
0.260206\\
0.056695\\
0.229539
\end{pmatrix} =
0.998747\cdot
\begin{pmatrix}
\color{gr} 0.454129\color{black} \\
0.260532\\
0.056766\\
0.229827
\end{pmatrix},
\end{equation*}
\begin{equation*}
\left[ \frac{{w}^{\prime}_i}{{w}^{\prime}_j} \right] =
\begin{pmatrix}
$\,\,$ 1 $\,\,$ & $\,\,$\color{gr} 1.7431\color{black} $\,\,$ & $\,\,$\color{gr} \color{blue} 8\color{black} $\,\,$ & $\,\,$\color{gr} 1.9760\color{black} $\,\,$ \\
$\,\,$\color{gr} 0.5737\color{black} $\,\,$ & $\,\,$ 1 $\,\,$ & $\,\,$4.5896$\,\,$ & $\,\,$1.1336  $\,\,$ \\
$\,\,$\color{gr} \color{blue}  1/8\color{black} $\,\,$ & $\,\,$0.2179$\,\,$ & $\,\,$ 1 $\,\,$ & $\,\,$0.2470 $\,\,$ \\
$\,\,$\color{gr} 0.5061\color{black} $\,\,$ & $\,\,$0.8821$\,\,$ & $\,\,$4.0487$\,\,$ & $\,\,$ 1  $\,\,$ \\
\end{pmatrix},
\end{equation*}
\end{example}
\newpage
\begin{example}
\begin{equation*}
\mathbf{A} =
\begin{pmatrix}
$\,\,$ 1 $\,\,$ & $\,\,$2$\,\,$ & $\,\,$8$\,\,$ & $\,\,$3 $\,\,$ \\
$\,\,$ 1/2$\,\,$ & $\,\,$ 1 $\,\,$ & $\,\,$3$\,\,$ & $\,\,$2 $\,\,$ \\
$\,\,$ 1/8$\,\,$ & $\,\,$ 1/3$\,\,$ & $\,\,$ 1 $\,\,$ & $\,\,$ 1/4 $\,\,$ \\
$\,\,$ 1/3$\,\,$ & $\,\,$ 1/2$\,\,$ & $\,\,$4$\,\,$ & $\,\,$ 1  $\,\,$ \\
\end{pmatrix},
\qquad
\lambda_{\max} =
4.0820,
\qquad
CR = 0.0309
\end{equation*}

\begin{equation*}
\mathbf{w}^{AMAST} =
\begin{pmatrix}
\color{red} 0.503620\color{black} \\
0.256132\\
0.063137\\
0.177111
\end{pmatrix}\end{equation*}
\begin{equation*}
\left[ \frac{{w}^{AMAST}_i}{{w}^{AMAST}_j} \right] =
\begin{pmatrix}
$\,\,$ 1 $\,\,$ & $\,\,$\color{red} 1.9663\color{black} $\,\,$ & $\,\,$\color{red} 7.9766\color{black} $\,\,$ & $\,\,$\color{red} 2.8435\color{black} $\,\,$ \\
$\,\,$\color{red} 0.5086\color{black} $\,\,$ & $\,\,$ 1 $\,\,$ & $\,\,$4.0567$\,\,$ & $\,\,$1.4462  $\,\,$ \\
$\,\,$\color{red} 0.1254\color{black} $\,\,$ & $\,\,$0.2465$\,\,$ & $\,\,$ 1 $\,\,$ & $\,\,$0.3565 $\,\,$ \\
$\,\,$\color{red} 0.3517\color{black} $\,\,$ & $\,\,$0.6915$\,\,$ & $\,\,$2.8052$\,\,$ & $\,\,$ 1  $\,\,$ \\
\end{pmatrix},
\end{equation*}

\begin{equation*}
\mathbf{w}^{\prime} =
\begin{pmatrix}
0.504353\\
0.255754\\
0.063044\\
0.176849
\end{pmatrix} =
0.998523\cdot
\begin{pmatrix}
\color{gr} 0.505099\color{black} \\
0.256132\\
0.063137\\
0.177111
\end{pmatrix},
\end{equation*}
\begin{equation*}
\left[ \frac{{w}^{\prime}_i}{{w}^{\prime}_j} \right] =
\begin{pmatrix}
$\,\,$ 1 $\,\,$ & $\,\,$\color{gr} 1.9720\color{black} $\,\,$ & $\,\,$\color{gr} \color{blue} 8\color{black} $\,\,$ & $\,\,$\color{gr} 2.8519\color{black} $\,\,$ \\
$\,\,$\color{gr} 0.5071\color{black} $\,\,$ & $\,\,$ 1 $\,\,$ & $\,\,$4.0567$\,\,$ & $\,\,$1.4462  $\,\,$ \\
$\,\,$\color{gr} \color{blue}  1/8\color{black} $\,\,$ & $\,\,$0.2465$\,\,$ & $\,\,$ 1 $\,\,$ & $\,\,$0.3565 $\,\,$ \\
$\,\,$\color{gr} 0.3506\color{black} $\,\,$ & $\,\,$0.6915$\,\,$ & $\,\,$2.8052$\,\,$ & $\,\,$ 1  $\,\,$ \\
\end{pmatrix},
\end{equation*}
\end{example}
\newpage
\begin{example}
\begin{equation*}
\mathbf{A} =
\begin{pmatrix}
$\,\,$ 1 $\,\,$ & $\,\,$2$\,\,$ & $\,\,$8$\,\,$ & $\,\,$4 $\,\,$ \\
$\,\,$ 1/2$\,\,$ & $\,\,$ 1 $\,\,$ & $\,\,$3$\,\,$ & $\,\,$3 $\,\,$ \\
$\,\,$ 1/8$\,\,$ & $\,\,$ 1/3$\,\,$ & $\,\,$ 1 $\,\,$ & $\,\,$ 1/3 $\,\,$ \\
$\,\,$ 1/4$\,\,$ & $\,\,$ 1/3$\,\,$ & $\,\,$3$\,\,$ & $\,\,$ 1  $\,\,$ \\
\end{pmatrix},
\qquad
\lambda_{\max} =
4.1031,
\qquad
CR = 0.0389
\end{equation*}

\begin{equation*}
\mathbf{w}^{AMAST} =
\begin{pmatrix}
\color{red} 0.523376\color{black} \\
0.275059\\
0.066040\\
0.135525
\end{pmatrix}\end{equation*}
\begin{equation*}
\left[ \frac{{w}^{AMAST}_i}{{w}^{AMAST}_j} \right] =
\begin{pmatrix}
$\,\,$ 1 $\,\,$ & $\,\,$\color{red} 1.9028\color{black} $\,\,$ & $\,\,$\color{red} 7.9251\color{black} $\,\,$ & $\,\,$\color{red} 3.8618\color{black} $\,\,$ \\
$\,\,$\color{red} 0.5255\color{black} $\,\,$ & $\,\,$ 1 $\,\,$ & $\,\,$4.1650$\,\,$ & $\,\,$2.0296  $\,\,$ \\
$\,\,$\color{red} 0.1262\color{black} $\,\,$ & $\,\,$0.2401$\,\,$ & $\,\,$ 1 $\,\,$ & $\,\,$0.4873 $\,\,$ \\
$\,\,$\color{red} 0.2589\color{black} $\,\,$ & $\,\,$0.4927$\,\,$ & $\,\,$2.0522$\,\,$ & $\,\,$ 1  $\,\,$ \\
\end{pmatrix},
\end{equation*}

\begin{equation*}
\mathbf{w}^{\prime} =
\begin{pmatrix}
0.525721\\
0.273706\\
0.065715\\
0.134858
\end{pmatrix} =
0.995081\cdot
\begin{pmatrix}
\color{gr} 0.528319\color{black} \\
0.275059\\
0.066040\\
0.135525
\end{pmatrix},
\end{equation*}
\begin{equation*}
\left[ \frac{{w}^{\prime}_i}{{w}^{\prime}_j} \right] =
\begin{pmatrix}
$\,\,$ 1 $\,\,$ & $\,\,$\color{gr} 1.9207\color{black} $\,\,$ & $\,\,$\color{gr} \color{blue} 8\color{black} $\,\,$ & $\,\,$\color{gr} 3.8983\color{black} $\,\,$ \\
$\,\,$\color{gr} 0.5206\color{black} $\,\,$ & $\,\,$ 1 $\,\,$ & $\,\,$4.1650$\,\,$ & $\,\,$2.0296  $\,\,$ \\
$\,\,$\color{gr} \color{blue}  1/8\color{black} $\,\,$ & $\,\,$0.2401$\,\,$ & $\,\,$ 1 $\,\,$ & $\,\,$0.4873 $\,\,$ \\
$\,\,$\color{gr} 0.2565\color{black} $\,\,$ & $\,\,$0.4927$\,\,$ & $\,\,$2.0522$\,\,$ & $\,\,$ 1  $\,\,$ \\
\end{pmatrix},
\end{equation*}
\end{example}
\newpage
\begin{example}
\begin{equation*}
\mathbf{A} =
\begin{pmatrix}
$\,\,$ 1 $\,\,$ & $\,\,$2$\,\,$ & $\,\,$8$\,\,$ & $\,\,$6 $\,\,$ \\
$\,\,$ 1/2$\,\,$ & $\,\,$ 1 $\,\,$ & $\,\,$3$\,\,$ & $\,\,$4 $\,\,$ \\
$\,\,$ 1/8$\,\,$ & $\,\,$ 1/3$\,\,$ & $\,\,$ 1 $\,\,$ & $\,\,$ 1/2 $\,\,$ \\
$\,\,$ 1/6$\,\,$ & $\,\,$ 1/4$\,\,$ & $\,\,$2$\,\,$ & $\,\,$ 1  $\,\,$ \\
\end{pmatrix},
\qquad
\lambda_{\max} =
4.0820,
\qquad
CR = 0.0309
\end{equation*}

\begin{equation*}
\mathbf{w}^{AMAST} =
\begin{pmatrix}
\color{red} 0.552249\color{black} \\
0.280633\\
0.069265\\
0.097852
\end{pmatrix}\end{equation*}
\begin{equation*}
\left[ \frac{{w}^{AMAST}_i}{{w}^{AMAST}_j} \right] =
\begin{pmatrix}
$\,\,$ 1 $\,\,$ & $\,\,$\color{red} 1.9679\color{black} $\,\,$ & $\,\,$\color{red} 7.9729\color{black} $\,\,$ & $\,\,$\color{red} 5.6437\color{black} $\,\,$ \\
$\,\,$\color{red} 0.5082\color{black} $\,\,$ & $\,\,$ 1 $\,\,$ & $\,\,$4.0516$\,\,$ & $\,\,$2.8679  $\,\,$ \\
$\,\,$\color{red} 0.1254\color{black} $\,\,$ & $\,\,$0.2468$\,\,$ & $\,\,$ 1 $\,\,$ & $\,\,$0.7079 $\,\,$ \\
$\,\,$\color{red} 0.1772\color{black} $\,\,$ & $\,\,$0.3487$\,\,$ & $\,\,$1.4127$\,\,$ & $\,\,$ 1  $\,\,$ \\
\end{pmatrix},
\end{equation*}

\begin{equation*}
\mathbf{w}^{\prime} =
\begin{pmatrix}
0.553087\\
0.280108\\
0.069136\\
0.097669
\end{pmatrix} =
0.998128\cdot
\begin{pmatrix}
\color{gr} 0.554124\color{black} \\
0.280633\\
0.069265\\
0.097852
\end{pmatrix},
\end{equation*}
\begin{equation*}
\left[ \frac{{w}^{\prime}_i}{{w}^{\prime}_j} \right] =
\begin{pmatrix}
$\,\,$ 1 $\,\,$ & $\,\,$\color{gr} 1.9745\color{black} $\,\,$ & $\,\,$\color{gr} \color{blue} 8\color{black} $\,\,$ & $\,\,$\color{gr} 5.6629\color{black} $\,\,$ \\
$\,\,$\color{gr} 0.5064\color{black} $\,\,$ & $\,\,$ 1 $\,\,$ & $\,\,$4.0516$\,\,$ & $\,\,$2.8679  $\,\,$ \\
$\,\,$\color{gr} \color{blue}  1/8\color{black} $\,\,$ & $\,\,$0.2468$\,\,$ & $\,\,$ 1 $\,\,$ & $\,\,$0.7079 $\,\,$ \\
$\,\,$\color{gr} 0.1766\color{black} $\,\,$ & $\,\,$0.3487$\,\,$ & $\,\,$1.4127$\,\,$ & $\,\,$ 1  $\,\,$ \\
\end{pmatrix},
\end{equation*}
\end{example}
\newpage
\begin{example}
\begin{equation*}
\mathbf{A} =
\begin{pmatrix}
$\,\,$ 1 $\,\,$ & $\,\,$2$\,\,$ & $\,\,$8$\,\,$ & $\,\,$6 $\,\,$ \\
$\,\,$ 1/2$\,\,$ & $\,\,$ 1 $\,\,$ & $\,\,$3$\,\,$ & $\,\,$5 $\,\,$ \\
$\,\,$ 1/8$\,\,$ & $\,\,$ 1/3$\,\,$ & $\,\,$ 1 $\,\,$ & $\,\,$ 1/2 $\,\,$ \\
$\,\,$ 1/6$\,\,$ & $\,\,$ 1/5$\,\,$ & $\,\,$2$\,\,$ & $\,\,$ 1  $\,\,$ \\
\end{pmatrix},
\qquad
\lambda_{\max} =
4.1252,
\qquad
CR = 0.0472
\end{equation*}

\begin{equation*}
\mathbf{w}^{AMAST} =
\begin{pmatrix}
\color{red} 0.543984\color{black} \\
0.294271\\
0.069072\\
0.092674
\end{pmatrix}\end{equation*}
\begin{equation*}
\left[ \frac{{w}^{AMAST}_i}{{w}^{AMAST}_j} \right] =
\begin{pmatrix}
$\,\,$ 1 $\,\,$ & $\,\,$\color{red} 1.8486\color{black} $\,\,$ & $\,\,$\color{red} 7.8757\color{black} $\,\,$ & $\,\,$\color{red} 5.8699\color{black} $\,\,$ \\
$\,\,$\color{red} 0.5410\color{black} $\,\,$ & $\,\,$ 1 $\,\,$ & $\,\,$4.2604$\,\,$ & $\,\,$3.1753  $\,\,$ \\
$\,\,$\color{red} 0.1270\color{black} $\,\,$ & $\,\,$0.2347$\,\,$ & $\,\,$ 1 $\,\,$ & $\,\,$0.7453 $\,\,$ \\
$\,\,$\color{red} 0.1704\color{black} $\,\,$ & $\,\,$0.3149$\,\,$ & $\,\,$1.3417$\,\,$ & $\,\,$ 1  $\,\,$ \\
\end{pmatrix},
\end{equation*}

\begin{equation*}
\mathbf{w}^{\prime} =
\begin{pmatrix}
0.547867\\
0.291765\\
0.068483\\
0.091885
\end{pmatrix} =
0.991485\cdot
\begin{pmatrix}
\color{gr} 0.552573\color{black} \\
0.294271\\
0.069072\\
0.092674
\end{pmatrix},
\end{equation*}
\begin{equation*}
\left[ \frac{{w}^{\prime}_i}{{w}^{\prime}_j} \right] =
\begin{pmatrix}
$\,\,$ 1 $\,\,$ & $\,\,$\color{gr} 1.8778\color{black} $\,\,$ & $\,\,$\color{gr} \color{blue} 8\color{black} $\,\,$ & $\,\,$\color{gr} 5.9626\color{black} $\,\,$ \\
$\,\,$\color{gr} 0.5325\color{black} $\,\,$ & $\,\,$ 1 $\,\,$ & $\,\,$4.2604$\,\,$ & $\,\,$3.1753  $\,\,$ \\
$\,\,$\color{gr} \color{blue}  1/8\color{black} $\,\,$ & $\,\,$0.2347$\,\,$ & $\,\,$ 1 $\,\,$ & $\,\,$0.7453 $\,\,$ \\
$\,\,$\color{gr} 0.1677\color{black} $\,\,$ & $\,\,$0.3149$\,\,$ & $\,\,$1.3417$\,\,$ & $\,\,$ 1  $\,\,$ \\
\end{pmatrix},
\end{equation*}
\end{example}
\newpage
\begin{example}
\begin{equation*}
\mathbf{A} =
\begin{pmatrix}
$\,\,$ 1 $\,\,$ & $\,\,$2$\,\,$ & $\,\,$8$\,\,$ & $\,\,$7 $\,\,$ \\
$\,\,$ 1/2$\,\,$ & $\,\,$ 1 $\,\,$ & $\,\,$3$\,\,$ & $\,\,$7 $\,\,$ \\
$\,\,$ 1/8$\,\,$ & $\,\,$ 1/3$\,\,$ & $\,\,$ 1 $\,\,$ & $\,\,$ 1/2 $\,\,$ \\
$\,\,$ 1/7$\,\,$ & $\,\,$ 1/7$\,\,$ & $\,\,$2$\,\,$ & $\,\,$ 1  $\,\,$ \\
\end{pmatrix},
\qquad
\lambda_{\max} =
4.2109,
\qquad
CR = 0.0795
\end{equation*}

\begin{equation*}
\mathbf{w}^{AMAST} =
\begin{pmatrix}
\color{red} 0.541387\color{black} \\
0.309458\\
0.067890\\
0.081264
\end{pmatrix}\end{equation*}
\begin{equation*}
\left[ \frac{{w}^{AMAST}_i}{{w}^{AMAST}_j} \right] =
\begin{pmatrix}
$\,\,$ 1 $\,\,$ & $\,\,$\color{red} 1.7495\color{black} $\,\,$ & $\,\,$\color{red} 7.9745\color{black} $\,\,$ & $\,\,$\color{red} 6.6620\color{black} $\,\,$ \\
$\,\,$\color{red} 0.5716\color{black} $\,\,$ & $\,\,$ 1 $\,\,$ & $\,\,$4.5582$\,\,$ & $\,\,$3.8080  $\,\,$ \\
$\,\,$\color{red} 0.1254\color{black} $\,\,$ & $\,\,$0.2194$\,\,$ & $\,\,$ 1 $\,\,$ & $\,\,$0.8354 $\,\,$ \\
$\,\,$\color{red} 0.1501\color{black} $\,\,$ & $\,\,$0.2626$\,\,$ & $\,\,$1.1970$\,\,$ & $\,\,$ 1  $\,\,$ \\
\end{pmatrix},
\end{equation*}

\begin{equation*}
\mathbf{w}^{\prime} =
\begin{pmatrix}
0.542181\\
0.308922\\
0.067773\\
0.081124
\end{pmatrix} =
0.998269\cdot
\begin{pmatrix}
\color{gr} 0.543121\color{black} \\
0.309458\\
0.067890\\
0.081264
\end{pmatrix},
\end{equation*}
\begin{equation*}
\left[ \frac{{w}^{\prime}_i}{{w}^{\prime}_j} \right] =
\begin{pmatrix}
$\,\,$ 1 $\,\,$ & $\,\,$\color{gr} 1.7551\color{black} $\,\,$ & $\,\,$\color{gr} \color{blue} 8\color{black} $\,\,$ & $\,\,$\color{gr} 6.6834\color{black} $\,\,$ \\
$\,\,$\color{gr} 0.5698\color{black} $\,\,$ & $\,\,$ 1 $\,\,$ & $\,\,$4.5582$\,\,$ & $\,\,$3.8080  $\,\,$ \\
$\,\,$\color{gr} \color{blue}  1/8\color{black} $\,\,$ & $\,\,$0.2194$\,\,$ & $\,\,$ 1 $\,\,$ & $\,\,$0.8354 $\,\,$ \\
$\,\,$\color{gr} 0.1496\color{black} $\,\,$ & $\,\,$0.2626$\,\,$ & $\,\,$1.1970$\,\,$ & $\,\,$ 1  $\,\,$ \\
\end{pmatrix},
\end{equation*}
\end{example}
\newpage
\begin{example}
\begin{equation*}
\mathbf{A} =
\begin{pmatrix}
$\,\,$ 1 $\,\,$ & $\,\,$2$\,\,$ & $\,\,$8$\,\,$ & $\,\,$7 $\,\,$ \\
$\,\,$ 1/2$\,\,$ & $\,\,$ 1 $\,\,$ & $\,\,$3$\,\,$ & $\,\,$8 $\,\,$ \\
$\,\,$ 1/8$\,\,$ & $\,\,$ 1/3$\,\,$ & $\,\,$ 1 $\,\,$ & $\,\,$ 1/2 $\,\,$ \\
$\,\,$ 1/7$\,\,$ & $\,\,$ 1/8$\,\,$ & $\,\,$2$\,\,$ & $\,\,$ 1  $\,\,$ \\
\end{pmatrix},
\qquad
\lambda_{\max} =
4.2536,
\qquad
CR = 0.0956
\end{equation*}

\begin{equation*}
\mathbf{w}^{AMAST} =
\begin{pmatrix}
\color{red} 0.535641\color{black} \\
0.317607\\
0.067807\\
0.078945
\end{pmatrix}\end{equation*}
\begin{equation*}
\left[ \frac{{w}^{AMAST}_i}{{w}^{AMAST}_j} \right] =
\begin{pmatrix}
$\,\,$ 1 $\,\,$ & $\,\,$\color{red} 1.6865\color{black} $\,\,$ & $\,\,$\color{red} 7.8994\color{black} $\,\,$ & $\,\,$\color{red} 6.7850\color{black} $\,\,$ \\
$\,\,$\color{red} 0.5929\color{black} $\,\,$ & $\,\,$ 1 $\,\,$ & $\,\,$4.6840$\,\,$ & $\,\,$4.0232  $\,\,$ \\
$\,\,$\color{red} 0.1266\color{black} $\,\,$ & $\,\,$0.2135$\,\,$ & $\,\,$ 1 $\,\,$ & $\,\,$0.8589 $\,\,$ \\
$\,\,$\color{red} 0.1474\color{black} $\,\,$ & $\,\,$0.2486$\,\,$ & $\,\,$1.1642$\,\,$ & $\,\,$ 1  $\,\,$ \\
\end{pmatrix},
\end{equation*}

\begin{equation*}
\mathbf{w}^{\prime} =
\begin{pmatrix}
0.538786\\
0.315456\\
0.067348\\
0.078410
\end{pmatrix} =
0.993227\cdot
\begin{pmatrix}
\color{gr} 0.542460\color{black} \\
0.317607\\
0.067807\\
0.078945
\end{pmatrix},
\end{equation*}
\begin{equation*}
\left[ \frac{{w}^{\prime}_i}{{w}^{\prime}_j} \right] =
\begin{pmatrix}
$\,\,$ 1 $\,\,$ & $\,\,$\color{gr} 1.7080\color{black} $\,\,$ & $\,\,$\color{gr} \color{blue} 8\color{black} $\,\,$ & $\,\,$\color{gr} 6.8714\color{black} $\,\,$ \\
$\,\,$\color{gr} 0.5855\color{black} $\,\,$ & $\,\,$ 1 $\,\,$ & $\,\,$4.6840$\,\,$ & $\,\,$4.0232  $\,\,$ \\
$\,\,$\color{gr} \color{blue}  1/8\color{black} $\,\,$ & $\,\,$0.2135$\,\,$ & $\,\,$ 1 $\,\,$ & $\,\,$0.8589 $\,\,$ \\
$\,\,$\color{gr} 0.1455\color{black} $\,\,$ & $\,\,$0.2486$\,\,$ & $\,\,$1.1642$\,\,$ & $\,\,$ 1  $\,\,$ \\
\end{pmatrix},
\end{equation*}
\end{example}
\newpage
\begin{example}
\begin{equation*}
\mathbf{A} =
\begin{pmatrix}
$\,\,$ 1 $\,\,$ & $\,\,$2$\,\,$ & $\,\,$9$\,\,$ & $\,\,$2 $\,\,$ \\
$\,\,$ 1/2$\,\,$ & $\,\,$ 1 $\,\,$ & $\,\,$3$\,\,$ & $\,\,$2 $\,\,$ \\
$\,\,$ 1/9$\,\,$ & $\,\,$ 1/3$\,\,$ & $\,\,$ 1 $\,\,$ & $\,\,$ 1/8 $\,\,$ \\
$\,\,$ 1/2$\,\,$ & $\,\,$ 1/2$\,\,$ & $\,\,$8$\,\,$ & $\,\,$ 1  $\,\,$ \\
\end{pmatrix},
\qquad
\lambda_{\max} =
4.2469,
\qquad
CR = 0.0931
\end{equation*}

\begin{equation*}
\mathbf{w}^{AMAST} =
\begin{pmatrix}
\color{red} 0.456232\color{black} \\
0.257000\\
0.052969\\
0.233799
\end{pmatrix}\end{equation*}
\begin{equation*}
\left[ \frac{{w}^{AMAST}_i}{{w}^{AMAST}_j} \right] =
\begin{pmatrix}
$\,\,$ 1 $\,\,$ & $\,\,$\color{red} 1.7752\color{black} $\,\,$ & $\,\,$\color{red} 8.6132\color{black} $\,\,$ & $\,\,$\color{red} 1.9514\color{black} $\,\,$ \\
$\,\,$\color{red} 0.5633\color{black} $\,\,$ & $\,\,$ 1 $\,\,$ & $\,\,$4.8519$\,\,$ & $\,\,$1.0992  $\,\,$ \\
$\,\,$\color{red} 0.1161\color{black} $\,\,$ & $\,\,$0.2061$\,\,$ & $\,\,$ 1 $\,\,$ & $\,\,$0.2266 $\,\,$ \\
$\,\,$\color{red} 0.5125\color{black} $\,\,$ & $\,\,$0.9097$\,\,$ & $\,\,$4.4139$\,\,$ & $\,\,$ 1  $\,\,$ \\
\end{pmatrix},
\end{equation*}

\begin{equation*}
\mathbf{w}^{\prime} =
\begin{pmatrix}
0.462343\\
0.254112\\
0.052374\\
0.231171
\end{pmatrix} =
0.988761\cdot
\begin{pmatrix}
\color{gr} 0.467598\color{black} \\
0.257000\\
0.052969\\
0.233799
\end{pmatrix},
\end{equation*}
\begin{equation*}
\left[ \frac{{w}^{\prime}_i}{{w}^{\prime}_j} \right] =
\begin{pmatrix}
$\,\,$ 1 $\,\,$ & $\,\,$\color{gr} 1.8194\color{black} $\,\,$ & $\,\,$\color{gr} 8.8278\color{black} $\,\,$ & $\,\,$\color{gr} \color{blue} 2\color{black} $\,\,$ \\
$\,\,$\color{gr} 0.5496\color{black} $\,\,$ & $\,\,$ 1 $\,\,$ & $\,\,$4.8519$\,\,$ & $\,\,$1.0992  $\,\,$ \\
$\,\,$\color{gr} 0.1133\color{black} $\,\,$ & $\,\,$0.2061$\,\,$ & $\,\,$ 1 $\,\,$ & $\,\,$0.2266 $\,\,$ \\
$\,\,$\color{gr} \color{blue}  1/2\color{black} $\,\,$ & $\,\,$0.9097$\,\,$ & $\,\,$4.4139$\,\,$ & $\,\,$ 1  $\,\,$ \\
\end{pmatrix},
\end{equation*}
\end{example}
\newpage
\begin{example}
\begin{equation*}
\mathbf{A} =
\begin{pmatrix}
$\,\,$ 1 $\,\,$ & $\,\,$2$\,\,$ & $\,\,$9$\,\,$ & $\,\,$3 $\,\,$ \\
$\,\,$ 1/2$\,\,$ & $\,\,$ 1 $\,\,$ & $\,\,$3$\,\,$ & $\,\,$3 $\,\,$ \\
$\,\,$ 1/9$\,\,$ & $\,\,$ 1/3$\,\,$ & $\,\,$ 1 $\,\,$ & $\,\,$ 1/5 $\,\,$ \\
$\,\,$ 1/3$\,\,$ & $\,\,$ 1/3$\,\,$ & $\,\,$5$\,\,$ & $\,\,$ 1  $\,\,$ \\
\end{pmatrix},
\qquad
\lambda_{\max} =
4.2277,
\qquad
CR = 0.0859
\end{equation*}

\begin{equation*}
\mathbf{w}^{AMAST} =
\begin{pmatrix}
\color{red} 0.495590\color{black} \\
0.277711\\
0.058078\\
0.168621
\end{pmatrix}\end{equation*}
\begin{equation*}
\left[ \frac{{w}^{AMAST}_i}{{w}^{AMAST}_j} \right] =
\begin{pmatrix}
$\,\,$ 1 $\,\,$ & $\,\,$\color{red} 1.7846\color{black} $\,\,$ & $\,\,$\color{red} 8.5332\color{black} $\,\,$ & $\,\,$\color{red} 2.9391\color{black} $\,\,$ \\
$\,\,$\color{red} 0.5604\color{black} $\,\,$ & $\,\,$ 1 $\,\,$ & $\,\,$4.7817$\,\,$ & $\,\,$1.6470  $\,\,$ \\
$\,\,$\color{red} 0.1172\color{black} $\,\,$ & $\,\,$0.2091$\,\,$ & $\,\,$ 1 $\,\,$ & $\,\,$0.3444 $\,\,$ \\
$\,\,$\color{red} 0.3402\color{black} $\,\,$ & $\,\,$0.6072$\,\,$ & $\,\,$2.9034$\,\,$ & $\,\,$ 1  $\,\,$ \\
\end{pmatrix},
\end{equation*}

\begin{equation*}
\mathbf{w}^{\prime} =
\begin{pmatrix}
0.500719\\
0.274887\\
0.057487\\
0.166906
\end{pmatrix} =
0.989832\cdot
\begin{pmatrix}
\color{gr} 0.505863\color{black} \\
0.277711\\
0.058078\\
0.168621
\end{pmatrix},
\end{equation*}
\begin{equation*}
\left[ \frac{{w}^{\prime}_i}{{w}^{\prime}_j} \right] =
\begin{pmatrix}
$\,\,$ 1 $\,\,$ & $\,\,$\color{gr} 1.8215\color{black} $\,\,$ & $\,\,$\color{gr} 8.7101\color{black} $\,\,$ & $\,\,$\color{gr} \color{blue} 3\color{black} $\,\,$ \\
$\,\,$\color{gr} 0.5490\color{black} $\,\,$ & $\,\,$ 1 $\,\,$ & $\,\,$4.7817$\,\,$ & $\,\,$1.6470  $\,\,$ \\
$\,\,$\color{gr} 0.1148\color{black} $\,\,$ & $\,\,$0.2091$\,\,$ & $\,\,$ 1 $\,\,$ & $\,\,$0.3444 $\,\,$ \\
$\,\,$\color{gr} \color{blue}  1/3\color{black} $\,\,$ & $\,\,$0.6072$\,\,$ & $\,\,$2.9034$\,\,$ & $\,\,$ 1  $\,\,$ \\
\end{pmatrix},
\end{equation*}
\end{example}
\newpage
\begin{example}
\begin{equation*}
\mathbf{A} =
\begin{pmatrix}
$\,\,$ 1 $\,\,$ & $\,\,$2$\,\,$ & $\,\,$9$\,\,$ & $\,\,$4 $\,\,$ \\
$\,\,$ 1/2$\,\,$ & $\,\,$ 1 $\,\,$ & $\,\,$3$\,\,$ & $\,\,$3 $\,\,$ \\
$\,\,$ 1/9$\,\,$ & $\,\,$ 1/3$\,\,$ & $\,\,$ 1 $\,\,$ & $\,\,$ 1/3 $\,\,$ \\
$\,\,$ 1/4$\,\,$ & $\,\,$ 1/3$\,\,$ & $\,\,$3$\,\,$ & $\,\,$ 1  $\,\,$ \\
\end{pmatrix},
\qquad
\lambda_{\max} =
4.1031,
\qquad
CR = 0.0389
\end{equation*}

\begin{equation*}
\mathbf{w}^{AMAST} =
\begin{pmatrix}
\color{red} 0.531453\color{black} \\
0.271785\\
0.063324\\
0.133439
\end{pmatrix}\end{equation*}
\begin{equation*}
\left[ \frac{{w}^{AMAST}_i}{{w}^{AMAST}_j} \right] =
\begin{pmatrix}
$\,\,$ 1 $\,\,$ & $\,\,$\color{red} 1.9554\color{black} $\,\,$ & $\,\,$\color{red} 8.3927\color{black} $\,\,$ & $\,\,$\color{red} 3.9828\color{black} $\,\,$ \\
$\,\,$\color{red} 0.5114\color{black} $\,\,$ & $\,\,$ 1 $\,\,$ & $\,\,$4.2920$\,\,$ & $\,\,$2.0368  $\,\,$ \\
$\,\,$\color{red} 0.1192\color{black} $\,\,$ & $\,\,$0.2330$\,\,$ & $\,\,$ 1 $\,\,$ & $\,\,$0.4746 $\,\,$ \\
$\,\,$\color{red} 0.2511\color{black} $\,\,$ & $\,\,$0.4910$\,\,$ & $\,\,$2.1073$\,\,$ & $\,\,$ 1  $\,\,$ \\
\end{pmatrix},
\end{equation*}

\begin{equation*}
\mathbf{w}^{\prime} =
\begin{pmatrix}
0.532529\\
0.271161\\
0.063178\\
0.133132
\end{pmatrix} =
0.997704\cdot
\begin{pmatrix}
\color{gr} 0.533754\color{black} \\
0.271785\\
0.063324\\
0.133439
\end{pmatrix},
\end{equation*}
\begin{equation*}
\left[ \frac{{w}^{\prime}_i}{{w}^{\prime}_j} \right] =
\begin{pmatrix}
$\,\,$ 1 $\,\,$ & $\,\,$\color{gr} 1.9639\color{black} $\,\,$ & $\,\,$\color{gr} 8.4290\color{black} $\,\,$ & $\,\,$\color{gr} \color{blue} 4\color{black} $\,\,$ \\
$\,\,$\color{gr} 0.5092\color{black} $\,\,$ & $\,\,$ 1 $\,\,$ & $\,\,$4.2920$\,\,$ & $\,\,$2.0368  $\,\,$ \\
$\,\,$\color{gr} 0.1186\color{black} $\,\,$ & $\,\,$0.2330$\,\,$ & $\,\,$ 1 $\,\,$ & $\,\,$0.4746 $\,\,$ \\
$\,\,$\color{gr} \color{blue}  1/4\color{black} $\,\,$ & $\,\,$0.4910$\,\,$ & $\,\,$2.1073$\,\,$ & $\,\,$ 1  $\,\,$ \\
\end{pmatrix},
\end{equation*}
\end{example}
\newpage
\begin{example}
\begin{equation*}
\mathbf{A} =
\begin{pmatrix}
$\,\,$ 1 $\,\,$ & $\,\,$2$\,\,$ & $\,\,$9$\,\,$ & $\,\,$4 $\,\,$ \\
$\,\,$ 1/2$\,\,$ & $\,\,$ 1 $\,\,$ & $\,\,$3$\,\,$ & $\,\,$3 $\,\,$ \\
$\,\,$ 1/9$\,\,$ & $\,\,$ 1/3$\,\,$ & $\,\,$ 1 $\,\,$ & $\,\,$ 1/4 $\,\,$ \\
$\,\,$ 1/4$\,\,$ & $\,\,$ 1/3$\,\,$ & $\,\,$4$\,\,$ & $\,\,$ 1  $\,\,$ \\
\end{pmatrix},
\qquad
\lambda_{\max} =
4.1664,
\qquad
CR = 0.0627
\end{equation*}

\begin{equation*}
\mathbf{w}^{AMAST} =
\begin{pmatrix}
\color{red} 0.524837\color{black} \\
0.270989\\
0.059651\\
0.144523
\end{pmatrix}\end{equation*}
\begin{equation*}
\left[ \frac{{w}^{AMAST}_i}{{w}^{AMAST}_j} \right] =
\begin{pmatrix}
$\,\,$ 1 $\,\,$ & $\,\,$\color{red} 1.9367\color{black} $\,\,$ & $\,\,$\color{red} 8.7984\color{black} $\,\,$ & $\,\,$\color{red} 3.6315\color{black} $\,\,$ \\
$\,\,$\color{red} 0.5163\color{black} $\,\,$ & $\,\,$ 1 $\,\,$ & $\,\,$4.5429$\,\,$ & $\,\,$1.8751  $\,\,$ \\
$\,\,$\color{red} 0.1137\color{black} $\,\,$ & $\,\,$0.2201$\,\,$ & $\,\,$ 1 $\,\,$ & $\,\,$0.4127 $\,\,$ \\
$\,\,$\color{red} 0.2754\color{black} $\,\,$ & $\,\,$0.5333$\,\,$ & $\,\,$2.4228$\,\,$ & $\,\,$ 1  $\,\,$ \\
\end{pmatrix},
\end{equation*}

\begin{equation*}
\mathbf{w}^{\prime} =
\begin{pmatrix}
0.530483\\
0.267769\\
0.058943\\
0.142805
\end{pmatrix} =
0.988117\cdot
\begin{pmatrix}
\color{gr} 0.536863\color{black} \\
0.270989\\
0.059651\\
0.144523
\end{pmatrix},
\end{equation*}
\begin{equation*}
\left[ \frac{{w}^{\prime}_i}{{w}^{\prime}_j} \right] =
\begin{pmatrix}
$\,\,$ 1 $\,\,$ & $\,\,$\color{gr} 1.9811\color{black} $\,\,$ & $\,\,$\color{gr} \color{blue} 9\color{black} $\,\,$ & $\,\,$\color{gr} 3.7147\color{black} $\,\,$ \\
$\,\,$\color{gr} 0.5048\color{black} $\,\,$ & $\,\,$ 1 $\,\,$ & $\,\,$4.5429$\,\,$ & $\,\,$1.8751  $\,\,$ \\
$\,\,$\color{gr} \color{blue}  1/9\color{black} $\,\,$ & $\,\,$0.2201$\,\,$ & $\,\,$ 1 $\,\,$ & $\,\,$0.4127 $\,\,$ \\
$\,\,$\color{gr} 0.2692\color{black} $\,\,$ & $\,\,$0.5333$\,\,$ & $\,\,$2.4228$\,\,$ & $\,\,$ 1  $\,\,$ \\
\end{pmatrix},
\end{equation*}
\end{example}
\newpage
\begin{example}
\begin{equation*}
\mathbf{A} =
\begin{pmatrix}
$\,\,$ 1 $\,\,$ & $\,\,$2$\,\,$ & $\,\,$9$\,\,$ & $\,\,$4 $\,\,$ \\
$\,\,$ 1/2$\,\,$ & $\,\,$ 1 $\,\,$ & $\,\,$3$\,\,$ & $\,\,$4 $\,\,$ \\
$\,\,$ 1/9$\,\,$ & $\,\,$ 1/3$\,\,$ & $\,\,$ 1 $\,\,$ & $\,\,$ 1/4 $\,\,$ \\
$\,\,$ 1/4$\,\,$ & $\,\,$ 1/4$\,\,$ & $\,\,$4$\,\,$ & $\,\,$ 1  $\,\,$ \\
\end{pmatrix},
\qquad
\lambda_{\max} =
4.2469,
\qquad
CR = 0.0931
\end{equation*}

\begin{equation*}
\mathbf{w}^{AMAST} =
\begin{pmatrix}
\color{red} 0.515475\color{black} \\
0.289040\\
0.059860\\
0.135626
\end{pmatrix}\end{equation*}
\begin{equation*}
\left[ \frac{{w}^{AMAST}_i}{{w}^{AMAST}_j} \right] =
\begin{pmatrix}
$\,\,$ 1 $\,\,$ & $\,\,$\color{red} 1.7834\color{black} $\,\,$ & $\,\,$\color{red} 8.6114\color{black} $\,\,$ & $\,\,$\color{red} 3.8007\color{black} $\,\,$ \\
$\,\,$\color{red} 0.5607\color{black} $\,\,$ & $\,\,$ 1 $\,\,$ & $\,\,$4.8286$\,\,$ & $\,\,$2.1312  $\,\,$ \\
$\,\,$\color{red} 0.1161\color{black} $\,\,$ & $\,\,$0.2071$\,\,$ & $\,\,$ 1 $\,\,$ & $\,\,$0.4414 $\,\,$ \\
$\,\,$\color{red} 0.2631\color{black} $\,\,$ & $\,\,$0.4692$\,\,$ & $\,\,$2.2657$\,\,$ & $\,\,$ 1  $\,\,$ \\
\end{pmatrix},
\end{equation*}

\begin{equation*}
\mathbf{w}^{\prime} =
\begin{pmatrix}
0.526490\\
0.282469\\
0.058499\\
0.132542
\end{pmatrix} =
0.977265\cdot
\begin{pmatrix}
\color{gr} 0.538738\color{black} \\
0.289040\\
0.059860\\
0.135626
\end{pmatrix},
\end{equation*}
\begin{equation*}
\left[ \frac{{w}^{\prime}_i}{{w}^{\prime}_j} \right] =
\begin{pmatrix}
$\,\,$ 1 $\,\,$ & $\,\,$\color{gr} 1.8639\color{black} $\,\,$ & $\,\,$\color{gr} \color{blue} 9\color{black} $\,\,$ & $\,\,$\color{gr} 3.9722\color{black} $\,\,$ \\
$\,\,$\color{gr} 0.5365\color{black} $\,\,$ & $\,\,$ 1 $\,\,$ & $\,\,$4.8286$\,\,$ & $\,\,$2.1312  $\,\,$ \\
$\,\,$\color{gr} \color{blue}  1/9\color{black} $\,\,$ & $\,\,$0.2071$\,\,$ & $\,\,$ 1 $\,\,$ & $\,\,$0.4414 $\,\,$ \\
$\,\,$\color{gr} 0.2517\color{black} $\,\,$ & $\,\,$0.4692$\,\,$ & $\,\,$2.2657$\,\,$ & $\,\,$ 1  $\,\,$ \\
\end{pmatrix},
\end{equation*}
\end{example}
\newpage
\begin{example}
\begin{equation*}
\mathbf{A} =
\begin{pmatrix}
$\,\,$ 1 $\,\,$ & $\,\,$2$\,\,$ & $\,\,$9$\,\,$ & $\,\,$5 $\,\,$ \\
$\,\,$ 1/2$\,\,$ & $\,\,$ 1 $\,\,$ & $\,\,$3$\,\,$ & $\,\,$4 $\,\,$ \\
$\,\,$ 1/9$\,\,$ & $\,\,$ 1/3$\,\,$ & $\,\,$ 1 $\,\,$ & $\,\,$ 1/3 $\,\,$ \\
$\,\,$ 1/5$\,\,$ & $\,\,$ 1/4$\,\,$ & $\,\,$3$\,\,$ & $\,\,$ 1  $\,\,$ \\
\end{pmatrix},
\qquad
\lambda_{\max} =
4.1655,
\qquad
CR = 0.0624
\end{equation*}

\begin{equation*}
\mathbf{w}^{AMAST} =
\begin{pmatrix}
\color{red} 0.539348\color{black} \\
0.282854\\
0.062203\\
0.115595
\end{pmatrix}\end{equation*}
\begin{equation*}
\left[ \frac{{w}^{AMAST}_i}{{w}^{AMAST}_j} \right] =
\begin{pmatrix}
$\,\,$ 1 $\,\,$ & $\,\,$\color{red} 1.9068\color{black} $\,\,$ & $\,\,$\color{red} 8.6708\color{black} $\,\,$ & $\,\,$\color{red} 4.6658\color{black} $\,\,$ \\
$\,\,$\color{red} 0.5244\color{black} $\,\,$ & $\,\,$ 1 $\,\,$ & $\,\,$4.5473$\,\,$ & $\,\,$2.4469  $\,\,$ \\
$\,\,$\color{red} 0.1153\color{black} $\,\,$ & $\,\,$0.2199$\,\,$ & $\,\,$ 1 $\,\,$ & $\,\,$0.5381 $\,\,$ \\
$\,\,$\color{red} 0.2143\color{black} $\,\,$ & $\,\,$0.4087$\,\,$ & $\,\,$1.8584$\,\,$ & $\,\,$ 1  $\,\,$ \\
\end{pmatrix},
\end{equation*}

\begin{equation*}
\mathbf{w}^{\prime} =
\begin{pmatrix}
0.548592\\
0.277178\\
0.060955\\
0.113275
\end{pmatrix} =
0.979933\cdot
\begin{pmatrix}
\color{gr} 0.559826\color{black} \\
0.282854\\
0.062203\\
0.115595
\end{pmatrix},
\end{equation*}
\begin{equation*}
\left[ \frac{{w}^{\prime}_i}{{w}^{\prime}_j} \right] =
\begin{pmatrix}
$\,\,$ 1 $\,\,$ & $\,\,$\color{gr} 1.9792\color{black} $\,\,$ & $\,\,$\color{gr} \color{blue} 9\color{black} $\,\,$ & $\,\,$\color{gr} 4.8430\color{black} $\,\,$ \\
$\,\,$\color{gr} 0.5053\color{black} $\,\,$ & $\,\,$ 1 $\,\,$ & $\,\,$4.5473$\,\,$ & $\,\,$2.4469  $\,\,$ \\
$\,\,$\color{gr} \color{blue}  1/9\color{black} $\,\,$ & $\,\,$0.2199$\,\,$ & $\,\,$ 1 $\,\,$ & $\,\,$0.5381 $\,\,$ \\
$\,\,$\color{gr} 0.2065\color{black} $\,\,$ & $\,\,$0.4087$\,\,$ & $\,\,$1.8584$\,\,$ & $\,\,$ 1  $\,\,$ \\
\end{pmatrix},
\end{equation*}
\end{example}
\newpage
\begin{example}
\begin{equation*}
\mathbf{A} =
\begin{pmatrix}
$\,\,$ 1 $\,\,$ & $\,\,$2$\,\,$ & $\,\,$9$\,\,$ & $\,\,$5 $\,\,$ \\
$\,\,$ 1/2$\,\,$ & $\,\,$ 1 $\,\,$ & $\,\,$3$\,\,$ & $\,\,$5 $\,\,$ \\
$\,\,$ 1/9$\,\,$ & $\,\,$ 1/3$\,\,$ & $\,\,$ 1 $\,\,$ & $\,\,$ 1/3 $\,\,$ \\
$\,\,$ 1/5$\,\,$ & $\,\,$ 1/5$\,\,$ & $\,\,$3$\,\,$ & $\,\,$ 1  $\,\,$ \\
\end{pmatrix},
\qquad
\lambda_{\max} =
4.2277,
\qquad
CR = 0.0859
\end{equation*}

\begin{equation*}
\mathbf{w}^{AMAST} =
\begin{pmatrix}
\color{red} 0.531040\color{black} \\
0.296730\\
0.062247\\
0.109982
\end{pmatrix}\end{equation*}
\begin{equation*}
\left[ \frac{{w}^{AMAST}_i}{{w}^{AMAST}_j} \right] =
\begin{pmatrix}
$\,\,$ 1 $\,\,$ & $\,\,$\color{red} 1.7896\color{black} $\,\,$ & $\,\,$\color{red} 8.5312\color{black} $\,\,$ & $\,\,$\color{red} 4.8284\color{black} $\,\,$ \\
$\,\,$\color{red} 0.5588\color{black} $\,\,$ & $\,\,$ 1 $\,\,$ & $\,\,$4.7670$\,\,$ & $\,\,$2.6980  $\,\,$ \\
$\,\,$\color{red} 0.1172\color{black} $\,\,$ & $\,\,$0.2098$\,\,$ & $\,\,$ 1 $\,\,$ & $\,\,$0.5660 $\,\,$ \\
$\,\,$\color{red} 0.2071\color{black} $\,\,$ & $\,\,$0.3706$\,\,$ & $\,\,$1.7669$\,\,$ & $\,\,$ 1  $\,\,$ \\
\end{pmatrix},
\end{equation*}

\begin{equation*}
\mathbf{w}^{\prime} =
\begin{pmatrix}
0.539726\\
0.291235\\
0.061094\\
0.107945
\end{pmatrix} =
0.981479\cdot
\begin{pmatrix}
\color{gr} 0.549911\color{black} \\
0.296730\\
0.062247\\
0.109982
\end{pmatrix},
\end{equation*}
\begin{equation*}
\left[ \frac{{w}^{\prime}_i}{{w}^{\prime}_j} \right] =
\begin{pmatrix}
$\,\,$ 1 $\,\,$ & $\,\,$\color{gr} 1.8532\color{black} $\,\,$ & $\,\,$\color{gr} 8.8343\color{black} $\,\,$ & $\,\,$\color{gr} \color{blue} 5\color{black} $\,\,$ \\
$\,\,$\color{gr} 0.5396\color{black} $\,\,$ & $\,\,$ 1 $\,\,$ & $\,\,$4.7670$\,\,$ & $\,\,$2.6980  $\,\,$ \\
$\,\,$\color{gr} 0.1132\color{black} $\,\,$ & $\,\,$0.2098$\,\,$ & $\,\,$ 1 $\,\,$ & $\,\,$0.5660 $\,\,$ \\
$\,\,$\color{gr} \color{blue}  1/5\color{black} $\,\,$ & $\,\,$0.3706$\,\,$ & $\,\,$1.7669$\,\,$ & $\,\,$ 1  $\,\,$ \\
\end{pmatrix},
\end{equation*}
\end{example}
\newpage
\begin{example}
\begin{equation*}
\mathbf{A} =
\begin{pmatrix}
$\,\,$ 1 $\,\,$ & $\,\,$2$\,\,$ & $\,\,$9$\,\,$ & $\,\,$6 $\,\,$ \\
$\,\,$ 1/2$\,\,$ & $\,\,$ 1 $\,\,$ & $\,\,$3$\,\,$ & $\,\,$5 $\,\,$ \\
$\,\,$ 1/9$\,\,$ & $\,\,$ 1/3$\,\,$ & $\,\,$ 1 $\,\,$ & $\,\,$ 1/3 $\,\,$ \\
$\,\,$ 1/6$\,\,$ & $\,\,$ 1/5$\,\,$ & $\,\,$3$\,\,$ & $\,\,$ 1  $\,\,$ \\
\end{pmatrix},
\qquad
\lambda_{\max} =
4.2277,
\qquad
CR = 0.0859
\end{equation*}

\begin{equation*}
\mathbf{w}^{AMAST} =
\begin{pmatrix}
\color{red} 0.544544\color{black} \\
0.290641\\
0.061358\\
0.103457
\end{pmatrix}\end{equation*}
\begin{equation*}
\left[ \frac{{w}^{AMAST}_i}{{w}^{AMAST}_j} \right] =
\begin{pmatrix}
$\,\,$ 1 $\,\,$ & $\,\,$\color{red} 1.8736\color{black} $\,\,$ & $\,\,$\color{red} 8.8748\color{black} $\,\,$ & $\,\,$\color{red} 5.2635\color{black} $\,\,$ \\
$\,\,$\color{red} 0.5337\color{black} $\,\,$ & $\,\,$ 1 $\,\,$ & $\,\,$4.7368$\,\,$ & $\,\,$2.8093  $\,\,$ \\
$\,\,$\color{red} 0.1127\color{black} $\,\,$ & $\,\,$0.2111$\,\,$ & $\,\,$ 1 $\,\,$ & $\,\,$0.5931 $\,\,$ \\
$\,\,$\color{red} 0.1900\color{black} $\,\,$ & $\,\,$0.3560$\,\,$ & $\,\,$1.6861$\,\,$ & $\,\,$ 1  $\,\,$ \\
\end{pmatrix},
\end{equation*}

\begin{equation*}
\mathbf{w}^{\prime} =
\begin{pmatrix}
0.548016\\
0.288425\\
0.060891\\
0.102668
\end{pmatrix} =
0.992376\cdot
\begin{pmatrix}
\color{gr} 0.552226\color{black} \\
0.290641\\
0.061358\\
0.103457
\end{pmatrix},
\end{equation*}
\begin{equation*}
\left[ \frac{{w}^{\prime}_i}{{w}^{\prime}_j} \right] =
\begin{pmatrix}
$\,\,$ 1 $\,\,$ & $\,\,$\color{gr} 1.9000\color{black} $\,\,$ & $\,\,$\color{gr} \color{blue} 9\color{black} $\,\,$ & $\,\,$\color{gr} 5.3377\color{black} $\,\,$ \\
$\,\,$\color{gr} 0.5263\color{black} $\,\,$ & $\,\,$ 1 $\,\,$ & $\,\,$4.7368$\,\,$ & $\,\,$2.8093  $\,\,$ \\
$\,\,$\color{gr} \color{blue}  1/9\color{black} $\,\,$ & $\,\,$0.2111$\,\,$ & $\,\,$ 1 $\,\,$ & $\,\,$0.5931 $\,\,$ \\
$\,\,$\color{gr} 0.1873\color{black} $\,\,$ & $\,\,$0.3560$\,\,$ & $\,\,$1.6861$\,\,$ & $\,\,$ 1  $\,\,$ \\
\end{pmatrix},
\end{equation*}
\end{example}
\newpage
\begin{example}
\begin{equation*}
\mathbf{A} =
\begin{pmatrix}
$\,\,$ 1 $\,\,$ & $\,\,$2$\,\,$ & $\,\,$9$\,\,$ & $\,\,$7 $\,\,$ \\
$\,\,$ 1/2$\,\,$ & $\,\,$ 1 $\,\,$ & $\,\,$3$\,\,$ & $\,\,$5 $\,\,$ \\
$\,\,$ 1/9$\,\,$ & $\,\,$ 1/3$\,\,$ & $\,\,$ 1 $\,\,$ & $\,\,$ 1/2 $\,\,$ \\
$\,\,$ 1/7$\,\,$ & $\,\,$ 1/5$\,\,$ & $\,\,$2$\,\,$ & $\,\,$ 1  $\,\,$ \\
\end{pmatrix},
\qquad
\lambda_{\max} =
4.1239,
\qquad
CR = 0.0467
\end{equation*}

\begin{equation*}
\mathbf{w}^{AMAST} =
\begin{pmatrix}
\color{red} 0.563299\color{black} \\
0.285209\\
0.065213\\
0.086278
\end{pmatrix}\end{equation*}
\begin{equation*}
\left[ \frac{{w}^{AMAST}_i}{{w}^{AMAST}_j} \right] =
\begin{pmatrix}
$\,\,$ 1 $\,\,$ & $\,\,$\color{red} 1.9750\color{black} $\,\,$ & $\,\,$\color{red} 8.6378\color{black} $\,\,$ & $\,\,$\color{red} 6.5288\color{black} $\,\,$ \\
$\,\,$\color{red} 0.5063\color{black} $\,\,$ & $\,\,$ 1 $\,\,$ & $\,\,$4.3735$\,\,$ & $\,\,$3.3057  $\,\,$ \\
$\,\,$\color{red} 0.1158\color{black} $\,\,$ & $\,\,$0.2287$\,\,$ & $\,\,$ 1 $\,\,$ & $\,\,$0.7558 $\,\,$ \\
$\,\,$\color{red} 0.1532\color{black} $\,\,$ & $\,\,$0.3025$\,\,$ & $\,\,$1.3230$\,\,$ & $\,\,$ 1  $\,\,$ \\
\end{pmatrix},
\end{equation*}

\begin{equation*}
\mathbf{w}^{\prime} =
\begin{pmatrix}
0.566386\\
0.283193\\
0.064752\\
0.085668
\end{pmatrix} =
0.992930\cdot
\begin{pmatrix}
\color{gr} 0.570419\color{black} \\
0.285209\\
0.065213\\
0.086278
\end{pmatrix},
\end{equation*}
\begin{equation*}
\left[ \frac{{w}^{\prime}_i}{{w}^{\prime}_j} \right] =
\begin{pmatrix}
$\,\,$ 1 $\,\,$ & $\,\,$\color{gr} \color{blue} 2\color{black} $\,\,$ & $\,\,$\color{gr} 8.7470\color{black} $\,\,$ & $\,\,$\color{gr} 6.6114\color{black} $\,\,$ \\
$\,\,$\color{gr} \color{blue}  1/2\color{black} $\,\,$ & $\,\,$ 1 $\,\,$ & $\,\,$4.3735$\,\,$ & $\,\,$3.3057  $\,\,$ \\
$\,\,$\color{gr} 0.1143\color{black} $\,\,$ & $\,\,$0.2287$\,\,$ & $\,\,$ 1 $\,\,$ & $\,\,$0.7558 $\,\,$ \\
$\,\,$\color{gr} 0.1513\color{black} $\,\,$ & $\,\,$0.3025$\,\,$ & $\,\,$1.3230$\,\,$ & $\,\,$ 1  $\,\,$ \\
\end{pmatrix},
\end{equation*}
\end{example}
\newpage
\begin{example}
\begin{equation*}
\mathbf{A} =
\begin{pmatrix}
$\,\,$ 1 $\,\,$ & $\,\,$2$\,\,$ & $\,\,$9$\,\,$ & $\,\,$7 $\,\,$ \\
$\,\,$ 1/2$\,\,$ & $\,\,$ 1 $\,\,$ & $\,\,$3$\,\,$ & $\,\,$6 $\,\,$ \\
$\,\,$ 1/9$\,\,$ & $\,\,$ 1/3$\,\,$ & $\,\,$ 1 $\,\,$ & $\,\,$ 1/2 $\,\,$ \\
$\,\,$ 1/7$\,\,$ & $\,\,$ 1/6$\,\,$ & $\,\,$2$\,\,$ & $\,\,$ 1  $\,\,$ \\
\end{pmatrix},
\qquad
\lambda_{\max} =
4.1658,
\qquad
CR = 0.0625
\end{equation*}

\begin{equation*}
\mathbf{w}^{AMAST} =
\begin{pmatrix}
\color{red} 0.555884\color{black} \\
0.296331\\
0.065127\\
0.082658
\end{pmatrix}\end{equation*}
\begin{equation*}
\left[ \frac{{w}^{AMAST}_i}{{w}^{AMAST}_j} \right] =
\begin{pmatrix}
$\,\,$ 1 $\,\,$ & $\,\,$\color{red} 1.8759\color{black} $\,\,$ & $\,\,$\color{red} 8.5354\color{black} $\,\,$ & $\,\,$\color{red} 6.7251\color{black} $\,\,$ \\
$\,\,$\color{red} 0.5331\color{black} $\,\,$ & $\,\,$ 1 $\,\,$ & $\,\,$4.5500$\,\,$ & $\,\,$3.5850  $\,\,$ \\
$\,\,$\color{red} 0.1172\color{black} $\,\,$ & $\,\,$0.2198$\,\,$ & $\,\,$ 1 $\,\,$ & $\,\,$0.7879 $\,\,$ \\
$\,\,$\color{red} 0.1487\color{black} $\,\,$ & $\,\,$0.2789$\,\,$ & $\,\,$1.2692$\,\,$ & $\,\,$ 1  $\,\,$ \\
\end{pmatrix},
\end{equation*}

\begin{equation*}
\mathbf{w}^{\prime} =
\begin{pmatrix}
0.565751\\
0.289747\\
0.063680\\
0.080822
\end{pmatrix} =
0.977783\cdot
\begin{pmatrix}
\color{gr} 0.578606\color{black} \\
0.296331\\
0.065127\\
0.082658
\end{pmatrix},
\end{equation*}
\begin{equation*}
\left[ \frac{{w}^{\prime}_i}{{w}^{\prime}_j} \right] =
\begin{pmatrix}
$\,\,$ 1 $\,\,$ & $\,\,$\color{gr} 1.9526\color{black} $\,\,$ & $\,\,$\color{gr} 8.8842\color{black} $\,\,$ & $\,\,$\color{gr} \color{blue} 7\color{black} $\,\,$ \\
$\,\,$\color{gr} 0.5121\color{black} $\,\,$ & $\,\,$ 1 $\,\,$ & $\,\,$4.5500$\,\,$ & $\,\,$3.5850  $\,\,$ \\
$\,\,$\color{gr} 0.1126\color{black} $\,\,$ & $\,\,$0.2198$\,\,$ & $\,\,$ 1 $\,\,$ & $\,\,$0.7879 $\,\,$ \\
$\,\,$\color{gr} \color{blue}  1/7\color{black} $\,\,$ & $\,\,$0.2789$\,\,$ & $\,\,$1.2692$\,\,$ & $\,\,$ 1  $\,\,$ \\
\end{pmatrix},
\end{equation*}
\end{example}
\newpage
\begin{example}
\begin{equation*}
\mathbf{A} =
\begin{pmatrix}
$\,\,$ 1 $\,\,$ & $\,\,$2$\,\,$ & $\,\,$9$\,\,$ & $\,\,$7 $\,\,$ \\
$\,\,$ 1/2$\,\,$ & $\,\,$ 1 $\,\,$ & $\,\,$3$\,\,$ & $\,\,$7 $\,\,$ \\
$\,\,$ 1/9$\,\,$ & $\,\,$ 1/3$\,\,$ & $\,\,$ 1 $\,\,$ & $\,\,$ 1/2 $\,\,$ \\
$\,\,$ 1/7$\,\,$ & $\,\,$ 1/7$\,\,$ & $\,\,$2$\,\,$ & $\,\,$ 1  $\,\,$ \\
\end{pmatrix},
\qquad
\lambda_{\max} =
4.2086,
\qquad
CR = 0.0786
\end{equation*}

\begin{equation*}
\mathbf{w}^{AMAST} =
\begin{pmatrix}
\color{red} 0.549309\color{black} \\
0.305795\\
0.065061\\
0.079835
\end{pmatrix}\end{equation*}
\begin{equation*}
\left[ \frac{{w}^{AMAST}_i}{{w}^{AMAST}_j} \right] =
\begin{pmatrix}
$\,\,$ 1 $\,\,$ & $\,\,$\color{red} 1.7963\color{black} $\,\,$ & $\,\,$\color{red} 8.4430\color{black} $\,\,$ & $\,\,$\color{red} 6.8806\color{black} $\,\,$ \\
$\,\,$\color{red} 0.5567\color{black} $\,\,$ & $\,\,$ 1 $\,\,$ & $\,\,$4.7001$\,\,$ & $\,\,$3.8303  $\,\,$ \\
$\,\,$\color{red} 0.1184\color{black} $\,\,$ & $\,\,$0.2128$\,\,$ & $\,\,$ 1 $\,\,$ & $\,\,$0.8149 $\,\,$ \\
$\,\,$\color{red} 0.1453\color{black} $\,\,$ & $\,\,$0.2611$\,\,$ & $\,\,$1.2271$\,\,$ & $\,\,$ 1  $\,\,$ \\
\end{pmatrix},
\end{equation*}

\begin{equation*}
\mathbf{w}^{\prime} =
\begin{pmatrix}
0.553566\\
0.302906\\
0.064446\\
0.079081
\end{pmatrix} =
0.990554\cdot
\begin{pmatrix}
\color{gr} 0.558845\color{black} \\
0.305795\\
0.065061\\
0.079835
\end{pmatrix},
\end{equation*}
\begin{equation*}
\left[ \frac{{w}^{\prime}_i}{{w}^{\prime}_j} \right] =
\begin{pmatrix}
$\,\,$ 1 $\,\,$ & $\,\,$\color{gr} 1.8275\color{black} $\,\,$ & $\,\,$\color{gr} 8.5896\color{black} $\,\,$ & $\,\,$\color{gr} \color{blue} 7\color{black} $\,\,$ \\
$\,\,$\color{gr} 0.5472\color{black} $\,\,$ & $\,\,$ 1 $\,\,$ & $\,\,$4.7001$\,\,$ & $\,\,$3.8303  $\,\,$ \\
$\,\,$\color{gr} 0.1164\color{black} $\,\,$ & $\,\,$0.2128$\,\,$ & $\,\,$ 1 $\,\,$ & $\,\,$0.8149 $\,\,$ \\
$\,\,$\color{gr} \color{blue}  1/7\color{black} $\,\,$ & $\,\,$0.2611$\,\,$ & $\,\,$1.2271$\,\,$ & $\,\,$ 1  $\,\,$ \\
\end{pmatrix},
\end{equation*}
\end{example}
\newpage
\begin{example}
\begin{equation*}
\mathbf{A} =
\begin{pmatrix}
$\,\,$ 1 $\,\,$ & $\,\,$2$\,\,$ & $\,\,$9$\,\,$ & $\,\,$7 $\,\,$ \\
$\,\,$ 1/2$\,\,$ & $\,\,$ 1 $\,\,$ & $\,\,$7$\,\,$ & $\,\,$2 $\,\,$ \\
$\,\,$ 1/9$\,\,$ & $\,\,$ 1/7$\,\,$ & $\,\,$ 1 $\,\,$ & $\,\,$ 1/6 $\,\,$ \\
$\,\,$ 1/7$\,\,$ & $\,\,$ 1/2$\,\,$ & $\,\,$6$\,\,$ & $\,\,$ 1  $\,\,$ \\
\end{pmatrix},
\qquad
\lambda_{\max} =
4.2059,
\qquad
CR = 0.0776
\end{equation*}

\begin{equation*}
\mathbf{w}^{AMAST} =
\begin{pmatrix}
0.546842\\
\color{red} 0.270463\color{black} \\
0.040176\\
0.142518
\end{pmatrix}\end{equation*}
\begin{equation*}
\left[ \frac{{w}^{AMAST}_i}{{w}^{AMAST}_j} \right] =
\begin{pmatrix}
$\,\,$ 1 $\,\,$ & $\,\,$\color{red} 2.0219\color{black} $\,\,$ & $\,\,$13.6110$\,\,$ & $\,\,$3.8370$\,\,$ \\
$\,\,$\color{red} 0.4946\color{black} $\,\,$ & $\,\,$ 1 $\,\,$ & $\,\,$\color{red} 6.7319\color{black} $\,\,$ & $\,\,$\color{red} 1.8977\color{black}   $\,\,$ \\
$\,\,$0.0735$\,\,$ & $\,\,$\color{red} 0.1485\color{black} $\,\,$ & $\,\,$ 1 $\,\,$ & $\,\,$0.2819 $\,\,$ \\
$\,\,$0.2606$\,\,$ & $\,\,$\color{red} 0.5269\color{black} $\,\,$ & $\,\,$3.5473$\,\,$ & $\,\,$ 1  $\,\,$ \\
\end{pmatrix},
\end{equation*}

\begin{equation*}
\mathbf{w}^{\prime} =
\begin{pmatrix}
0.545229\\
0.272615\\
0.040058\\
0.142098
\end{pmatrix} =
0.997051\cdot
\begin{pmatrix}
0.546842\\
\color{gr} 0.273421\color{black} \\
0.040176\\
0.142518
\end{pmatrix},
\end{equation*}
\begin{equation*}
\left[ \frac{{w}^{\prime}_i}{{w}^{\prime}_j} \right] =
\begin{pmatrix}
$\,\,$ 1 $\,\,$ & $\,\,$\color{gr} \color{blue} 2\color{black} $\,\,$ & $\,\,$13.6110$\,\,$ & $\,\,$3.8370$\,\,$ \\
$\,\,$\color{gr} \color{blue}  1/2\color{black} $\,\,$ & $\,\,$ 1 $\,\,$ & $\,\,$\color{gr} 6.8055\color{black} $\,\,$ & $\,\,$\color{gr} 1.9185\color{black}   $\,\,$ \\
$\,\,$0.0735$\,\,$ & $\,\,$\color{gr} 0.1469\color{black} $\,\,$ & $\,\,$ 1 $\,\,$ & $\,\,$0.2819 $\,\,$ \\
$\,\,$0.2606$\,\,$ & $\,\,$\color{gr} 0.5212\color{black} $\,\,$ & $\,\,$3.5473$\,\,$ & $\,\,$ 1  $\,\,$ \\
\end{pmatrix},
\end{equation*}
\end{example}
\newpage
\begin{example}
\begin{equation*}
\mathbf{A} =
\begin{pmatrix}
$\,\,$ 1 $\,\,$ & $\,\,$2$\,\,$ & $\,\,$9$\,\,$ & $\,\,$7 $\,\,$ \\
$\,\,$ 1/2$\,\,$ & $\,\,$ 1 $\,\,$ & $\,\,$7$\,\,$ & $\,\,$2 $\,\,$ \\
$\,\,$ 1/9$\,\,$ & $\,\,$ 1/7$\,\,$ & $\,\,$ 1 $\,\,$ & $\,\,$ 1/7 $\,\,$ \\
$\,\,$ 1/7$\,\,$ & $\,\,$ 1/2$\,\,$ & $\,\,$7$\,\,$ & $\,\,$ 1  $\,\,$ \\
\end{pmatrix},
\qquad
\lambda_{\max} =
4.2526,
\qquad
CR = 0.0952
\end{equation*}

\begin{equation*}
\mathbf{w}^{AMAST} =
\begin{pmatrix}
0.542796\\
\color{red} 0.269083\color{black} \\
0.039015\\
0.149107
\end{pmatrix}\end{equation*}
\begin{equation*}
\left[ \frac{{w}^{AMAST}_i}{{w}^{AMAST}_j} \right] =
\begin{pmatrix}
$\,\,$ 1 $\,\,$ & $\,\,$\color{red} 2.0172\color{black} $\,\,$ & $\,\,$13.9125$\,\,$ & $\,\,$3.6403$\,\,$ \\
$\,\,$\color{red} 0.4957\color{black} $\,\,$ & $\,\,$ 1 $\,\,$ & $\,\,$\color{red} 6.8969\color{black} $\,\,$ & $\,\,$\color{red} 1.8046\color{black}   $\,\,$ \\
$\,\,$0.0719$\,\,$ & $\,\,$\color{red} 0.1450\color{black} $\,\,$ & $\,\,$ 1 $\,\,$ & $\,\,$0.2617 $\,\,$ \\
$\,\,$0.2747$\,\,$ & $\,\,$\color{red} 0.5541\color{black} $\,\,$ & $\,\,$3.8218$\,\,$ & $\,\,$ 1  $\,\,$ \\
\end{pmatrix},
\end{equation*}

\begin{equation*}
\mathbf{w}^{\prime} =
\begin{pmatrix}
0.541542\\
0.270771\\
0.038925\\
0.148762
\end{pmatrix} =
0.997690\cdot
\begin{pmatrix}
0.542796\\
\color{gr} 0.271398\color{black} \\
0.039015\\
0.149107
\end{pmatrix},
\end{equation*}
\begin{equation*}
\left[ \frac{{w}^{\prime}_i}{{w}^{\prime}_j} \right] =
\begin{pmatrix}
$\,\,$ 1 $\,\,$ & $\,\,$\color{gr} \color{blue} 2\color{black} $\,\,$ & $\,\,$13.9125$\,\,$ & $\,\,$3.6403$\,\,$ \\
$\,\,$\color{gr} \color{blue}  1/2\color{black} $\,\,$ & $\,\,$ 1 $\,\,$ & $\,\,$\color{gr} 6.9562\color{black} $\,\,$ & $\,\,$\color{gr} 1.8202\color{black}   $\,\,$ \\
$\,\,$0.0719$\,\,$ & $\,\,$\color{gr} 0.1438\color{black} $\,\,$ & $\,\,$ 1 $\,\,$ & $\,\,$0.2617 $\,\,$ \\
$\,\,$0.2747$\,\,$ & $\,\,$\color{gr} 0.5494\color{black} $\,\,$ & $\,\,$3.8218$\,\,$ & $\,\,$ 1  $\,\,$ \\
\end{pmatrix},
\end{equation*}
\end{example}
\newpage
\begin{example}
\begin{equation*}
\mathbf{A} =
\begin{pmatrix}
$\,\,$ 1 $\,\,$ & $\,\,$2$\,\,$ & $\,\,$9$\,\,$ & $\,\,$8 $\,\,$ \\
$\,\,$ 1/2$\,\,$ & $\,\,$ 1 $\,\,$ & $\,\,$3$\,\,$ & $\,\,$6 $\,\,$ \\
$\,\,$ 1/9$\,\,$ & $\,\,$ 1/3$\,\,$ & $\,\,$ 1 $\,\,$ & $\,\,$ 1/2 $\,\,$ \\
$\,\,$ 1/8$\,\,$ & $\,\,$ 1/6$\,\,$ & $\,\,$2$\,\,$ & $\,\,$ 1  $\,\,$ \\
\end{pmatrix},
\qquad
\lambda_{\max} =
4.1664,
\qquad
CR = 0.0627
\end{equation*}

\begin{equation*}
\mathbf{w}^{AMAST} =
\begin{pmatrix}
\color{red} 0.565343\color{black} \\
0.291451\\
0.064310\\
0.078896
\end{pmatrix}\end{equation*}
\begin{equation*}
\left[ \frac{{w}^{AMAST}_i}{{w}^{AMAST}_j} \right] =
\begin{pmatrix}
$\,\,$ 1 $\,\,$ & $\,\,$\color{red} 1.9398\color{black} $\,\,$ & $\,\,$\color{red} 8.7909\color{black} $\,\,$ & $\,\,$\color{red} 7.1657\color{black} $\,\,$ \\
$\,\,$\color{red} 0.5155\color{black} $\,\,$ & $\,\,$ 1 $\,\,$ & $\,\,$4.5320$\,\,$ & $\,\,$3.6941  $\,\,$ \\
$\,\,$\color{red} 0.1138\color{black} $\,\,$ & $\,\,$0.2207$\,\,$ & $\,\,$ 1 $\,\,$ & $\,\,$0.8151 $\,\,$ \\
$\,\,$\color{red} 0.1396\color{black} $\,\,$ & $\,\,$0.2707$\,\,$ & $\,\,$1.2268$\,\,$ & $\,\,$ 1  $\,\,$ \\
\end{pmatrix},
\end{equation*}

\begin{equation*}
\mathbf{w}^{\prime} =
\begin{pmatrix}
0.571110\\
0.287584\\
0.063457\\
0.077849
\end{pmatrix} =
0.986731\cdot
\begin{pmatrix}
\color{gr} 0.578790\color{black} \\
0.291451\\
0.064310\\
0.078896
\end{pmatrix},
\end{equation*}
\begin{equation*}
\left[ \frac{{w}^{\prime}_i}{{w}^{\prime}_j} \right] =
\begin{pmatrix}
$\,\,$ 1 $\,\,$ & $\,\,$\color{gr} 1.9859\color{black} $\,\,$ & $\,\,$\color{gr} \color{blue} 9\color{black} $\,\,$ & $\,\,$\color{gr} 7.3361\color{black} $\,\,$ \\
$\,\,$\color{gr} 0.5036\color{black} $\,\,$ & $\,\,$ 1 $\,\,$ & $\,\,$4.5320$\,\,$ & $\,\,$3.6941  $\,\,$ \\
$\,\,$\color{gr} \color{blue}  1/9\color{black} $\,\,$ & $\,\,$0.2207$\,\,$ & $\,\,$ 1 $\,\,$ & $\,\,$0.8151 $\,\,$ \\
$\,\,$\color{gr} 0.1363\color{black} $\,\,$ & $\,\,$0.2707$\,\,$ & $\,\,$1.2268$\,\,$ & $\,\,$ 1  $\,\,$ \\
\end{pmatrix},
\end{equation*}
\end{example}
\newpage
\begin{example}
\begin{equation*}
\mathbf{A} =
\begin{pmatrix}
$\,\,$ 1 $\,\,$ & $\,\,$2$\,\,$ & $\,\,$9$\,\,$ & $\,\,$8 $\,\,$ \\
$\,\,$ 1/2$\,\,$ & $\,\,$ 1 $\,\,$ & $\,\,$3$\,\,$ & $\,\,$7 $\,\,$ \\
$\,\,$ 1/9$\,\,$ & $\,\,$ 1/3$\,\,$ & $\,\,$ 1 $\,\,$ & $\,\,$ 1/2 $\,\,$ \\
$\,\,$ 1/8$\,\,$ & $\,\,$ 1/7$\,\,$ & $\,\,$2$\,\,$ & $\,\,$ 1  $\,\,$ \\
\end{pmatrix},
\qquad
\lambda_{\max} =
4.2065,
\qquad
CR = 0.0779
\end{equation*}

\begin{equation*}
\mathbf{w}^{AMAST} =
\begin{pmatrix}
\color{red} 0.558737\color{black} \\
0.300825\\
0.064263\\
0.076175
\end{pmatrix}\end{equation*}
\begin{equation*}
\left[ \frac{{w}^{AMAST}_i}{{w}^{AMAST}_j} \right] =
\begin{pmatrix}
$\,\,$ 1 $\,\,$ & $\,\,$\color{red} 1.8574\color{black} $\,\,$ & $\,\,$\color{red} 8.6946\color{black} $\,\,$ & $\,\,$\color{red} 7.3349\color{black} $\,\,$ \\
$\,\,$\color{red} 0.5384\color{black} $\,\,$ & $\,\,$ 1 $\,\,$ & $\,\,$4.6812$\,\,$ & $\,\,$3.9491  $\,\,$ \\
$\,\,$\color{red} 0.1150\color{black} $\,\,$ & $\,\,$0.2136$\,\,$ & $\,\,$ 1 $\,\,$ & $\,\,$0.8436 $\,\,$ \\
$\,\,$\color{red} 0.1363\color{black} $\,\,$ & $\,\,$0.2532$\,\,$ & $\,\,$1.1854$\,\,$ & $\,\,$ 1  $\,\,$ \\
\end{pmatrix},
\end{equation*}

\begin{equation*}
\mathbf{w}^{\prime} =
\begin{pmatrix}
0.567232\\
0.295034\\
0.063026\\
0.074709
\end{pmatrix} =
0.980750\cdot
\begin{pmatrix}
\color{gr} 0.578365\color{black} \\
0.300825\\
0.064263\\
0.076175
\end{pmatrix},
\end{equation*}
\begin{equation*}
\left[ \frac{{w}^{\prime}_i}{{w}^{\prime}_j} \right] =
\begin{pmatrix}
$\,\,$ 1 $\,\,$ & $\,\,$\color{gr} 1.9226\color{black} $\,\,$ & $\,\,$\color{gr} \color{blue} 9\color{black} $\,\,$ & $\,\,$\color{gr} 7.5926\color{black} $\,\,$ \\
$\,\,$\color{gr} 0.5201\color{black} $\,\,$ & $\,\,$ 1 $\,\,$ & $\,\,$4.6812$\,\,$ & $\,\,$3.9491  $\,\,$ \\
$\,\,$\color{gr} \color{blue}  1/9\color{black} $\,\,$ & $\,\,$0.2136$\,\,$ & $\,\,$ 1 $\,\,$ & $\,\,$0.8436 $\,\,$ \\
$\,\,$\color{gr} 0.1317\color{black} $\,\,$ & $\,\,$0.2532$\,\,$ & $\,\,$1.1854$\,\,$ & $\,\,$ 1  $\,\,$ \\
\end{pmatrix},
\end{equation*}
\end{example}
\newpage
\begin{example}
\begin{equation*}
\mathbf{A} =
\begin{pmatrix}
$\,\,$ 1 $\,\,$ & $\,\,$2$\,\,$ & $\,\,$9$\,\,$ & $\,\,$8 $\,\,$ \\
$\,\,$ 1/2$\,\,$ & $\,\,$ 1 $\,\,$ & $\,\,$3$\,\,$ & $\,\,$8 $\,\,$ \\
$\,\,$ 1/9$\,\,$ & $\,\,$ 1/3$\,\,$ & $\,\,$ 1 $\,\,$ & $\,\,$ 1/2 $\,\,$ \\
$\,\,$ 1/8$\,\,$ & $\,\,$ 1/8$\,\,$ & $\,\,$2$\,\,$ & $\,\,$ 1  $\,\,$ \\
\end{pmatrix},
\qquad
\lambda_{\max} =
4.2469,
\qquad
CR = 0.0931
\end{equation*}

\begin{equation*}
\mathbf{w}^{AMAST} =
\begin{pmatrix}
\color{red} 0.552835\color{black} \\
0.308959\\
0.064227\\
0.073979
\end{pmatrix}\end{equation*}
\begin{equation*}
\left[ \frac{{w}^{AMAST}_i}{{w}^{AMAST}_j} \right] =
\begin{pmatrix}
$\,\,$ 1 $\,\,$ & $\,\,$\color{red} 1.7893\color{black} $\,\,$ & $\,\,$\color{red} 8.6075\color{black} $\,\,$ & $\,\,$\color{red} 7.4728\color{black} $\,\,$ \\
$\,\,$\color{red} 0.5589\color{black} $\,\,$ & $\,\,$ 1 $\,\,$ & $\,\,$4.8104$\,\,$ & $\,\,$4.1763  $\,\,$ \\
$\,\,$\color{red} 0.1162\color{black} $\,\,$ & $\,\,$0.2079$\,\,$ & $\,\,$ 1 $\,\,$ & $\,\,$0.8682 $\,\,$ \\
$\,\,$\color{red} 0.1338\color{black} $\,\,$ & $\,\,$0.2394$\,\,$ & $\,\,$1.1518$\,\,$ & $\,\,$ 1  $\,\,$ \\
\end{pmatrix},
\end{equation*}

\begin{equation*}
\mathbf{w}^{\prime} =
\begin{pmatrix}
0.563831\\
0.301361\\
0.062648\\
0.072160
\end{pmatrix} =
0.975410\cdot
\begin{pmatrix}
\color{gr} 0.578045\color{black} \\
0.308959\\
0.064227\\
0.073979
\end{pmatrix},
\end{equation*}
\begin{equation*}
\left[ \frac{{w}^{\prime}_i}{{w}^{\prime}_j} \right] =
\begin{pmatrix}
$\,\,$ 1 $\,\,$ & $\,\,$\color{gr} 1.8709\color{black} $\,\,$ & $\,\,$\color{gr} \color{blue} 9\color{black} $\,\,$ & $\,\,$\color{gr} 7.8136\color{black} $\,\,$ \\
$\,\,$\color{gr} 0.5345\color{black} $\,\,$ & $\,\,$ 1 $\,\,$ & $\,\,$4.8104$\,\,$ & $\,\,$4.1763  $\,\,$ \\
$\,\,$\color{gr} \color{blue}  1/9\color{black} $\,\,$ & $\,\,$0.2079$\,\,$ & $\,\,$ 1 $\,\,$ & $\,\,$0.8682 $\,\,$ \\
$\,\,$\color{gr} 0.1280\color{black} $\,\,$ & $\,\,$0.2394$\,\,$ & $\,\,$1.1518$\,\,$ & $\,\,$ 1  $\,\,$ \\
\end{pmatrix},
\end{equation*}
\end{example}
\newpage
\begin{example}
\begin{equation*}
\mathbf{A} =
\begin{pmatrix}
$\,\,$ 1 $\,\,$ & $\,\,$2$\,\,$ & $\,\,$9$\,\,$ & $\,\,$9 $\,\,$ \\
$\,\,$ 1/2$\,\,$ & $\,\,$ 1 $\,\,$ & $\,\,$6$\,\,$ & $\,\,$3 $\,\,$ \\
$\,\,$ 1/9$\,\,$ & $\,\,$ 1/6$\,\,$ & $\,\,$ 1 $\,\,$ & $\,\,$ 1/3 $\,\,$ \\
$\,\,$ 1/9$\,\,$ & $\,\,$ 1/3$\,\,$ & $\,\,$3$\,\,$ & $\,\,$ 1  $\,\,$ \\
\end{pmatrix},
\qquad
\lambda_{\max} =
4.1031,
\qquad
CR = 0.0389
\end{equation*}

\begin{equation*}
\mathbf{w}^{AMAST} =
\begin{pmatrix}
0.573644\\
\color{red} 0.281922\color{black} \\
0.047131\\
0.097302
\end{pmatrix}\end{equation*}
\begin{equation*}
\left[ \frac{{w}^{AMAST}_i}{{w}^{AMAST}_j} \right] =
\begin{pmatrix}
$\,\,$ 1 $\,\,$ & $\,\,$\color{red} 2.0348\color{black} $\,\,$ & $\,\,$12.1712$\,\,$ & $\,\,$5.8955$\,\,$ \\
$\,\,$\color{red} 0.4915\color{black} $\,\,$ & $\,\,$ 1 $\,\,$ & $\,\,$\color{red} 5.9816\color{black} $\,\,$ & $\,\,$\color{red} 2.8974\color{black}   $\,\,$ \\
$\,\,$0.0822$\,\,$ & $\,\,$\color{red} 0.1672\color{black} $\,\,$ & $\,\,$ 1 $\,\,$ & $\,\,$0.4844 $\,\,$ \\
$\,\,$0.1696$\,\,$ & $\,\,$\color{red} 0.3451\color{black} $\,\,$ & $\,\,$2.0645$\,\,$ & $\,\,$ 1  $\,\,$ \\
\end{pmatrix},
\end{equation*}

\begin{equation*}
\mathbf{w}^{\prime} =
\begin{pmatrix}
0.573148\\
0.282543\\
0.047091\\
0.097218
\end{pmatrix} =
0.999135\cdot
\begin{pmatrix}
0.573644\\
\color{gr} 0.282788\color{black} \\
0.047131\\
0.097302
\end{pmatrix},
\end{equation*}
\begin{equation*}
\left[ \frac{{w}^{\prime}_i}{{w}^{\prime}_j} \right] =
\begin{pmatrix}
$\,\,$ 1 $\,\,$ & $\,\,$\color{gr} 2.0285\color{black} $\,\,$ & $\,\,$12.1712$\,\,$ & $\,\,$5.8955$\,\,$ \\
$\,\,$\color{gr} 0.4930\color{black} $\,\,$ & $\,\,$ 1 $\,\,$ & $\,\,$\color{gr} \color{blue} 6\color{black} $\,\,$ & $\,\,$\color{gr} 2.9063\color{black}   $\,\,$ \\
$\,\,$0.0822$\,\,$ & $\,\,$\color{gr} \color{blue}  1/6\color{black} $\,\,$ & $\,\,$ 1 $\,\,$ & $\,\,$0.4844 $\,\,$ \\
$\,\,$0.1696$\,\,$ & $\,\,$\color{gr} 0.3441\color{black} $\,\,$ & $\,\,$2.0645$\,\,$ & $\,\,$ 1  $\,\,$ \\
\end{pmatrix},
\end{equation*}
\end{example}
\newpage
\begin{example}
\begin{equation*}
\mathbf{A} =
\begin{pmatrix}
$\,\,$ 1 $\,\,$ & $\,\,$3$\,\,$ & $\,\,$4$\,\,$ & $\,\,$2 $\,\,$ \\
$\,\,$ 1/3$\,\,$ & $\,\,$ 1 $\,\,$ & $\,\,$1$\,\,$ & $\,\,$1 $\,\,$ \\
$\,\,$ 1/4$\,\,$ & $\,\,$ 1 $\,\,$ & $\,\,$ 1 $\,\,$ & $\,\,$ 1/3 $\,\,$ \\
$\,\,$ 1/2$\,\,$ & $\,\,$ 1 $\,\,$ & $\,\,$3$\,\,$ & $\,\,$ 1  $\,\,$ \\
\end{pmatrix},
\qquad
\lambda_{\max} =
4.1031,
\qquad
CR = 0.0389
\end{equation*}

\begin{equation*}
\mathbf{w}^{AMAST} =
\begin{pmatrix}
\color{red} 0.472518\color{black} \\
0.167003\\
0.118609\\
0.241870
\end{pmatrix}\end{equation*}
\begin{equation*}
\left[ \frac{{w}^{AMAST}_i}{{w}^{AMAST}_j} \right] =
\begin{pmatrix}
$\,\,$ 1 $\,\,$ & $\,\,$\color{red} 2.8294\color{black} $\,\,$ & $\,\,$\color{red} 3.9838\color{black} $\,\,$ & $\,\,$\color{red} 1.9536\color{black} $\,\,$ \\
$\,\,$\color{red} 0.3534\color{black} $\,\,$ & $\,\,$ 1 $\,\,$ & $\,\,$1.4080$\,\,$ & $\,\,$0.6905  $\,\,$ \\
$\,\,$\color{red} 0.2510\color{black} $\,\,$ & $\,\,$0.7102$\,\,$ & $\,\,$ 1 $\,\,$ & $\,\,$0.4904 $\,\,$ \\
$\,\,$\color{red} 0.5119\color{black} $\,\,$ & $\,\,$1.4483$\,\,$ & $\,\,$2.0392$\,\,$ & $\,\,$ 1  $\,\,$ \\
\end{pmatrix},
\end{equation*}

\begin{equation*}
\mathbf{w}^{\prime} =
\begin{pmatrix}
0.473528\\
0.166683\\
0.118382\\
0.241407
\end{pmatrix} =
0.998086\cdot
\begin{pmatrix}
\color{gr} 0.474436\color{black} \\
0.167003\\
0.118609\\
0.241870
\end{pmatrix},
\end{equation*}
\begin{equation*}
\left[ \frac{{w}^{\prime}_i}{{w}^{\prime}_j} \right] =
\begin{pmatrix}
$\,\,$ 1 $\,\,$ & $\,\,$\color{gr} 2.8409\color{black} $\,\,$ & $\,\,$\color{gr} \color{blue} 4\color{black} $\,\,$ & $\,\,$\color{gr} 1.9615\color{black} $\,\,$ \\
$\,\,$\color{gr} 0.3520\color{black} $\,\,$ & $\,\,$ 1 $\,\,$ & $\,\,$1.4080$\,\,$ & $\,\,$0.6905  $\,\,$ \\
$\,\,$\color{gr} \color{blue}  1/4\color{black} $\,\,$ & $\,\,$0.7102$\,\,$ & $\,\,$ 1 $\,\,$ & $\,\,$0.4904 $\,\,$ \\
$\,\,$\color{gr} 0.5098\color{black} $\,\,$ & $\,\,$1.4483$\,\,$ & $\,\,$2.0392$\,\,$ & $\,\,$ 1  $\,\,$ \\
\end{pmatrix},
\end{equation*}
\end{example}
\newpage
\begin{example}
\begin{equation*}
\mathbf{A} =
\begin{pmatrix}
$\,\,$ 1 $\,\,$ & $\,\,$3$\,\,$ & $\,\,$5$\,\,$ & $\,\,$2 $\,\,$ \\
$\,\,$ 1/3$\,\,$ & $\,\,$ 1 $\,\,$ & $\,\,$1$\,\,$ & $\,\,$1 $\,\,$ \\
$\,\,$ 1/5$\,\,$ & $\,\,$ 1 $\,\,$ & $\,\,$ 1 $\,\,$ & $\,\,$ 1/4 $\,\,$ \\
$\,\,$ 1/2$\,\,$ & $\,\,$ 1 $\,\,$ & $\,\,$4$\,\,$ & $\,\,$ 1  $\,\,$ \\
\end{pmatrix},
\qquad
\lambda_{\max} =
4.1655,
\qquad
CR = 0.0624
\end{equation*}

\begin{equation*}
\mathbf{w}^{AMAST} =
\begin{pmatrix}
\color{red} 0.480787\color{black} \\
0.163660\\
0.103003\\
0.252550
\end{pmatrix}\end{equation*}
\begin{equation*}
\left[ \frac{{w}^{AMAST}_i}{{w}^{AMAST}_j} \right] =
\begin{pmatrix}
$\,\,$ 1 $\,\,$ & $\,\,$\color{red} 2.9377\color{black} $\,\,$ & $\,\,$\color{red} 4.6677\color{black} $\,\,$ & $\,\,$\color{red} 1.9037\color{black} $\,\,$ \\
$\,\,$\color{red} 0.3404\color{black} $\,\,$ & $\,\,$ 1 $\,\,$ & $\,\,$1.5889$\,\,$ & $\,\,$0.6480  $\,\,$ \\
$\,\,$\color{red} 0.2142\color{black} $\,\,$ & $\,\,$0.6294$\,\,$ & $\,\,$ 1 $\,\,$ & $\,\,$0.4079 $\,\,$ \\
$\,\,$\color{red} 0.5253\color{black} $\,\,$ & $\,\,$1.5431$\,\,$ & $\,\,$2.4519$\,\,$ & $\,\,$ 1  $\,\,$ \\
\end{pmatrix},
\end{equation*}

\begin{equation*}
\mathbf{w}^{\prime} =
\begin{pmatrix}
0.486026\\
0.162009\\
0.101964\\
0.250002
\end{pmatrix} =
0.989910\cdot
\begin{pmatrix}
\color{gr} 0.490980\color{black} \\
0.163660\\
0.103003\\
0.252550
\end{pmatrix},
\end{equation*}
\begin{equation*}
\left[ \frac{{w}^{\prime}_i}{{w}^{\prime}_j} \right] =
\begin{pmatrix}
$\,\,$ 1 $\,\,$ & $\,\,$\color{gr} \color{blue} 3\color{black} $\,\,$ & $\,\,$\color{gr} 4.7666\color{black} $\,\,$ & $\,\,$\color{gr} 1.9441\color{black} $\,\,$ \\
$\,\,$\color{gr} \color{blue}  1/3\color{black} $\,\,$ & $\,\,$ 1 $\,\,$ & $\,\,$1.5889$\,\,$ & $\,\,$0.6480  $\,\,$ \\
$\,\,$\color{gr} 0.2098\color{black} $\,\,$ & $\,\,$0.6294$\,\,$ & $\,\,$ 1 $\,\,$ & $\,\,$0.4079 $\,\,$ \\
$\,\,$\color{gr} 0.5144\color{black} $\,\,$ & $\,\,$1.5431$\,\,$ & $\,\,$2.4519$\,\,$ & $\,\,$ 1  $\,\,$ \\
\end{pmatrix},
\end{equation*}
\end{example}
\newpage
\begin{example}
\begin{equation*}
\mathbf{A} =
\begin{pmatrix}
$\,\,$ 1 $\,\,$ & $\,\,$3$\,\,$ & $\,\,$5$\,\,$ & $\,\,$2 $\,\,$ \\
$\,\,$ 1/3$\,\,$ & $\,\,$ 1 $\,\,$ & $\,\,$1$\,\,$ & $\,\,$1 $\,\,$ \\
$\,\,$ 1/5$\,\,$ & $\,\,$ 1 $\,\,$ & $\,\,$ 1 $\,\,$ & $\,\,$ 1/5 $\,\,$ \\
$\,\,$ 1/2$\,\,$ & $\,\,$ 1 $\,\,$ & $\,\,$5$\,\,$ & $\,\,$ 1  $\,\,$ \\
\end{pmatrix},
\qquad
\lambda_{\max} =
4.2277,
\qquad
CR = 0.0859
\end{equation*}

\begin{equation*}
\mathbf{w}^{AMAST} =
\begin{pmatrix}
\color{red} 0.473588\color{black} \\
0.163100\\
0.098037\\
0.265274
\end{pmatrix}\end{equation*}
\begin{equation*}
\left[ \frac{{w}^{AMAST}_i}{{w}^{AMAST}_j} \right] =
\begin{pmatrix}
$\,\,$ 1 $\,\,$ & $\,\,$\color{red} 2.9037\color{black} $\,\,$ & $\,\,$\color{red} 4.8307\color{black} $\,\,$ & $\,\,$\color{red} 1.7853\color{black} $\,\,$ \\
$\,\,$\color{red} 0.3444\color{black} $\,\,$ & $\,\,$ 1 $\,\,$ & $\,\,$1.6637$\,\,$ & $\,\,$0.6148  $\,\,$ \\
$\,\,$\color{red} 0.2070\color{black} $\,\,$ & $\,\,$0.6011$\,\,$ & $\,\,$ 1 $\,\,$ & $\,\,$0.3696 $\,\,$ \\
$\,\,$\color{red} 0.5601\color{black} $\,\,$ & $\,\,$1.6264$\,\,$ & $\,\,$2.7058$\,\,$ & $\,\,$ 1  $\,\,$ \\
\end{pmatrix},
\end{equation*}

\begin{equation*}
\mathbf{w}^{\prime} =
\begin{pmatrix}
0.481732\\
0.160577\\
0.096521\\
0.261170
\end{pmatrix} =
0.984530\cdot
\begin{pmatrix}
\color{gr} 0.489301\color{black} \\
0.163100\\
0.098037\\
0.265274
\end{pmatrix},
\end{equation*}
\begin{equation*}
\left[ \frac{{w}^{\prime}_i}{{w}^{\prime}_j} \right] =
\begin{pmatrix}
$\,\,$ 1 $\,\,$ & $\,\,$\color{gr} \color{blue} 3\color{black} $\,\,$ & $\,\,$\color{gr} 4.9910\color{black} $\,\,$ & $\,\,$\color{gr} 1.8445\color{black} $\,\,$ \\
$\,\,$\color{gr} \color{blue}  1/3\color{black} $\,\,$ & $\,\,$ 1 $\,\,$ & $\,\,$1.6637$\,\,$ & $\,\,$0.6148  $\,\,$ \\
$\,\,$\color{gr} 0.2004\color{black} $\,\,$ & $\,\,$0.6011$\,\,$ & $\,\,$ 1 $\,\,$ & $\,\,$0.3696 $\,\,$ \\
$\,\,$\color{gr} 0.5421\color{black} $\,\,$ & $\,\,$1.6264$\,\,$ & $\,\,$2.7058$\,\,$ & $\,\,$ 1  $\,\,$ \\
\end{pmatrix},
\end{equation*}
\end{example}
\newpage
\begin{example}
\begin{equation*}
\mathbf{A} =
\begin{pmatrix}
$\,\,$ 1 $\,\,$ & $\,\,$3$\,\,$ & $\,\,$5$\,\,$ & $\,\,$4 $\,\,$ \\
$\,\,$ 1/3$\,\,$ & $\,\,$ 1 $\,\,$ & $\,\,$1$\,\,$ & $\,\,$2 $\,\,$ \\
$\,\,$ 1/5$\,\,$ & $\,\,$ 1 $\,\,$ & $\,\,$ 1 $\,\,$ & $\,\,$ 1/2 $\,\,$ \\
$\,\,$ 1/4$\,\,$ & $\,\,$ 1/2$\,\,$ & $\,\,$2$\,\,$ & $\,\,$ 1  $\,\,$ \\
\end{pmatrix},
\qquad
\lambda_{\max} =
4.1655,
\qquad
CR = 0.0624
\end{equation*}

\begin{equation*}
\mathbf{w}^{AMAST} =
\begin{pmatrix}
\color{red} 0.548822\color{black} \\
0.186704\\
0.117588\\
0.146886
\end{pmatrix}\end{equation*}
\begin{equation*}
\left[ \frac{{w}^{AMAST}_i}{{w}^{AMAST}_j} \right] =
\begin{pmatrix}
$\,\,$ 1 $\,\,$ & $\,\,$\color{red} 2.9395\color{black} $\,\,$ & $\,\,$\color{red} 4.6673\color{black} $\,\,$ & $\,\,$\color{red} 3.7364\color{black} $\,\,$ \\
$\,\,$\color{red} 0.3402\color{black} $\,\,$ & $\,\,$ 1 $\,\,$ & $\,\,$1.5878$\,\,$ & $\,\,$1.2711  $\,\,$ \\
$\,\,$\color{red} 0.2143\color{black} $\,\,$ & $\,\,$0.6298$\,\,$ & $\,\,$ 1 $\,\,$ & $\,\,$0.8005 $\,\,$ \\
$\,\,$\color{red} 0.2676\color{black} $\,\,$ & $\,\,$0.7867$\,\,$ & $\,\,$1.2492$\,\,$ & $\,\,$ 1  $\,\,$ \\
\end{pmatrix},
\end{equation*}

\begin{equation*}
\mathbf{w}^{\prime} =
\begin{pmatrix}
0.553859\\
0.184620\\
0.116275\\
0.145246
\end{pmatrix} =
0.988835\cdot
\begin{pmatrix}
\color{gr} 0.560113\color{black} \\
0.186704\\
0.117588\\
0.146886
\end{pmatrix},
\end{equation*}
\begin{equation*}
\left[ \frac{{w}^{\prime}_i}{{w}^{\prime}_j} \right] =
\begin{pmatrix}
$\,\,$ 1 $\,\,$ & $\,\,$\color{gr} \color{blue} 3\color{black} $\,\,$ & $\,\,$\color{gr} 4.7633\color{black} $\,\,$ & $\,\,$\color{gr} 3.8133\color{black} $\,\,$ \\
$\,\,$\color{gr} \color{blue}  1/3\color{black} $\,\,$ & $\,\,$ 1 $\,\,$ & $\,\,$1.5878$\,\,$ & $\,\,$1.2711  $\,\,$ \\
$\,\,$\color{gr} 0.2099\color{black} $\,\,$ & $\,\,$0.6298$\,\,$ & $\,\,$ 1 $\,\,$ & $\,\,$0.8005 $\,\,$ \\
$\,\,$\color{gr} 0.2622\color{black} $\,\,$ & $\,\,$0.7867$\,\,$ & $\,\,$1.2492$\,\,$ & $\,\,$ 1  $\,\,$ \\
\end{pmatrix},
\end{equation*}
\end{example}
\newpage
\begin{example}
\begin{equation*}
\mathbf{A} =
\begin{pmatrix}
$\,\,$ 1 $\,\,$ & $\,\,$4$\,\,$ & $\,\,$6$\,\,$ & $\,\,$2 $\,\,$ \\
$\,\,$ 1/4$\,\,$ & $\,\,$ 1 $\,\,$ & $\,\,$1$\,\,$ & $\,\,$1 $\,\,$ \\
$\,\,$ 1/6$\,\,$ & $\,\,$ 1 $\,\,$ & $\,\,$ 1 $\,\,$ & $\,\,$ 1/5 $\,\,$ \\
$\,\,$ 1/2$\,\,$ & $\,\,$ 1 $\,\,$ & $\,\,$5$\,\,$ & $\,\,$ 1  $\,\,$ \\
\end{pmatrix},
\qquad
\lambda_{\max} =
4.2277,
\qquad
CR = 0.0859
\end{equation*}

\begin{equation*}
\mathbf{w}^{AMAST} =
\begin{pmatrix}
\color{red} 0.508720\color{black} \\
0.146718\\
0.089203\\
0.255360
\end{pmatrix}\end{equation*}
\begin{equation*}
\left[ \frac{{w}^{AMAST}_i}{{w}^{AMAST}_j} \right] =
\begin{pmatrix}
$\,\,$ 1 $\,\,$ & $\,\,$\color{red} 3.4673\color{black} $\,\,$ & $\,\,$\color{red} 5.7030\color{black} $\,\,$ & $\,\,$\color{red} 1.9922\color{black} $\,\,$ \\
$\,\,$\color{red} 0.2884\color{black} $\,\,$ & $\,\,$ 1 $\,\,$ & $\,\,$1.6448$\,\,$ & $\,\,$0.5746  $\,\,$ \\
$\,\,$\color{red} 0.1753\color{black} $\,\,$ & $\,\,$0.6080$\,\,$ & $\,\,$ 1 $\,\,$ & $\,\,$0.3493 $\,\,$ \\
$\,\,$\color{red} 0.5020\color{black} $\,\,$ & $\,\,$1.7405$\,\,$ & $\,\,$2.8627$\,\,$ & $\,\,$ 1  $\,\,$ \\
\end{pmatrix},
\end{equation*}

\begin{equation*}
\mathbf{w}^{\prime} =
\begin{pmatrix}
0.509700\\
0.146425\\
0.089025\\
0.254850
\end{pmatrix} =
0.998004\cdot
\begin{pmatrix}
\color{gr} 0.510720\color{black} \\
0.146718\\
0.089203\\
0.255360
\end{pmatrix},
\end{equation*}
\begin{equation*}
\left[ \frac{{w}^{\prime}_i}{{w}^{\prime}_j} \right] =
\begin{pmatrix}
$\,\,$ 1 $\,\,$ & $\,\,$\color{gr} 3.4810\color{black} $\,\,$ & $\,\,$\color{gr} 5.7254\color{black} $\,\,$ & $\,\,$\color{gr} \color{blue} 2\color{black} $\,\,$ \\
$\,\,$\color{gr} 0.2873\color{black} $\,\,$ & $\,\,$ 1 $\,\,$ & $\,\,$1.6448$\,\,$ & $\,\,$0.5746  $\,\,$ \\
$\,\,$\color{gr} 0.1747\color{black} $\,\,$ & $\,\,$0.6080$\,\,$ & $\,\,$ 1 $\,\,$ & $\,\,$0.3493 $\,\,$ \\
$\,\,$\color{gr} \color{blue}  1/2\color{black} $\,\,$ & $\,\,$1.7405$\,\,$ & $\,\,$2.8627$\,\,$ & $\,\,$ 1  $\,\,$ \\
\end{pmatrix},
\end{equation*}
\end{example}
\newpage
\begin{example}
\begin{equation*}
\mathbf{A} =
\begin{pmatrix}
$\,\,$ 1 $\,\,$ & $\,\,$4$\,\,$ & $\,\,$6$\,\,$ & $\,\,$3 $\,\,$ \\
$\,\,$ 1/4$\,\,$ & $\,\,$ 1 $\,\,$ & $\,\,$1$\,\,$ & $\,\,$1 $\,\,$ \\
$\,\,$ 1/6$\,\,$ & $\,\,$ 1 $\,\,$ & $\,\,$ 1 $\,\,$ & $\,\,$ 1/3 $\,\,$ \\
$\,\,$ 1/3$\,\,$ & $\,\,$ 1 $\,\,$ & $\,\,$3$\,\,$ & $\,\,$ 1  $\,\,$ \\
\end{pmatrix},
\qquad
\lambda_{\max} =
4.1031,
\qquad
CR = 0.0389
\end{equation*}

\begin{equation*}
\mathbf{w}^{AMAST} =
\begin{pmatrix}
\color{red} 0.562544\color{black} \\
0.141204\\
0.097574\\
0.198678
\end{pmatrix}\end{equation*}
\begin{equation*}
\left[ \frac{{w}^{AMAST}_i}{{w}^{AMAST}_j} \right] =
\begin{pmatrix}
$\,\,$ 1 $\,\,$ & $\,\,$\color{red} 3.9839\color{black} $\,\,$ & $\,\,$\color{red} 5.7653\color{black} $\,\,$ & $\,\,$\color{red} 2.8314\color{black} $\,\,$ \\
$\,\,$\color{red} 0.2510\color{black} $\,\,$ & $\,\,$ 1 $\,\,$ & $\,\,$1.4472$\,\,$ & $\,\,$0.7107  $\,\,$ \\
$\,\,$\color{red} 0.1735\color{black} $\,\,$ & $\,\,$0.6910$\,\,$ & $\,\,$ 1 $\,\,$ & $\,\,$0.4911 $\,\,$ \\
$\,\,$\color{red} 0.3532\color{black} $\,\,$ & $\,\,$1.4070$\,\,$ & $\,\,$2.0362$\,\,$ & $\,\,$ 1  $\,\,$ \\
\end{pmatrix},
\end{equation*}

\begin{equation*}
\mathbf{w}^{\prime} =
\begin{pmatrix}
0.563536\\
0.140884\\
0.097353\\
0.198228
\end{pmatrix} =
0.997734\cdot
\begin{pmatrix}
\color{gr} 0.564816\color{black} \\
0.141204\\
0.097574\\
0.198678
\end{pmatrix},
\end{equation*}
\begin{equation*}
\left[ \frac{{w}^{\prime}_i}{{w}^{\prime}_j} \right] =
\begin{pmatrix}
$\,\,$ 1 $\,\,$ & $\,\,$\color{gr} \color{blue} 4\color{black} $\,\,$ & $\,\,$\color{gr} 5.7886\color{black} $\,\,$ & $\,\,$\color{gr} 2.8429\color{black} $\,\,$ \\
$\,\,$\color{gr} \color{blue}  1/4\color{black} $\,\,$ & $\,\,$ 1 $\,\,$ & $\,\,$1.4472$\,\,$ & $\,\,$0.7107  $\,\,$ \\
$\,\,$\color{gr} 0.1728\color{black} $\,\,$ & $\,\,$0.6910$\,\,$ & $\,\,$ 1 $\,\,$ & $\,\,$0.4911 $\,\,$ \\
$\,\,$\color{gr} 0.3518\color{black} $\,\,$ & $\,\,$1.4070$\,\,$ & $\,\,$2.0362$\,\,$ & $\,\,$ 1  $\,\,$ \\
\end{pmatrix},
\end{equation*}
\end{example}
\newpage
\begin{example}
\begin{equation*}
\mathbf{A} =
\begin{pmatrix}
$\,\,$ 1 $\,\,$ & $\,\,$4$\,\,$ & $\,\,$6$\,\,$ & $\,\,$5 $\,\,$ \\
$\,\,$ 1/4$\,\,$ & $\,\,$ 1 $\,\,$ & $\,\,$1$\,\,$ & $\,\,$2 $\,\,$ \\
$\,\,$ 1/6$\,\,$ & $\,\,$ 1 $\,\,$ & $\,\,$ 1 $\,\,$ & $\,\,$ 1/2 $\,\,$ \\
$\,\,$ 1/5$\,\,$ & $\,\,$ 1/2$\,\,$ & $\,\,$2$\,\,$ & $\,\,$ 1  $\,\,$ \\
\end{pmatrix},
\qquad
\lambda_{\max} =
4.1655,
\qquad
CR = 0.0624
\end{equation*}

\begin{equation*}
\mathbf{w}^{AMAST} =
\begin{pmatrix}
\color{red} 0.604451\color{black} \\
0.161746\\
0.104262\\
0.129540
\end{pmatrix}\end{equation*}
\begin{equation*}
\left[ \frac{{w}^{AMAST}_i}{{w}^{AMAST}_j} \right] =
\begin{pmatrix}
$\,\,$ 1 $\,\,$ & $\,\,$\color{red} 3.7370\color{black} $\,\,$ & $\,\,$\color{red} 5.7974\color{black} $\,\,$ & $\,\,$\color{red} 4.6661\color{black} $\,\,$ \\
$\,\,$\color{red} 0.2676\color{black} $\,\,$ & $\,\,$ 1 $\,\,$ & $\,\,$1.5513$\,\,$ & $\,\,$1.2486  $\,\,$ \\
$\,\,$\color{red} 0.1725\color{black} $\,\,$ & $\,\,$0.6446$\,\,$ & $\,\,$ 1 $\,\,$ & $\,\,$0.8049 $\,\,$ \\
$\,\,$\color{red} 0.2143\color{black} $\,\,$ & $\,\,$0.8009$\,\,$ & $\,\,$1.2424$\,\,$ & $\,\,$ 1  $\,\,$ \\
\end{pmatrix},
\end{equation*}

\begin{equation*}
\mathbf{w}^{\prime} =
\begin{pmatrix}
0.612633\\
0.158400\\
0.102106\\
0.126861
\end{pmatrix} =
0.979315\cdot
\begin{pmatrix}
\color{gr} 0.625573\color{black} \\
0.161746\\
0.104262\\
0.129540
\end{pmatrix},
\end{equation*}
\begin{equation*}
\left[ \frac{{w}^{\prime}_i}{{w}^{\prime}_j} \right] =
\begin{pmatrix}
$\,\,$ 1 $\,\,$ & $\,\,$\color{gr} 3.8676\color{black} $\,\,$ & $\,\,$\color{gr} \color{blue} 6\color{black} $\,\,$ & $\,\,$\color{gr} 4.8292\color{black} $\,\,$ \\
$\,\,$\color{gr} 0.2586\color{black} $\,\,$ & $\,\,$ 1 $\,\,$ & $\,\,$1.5513$\,\,$ & $\,\,$1.2486  $\,\,$ \\
$\,\,$\color{gr} \color{blue}  1/6\color{black} $\,\,$ & $\,\,$0.6446$\,\,$ & $\,\,$ 1 $\,\,$ & $\,\,$0.8049 $\,\,$ \\
$\,\,$\color{gr} 0.2071\color{black} $\,\,$ & $\,\,$0.8009$\,\,$ & $\,\,$1.2424$\,\,$ & $\,\,$ 1  $\,\,$ \\
\end{pmatrix},
\end{equation*}
\end{example}
\newpage
\begin{example}
\begin{equation*}
\mathbf{A} =
\begin{pmatrix}
$\,\,$ 1 $\,\,$ & $\,\,$4$\,\,$ & $\,\,$7$\,\,$ & $\,\,$5 $\,\,$ \\
$\,\,$ 1/4$\,\,$ & $\,\,$ 1 $\,\,$ & $\,\,$1$\,\,$ & $\,\,$2 $\,\,$ \\
$\,\,$ 1/7$\,\,$ & $\,\,$ 1 $\,\,$ & $\,\,$ 1 $\,\,$ & $\,\,$ 1/2 $\,\,$ \\
$\,\,$ 1/5$\,\,$ & $\,\,$ 1/2$\,\,$ & $\,\,$2$\,\,$ & $\,\,$ 1  $\,\,$ \\
\end{pmatrix},
\qquad
\lambda_{\max} =
4.1665,
\qquad
CR = 0.0628
\end{equation*}

\begin{equation*}
\mathbf{w}^{AMAST} =
\begin{pmatrix}
\color{red} 0.615586\color{black} \\
0.159123\\
0.098674\\
0.126617
\end{pmatrix}\end{equation*}
\begin{equation*}
\left[ \frac{{w}^{AMAST}_i}{{w}^{AMAST}_j} \right] =
\begin{pmatrix}
$\,\,$ 1 $\,\,$ & $\,\,$\color{red} 3.8686\color{black} $\,\,$ & $\,\,$\color{red} 6.2386\color{black} $\,\,$ & $\,\,$\color{red} 4.8618\color{black} $\,\,$ \\
$\,\,$\color{red} 0.2585\color{black} $\,\,$ & $\,\,$ 1 $\,\,$ & $\,\,$1.6126$\,\,$ & $\,\,$1.2567  $\,\,$ \\
$\,\,$\color{red} 0.1603\color{black} $\,\,$ & $\,\,$0.6201$\,\,$ & $\,\,$ 1 $\,\,$ & $\,\,$0.7793 $\,\,$ \\
$\,\,$\color{red} 0.2057\color{black} $\,\,$ & $\,\,$0.7957$\,\,$ & $\,\,$1.2832$\,\,$ & $\,\,$ 1  $\,\,$ \\
\end{pmatrix},
\end{equation*}

\begin{equation*}
\mathbf{w}^{\prime} =
\begin{pmatrix}
0.622197\\
0.156387\\
0.096977\\
0.124439
\end{pmatrix} =
0.982804\cdot
\begin{pmatrix}
\color{gr} 0.633083\color{black} \\
0.159123\\
0.098674\\
0.126617
\end{pmatrix},
\end{equation*}
\begin{equation*}
\left[ \frac{{w}^{\prime}_i}{{w}^{\prime}_j} \right] =
\begin{pmatrix}
$\,\,$ 1 $\,\,$ & $\,\,$\color{gr} 3.9786\color{black} $\,\,$ & $\,\,$\color{gr} 6.4159\color{black} $\,\,$ & $\,\,$\color{gr} \color{blue} 5\color{black} $\,\,$ \\
$\,\,$\color{gr} 0.2513\color{black} $\,\,$ & $\,\,$ 1 $\,\,$ & $\,\,$1.6126$\,\,$ & $\,\,$1.2567  $\,\,$ \\
$\,\,$\color{gr} 0.1559\color{black} $\,\,$ & $\,\,$0.6201$\,\,$ & $\,\,$ 1 $\,\,$ & $\,\,$0.7793 $\,\,$ \\
$\,\,$\color{gr} \color{blue}  1/5\color{black} $\,\,$ & $\,\,$0.7957$\,\,$ & $\,\,$1.2832$\,\,$ & $\,\,$ 1  $\,\,$ \\
\end{pmatrix},
\end{equation*}
\end{example}
\newpage
\begin{example}
\begin{equation*}
\mathbf{A} =
\begin{pmatrix}
$\,\,$ 1 $\,\,$ & $\,\,$5$\,\,$ & $\,\,$7$\,\,$ & $\,\,$3 $\,\,$ \\
$\,\,$ 1/5$\,\,$ & $\,\,$ 1 $\,\,$ & $\,\,$1$\,\,$ & $\,\,$1 $\,\,$ \\
$\,\,$ 1/7$\,\,$ & $\,\,$ 1 $\,\,$ & $\,\,$ 1 $\,\,$ & $\,\,$ 1/4 $\,\,$ \\
$\,\,$ 1/3$\,\,$ & $\,\,$ 1 $\,\,$ & $\,\,$4$\,\,$ & $\,\,$ 1  $\,\,$ \\
\end{pmatrix},
\qquad
\lambda_{\max} =
4.1667,
\qquad
CR = 0.0629
\end{equation*}

\begin{equation*}
\mathbf{w}^{AMAST} =
\begin{pmatrix}
\color{red} 0.582259\color{black} \\
0.128652\\
0.084139\\
0.204951
\end{pmatrix}\end{equation*}
\begin{equation*}
\left[ \frac{{w}^{AMAST}_i}{{w}^{AMAST}_j} \right] =
\begin{pmatrix}
$\,\,$ 1 $\,\,$ & $\,\,$\color{red} 4.5259\color{black} $\,\,$ & $\,\,$\color{red} 6.9202\color{black} $\,\,$ & $\,\,$\color{red} 2.8410\color{black} $\,\,$ \\
$\,\,$\color{red} 0.2210\color{black} $\,\,$ & $\,\,$ 1 $\,\,$ & $\,\,$1.5290$\,\,$ & $\,\,$0.6277  $\,\,$ \\
$\,\,$\color{red} 0.1445\color{black} $\,\,$ & $\,\,$0.6540$\,\,$ & $\,\,$ 1 $\,\,$ & $\,\,$0.4105 $\,\,$ \\
$\,\,$\color{red} 0.3520\color{black} $\,\,$ & $\,\,$1.5931$\,\,$ & $\,\,$2.4359$\,\,$ & $\,\,$ 1  $\,\,$ \\
\end{pmatrix},
\end{equation*}

\begin{equation*}
\mathbf{w}^{\prime} =
\begin{pmatrix}
0.585044\\
0.127794\\
0.083578\\
0.203585
\end{pmatrix} =
0.993334\cdot
\begin{pmatrix}
\color{gr} 0.588970\color{black} \\
0.128652\\
0.084139\\
0.204951
\end{pmatrix},
\end{equation*}
\begin{equation*}
\left[ \frac{{w}^{\prime}_i}{{w}^{\prime}_j} \right] =
\begin{pmatrix}
$\,\,$ 1 $\,\,$ & $\,\,$\color{gr} 4.5780\color{black} $\,\,$ & $\,\,$\color{gr} \color{blue} 7\color{black} $\,\,$ & $\,\,$\color{gr} 2.8737\color{black} $\,\,$ \\
$\,\,$\color{gr} 0.2184\color{black} $\,\,$ & $\,\,$ 1 $\,\,$ & $\,\,$1.5290$\,\,$ & $\,\,$0.6277  $\,\,$ \\
$\,\,$\color{gr} \color{blue}  1/7\color{black} $\,\,$ & $\,\,$0.6540$\,\,$ & $\,\,$ 1 $\,\,$ & $\,\,$0.4105 $\,\,$ \\
$\,\,$\color{gr} 0.3480\color{black} $\,\,$ & $\,\,$1.5931$\,\,$ & $\,\,$2.4359$\,\,$ & $\,\,$ 1  $\,\,$ \\
\end{pmatrix},
\end{equation*}
\end{example}
\newpage
\begin{example}
\begin{equation*}
\mathbf{A} =
\begin{pmatrix}
$\,\,$ 1 $\,\,$ & $\,\,$5$\,\,$ & $\,\,$7$\,\,$ & $\,\,$6 $\,\,$ \\
$\,\,$ 1/5$\,\,$ & $\,\,$ 1 $\,\,$ & $\,\,$1$\,\,$ & $\,\,$2 $\,\,$ \\
$\,\,$ 1/7$\,\,$ & $\,\,$ 1 $\,\,$ & $\,\,$ 1 $\,\,$ & $\,\,$ 1/2 $\,\,$ \\
$\,\,$ 1/6$\,\,$ & $\,\,$ 1/2$\,\,$ & $\,\,$2$\,\,$ & $\,\,$ 1  $\,\,$ \\
\end{pmatrix},
\qquad
\lambda_{\max} =
4.1667,
\qquad
CR = 0.0629
\end{equation*}

\begin{equation*}
\mathbf{w}^{AMAST} =
\begin{pmatrix}
\color{red} 0.647371\color{black} \\
0.143130\\
0.093615\\
0.115883
\end{pmatrix}\end{equation*}
\begin{equation*}
\left[ \frac{{w}^{AMAST}_i}{{w}^{AMAST}_j} \right] =
\begin{pmatrix}
$\,\,$ 1 $\,\,$ & $\,\,$\color{red} 4.5229\color{black} $\,\,$ & $\,\,$\color{red} 6.9152\color{black} $\,\,$ & $\,\,$\color{red} 5.5864\color{black} $\,\,$ \\
$\,\,$\color{red} 0.2211\color{black} $\,\,$ & $\,\,$ 1 $\,\,$ & $\,\,$1.5289$\,\,$ & $\,\,$1.2351  $\,\,$ \\
$\,\,$\color{red} 0.1446\color{black} $\,\,$ & $\,\,$0.6541$\,\,$ & $\,\,$ 1 $\,\,$ & $\,\,$0.8078 $\,\,$ \\
$\,\,$\color{red} 0.1790\color{black} $\,\,$ & $\,\,$0.8096$\,\,$ & $\,\,$1.2379$\,\,$ & $\,\,$ 1  $\,\,$ \\
\end{pmatrix},
\end{equation*}

\begin{equation*}
\mathbf{w}^{\prime} =
\begin{pmatrix}
0.650148\\
0.142003\\
0.092878\\
0.114970
\end{pmatrix} =
0.992126\cdot
\begin{pmatrix}
\color{gr} 0.655308\color{black} \\
0.143130\\
0.093615\\
0.115883
\end{pmatrix},
\end{equation*}
\begin{equation*}
\left[ \frac{{w}^{\prime}_i}{{w}^{\prime}_j} \right] =
\begin{pmatrix}
$\,\,$ 1 $\,\,$ & $\,\,$\color{gr} 4.5784\color{black} $\,\,$ & $\,\,$\color{gr} \color{blue} 7\color{black} $\,\,$ & $\,\,$\color{gr} 5.6549\color{black} $\,\,$ \\
$\,\,$\color{gr} 0.2184\color{black} $\,\,$ & $\,\,$ 1 $\,\,$ & $\,\,$1.5289$\,\,$ & $\,\,$1.2351  $\,\,$ \\
$\,\,$\color{gr} \color{blue}  1/7\color{black} $\,\,$ & $\,\,$0.6541$\,\,$ & $\,\,$ 1 $\,\,$ & $\,\,$0.8078 $\,\,$ \\
$\,\,$\color{gr} 0.1768\color{black} $\,\,$ & $\,\,$0.8096$\,\,$ & $\,\,$1.2379$\,\,$ & $\,\,$ 1  $\,\,$ \\
\end{pmatrix},
\end{equation*}
\end{example}
\newpage
\begin{example}
\begin{equation*}
\mathbf{A} =
\begin{pmatrix}
$\,\,$ 1 $\,\,$ & $\,\,$5$\,\,$ & $\,\,$8$\,\,$ & $\,\,$1 $\,\,$ \\
$\,\,$ 1/5$\,\,$ & $\,\,$ 1 $\,\,$ & $\,\,$3$\,\,$ & $\,\,$1 $\,\,$ \\
$\,\,$ 1/8$\,\,$ & $\,\,$ 1/3$\,\,$ & $\,\,$ 1 $\,\,$ & $\,\,$ 1/5 $\,\,$ \\
$\,\,$ 1 $\,\,$ & $\,\,$ 1 $\,\,$ & $\,\,$5$\,\,$ & $\,\,$ 1  $\,\,$ \\
\end{pmatrix},
\qquad
\lambda_{\max} =
4.2259,
\qquad
CR = 0.0852
\end{equation*}

\begin{equation*}
\mathbf{w}^{AMAST} =
\begin{pmatrix}
0.473274\\
0.176748\\
\color{red} 0.058095\color{black} \\
0.291882
\end{pmatrix}\end{equation*}
\begin{equation*}
\left[ \frac{{w}^{AMAST}_i}{{w}^{AMAST}_j} \right] =
\begin{pmatrix}
$\,\,$ 1 $\,\,$ & $\,\,$2.6777$\,\,$ & $\,\,$\color{red} 8.1465\color{black} $\,\,$ & $\,\,$1.6215$\,\,$ \\
$\,\,$0.3735$\,\,$ & $\,\,$ 1 $\,\,$ & $\,\,$\color{red} 3.0424\color{black} $\,\,$ & $\,\,$0.6055  $\,\,$ \\
$\,\,$\color{red} 0.1228\color{black} $\,\,$ & $\,\,$\color{red} 0.3287\color{black} $\,\,$ & $\,\,$ 1 $\,\,$ & $\,\,$\color{red} 0.1990\color{black}  $\,\,$ \\
$\,\,$0.6167$\,\,$ & $\,\,$1.6514$\,\,$ & $\,\,$\color{red} 5.0242\color{black} $\,\,$ & $\,\,$ 1  $\,\,$ \\
\end{pmatrix},
\end{equation*}

\begin{equation*}
\mathbf{w}^{\prime} =
\begin{pmatrix}
0.473141\\
0.176698\\
0.058360\\
0.291800
\end{pmatrix} =
0.999719\cdot
\begin{pmatrix}
0.473274\\
0.176748\\
\color{gr} 0.058376\color{black} \\
0.291882
\end{pmatrix},
\end{equation*}
\begin{equation*}
\left[ \frac{{w}^{\prime}_i}{{w}^{\prime}_j} \right] =
\begin{pmatrix}
$\,\,$ 1 $\,\,$ & $\,\,$2.6777$\,\,$ & $\,\,$\color{gr} 8.1073\color{black} $\,\,$ & $\,\,$1.6215$\,\,$ \\
$\,\,$0.3735$\,\,$ & $\,\,$ 1 $\,\,$ & $\,\,$\color{gr} 3.0277\color{black} $\,\,$ & $\,\,$0.6055  $\,\,$ \\
$\,\,$\color{gr} 0.1233\color{black} $\,\,$ & $\,\,$\color{gr} 0.3303\color{black} $\,\,$ & $\,\,$ 1 $\,\,$ & $\,\,$\color{gr} \color{blue}  1/5\color{black}  $\,\,$ \\
$\,\,$0.6167$\,\,$ & $\,\,$1.6514$\,\,$ & $\,\,$\color{gr} \color{blue} 5\color{black} $\,\,$ & $\,\,$ 1  $\,\,$ \\
\end{pmatrix},
\end{equation*}
\end{example}
\newpage
\begin{example}
\begin{equation*}
\mathbf{A} =
\begin{pmatrix}
$\,\,$ 1 $\,\,$ & $\,\,$5$\,\,$ & $\,\,$8$\,\,$ & $\,\,$3 $\,\,$ \\
$\,\,$ 1/5$\,\,$ & $\,\,$ 1 $\,\,$ & $\,\,$1$\,\,$ & $\,\,$1 $\,\,$ \\
$\,\,$ 1/8$\,\,$ & $\,\,$ 1 $\,\,$ & $\,\,$ 1 $\,\,$ & $\,\,$ 1/4 $\,\,$ \\
$\,\,$ 1/3$\,\,$ & $\,\,$ 1 $\,\,$ & $\,\,$4$\,\,$ & $\,\,$ 1  $\,\,$ \\
\end{pmatrix},
\qquad
\lambda_{\max} =
4.1655,
\qquad
CR = 0.0624
\end{equation*}

\begin{equation*}
\mathbf{w}^{AMAST} =
\begin{pmatrix}
\color{red} 0.591780\color{black} \\
0.126850\\
0.080180\\
0.201190
\end{pmatrix}\end{equation*}
\begin{equation*}
\left[ \frac{{w}^{AMAST}_i}{{w}^{AMAST}_j} \right] =
\begin{pmatrix}
$\,\,$ 1 $\,\,$ & $\,\,$\color{red} 4.6652\color{black} $\,\,$ & $\,\,$\color{red} 7.3807\color{black} $\,\,$ & $\,\,$\color{red} 2.9414\color{black} $\,\,$ \\
$\,\,$\color{red} 0.2144\color{black} $\,\,$ & $\,\,$ 1 $\,\,$ & $\,\,$1.5821$\,\,$ & $\,\,$0.6305  $\,\,$ \\
$\,\,$\color{red} 0.1355\color{black} $\,\,$ & $\,\,$0.6321$\,\,$ & $\,\,$ 1 $\,\,$ & $\,\,$0.3985 $\,\,$ \\
$\,\,$\color{red} 0.3400\color{black} $\,\,$ & $\,\,$1.5860$\,\,$ & $\,\,$2.5092$\,\,$ & $\,\,$ 1  $\,\,$ \\
\end{pmatrix},
\end{equation*}

\begin{equation*}
\mathbf{w}^{\prime} =
\begin{pmatrix}
0.596537\\
0.125372\\
0.079246\\
0.198846
\end{pmatrix} =
0.988347\cdot
\begin{pmatrix}
\color{gr} 0.603570\color{black} \\
0.126850\\
0.080180\\
0.201190
\end{pmatrix},
\end{equation*}
\begin{equation*}
\left[ \frac{{w}^{\prime}_i}{{w}^{\prime}_j} \right] =
\begin{pmatrix}
$\,\,$ 1 $\,\,$ & $\,\,$\color{gr} 4.7581\color{black} $\,\,$ & $\,\,$\color{gr} 7.5277\color{black} $\,\,$ & $\,\,$\color{gr} \color{blue} 3\color{black} $\,\,$ \\
$\,\,$\color{gr} 0.2102\color{black} $\,\,$ & $\,\,$ 1 $\,\,$ & $\,\,$1.5821$\,\,$ & $\,\,$0.6305  $\,\,$ \\
$\,\,$\color{gr} 0.1328\color{black} $\,\,$ & $\,\,$0.6321$\,\,$ & $\,\,$ 1 $\,\,$ & $\,\,$0.3985 $\,\,$ \\
$\,\,$\color{gr} \color{blue}  1/3\color{black} $\,\,$ & $\,\,$1.5860$\,\,$ & $\,\,$2.5092$\,\,$ & $\,\,$ 1  $\,\,$ \\
\end{pmatrix},
\end{equation*}
\end{example}
\newpage
\begin{example}
\begin{equation*}
\mathbf{A} =
\begin{pmatrix}
$\,\,$ 1 $\,\,$ & $\,\,$5$\,\,$ & $\,\,$8$\,\,$ & $\,\,$3 $\,\,$ \\
$\,\,$ 1/5$\,\,$ & $\,\,$ 1 $\,\,$ & $\,\,$1$\,\,$ & $\,\,$1 $\,\,$ \\
$\,\,$ 1/8$\,\,$ & $\,\,$ 1 $\,\,$ & $\,\,$ 1 $\,\,$ & $\,\,$ 1/5 $\,\,$ \\
$\,\,$ 1/3$\,\,$ & $\,\,$ 1 $\,\,$ & $\,\,$5$\,\,$ & $\,\,$ 1  $\,\,$ \\
\end{pmatrix},
\qquad
\lambda_{\max} =
4.2259,
\qquad
CR = 0.0852
\end{equation*}

\begin{equation*}
\mathbf{w}^{AMAST} =
\begin{pmatrix}
\color{red} 0.583733\color{black} \\
0.127005\\
0.076507\\
0.212755
\end{pmatrix}\end{equation*}
\begin{equation*}
\left[ \frac{{w}^{AMAST}_i}{{w}^{AMAST}_j} \right] =
\begin{pmatrix}
$\,\,$ 1 $\,\,$ & $\,\,$\color{red} 4.5961\color{black} $\,\,$ & $\,\,$\color{red} 7.6298\color{black} $\,\,$ & $\,\,$\color{red} 2.7437\color{black} $\,\,$ \\
$\,\,$\color{red} 0.2176\color{black} $\,\,$ & $\,\,$ 1 $\,\,$ & $\,\,$1.6600$\,\,$ & $\,\,$0.5970  $\,\,$ \\
$\,\,$\color{red} 0.1311\color{black} $\,\,$ & $\,\,$0.6024$\,\,$ & $\,\,$ 1 $\,\,$ & $\,\,$0.3596 $\,\,$ \\
$\,\,$\color{red} 0.3645\color{black} $\,\,$ & $\,\,$1.6752$\,\,$ & $\,\,$2.7809$\,\,$ & $\,\,$ 1  $\,\,$ \\
\end{pmatrix},
\end{equation*}

\begin{equation*}
\mathbf{w}^{\prime} =
\begin{pmatrix}
0.595199\\
0.123507\\
0.074400\\
0.206895
\end{pmatrix} =
0.972456\cdot
\begin{pmatrix}
\color{gr} 0.612057\color{black} \\
0.127005\\
0.076507\\
0.212755
\end{pmatrix},
\end{equation*}
\begin{equation*}
\left[ \frac{{w}^{\prime}_i}{{w}^{\prime}_j} \right] =
\begin{pmatrix}
$\,\,$ 1 $\,\,$ & $\,\,$\color{gr} 4.8192\color{black} $\,\,$ & $\,\,$\color{gr} \color{blue} 8\color{black} $\,\,$ & $\,\,$\color{gr} 2.8768\color{black} $\,\,$ \\
$\,\,$\color{gr} 0.2075\color{black} $\,\,$ & $\,\,$ 1 $\,\,$ & $\,\,$1.6600$\,\,$ & $\,\,$0.5970  $\,\,$ \\
$\,\,$\color{gr} \color{blue}  1/8\color{black} $\,\,$ & $\,\,$0.6024$\,\,$ & $\,\,$ 1 $\,\,$ & $\,\,$0.3596 $\,\,$ \\
$\,\,$\color{gr} 0.3476\color{black} $\,\,$ & $\,\,$1.6752$\,\,$ & $\,\,$2.7809$\,\,$ & $\,\,$ 1  $\,\,$ \\
\end{pmatrix},
\end{equation*}
\end{example}
\newpage
\begin{example}
\begin{equation*}
\mathbf{A} =
\begin{pmatrix}
$\,\,$ 1 $\,\,$ & $\,\,$5$\,\,$ & $\,\,$8$\,\,$ & $\,\,$6 $\,\,$ \\
$\,\,$ 1/5$\,\,$ & $\,\,$ 1 $\,\,$ & $\,\,$1$\,\,$ & $\,\,$2 $\,\,$ \\
$\,\,$ 1/8$\,\,$ & $\,\,$ 1 $\,\,$ & $\,\,$ 1 $\,\,$ & $\,\,$ 1/2 $\,\,$ \\
$\,\,$ 1/6$\,\,$ & $\,\,$ 1/2$\,\,$ & $\,\,$2$\,\,$ & $\,\,$ 1  $\,\,$ \\
\end{pmatrix},
\qquad
\lambda_{\max} =
4.1655,
\qquad
CR = 0.0624
\end{equation*}

\begin{equation*}
\mathbf{w}^{AMAST} =
\begin{pmatrix}
\color{red} 0.656762\color{black} \\
0.140815\\
0.089053\\
0.113371
\end{pmatrix}\end{equation*}
\begin{equation*}
\left[ \frac{{w}^{AMAST}_i}{{w}^{AMAST}_j} \right] =
\begin{pmatrix}
$\,\,$ 1 $\,\,$ & $\,\,$\color{red} 4.6640\color{black} $\,\,$ & $\,\,$\color{red} 7.3750\color{black} $\,\,$ & $\,\,$\color{red} 5.7931\color{black} $\,\,$ \\
$\,\,$\color{red} 0.2144\color{black} $\,\,$ & $\,\,$ 1 $\,\,$ & $\,\,$1.5813$\,\,$ & $\,\,$1.2421  $\,\,$ \\
$\,\,$\color{red} 0.1356\color{black} $\,\,$ & $\,\,$0.6324$\,\,$ & $\,\,$ 1 $\,\,$ & $\,\,$0.7855 $\,\,$ \\
$\,\,$\color{red} 0.1726\color{black} $\,\,$ & $\,\,$0.8051$\,\,$ & $\,\,$1.2731$\,\,$ & $\,\,$ 1  $\,\,$ \\
\end{pmatrix},
\end{equation*}

\begin{equation*}
\mathbf{w}^{\prime} =
\begin{pmatrix}
0.664630\\
0.137587\\
0.087011\\
0.110772
\end{pmatrix} =
0.977076\cdot
\begin{pmatrix}
\color{gr} 0.680223\color{black} \\
0.140815\\
0.089053\\
0.113371
\end{pmatrix},
\end{equation*}
\begin{equation*}
\left[ \frac{{w}^{\prime}_i}{{w}^{\prime}_j} \right] =
\begin{pmatrix}
$\,\,$ 1 $\,\,$ & $\,\,$\color{gr} 4.8306\color{black} $\,\,$ & $\,\,$\color{gr} 7.6384\color{black} $\,\,$ & $\,\,$\color{gr} \color{blue} 6\color{black} $\,\,$ \\
$\,\,$\color{gr} 0.2070\color{black} $\,\,$ & $\,\,$ 1 $\,\,$ & $\,\,$1.5813$\,\,$ & $\,\,$1.2421  $\,\,$ \\
$\,\,$\color{gr} 0.1309\color{black} $\,\,$ & $\,\,$0.6324$\,\,$ & $\,\,$ 1 $\,\,$ & $\,\,$0.7855 $\,\,$ \\
$\,\,$\color{gr} \color{blue}  1/6\color{black} $\,\,$ & $\,\,$0.8051$\,\,$ & $\,\,$1.2731$\,\,$ & $\,\,$ 1  $\,\,$ \\
\end{pmatrix},
\end{equation*}
\end{example}
\newpage
\begin{example}
\begin{equation*}
\mathbf{A} =
\begin{pmatrix}
$\,\,$ 1 $\,\,$ & $\,\,$5$\,\,$ & $\,\,$8$\,\,$ & $\,\,$7 $\,\,$ \\
$\,\,$ 1/5$\,\,$ & $\,\,$ 1 $\,\,$ & $\,\,$1$\,\,$ & $\,\,$2 $\,\,$ \\
$\,\,$ 1/8$\,\,$ & $\,\,$ 1 $\,\,$ & $\,\,$ 1 $\,\,$ & $\,\,$ 1/2 $\,\,$ \\
$\,\,$ 1/7$\,\,$ & $\,\,$ 1/2$\,\,$ & $\,\,$2$\,\,$ & $\,\,$ 1  $\,\,$ \\
\end{pmatrix},
\qquad
\lambda_{\max} =
4.1665,
\qquad
CR = 0.0628
\end{equation*}

\begin{equation*}
\mathbf{w}^{AMAST} =
\begin{pmatrix}
\color{red} 0.667964\color{black} \\
0.137426\\
0.087446\\
0.107163
\end{pmatrix}\end{equation*}
\begin{equation*}
\left[ \frac{{w}^{AMAST}_i}{{w}^{AMAST}_j} \right] =
\begin{pmatrix}
$\,\,$ 1 $\,\,$ & $\,\,$\color{red} 4.8605\color{black} $\,\,$ & $\,\,$\color{red} 7.6386\color{black} $\,\,$ & $\,\,$\color{red} 6.2331\color{black} $\,\,$ \\
$\,\,$\color{red} 0.2057\color{black} $\,\,$ & $\,\,$ 1 $\,\,$ & $\,\,$1.5715$\,\,$ & $\,\,$1.2824  $\,\,$ \\
$\,\,$\color{red} 0.1309\color{black} $\,\,$ & $\,\,$0.6363$\,\,$ & $\,\,$ 1 $\,\,$ & $\,\,$0.8160 $\,\,$ \\
$\,\,$\color{red} 0.1604\color{black} $\,\,$ & $\,\,$0.7798$\,\,$ & $\,\,$1.2255$\,\,$ & $\,\,$ 1  $\,\,$ \\
\end{pmatrix},
\end{equation*}

\begin{equation*}
\mathbf{w}^{\prime} =
\begin{pmatrix}
0.674208\\
0.134842\\
0.085802\\
0.105148
\end{pmatrix} =
0.981196\cdot
\begin{pmatrix}
\color{gr} 0.687129\color{black} \\
0.137426\\
0.087446\\
0.107163
\end{pmatrix},
\end{equation*}
\begin{equation*}
\left[ \frac{{w}^{\prime}_i}{{w}^{\prime}_j} \right] =
\begin{pmatrix}
$\,\,$ 1 $\,\,$ & $\,\,$\color{gr} \color{blue} 5\color{black} $\,\,$ & $\,\,$\color{gr} 7.8577\color{black} $\,\,$ & $\,\,$\color{gr} 6.4120\color{black} $\,\,$ \\
$\,\,$\color{gr} \color{blue}  1/5\color{black} $\,\,$ & $\,\,$ 1 $\,\,$ & $\,\,$1.5715$\,\,$ & $\,\,$1.2824  $\,\,$ \\
$\,\,$\color{gr} 0.1273\color{black} $\,\,$ & $\,\,$0.6363$\,\,$ & $\,\,$ 1 $\,\,$ & $\,\,$0.8160 $\,\,$ \\
$\,\,$\color{gr} 0.1560\color{black} $\,\,$ & $\,\,$0.7798$\,\,$ & $\,\,$1.2255$\,\,$ & $\,\,$ 1  $\,\,$ \\
\end{pmatrix},
\end{equation*}
\end{example}
\newpage
\begin{example}
\begin{equation*}
\mathbf{A} =
\begin{pmatrix}
$\,\,$ 1 $\,\,$ & $\,\,$5$\,\,$ & $\,\,$9$\,\,$ & $\,\,$3 $\,\,$ \\
$\,\,$ 1/5$\,\,$ & $\,\,$ 1 $\,\,$ & $\,\,$1$\,\,$ & $\,\,$1 $\,\,$ \\
$\,\,$ 1/9$\,\,$ & $\,\,$ 1 $\,\,$ & $\,\,$ 1 $\,\,$ & $\,\,$ 1/5 $\,\,$ \\
$\,\,$ 1/3$\,\,$ & $\,\,$ 1 $\,\,$ & $\,\,$5$\,\,$ & $\,\,$ 1  $\,\,$ \\
\end{pmatrix},
\qquad
\lambda_{\max} =
4.2253,
\qquad
CR = 0.0849
\end{equation*}

\begin{equation*}
\mathbf{w}^{AMAST} =
\begin{pmatrix}
\color{red} 0.591889\color{black} \\
0.125495\\
0.073383\\
0.209234
\end{pmatrix}\end{equation*}
\begin{equation*}
\left[ \frac{{w}^{AMAST}_i}{{w}^{AMAST}_j} \right] =
\begin{pmatrix}
$\,\,$ 1 $\,\,$ & $\,\,$\color{red} 4.7164\color{black} $\,\,$ & $\,\,$\color{red} 8.0657\color{black} $\,\,$ & $\,\,$\color{red} 2.8288\color{black} $\,\,$ \\
$\,\,$\color{red} 0.2120\color{black} $\,\,$ & $\,\,$ 1 $\,\,$ & $\,\,$1.7101$\,\,$ & $\,\,$0.5998  $\,\,$ \\
$\,\,$\color{red} 0.1240\color{black} $\,\,$ & $\,\,$0.5847$\,\,$ & $\,\,$ 1 $\,\,$ & $\,\,$0.3507 $\,\,$ \\
$\,\,$\color{red} 0.3535\color{black} $\,\,$ & $\,\,$1.6673$\,\,$ & $\,\,$2.8513$\,\,$ & $\,\,$ 1  $\,\,$ \\
\end{pmatrix},
\end{equation*}

\begin{equation*}
\mathbf{w}^{\prime} =
\begin{pmatrix}
0.605912\\
0.121182\\
0.070861\\
0.202044
\end{pmatrix} =
0.965638\cdot
\begin{pmatrix}
\color{gr} 0.627473\color{black} \\
0.125495\\
0.073383\\
0.209234
\end{pmatrix},
\end{equation*}
\begin{equation*}
\left[ \frac{{w}^{\prime}_i}{{w}^{\prime}_j} \right] =
\begin{pmatrix}
$\,\,$ 1 $\,\,$ & $\,\,$\color{gr} \color{blue} 5\color{black} $\,\,$ & $\,\,$\color{gr} 8.5507\color{black} $\,\,$ & $\,\,$\color{gr} 2.9989\color{black} $\,\,$ \\
$\,\,$\color{gr} \color{blue}  1/5\color{black} $\,\,$ & $\,\,$ 1 $\,\,$ & $\,\,$1.7101$\,\,$ & $\,\,$0.5998  $\,\,$ \\
$\,\,$\color{gr} 0.1169\color{black} $\,\,$ & $\,\,$0.5847$\,\,$ & $\,\,$ 1 $\,\,$ & $\,\,$0.3507 $\,\,$ \\
$\,\,$\color{gr} 0.3335\color{black} $\,\,$ & $\,\,$1.6673$\,\,$ & $\,\,$2.8513$\,\,$ & $\,\,$ 1  $\,\,$ \\
\end{pmatrix},
\end{equation*}
\end{example}
\newpage
\begin{example}
\begin{equation*}
\mathbf{A} =
\begin{pmatrix}
$\,\,$ 1 $\,\,$ & $\,\,$5$\,\,$ & $\,\,$9$\,\,$ & $\,\,$6 $\,\,$ \\
$\,\,$ 1/5$\,\,$ & $\,\,$ 1 $\,\,$ & $\,\,$1$\,\,$ & $\,\,$2 $\,\,$ \\
$\,\,$ 1/9$\,\,$ & $\,\,$ 1 $\,\,$ & $\,\,$ 1 $\,\,$ & $\,\,$ 1/2 $\,\,$ \\
$\,\,$ 1/6$\,\,$ & $\,\,$ 1/2$\,\,$ & $\,\,$2$\,\,$ & $\,\,$ 1  $\,\,$ \\
\end{pmatrix},
\qquad
\lambda_{\max} =
4.1683,
\qquad
CR = 0.0634
\end{equation*}

\begin{equation*}
\mathbf{w}^{AMAST} =
\begin{pmatrix}
\color{red} 0.664599\color{black} \\
0.138832\\
0.085331\\
0.111238
\end{pmatrix}\end{equation*}
\begin{equation*}
\left[ \frac{{w}^{AMAST}_i}{{w}^{AMAST}_j} \right] =
\begin{pmatrix}
$\,\,$ 1 $\,\,$ & $\,\,$\color{red} 4.7871\color{black} $\,\,$ & $\,\,$\color{red} 7.7885\color{black} $\,\,$ & $\,\,$\color{red} 5.9746\color{black} $\,\,$ \\
$\,\,$\color{red} 0.2089\color{black} $\,\,$ & $\,\,$ 1 $\,\,$ & $\,\,$1.6270$\,\,$ & $\,\,$1.2481  $\,\,$ \\
$\,\,$\color{red} 0.1284\color{black} $\,\,$ & $\,\,$0.6146$\,\,$ & $\,\,$ 1 $\,\,$ & $\,\,$0.7671 $\,\,$ \\
$\,\,$\color{red} 0.1674\color{black} $\,\,$ & $\,\,$0.8012$\,\,$ & $\,\,$1.3036$\,\,$ & $\,\,$ 1  $\,\,$ \\
\end{pmatrix},
\end{equation*}

\begin{equation*}
\mathbf{w}^{\prime} =
\begin{pmatrix}
0.665545\\
0.138440\\
0.085090\\
0.110924
\end{pmatrix} =
0.997179\cdot
\begin{pmatrix}
\color{gr} 0.667428\color{black} \\
0.138832\\
0.085331\\
0.111238
\end{pmatrix},
\end{equation*}
\begin{equation*}
\left[ \frac{{w}^{\prime}_i}{{w}^{\prime}_j} \right] =
\begin{pmatrix}
$\,\,$ 1 $\,\,$ & $\,\,$\color{gr} 4.8075\color{black} $\,\,$ & $\,\,$\color{gr} 7.8216\color{black} $\,\,$ & $\,\,$\color{gr} \color{blue} 6\color{black} $\,\,$ \\
$\,\,$\color{gr} 0.2080\color{black} $\,\,$ & $\,\,$ 1 $\,\,$ & $\,\,$1.6270$\,\,$ & $\,\,$1.2481  $\,\,$ \\
$\,\,$\color{gr} 0.1279\color{black} $\,\,$ & $\,\,$0.6146$\,\,$ & $\,\,$ 1 $\,\,$ & $\,\,$0.7671 $\,\,$ \\
$\,\,$\color{gr} \color{blue}  1/6\color{black} $\,\,$ & $\,\,$0.8012$\,\,$ & $\,\,$1.3036$\,\,$ & $\,\,$ 1  $\,\,$ \\
\end{pmatrix},
\end{equation*}
\end{example}
\newpage
\begin{example}
\begin{equation*}
\mathbf{A} =
\begin{pmatrix}
$\,\,$ 1 $\,\,$ & $\,\,$5$\,\,$ & $\,\,$9$\,\,$ & $\,\,$7 $\,\,$ \\
$\,\,$ 1/5$\,\,$ & $\,\,$ 1 $\,\,$ & $\,\,$1$\,\,$ & $\,\,$2 $\,\,$ \\
$\,\,$ 1/9$\,\,$ & $\,\,$ 1 $\,\,$ & $\,\,$ 1 $\,\,$ & $\,\,$ 1/2 $\,\,$ \\
$\,\,$ 1/7$\,\,$ & $\,\,$ 1/2$\,\,$ & $\,\,$2$\,\,$ & $\,\,$ 1  $\,\,$ \\
\end{pmatrix},
\qquad
\lambda_{\max} =
4.1669,
\qquad
CR = 0.0629
\end{equation*}

\begin{equation*}
\mathbf{w}^{AMAST} =
\begin{pmatrix}
\color{red} 0.675877\color{black} \\
0.135431\\
0.083689\\
0.105003
\end{pmatrix}\end{equation*}
\begin{equation*}
\left[ \frac{{w}^{AMAST}_i}{{w}^{AMAST}_j} \right] =
\begin{pmatrix}
$\,\,$ 1 $\,\,$ & $\,\,$\color{red} 4.9906\color{black} $\,\,$ & $\,\,$\color{red} 8.0761\color{black} $\,\,$ & $\,\,$\color{red} 6.4367\color{black} $\,\,$ \\
$\,\,$\color{red} 0.2004\color{black} $\,\,$ & $\,\,$ 1 $\,\,$ & $\,\,$1.6183$\,\,$ & $\,\,$1.2898  $\,\,$ \\
$\,\,$\color{red} 0.1238\color{black} $\,\,$ & $\,\,$0.6179$\,\,$ & $\,\,$ 1 $\,\,$ & $\,\,$0.7970 $\,\,$ \\
$\,\,$\color{red} 0.1554\color{black} $\,\,$ & $\,\,$0.7753$\,\,$ & $\,\,$1.2547$\,\,$ & $\,\,$ 1  $\,\,$ \\
\end{pmatrix},
\end{equation*}

\begin{equation*}
\mathbf{w}^{\prime} =
\begin{pmatrix}
0.676291\\
0.135258\\
0.083582\\
0.104869
\end{pmatrix} =
0.998723\cdot
\begin{pmatrix}
\color{gr} 0.677156\color{black} \\
0.135431\\
0.083689\\
0.105003
\end{pmatrix},
\end{equation*}
\begin{equation*}
\left[ \frac{{w}^{\prime}_i}{{w}^{\prime}_j} \right] =
\begin{pmatrix}
$\,\,$ 1 $\,\,$ & $\,\,$\color{gr} \color{blue} 5\color{black} $\,\,$ & $\,\,$\color{gr} 8.0914\color{black} $\,\,$ & $\,\,$\color{gr} 6.4489\color{black} $\,\,$ \\
$\,\,$\color{gr} \color{blue}  1/5\color{black} $\,\,$ & $\,\,$ 1 $\,\,$ & $\,\,$1.6183$\,\,$ & $\,\,$1.2898  $\,\,$ \\
$\,\,$\color{gr} 0.1236\color{black} $\,\,$ & $\,\,$0.6179$\,\,$ & $\,\,$ 1 $\,\,$ & $\,\,$0.7970 $\,\,$ \\
$\,\,$\color{gr} 0.1551\color{black} $\,\,$ & $\,\,$0.7753$\,\,$ & $\,\,$1.2547$\,\,$ & $\,\,$ 1  $\,\,$ \\
\end{pmatrix},
\end{equation*}
\end{example}
\newpage
\begin{example}
\begin{equation*}
\mathbf{A} =
\begin{pmatrix}
$\,\,$ 1 $\,\,$ & $\,\,$6$\,\,$ & $\,\,$8$\,\,$ & $\,\,$4 $\,\,$ \\
$\,\,$ 1/6$\,\,$ & $\,\,$ 1 $\,\,$ & $\,\,$1$\,\,$ & $\,\,$1 $\,\,$ \\
$\,\,$ 1/8$\,\,$ & $\,\,$ 1 $\,\,$ & $\,\,$ 1 $\,\,$ & $\,\,$ 1/3 $\,\,$ \\
$\,\,$ 1/4$\,\,$ & $\,\,$ 1 $\,\,$ & $\,\,$3$\,\,$ & $\,\,$ 1  $\,\,$ \\
\end{pmatrix},
\qquad
\lambda_{\max} =
4.1031,
\qquad
CR = 0.0389
\end{equation*}

\begin{equation*}
\mathbf{w}^{AMAST} =
\begin{pmatrix}
\color{red} 0.639479\color{black} \\
0.114086\\
0.080856\\
0.165579
\end{pmatrix}\end{equation*}
\begin{equation*}
\left[ \frac{{w}^{AMAST}_i}{{w}^{AMAST}_j} \right] =
\begin{pmatrix}
$\,\,$ 1 $\,\,$ & $\,\,$\color{red} 5.6052\color{black} $\,\,$ & $\,\,$\color{red} 7.9088\color{black} $\,\,$ & $\,\,$\color{red} 3.8621\color{black} $\,\,$ \\
$\,\,$\color{red} 0.1784\color{black} $\,\,$ & $\,\,$ 1 $\,\,$ & $\,\,$1.4110$\,\,$ & $\,\,$0.6890  $\,\,$ \\
$\,\,$\color{red} 0.1264\color{black} $\,\,$ & $\,\,$0.7087$\,\,$ & $\,\,$ 1 $\,\,$ & $\,\,$0.4883 $\,\,$ \\
$\,\,$\color{red} 0.2589\color{black} $\,\,$ & $\,\,$1.4514$\,\,$ & $\,\,$2.0478$\,\,$ & $\,\,$ 1  $\,\,$ \\
\end{pmatrix},
\end{equation*}

\begin{equation*}
\mathbf{w}^{\prime} =
\begin{pmatrix}
0.642117\\
0.113251\\
0.080265\\
0.164367
\end{pmatrix} =
0.992682\cdot
\begin{pmatrix}
\color{gr} 0.646851\color{black} \\
0.114086\\
0.080856\\
0.165579
\end{pmatrix},
\end{equation*}
\begin{equation*}
\left[ \frac{{w}^{\prime}_i}{{w}^{\prime}_j} \right] =
\begin{pmatrix}
$\,\,$ 1 $\,\,$ & $\,\,$\color{gr} 5.6699\color{black} $\,\,$ & $\,\,$\color{gr} \color{blue} 8\color{black} $\,\,$ & $\,\,$\color{gr} 3.9066\color{black} $\,\,$ \\
$\,\,$\color{gr} 0.1764\color{black} $\,\,$ & $\,\,$ 1 $\,\,$ & $\,\,$1.4110$\,\,$ & $\,\,$0.6890  $\,\,$ \\
$\,\,$\color{gr} \color{blue}  1/8\color{black} $\,\,$ & $\,\,$0.7087$\,\,$ & $\,\,$ 1 $\,\,$ & $\,\,$0.4883 $\,\,$ \\
$\,\,$\color{gr} 0.2560\color{black} $\,\,$ & $\,\,$1.4514$\,\,$ & $\,\,$2.0478$\,\,$ & $\,\,$ 1  $\,\,$ \\
\end{pmatrix},
\end{equation*}
\end{example}
\newpage
\begin{example}
\begin{equation*}
\mathbf{A} =
\begin{pmatrix}
$\,\,$ 1 $\,\,$ & $\,\,$6$\,\,$ & $\,\,$9$\,\,$ & $\,\,$3 $\,\,$ \\
$\,\,$ 1/6$\,\,$ & $\,\,$ 1 $\,\,$ & $\,\,$1$\,\,$ & $\,\,$1 $\,\,$ \\
$\,\,$ 1/9$\,\,$ & $\,\,$ 1 $\,\,$ & $\,\,$ 1 $\,\,$ & $\,\,$ 1/5 $\,\,$ \\
$\,\,$ 1/3$\,\,$ & $\,\,$ 1 $\,\,$ & $\,\,$5$\,\,$ & $\,\,$ 1  $\,\,$ \\
\end{pmatrix},
\qquad
\lambda_{\max} =
4.2277,
\qquad
CR = 0.0859
\end{equation*}

\begin{equation*}
\mathbf{w}^{AMAST} =
\begin{pmatrix}
\color{red} 0.605204\color{black} \\
0.117831\\
0.071364\\
0.205601
\end{pmatrix}\end{equation*}
\begin{equation*}
\left[ \frac{{w}^{AMAST}_i}{{w}^{AMAST}_j} \right] =
\begin{pmatrix}
$\,\,$ 1 $\,\,$ & $\,\,$\color{red} 5.1362\color{black} $\,\,$ & $\,\,$\color{red} 8.4806\color{black} $\,\,$ & $\,\,$\color{red} 2.9436\color{black} $\,\,$ \\
$\,\,$\color{red} 0.1947\color{black} $\,\,$ & $\,\,$ 1 $\,\,$ & $\,\,$1.6511$\,\,$ & $\,\,$0.5731  $\,\,$ \\
$\,\,$\color{red} 0.1179\color{black} $\,\,$ & $\,\,$0.6056$\,\,$ & $\,\,$ 1 $\,\,$ & $\,\,$0.3471 $\,\,$ \\
$\,\,$\color{red} 0.3397\color{black} $\,\,$ & $\,\,$1.7449$\,\,$ & $\,\,$2.8810$\,\,$ & $\,\,$ 1  $\,\,$ \\
\end{pmatrix},
\end{equation*}

\begin{equation*}
\mathbf{w}^{\prime} =
\begin{pmatrix}
0.609731\\
0.116480\\
0.070545\\
0.203244
\end{pmatrix} =
0.988533\cdot
\begin{pmatrix}
\color{gr} 0.616804\color{black} \\
0.117831\\
0.071364\\
0.205601
\end{pmatrix},
\end{equation*}
\begin{equation*}
\left[ \frac{{w}^{\prime}_i}{{w}^{\prime}_j} \right] =
\begin{pmatrix}
$\,\,$ 1 $\,\,$ & $\,\,$\color{gr} 5.2346\color{black} $\,\,$ & $\,\,$\color{gr} 8.6431\color{black} $\,\,$ & $\,\,$\color{gr} \color{blue} 3\color{black} $\,\,$ \\
$\,\,$\color{gr} 0.1910\color{black} $\,\,$ & $\,\,$ 1 $\,\,$ & $\,\,$1.6511$\,\,$ & $\,\,$0.5731  $\,\,$ \\
$\,\,$\color{gr} 0.1157\color{black} $\,\,$ & $\,\,$0.6056$\,\,$ & $\,\,$ 1 $\,\,$ & $\,\,$0.3471 $\,\,$ \\
$\,\,$\color{gr} \color{blue}  1/3\color{black} $\,\,$ & $\,\,$1.7449$\,\,$ & $\,\,$2.8810$\,\,$ & $\,\,$ 1  $\,\,$ \\
\end{pmatrix},
\end{equation*}
\end{example}
\newpage
\begin{example}
\begin{equation*}
\mathbf{A} =
\begin{pmatrix}
$\,\,$ 1 $\,\,$ & $\,\,$6$\,\,$ & $\,\,$9$\,\,$ & $\,\,$4 $\,\,$ \\
$\,\,$ 1/6$\,\,$ & $\,\,$ 1 $\,\,$ & $\,\,$1$\,\,$ & $\,\,$1 $\,\,$ \\
$\,\,$ 1/9$\,\,$ & $\,\,$ 1 $\,\,$ & $\,\,$ 1 $\,\,$ & $\,\,$ 1/3 $\,\,$ \\
$\,\,$ 1/4$\,\,$ & $\,\,$ 1 $\,\,$ & $\,\,$3$\,\,$ & $\,\,$ 1  $\,\,$ \\
\end{pmatrix},
\qquad
\lambda_{\max} =
4.1031,
\qquad
CR = 0.0389
\end{equation*}

\begin{equation*}
\mathbf{w}^{AMAST} =
\begin{pmatrix}
\color{red} 0.647631\color{black} \\
0.112425\\
0.077342\\
0.162603
\end{pmatrix}\end{equation*}
\begin{equation*}
\left[ \frac{{w}^{AMAST}_i}{{w}^{AMAST}_j} \right] =
\begin{pmatrix}
$\,\,$ 1 $\,\,$ & $\,\,$\color{red} 5.7606\color{black} $\,\,$ & $\,\,$\color{red} 8.3736\color{black} $\,\,$ & $\,\,$\color{red} 3.9829\color{black} $\,\,$ \\
$\,\,$\color{red} 0.1736\color{black} $\,\,$ & $\,\,$ 1 $\,\,$ & $\,\,$1.4536$\,\,$ & $\,\,$0.6914  $\,\,$ \\
$\,\,$\color{red} 0.1194\color{black} $\,\,$ & $\,\,$0.6879$\,\,$ & $\,\,$ 1 $\,\,$ & $\,\,$0.4756 $\,\,$ \\
$\,\,$\color{red} 0.2511\color{black} $\,\,$ & $\,\,$1.4463$\,\,$ & $\,\,$2.1024$\,\,$ & $\,\,$ 1  $\,\,$ \\
\end{pmatrix},
\end{equation*}

\begin{equation*}
\mathbf{w}^{\prime} =
\begin{pmatrix}
0.648608\\
0.112113\\
0.077127\\
0.162152
\end{pmatrix} =
0.997228\cdot
\begin{pmatrix}
\color{gr} 0.650411\color{black} \\
0.112425\\
0.077342\\
0.162603
\end{pmatrix},
\end{equation*}
\begin{equation*}
\left[ \frac{{w}^{\prime}_i}{{w}^{\prime}_j} \right] =
\begin{pmatrix}
$\,\,$ 1 $\,\,$ & $\,\,$\color{gr} 5.7853\color{black} $\,\,$ & $\,\,$\color{gr} 8.4096\color{black} $\,\,$ & $\,\,$\color{gr} \color{blue} 4\color{black} $\,\,$ \\
$\,\,$\color{gr} 0.1729\color{black} $\,\,$ & $\,\,$ 1 $\,\,$ & $\,\,$1.4536$\,\,$ & $\,\,$0.6914  $\,\,$ \\
$\,\,$\color{gr} 0.1189\color{black} $\,\,$ & $\,\,$0.6879$\,\,$ & $\,\,$ 1 $\,\,$ & $\,\,$0.4756 $\,\,$ \\
$\,\,$\color{gr} \color{blue}  1/4\color{black} $\,\,$ & $\,\,$1.4463$\,\,$ & $\,\,$2.1024$\,\,$ & $\,\,$ 1  $\,\,$ \\
\end{pmatrix},
\end{equation*}
\end{example}
\newpage
\begin{example}
\begin{equation*}
\mathbf{A} =
\begin{pmatrix}
$\,\,$ 1 $\,\,$ & $\,\,$6$\,\,$ & $\,\,$9$\,\,$ & $\,\,$4 $\,\,$ \\
$\,\,$ 1/6$\,\,$ & $\,\,$ 1 $\,\,$ & $\,\,$1$\,\,$ & $\,\,$1 $\,\,$ \\
$\,\,$ 1/9$\,\,$ & $\,\,$ 1 $\,\,$ & $\,\,$ 1 $\,\,$ & $\,\,$ 1/4 $\,\,$ \\
$\,\,$ 1/4$\,\,$ & $\,\,$ 1 $\,\,$ & $\,\,$4$\,\,$ & $\,\,$ 1  $\,\,$ \\
\end{pmatrix},
\qquad
\lambda_{\max} =
4.1664,
\qquad
CR = 0.0627
\end{equation*}

\begin{equation*}
\mathbf{w}^{AMAST} =
\begin{pmatrix}
\color{red} 0.638607\color{black} \\
0.112876\\
0.072796\\
0.175721
\end{pmatrix}\end{equation*}
\begin{equation*}
\left[ \frac{{w}^{AMAST}_i}{{w}^{AMAST}_j} \right] =
\begin{pmatrix}
$\,\,$ 1 $\,\,$ & $\,\,$\color{red} 5.6576\color{black} $\,\,$ & $\,\,$\color{red} 8.7726\color{black} $\,\,$ & $\,\,$\color{red} 3.6342\color{black} $\,\,$ \\
$\,\,$\color{red} 0.1768\color{black} $\,\,$ & $\,\,$ 1 $\,\,$ & $\,\,$1.5506$\,\,$ & $\,\,$0.6424  $\,\,$ \\
$\,\,$\color{red} 0.1140\color{black} $\,\,$ & $\,\,$0.6449$\,\,$ & $\,\,$ 1 $\,\,$ & $\,\,$0.4143 $\,\,$ \\
$\,\,$\color{red} 0.2752\color{black} $\,\,$ & $\,\,$1.5568$\,\,$ & $\,\,$2.4139$\,\,$ & $\,\,$ 1  $\,\,$ \\
\end{pmatrix},
\end{equation*}

\begin{equation*}
\mathbf{w}^{\prime} =
\begin{pmatrix}
0.644493\\
0.111038\\
0.071610\\
0.172859
\end{pmatrix} =
0.983713\cdot
\begin{pmatrix}
\color{gr} 0.655163\color{black} \\
0.112876\\
0.072796\\
0.175721
\end{pmatrix},
\end{equation*}
\begin{equation*}
\left[ \frac{{w}^{\prime}_i}{{w}^{\prime}_j} \right] =
\begin{pmatrix}
$\,\,$ 1 $\,\,$ & $\,\,$\color{gr} 5.8043\color{black} $\,\,$ & $\,\,$\color{gr} \color{blue} 9\color{black} $\,\,$ & $\,\,$\color{gr} 3.7284\color{black} $\,\,$ \\
$\,\,$\color{gr} 0.1723\color{black} $\,\,$ & $\,\,$ 1 $\,\,$ & $\,\,$1.5506$\,\,$ & $\,\,$0.6424  $\,\,$ \\
$\,\,$\color{gr} \color{blue}  1/9\color{black} $\,\,$ & $\,\,$0.6449$\,\,$ & $\,\,$ 1 $\,\,$ & $\,\,$0.4143 $\,\,$ \\
$\,\,$\color{gr} 0.2682\color{black} $\,\,$ & $\,\,$1.5568$\,\,$ & $\,\,$2.4139$\,\,$ & $\,\,$ 1  $\,\,$ \\
\end{pmatrix},
\end{equation*}
\end{example}
\newpage
\begin{example}
\begin{equation*}
\mathbf{A} =
\begin{pmatrix}
$\,\,$ 1 $\,\,$ & $\,\,$6$\,\,$ & $\,\,$9$\,\,$ & $\,\,$7 $\,\,$ \\
$\,\,$ 1/6$\,\,$ & $\,\,$ 1 $\,\,$ & $\,\,$1$\,\,$ & $\,\,$2 $\,\,$ \\
$\,\,$ 1/9$\,\,$ & $\,\,$ 1 $\,\,$ & $\,\,$ 1 $\,\,$ & $\,\,$ 1/2 $\,\,$ \\
$\,\,$ 1/7$\,\,$ & $\,\,$ 1/2$\,\,$ & $\,\,$2$\,\,$ & $\,\,$ 1  $\,\,$ \\
\end{pmatrix},
\qquad
\lambda_{\max} =
4.1658,
\qquad
CR = 0.0625
\end{equation*}

\begin{equation*}
\mathbf{w}^{AMAST} =
\begin{pmatrix}
\color{red} 0.689634\color{black} \\
0.126543\\
0.081140\\
0.102683
\end{pmatrix}\end{equation*}
\begin{equation*}
\left[ \frac{{w}^{AMAST}_i}{{w}^{AMAST}_j} \right] =
\begin{pmatrix}
$\,\,$ 1 $\,\,$ & $\,\,$\color{red} 5.4498\color{black} $\,\,$ & $\,\,$\color{red} 8.4994\color{black} $\,\,$ & $\,\,$\color{red} 6.7161\color{black} $\,\,$ \\
$\,\,$\color{red} 0.1835\color{black} $\,\,$ & $\,\,$ 1 $\,\,$ & $\,\,$1.5596$\,\,$ & $\,\,$1.2324  $\,\,$ \\
$\,\,$\color{red} 0.1177\color{black} $\,\,$ & $\,\,$0.6412$\,\,$ & $\,\,$ 1 $\,\,$ & $\,\,$0.7902 $\,\,$ \\
$\,\,$\color{red} 0.1489\color{black} $\,\,$ & $\,\,$0.8114$\,\,$ & $\,\,$1.2655$\,\,$ & $\,\,$ 1  $\,\,$ \\
\end{pmatrix},
\end{equation*}

\begin{equation*}
\mathbf{w}^{\prime} =
\begin{pmatrix}
0.698425\\
0.122959\\
0.078841\\
0.099775
\end{pmatrix} =
0.971675\cdot
\begin{pmatrix}
\color{gr} 0.718784\color{black} \\
0.126543\\
0.081140\\
0.102683
\end{pmatrix},
\end{equation*}
\begin{equation*}
\left[ \frac{{w}^{\prime}_i}{{w}^{\prime}_j} \right] =
\begin{pmatrix}
$\,\,$ 1 $\,\,$ & $\,\,$\color{gr} 5.6801\color{black} $\,\,$ & $\,\,$\color{gr} 8.8586\color{black} $\,\,$ & $\,\,$\color{gr} \color{blue} 7\color{black} $\,\,$ \\
$\,\,$\color{gr} 0.1761\color{black} $\,\,$ & $\,\,$ 1 $\,\,$ & $\,\,$1.5596$\,\,$ & $\,\,$1.2324  $\,\,$ \\
$\,\,$\color{gr} 0.1129\color{black} $\,\,$ & $\,\,$0.6412$\,\,$ & $\,\,$ 1 $\,\,$ & $\,\,$0.7902 $\,\,$ \\
$\,\,$\color{gr} \color{blue}  1/7\color{black} $\,\,$ & $\,\,$0.8114$\,\,$ & $\,\,$1.2655$\,\,$ & $\,\,$ 1  $\,\,$ \\
\end{pmatrix},
\end{equation*}
\end{example}
\newpage
\begin{example}
\begin{equation*}
\mathbf{A} =
\begin{pmatrix}
$\,\,$ 1 $\,\,$ & $\,\,$6$\,\,$ & $\,\,$9$\,\,$ & $\,\,$8 $\,\,$ \\
$\,\,$ 1/6$\,\,$ & $\,\,$ 1 $\,\,$ & $\,\,$1$\,\,$ & $\,\,$2 $\,\,$ \\
$\,\,$ 1/9$\,\,$ & $\,\,$ 1 $\,\,$ & $\,\,$ 1 $\,\,$ & $\,\,$ 1/2 $\,\,$ \\
$\,\,$ 1/8$\,\,$ & $\,\,$ 1/2$\,\,$ & $\,\,$2$\,\,$ & $\,\,$ 1  $\,\,$ \\
\end{pmatrix},
\qquad
\lambda_{\max} =
4.1664,
\qquad
CR = 0.0627
\end{equation*}

\begin{equation*}
\mathbf{w}^{AMAST} =
\begin{pmatrix}
\color{red} 0.698881\color{black} \\
0.123672\\
0.079763\\
0.097684
\end{pmatrix}\end{equation*}
\begin{equation*}
\left[ \frac{{w}^{AMAST}_i}{{w}^{AMAST}_j} \right] =
\begin{pmatrix}
$\,\,$ 1 $\,\,$ & $\,\,$\color{red} 5.6511\color{black} $\,\,$ & $\,\,$\color{red} 8.7620\color{black} $\,\,$ & $\,\,$\color{red} 7.1545\color{black} $\,\,$ \\
$\,\,$\color{red} 0.1770\color{black} $\,\,$ & $\,\,$ 1 $\,\,$ & $\,\,$1.5505$\,\,$ & $\,\,$1.2660  $\,\,$ \\
$\,\,$\color{red} 0.1141\color{black} $\,\,$ & $\,\,$0.6450$\,\,$ & $\,\,$ 1 $\,\,$ & $\,\,$0.8165 $\,\,$ \\
$\,\,$\color{red} 0.1398\color{black} $\,\,$ & $\,\,$0.7899$\,\,$ & $\,\,$1.2247$\,\,$ & $\,\,$ 1  $\,\,$ \\
\end{pmatrix},
\end{equation*}

\begin{equation*}
\mathbf{w}^{\prime} =
\begin{pmatrix}
0.704491\\
0.121368\\
0.078277\\
0.095864
\end{pmatrix} =
0.981370\cdot
\begin{pmatrix}
\color{gr} 0.717865\color{black} \\
0.123672\\
0.079763\\
0.097684
\end{pmatrix},
\end{equation*}
\begin{equation*}
\left[ \frac{{w}^{\prime}_i}{{w}^{\prime}_j} \right] =
\begin{pmatrix}
$\,\,$ 1 $\,\,$ & $\,\,$\color{gr} 5.8046\color{black} $\,\,$ & $\,\,$\color{gr} \color{blue} 9\color{black} $\,\,$ & $\,\,$\color{gr} 7.3489\color{black} $\,\,$ \\
$\,\,$\color{gr} 0.1723\color{black} $\,\,$ & $\,\,$ 1 $\,\,$ & $\,\,$1.5505$\,\,$ & $\,\,$1.2660  $\,\,$ \\
$\,\,$\color{gr} \color{blue}  1/9\color{black} $\,\,$ & $\,\,$0.6450$\,\,$ & $\,\,$ 1 $\,\,$ & $\,\,$0.8165 $\,\,$ \\
$\,\,$\color{gr} 0.1361\color{black} $\,\,$ & $\,\,$0.7899$\,\,$ & $\,\,$1.2247$\,\,$ & $\,\,$ 1  $\,\,$ \\
\end{pmatrix},
\end{equation*}
\end{example}
\newpage
\begin{example}
\begin{equation*}
\mathbf{A} =
\begin{pmatrix}
$\,\,$ 1 $\,\,$ & $\,\,$6$\,\,$ & $\,\,$9$\,\,$ & $\,\,$9 $\,\,$ \\
$\,\,$ 1/6$\,\,$ & $\,\,$ 1 $\,\,$ & $\,\,$1$\,\,$ & $\,\,$2 $\,\,$ \\
$\,\,$ 1/9$\,\,$ & $\,\,$ 1 $\,\,$ & $\,\,$ 1 $\,\,$ & $\,\,$ 1/2 $\,\,$ \\
$\,\,$ 1/9$\,\,$ & $\,\,$ 1/2$\,\,$ & $\,\,$2$\,\,$ & $\,\,$ 1  $\,\,$ \\
\end{pmatrix},
\qquad
\lambda_{\max} =
4.1707,
\qquad
CR = 0.0644
\end{equation*}

\begin{equation*}
\mathbf{w}^{AMAST} =
\begin{pmatrix}
\color{red} 0.706532\color{black} \\
0.121250\\
0.078601\\
0.093618
\end{pmatrix}\end{equation*}
\begin{equation*}
\left[ \frac{{w}^{AMAST}_i}{{w}^{AMAST}_j} \right] =
\begin{pmatrix}
$\,\,$ 1 $\,\,$ & $\,\,$\color{red} 5.8271\color{black} $\,\,$ & $\,\,$\color{red} 8.9889\color{black} $\,\,$ & $\,\,$\color{red} 7.5470\color{black} $\,\,$ \\
$\,\,$\color{red} 0.1716\color{black} $\,\,$ & $\,\,$ 1 $\,\,$ & $\,\,$1.5426$\,\,$ & $\,\,$1.2952  $\,\,$ \\
$\,\,$\color{red} 0.1112\color{black} $\,\,$ & $\,\,$0.6483$\,\,$ & $\,\,$ 1 $\,\,$ & $\,\,$0.8396 $\,\,$ \\
$\,\,$\color{red} 0.1325\color{black} $\,\,$ & $\,\,$0.7721$\,\,$ & $\,\,$1.1911$\,\,$ & $\,\,$ 1  $\,\,$ \\
\end{pmatrix},
\end{equation*}

\begin{equation*}
\mathbf{w}^{\prime} =
\begin{pmatrix}
0.706788\\
0.121144\\
0.078532\\
0.093536
\end{pmatrix} =
0.999126\cdot
\begin{pmatrix}
\color{gr} 0.707406\color{black} \\
0.121250\\
0.078601\\
0.093618
\end{pmatrix},
\end{equation*}
\begin{equation*}
\left[ \frac{{w}^{\prime}_i}{{w}^{\prime}_j} \right] =
\begin{pmatrix}
$\,\,$ 1 $\,\,$ & $\,\,$\color{gr} 5.8343\color{black} $\,\,$ & $\,\,$\color{gr} \color{blue} 9\color{black} $\,\,$ & $\,\,$\color{gr} 7.5563\color{black} $\,\,$ \\
$\,\,$\color{gr} 0.1714\color{black} $\,\,$ & $\,\,$ 1 $\,\,$ & $\,\,$1.5426$\,\,$ & $\,\,$1.2952  $\,\,$ \\
$\,\,$\color{gr} \color{blue}  1/9\color{black} $\,\,$ & $\,\,$0.6483$\,\,$ & $\,\,$ 1 $\,\,$ & $\,\,$0.8396 $\,\,$ \\
$\,\,$\color{gr} 0.1323\color{black} $\,\,$ & $\,\,$0.7721$\,\,$ & $\,\,$1.1911$\,\,$ & $\,\,$ 1  $\,\,$ \\
\end{pmatrix},
\end{equation*}
\end{example}
\newpage
\begin{example}
\begin{equation*}
\mathbf{A} =
\begin{pmatrix}
$\,\,$ 1 $\,\,$ & $\,\,$7$\,\,$ & $\,\,$9$\,\,$ & $\,\,$7 $\,\,$ \\
$\,\,$ 1/7$\,\,$ & $\,\,$ 1 $\,\,$ & $\,\,$1$\,\,$ & $\,\,$2 $\,\,$ \\
$\,\,$ 1/9$\,\,$ & $\,\,$ 1 $\,\,$ & $\,\,$ 1 $\,\,$ & $\,\,$ 1/2 $\,\,$ \\
$\,\,$ 1/7$\,\,$ & $\,\,$ 1/2$\,\,$ & $\,\,$2$\,\,$ & $\,\,$ 1  $\,\,$ \\
\end{pmatrix},
\qquad
\lambda_{\max} =
4.1714,
\qquad
CR = 0.0646
\end{equation*}

\begin{equation*}
\mathbf{w}^{AMAST} =
\begin{pmatrix}
\color{red} 0.700319\color{black} \\
0.119756\\
0.079113\\
0.100812
\end{pmatrix}\end{equation*}
\begin{equation*}
\left[ \frac{{w}^{AMAST}_i}{{w}^{AMAST}_j} \right] =
\begin{pmatrix}
$\,\,$ 1 $\,\,$ & $\,\,$\color{red} 5.8479\color{black} $\,\,$ & $\,\,$\color{red} 8.8522\color{black} $\,\,$ & $\,\,$\color{red} 6.9467\color{black} $\,\,$ \\
$\,\,$\color{red} 0.1710\color{black} $\,\,$ & $\,\,$ 1 $\,\,$ & $\,\,$1.5137$\,\,$ & $\,\,$1.1879  $\,\,$ \\
$\,\,$\color{red} 0.1130\color{black} $\,\,$ & $\,\,$0.6606$\,\,$ & $\,\,$ 1 $\,\,$ & $\,\,$0.7848 $\,\,$ \\
$\,\,$\color{red} 0.1440\color{black} $\,\,$ & $\,\,$0.8418$\,\,$ & $\,\,$1.2743$\,\,$ & $\,\,$ 1  $\,\,$ \\
\end{pmatrix},
\end{equation*}

\begin{equation*}
\mathbf{w}^{\prime} =
\begin{pmatrix}
0.701919\\
0.119116\\
0.078690\\
0.100274
\end{pmatrix} =
0.994660\cdot
\begin{pmatrix}
\color{gr} 0.705687\color{black} \\
0.119756\\
0.079113\\
0.100812
\end{pmatrix},
\end{equation*}
\begin{equation*}
\left[ \frac{{w}^{\prime}_i}{{w}^{\prime}_j} \right] =
\begin{pmatrix}
$\,\,$ 1 $\,\,$ & $\,\,$\color{gr} 5.8927\color{black} $\,\,$ & $\,\,$\color{gr} 8.9200\color{black} $\,\,$ & $\,\,$\color{gr} \color{blue} 7\color{black} $\,\,$ \\
$\,\,$\color{gr} 0.1697\color{black} $\,\,$ & $\,\,$ 1 $\,\,$ & $\,\,$1.5137$\,\,$ & $\,\,$1.1879  $\,\,$ \\
$\,\,$\color{gr} 0.1121\color{black} $\,\,$ & $\,\,$0.6606$\,\,$ & $\,\,$ 1 $\,\,$ & $\,\,$0.7848 $\,\,$ \\
$\,\,$\color{gr} \color{blue}  1/7\color{black} $\,\,$ & $\,\,$0.8418$\,\,$ & $\,\,$1.2743$\,\,$ & $\,\,$ 1  $\,\,$ \\
\end{pmatrix},
\end{equation*}
\end{example}

\newpage
\section{Inef{\kern0pt}f{\kern0pt}icient cosine weight vector}
\begin{example}
\begin{equation*}
\mathbf{A} =
\begin{pmatrix}
$\,\,$ 1 $\,\,$ & $\,\,$1$\,\,$ & $\,\,$2$\,\,$ & $\,\,$1 $\,\,$ \\
$\,\,$ 1 $\,\,$ & $\,\,$ 1 $\,\,$ & $\,\,$3$\,\,$ & $\,\,$5 $\,\,$ \\
$\,\,$ 1/2$\,\,$ & $\,\,$ 1/3$\,\,$ & $\,\,$ 1 $\,\,$ & $\,\,$1 $\,\,$ \\
$\,\,$ 1 $\,\,$ & $\,\,$ 1/5$\,\,$ & $\,\,$ 1 $\,\,$ & $\,\,$ 1  $\,\,$ \\
\end{pmatrix},
\qquad
\lambda_{\max} =
4.2277,
\qquad
CR = 0.0859
\end{equation*}

\begin{equation*}
\mathbf{w}^{cos} =
\begin{pmatrix}
0.277891\\
0.423016\\
\color{red} 0.136206\color{black} \\
0.162886
\end{pmatrix}\end{equation*}
\begin{equation*}
\left[ \frac{{w}^{cos}_i}{{w}^{cos}_j} \right] =
\begin{pmatrix}
$\,\,$ 1 $\,\,$ & $\,\,$0.6569$\,\,$ & $\,\,$\color{red} 2.0402\color{black} $\,\,$ & $\,\,$1.7060$\,\,$ \\
$\,\,$1.5222$\,\,$ & $\,\,$ 1 $\,\,$ & $\,\,$\color{red} 3.1057\color{black} $\,\,$ & $\,\,$2.5970  $\,\,$ \\
$\,\,$\color{red} 0.4901\color{black} $\,\,$ & $\,\,$\color{red} 0.3220\color{black} $\,\,$ & $\,\,$ 1 $\,\,$ & $\,\,$\color{red} 0.8362\color{black}  $\,\,$ \\
$\,\,$0.5862$\,\,$ & $\,\,$0.3851$\,\,$ & $\,\,$\color{red} 1.1959\color{black} $\,\,$ & $\,\,$ 1  $\,\,$ \\
\end{pmatrix},
\end{equation*}

\begin{equation*}
\mathbf{w}^{\prime} =
\begin{pmatrix}
0.277132\\
0.421860\\
0.138566\\
0.162441
\end{pmatrix} =
0.997268\cdot
\begin{pmatrix}
0.277891\\
0.423016\\
\color{gr} 0.138946\color{black} \\
0.162886
\end{pmatrix},
\end{equation*}
\begin{equation*}
\left[ \frac{{w}^{\prime}_i}{{w}^{\prime}_j} \right] =
\begin{pmatrix}
$\,\,$ 1 $\,\,$ & $\,\,$0.6569$\,\,$ & $\,\,$\color{gr} \color{blue} 2\color{black} $\,\,$ & $\,\,$1.7060$\,\,$ \\
$\,\,$1.5222$\,\,$ & $\,\,$ 1 $\,\,$ & $\,\,$\color{gr} 3.0445\color{black} $\,\,$ & $\,\,$2.5970  $\,\,$ \\
$\,\,$\color{gr} \color{blue}  1/2\color{black} $\,\,$ & $\,\,$\color{gr} 0.3285\color{black} $\,\,$ & $\,\,$ 1 $\,\,$ & $\,\,$\color{gr} 0.8530\color{black}  $\,\,$ \\
$\,\,$0.5862$\,\,$ & $\,\,$0.3851$\,\,$ & $\,\,$\color{gr} 1.1723\color{black} $\,\,$ & $\,\,$ 1  $\,\,$ \\
\end{pmatrix},
\end{equation*}
\end{example}
\newpage
\begin{example}
\begin{equation*}
\mathbf{A} =
\begin{pmatrix}
$\,\,$ 1 $\,\,$ & $\,\,$1$\,\,$ & $\,\,$3$\,\,$ & $\,\,$2 $\,\,$ \\
$\,\,$ 1 $\,\,$ & $\,\,$ 1 $\,\,$ & $\,\,$4$\,\,$ & $\,\,$6 $\,\,$ \\
$\,\,$ 1/3$\,\,$ & $\,\,$ 1/4$\,\,$ & $\,\,$ 1 $\,\,$ & $\,\,$1 $\,\,$ \\
$\,\,$ 1/2$\,\,$ & $\,\,$ 1/6$\,\,$ & $\,\,$ 1 $\,\,$ & $\,\,$ 1  $\,\,$ \\
\end{pmatrix},
\qquad
\lambda_{\max} =
4.1031,
\qquad
CR = 0.0389
\end{equation*}

\begin{equation*}
\mathbf{w}^{cos} =
\begin{pmatrix}
0.328166\\
0.447400\\
\color{red} 0.108475\color{black} \\
0.115960
\end{pmatrix}\end{equation*}
\begin{equation*}
\left[ \frac{{w}^{cos}_i}{{w}^{cos}_j} \right] =
\begin{pmatrix}
$\,\,$ 1 $\,\,$ & $\,\,$0.7335$\,\,$ & $\,\,$\color{red} 3.0253\color{black} $\,\,$ & $\,\,$2.8300$\,\,$ \\
$\,\,$1.3633$\,\,$ & $\,\,$ 1 $\,\,$ & $\,\,$\color{red} 4.1245\color{black} $\,\,$ & $\,\,$3.8582  $\,\,$ \\
$\,\,$\color{red} 0.3305\color{black} $\,\,$ & $\,\,$\color{red} 0.2425\color{black} $\,\,$ & $\,\,$ 1 $\,\,$ & $\,\,$\color{red} 0.9355\color{black}  $\,\,$ \\
$\,\,$0.3534$\,\,$ & $\,\,$0.2592$\,\,$ & $\,\,$\color{red} 1.0690\color{black} $\,\,$ & $\,\,$ 1  $\,\,$ \\
\end{pmatrix},
\end{equation*}

\begin{equation*}
\mathbf{w}^{\prime} =
\begin{pmatrix}
0.327866\\
0.446991\\
0.109289\\
0.115854
\end{pmatrix} =
0.999087\cdot
\begin{pmatrix}
0.328166\\
0.447400\\
\color{gr} 0.109389\color{black} \\
0.115960
\end{pmatrix},
\end{equation*}
\begin{equation*}
\left[ \frac{{w}^{\prime}_i}{{w}^{\prime}_j} \right] =
\begin{pmatrix}
$\,\,$ 1 $\,\,$ & $\,\,$0.7335$\,\,$ & $\,\,$\color{gr} \color{blue} 3\color{black} $\,\,$ & $\,\,$2.8300$\,\,$ \\
$\,\,$1.3633$\,\,$ & $\,\,$ 1 $\,\,$ & $\,\,$\color{gr} 4.0900\color{black} $\,\,$ & $\,\,$3.8582  $\,\,$ \\
$\,\,$\color{gr} \color{blue}  1/3\color{black} $\,\,$ & $\,\,$\color{gr} 0.2445\color{black} $\,\,$ & $\,\,$ 1 $\,\,$ & $\,\,$\color{gr} 0.9433\color{black}  $\,\,$ \\
$\,\,$0.3534$\,\,$ & $\,\,$0.2592$\,\,$ & $\,\,$\color{gr} 1.0601\color{black} $\,\,$ & $\,\,$ 1  $\,\,$ \\
\end{pmatrix},
\end{equation*}
\end{example}
\newpage
\begin{example}
\begin{equation*}
\mathbf{A} =
\begin{pmatrix}
$\,\,$ 1 $\,\,$ & $\,\,$1$\,\,$ & $\,\,$3$\,\,$ & $\,\,$2 $\,\,$ \\
$\,\,$ 1 $\,\,$ & $\,\,$ 1 $\,\,$ & $\,\,$4$\,\,$ & $\,\,$7 $\,\,$ \\
$\,\,$ 1/3$\,\,$ & $\,\,$ 1/4$\,\,$ & $\,\,$ 1 $\,\,$ & $\,\,$1 $\,\,$ \\
$\,\,$ 1/2$\,\,$ & $\,\,$ 1/7$\,\,$ & $\,\,$ 1 $\,\,$ & $\,\,$ 1  $\,\,$ \\
\end{pmatrix},
\qquad
\lambda_{\max} =
4.1365,
\qquad
CR = 0.0515
\end{equation*}

\begin{equation*}
\mathbf{w}^{cos} =
\begin{pmatrix}
0.326135\\
0.455155\\
\color{red} 0.106807\color{black} \\
0.111902
\end{pmatrix}\end{equation*}
\begin{equation*}
\left[ \frac{{w}^{cos}_i}{{w}^{cos}_j} \right] =
\begin{pmatrix}
$\,\,$ 1 $\,\,$ & $\,\,$0.7165$\,\,$ & $\,\,$\color{red} 3.0535\color{black} $\,\,$ & $\,\,$2.9145$\,\,$ \\
$\,\,$1.3956$\,\,$ & $\,\,$ 1 $\,\,$ & $\,\,$\color{red} 4.2615\color{black} $\,\,$ & $\,\,$4.0674  $\,\,$ \\
$\,\,$\color{red} 0.3275\color{black} $\,\,$ & $\,\,$\color{red} 0.2347\color{black} $\,\,$ & $\,\,$ 1 $\,\,$ & $\,\,$\color{red} 0.9545\color{black}  $\,\,$ \\
$\,\,$0.3431$\,\,$ & $\,\,$0.2459$\,\,$ & $\,\,$\color{red} 1.0477\color{black} $\,\,$ & $\,\,$ 1  $\,\,$ \\
\end{pmatrix},
\end{equation*}

\begin{equation*}
\mathbf{w}^{\prime} =
\begin{pmatrix}
0.325515\\
0.454290\\
0.108505\\
0.111690
\end{pmatrix} =
0.998099\cdot
\begin{pmatrix}
0.326135\\
0.455155\\
\color{gr} 0.108712\color{black} \\
0.111902
\end{pmatrix},
\end{equation*}
\begin{equation*}
\left[ \frac{{w}^{\prime}_i}{{w}^{\prime}_j} \right] =
\begin{pmatrix}
$\,\,$ 1 $\,\,$ & $\,\,$0.7165$\,\,$ & $\,\,$\color{gr} \color{blue} 3\color{black} $\,\,$ & $\,\,$2.9145$\,\,$ \\
$\,\,$1.3956$\,\,$ & $\,\,$ 1 $\,\,$ & $\,\,$\color{gr} 4.1868\color{black} $\,\,$ & $\,\,$4.0674  $\,\,$ \\
$\,\,$\color{gr} \color{blue}  1/3\color{black} $\,\,$ & $\,\,$\color{gr} 0.2388\color{black} $\,\,$ & $\,\,$ 1 $\,\,$ & $\,\,$\color{gr} 0.9715\color{black}  $\,\,$ \\
$\,\,$0.3431$\,\,$ & $\,\,$0.2459$\,\,$ & $\,\,$\color{gr} 1.0294\color{black} $\,\,$ & $\,\,$ 1  $\,\,$ \\
\end{pmatrix},
\end{equation*}
\end{example}
\newpage
\begin{example}
\begin{equation*}
\mathbf{A} =
\begin{pmatrix}
$\,\,$ 1 $\,\,$ & $\,\,$1$\,\,$ & $\,\,$3$\,\,$ & $\,\,$2 $\,\,$ \\
$\,\,$ 1 $\,\,$ & $\,\,$ 1 $\,\,$ & $\,\,$4$\,\,$ & $\,\,$8 $\,\,$ \\
$\,\,$ 1/3$\,\,$ & $\,\,$ 1/4$\,\,$ & $\,\,$ 1 $\,\,$ & $\,\,$1 $\,\,$ \\
$\,\,$ 1/2$\,\,$ & $\,\,$ 1/8$\,\,$ & $\,\,$ 1 $\,\,$ & $\,\,$ 1  $\,\,$ \\
\end{pmatrix},
\qquad
\lambda_{\max} =
4.1707,
\qquad
CR = 0.0644
\end{equation*}

\begin{equation*}
\mathbf{w}^{cos} =
\begin{pmatrix}
0.324566\\
0.461187\\
\color{red} 0.105490\color{black} \\
0.108758
\end{pmatrix}\end{equation*}
\begin{equation*}
\left[ \frac{{w}^{cos}_i}{{w}^{cos}_j} \right] =
\begin{pmatrix}
$\,\,$ 1 $\,\,$ & $\,\,$0.7038$\,\,$ & $\,\,$\color{red} 3.0767\color{black} $\,\,$ & $\,\,$2.9843$\,\,$ \\
$\,\,$1.4209$\,\,$ & $\,\,$ 1 $\,\,$ & $\,\,$\color{red} 4.3719\color{black} $\,\,$ & $\,\,$4.2405  $\,\,$ \\
$\,\,$\color{red} 0.3250\color{black} $\,\,$ & $\,\,$\color{red} 0.2287\color{black} $\,\,$ & $\,\,$ 1 $\,\,$ & $\,\,$\color{red} 0.9700\color{black}  $\,\,$ \\
$\,\,$0.3351$\,\,$ & $\,\,$0.2358$\,\,$ & $\,\,$\color{red} 1.0310\color{black} $\,\,$ & $\,\,$ 1  $\,\,$ \\
\end{pmatrix},
\end{equation*}

\begin{equation*}
\mathbf{w}^{\prime} =
\begin{pmatrix}
0.323692\\
0.459946\\
0.107897\\
0.108465
\end{pmatrix} =
0.997309\cdot
\begin{pmatrix}
0.324566\\
0.461187\\
\color{gr} 0.108189\color{black} \\
0.108758
\end{pmatrix},
\end{equation*}
\begin{equation*}
\left[ \frac{{w}^{\prime}_i}{{w}^{\prime}_j} \right] =
\begin{pmatrix}
$\,\,$ 1 $\,\,$ & $\,\,$0.7038$\,\,$ & $\,\,$\color{gr} \color{blue} 3\color{black} $\,\,$ & $\,\,$2.9843$\,\,$ \\
$\,\,$1.4209$\,\,$ & $\,\,$ 1 $\,\,$ & $\,\,$\color{gr} 4.2628\color{black} $\,\,$ & $\,\,$4.2405  $\,\,$ \\
$\,\,$\color{gr} \color{blue}  1/3\color{black} $\,\,$ & $\,\,$\color{gr} 0.2346\color{black} $\,\,$ & $\,\,$ 1 $\,\,$ & $\,\,$\color{gr} 0.9948\color{black}  $\,\,$ \\
$\,\,$0.3351$\,\,$ & $\,\,$0.2358$\,\,$ & $\,\,$\color{gr} 1.0053\color{black} $\,\,$ & $\,\,$ 1  $\,\,$ \\
\end{pmatrix},
\end{equation*}
\end{example}
\newpage
\begin{example}
\begin{equation*}
\mathbf{A} =
\begin{pmatrix}
$\,\,$ 1 $\,\,$ & $\,\,$1$\,\,$ & $\,\,$3$\,\,$ & $\,\,$2 $\,\,$ \\
$\,\,$ 1 $\,\,$ & $\,\,$ 1 $\,\,$ & $\,\,$4$\,\,$ & $\,\,$9 $\,\,$ \\
$\,\,$ 1/3$\,\,$ & $\,\,$ 1/4$\,\,$ & $\,\,$ 1 $\,\,$ & $\,\,$1 $\,\,$ \\
$\,\,$ 1/2$\,\,$ & $\,\,$ 1/9$\,\,$ & $\,\,$ 1 $\,\,$ & $\,\,$ 1  $\,\,$ \\
\end{pmatrix},
\qquad
\lambda_{\max} =
4.2052,
\qquad
CR = 0.0774
\end{equation*}

\begin{equation*}
\mathbf{w}^{cos} =
\begin{pmatrix}
0.323323\\
0.465999\\
\color{red} 0.104428\color{black} \\
0.106250
\end{pmatrix}\end{equation*}
\begin{equation*}
\left[ \frac{{w}^{cos}_i}{{w}^{cos}_j} \right] =
\begin{pmatrix}
$\,\,$ 1 $\,\,$ & $\,\,$0.6938$\,\,$ & $\,\,$\color{red} 3.0961\color{black} $\,\,$ & $\,\,$3.0430$\,\,$ \\
$\,\,$1.4413$\,\,$ & $\,\,$ 1 $\,\,$ & $\,\,$\color{red} 4.4624\color{black} $\,\,$ & $\,\,$4.3859  $\,\,$ \\
$\,\,$\color{red} 0.3230\color{black} $\,\,$ & $\,\,$\color{red} 0.2241\color{black} $\,\,$ & $\,\,$ 1 $\,\,$ & $\,\,$\color{red} 0.9828\color{black}  $\,\,$ \\
$\,\,$0.3286$\,\,$ & $\,\,$0.2280$\,\,$ & $\,\,$\color{red} 1.0175\color{black} $\,\,$ & $\,\,$ 1  $\,\,$ \\
\end{pmatrix},
\end{equation*}

\begin{equation*}
\mathbf{w}^{\prime} =
\begin{pmatrix}
0.322735\\
0.465151\\
0.106057\\
0.106057
\end{pmatrix} =
0.998181\cdot
\begin{pmatrix}
0.323323\\
0.465999\\
\color{gr} 0.106250\color{black} \\
0.106250
\end{pmatrix},
\end{equation*}
\begin{equation*}
\left[ \frac{{w}^{\prime}_i}{{w}^{\prime}_j} \right] =
\begin{pmatrix}
$\,\,$ 1 $\,\,$ & $\,\,$0.6938$\,\,$ & $\,\,$\color{gr} 3.0430\color{black} $\,\,$ & $\,\,$3.0430$\,\,$ \\
$\,\,$1.4413$\,\,$ & $\,\,$ 1 $\,\,$ & $\,\,$\color{gr} 4.3859\color{black} $\,\,$ & $\,\,$4.3859  $\,\,$ \\
$\,\,$\color{gr} 0.3286\color{black} $\,\,$ & $\,\,$\color{gr} 0.2280\color{black} $\,\,$ & $\,\,$ 1 $\,\,$ & $\,\,$\color{gr} \color{blue} 1\color{black}  $\,\,$ \\
$\,\,$0.3286$\,\,$ & $\,\,$0.2280$\,\,$ & $\,\,$\color{gr} \color{blue} 1\color{black} $\,\,$ & $\,\,$ 1  $\,\,$ \\
\end{pmatrix},
\end{equation*}
\end{example}
\newpage
\begin{example}
\begin{equation*}
\mathbf{A} =
\begin{pmatrix}
$\,\,$ 1 $\,\,$ & $\,\,$1$\,\,$ & $\,\,$4$\,\,$ & $\,\,$3 $\,\,$ \\
$\,\,$ 1 $\,\,$ & $\,\,$ 1 $\,\,$ & $\,\,$5$\,\,$ & $\,\,$7 $\,\,$ \\
$\,\,$ 1/4$\,\,$ & $\,\,$ 1/5$\,\,$ & $\,\,$ 1 $\,\,$ & $\,\,$1 $\,\,$ \\
$\,\,$ 1/3$\,\,$ & $\,\,$ 1/7$\,\,$ & $\,\,$ 1 $\,\,$ & $\,\,$ 1  $\,\,$ \\
\end{pmatrix},
\qquad
\lambda_{\max} =
4.0609,
\qquad
CR = 0.0230
\end{equation*}

\begin{equation*}
\mathbf{w}^{cos} =
\begin{pmatrix}
0.358827\\
0.460005\\
\color{red} 0.089320\color{black} \\
0.091848
\end{pmatrix}\end{equation*}
\begin{equation*}
\left[ \frac{{w}^{cos}_i}{{w}^{cos}_j} \right] =
\begin{pmatrix}
$\,\,$ 1 $\,\,$ & $\,\,$0.7801$\,\,$ & $\,\,$\color{red} 4.0173\color{black} $\,\,$ & $\,\,$3.9068$\,\,$ \\
$\,\,$1.2820$\,\,$ & $\,\,$ 1 $\,\,$ & $\,\,$\color{red} 5.1501\color{black} $\,\,$ & $\,\,$5.0083  $\,\,$ \\
$\,\,$\color{red} 0.2489\color{black} $\,\,$ & $\,\,$\color{red} 0.1942\color{black} $\,\,$ & $\,\,$ 1 $\,\,$ & $\,\,$\color{red} 0.9725\color{black}  $\,\,$ \\
$\,\,$0.2560$\,\,$ & $\,\,$0.1997$\,\,$ & $\,\,$\color{red} 1.0283\color{black} $\,\,$ & $\,\,$ 1  $\,\,$ \\
\end{pmatrix},
\end{equation*}

\begin{equation*}
\mathbf{w}^{\prime} =
\begin{pmatrix}
0.358688\\
0.459827\\
0.089672\\
0.091812
\end{pmatrix} =
0.999613\cdot
\begin{pmatrix}
0.358827\\
0.460005\\
\color{gr} 0.089707\color{black} \\
0.091848
\end{pmatrix},
\end{equation*}
\begin{equation*}
\left[ \frac{{w}^{\prime}_i}{{w}^{\prime}_j} \right] =
\begin{pmatrix}
$\,\,$ 1 $\,\,$ & $\,\,$0.7801$\,\,$ & $\,\,$\color{gr} \color{blue} 4\color{black} $\,\,$ & $\,\,$3.9068$\,\,$ \\
$\,\,$1.2820$\,\,$ & $\,\,$ 1 $\,\,$ & $\,\,$\color{gr} 5.1279\color{black} $\,\,$ & $\,\,$5.0083  $\,\,$ \\
$\,\,$\color{gr} \color{blue}  1/4\color{black} $\,\,$ & $\,\,$\color{gr} 0.1950\color{black} $\,\,$ & $\,\,$ 1 $\,\,$ & $\,\,$\color{gr} 0.9767\color{black}  $\,\,$ \\
$\,\,$0.2560$\,\,$ & $\,\,$0.1997$\,\,$ & $\,\,$\color{gr} 1.0239\color{black} $\,\,$ & $\,\,$ 1  $\,\,$ \\
\end{pmatrix},
\end{equation*}
\end{example}
\newpage
\begin{example}
\begin{equation*}
\mathbf{A} =
\begin{pmatrix}
$\,\,$ 1 $\,\,$ & $\,\,$1$\,\,$ & $\,\,$4$\,\,$ & $\,\,$3 $\,\,$ \\
$\,\,$ 1 $\,\,$ & $\,\,$ 1 $\,\,$ & $\,\,$5$\,\,$ & $\,\,$8 $\,\,$ \\
$\,\,$ 1/4$\,\,$ & $\,\,$ 1/5$\,\,$ & $\,\,$ 1 $\,\,$ & $\,\,$1 $\,\,$ \\
$\,\,$ 1/3$\,\,$ & $\,\,$ 1/8$\,\,$ & $\,\,$ 1 $\,\,$ & $\,\,$ 1  $\,\,$ \\
\end{pmatrix},
\qquad
\lambda_{\max} =
4.0835,
\qquad
CR = 0.0315
\end{equation*}

\begin{equation*}
\mathbf{w}^{cos} =
\begin{pmatrix}
0.355922\\
0.467308\\
\color{red} 0.088064\color{black} \\
0.088707
\end{pmatrix}\end{equation*}
\begin{equation*}
\left[ \frac{{w}^{cos}_i}{{w}^{cos}_j} \right] =
\begin{pmatrix}
$\,\,$ 1 $\,\,$ & $\,\,$0.7616$\,\,$ & $\,\,$\color{red} 4.0416\color{black} $\,\,$ & $\,\,$4.0123$\,\,$ \\
$\,\,$1.3129$\,\,$ & $\,\,$ 1 $\,\,$ & $\,\,$\color{red} 5.3065\color{black} $\,\,$ & $\,\,$5.2680  $\,\,$ \\
$\,\,$\color{red} 0.2474\color{black} $\,\,$ & $\,\,$\color{red} 0.1884\color{black} $\,\,$ & $\,\,$ 1 $\,\,$ & $\,\,$\color{red} 0.9928\color{black}  $\,\,$ \\
$\,\,$0.2492$\,\,$ & $\,\,$0.1898$\,\,$ & $\,\,$\color{red} 1.0073\color{black} $\,\,$ & $\,\,$ 1  $\,\,$ \\
\end{pmatrix},
\end{equation*}

\begin{equation*}
\mathbf{w}^{\prime} =
\begin{pmatrix}
0.355693\\
0.467007\\
0.088650\\
0.088650
\end{pmatrix} =
0.999357\cdot
\begin{pmatrix}
0.355922\\
0.467308\\
\color{gr} 0.088707\color{black} \\
0.088707
\end{pmatrix},
\end{equation*}
\begin{equation*}
\left[ \frac{{w}^{\prime}_i}{{w}^{\prime}_j} \right] =
\begin{pmatrix}
$\,\,$ 1 $\,\,$ & $\,\,$0.7616$\,\,$ & $\,\,$\color{gr} 4.0123\color{black} $\,\,$ & $\,\,$4.0123$\,\,$ \\
$\,\,$1.3129$\,\,$ & $\,\,$ 1 $\,\,$ & $\,\,$\color{gr} 5.2680\color{black} $\,\,$ & $\,\,$5.2680  $\,\,$ \\
$\,\,$\color{gr} 0.2492\color{black} $\,\,$ & $\,\,$\color{gr} 0.1898\color{black} $\,\,$ & $\,\,$ 1 $\,\,$ & $\,\,$\color{gr} \color{blue} 1\color{black}  $\,\,$ \\
$\,\,$0.2492$\,\,$ & $\,\,$0.1898$\,\,$ & $\,\,$\color{gr} \color{blue} 1\color{black} $\,\,$ & $\,\,$ 1  $\,\,$ \\
\end{pmatrix},
\end{equation*}
\end{example}
\newpage
\begin{example}
\begin{equation*}
\mathbf{A} =
\begin{pmatrix}
$\,\,$ 1 $\,\,$ & $\,\,$1$\,\,$ & $\,\,$5$\,\,$ & $\,\,$2 $\,\,$ \\
$\,\,$ 1 $\,\,$ & $\,\,$ 1 $\,\,$ & $\,\,$3$\,\,$ & $\,\,$3 $\,\,$ \\
$\,\,$ 1/5$\,\,$ & $\,\,$ 1/3$\,\,$ & $\,\,$ 1 $\,\,$ & $\,\,$2 $\,\,$ \\
$\,\,$ 1/2$\,\,$ & $\,\,$ 1/3$\,\,$ & $\,\,$ 1/2$\,\,$ & $\,\,$ 1  $\,\,$ \\
\end{pmatrix},
\qquad
\lambda_{\max} =
4.2277,
\qquad
CR = 0.0859
\end{equation*}

\begin{equation*}
\mathbf{w}^{cos} =
\begin{pmatrix}
0.374739\\
\color{red} 0.360408\color{black} \\
0.141045\\
0.123808
\end{pmatrix}\end{equation*}
\begin{equation*}
\left[ \frac{{w}^{cos}_i}{{w}^{cos}_j} \right] =
\begin{pmatrix}
$\,\,$ 1 $\,\,$ & $\,\,$\color{red} 1.0398\color{black} $\,\,$ & $\,\,$2.6569$\,\,$ & $\,\,$3.0268$\,\,$ \\
$\,\,$\color{red} 0.9618\color{black} $\,\,$ & $\,\,$ 1 $\,\,$ & $\,\,$\color{red} 2.5553\color{black} $\,\,$ & $\,\,$\color{red} 2.9110\color{black}   $\,\,$ \\
$\,\,$0.3764$\,\,$ & $\,\,$\color{red} 0.3913\color{black} $\,\,$ & $\,\,$ 1 $\,\,$ & $\,\,$1.1392 $\,\,$ \\
$\,\,$0.3304$\,\,$ & $\,\,$\color{red} 0.3435\color{black} $\,\,$ & $\,\,$0.8778$\,\,$ & $\,\,$ 1  $\,\,$ \\
\end{pmatrix},
\end{equation*}

\begin{equation*}
\mathbf{w}^{\prime} =
\begin{pmatrix}
0.370655\\
0.367377\\
0.139508\\
0.122459
\end{pmatrix} =
0.989103\cdot
\begin{pmatrix}
0.374739\\
\color{gr} 0.371425\color{black} \\
0.141045\\
0.123808
\end{pmatrix},
\end{equation*}
\begin{equation*}
\left[ \frac{{w}^{\prime}_i}{{w}^{\prime}_j} \right] =
\begin{pmatrix}
$\,\,$ 1 $\,\,$ & $\,\,$\color{gr} 1.0089\color{black} $\,\,$ & $\,\,$2.6569$\,\,$ & $\,\,$3.0268$\,\,$ \\
$\,\,$\color{gr} 0.9912\color{black} $\,\,$ & $\,\,$ 1 $\,\,$ & $\,\,$\color{gr} 2.6334\color{black} $\,\,$ & $\,\,$\color{gr} \color{blue} 3\color{black}   $\,\,$ \\
$\,\,$0.3764$\,\,$ & $\,\,$\color{gr} 0.3797\color{black} $\,\,$ & $\,\,$ 1 $\,\,$ & $\,\,$1.1392 $\,\,$ \\
$\,\,$0.3304$\,\,$ & $\,\,$\color{gr} \color{blue}  1/3\color{black} $\,\,$ & $\,\,$0.8778$\,\,$ & $\,\,$ 1  $\,\,$ \\
\end{pmatrix},
\end{equation*}
\end{example}
\newpage
\begin{example}
\begin{equation*}
\mathbf{A} =
\begin{pmatrix}
$\,\,$ 1 $\,\,$ & $\,\,$1$\,\,$ & $\,\,$5$\,\,$ & $\,\,$4 $\,\,$ \\
$\,\,$ 1 $\,\,$ & $\,\,$ 1 $\,\,$ & $\,\,$6$\,\,$ & $\,\,$8 $\,\,$ \\
$\,\,$ 1/5$\,\,$ & $\,\,$ 1/6$\,\,$ & $\,\,$ 1 $\,\,$ & $\,\,$1 $\,\,$ \\
$\,\,$ 1/4$\,\,$ & $\,\,$ 1/8$\,\,$ & $\,\,$ 1 $\,\,$ & $\,\,$ 1  $\,\,$ \\
\end{pmatrix},
\qquad
\lambda_{\max} =
4.0407,
\qquad
CR = 0.0153
\end{equation*}

\begin{equation*}
\mathbf{w}^{cos} =
\begin{pmatrix}
0.379959\\
0.467614\\
\color{red} 0.075802\color{black} \\
0.076625
\end{pmatrix}\end{equation*}
\begin{equation*}
\left[ \frac{{w}^{cos}_i}{{w}^{cos}_j} \right] =
\begin{pmatrix}
$\,\,$ 1 $\,\,$ & $\,\,$0.8125$\,\,$ & $\,\,$\color{red} 5.0125\color{black} $\,\,$ & $\,\,$4.9587$\,\,$ \\
$\,\,$1.2307$\,\,$ & $\,\,$ 1 $\,\,$ & $\,\,$\color{red} 6.1689\color{black} $\,\,$ & $\,\,$6.1026  $\,\,$ \\
$\,\,$\color{red} 0.1995\color{black} $\,\,$ & $\,\,$\color{red} 0.1621\color{black} $\,\,$ & $\,\,$ 1 $\,\,$ & $\,\,$\color{red} 0.9893\color{black}  $\,\,$ \\
$\,\,$0.2017$\,\,$ & $\,\,$0.1639$\,\,$ & $\,\,$\color{red} 1.0109\color{black} $\,\,$ & $\,\,$ 1  $\,\,$ \\
\end{pmatrix},
\end{equation*}

\begin{equation*}
\mathbf{w}^{\prime} =
\begin{pmatrix}
0.379887\\
0.467525\\
0.075977\\
0.076611
\end{pmatrix} =
0.999810\cdot
\begin{pmatrix}
0.379959\\
0.467614\\
\color{gr} 0.075992\color{black} \\
0.076625
\end{pmatrix},
\end{equation*}
\begin{equation*}
\left[ \frac{{w}^{\prime}_i}{{w}^{\prime}_j} \right] =
\begin{pmatrix}
$\,\,$ 1 $\,\,$ & $\,\,$0.8125$\,\,$ & $\,\,$\color{gr} \color{blue} 5\color{black} $\,\,$ & $\,\,$4.9587$\,\,$ \\
$\,\,$1.2307$\,\,$ & $\,\,$ 1 $\,\,$ & $\,\,$\color{gr} 6.1535\color{black} $\,\,$ & $\,\,$6.1026  $\,\,$ \\
$\,\,$\color{gr} \color{blue}  1/5\color{black} $\,\,$ & $\,\,$\color{gr} 0.1625\color{black} $\,\,$ & $\,\,$ 1 $\,\,$ & $\,\,$\color{gr} 0.9917\color{black}  $\,\,$ \\
$\,\,$0.2017$\,\,$ & $\,\,$0.1639$\,\,$ & $\,\,$\color{gr} 1.0083\color{black} $\,\,$ & $\,\,$ 1  $\,\,$ \\
\end{pmatrix},
\end{equation*}
\end{example}
\newpage
\begin{example}
\begin{equation*}
\mathbf{A} =
\begin{pmatrix}
$\,\,$ 1 $\,\,$ & $\,\,$1$\,\,$ & $\,\,$6$\,\,$ & $\,\,$2 $\,\,$ \\
$\,\,$ 1 $\,\,$ & $\,\,$ 1 $\,\,$ & $\,\,$4$\,\,$ & $\,\,$3 $\,\,$ \\
$\,\,$ 1/6$\,\,$ & $\,\,$ 1/4$\,\,$ & $\,\,$ 1 $\,\,$ & $\,\,$1 $\,\,$ \\
$\,\,$ 1/2$\,\,$ & $\,\,$ 1/3$\,\,$ & $\,\,$ 1 $\,\,$ & $\,\,$ 1  $\,\,$ \\
\end{pmatrix},
\qquad
\lambda_{\max} =
4.1031,
\qquad
CR = 0.0389
\end{equation*}

\begin{equation*}
\mathbf{w}^{cos} =
\begin{pmatrix}
0.384188\\
\color{red} 0.382181\color{black} \\
0.096887\\
0.136744
\end{pmatrix}\end{equation*}
\begin{equation*}
\left[ \frac{{w}^{cos}_i}{{w}^{cos}_j} \right] =
\begin{pmatrix}
$\,\,$ 1 $\,\,$ & $\,\,$\color{red} 1.0053\color{black} $\,\,$ & $\,\,$3.9653$\,\,$ & $\,\,$2.8096$\,\,$ \\
$\,\,$\color{red} 0.9948\color{black} $\,\,$ & $\,\,$ 1 $\,\,$ & $\,\,$\color{red} 3.9446\color{black} $\,\,$ & $\,\,$\color{red} 2.7949\color{black}   $\,\,$ \\
$\,\,$0.2522$\,\,$ & $\,\,$\color{red} 0.2535\color{black} $\,\,$ & $\,\,$ 1 $\,\,$ & $\,\,$0.7085 $\,\,$ \\
$\,\,$0.3559$\,\,$ & $\,\,$\color{red} 0.3578\color{black} $\,\,$ & $\,\,$1.4114$\,\,$ & $\,\,$ 1  $\,\,$ \\
\end{pmatrix},
\end{equation*}

\begin{equation*}
\mathbf{w}^{\prime} =
\begin{pmatrix}
0.383419\\
0.383419\\
0.096692\\
0.136470
\end{pmatrix} =
0.997997\cdot
\begin{pmatrix}
0.384188\\
\color{gr} 0.384188\color{black} \\
0.096887\\
0.136744
\end{pmatrix},
\end{equation*}
\begin{equation*}
\left[ \frac{{w}^{\prime}_i}{{w}^{\prime}_j} \right] =
\begin{pmatrix}
$\,\,$ 1 $\,\,$ & $\,\,$\color{gr} \color{blue} 1\color{black} $\,\,$ & $\,\,$3.9653$\,\,$ & $\,\,$2.8096$\,\,$ \\
$\,\,$\color{gr} \color{blue} 1\color{black} $\,\,$ & $\,\,$ 1 $\,\,$ & $\,\,$\color{gr} 3.9653\color{black} $\,\,$ & $\,\,$\color{gr} 2.8096\color{black}   $\,\,$ \\
$\,\,$0.2522$\,\,$ & $\,\,$\color{gr} 0.2522\color{black} $\,\,$ & $\,\,$ 1 $\,\,$ & $\,\,$0.7085 $\,\,$ \\
$\,\,$0.3559$\,\,$ & $\,\,$\color{gr} 0.3559\color{black} $\,\,$ & $\,\,$1.4114$\,\,$ & $\,\,$ 1  $\,\,$ \\
\end{pmatrix},
\end{equation*}
\end{example}
\newpage
\begin{example}
\begin{equation*}
\mathbf{A} =
\begin{pmatrix}
$\,\,$ 1 $\,\,$ & $\,\,$1$\,\,$ & $\,\,$6$\,\,$ & $\,\,$5 $\,\,$ \\
$\,\,$ 1 $\,\,$ & $\,\,$ 1 $\,\,$ & $\,\,$7$\,\,$ & $\,\,$9 $\,\,$ \\
$\,\,$ 1/6$\,\,$ & $\,\,$ 1/7$\,\,$ & $\,\,$ 1 $\,\,$ & $\,\,$1 $\,\,$ \\
$\,\,$ 1/5$\,\,$ & $\,\,$ 1/9$\,\,$ & $\,\,$ 1 $\,\,$ & $\,\,$ 1  $\,\,$ \\
\end{pmatrix},
\qquad
\lambda_{\max} =
4.0293,
\qquad
CR = 0.0110
\end{equation*}

\begin{equation*}
\mathbf{w}^{cos} =
\begin{pmatrix}
0.395513\\
0.472713\\
\color{red} 0.065815\color{black} \\
0.065958
\end{pmatrix}\end{equation*}
\begin{equation*}
\left[ \frac{{w}^{cos}_i}{{w}^{cos}_j} \right] =
\begin{pmatrix}
$\,\,$ 1 $\,\,$ & $\,\,$0.8367$\,\,$ & $\,\,$\color{red} 6.0095\color{black} $\,\,$ & $\,\,$5.9964$\,\,$ \\
$\,\,$1.1952$\,\,$ & $\,\,$ 1 $\,\,$ & $\,\,$\color{red} 7.1825\color{black} $\,\,$ & $\,\,$7.1668  $\,\,$ \\
$\,\,$\color{red} 0.1664\color{black} $\,\,$ & $\,\,$\color{red} 0.1392\color{black} $\,\,$ & $\,\,$ 1 $\,\,$ & $\,\,$\color{red} 0.9978\color{black}  $\,\,$ \\
$\,\,$0.1668$\,\,$ & $\,\,$0.1395$\,\,$ & $\,\,$\color{red} 1.0022\color{black} $\,\,$ & $\,\,$ 1  $\,\,$ \\
\end{pmatrix},
\end{equation*}

\begin{equation*}
\mathbf{w}^{\prime} =
\begin{pmatrix}
0.395472\\
0.472664\\
0.065912\\
0.065952
\end{pmatrix} =
0.999896\cdot
\begin{pmatrix}
0.395513\\
0.472713\\
\color{gr} 0.065919\color{black} \\
0.065958
\end{pmatrix},
\end{equation*}
\begin{equation*}
\left[ \frac{{w}^{\prime}_i}{{w}^{\prime}_j} \right] =
\begin{pmatrix}
$\,\,$ 1 $\,\,$ & $\,\,$0.8367$\,\,$ & $\,\,$\color{gr} \color{blue} 6\color{black} $\,\,$ & $\,\,$5.9964$\,\,$ \\
$\,\,$1.1952$\,\,$ & $\,\,$ 1 $\,\,$ & $\,\,$\color{gr} 7.1711\color{black} $\,\,$ & $\,\,$7.1668  $\,\,$ \\
$\,\,$\color{gr} \color{blue}  1/6\color{black} $\,\,$ & $\,\,$\color{gr} 0.1394\color{black} $\,\,$ & $\,\,$ 1 $\,\,$ & $\,\,$\color{gr} 0.9994\color{black}  $\,\,$ \\
$\,\,$0.1668$\,\,$ & $\,\,$0.1395$\,\,$ & $\,\,$\color{gr} 1.0006\color{black} $\,\,$ & $\,\,$ 1  $\,\,$ \\
\end{pmatrix},
\end{equation*}
\end{example}
\newpage
\begin{example}
\begin{equation*}
\mathbf{A} =
\begin{pmatrix}
$\,\,$ 1 $\,\,$ & $\,\,$1$\,\,$ & $\,\,$7$\,\,$ & $\,\,$3 $\,\,$ \\
$\,\,$ 1 $\,\,$ & $\,\,$ 1 $\,\,$ & $\,\,$4$\,\,$ & $\,\,$5 $\,\,$ \\
$\,\,$ 1/7$\,\,$ & $\,\,$ 1/4$\,\,$ & $\,\,$ 1 $\,\,$ & $\,\,$2 $\,\,$ \\
$\,\,$ 1/3$\,\,$ & $\,\,$ 1/5$\,\,$ & $\,\,$ 1/2$\,\,$ & $\,\,$ 1  $\,\,$ \\
\end{pmatrix},
\qquad
\lambda_{\max} =
4.2057,
\qquad
CR = 0.0776
\end{equation*}

\begin{equation*}
\mathbf{w}^{cos} =
\begin{pmatrix}
0.406263\\
\color{red} 0.399032\color{black} \\
0.106748\\
0.087957
\end{pmatrix}\end{equation*}
\begin{equation*}
\left[ \frac{{w}^{cos}_i}{{w}^{cos}_j} \right] =
\begin{pmatrix}
$\,\,$ 1 $\,\,$ & $\,\,$\color{red} 1.0181\color{black} $\,\,$ & $\,\,$3.8058$\,\,$ & $\,\,$4.6189$\,\,$ \\
$\,\,$\color{red} 0.9822\color{black} $\,\,$ & $\,\,$ 1 $\,\,$ & $\,\,$\color{red} 3.7381\color{black} $\,\,$ & $\,\,$\color{red} 4.5367\color{black}   $\,\,$ \\
$\,\,$0.2628$\,\,$ & $\,\,$\color{red} 0.2675\color{black} $\,\,$ & $\,\,$ 1 $\,\,$ & $\,\,$1.2136 $\,\,$ \\
$\,\,$0.2165$\,\,$ & $\,\,$\color{red} 0.2204\color{black} $\,\,$ & $\,\,$0.8240$\,\,$ & $\,\,$ 1  $\,\,$ \\
\end{pmatrix},
\end{equation*}

\begin{equation*}
\mathbf{w}^{\prime} =
\begin{pmatrix}
0.403347\\
0.403347\\
0.105981\\
0.087326
\end{pmatrix} =
0.992821\cdot
\begin{pmatrix}
0.406263\\
\color{gr} 0.406263\color{black} \\
0.106748\\
0.087957
\end{pmatrix},
\end{equation*}
\begin{equation*}
\left[ \frac{{w}^{\prime}_i}{{w}^{\prime}_j} \right] =
\begin{pmatrix}
$\,\,$ 1 $\,\,$ & $\,\,$\color{gr} \color{blue} 1\color{black} $\,\,$ & $\,\,$3.8058$\,\,$ & $\,\,$4.6189$\,\,$ \\
$\,\,$\color{gr} \color{blue} 1\color{black} $\,\,$ & $\,\,$ 1 $\,\,$ & $\,\,$\color{gr} 3.8058\color{black} $\,\,$ & $\,\,$\color{gr} 4.6189\color{black}   $\,\,$ \\
$\,\,$0.2628$\,\,$ & $\,\,$\color{gr} 0.2628\color{black} $\,\,$ & $\,\,$ 1 $\,\,$ & $\,\,$1.2136 $\,\,$ \\
$\,\,$0.2165$\,\,$ & $\,\,$\color{gr} 0.2165\color{black} $\,\,$ & $\,\,$0.8240$\,\,$ & $\,\,$ 1  $\,\,$ \\
\end{pmatrix},
\end{equation*}
\end{example}
\newpage
\begin{example}
\begin{equation*}
\mathbf{A} =
\begin{pmatrix}
$\,\,$ 1 $\,\,$ & $\,\,$1$\,\,$ & $\,\,$8$\,\,$ & $\,\,$2 $\,\,$ \\
$\,\,$ 1 $\,\,$ & $\,\,$ 1 $\,\,$ & $\,\,$5$\,\,$ & $\,\,$3 $\,\,$ \\
$\,\,$ 1/8$\,\,$ & $\,\,$ 1/5$\,\,$ & $\,\,$ 1 $\,\,$ & $\,\,$1 $\,\,$ \\
$\,\,$ 1/2$\,\,$ & $\,\,$ 1/3$\,\,$ & $\,\,$ 1 $\,\,$ & $\,\,$ 1  $\,\,$ \\
\end{pmatrix},
\qquad
\lambda_{\max} =
4.1655,
\qquad
CR = 0.0624
\end{equation*}

\begin{equation*}
\mathbf{w}^{cos} =
\begin{pmatrix}
0.394277\\
\color{red} 0.386061\color{black} \\
0.085014\\
0.134648
\end{pmatrix}\end{equation*}
\begin{equation*}
\left[ \frac{{w}^{cos}_i}{{w}^{cos}_j} \right] =
\begin{pmatrix}
$\,\,$ 1 $\,\,$ & $\,\,$\color{red} 1.0213\color{black} $\,\,$ & $\,\,$4.6378$\,\,$ & $\,\,$2.9282$\,\,$ \\
$\,\,$\color{red} 0.9792\color{black} $\,\,$ & $\,\,$ 1 $\,\,$ & $\,\,$\color{red} 4.5411\color{black} $\,\,$ & $\,\,$\color{red} 2.8672\color{black}   $\,\,$ \\
$\,\,$0.2156$\,\,$ & $\,\,$\color{red} 0.2202\color{black} $\,\,$ & $\,\,$ 1 $\,\,$ & $\,\,$0.6314 $\,\,$ \\
$\,\,$0.3415$\,\,$ & $\,\,$\color{red} 0.3488\color{black} $\,\,$ & $\,\,$1.5838$\,\,$ & $\,\,$ 1  $\,\,$ \\
\end{pmatrix},
\end{equation*}

\begin{equation*}
\mathbf{w}^{\prime} =
\begin{pmatrix}
0.391064\\
0.391064\\
0.084321\\
0.133551
\end{pmatrix} =
0.991851\cdot
\begin{pmatrix}
0.394277\\
\color{gr} 0.394277\color{black} \\
0.085014\\
0.134648
\end{pmatrix},
\end{equation*}
\begin{equation*}
\left[ \frac{{w}^{\prime}_i}{{w}^{\prime}_j} \right] =
\begin{pmatrix}
$\,\,$ 1 $\,\,$ & $\,\,$\color{gr} \color{blue} 1\color{black} $\,\,$ & $\,\,$4.6378$\,\,$ & $\,\,$2.9282$\,\,$ \\
$\,\,$\color{gr} \color{blue} 1\color{black} $\,\,$ & $\,\,$ 1 $\,\,$ & $\,\,$\color{gr} 4.6378\color{black} $\,\,$ & $\,\,$\color{gr} 2.9282\color{black}   $\,\,$ \\
$\,\,$0.2156$\,\,$ & $\,\,$\color{gr} 0.2156\color{black} $\,\,$ & $\,\,$ 1 $\,\,$ & $\,\,$0.6314 $\,\,$ \\
$\,\,$0.3415$\,\,$ & $\,\,$\color{gr} 0.3415\color{black} $\,\,$ & $\,\,$1.5838$\,\,$ & $\,\,$ 1  $\,\,$ \\
\end{pmatrix},
\end{equation*}
\end{example}
\newpage
\begin{example}
\begin{equation*}
\mathbf{A} =
\begin{pmatrix}
$\,\,$ 1 $\,\,$ & $\,\,$1$\,\,$ & $\,\,$8$\,\,$ & $\,\,$3 $\,\,$ \\
$\,\,$ 1 $\,\,$ & $\,\,$ 1 $\,\,$ & $\,\,$4$\,\,$ & $\,\,$4 $\,\,$ \\
$\,\,$ 1/8$\,\,$ & $\,\,$ 1/4$\,\,$ & $\,\,$ 1 $\,\,$ & $\,\,$2 $\,\,$ \\
$\,\,$ 1/3$\,\,$ & $\,\,$ 1/4$\,\,$ & $\,\,$ 1/2$\,\,$ & $\,\,$ 1  $\,\,$ \\
\end{pmatrix},
\qquad
\lambda_{\max} =
4.2512,
\qquad
CR = 0.0947
\end{equation*}

\begin{equation*}
\mathbf{w}^{cos} =
\begin{pmatrix}
0.417444\\
\color{red} 0.378601\color{black} \\
0.109055\\
0.094899
\end{pmatrix}\end{equation*}
\begin{equation*}
\left[ \frac{{w}^{cos}_i}{{w}^{cos}_j} \right] =
\begin{pmatrix}
$\,\,$ 1 $\,\,$ & $\,\,$\color{red} 1.1026\color{black} $\,\,$ & $\,\,$3.8278$\,\,$ & $\,\,$4.3988$\,\,$ \\
$\,\,$\color{red} 0.9070\color{black} $\,\,$ & $\,\,$ 1 $\,\,$ & $\,\,$\color{red} 3.4717\color{black} $\,\,$ & $\,\,$\color{red} 3.9895\color{black}   $\,\,$ \\
$\,\,$0.2612$\,\,$ & $\,\,$\color{red} 0.2880\color{black} $\,\,$ & $\,\,$ 1 $\,\,$ & $\,\,$1.1492 $\,\,$ \\
$\,\,$0.2273$\,\,$ & $\,\,$\color{red} 0.2507\color{black} $\,\,$ & $\,\,$0.8702$\,\,$ & $\,\,$ 1  $\,\,$ \\
\end{pmatrix},
\end{equation*}

\begin{equation*}
\mathbf{w}^{\prime} =
\begin{pmatrix}
0.417029\\
0.379219\\
0.108947\\
0.094805
\end{pmatrix} =
0.999006\cdot
\begin{pmatrix}
0.417444\\
\color{gr} 0.379596\color{black} \\
0.109055\\
0.094899
\end{pmatrix},
\end{equation*}
\begin{equation*}
\left[ \frac{{w}^{\prime}_i}{{w}^{\prime}_j} \right] =
\begin{pmatrix}
$\,\,$ 1 $\,\,$ & $\,\,$\color{gr} 1.0997\color{black} $\,\,$ & $\,\,$3.8278$\,\,$ & $\,\,$4.3988$\,\,$ \\
$\,\,$\color{gr} 0.9093\color{black} $\,\,$ & $\,\,$ 1 $\,\,$ & $\,\,$\color{gr} 3.4808\color{black} $\,\,$ & $\,\,$\color{gr} \color{blue} 4\color{black}   $\,\,$ \\
$\,\,$0.2612$\,\,$ & $\,\,$\color{gr} 0.2873\color{black} $\,\,$ & $\,\,$ 1 $\,\,$ & $\,\,$1.1492 $\,\,$ \\
$\,\,$0.2273$\,\,$ & $\,\,$\color{gr} \color{blue}  1/4\color{black} $\,\,$ & $\,\,$0.8702$\,\,$ & $\,\,$ 1  $\,\,$ \\
\end{pmatrix},
\end{equation*}
\end{example}
\newpage
\begin{example}
\begin{equation*}
\mathbf{A} =
\begin{pmatrix}
$\,\,$ 1 $\,\,$ & $\,\,$1$\,\,$ & $\,\,$8$\,\,$ & $\,\,$3 $\,\,$ \\
$\,\,$ 1 $\,\,$ & $\,\,$ 1 $\,\,$ & $\,\,$4$\,\,$ & $\,\,$5 $\,\,$ \\
$\,\,$ 1/8$\,\,$ & $\,\,$ 1/4$\,\,$ & $\,\,$ 1 $\,\,$ & $\,\,$2 $\,\,$ \\
$\,\,$ 1/3$\,\,$ & $\,\,$ 1/5$\,\,$ & $\,\,$ 1/2$\,\,$ & $\,\,$ 1  $\,\,$ \\
\end{pmatrix},
\qquad
\lambda_{\max} =
4.2460,
\qquad
CR = 0.0928
\end{equation*}

\begin{equation*}
\mathbf{w}^{cos} =
\begin{pmatrix}
0.413488\\
\color{red} 0.394873\color{black} \\
0.103900\\
0.087739
\end{pmatrix}\end{equation*}
\begin{equation*}
\left[ \frac{{w}^{cos}_i}{{w}^{cos}_j} \right] =
\begin{pmatrix}
$\,\,$ 1 $\,\,$ & $\,\,$\color{red} 1.0471\color{black} $\,\,$ & $\,\,$3.9797$\,\,$ & $\,\,$4.7127$\,\,$ \\
$\,\,$\color{red} 0.9550\color{black} $\,\,$ & $\,\,$ 1 $\,\,$ & $\,\,$\color{red} 3.8005\color{black} $\,\,$ & $\,\,$\color{red} 4.5005\color{black}   $\,\,$ \\
$\,\,$0.2513$\,\,$ & $\,\,$\color{red} 0.2631\color{black} $\,\,$ & $\,\,$ 1 $\,\,$ & $\,\,$1.1842 $\,\,$ \\
$\,\,$0.2122$\,\,$ & $\,\,$\color{red} 0.2222\color{black} $\,\,$ & $\,\,$0.8445$\,\,$ & $\,\,$ 1  $\,\,$ \\
\end{pmatrix},
\end{equation*}

\begin{equation*}
\mathbf{w}^{\prime} =
\begin{pmatrix}
0.405931\\
0.405931\\
0.102002\\
0.086135
\end{pmatrix} =
0.981725\cdot
\begin{pmatrix}
0.413488\\
\color{gr} 0.413488\color{black} \\
0.103900\\
0.087739
\end{pmatrix},
\end{equation*}
\begin{equation*}
\left[ \frac{{w}^{\prime}_i}{{w}^{\prime}_j} \right] =
\begin{pmatrix}
$\,\,$ 1 $\,\,$ & $\,\,$\color{gr} \color{blue} 1\color{black} $\,\,$ & $\,\,$3.9797$\,\,$ & $\,\,$4.7127$\,\,$ \\
$\,\,$\color{gr} \color{blue} 1\color{black} $\,\,$ & $\,\,$ 1 $\,\,$ & $\,\,$\color{gr} 3.9797\color{black} $\,\,$ & $\,\,$\color{gr} 4.7127\color{black}   $\,\,$ \\
$\,\,$0.2513$\,\,$ & $\,\,$\color{gr} 0.2513\color{black} $\,\,$ & $\,\,$ 1 $\,\,$ & $\,\,$1.1842 $\,\,$ \\
$\,\,$0.2122$\,\,$ & $\,\,$\color{gr} 0.2122\color{black} $\,\,$ & $\,\,$0.8445$\,\,$ & $\,\,$ 1  $\,\,$ \\
\end{pmatrix},
\end{equation*}
\end{example}
\newpage
\begin{example}
\begin{equation*}
\mathbf{A} =
\begin{pmatrix}
$\,\,$ 1 $\,\,$ & $\,\,$1$\,\,$ & $\,\,$8$\,\,$ & $\,\,$3 $\,\,$ \\
$\,\,$ 1 $\,\,$ & $\,\,$ 1 $\,\,$ & $\,\,$6$\,\,$ & $\,\,$4 $\,\,$ \\
$\,\,$ 1/8$\,\,$ & $\,\,$ 1/6$\,\,$ & $\,\,$ 1 $\,\,$ & $\,\,$1 $\,\,$ \\
$\,\,$ 1/3$\,\,$ & $\,\,$ 1/4$\,\,$ & $\,\,$ 1 $\,\,$ & $\,\,$ 1  $\,\,$ \\
\end{pmatrix},
\qquad
\lambda_{\max} =
4.0820,
\qquad
CR = 0.0309
\end{equation*}

\begin{equation*}
\mathbf{w}^{cos} =
\begin{pmatrix}
0.411604\\
\color{red} 0.410768\color{black} \\
0.073793\\
0.103836
\end{pmatrix}\end{equation*}
\begin{equation*}
\left[ \frac{{w}^{cos}_i}{{w}^{cos}_j} \right] =
\begin{pmatrix}
$\,\,$ 1 $\,\,$ & $\,\,$\color{red} 1.0020\color{black} $\,\,$ & $\,\,$5.5778$\,\,$ & $\,\,$3.9640$\,\,$ \\
$\,\,$\color{red} 0.9980\color{black} $\,\,$ & $\,\,$ 1 $\,\,$ & $\,\,$\color{red} 5.5665\color{black} $\,\,$ & $\,\,$\color{red} 3.9559\color{black}   $\,\,$ \\
$\,\,$0.1793$\,\,$ & $\,\,$\color{red} 0.1796\color{black} $\,\,$ & $\,\,$ 1 $\,\,$ & $\,\,$0.7107 $\,\,$ \\
$\,\,$0.2523$\,\,$ & $\,\,$\color{red} 0.2528\color{black} $\,\,$ & $\,\,$1.4071$\,\,$ & $\,\,$ 1  $\,\,$ \\
\end{pmatrix},
\end{equation*}

\begin{equation*}
\mathbf{w}^{\prime} =
\begin{pmatrix}
0.411260\\
0.411260\\
0.073731\\
0.103749
\end{pmatrix} =
0.999165\cdot
\begin{pmatrix}
0.411604\\
\color{gr} 0.411604\color{black} \\
0.073793\\
0.103836
\end{pmatrix},
\end{equation*}
\begin{equation*}
\left[ \frac{{w}^{\prime}_i}{{w}^{\prime}_j} \right] =
\begin{pmatrix}
$\,\,$ 1 $\,\,$ & $\,\,$\color{gr} \color{blue} 1\color{black} $\,\,$ & $\,\,$5.5778$\,\,$ & $\,\,$3.9640$\,\,$ \\
$\,\,$\color{gr} \color{blue} 1\color{black} $\,\,$ & $\,\,$ 1 $\,\,$ & $\,\,$\color{gr} 5.5778\color{black} $\,\,$ & $\,\,$\color{gr} 3.9640\color{black}   $\,\,$ \\
$\,\,$0.1793$\,\,$ & $\,\,$\color{gr} 0.1793\color{black} $\,\,$ & $\,\,$ 1 $\,\,$ & $\,\,$0.7107 $\,\,$ \\
$\,\,$0.2523$\,\,$ & $\,\,$\color{gr} 0.2523\color{black} $\,\,$ & $\,\,$1.4071$\,\,$ & $\,\,$ 1  $\,\,$ \\
\end{pmatrix},
\end{equation*}
\end{example}
\newpage
\begin{example}
\begin{equation*}
\mathbf{A} =
\begin{pmatrix}
$\,\,$ 1 $\,\,$ & $\,\,$1$\,\,$ & $\,\,$8$\,\,$ & $\,\,$4 $\,\,$ \\
$\,\,$ 1 $\,\,$ & $\,\,$ 1 $\,\,$ & $\,\,$5$\,\,$ & $\,\,$6 $\,\,$ \\
$\,\,$ 1/8$\,\,$ & $\,\,$ 1/5$\,\,$ & $\,\,$ 1 $\,\,$ & $\,\,$2 $\,\,$ \\
$\,\,$ 1/4$\,\,$ & $\,\,$ 1/6$\,\,$ & $\,\,$ 1/2$\,\,$ & $\,\,$ 1  $\,\,$ \\
\end{pmatrix},
\qquad
\lambda_{\max} =
4.1655,
\qquad
CR = 0.0624
\end{equation*}

\begin{equation*}
\mathbf{w}^{cos} =
\begin{pmatrix}
0.422076\\
\color{red} 0.414297\color{black} \\
0.091105\\
0.072523
\end{pmatrix}\end{equation*}
\begin{equation*}
\left[ \frac{{w}^{cos}_i}{{w}^{cos}_j} \right] =
\begin{pmatrix}
$\,\,$ 1 $\,\,$ & $\,\,$\color{red} 1.0188\color{black} $\,\,$ & $\,\,$4.6329$\,\,$ & $\,\,$5.8199$\,\,$ \\
$\,\,$\color{red} 0.9816\color{black} $\,\,$ & $\,\,$ 1 $\,\,$ & $\,\,$\color{red} 4.5475\color{black} $\,\,$ & $\,\,$\color{red} 5.7127\color{black}   $\,\,$ \\
$\,\,$0.2158$\,\,$ & $\,\,$\color{red} 0.2199\color{black} $\,\,$ & $\,\,$ 1 $\,\,$ & $\,\,$1.2562 $\,\,$ \\
$\,\,$0.1718$\,\,$ & $\,\,$\color{red} 0.1750\color{black} $\,\,$ & $\,\,$0.7960$\,\,$ & $\,\,$ 1  $\,\,$ \\
\end{pmatrix},
\end{equation*}

\begin{equation*}
\mathbf{w}^{\prime} =
\begin{pmatrix}
0.418818\\
0.418818\\
0.090401\\
0.071963
\end{pmatrix} =
0.992282\cdot
\begin{pmatrix}
0.422076\\
\color{gr} 0.422076\color{black} \\
0.091105\\
0.072523
\end{pmatrix},
\end{equation*}
\begin{equation*}
\left[ \frac{{w}^{\prime}_i}{{w}^{\prime}_j} \right] =
\begin{pmatrix}
$\,\,$ 1 $\,\,$ & $\,\,$\color{gr} \color{blue} 1\color{black} $\,\,$ & $\,\,$4.6329$\,\,$ & $\,\,$5.8199$\,\,$ \\
$\,\,$\color{gr} \color{blue} 1\color{black} $\,\,$ & $\,\,$ 1 $\,\,$ & $\,\,$\color{gr} 4.6329\color{black} $\,\,$ & $\,\,$\color{gr} 5.8199\color{black}   $\,\,$ \\
$\,\,$0.2158$\,\,$ & $\,\,$\color{gr} 0.2158\color{black} $\,\,$ & $\,\,$ 1 $\,\,$ & $\,\,$1.2562 $\,\,$ \\
$\,\,$0.1718$\,\,$ & $\,\,$\color{gr} 0.1718\color{black} $\,\,$ & $\,\,$0.7960$\,\,$ & $\,\,$ 1  $\,\,$ \\
\end{pmatrix},
\end{equation*}
\end{example}
\newpage
\begin{example}
\begin{equation*}
\mathbf{A} =
\begin{pmatrix}
$\,\,$ 1 $\,\,$ & $\,\,$1$\,\,$ & $\,\,$9$\,\,$ & $\,\,$2 $\,\,$ \\
$\,\,$ 1 $\,\,$ & $\,\,$ 1 $\,\,$ & $\,\,$5$\,\,$ & $\,\,$3 $\,\,$ \\
$\,\,$ 1/9$\,\,$ & $\,\,$ 1/5$\,\,$ & $\,\,$ 1 $\,\,$ & $\,\,$1 $\,\,$ \\
$\,\,$ 1/2$\,\,$ & $\,\,$ 1/3$\,\,$ & $\,\,$ 1 $\,\,$ & $\,\,$ 1  $\,\,$ \\
\end{pmatrix},
\qquad
\lambda_{\max} =
4.1966,
\qquad
CR = 0.0741
\end{equation*}

\begin{equation*}
\mathbf{w}^{cos} =
\begin{pmatrix}
0.400732\\
\color{red} 0.382128\color{black} \\
0.082913\\
0.134227
\end{pmatrix}\end{equation*}
\begin{equation*}
\left[ \frac{{w}^{cos}_i}{{w}^{cos}_j} \right] =
\begin{pmatrix}
$\,\,$ 1 $\,\,$ & $\,\,$\color{red} 1.0487\color{black} $\,\,$ & $\,\,$4.8332$\,\,$ & $\,\,$2.9855$\,\,$ \\
$\,\,$\color{red} 0.9536\color{black} $\,\,$ & $\,\,$ 1 $\,\,$ & $\,\,$\color{red} 4.6088\color{black} $\,\,$ & $\,\,$\color{red} 2.8469\color{black}   $\,\,$ \\
$\,\,$0.2069$\,\,$ & $\,\,$\color{red} 0.2170\color{black} $\,\,$ & $\,\,$ 1 $\,\,$ & $\,\,$0.6177 $\,\,$ \\
$\,\,$0.3350$\,\,$ & $\,\,$\color{red} 0.3513\color{black} $\,\,$ & $\,\,$1.6189$\,\,$ & $\,\,$ 1  $\,\,$ \\
\end{pmatrix},
\end{equation*}

\begin{equation*}
\mathbf{w}^{\prime} =
\begin{pmatrix}
0.393413\\
0.393413\\
0.081399\\
0.131775
\end{pmatrix} =
0.981735\cdot
\begin{pmatrix}
0.400732\\
\color{gr} 0.400732\color{black} \\
0.082913\\
0.134227
\end{pmatrix},
\end{equation*}
\begin{equation*}
\left[ \frac{{w}^{\prime}_i}{{w}^{\prime}_j} \right] =
\begin{pmatrix}
$\,\,$ 1 $\,\,$ & $\,\,$\color{gr} \color{blue} 1\color{black} $\,\,$ & $\,\,$4.8332$\,\,$ & $\,\,$2.9855$\,\,$ \\
$\,\,$\color{gr} \color{blue} 1\color{black} $\,\,$ & $\,\,$ 1 $\,\,$ & $\,\,$\color{gr} 4.8332\color{black} $\,\,$ & $\,\,$\color{gr} 2.9855\color{black}   $\,\,$ \\
$\,\,$0.2069$\,\,$ & $\,\,$\color{gr} 0.2069\color{black} $\,\,$ & $\,\,$ 1 $\,\,$ & $\,\,$0.6177 $\,\,$ \\
$\,\,$0.3350$\,\,$ & $\,\,$\color{gr} 0.3350\color{black} $\,\,$ & $\,\,$1.6189$\,\,$ & $\,\,$ 1  $\,\,$ \\
\end{pmatrix},
\end{equation*}
\end{example}
\newpage
\begin{example}
\begin{equation*}
\mathbf{A} =
\begin{pmatrix}
$\,\,$ 1 $\,\,$ & $\,\,$1$\,\,$ & $\,\,$9$\,\,$ & $\,\,$2 $\,\,$ \\
$\,\,$ 1 $\,\,$ & $\,\,$ 1 $\,\,$ & $\,\,$6$\,\,$ & $\,\,$3 $\,\,$ \\
$\,\,$ 1/9$\,\,$ & $\,\,$ 1/6$\,\,$ & $\,\,$ 1 $\,\,$ & $\,\,$1 $\,\,$ \\
$\,\,$ 1/2$\,\,$ & $\,\,$ 1/3$\,\,$ & $\,\,$ 1 $\,\,$ & $\,\,$ 1  $\,\,$ \\
\end{pmatrix},
\qquad
\lambda_{\max} =
4.1990,
\qquad
CR = 0.0750
\end{equation*}

\begin{equation*}
\mathbf{w}^{cos} =
\begin{pmatrix}
0.394944\\
\color{red} 0.392476\color{black} \\
0.078944\\
0.133635
\end{pmatrix}\end{equation*}
\begin{equation*}
\left[ \frac{{w}^{cos}_i}{{w}^{cos}_j} \right] =
\begin{pmatrix}
$\,\,$ 1 $\,\,$ & $\,\,$\color{red} 1.0063\color{black} $\,\,$ & $\,\,$5.0028$\,\,$ & $\,\,$2.9554$\,\,$ \\
$\,\,$\color{red} 0.9938\color{black} $\,\,$ & $\,\,$ 1 $\,\,$ & $\,\,$\color{red} 4.9716\color{black} $\,\,$ & $\,\,$\color{red} 2.9369\color{black}   $\,\,$ \\
$\,\,$0.1999$\,\,$ & $\,\,$\color{red} 0.2011\color{black} $\,\,$ & $\,\,$ 1 $\,\,$ & $\,\,$0.5907 $\,\,$ \\
$\,\,$0.3384$\,\,$ & $\,\,$\color{red} 0.3405\color{black} $\,\,$ & $\,\,$1.6928$\,\,$ & $\,\,$ 1  $\,\,$ \\
\end{pmatrix},
\end{equation*}

\begin{equation*}
\mathbf{w}^{\prime} =
\begin{pmatrix}
0.393972\\
0.393972\\
0.078750\\
0.133306
\end{pmatrix} =
0.997538\cdot
\begin{pmatrix}
0.394944\\
\color{gr} 0.394944\color{black} \\
0.078944\\
0.133635
\end{pmatrix},
\end{equation*}
\begin{equation*}
\left[ \frac{{w}^{\prime}_i}{{w}^{\prime}_j} \right] =
\begin{pmatrix}
$\,\,$ 1 $\,\,$ & $\,\,$\color{gr} \color{blue} 1\color{black} $\,\,$ & $\,\,$5.0028$\,\,$ & $\,\,$2.9554$\,\,$ \\
$\,\,$\color{gr} \color{blue} 1\color{black} $\,\,$ & $\,\,$ 1 $\,\,$ & $\,\,$\color{gr} 5.0028\color{black} $\,\,$ & $\,\,$\color{gr} 2.9554\color{black}   $\,\,$ \\
$\,\,$0.1999$\,\,$ & $\,\,$\color{gr} 0.1999\color{black} $\,\,$ & $\,\,$ 1 $\,\,$ & $\,\,$0.5907 $\,\,$ \\
$\,\,$0.3384$\,\,$ & $\,\,$\color{gr} 0.3384\color{black} $\,\,$ & $\,\,$1.6928$\,\,$ & $\,\,$ 1  $\,\,$ \\
\end{pmatrix},
\end{equation*}
\end{example}
\newpage
\begin{example}
\begin{equation*}
\mathbf{A} =
\begin{pmatrix}
$\,\,$ 1 $\,\,$ & $\,\,$1$\,\,$ & $\,\,$9$\,\,$ & $\,\,$3 $\,\,$ \\
$\,\,$ 1 $\,\,$ & $\,\,$ 1 $\,\,$ & $\,\,$6$\,\,$ & $\,\,$4 $\,\,$ \\
$\,\,$ 1/9$\,\,$ & $\,\,$ 1/6$\,\,$ & $\,\,$ 1 $\,\,$ & $\,\,$1 $\,\,$ \\
$\,\,$ 1/3$\,\,$ & $\,\,$ 1/4$\,\,$ & $\,\,$ 1 $\,\,$ & $\,\,$ 1  $\,\,$ \\
\end{pmatrix},
\qquad
\lambda_{\max} =
4.1031,
\qquad
CR = 0.0389
\end{equation*}

\begin{equation*}
\mathbf{w}^{cos} =
\begin{pmatrix}
0.418753\\
\color{red} 0.406318\color{black} \\
0.071640\\
0.103289
\end{pmatrix}\end{equation*}
\begin{equation*}
\left[ \frac{{w}^{cos}_i}{{w}^{cos}_j} \right] =
\begin{pmatrix}
$\,\,$ 1 $\,\,$ & $\,\,$\color{red} 1.0306\color{black} $\,\,$ & $\,\,$5.8453$\,\,$ & $\,\,$4.0542$\,\,$ \\
$\,\,$\color{red} 0.9703\color{black} $\,\,$ & $\,\,$ 1 $\,\,$ & $\,\,$\color{red} 5.6717\color{black} $\,\,$ & $\,\,$\color{red} 3.9338\color{black}   $\,\,$ \\
$\,\,$0.1711$\,\,$ & $\,\,$\color{red} 0.1763\color{black} $\,\,$ & $\,\,$ 1 $\,\,$ & $\,\,$0.6936 $\,\,$ \\
$\,\,$0.2467$\,\,$ & $\,\,$\color{red} 0.2542\color{black} $\,\,$ & $\,\,$1.4418$\,\,$ & $\,\,$ 1  $\,\,$ \\
\end{pmatrix},
\end{equation*}

\begin{equation*}
\mathbf{w}^{\prime} =
\begin{pmatrix}
0.415908\\
0.410351\\
0.071153\\
0.102588
\end{pmatrix} =
0.993207\cdot
\begin{pmatrix}
0.418753\\
\color{gr} 0.413157\color{black} \\
0.071640\\
0.103289
\end{pmatrix},
\end{equation*}
\begin{equation*}
\left[ \frac{{w}^{\prime}_i}{{w}^{\prime}_j} \right] =
\begin{pmatrix}
$\,\,$ 1 $\,\,$ & $\,\,$\color{gr} 1.0135\color{black} $\,\,$ & $\,\,$5.8453$\,\,$ & $\,\,$4.0542$\,\,$ \\
$\,\,$\color{gr} 0.9866\color{black} $\,\,$ & $\,\,$ 1 $\,\,$ & $\,\,$\color{gr} 5.7672\color{black} $\,\,$ & $\,\,$\color{gr} \color{blue} 4\color{black}   $\,\,$ \\
$\,\,$0.1711$\,\,$ & $\,\,$\color{gr} 0.1734\color{black} $\,\,$ & $\,\,$ 1 $\,\,$ & $\,\,$0.6936 $\,\,$ \\
$\,\,$0.2467$\,\,$ & $\,\,$\color{gr} \color{blue}  1/4\color{black} $\,\,$ & $\,\,$1.4418$\,\,$ & $\,\,$ 1  $\,\,$ \\
\end{pmatrix},
\end{equation*}
\end{example}
\newpage
\begin{example}
\begin{equation*}
\mathbf{A} =
\begin{pmatrix}
$\,\,$ 1 $\,\,$ & $\,\,$1$\,\,$ & $\,\,$9$\,\,$ & $\,\,$4 $\,\,$ \\
$\,\,$ 1 $\,\,$ & $\,\,$ 1 $\,\,$ & $\,\,$5$\,\,$ & $\,\,$6 $\,\,$ \\
$\,\,$ 1/9$\,\,$ & $\,\,$ 1/5$\,\,$ & $\,\,$ 1 $\,\,$ & $\,\,$2 $\,\,$ \\
$\,\,$ 1/4$\,\,$ & $\,\,$ 1/6$\,\,$ & $\,\,$ 1/2$\,\,$ & $\,\,$ 1  $\,\,$ \\
\end{pmatrix},
\qquad
\lambda_{\max} =
4.1966,
\qquad
CR = 0.0741
\end{equation*}

\begin{equation*}
\mathbf{w}^{cos} =
\begin{pmatrix}
0.428766\\
\color{red} 0.410109\color{black} \\
0.088826\\
0.072298
\end{pmatrix}\end{equation*}
\begin{equation*}
\left[ \frac{{w}^{cos}_i}{{w}^{cos}_j} \right] =
\begin{pmatrix}
$\,\,$ 1 $\,\,$ & $\,\,$\color{red} 1.0455\color{black} $\,\,$ & $\,\,$4.8270$\,\,$ & $\,\,$5.9305$\,\,$ \\
$\,\,$\color{red} 0.9565\color{black} $\,\,$ & $\,\,$ 1 $\,\,$ & $\,\,$\color{red} 4.6170\color{black} $\,\,$ & $\,\,$\color{red} 5.6725\color{black}   $\,\,$ \\
$\,\,$0.2072$\,\,$ & $\,\,$\color{red} 0.2166\color{black} $\,\,$ & $\,\,$ 1 $\,\,$ & $\,\,$1.2286 $\,\,$ \\
$\,\,$0.1686$\,\,$ & $\,\,$\color{red} 0.1763\color{black} $\,\,$ & $\,\,$0.8139$\,\,$ & $\,\,$ 1  $\,\,$ \\
\end{pmatrix},
\end{equation*}

\begin{equation*}
\mathbf{w}^{\prime} =
\begin{pmatrix}
0.420913\\
0.420913\\
0.087199\\
0.070974
\end{pmatrix} =
0.981685\cdot
\begin{pmatrix}
0.428766\\
\color{gr} 0.428766\color{black} \\
0.088826\\
0.072298
\end{pmatrix},
\end{equation*}
\begin{equation*}
\left[ \frac{{w}^{\prime}_i}{{w}^{\prime}_j} \right] =
\begin{pmatrix}
$\,\,$ 1 $\,\,$ & $\,\,$\color{gr} \color{blue} 1\color{black} $\,\,$ & $\,\,$4.8270$\,\,$ & $\,\,$5.9305$\,\,$ \\
$\,\,$\color{gr} \color{blue} 1\color{black} $\,\,$ & $\,\,$ 1 $\,\,$ & $\,\,$\color{gr} 4.8270\color{black} $\,\,$ & $\,\,$\color{gr} 5.9305\color{black}   $\,\,$ \\
$\,\,$0.2072$\,\,$ & $\,\,$\color{gr} 0.2072\color{black} $\,\,$ & $\,\,$ 1 $\,\,$ & $\,\,$1.2286 $\,\,$ \\
$\,\,$0.1686$\,\,$ & $\,\,$\color{gr} 0.1686\color{black} $\,\,$ & $\,\,$0.8139$\,\,$ & $\,\,$ 1  $\,\,$ \\
\end{pmatrix},
\end{equation*}
\end{example}
\newpage
\begin{example}
\begin{equation*}
\mathbf{A} =
\begin{pmatrix}
$\,\,$ 1 $\,\,$ & $\,\,$1$\,\,$ & $\,\,$9$\,\,$ & $\,\,$4 $\,\,$ \\
$\,\,$ 1 $\,\,$ & $\,\,$ 1 $\,\,$ & $\,\,$5$\,\,$ & $\,\,$7 $\,\,$ \\
$\,\,$ 1/9$\,\,$ & $\,\,$ 1/5$\,\,$ & $\,\,$ 1 $\,\,$ & $\,\,$2 $\,\,$ \\
$\,\,$ 1/4$\,\,$ & $\,\,$ 1/7$\,\,$ & $\,\,$ 1/2$\,\,$ & $\,\,$ 1  $\,\,$ \\
\end{pmatrix},
\qquad
\lambda_{\max} =
4.1975,
\qquad
CR = 0.0745
\end{equation*}

\begin{equation*}
\mathbf{w}^{cos} =
\begin{pmatrix}
0.425071\\
\color{red} 0.420775\color{black} \\
0.085708\\
0.068446
\end{pmatrix}\end{equation*}
\begin{equation*}
\left[ \frac{{w}^{cos}_i}{{w}^{cos}_j} \right] =
\begin{pmatrix}
$\,\,$ 1 $\,\,$ & $\,\,$\color{red} 1.0102\color{black} $\,\,$ & $\,\,$4.9595$\,\,$ & $\,\,$6.2103$\,\,$ \\
$\,\,$\color{red} 0.9899\color{black} $\,\,$ & $\,\,$ 1 $\,\,$ & $\,\,$\color{red} 4.9094\color{black} $\,\,$ & $\,\,$\color{red} 6.1475\color{black}   $\,\,$ \\
$\,\,$0.2016$\,\,$ & $\,\,$\color{red} 0.2037\color{black} $\,\,$ & $\,\,$ 1 $\,\,$ & $\,\,$1.2522 $\,\,$ \\
$\,\,$0.1610$\,\,$ & $\,\,$\color{red} 0.1627\color{black} $\,\,$ & $\,\,$0.7986$\,\,$ & $\,\,$ 1  $\,\,$ \\
\end{pmatrix},
\end{equation*}

\begin{equation*}
\mathbf{w}^{\prime} =
\begin{pmatrix}
0.423253\\
0.423253\\
0.085341\\
0.068153
\end{pmatrix} =
0.995722\cdot
\begin{pmatrix}
0.425071\\
\color{gr} 0.425071\color{black} \\
0.085708\\
0.068446
\end{pmatrix},
\end{equation*}
\begin{equation*}
\left[ \frac{{w}^{\prime}_i}{{w}^{\prime}_j} \right] =
\begin{pmatrix}
$\,\,$ 1 $\,\,$ & $\,\,$\color{gr} \color{blue} 1\color{black} $\,\,$ & $\,\,$4.9595$\,\,$ & $\,\,$6.2103$\,\,$ \\
$\,\,$\color{gr} \color{blue} 1\color{black} $\,\,$ & $\,\,$ 1 $\,\,$ & $\,\,$\color{gr} 4.9595\color{black} $\,\,$ & $\,\,$\color{gr} 6.2103\color{black}   $\,\,$ \\
$\,\,$0.2016$\,\,$ & $\,\,$\color{gr} 0.2016\color{black} $\,\,$ & $\,\,$ 1 $\,\,$ & $\,\,$1.2522 $\,\,$ \\
$\,\,$0.1610$\,\,$ & $\,\,$\color{gr} 0.1610\color{black} $\,\,$ & $\,\,$0.7986$\,\,$ & $\,\,$ 1  $\,\,$ \\
\end{pmatrix},
\end{equation*}
\end{example}
\newpage
\begin{example}
\begin{equation*}
\mathbf{A} =
\begin{pmatrix}
$\,\,$ 1 $\,\,$ & $\,\,$1$\,\,$ & $\,\,$9$\,\,$ & $\,\,$4 $\,\,$ \\
$\,\,$ 1 $\,\,$ & $\,\,$ 1 $\,\,$ & $\,\,$6$\,\,$ & $\,\,$6 $\,\,$ \\
$\,\,$ 1/9$\,\,$ & $\,\,$ 1/6$\,\,$ & $\,\,$ 1 $\,\,$ & $\,\,$2 $\,\,$ \\
$\,\,$ 1/4$\,\,$ & $\,\,$ 1/6$\,\,$ & $\,\,$ 1/2$\,\,$ & $\,\,$ 1  $\,\,$ \\
\end{pmatrix},
\qquad
\lambda_{\max} =
4.1990,
\qquad
CR = 0.0750
\end{equation*}

\begin{equation*}
\mathbf{w}^{cos} =
\begin{pmatrix}
0.422569\\
\color{red} 0.420882\color{black} \\
0.084578\\
0.071972
\end{pmatrix}\end{equation*}
\begin{equation*}
\left[ \frac{{w}^{cos}_i}{{w}^{cos}_j} \right] =
\begin{pmatrix}
$\,\,$ 1 $\,\,$ & $\,\,$\color{red} 1.0040\color{black} $\,\,$ & $\,\,$4.9962$\,\,$ & $\,\,$5.8713$\,\,$ \\
$\,\,$\color{red} 0.9960\color{black} $\,\,$ & $\,\,$ 1 $\,\,$ & $\,\,$\color{red} 4.9763\color{black} $\,\,$ & $\,\,$\color{red} 5.8479\color{black}   $\,\,$ \\
$\,\,$0.2002$\,\,$ & $\,\,$\color{red} 0.2010\color{black} $\,\,$ & $\,\,$ 1 $\,\,$ & $\,\,$1.1751 $\,\,$ \\
$\,\,$0.1703$\,\,$ & $\,\,$\color{red} 0.1710\color{black} $\,\,$ & $\,\,$0.8510$\,\,$ & $\,\,$ 1  $\,\,$ \\
\end{pmatrix},
\end{equation*}

\begin{equation*}
\mathbf{w}^{\prime} =
\begin{pmatrix}
0.421857\\
0.421857\\
0.084435\\
0.071851
\end{pmatrix} =
0.998316\cdot
\begin{pmatrix}
0.422569\\
\color{gr} 0.422569\color{black} \\
0.084578\\
0.071972
\end{pmatrix},
\end{equation*}
\begin{equation*}
\left[ \frac{{w}^{\prime}_i}{{w}^{\prime}_j} \right] =
\begin{pmatrix}
$\,\,$ 1 $\,\,$ & $\,\,$\color{gr} \color{blue} 1\color{black} $\,\,$ & $\,\,$4.9962$\,\,$ & $\,\,$5.8713$\,\,$ \\
$\,\,$\color{gr} \color{blue} 1\color{black} $\,\,$ & $\,\,$ 1 $\,\,$ & $\,\,$\color{gr} 4.9962\color{black} $\,\,$ & $\,\,$\color{gr} 5.8713\color{black}   $\,\,$ \\
$\,\,$0.2002$\,\,$ & $\,\,$\color{gr} 0.2002\color{black} $\,\,$ & $\,\,$ 1 $\,\,$ & $\,\,$1.1751 $\,\,$ \\
$\,\,$0.1703$\,\,$ & $\,\,$\color{gr} 0.1703\color{black} $\,\,$ & $\,\,$0.8510$\,\,$ & $\,\,$ 1  $\,\,$ \\
\end{pmatrix},
\end{equation*}
\end{example}
\newpage
\begin{example}
\begin{equation*}
\mathbf{A} =
\begin{pmatrix}
$\,\,$ 1 $\,\,$ & $\,\,$1$\,\,$ & $\,\,$9$\,\,$ & $\,\,$5 $\,\,$ \\
$\,\,$ 1 $\,\,$ & $\,\,$ 1 $\,\,$ & $\,\,$6$\,\,$ & $\,\,$7 $\,\,$ \\
$\,\,$ 1/9$\,\,$ & $\,\,$ 1/6$\,\,$ & $\,\,$ 1 $\,\,$ & $\,\,$2 $\,\,$ \\
$\,\,$ 1/5$\,\,$ & $\,\,$ 1/7$\,\,$ & $\,\,$ 1/2$\,\,$ & $\,\,$ 1  $\,\,$ \\
\end{pmatrix},
\qquad
\lambda_{\max} =
4.1417,
\qquad
CR = 0.0534
\end{equation*}

\begin{equation*}
\mathbf{w}^{cos} =
\begin{pmatrix}
0.433298\\
\color{red} 0.425390\color{black} \\
0.079446\\
0.061865
\end{pmatrix}\end{equation*}
\begin{equation*}
\left[ \frac{{w}^{cos}_i}{{w}^{cos}_j} \right] =
\begin{pmatrix}
$\,\,$ 1 $\,\,$ & $\,\,$\color{red} 1.0186\color{black} $\,\,$ & $\,\,$5.4540$\,\,$ & $\,\,$7.0039$\,\,$ \\
$\,\,$\color{red} 0.9817\color{black} $\,\,$ & $\,\,$ 1 $\,\,$ & $\,\,$\color{red} 5.3545\color{black} $\,\,$ & $\,\,$\color{red} 6.8761\color{black}   $\,\,$ \\
$\,\,$0.1834$\,\,$ & $\,\,$\color{red} 0.1868\color{black} $\,\,$ & $\,\,$ 1 $\,\,$ & $\,\,$1.2842 $\,\,$ \\
$\,\,$0.1428$\,\,$ & $\,\,$\color{red} 0.1454\color{black} $\,\,$ & $\,\,$0.7787$\,\,$ & $\,\,$ 1  $\,\,$ \\
\end{pmatrix},
\end{equation*}

\begin{equation*}
\mathbf{w}^{\prime} =
\begin{pmatrix}
0.430002\\
0.429762\\
0.078842\\
0.061395
\end{pmatrix} =
0.992392\cdot
\begin{pmatrix}
0.433298\\
\color{gr} 0.433057\color{black} \\
0.079446\\
0.061865
\end{pmatrix},
\end{equation*}
\begin{equation*}
\left[ \frac{{w}^{\prime}_i}{{w}^{\prime}_j} \right] =
\begin{pmatrix}
$\,\,$ 1 $\,\,$ & $\,\,$\color{gr} 1.0006\color{black} $\,\,$ & $\,\,$5.4540$\,\,$ & $\,\,$7.0039$\,\,$ \\
$\,\,$\color{gr} 0.9994\color{black} $\,\,$ & $\,\,$ 1 $\,\,$ & $\,\,$\color{gr} 5.4510\color{black} $\,\,$ & $\,\,$\color{gr} \color{blue} 7\color{black}   $\,\,$ \\
$\,\,$0.1834$\,\,$ & $\,\,$\color{gr} 0.1835\color{black} $\,\,$ & $\,\,$ 1 $\,\,$ & $\,\,$1.2842 $\,\,$ \\
$\,\,$0.1428$\,\,$ & $\,\,$\color{gr} \color{blue}  1/7\color{black} $\,\,$ & $\,\,$0.7787$\,\,$ & $\,\,$ 1  $\,\,$ \\
\end{pmatrix},
\end{equation*}
\end{example}
\newpage
\begin{example}
\begin{equation*}
\mathbf{A} =
\begin{pmatrix}
$\,\,$ 1 $\,\,$ & $\,\,$2$\,\,$ & $\,\,$1$\,\,$ & $\,\,$3 $\,\,$ \\
$\,\,$ 1/2$\,\,$ & $\,\,$ 1 $\,\,$ & $\,\,$2$\,\,$ & $\,\,$2 $\,\,$ \\
$\,\,$ 1 $\,\,$ & $\,\,$ 1/2$\,\,$ & $\,\,$ 1 $\,\,$ & $\,\,$2 $\,\,$ \\
$\,\,$ 1/3$\,\,$ & $\,\,$ 1/2$\,\,$ & $\,\,$ 1/2$\,\,$ & $\,\,$ 1  $\,\,$ \\
\end{pmatrix},
\qquad
\lambda_{\max} =
4.1707,
\qquad
CR = 0.0644
\end{equation*}

\begin{equation*}
\mathbf{w}^{cos} =
\begin{pmatrix}
0.360837\\
0.279636\\
0.239857\\
\color{red} 0.119671\color{black}
\end{pmatrix}\end{equation*}
\begin{equation*}
\left[ \frac{{w}^{cos}_i}{{w}^{cos}_j} \right] =
\begin{pmatrix}
$\,\,$ 1 $\,\,$ & $\,\,$1.2904$\,\,$ & $\,\,$1.5044$\,\,$ & $\,\,$\color{red} 3.0152\color{black} $\,\,$ \\
$\,\,$0.7750$\,\,$ & $\,\,$ 1 $\,\,$ & $\,\,$1.1658$\,\,$ & $\,\,$\color{red} 2.3367\color{black}   $\,\,$ \\
$\,\,$0.6647$\,\,$ & $\,\,$0.8577$\,\,$ & $\,\,$ 1 $\,\,$ & $\,\,$\color{red} 2.0043\color{black}  $\,\,$ \\
$\,\,$\color{red} 0.3316\color{black} $\,\,$ & $\,\,$\color{red} 0.4280\color{black} $\,\,$ & $\,\,$\color{red} 0.4989\color{black} $\,\,$ & $\,\,$ 1  $\,\,$ \\
\end{pmatrix},
\end{equation*}

\begin{equation*}
\mathbf{w}^{\prime} =
\begin{pmatrix}
0.360744\\
0.279564\\
0.239795\\
0.119898
\end{pmatrix} =
0.999742\cdot
\begin{pmatrix}
0.360837\\
0.279636\\
0.239857\\
\color{gr} 0.119928\color{black}
\end{pmatrix},
\end{equation*}
\begin{equation*}
\left[ \frac{{w}^{\prime}_i}{{w}^{\prime}_j} \right] =
\begin{pmatrix}
$\,\,$ 1 $\,\,$ & $\,\,$1.2904$\,\,$ & $\,\,$1.5044$\,\,$ & $\,\,$\color{gr} 3.0088\color{black} $\,\,$ \\
$\,\,$0.7750$\,\,$ & $\,\,$ 1 $\,\,$ & $\,\,$1.1658$\,\,$ & $\,\,$\color{gr} 2.3317\color{black}   $\,\,$ \\
$\,\,$0.6647$\,\,$ & $\,\,$0.8577$\,\,$ & $\,\,$ 1 $\,\,$ & $\,\,$\color{gr} \color{blue} 2\color{black}  $\,\,$ \\
$\,\,$\color{gr} 0.3324\color{black} $\,\,$ & $\,\,$\color{gr} 0.4289\color{black} $\,\,$ & $\,\,$\color{gr} \color{blue}  1/2\color{black} $\,\,$ & $\,\,$ 1  $\,\,$ \\
\end{pmatrix},
\end{equation*}
\end{example}
\newpage
\begin{example}
\begin{equation*}
\mathbf{A} =
\begin{pmatrix}
$\,\,$ 1 $\,\,$ & $\,\,$2$\,\,$ & $\,\,$1$\,\,$ & $\,\,$5 $\,\,$ \\
$\,\,$ 1/2$\,\,$ & $\,\,$ 1 $\,\,$ & $\,\,$2$\,\,$ & $\,\,$4 $\,\,$ \\
$\,\,$ 1 $\,\,$ & $\,\,$ 1/2$\,\,$ & $\,\,$ 1 $\,\,$ & $\,\,$3 $\,\,$ \\
$\,\,$ 1/5$\,\,$ & $\,\,$ 1/4$\,\,$ & $\,\,$ 1/3$\,\,$ & $\,\,$ 1  $\,\,$ \\
\end{pmatrix},
\qquad
\lambda_{\max} =
4.1655,
\qquad
CR = 0.0624
\end{equation*}

\begin{equation*}
\mathbf{w}^{cos} =
\begin{pmatrix}
0.377008\\
0.305452\\
0.243734\\
\color{red} 0.073806\color{black}
\end{pmatrix}\end{equation*}
\begin{equation*}
\left[ \frac{{w}^{cos}_i}{{w}^{cos}_j} \right] =
\begin{pmatrix}
$\,\,$ 1 $\,\,$ & $\,\,$1.2343$\,\,$ & $\,\,$1.5468$\,\,$ & $\,\,$\color{red} 5.1081\color{black} $\,\,$ \\
$\,\,$0.8102$\,\,$ & $\,\,$ 1 $\,\,$ & $\,\,$1.2532$\,\,$ & $\,\,$\color{red} 4.1386\color{black}   $\,\,$ \\
$\,\,$0.6465$\,\,$ & $\,\,$0.7979$\,\,$ & $\,\,$ 1 $\,\,$ & $\,\,$\color{red} 3.3024\color{black}  $\,\,$ \\
$\,\,$\color{red} 0.1958\color{black} $\,\,$ & $\,\,$\color{red} 0.2416\color{black} $\,\,$ & $\,\,$\color{red} 0.3028\color{black} $\,\,$ & $\,\,$ 1  $\,\,$ \\
\end{pmatrix},
\end{equation*}

\begin{equation*}
\mathbf{w}^{\prime} =
\begin{pmatrix}
0.376407\\
0.304966\\
0.243346\\
0.075281
\end{pmatrix} =
0.998407\cdot
\begin{pmatrix}
0.377008\\
0.305452\\
0.243734\\
\color{gr} 0.075402\color{black}
\end{pmatrix},
\end{equation*}
\begin{equation*}
\left[ \frac{{w}^{\prime}_i}{{w}^{\prime}_j} \right] =
\begin{pmatrix}
$\,\,$ 1 $\,\,$ & $\,\,$1.2343$\,\,$ & $\,\,$1.5468$\,\,$ & $\,\,$\color{gr} \color{blue} 5\color{black} $\,\,$ \\
$\,\,$0.8102$\,\,$ & $\,\,$ 1 $\,\,$ & $\,\,$1.2532$\,\,$ & $\,\,$\color{gr} 4.0510\color{black}   $\,\,$ \\
$\,\,$0.6465$\,\,$ & $\,\,$0.7979$\,\,$ & $\,\,$ 1 $\,\,$ & $\,\,$\color{gr} 3.2325\color{black}  $\,\,$ \\
$\,\,$\color{gr} \color{blue}  1/5\color{black} $\,\,$ & $\,\,$\color{gr} 0.2469\color{black} $\,\,$ & $\,\,$\color{gr} 0.3094\color{black} $\,\,$ & $\,\,$ 1  $\,\,$ \\
\end{pmatrix},
\end{equation*}
\end{example}
\newpage
\begin{example}
\begin{equation*}
\mathbf{A} =
\begin{pmatrix}
$\,\,$ 1 $\,\,$ & $\,\,$2$\,\,$ & $\,\,$1$\,\,$ & $\,\,$6 $\,\,$ \\
$\,\,$ 1/2$\,\,$ & $\,\,$ 1 $\,\,$ & $\,\,$2$\,\,$ & $\,\,$4 $\,\,$ \\
$\,\,$ 1 $\,\,$ & $\,\,$ 1/2$\,\,$ & $\,\,$ 1 $\,\,$ & $\,\,$4 $\,\,$ \\
$\,\,$ 1/6$\,\,$ & $\,\,$ 1/4$\,\,$ & $\,\,$ 1/4$\,\,$ & $\,\,$ 1  $\,\,$ \\
\end{pmatrix},
\qquad
\lambda_{\max} =
4.1707,
\qquad
CR = 0.0644
\end{equation*}

\begin{equation*}
\mathbf{w}^{cos} =
\begin{pmatrix}
0.383895\\
0.297292\\
0.255164\\
\color{red} 0.063649\color{black}
\end{pmatrix}\end{equation*}
\begin{equation*}
\left[ \frac{{w}^{cos}_i}{{w}^{cos}_j} \right] =
\begin{pmatrix}
$\,\,$ 1 $\,\,$ & $\,\,$1.2913$\,\,$ & $\,\,$1.5045$\,\,$ & $\,\,$\color{red} 6.0314\color{black} $\,\,$ \\
$\,\,$0.7744$\,\,$ & $\,\,$ 1 $\,\,$ & $\,\,$1.1651$\,\,$ & $\,\,$\color{red} 4.6708\color{black}   $\,\,$ \\
$\,\,$0.6647$\,\,$ & $\,\,$0.8583$\,\,$ & $\,\,$ 1 $\,\,$ & $\,\,$\color{red} 4.0089\color{black}  $\,\,$ \\
$\,\,$\color{red} 0.1658\color{black} $\,\,$ & $\,\,$\color{red} 0.2141\color{black} $\,\,$ & $\,\,$\color{red} 0.2494\color{black} $\,\,$ & $\,\,$ 1  $\,\,$ \\
\end{pmatrix},
\end{equation*}

\begin{equation*}
\mathbf{w}^{\prime} =
\begin{pmatrix}
0.383841\\
0.297250\\
0.255127\\
0.063782
\end{pmatrix} =
0.999858\cdot
\begin{pmatrix}
0.383895\\
0.297292\\
0.255164\\
\color{gr} 0.063791\color{black}
\end{pmatrix},
\end{equation*}
\begin{equation*}
\left[ \frac{{w}^{\prime}_i}{{w}^{\prime}_j} \right] =
\begin{pmatrix}
$\,\,$ 1 $\,\,$ & $\,\,$1.2913$\,\,$ & $\,\,$1.5045$\,\,$ & $\,\,$\color{gr} 6.0180\color{black} $\,\,$ \\
$\,\,$0.7744$\,\,$ & $\,\,$ 1 $\,\,$ & $\,\,$1.1651$\,\,$ & $\,\,$\color{gr} 4.6604\color{black}   $\,\,$ \\
$\,\,$0.6647$\,\,$ & $\,\,$0.8583$\,\,$ & $\,\,$ 1 $\,\,$ & $\,\,$\color{gr} \color{blue} 4\color{black}  $\,\,$ \\
$\,\,$\color{gr} 0.1662\color{black} $\,\,$ & $\,\,$\color{gr} 0.2146\color{black} $\,\,$ & $\,\,$\color{gr} \color{blue}  1/4\color{black} $\,\,$ & $\,\,$ 1  $\,\,$ \\
\end{pmatrix},
\end{equation*}
\end{example}
\newpage
\begin{example}
\begin{equation*}
\mathbf{A} =
\begin{pmatrix}
$\,\,$ 1 $\,\,$ & $\,\,$2$\,\,$ & $\,\,$1$\,\,$ & $\,\,$6 $\,\,$ \\
$\,\,$ 1/2$\,\,$ & $\,\,$ 1 $\,\,$ & $\,\,$2$\,\,$ & $\,\,$5 $\,\,$ \\
$\,\,$ 1 $\,\,$ & $\,\,$ 1/2$\,\,$ & $\,\,$ 1 $\,\,$ & $\,\,$4 $\,\,$ \\
$\,\,$ 1/6$\,\,$ & $\,\,$ 1/5$\,\,$ & $\,\,$ 1/4$\,\,$ & $\,\,$ 1  $\,\,$ \\
\end{pmatrix},
\qquad
\lambda_{\max} =
4.1655,
\qquad
CR = 0.0624
\end{equation*}

\begin{equation*}
\mathbf{w}^{cos} =
\begin{pmatrix}
0.378639\\
0.310197\\
0.251547\\
\color{red} 0.059617\color{black}
\end{pmatrix}\end{equation*}
\begin{equation*}
\left[ \frac{{w}^{cos}_i}{{w}^{cos}_j} \right] =
\begin{pmatrix}
$\,\,$ 1 $\,\,$ & $\,\,$1.2206$\,\,$ & $\,\,$1.5052$\,\,$ & $\,\,$\color{red} 6.3512\color{black} $\,\,$ \\
$\,\,$0.8192$\,\,$ & $\,\,$ 1 $\,\,$ & $\,\,$1.2332$\,\,$ & $\,\,$\color{red} 5.2032\color{black}   $\,\,$ \\
$\,\,$0.6643$\,\,$ & $\,\,$0.8109$\,\,$ & $\,\,$ 1 $\,\,$ & $\,\,$\color{red} 4.2194\color{black}  $\,\,$ \\
$\,\,$\color{red} 0.1574\color{black} $\,\,$ & $\,\,$\color{red} 0.1922\color{black} $\,\,$ & $\,\,$\color{red} 0.2370\color{black} $\,\,$ & $\,\,$ 1  $\,\,$ \\
\end{pmatrix},
\end{equation*}

\begin{equation*}
\mathbf{w}^{\prime} =
\begin{pmatrix}
0.377724\\
0.309447\\
0.250939\\
0.061889
\end{pmatrix} =
0.997583\cdot
\begin{pmatrix}
0.378639\\
0.310197\\
0.251547\\
\color{gr} 0.062039\color{black}
\end{pmatrix},
\end{equation*}
\begin{equation*}
\left[ \frac{{w}^{\prime}_i}{{w}^{\prime}_j} \right] =
\begin{pmatrix}
$\,\,$ 1 $\,\,$ & $\,\,$1.2206$\,\,$ & $\,\,$1.5052$\,\,$ & $\,\,$\color{gr} 6.1032\color{black} $\,\,$ \\
$\,\,$0.8192$\,\,$ & $\,\,$ 1 $\,\,$ & $\,\,$1.2332$\,\,$ & $\,\,$\color{gr} \color{blue} 5\color{black}   $\,\,$ \\
$\,\,$0.6643$\,\,$ & $\,\,$0.8109$\,\,$ & $\,\,$ 1 $\,\,$ & $\,\,$\color{gr} 4.0546\color{black}  $\,\,$ \\
$\,\,$\color{gr} 0.1638\color{black} $\,\,$ & $\,\,$\color{gr} \color{blue}  1/5\color{black} $\,\,$ & $\,\,$\color{gr} 0.2466\color{black} $\,\,$ & $\,\,$ 1  $\,\,$ \\
\end{pmatrix},
\end{equation*}
\end{example}
\newpage
\begin{example}
\begin{equation*}
\mathbf{A} =
\begin{pmatrix}
$\,\,$ 1 $\,\,$ & $\,\,$2$\,\,$ & $\,\,$1$\,\,$ & $\,\,$7 $\,\,$ \\
$\,\,$ 1/2$\,\,$ & $\,\,$ 1 $\,\,$ & $\,\,$2$\,\,$ & $\,\,$6 $\,\,$ \\
$\,\,$ 1 $\,\,$ & $\,\,$ 1/2$\,\,$ & $\,\,$ 1 $\,\,$ & $\,\,$5 $\,\,$ \\
$\,\,$ 1/7$\,\,$ & $\,\,$ 1/6$\,\,$ & $\,\,$ 1/5$\,\,$ & $\,\,$ 1  $\,\,$ \\
\end{pmatrix},
\qquad
\lambda_{\max} =
4.1667,
\qquad
CR = 0.0629
\end{equation*}

\begin{equation*}
\mathbf{w}^{cos} =
\begin{pmatrix}
0.379685\\
0.313361\\
0.256876\\
\color{red} 0.050078\color{black}
\end{pmatrix}\end{equation*}
\begin{equation*}
\left[ \frac{{w}^{cos}_i}{{w}^{cos}_j} \right] =
\begin{pmatrix}
$\,\,$ 1 $\,\,$ & $\,\,$1.2117$\,\,$ & $\,\,$1.4781$\,\,$ & $\,\,$\color{red} 7.5819\color{black} $\,\,$ \\
$\,\,$0.8253$\,\,$ & $\,\,$ 1 $\,\,$ & $\,\,$1.2199$\,\,$ & $\,\,$\color{red} 6.2575\color{black}   $\,\,$ \\
$\,\,$0.6766$\,\,$ & $\,\,$0.8197$\,\,$ & $\,\,$ 1 $\,\,$ & $\,\,$\color{red} 5.1296\color{black}  $\,\,$ \\
$\,\,$\color{red} 0.1319\color{black} $\,\,$ & $\,\,$\color{red} 0.1598\color{black} $\,\,$ & $\,\,$\color{red} 0.1949\color{black} $\,\,$ & $\,\,$ 1  $\,\,$ \\
\end{pmatrix},
\end{equation*}

\begin{equation*}
\mathbf{w}^{\prime} =
\begin{pmatrix}
0.379193\\
0.312955\\
0.256543\\
0.051309
\end{pmatrix} =
0.998704\cdot
\begin{pmatrix}
0.379685\\
0.313361\\
0.256876\\
\color{gr} 0.051375\color{black}
\end{pmatrix},
\end{equation*}
\begin{equation*}
\left[ \frac{{w}^{\prime}_i}{{w}^{\prime}_j} \right] =
\begin{pmatrix}
$\,\,$ 1 $\,\,$ & $\,\,$1.2117$\,\,$ & $\,\,$1.4781$\,\,$ & $\,\,$\color{gr} 7.3904\color{black} $\,\,$ \\
$\,\,$0.8253$\,\,$ & $\,\,$ 1 $\,\,$ & $\,\,$1.2199$\,\,$ & $\,\,$\color{gr} 6.0995\color{black}   $\,\,$ \\
$\,\,$0.6766$\,\,$ & $\,\,$0.8197$\,\,$ & $\,\,$ 1 $\,\,$ & $\,\,$\color{gr} \color{blue} 5\color{black}  $\,\,$ \\
$\,\,$\color{gr} 0.1353\color{black} $\,\,$ & $\,\,$\color{gr} 0.1639\color{black} $\,\,$ & $\,\,$\color{gr} \color{blue}  1/5\color{black} $\,\,$ & $\,\,$ 1  $\,\,$ \\
\end{pmatrix},
\end{equation*}
\end{example}
\newpage
\begin{example}
\begin{equation*}
\mathbf{A} =
\begin{pmatrix}
$\,\,$ 1 $\,\,$ & $\,\,$2$\,\,$ & $\,\,$1$\,\,$ & $\,\,$8 $\,\,$ \\
$\,\,$ 1/2$\,\,$ & $\,\,$ 1 $\,\,$ & $\,\,$2$\,\,$ & $\,\,$6 $\,\,$ \\
$\,\,$ 1 $\,\,$ & $\,\,$ 1/2$\,\,$ & $\,\,$ 1 $\,\,$ & $\,\,$5 $\,\,$ \\
$\,\,$ 1/8$\,\,$ & $\,\,$ 1/6$\,\,$ & $\,\,$ 1/5$\,\,$ & $\,\,$ 1  $\,\,$ \\
\end{pmatrix},
\qquad
\lambda_{\max} =
4.1655,
\qquad
CR = 0.0624
\end{equation*}

\begin{equation*}
\mathbf{w}^{cos} =
\begin{pmatrix}
0.388617\\
0.309745\\
0.253903\\
\color{red} 0.047734\color{black}
\end{pmatrix}\end{equation*}
\begin{equation*}
\left[ \frac{{w}^{cos}_i}{{w}^{cos}_j} \right] =
\begin{pmatrix}
$\,\,$ 1 $\,\,$ & $\,\,$1.2546$\,\,$ & $\,\,$1.5306$\,\,$ & $\,\,$\color{red} 8.1412\color{black} $\,\,$ \\
$\,\,$0.7970$\,\,$ & $\,\,$ 1 $\,\,$ & $\,\,$1.2199$\,\,$ & $\,\,$\color{red} 6.4889\color{black}   $\,\,$ \\
$\,\,$0.6534$\,\,$ & $\,\,$0.8197$\,\,$ & $\,\,$ 1 $\,\,$ & $\,\,$\color{red} 5.3191\color{black}  $\,\,$ \\
$\,\,$\color{red} 0.1228\color{black} $\,\,$ & $\,\,$\color{red} 0.1541\color{black} $\,\,$ & $\,\,$\color{red} 0.1880\color{black} $\,\,$ & $\,\,$ 1  $\,\,$ \\
\end{pmatrix},
\end{equation*}

\begin{equation*}
\mathbf{w}^{\prime} =
\begin{pmatrix}
0.388290\\
0.309485\\
0.253689\\
0.048536
\end{pmatrix} =
0.999158\cdot
\begin{pmatrix}
0.388617\\
0.309745\\
0.253903\\
\color{gr} 0.048577\color{black}
\end{pmatrix},
\end{equation*}
\begin{equation*}
\left[ \frac{{w}^{\prime}_i}{{w}^{\prime}_j} \right] =
\begin{pmatrix}
$\,\,$ 1 $\,\,$ & $\,\,$1.2546$\,\,$ & $\,\,$1.5306$\,\,$ & $\,\,$\color{gr} \color{blue} 8\color{black} $\,\,$ \\
$\,\,$0.7970$\,\,$ & $\,\,$ 1 $\,\,$ & $\,\,$1.2199$\,\,$ & $\,\,$\color{gr} 6.3764\color{black}   $\,\,$ \\
$\,\,$0.6534$\,\,$ & $\,\,$0.8197$\,\,$ & $\,\,$ 1 $\,\,$ & $\,\,$\color{gr} 5.2268\color{black}  $\,\,$ \\
$\,\,$\color{gr} \color{blue}  1/8\color{black} $\,\,$ & $\,\,$\color{gr} 0.1568\color{black} $\,\,$ & $\,\,$\color{gr} 0.1913\color{black} $\,\,$ & $\,\,$ 1  $\,\,$ \\
\end{pmatrix},
\end{equation*}
\end{example}
\newpage
\begin{example}
\begin{equation*}
\mathbf{A} =
\begin{pmatrix}
$\,\,$ 1 $\,\,$ & $\,\,$2$\,\,$ & $\,\,$1$\,\,$ & $\,\,$8 $\,\,$ \\
$\,\,$ 1/2$\,\,$ & $\,\,$ 1 $\,\,$ & $\,\,$2$\,\,$ & $\,\,$7 $\,\,$ \\
$\,\,$ 1 $\,\,$ & $\,\,$ 1/2$\,\,$ & $\,\,$ 1 $\,\,$ & $\,\,$6 $\,\,$ \\
$\,\,$ 1/8$\,\,$ & $\,\,$ 1/7$\,\,$ & $\,\,$ 1/6$\,\,$ & $\,\,$ 1  $\,\,$ \\
\end{pmatrix},
\qquad
\lambda_{\max} =
4.1681,
\qquad
CR = 0.0634
\end{equation*}

\begin{equation*}
\mathbf{w}^{cos} =
\begin{pmatrix}
0.380421\\
0.315634\\
0.260745\\
\color{red} 0.043201\color{black}
\end{pmatrix}\end{equation*}
\begin{equation*}
\left[ \frac{{w}^{cos}_i}{{w}^{cos}_j} \right] =
\begin{pmatrix}
$\,\,$ 1 $\,\,$ & $\,\,$1.2053$\,\,$ & $\,\,$1.4590$\,\,$ & $\,\,$\color{red} 8.8059\color{black} $\,\,$ \\
$\,\,$0.8297$\,\,$ & $\,\,$ 1 $\,\,$ & $\,\,$1.2105$\,\,$ & $\,\,$\color{red} 7.3063\color{black}   $\,\,$ \\
$\,\,$0.6854$\,\,$ & $\,\,$0.8261$\,\,$ & $\,\,$ 1 $\,\,$ & $\,\,$\color{red} 6.0357\color{black}  $\,\,$ \\
$\,\,$\color{red} 0.1136\color{black} $\,\,$ & $\,\,$\color{red} 0.1369\color{black} $\,\,$ & $\,\,$\color{red} 0.1657\color{black} $\,\,$ & $\,\,$ 1  $\,\,$ \\
\end{pmatrix},
\end{equation*}

\begin{equation*}
\mathbf{w}^{\prime} =
\begin{pmatrix}
0.380323\\
0.315553\\
0.260678\\
0.043446
\end{pmatrix} =
0.999743\cdot
\begin{pmatrix}
0.380421\\
0.315634\\
0.260745\\
\color{gr} 0.043458\color{black}
\end{pmatrix},
\end{equation*}
\begin{equation*}
\left[ \frac{{w}^{\prime}_i}{{w}^{\prime}_j} \right] =
\begin{pmatrix}
$\,\,$ 1 $\,\,$ & $\,\,$1.2053$\,\,$ & $\,\,$1.4590$\,\,$ & $\,\,$\color{gr} 8.7539\color{black} $\,\,$ \\
$\,\,$0.8297$\,\,$ & $\,\,$ 1 $\,\,$ & $\,\,$1.2105$\,\,$ & $\,\,$\color{gr} 7.2630\color{black}   $\,\,$ \\
$\,\,$0.6854$\,\,$ & $\,\,$0.8261$\,\,$ & $\,\,$ 1 $\,\,$ & $\,\,$\color{gr} \color{blue} 6\color{black}  $\,\,$ \\
$\,\,$\color{gr} 0.1142\color{black} $\,\,$ & $\,\,$\color{gr} 0.1377\color{black} $\,\,$ & $\,\,$\color{gr} \color{blue}  1/6\color{black} $\,\,$ & $\,\,$ 1  $\,\,$ \\
\end{pmatrix},
\end{equation*}
\end{example}
\newpage
\begin{example}
\begin{equation*}
\mathbf{A} =
\begin{pmatrix}
$\,\,$ 1 $\,\,$ & $\,\,$2$\,\,$ & $\,\,$1$\,\,$ & $\,\,$9 $\,\,$ \\
$\,\,$ 1/2$\,\,$ & $\,\,$ 1 $\,\,$ & $\,\,$2$\,\,$ & $\,\,$6 $\,\,$ \\
$\,\,$ 1 $\,\,$ & $\,\,$ 1/2$\,\,$ & $\,\,$ 1 $\,\,$ & $\,\,$6 $\,\,$ \\
$\,\,$ 1/9$\,\,$ & $\,\,$ 1/6$\,\,$ & $\,\,$ 1/6$\,\,$ & $\,\,$ 1  $\,\,$ \\
\end{pmatrix},
\qquad
\lambda_{\max} =
4.1707,
\qquad
CR = 0.0644
\end{equation*}

\begin{equation*}
\mathbf{w}^{cos} =
\begin{pmatrix}
0.392235\\
0.303709\\
0.260703\\
\color{red} 0.043353\color{black}
\end{pmatrix}\end{equation*}
\begin{equation*}
\left[ \frac{{w}^{cos}_i}{{w}^{cos}_j} \right] =
\begin{pmatrix}
$\,\,$ 1 $\,\,$ & $\,\,$1.2915$\,\,$ & $\,\,$1.5045$\,\,$ & $\,\,$\color{red} 9.0474\color{black} $\,\,$ \\
$\,\,$0.7743$\,\,$ & $\,\,$ 1 $\,\,$ & $\,\,$1.1650$\,\,$ & $\,\,$\color{red} 7.0055\color{black}   $\,\,$ \\
$\,\,$0.6647$\,\,$ & $\,\,$0.8584$\,\,$ & $\,\,$ 1 $\,\,$ & $\,\,$\color{red} 6.0135\color{black}  $\,\,$ \\
$\,\,$\color{red} 0.1105\color{black} $\,\,$ & $\,\,$\color{red} 0.1427\color{black} $\,\,$ & $\,\,$\color{red} 0.1663\color{black} $\,\,$ & $\,\,$ 1  $\,\,$ \\
\end{pmatrix},
\end{equation*}

\begin{equation*}
\mathbf{w}^{\prime} =
\begin{pmatrix}
0.392197\\
0.303680\\
0.260677\\
0.043446
\end{pmatrix} =
0.999903\cdot
\begin{pmatrix}
0.392235\\
0.303709\\
0.260703\\
\color{gr} 0.043450\color{black}
\end{pmatrix},
\end{equation*}
\begin{equation*}
\left[ \frac{{w}^{\prime}_i}{{w}^{\prime}_j} \right] =
\begin{pmatrix}
$\,\,$ 1 $\,\,$ & $\,\,$1.2915$\,\,$ & $\,\,$1.5045$\,\,$ & $\,\,$\color{gr} 9.0272\color{black} $\,\,$ \\
$\,\,$0.7743$\,\,$ & $\,\,$ 1 $\,\,$ & $\,\,$1.1650$\,\,$ & $\,\,$\color{gr} 6.9898\color{black}   $\,\,$ \\
$\,\,$0.6647$\,\,$ & $\,\,$0.8584$\,\,$ & $\,\,$ 1 $\,\,$ & $\,\,$\color{gr} \color{blue} 6\color{black}  $\,\,$ \\
$\,\,$\color{gr} 0.1108\color{black} $\,\,$ & $\,\,$\color{gr} 0.1431\color{black} $\,\,$ & $\,\,$\color{gr} \color{blue}  1/6\color{black} $\,\,$ & $\,\,$ 1  $\,\,$ \\
\end{pmatrix},
\end{equation*}
\end{example}
\newpage
\begin{example}
\begin{equation*}
\mathbf{A} =
\begin{pmatrix}
$\,\,$ 1 $\,\,$ & $\,\,$2$\,\,$ & $\,\,$1$\,\,$ & $\,\,$9 $\,\,$ \\
$\,\,$ 1/2$\,\,$ & $\,\,$ 1 $\,\,$ & $\,\,$2$\,\,$ & $\,\,$7 $\,\,$ \\
$\,\,$ 1 $\,\,$ & $\,\,$ 1/2$\,\,$ & $\,\,$ 1 $\,\,$ & $\,\,$6 $\,\,$ \\
$\,\,$ 1/9$\,\,$ & $\,\,$ 1/7$\,\,$ & $\,\,$ 1/6$\,\,$ & $\,\,$ 1  $\,\,$ \\
\end{pmatrix},
\qquad
\lambda_{\max} =
4.1658,
\qquad
CR = 0.0625
\end{equation*}

\begin{equation*}
\mathbf{w}^{cos} =
\begin{pmatrix}
0.388220\\
0.312412\\
0.257989\\
\color{red} 0.041379\color{black}
\end{pmatrix}\end{equation*}
\begin{equation*}
\left[ \frac{{w}^{cos}_i}{{w}^{cos}_j} \right] =
\begin{pmatrix}
$\,\,$ 1 $\,\,$ & $\,\,$1.2427$\,\,$ & $\,\,$1.5048$\,\,$ & $\,\,$\color{red} 9.3820\color{black} $\,\,$ \\
$\,\,$0.8047$\,\,$ & $\,\,$ 1 $\,\,$ & $\,\,$1.2110$\,\,$ & $\,\,$\color{red} 7.5500\color{black}   $\,\,$ \\
$\,\,$0.6645$\,\,$ & $\,\,$0.8258$\,\,$ & $\,\,$ 1 $\,\,$ & $\,\,$\color{red} 6.2347\color{black}  $\,\,$ \\
$\,\,$\color{red} 0.1066\color{black} $\,\,$ & $\,\,$\color{red} 0.1325\color{black} $\,\,$ & $\,\,$\color{red} 0.1604\color{black} $\,\,$ & $\,\,$ 1  $\,\,$ \\
\end{pmatrix},
\end{equation*}

\begin{equation*}
\mathbf{w}^{\prime} =
\begin{pmatrix}
0.387593\\
0.311907\\
0.257572\\
0.042929
\end{pmatrix} =
0.998384\cdot
\begin{pmatrix}
0.388220\\
0.312412\\
0.257989\\
\color{gr} 0.042998\color{black}
\end{pmatrix},
\end{equation*}
\begin{equation*}
\left[ \frac{{w}^{\prime}_i}{{w}^{\prime}_j} \right] =
\begin{pmatrix}
$\,\,$ 1 $\,\,$ & $\,\,$1.2427$\,\,$ & $\,\,$1.5048$\,\,$ & $\,\,$\color{gr} 9.0288\color{black} $\,\,$ \\
$\,\,$0.8047$\,\,$ & $\,\,$ 1 $\,\,$ & $\,\,$1.2110$\,\,$ & $\,\,$\color{gr} 7.2657\color{black}   $\,\,$ \\
$\,\,$0.6645$\,\,$ & $\,\,$0.8258$\,\,$ & $\,\,$ 1 $\,\,$ & $\,\,$\color{gr} \color{blue} 6\color{black}  $\,\,$ \\
$\,\,$\color{gr} 0.1108\color{black} $\,\,$ & $\,\,$\color{gr} 0.1376\color{black} $\,\,$ & $\,\,$\color{gr} \color{blue}  1/6\color{black} $\,\,$ & $\,\,$ 1  $\,\,$ \\
\end{pmatrix},
\end{equation*}
\end{example}
\newpage
\begin{example}
\begin{equation*}
\mathbf{A} =
\begin{pmatrix}
$\,\,$ 1 $\,\,$ & $\,\,$2$\,\,$ & $\,\,$1$\,\,$ & $\,\,$9 $\,\,$ \\
$\,\,$ 1/2$\,\,$ & $\,\,$ 1 $\,\,$ & $\,\,$2$\,\,$ & $\,\,$8 $\,\,$ \\
$\,\,$ 1 $\,\,$ & $\,\,$ 1/2$\,\,$ & $\,\,$ 1 $\,\,$ & $\,\,$6 $\,\,$ \\
$\,\,$ 1/9$\,\,$ & $\,\,$ 1/8$\,\,$ & $\,\,$ 1/6$\,\,$ & $\,\,$ 1  $\,\,$ \\
\end{pmatrix},
\qquad
\lambda_{\max} =
4.1664,
\qquad
CR = 0.0627
\end{equation*}

\begin{equation*}
\mathbf{w}^{cos} =
\begin{pmatrix}
0.384411\\
0.320359\\
0.255424\\
\color{red} 0.039806\color{black}
\end{pmatrix}\end{equation*}
\begin{equation*}
\left[ \frac{{w}^{cos}_i}{{w}^{cos}_j} \right] =
\begin{pmatrix}
$\,\,$ 1 $\,\,$ & $\,\,$1.1999$\,\,$ & $\,\,$1.5050$\,\,$ & $\,\,$\color{red} 9.6571\color{black} $\,\,$ \\
$\,\,$0.8334$\,\,$ & $\,\,$ 1 $\,\,$ & $\,\,$1.2542$\,\,$ & $\,\,$\color{red} 8.0480\color{black}   $\,\,$ \\
$\,\,$0.6645$\,\,$ & $\,\,$0.7973$\,\,$ & $\,\,$ 1 $\,\,$ & $\,\,$\color{red} 6.4167\color{black}  $\,\,$ \\
$\,\,$\color{red} 0.1036\color{black} $\,\,$ & $\,\,$\color{red} 0.1243\color{black} $\,\,$ & $\,\,$\color{red} 0.1558\color{black} $\,\,$ & $\,\,$ 1  $\,\,$ \\
\end{pmatrix},
\end{equation*}

\begin{equation*}
\mathbf{w}^{\prime} =
\begin{pmatrix}
0.384319\\
0.320283\\
0.255363\\
0.040035
\end{pmatrix} =
0.999761\cdot
\begin{pmatrix}
0.384411\\
0.320359\\
0.255424\\
\color{gr} 0.040045\color{black}
\end{pmatrix},
\end{equation*}
\begin{equation*}
\left[ \frac{{w}^{\prime}_i}{{w}^{\prime}_j} \right] =
\begin{pmatrix}
$\,\,$ 1 $\,\,$ & $\,\,$1.1999$\,\,$ & $\,\,$1.5050$\,\,$ & $\,\,$\color{gr} 9.5995\color{black} $\,\,$ \\
$\,\,$0.8334$\,\,$ & $\,\,$ 1 $\,\,$ & $\,\,$1.2542$\,\,$ & $\,\,$\color{gr} \color{blue} 8\color{black}   $\,\,$ \\
$\,\,$0.6645$\,\,$ & $\,\,$0.7973$\,\,$ & $\,\,$ 1 $\,\,$ & $\,\,$\color{gr} 6.3784\color{black}  $\,\,$ \\
$\,\,$\color{gr} 0.1042\color{black} $\,\,$ & $\,\,$\color{gr} \color{blue}  1/8\color{black} $\,\,$ & $\,\,$\color{gr} 0.1568\color{black} $\,\,$ & $\,\,$ 1  $\,\,$ \\
\end{pmatrix},
\end{equation*}
\end{example}
\newpage
\begin{example}
\begin{equation*}
\mathbf{A} =
\begin{pmatrix}
$\,\,$ 1 $\,\,$ & $\,\,$2$\,\,$ & $\,\,$2$\,\,$ & $\,\,$3 $\,\,$ \\
$\,\,$ 1/2$\,\,$ & $\,\,$ 1 $\,\,$ & $\,\,$4$\,\,$ & $\,\,$2 $\,\,$ \\
$\,\,$ 1/2$\,\,$ & $\,\,$ 1/4$\,\,$ & $\,\,$ 1 $\,\,$ & $\,\,$1 $\,\,$ \\
$\,\,$ 1/3$\,\,$ & $\,\,$ 1/2$\,\,$ & $\,\,$ 1 $\,\,$ & $\,\,$ 1  $\,\,$ \\
\end{pmatrix},
\qquad
\lambda_{\max} =
4.1707,
\qquad
CR = 0.0644
\end{equation*}

\begin{equation*}
\mathbf{w}^{cos} =
\begin{pmatrix}
0.409351\\
0.314832\\
0.139545\\
\color{red} 0.136272\color{black}
\end{pmatrix}\end{equation*}
\begin{equation*}
\left[ \frac{{w}^{cos}_i}{{w}^{cos}_j} \right] =
\begin{pmatrix}
$\,\,$ 1 $\,\,$ & $\,\,$1.3002$\,\,$ & $\,\,$2.9335$\,\,$ & $\,\,$\color{red} 3.0039\color{black} $\,\,$ \\
$\,\,$0.7691$\,\,$ & $\,\,$ 1 $\,\,$ & $\,\,$2.2561$\,\,$ & $\,\,$\color{red} 2.3103\color{black}   $\,\,$ \\
$\,\,$0.3409$\,\,$ & $\,\,$0.4432$\,\,$ & $\,\,$ 1 $\,\,$ & $\,\,$\color{red} 1.0240\color{black}  $\,\,$ \\
$\,\,$\color{red} 0.3329\color{black} $\,\,$ & $\,\,$\color{red} 0.4328\color{black} $\,\,$ & $\,\,$\color{red} 0.9765\color{black} $\,\,$ & $\,\,$ 1  $\,\,$ \\
\end{pmatrix},
\end{equation*}

\begin{equation*}
\mathbf{w}^{\prime} =
\begin{pmatrix}
0.409278\\
0.314776\\
0.139520\\
0.136426
\end{pmatrix} =
0.999822\cdot
\begin{pmatrix}
0.409351\\
0.314832\\
0.139545\\
\color{gr} 0.136450\color{black}
\end{pmatrix},
\end{equation*}
\begin{equation*}
\left[ \frac{{w}^{\prime}_i}{{w}^{\prime}_j} \right] =
\begin{pmatrix}
$\,\,$ 1 $\,\,$ & $\,\,$1.3002$\,\,$ & $\,\,$2.9335$\,\,$ & $\,\,$\color{gr} \color{blue} 3\color{black} $\,\,$ \\
$\,\,$0.7691$\,\,$ & $\,\,$ 1 $\,\,$ & $\,\,$2.2561$\,\,$ & $\,\,$\color{gr} 2.3073\color{black}   $\,\,$ \\
$\,\,$0.3409$\,\,$ & $\,\,$0.4432$\,\,$ & $\,\,$ 1 $\,\,$ & $\,\,$\color{gr} 1.0227\color{black}  $\,\,$ \\
$\,\,$\color{gr} \color{blue}  1/3\color{black} $\,\,$ & $\,\,$\color{gr} 0.4334\color{black} $\,\,$ & $\,\,$\color{gr} 0.9778\color{black} $\,\,$ & $\,\,$ 1  $\,\,$ \\
\end{pmatrix},
\end{equation*}
\end{example}
\newpage
\begin{example}
\begin{equation*}
\mathbf{A} =
\begin{pmatrix}
$\,\,$ 1 $\,\,$ & $\,\,$2$\,\,$ & $\,\,$2$\,\,$ & $\,\,$3 $\,\,$ \\
$\,\,$ 1/2$\,\,$ & $\,\,$ 1 $\,\,$ & $\,\,$5$\,\,$ & $\,\,$2 $\,\,$ \\
$\,\,$ 1/2$\,\,$ & $\,\,$ 1/5$\,\,$ & $\,\,$ 1 $\,\,$ & $\,\,$1 $\,\,$ \\
$\,\,$ 1/3$\,\,$ & $\,\,$ 1/2$\,\,$ & $\,\,$ 1 $\,\,$ & $\,\,$ 1  $\,\,$ \\
\end{pmatrix},
\qquad
\lambda_{\max} =
4.2394,
\qquad
CR = 0.0903
\end{equation*}

\begin{equation*}
\mathbf{w}^{cos} =
\begin{pmatrix}
0.406241\\
0.326471\\
0.133707\\
\color{red} 0.133581\color{black}
\end{pmatrix}\end{equation*}
\begin{equation*}
\left[ \frac{{w}^{cos}_i}{{w}^{cos}_j} \right] =
\begin{pmatrix}
$\,\,$ 1 $\,\,$ & $\,\,$1.2443$\,\,$ & $\,\,$3.0383$\,\,$ & $\,\,$\color{red} 3.0412\color{black} $\,\,$ \\
$\,\,$0.8036$\,\,$ & $\,\,$ 1 $\,\,$ & $\,\,$2.4417$\,\,$ & $\,\,$\color{red} 2.4440\color{black}   $\,\,$ \\
$\,\,$0.3291$\,\,$ & $\,\,$0.4096$\,\,$ & $\,\,$ 1 $\,\,$ & $\,\,$\color{red} 1.0009\color{black}  $\,\,$ \\
$\,\,$\color{red} 0.3288\color{black} $\,\,$ & $\,\,$\color{red} 0.4092\color{black} $\,\,$ & $\,\,$\color{red} 0.9991\color{black} $\,\,$ & $\,\,$ 1  $\,\,$ \\
\end{pmatrix},
\end{equation*}

\begin{equation*}
\mathbf{w}^{\prime} =
\begin{pmatrix}
0.406189\\
0.326430\\
0.133690\\
0.133690
\end{pmatrix} =
0.999873\cdot
\begin{pmatrix}
0.406241\\
0.326471\\
0.133707\\
\color{gr} 0.133707\color{black}
\end{pmatrix},
\end{equation*}
\begin{equation*}
\left[ \frac{{w}^{\prime}_i}{{w}^{\prime}_j} \right] =
\begin{pmatrix}
$\,\,$ 1 $\,\,$ & $\,\,$1.2443$\,\,$ & $\,\,$3.0383$\,\,$ & $\,\,$\color{gr} 3.0383\color{black} $\,\,$ \\
$\,\,$0.8036$\,\,$ & $\,\,$ 1 $\,\,$ & $\,\,$2.4417$\,\,$ & $\,\,$\color{gr} 2.4417\color{black}   $\,\,$ \\
$\,\,$0.3291$\,\,$ & $\,\,$0.4096$\,\,$ & $\,\,$ 1 $\,\,$ & $\,\,$\color{gr} \color{blue} 1\color{black}  $\,\,$ \\
$\,\,$\color{gr} 0.3291\color{black} $\,\,$ & $\,\,$\color{gr} 0.4096\color{black} $\,\,$ & $\,\,$\color{gr} \color{blue} 1\color{black} $\,\,$ & $\,\,$ 1  $\,\,$ \\
\end{pmatrix},
\end{equation*}
\end{example}
\newpage
\begin{example}
\begin{equation*}
\mathbf{A} =
\begin{pmatrix}
$\,\,$ 1 $\,\,$ & $\,\,$2$\,\,$ & $\,\,$3$\,\,$ & $\,\,$4 $\,\,$ \\
$\,\,$ 1/2$\,\,$ & $\,\,$ 1 $\,\,$ & $\,\,$5$\,\,$ & $\,\,$3 $\,\,$ \\
$\,\,$ 1/3$\,\,$ & $\,\,$ 1/5$\,\,$ & $\,\,$ 1 $\,\,$ & $\,\,$1 $\,\,$ \\
$\,\,$ 1/4$\,\,$ & $\,\,$ 1/3$\,\,$ & $\,\,$ 1 $\,\,$ & $\,\,$ 1  $\,\,$ \\
\end{pmatrix},
\qquad
\lambda_{\max} =
4.1252,
\qquad
CR = 0.0472
\end{equation*}

\begin{equation*}
\mathbf{w}^{cos} =
\begin{pmatrix}
0.446032\\
0.338777\\
0.108423\\
\color{red} 0.106768\color{black}
\end{pmatrix}\end{equation*}
\begin{equation*}
\left[ \frac{{w}^{cos}_i}{{w}^{cos}_j} \right] =
\begin{pmatrix}
$\,\,$ 1 $\,\,$ & $\,\,$1.3166$\,\,$ & $\,\,$4.1138$\,\,$ & $\,\,$\color{red} 4.1776\color{black} $\,\,$ \\
$\,\,$0.7595$\,\,$ & $\,\,$ 1 $\,\,$ & $\,\,$3.1246$\,\,$ & $\,\,$\color{red} 3.1730\color{black}   $\,\,$ \\
$\,\,$0.2431$\,\,$ & $\,\,$0.3200$\,\,$ & $\,\,$ 1 $\,\,$ & $\,\,$\color{red} 1.0155\color{black}  $\,\,$ \\
$\,\,$\color{red} 0.2394\color{black} $\,\,$ & $\,\,$\color{red} 0.3152\color{black} $\,\,$ & $\,\,$\color{red} 0.9847\color{black} $\,\,$ & $\,\,$ 1  $\,\,$ \\
\end{pmatrix},
\end{equation*}

\begin{equation*}
\mathbf{w}^{\prime} =
\begin{pmatrix}
0.445295\\
0.338217\\
0.108244\\
0.108244
\end{pmatrix} =
0.998347\cdot
\begin{pmatrix}
0.446032\\
0.338777\\
0.108423\\
\color{gr} 0.108423\color{black}
\end{pmatrix},
\end{equation*}
\begin{equation*}
\left[ \frac{{w}^{\prime}_i}{{w}^{\prime}_j} \right] =
\begin{pmatrix}
$\,\,$ 1 $\,\,$ & $\,\,$1.3166$\,\,$ & $\,\,$4.1138$\,\,$ & $\,\,$\color{gr} 4.1138\color{black} $\,\,$ \\
$\,\,$0.7595$\,\,$ & $\,\,$ 1 $\,\,$ & $\,\,$3.1246$\,\,$ & $\,\,$\color{gr} 3.1246\color{black}   $\,\,$ \\
$\,\,$0.2431$\,\,$ & $\,\,$0.3200$\,\,$ & $\,\,$ 1 $\,\,$ & $\,\,$\color{gr} \color{blue} 1\color{black}  $\,\,$ \\
$\,\,$\color{gr} 0.2431\color{black} $\,\,$ & $\,\,$\color{gr} 0.3200\color{black} $\,\,$ & $\,\,$\color{gr} \color{blue} 1\color{black} $\,\,$ & $\,\,$ 1  $\,\,$ \\
\end{pmatrix},
\end{equation*}
\end{example}
\newpage
\begin{example}
\begin{equation*}
\mathbf{A} =
\begin{pmatrix}
$\,\,$ 1 $\,\,$ & $\,\,$2$\,\,$ & $\,\,$3$\,\,$ & $\,\,$6 $\,\,$ \\
$\,\,$ 1/2$\,\,$ & $\,\,$ 1 $\,\,$ & $\,\,$2$\,\,$ & $\,\,$2 $\,\,$ \\
$\,\,$ 1/3$\,\,$ & $\,\,$ 1/2$\,\,$ & $\,\,$ 1 $\,\,$ & $\,\,$4 $\,\,$ \\
$\,\,$ 1/6$\,\,$ & $\,\,$ 1/2$\,\,$ & $\,\,$ 1/4$\,\,$ & $\,\,$ 1  $\,\,$ \\
\end{pmatrix},
\qquad
\lambda_{\max} =
4.1707,
\qquad
CR = 0.0644
\end{equation*}

\begin{equation*}
\mathbf{w}^{cos} =
\begin{pmatrix}
\color{red} 0.485340\color{black} \\
0.242800\\
0.190323\\
0.081538
\end{pmatrix}\end{equation*}
\begin{equation*}
\left[ \frac{{w}^{cos}_i}{{w}^{cos}_j} \right] =
\begin{pmatrix}
$\,\,$ 1 $\,\,$ & $\,\,$\color{red} 1.9989\color{black} $\,\,$ & $\,\,$\color{red} 2.5501\color{black} $\,\,$ & $\,\,$\color{red} 5.9523\color{black} $\,\,$ \\
$\,\,$\color{red} 0.5003\color{black} $\,\,$ & $\,\,$ 1 $\,\,$ & $\,\,$1.2757$\,\,$ & $\,\,$2.9778  $\,\,$ \\
$\,\,$\color{red} 0.3921\color{black} $\,\,$ & $\,\,$0.7839$\,\,$ & $\,\,$ 1 $\,\,$ & $\,\,$2.3342 $\,\,$ \\
$\,\,$\color{red} 0.1680\color{black} $\,\,$ & $\,\,$0.3358$\,\,$ & $\,\,$0.4284$\,\,$ & $\,\,$ 1  $\,\,$ \\
\end{pmatrix},
\end{equation*}

\begin{equation*}
\mathbf{w}^{\prime} =
\begin{pmatrix}
0.485473\\
0.242737\\
0.190273\\
0.081517
\end{pmatrix} =
0.999741\cdot
\begin{pmatrix}
\color{gr} 0.485599\color{black} \\
0.242800\\
0.190323\\
0.081538
\end{pmatrix},
\end{equation*}
\begin{equation*}
\left[ \frac{{w}^{\prime}_i}{{w}^{\prime}_j} \right] =
\begin{pmatrix}
$\,\,$ 1 $\,\,$ & $\,\,$\color{gr} \color{blue} 2\color{black} $\,\,$ & $\,\,$\color{gr} 2.5515\color{black} $\,\,$ & $\,\,$\color{gr} 5.9555\color{black} $\,\,$ \\
$\,\,$\color{gr} \color{blue}  1/2\color{black} $\,\,$ & $\,\,$ 1 $\,\,$ & $\,\,$1.2757$\,\,$ & $\,\,$2.9778  $\,\,$ \\
$\,\,$\color{gr} 0.3919\color{black} $\,\,$ & $\,\,$0.7839$\,\,$ & $\,\,$ 1 $\,\,$ & $\,\,$2.3342 $\,\,$ \\
$\,\,$\color{gr} 0.1679\color{black} $\,\,$ & $\,\,$0.3358$\,\,$ & $\,\,$0.4284$\,\,$ & $\,\,$ 1  $\,\,$ \\
\end{pmatrix},
\end{equation*}
\end{example}
\newpage
\begin{example}
\begin{equation*}
\mathbf{A} =
\begin{pmatrix}
$\,\,$ 1 $\,\,$ & $\,\,$2$\,\,$ & $\,\,$3$\,\,$ & $\,\,$8 $\,\,$ \\
$\,\,$ 1/2$\,\,$ & $\,\,$ 1 $\,\,$ & $\,\,$1$\,\,$ & $\,\,$6 $\,\,$ \\
$\,\,$ 1/3$\,\,$ & $\,\,$ 1 $\,\,$ & $\,\,$ 1 $\,\,$ & $\,\,$2 $\,\,$ \\
$\,\,$ 1/8$\,\,$ & $\,\,$ 1/6$\,\,$ & $\,\,$ 1/2$\,\,$ & $\,\,$ 1  $\,\,$ \\
\end{pmatrix},
\qquad
\lambda_{\max} =
4.1031,
\qquad
CR = 0.0389
\end{equation*}

\begin{equation*}
\mathbf{w}^{cos} =
\begin{pmatrix}
\color{red} 0.501433\color{black} \\
0.257635\\
0.177754\\
0.063178
\end{pmatrix}\end{equation*}
\begin{equation*}
\left[ \frac{{w}^{cos}_i}{{w}^{cos}_j} \right] =
\begin{pmatrix}
$\,\,$ 1 $\,\,$ & $\,\,$\color{red} 1.9463\color{black} $\,\,$ & $\,\,$\color{red} 2.8209\color{black} $\,\,$ & $\,\,$\color{red} 7.9368\color{black} $\,\,$ \\
$\,\,$\color{red} 0.5138\color{black} $\,\,$ & $\,\,$ 1 $\,\,$ & $\,\,$1.4494$\,\,$ & $\,\,$4.0779  $\,\,$ \\
$\,\,$\color{red} 0.3545\color{black} $\,\,$ & $\,\,$0.6899$\,\,$ & $\,\,$ 1 $\,\,$ & $\,\,$2.8135 $\,\,$ \\
$\,\,$\color{red} 0.1260\color{black} $\,\,$ & $\,\,$0.2452$\,\,$ & $\,\,$0.3554$\,\,$ & $\,\,$ 1  $\,\,$ \\
\end{pmatrix},
\end{equation*}

\begin{equation*}
\mathbf{w}^{\prime} =
\begin{pmatrix}
0.503416\\
0.256611\\
0.177047\\
0.062927
\end{pmatrix} =
0.996024\cdot
\begin{pmatrix}
\color{gr} 0.505425\color{black} \\
0.257635\\
0.177754\\
0.063178
\end{pmatrix},
\end{equation*}
\begin{equation*}
\left[ \frac{{w}^{\prime}_i}{{w}^{\prime}_j} \right] =
\begin{pmatrix}
$\,\,$ 1 $\,\,$ & $\,\,$\color{gr} 1.9618\color{black} $\,\,$ & $\,\,$\color{gr} 2.8434\color{black} $\,\,$ & $\,\,$\color{gr} \color{blue} 8\color{black} $\,\,$ \\
$\,\,$\color{gr} 0.5097\color{black} $\,\,$ & $\,\,$ 1 $\,\,$ & $\,\,$1.4494$\,\,$ & $\,\,$4.0779  $\,\,$ \\
$\,\,$\color{gr} 0.3517\color{black} $\,\,$ & $\,\,$0.6899$\,\,$ & $\,\,$ 1 $\,\,$ & $\,\,$2.8135 $\,\,$ \\
$\,\,$\color{gr} \color{blue}  1/8\color{black} $\,\,$ & $\,\,$0.2452$\,\,$ & $\,\,$0.3554$\,\,$ & $\,\,$ 1  $\,\,$ \\
\end{pmatrix},
\end{equation*}
\end{example}
\newpage
\begin{example}
\begin{equation*}
\mathbf{A} =
\begin{pmatrix}
$\,\,$ 1 $\,\,$ & $\,\,$2$\,\,$ & $\,\,$3$\,\,$ & $\,\,$9 $\,\,$ \\
$\,\,$ 1/2$\,\,$ & $\,\,$ 1 $\,\,$ & $\,\,$1$\,\,$ & $\,\,$7 $\,\,$ \\
$\,\,$ 1/3$\,\,$ & $\,\,$ 1 $\,\,$ & $\,\,$ 1 $\,\,$ & $\,\,$2 $\,\,$ \\
$\,\,$ 1/9$\,\,$ & $\,\,$ 1/7$\,\,$ & $\,\,$ 1/2$\,\,$ & $\,\,$ 1  $\,\,$ \\
\end{pmatrix},
\qquad
\lambda_{\max} =
4.1342,
\qquad
CR = 0.0506
\end{equation*}

\begin{equation*}
\mathbf{w}^{cos} =
\begin{pmatrix}
\color{red} 0.503940\color{black} \\
0.261950\\
0.175497\\
0.058612
\end{pmatrix}\end{equation*}
\begin{equation*}
\left[ \frac{{w}^{cos}_i}{{w}^{cos}_j} \right] =
\begin{pmatrix}
$\,\,$ 1 $\,\,$ & $\,\,$\color{red} 1.9238\color{black} $\,\,$ & $\,\,$\color{red} 2.8715\color{black} $\,\,$ & $\,\,$\color{red} 8.5978\color{black} $\,\,$ \\
$\,\,$\color{red} 0.5198\color{black} $\,\,$ & $\,\,$ 1 $\,\,$ & $\,\,$1.4926$\,\,$ & $\,\,$4.4692  $\,\,$ \\
$\,\,$\color{red} 0.3483\color{black} $\,\,$ & $\,\,$0.6700$\,\,$ & $\,\,$ 1 $\,\,$ & $\,\,$2.9942 $\,\,$ \\
$\,\,$\color{red} 0.1163\color{black} $\,\,$ & $\,\,$0.2238$\,\,$ & $\,\,$0.3340$\,\,$ & $\,\,$ 1  $\,\,$ \\
\end{pmatrix},
\end{equation*}

\begin{equation*}
\mathbf{w}^{\prime} =
\begin{pmatrix}
0.513648\\
0.256824\\
0.172063\\
0.057465
\end{pmatrix} =
0.980430\cdot
\begin{pmatrix}
\color{gr} 0.523900\color{black} \\
0.261950\\
0.175497\\
0.058612
\end{pmatrix},
\end{equation*}
\begin{equation*}
\left[ \frac{{w}^{\prime}_i}{{w}^{\prime}_j} \right] =
\begin{pmatrix}
$\,\,$ 1 $\,\,$ & $\,\,$\color{gr} \color{blue} 2\color{black} $\,\,$ & $\,\,$\color{gr} 2.9852\color{black} $\,\,$ & $\,\,$\color{gr} 8.9384\color{black} $\,\,$ \\
$\,\,$\color{gr} \color{blue}  1/2\color{black} $\,\,$ & $\,\,$ 1 $\,\,$ & $\,\,$1.4926$\,\,$ & $\,\,$4.4692  $\,\,$ \\
$\,\,$\color{gr} 0.3350\color{black} $\,\,$ & $\,\,$0.6700$\,\,$ & $\,\,$ 1 $\,\,$ & $\,\,$2.9942 $\,\,$ \\
$\,\,$\color{gr} 0.1119\color{black} $\,\,$ & $\,\,$0.2238$\,\,$ & $\,\,$0.3340$\,\,$ & $\,\,$ 1  $\,\,$ \\
\end{pmatrix},
\end{equation*}
\end{example}
\newpage
\begin{example}
\begin{equation*}
\mathbf{A} =
\begin{pmatrix}
$\,\,$ 1 $\,\,$ & $\,\,$2$\,\,$ & $\,\,$3$\,\,$ & $\,\,$9 $\,\,$ \\
$\,\,$ 1/2$\,\,$ & $\,\,$ 1 $\,\,$ & $\,\,$1$\,\,$ & $\,\,$8 $\,\,$ \\
$\,\,$ 1/3$\,\,$ & $\,\,$ 1 $\,\,$ & $\,\,$ 1 $\,\,$ & $\,\,$2 $\,\,$ \\
$\,\,$ 1/9$\,\,$ & $\,\,$ 1/8$\,\,$ & $\,\,$ 1/2$\,\,$ & $\,\,$ 1  $\,\,$ \\
\end{pmatrix},
\qquad
\lambda_{\max} =
4.1664,
\qquad
CR = 0.0627
\end{equation*}

\begin{equation*}
\mathbf{w}^{cos} =
\begin{pmatrix}
\color{red} 0.498655\color{black} \\
0.269997\\
0.174422\\
0.056926
\end{pmatrix}\end{equation*}
\begin{equation*}
\left[ \frac{{w}^{cos}_i}{{w}^{cos}_j} \right] =
\begin{pmatrix}
$\,\,$ 1 $\,\,$ & $\,\,$\color{red} 1.8469\color{black} $\,\,$ & $\,\,$\color{red} 2.8589\color{black} $\,\,$ & $\,\,$\color{red} 8.7597\color{black} $\,\,$ \\
$\,\,$\color{red} 0.5415\color{black} $\,\,$ & $\,\,$ 1 $\,\,$ & $\,\,$1.5480$\,\,$ & $\,\,$4.7430  $\,\,$ \\
$\,\,$\color{red} 0.3498\color{black} $\,\,$ & $\,\,$0.6460$\,\,$ & $\,\,$ 1 $\,\,$ & $\,\,$3.0640 $\,\,$ \\
$\,\,$\color{red} 0.1142\color{black} $\,\,$ & $\,\,$0.2108$\,\,$ & $\,\,$0.3264$\,\,$ & $\,\,$ 1  $\,\,$ \\
\end{pmatrix},
\end{equation*}

\begin{equation*}
\mathbf{w}^{\prime} =
\begin{pmatrix}
0.505420\\
0.266354\\
0.172068\\
0.056158
\end{pmatrix} =
0.986507\cdot
\begin{pmatrix}
\color{gr} 0.512333\color{black} \\
0.269997\\
0.174422\\
0.056926
\end{pmatrix},
\end{equation*}
\begin{equation*}
\left[ \frac{{w}^{\prime}_i}{{w}^{\prime}_j} \right] =
\begin{pmatrix}
$\,\,$ 1 $\,\,$ & $\,\,$\color{gr} 1.8976\color{black} $\,\,$ & $\,\,$\color{gr} 2.9373\color{black} $\,\,$ & $\,\,$\color{gr} \color{blue} 9\color{black} $\,\,$ \\
$\,\,$\color{gr} 0.5270\color{black} $\,\,$ & $\,\,$ 1 $\,\,$ & $\,\,$1.5480$\,\,$ & $\,\,$4.7430  $\,\,$ \\
$\,\,$\color{gr} 0.3404\color{black} $\,\,$ & $\,\,$0.6460$\,\,$ & $\,\,$ 1 $\,\,$ & $\,\,$3.0640 $\,\,$ \\
$\,\,$\color{gr} \color{blue}  1/9\color{black} $\,\,$ & $\,\,$0.2108$\,\,$ & $\,\,$0.3264$\,\,$ & $\,\,$ 1  $\,\,$ \\
\end{pmatrix},
\end{equation*}
\end{example}
\newpage
\begin{example}
\begin{equation*}
\mathbf{A} =
\begin{pmatrix}
$\,\,$ 1 $\,\,$ & $\,\,$2$\,\,$ & $\,\,$3$\,\,$ & $\,\,$9 $\,\,$ \\
$\,\,$ 1/2$\,\,$ & $\,\,$ 1 $\,\,$ & $\,\,$1$\,\,$ & $\,\,$9 $\,\,$ \\
$\,\,$ 1/3$\,\,$ & $\,\,$ 1 $\,\,$ & $\,\,$ 1 $\,\,$ & $\,\,$2 $\,\,$ \\
$\,\,$ 1/9$\,\,$ & $\,\,$ 1/9$\,\,$ & $\,\,$ 1/2$\,\,$ & $\,\,$ 1  $\,\,$ \\
\end{pmatrix},
\qquad
\lambda_{\max} =
4.1990,
\qquad
CR = 0.0750
\end{equation*}

\begin{equation*}
\mathbf{w}^{cos} =
\begin{pmatrix}
\color{red} 0.493865\color{black} \\
0.277104\\
0.173494\\
0.055537
\end{pmatrix}\end{equation*}
\begin{equation*}
\left[ \frac{{w}^{cos}_i}{{w}^{cos}_j} \right] =
\begin{pmatrix}
$\,\,$ 1 $\,\,$ & $\,\,$\color{red} 1.7822\color{black} $\,\,$ & $\,\,$\color{red} 2.8466\color{black} $\,\,$ & $\,\,$\color{red} 8.8926\color{black} $\,\,$ \\
$\,\,$\color{red} 0.5611\color{black} $\,\,$ & $\,\,$ 1 $\,\,$ & $\,\,$1.5972$\,\,$ & $\,\,$4.9895  $\,\,$ \\
$\,\,$\color{red} 0.3513\color{black} $\,\,$ & $\,\,$0.6261$\,\,$ & $\,\,$ 1 $\,\,$ & $\,\,$3.1239 $\,\,$ \\
$\,\,$\color{red} 0.1125\color{black} $\,\,$ & $\,\,$0.2004$\,\,$ & $\,\,$0.3201$\,\,$ & $\,\,$ 1  $\,\,$ \\
\end{pmatrix},
\end{equation*}

\begin{equation*}
\mathbf{w}^{\prime} =
\begin{pmatrix}
0.496867\\
0.275460\\
0.172465\\
0.055207
\end{pmatrix} =
0.994069\cdot
\begin{pmatrix}
\color{gr} 0.499832\color{black} \\
0.277104\\
0.173494\\
0.055537
\end{pmatrix},
\end{equation*}
\begin{equation*}
\left[ \frac{{w}^{\prime}_i}{{w}^{\prime}_j} \right] =
\begin{pmatrix}
$\,\,$ 1 $\,\,$ & $\,\,$\color{gr} 1.8038\color{black} $\,\,$ & $\,\,$\color{gr} 2.8810\color{black} $\,\,$ & $\,\,$\color{gr} \color{blue} 9\color{black} $\,\,$ \\
$\,\,$\color{gr} 0.5544\color{black} $\,\,$ & $\,\,$ 1 $\,\,$ & $\,\,$1.5972$\,\,$ & $\,\,$4.9895  $\,\,$ \\
$\,\,$\color{gr} 0.3471\color{black} $\,\,$ & $\,\,$0.6261$\,\,$ & $\,\,$ 1 $\,\,$ & $\,\,$3.1239 $\,\,$ \\
$\,\,$\color{gr} \color{blue}  1/9\color{black} $\,\,$ & $\,\,$0.2004$\,\,$ & $\,\,$0.3201$\,\,$ & $\,\,$ 1  $\,\,$ \\
\end{pmatrix},
\end{equation*}
\end{example}
\newpage
\begin{example}
\begin{equation*}
\mathbf{A} =
\begin{pmatrix}
$\,\,$ 1 $\,\,$ & $\,\,$2$\,\,$ & $\,\,$4$\,\,$ & $\,\,$3 $\,\,$ \\
$\,\,$ 1/2$\,\,$ & $\,\,$ 1 $\,\,$ & $\,\,$4$\,\,$ & $\,\,$8 $\,\,$ \\
$\,\,$ 1/4$\,\,$ & $\,\,$ 1/4$\,\,$ & $\,\,$ 1 $\,\,$ & $\,\,$1 $\,\,$ \\
$\,\,$ 1/3$\,\,$ & $\,\,$ 1/8$\,\,$ & $\,\,$ 1 $\,\,$ & $\,\,$ 1  $\,\,$ \\
\end{pmatrix},
\qquad
\lambda_{\max} =
4.2512,
\qquad
CR = 0.0947
\end{equation*}

\begin{equation*}
\mathbf{w}^{cos} =
\begin{pmatrix}
0.426950\\
0.382759\\
\color{red} 0.094033\color{black} \\
0.096258
\end{pmatrix}\end{equation*}
\begin{equation*}
\left[ \frac{{w}^{cos}_i}{{w}^{cos}_j} \right] =
\begin{pmatrix}
$\,\,$ 1 $\,\,$ & $\,\,$1.1155$\,\,$ & $\,\,$\color{red} 4.5405\color{black} $\,\,$ & $\,\,$4.4355$\,\,$ \\
$\,\,$0.8965$\,\,$ & $\,\,$ 1 $\,\,$ & $\,\,$\color{red} 4.0705\color{black} $\,\,$ & $\,\,$3.9764  $\,\,$ \\
$\,\,$\color{red} 0.2202\color{black} $\,\,$ & $\,\,$\color{red} 0.2457\color{black} $\,\,$ & $\,\,$ 1 $\,\,$ & $\,\,$\color{red} 0.9769\color{black}  $\,\,$ \\
$\,\,$0.2255$\,\,$ & $\,\,$0.2515$\,\,$ & $\,\,$\color{red} 1.0237\color{black} $\,\,$ & $\,\,$ 1  $\,\,$ \\
\end{pmatrix},
\end{equation*}

\begin{equation*}
\mathbf{w}^{\prime} =
\begin{pmatrix}
0.426244\\
0.382126\\
0.095532\\
0.096098
\end{pmatrix} =
0.998346\cdot
\begin{pmatrix}
0.426950\\
0.382759\\
\color{gr} 0.095690\color{black} \\
0.096258
\end{pmatrix},
\end{equation*}
\begin{equation*}
\left[ \frac{{w}^{\prime}_i}{{w}^{\prime}_j} \right] =
\begin{pmatrix}
$\,\,$ 1 $\,\,$ & $\,\,$1.1155$\,\,$ & $\,\,$\color{gr} 4.4618\color{black} $\,\,$ & $\,\,$4.4355$\,\,$ \\
$\,\,$0.8965$\,\,$ & $\,\,$ 1 $\,\,$ & $\,\,$\color{gr} \color{blue} 4\color{black} $\,\,$ & $\,\,$3.9764  $\,\,$ \\
$\,\,$\color{gr} 0.2241\color{black} $\,\,$ & $\,\,$\color{gr} \color{blue}  1/4\color{black} $\,\,$ & $\,\,$ 1 $\,\,$ & $\,\,$\color{gr} 0.9941\color{black}  $\,\,$ \\
$\,\,$0.2255$\,\,$ & $\,\,$0.2515$\,\,$ & $\,\,$\color{gr} 1.0059\color{black} $\,\,$ & $\,\,$ 1  $\,\,$ \\
\end{pmatrix},
\end{equation*}
\end{example}
\newpage
\begin{example}
\begin{equation*}
\mathbf{A} =
\begin{pmatrix}
$\,\,$ 1 $\,\,$ & $\,\,$2$\,\,$ & $\,\,$4$\,\,$ & $\,\,$5 $\,\,$ \\
$\,\,$ 1/2$\,\,$ & $\,\,$ 1 $\,\,$ & $\,\,$6$\,\,$ & $\,\,$4 $\,\,$ \\
$\,\,$ 1/4$\,\,$ & $\,\,$ 1/6$\,\,$ & $\,\,$ 1 $\,\,$ & $\,\,$1 $\,\,$ \\
$\,\,$ 1/5$\,\,$ & $\,\,$ 1/4$\,\,$ & $\,\,$ 1 $\,\,$ & $\,\,$ 1  $\,\,$ \\
\end{pmatrix},
\qquad
\lambda_{\max} =
4.1046,
\qquad
CR = 0.0395
\end{equation*}

\begin{equation*}
\mathbf{w}^{cos} =
\begin{pmatrix}
0.469601\\
0.353707\\
0.088812\\
\color{red} 0.087880\color{black}
\end{pmatrix}\end{equation*}
\begin{equation*}
\left[ \frac{{w}^{cos}_i}{{w}^{cos}_j} \right] =
\begin{pmatrix}
$\,\,$ 1 $\,\,$ & $\,\,$1.3277$\,\,$ & $\,\,$5.2876$\,\,$ & $\,\,$\color{red} 5.3436\color{black} $\,\,$ \\
$\,\,$0.7532$\,\,$ & $\,\,$ 1 $\,\,$ & $\,\,$3.9826$\,\,$ & $\,\,$\color{red} 4.0249\color{black}   $\,\,$ \\
$\,\,$0.1891$\,\,$ & $\,\,$0.2511$\,\,$ & $\,\,$ 1 $\,\,$ & $\,\,$\color{red} 1.0106\color{black}  $\,\,$ \\
$\,\,$\color{red} 0.1871\color{black} $\,\,$ & $\,\,$\color{red} 0.2485\color{black} $\,\,$ & $\,\,$\color{red} 0.9895\color{black} $\,\,$ & $\,\,$ 1  $\,\,$ \\
\end{pmatrix},
\end{equation*}

\begin{equation*}
\mathbf{w}^{\prime} =
\begin{pmatrix}
0.469344\\
0.353513\\
0.088764\\
0.088378
\end{pmatrix} =
0.999454\cdot
\begin{pmatrix}
0.469601\\
0.353707\\
0.088812\\
\color{gr} 0.088427\color{black}
\end{pmatrix},
\end{equation*}
\begin{equation*}
\left[ \frac{{w}^{\prime}_i}{{w}^{\prime}_j} \right] =
\begin{pmatrix}
$\,\,$ 1 $\,\,$ & $\,\,$1.3277$\,\,$ & $\,\,$5.2876$\,\,$ & $\,\,$\color{gr} 5.3106\color{black} $\,\,$ \\
$\,\,$0.7532$\,\,$ & $\,\,$ 1 $\,\,$ & $\,\,$3.9826$\,\,$ & $\,\,$\color{gr} \color{blue} 4\color{black}   $\,\,$ \\
$\,\,$0.1891$\,\,$ & $\,\,$0.2511$\,\,$ & $\,\,$ 1 $\,\,$ & $\,\,$\color{gr} 1.0044\color{black}  $\,\,$ \\
$\,\,$\color{gr} 0.1883\color{black} $\,\,$ & $\,\,$\color{gr} \color{blue}  1/4\color{black} $\,\,$ & $\,\,$\color{gr} 0.9957\color{black} $\,\,$ & $\,\,$ 1  $\,\,$ \\
\end{pmatrix},
\end{equation*}
\end{example}
\newpage
\begin{example}
\begin{equation*}
\mathbf{A} =
\begin{pmatrix}
$\,\,$ 1 $\,\,$ & $\,\,$2$\,\,$ & $\,\,$4$\,\,$ & $\,\,$6 $\,\,$ \\
$\,\,$ 1/2$\,\,$ & $\,\,$ 1 $\,\,$ & $\,\,$3$\,\,$ & $\,\,$2 $\,\,$ \\
$\,\,$ 1/4$\,\,$ & $\,\,$ 1/3$\,\,$ & $\,\,$ 1 $\,\,$ & $\,\,$2 $\,\,$ \\
$\,\,$ 1/6$\,\,$ & $\,\,$ 1/2$\,\,$ & $\,\,$ 1/2$\,\,$ & $\,\,$ 1  $\,\,$ \\
\end{pmatrix},
\qquad
\lambda_{\max} =
4.1031,
\qquad
CR = 0.0389
\end{equation*}

\begin{equation*}
\mathbf{w}^{cos} =
\begin{pmatrix}
\color{red} 0.514784\color{black} \\
0.264368\\
0.129082\\
0.091766
\end{pmatrix}\end{equation*}
\begin{equation*}
\left[ \frac{{w}^{cos}_i}{{w}^{cos}_j} \right] =
\begin{pmatrix}
$\,\,$ 1 $\,\,$ & $\,\,$\color{red} 1.9472\color{black} $\,\,$ & $\,\,$\color{red} 3.9880\color{black} $\,\,$ & $\,\,$\color{red} 5.6098\color{black} $\,\,$ \\
$\,\,$\color{red} 0.5136\color{black} $\,\,$ & $\,\,$ 1 $\,\,$ & $\,\,$2.0481$\,\,$ & $\,\,$2.8809  $\,\,$ \\
$\,\,$\color{red} 0.2508\color{black} $\,\,$ & $\,\,$0.4883$\,\,$ & $\,\,$ 1 $\,\,$ & $\,\,$1.4066 $\,\,$ \\
$\,\,$\color{red} 0.1783\color{black} $\,\,$ & $\,\,$0.3471$\,\,$ & $\,\,$0.7109$\,\,$ & $\,\,$ 1  $\,\,$ \\
\end{pmatrix},
\end{equation*}

\begin{equation*}
\mathbf{w}^{\prime} =
\begin{pmatrix}
0.515533\\
0.263960\\
0.128883\\
0.091624
\end{pmatrix} =
0.998457\cdot
\begin{pmatrix}
\color{gr} 0.516330\color{black} \\
0.264368\\
0.129082\\
0.091766
\end{pmatrix},
\end{equation*}
\begin{equation*}
\left[ \frac{{w}^{\prime}_i}{{w}^{\prime}_j} \right] =
\begin{pmatrix}
$\,\,$ 1 $\,\,$ & $\,\,$\color{gr} 1.9531\color{black} $\,\,$ & $\,\,$\color{gr} \color{blue} 4\color{black} $\,\,$ & $\,\,$\color{gr} 5.6266\color{black} $\,\,$ \\
$\,\,$\color{gr} 0.5120\color{black} $\,\,$ & $\,\,$ 1 $\,\,$ & $\,\,$2.0481$\,\,$ & $\,\,$2.8809  $\,\,$ \\
$\,\,$\color{gr} \color{blue}  1/4\color{black} $\,\,$ & $\,\,$0.4883$\,\,$ & $\,\,$ 1 $\,\,$ & $\,\,$1.4066 $\,\,$ \\
$\,\,$\color{gr} 0.1777\color{black} $\,\,$ & $\,\,$0.3471$\,\,$ & $\,\,$0.7109$\,\,$ & $\,\,$ 1  $\,\,$ \\
\end{pmatrix},
\end{equation*}
\end{example}
\newpage
\begin{example}
\begin{equation*}
\mathbf{A} =
\begin{pmatrix}
$\,\,$ 1 $\,\,$ & $\,\,$2$\,\,$ & $\,\,$4$\,\,$ & $\,\,$6 $\,\,$ \\
$\,\,$ 1/2$\,\,$ & $\,\,$ 1 $\,\,$ & $\,\,$3$\,\,$ & $\,\,$2 $\,\,$ \\
$\,\,$ 1/4$\,\,$ & $\,\,$ 1/3$\,\,$ & $\,\,$ 1 $\,\,$ & $\,\,$3 $\,\,$ \\
$\,\,$ 1/6$\,\,$ & $\,\,$ 1/2$\,\,$ & $\,\,$ 1/3$\,\,$ & $\,\,$ 1  $\,\,$ \\
\end{pmatrix},
\qquad
\lambda_{\max} =
4.1990,
\qquad
CR = 0.0750
\end{equation*}

\begin{equation*}
\mathbf{w}^{cos} =
\begin{pmatrix}
\color{red} 0.505929\color{black} \\
0.261140\\
0.147592\\
0.085338
\end{pmatrix}\end{equation*}
\begin{equation*}
\left[ \frac{{w}^{cos}_i}{{w}^{cos}_j} \right] =
\begin{pmatrix}
$\,\,$ 1 $\,\,$ & $\,\,$\color{red} 1.9374\color{black} $\,\,$ & $\,\,$\color{red} 3.4279\color{black} $\,\,$ & $\,\,$\color{red} 5.9285\color{black} $\,\,$ \\
$\,\,$\color{red} 0.5162\color{black} $\,\,$ & $\,\,$ 1 $\,\,$ & $\,\,$1.7693$\,\,$ & $\,\,$3.0601  $\,\,$ \\
$\,\,$\color{red} 0.2917\color{black} $\,\,$ & $\,\,$0.5652$\,\,$ & $\,\,$ 1 $\,\,$ & $\,\,$1.7295 $\,\,$ \\
$\,\,$\color{red} 0.1687\color{black} $\,\,$ & $\,\,$0.3268$\,\,$ & $\,\,$0.5782$\,\,$ & $\,\,$ 1  $\,\,$ \\
\end{pmatrix},
\end{equation*}

\begin{equation*}
\mathbf{w}^{\prime} =
\begin{pmatrix}
0.508925\\
0.259557\\
0.146698\\
0.084821
\end{pmatrix} =
0.993938\cdot
\begin{pmatrix}
\color{gr} 0.512029\color{black} \\
0.261140\\
0.147592\\
0.085338
\end{pmatrix},
\end{equation*}
\begin{equation*}
\left[ \frac{{w}^{\prime}_i}{{w}^{\prime}_j} \right] =
\begin{pmatrix}
$\,\,$ 1 $\,\,$ & $\,\,$\color{gr} 1.9607\color{black} $\,\,$ & $\,\,$\color{gr} 3.4692\color{black} $\,\,$ & $\,\,$\color{gr} \color{blue} 6\color{black} $\,\,$ \\
$\,\,$\color{gr} 0.5100\color{black} $\,\,$ & $\,\,$ 1 $\,\,$ & $\,\,$1.7693$\,\,$ & $\,\,$3.0601  $\,\,$ \\
$\,\,$\color{gr} 0.2883\color{black} $\,\,$ & $\,\,$0.5652$\,\,$ & $\,\,$ 1 $\,\,$ & $\,\,$1.7295 $\,\,$ \\
$\,\,$\color{gr} \color{blue}  1/6\color{black} $\,\,$ & $\,\,$0.3268$\,\,$ & $\,\,$0.5782$\,\,$ & $\,\,$ 1  $\,\,$ \\
\end{pmatrix},
\end{equation*}
\end{example}
\newpage
\begin{example}
\begin{equation*}
\mathbf{A} =
\begin{pmatrix}
$\,\,$ 1 $\,\,$ & $\,\,$2$\,\,$ & $\,\,$4$\,\,$ & $\,\,$6 $\,\,$ \\
$\,\,$ 1/2$\,\,$ & $\,\,$ 1 $\,\,$ & $\,\,$8$\,\,$ & $\,\,$4 $\,\,$ \\
$\,\,$ 1/4$\,\,$ & $\,\,$ 1/8$\,\,$ & $\,\,$ 1 $\,\,$ & $\,\,$1 $\,\,$ \\
$\,\,$ 1/6$\,\,$ & $\,\,$ 1/4$\,\,$ & $\,\,$ 1 $\,\,$ & $\,\,$ 1  $\,\,$ \\
\end{pmatrix},
\qquad
\lambda_{\max} =
4.1707,
\qquad
CR = 0.0644
\end{equation*}

\begin{equation*}
\mathbf{w}^{cos} =
\begin{pmatrix}
0.475265\\
0.363759\\
0.081792\\
\color{red} 0.079184\color{black}
\end{pmatrix}\end{equation*}
\begin{equation*}
\left[ \frac{{w}^{cos}_i}{{w}^{cos}_j} \right] =
\begin{pmatrix}
$\,\,$ 1 $\,\,$ & $\,\,$1.3065$\,\,$ & $\,\,$5.8107$\,\,$ & $\,\,$\color{red} 6.0020\color{black} $\,\,$ \\
$\,\,$0.7654$\,\,$ & $\,\,$ 1 $\,\,$ & $\,\,$4.4474$\,\,$ & $\,\,$\color{red} 4.5938\color{black}   $\,\,$ \\
$\,\,$0.1721$\,\,$ & $\,\,$0.2249$\,\,$ & $\,\,$ 1 $\,\,$ & $\,\,$\color{red} 1.0329\color{black}  $\,\,$ \\
$\,\,$\color{red} 0.1666\color{black} $\,\,$ & $\,\,$\color{red} 0.2177\color{black} $\,\,$ & $\,\,$\color{red} 0.9681\color{black} $\,\,$ & $\,\,$ 1  $\,\,$ \\
\end{pmatrix},
\end{equation*}

\begin{equation*}
\mathbf{w}^{\prime} =
\begin{pmatrix}
0.475252\\
0.363749\\
0.081789\\
0.079209
\end{pmatrix} =
0.999973\cdot
\begin{pmatrix}
0.475265\\
0.363759\\
0.081792\\
\color{gr} 0.079211\color{black}
\end{pmatrix},
\end{equation*}
\begin{equation*}
\left[ \frac{{w}^{\prime}_i}{{w}^{\prime}_j} \right] =
\begin{pmatrix}
$\,\,$ 1 $\,\,$ & $\,\,$1.3065$\,\,$ & $\,\,$5.8107$\,\,$ & $\,\,$\color{gr} \color{blue} 6\color{black} $\,\,$ \\
$\,\,$0.7654$\,\,$ & $\,\,$ 1 $\,\,$ & $\,\,$4.4474$\,\,$ & $\,\,$\color{gr} 4.5923\color{black}   $\,\,$ \\
$\,\,$0.1721$\,\,$ & $\,\,$0.2249$\,\,$ & $\,\,$ 1 $\,\,$ & $\,\,$\color{gr} 1.0326\color{black}  $\,\,$ \\
$\,\,$\color{gr} \color{blue}  1/6\color{black} $\,\,$ & $\,\,$\color{gr} 0.2178\color{black} $\,\,$ & $\,\,$\color{gr} 0.9684\color{black} $\,\,$ & $\,\,$ 1  $\,\,$ \\
\end{pmatrix},
\end{equation*}
\end{example}
\newpage
\begin{example}
\begin{equation*}
\mathbf{A} =
\begin{pmatrix}
$\,\,$ 1 $\,\,$ & $\,\,$2$\,\,$ & $\,\,$4$\,\,$ & $\,\,$6 $\,\,$ \\
$\,\,$ 1/2$\,\,$ & $\,\,$ 1 $\,\,$ & $\,\,$9$\,\,$ & $\,\,$4 $\,\,$ \\
$\,\,$ 1/4$\,\,$ & $\,\,$ 1/9$\,\,$ & $\,\,$ 1 $\,\,$ & $\,\,$1 $\,\,$ \\
$\,\,$ 1/6$\,\,$ & $\,\,$ 1/4$\,\,$ & $\,\,$ 1 $\,\,$ & $\,\,$ 1  $\,\,$ \\
\end{pmatrix},
\qquad
\lambda_{\max} =
4.2052,
\qquad
CR = 0.0774
\end{equation*}

\begin{equation*}
\mathbf{w}^{cos} =
\begin{pmatrix}
0.472334\\
0.369690\\
0.079810\\
\color{red} 0.078167\color{black}
\end{pmatrix}\end{equation*}
\begin{equation*}
\left[ \frac{{w}^{cos}_i}{{w}^{cos}_j} \right] =
\begin{pmatrix}
$\,\,$ 1 $\,\,$ & $\,\,$1.2776$\,\,$ & $\,\,$5.9183$\,\,$ & $\,\,$\color{red} 6.0426\color{black} $\,\,$ \\
$\,\,$0.7827$\,\,$ & $\,\,$ 1 $\,\,$ & $\,\,$4.6321$\,\,$ & $\,\,$\color{red} 4.7295\color{black}   $\,\,$ \\
$\,\,$0.1690$\,\,$ & $\,\,$0.2159$\,\,$ & $\,\,$ 1 $\,\,$ & $\,\,$\color{red} 1.0210\color{black}  $\,\,$ \\
$\,\,$\color{red} 0.1655\color{black} $\,\,$ & $\,\,$\color{red} 0.2114\color{black} $\,\,$ & $\,\,$\color{red} 0.9794\color{black} $\,\,$ & $\,\,$ 1  $\,\,$ \\
\end{pmatrix},
\end{equation*}

\begin{equation*}
\mathbf{w}^{\prime} =
\begin{pmatrix}
0.472072\\
0.369484\\
0.079765\\
0.078679
\end{pmatrix} =
0.999445\cdot
\begin{pmatrix}
0.472334\\
0.369690\\
0.079810\\
\color{gr} 0.078722\color{black}
\end{pmatrix},
\end{equation*}
\begin{equation*}
\left[ \frac{{w}^{\prime}_i}{{w}^{\prime}_j} \right] =
\begin{pmatrix}
$\,\,$ 1 $\,\,$ & $\,\,$1.2776$\,\,$ & $\,\,$5.9183$\,\,$ & $\,\,$\color{gr} \color{blue} 6\color{black} $\,\,$ \\
$\,\,$0.7827$\,\,$ & $\,\,$ 1 $\,\,$ & $\,\,$4.6321$\,\,$ & $\,\,$\color{gr} 4.6961\color{black}   $\,\,$ \\
$\,\,$0.1690$\,\,$ & $\,\,$0.2159$\,\,$ & $\,\,$ 1 $\,\,$ & $\,\,$\color{gr} 1.0138\color{black}  $\,\,$ \\
$\,\,$\color{gr} \color{blue}  1/6\color{black} $\,\,$ & $\,\,$\color{gr} 0.2129\color{black} $\,\,$ & $\,\,$\color{gr} 0.9864\color{black} $\,\,$ & $\,\,$ 1  $\,\,$ \\
\end{pmatrix},
\end{equation*}
\end{example}
\newpage
\begin{example}
\begin{equation*}
\mathbf{A} =
\begin{pmatrix}
$\,\,$ 1 $\,\,$ & $\,\,$2$\,\,$ & $\,\,$4$\,\,$ & $\,\,$7 $\,\,$ \\
$\,\,$ 1/2$\,\,$ & $\,\,$ 1 $\,\,$ & $\,\,$3$\,\,$ & $\,\,$2 $\,\,$ \\
$\,\,$ 1/4$\,\,$ & $\,\,$ 1/3$\,\,$ & $\,\,$ 1 $\,\,$ & $\,\,$3 $\,\,$ \\
$\,\,$ 1/7$\,\,$ & $\,\,$ 1/2$\,\,$ & $\,\,$ 1/3$\,\,$ & $\,\,$ 1  $\,\,$ \\
\end{pmatrix},
\qquad
\lambda_{\max} =
4.1964,
\qquad
CR = 0.0741
\end{equation*}

\begin{equation*}
\mathbf{w}^{cos} =
\begin{pmatrix}
\color{red} 0.517181\color{black} \\
0.259614\\
0.142322\\
0.080884
\end{pmatrix}\end{equation*}
\begin{equation*}
\left[ \frac{{w}^{cos}_i}{{w}^{cos}_j} \right] =
\begin{pmatrix}
$\,\,$ 1 $\,\,$ & $\,\,$\color{red} 1.9921\color{black} $\,\,$ & $\,\,$\color{red} 3.6339\color{black} $\,\,$ & $\,\,$\color{red} 6.3941\color{black} $\,\,$ \\
$\,\,$\color{red} 0.5020\color{black} $\,\,$ & $\,\,$ 1 $\,\,$ & $\,\,$1.8241$\,\,$ & $\,\,$3.2097  $\,\,$ \\
$\,\,$\color{red} 0.2752\color{black} $\,\,$ & $\,\,$0.5482$\,\,$ & $\,\,$ 1 $\,\,$ & $\,\,$1.7596 $\,\,$ \\
$\,\,$\color{red} 0.1564\color{black} $\,\,$ & $\,\,$0.3116$\,\,$ & $\,\,$0.5683$\,\,$ & $\,\,$ 1  $\,\,$ \\
\end{pmatrix},
\end{equation*}

\begin{equation*}
\mathbf{w}^{\prime} =
\begin{pmatrix}
0.518167\\
0.259083\\
0.142031\\
0.080719
\end{pmatrix} =
0.997958\cdot
\begin{pmatrix}
\color{gr} 0.519227\color{black} \\
0.259614\\
0.142322\\
0.080884
\end{pmatrix},
\end{equation*}
\begin{equation*}
\left[ \frac{{w}^{\prime}_i}{{w}^{\prime}_j} \right] =
\begin{pmatrix}
$\,\,$ 1 $\,\,$ & $\,\,$\color{gr} \color{blue} 2\color{black} $\,\,$ & $\,\,$\color{gr} 3.6483\color{black} $\,\,$ & $\,\,$\color{gr} 6.4194\color{black} $\,\,$ \\
$\,\,$\color{gr} \color{blue}  1/2\color{black} $\,\,$ & $\,\,$ 1 $\,\,$ & $\,\,$1.8241$\,\,$ & $\,\,$3.2097  $\,\,$ \\
$\,\,$\color{gr} 0.2741\color{black} $\,\,$ & $\,\,$0.5482$\,\,$ & $\,\,$ 1 $\,\,$ & $\,\,$1.7596 $\,\,$ \\
$\,\,$\color{gr} 0.1558\color{black} $\,\,$ & $\,\,$0.3116$\,\,$ & $\,\,$0.5683$\,\,$ & $\,\,$ 1  $\,\,$ \\
\end{pmatrix},
\end{equation*}
\end{example}
\newpage
\begin{example}
\begin{equation*}
\mathbf{A} =
\begin{pmatrix}
$\,\,$ 1 $\,\,$ & $\,\,$2$\,\,$ & $\,\,$5$\,\,$ & $\,\,$3 $\,\,$ \\
$\,\,$ 1/2$\,\,$ & $\,\,$ 1 $\,\,$ & $\,\,$4$\,\,$ & $\,\,$6 $\,\,$ \\
$\,\,$ 1/5$\,\,$ & $\,\,$ 1/4$\,\,$ & $\,\,$ 1 $\,\,$ & $\,\,$1 $\,\,$ \\
$\,\,$ 1/3$\,\,$ & $\,\,$ 1/6$\,\,$ & $\,\,$ 1 $\,\,$ & $\,\,$ 1  $\,\,$ \\
\end{pmatrix},
\qquad
\lambda_{\max} =
4.1655,
\qquad
CR = 0.0624
\end{equation*}

\begin{equation*}
\mathbf{w}^{cos} =
\begin{pmatrix}
0.449999\\
0.360879\\
\color{red} 0.088750\color{black} \\
0.100372
\end{pmatrix}\end{equation*}
\begin{equation*}
\left[ \frac{{w}^{cos}_i}{{w}^{cos}_j} \right] =
\begin{pmatrix}
$\,\,$ 1 $\,\,$ & $\,\,$1.2470$\,\,$ & $\,\,$\color{red} 5.0704\color{black} $\,\,$ & $\,\,$4.4833$\,\,$ \\
$\,\,$0.8020$\,\,$ & $\,\,$ 1 $\,\,$ & $\,\,$\color{red} 4.0662\color{black} $\,\,$ & $\,\,$3.5954  $\,\,$ \\
$\,\,$\color{red} 0.1972\color{black} $\,\,$ & $\,\,$\color{red} 0.2459\color{black} $\,\,$ & $\,\,$ 1 $\,\,$ & $\,\,$\color{red} 0.8842\color{black}  $\,\,$ \\
$\,\,$0.2231$\,\,$ & $\,\,$0.2781$\,\,$ & $\,\,$\color{red} 1.1309\color{black} $\,\,$ & $\,\,$ 1  $\,\,$ \\
\end{pmatrix},
\end{equation*}

\begin{equation*}
\mathbf{w}^{\prime} =
\begin{pmatrix}
0.449437\\
0.360428\\
0.089887\\
0.100247
\end{pmatrix} =
0.998752\cdot
\begin{pmatrix}
0.449999\\
0.360879\\
\color{gr} 0.090000\color{black} \\
0.100372
\end{pmatrix},
\end{equation*}
\begin{equation*}
\left[ \frac{{w}^{\prime}_i}{{w}^{\prime}_j} \right] =
\begin{pmatrix}
$\,\,$ 1 $\,\,$ & $\,\,$1.2470$\,\,$ & $\,\,$\color{gr} \color{blue} 5\color{black} $\,\,$ & $\,\,$4.4833$\,\,$ \\
$\,\,$0.8020$\,\,$ & $\,\,$ 1 $\,\,$ & $\,\,$\color{gr} 4.0098\color{black} $\,\,$ & $\,\,$3.5954  $\,\,$ \\
$\,\,$\color{gr} \color{blue}  1/5\color{black} $\,\,$ & $\,\,$\color{gr} 0.2494\color{black} $\,\,$ & $\,\,$ 1 $\,\,$ & $\,\,$\color{gr} 0.8967\color{black}  $\,\,$ \\
$\,\,$0.2231$\,\,$ & $\,\,$0.2781$\,\,$ & $\,\,$\color{gr} 1.1153\color{black} $\,\,$ & $\,\,$ 1  $\,\,$ \\
\end{pmatrix},
\end{equation*}
\end{example}
\newpage
\begin{example}
\begin{equation*}
\mathbf{A} =
\begin{pmatrix}
$\,\,$ 1 $\,\,$ & $\,\,$2$\,\,$ & $\,\,$5$\,\,$ & $\,\,$3 $\,\,$ \\
$\,\,$ 1/2$\,\,$ & $\,\,$ 1 $\,\,$ & $\,\,$4$\,\,$ & $\,\,$7 $\,\,$ \\
$\,\,$ 1/5$\,\,$ & $\,\,$ 1/4$\,\,$ & $\,\,$ 1 $\,\,$ & $\,\,$1 $\,\,$ \\
$\,\,$ 1/3$\,\,$ & $\,\,$ 1/7$\,\,$ & $\,\,$ 1 $\,\,$ & $\,\,$ 1  $\,\,$ \\
\end{pmatrix},
\qquad
\lambda_{\max} =
4.2057,
\qquad
CR = 0.0776
\end{equation*}

\begin{equation*}
\mathbf{w}^{cos} =
\begin{pmatrix}
0.446820\\
0.368973\\
\color{red} 0.087055\color{black} \\
0.097152
\end{pmatrix}\end{equation*}
\begin{equation*}
\left[ \frac{{w}^{cos}_i}{{w}^{cos}_j} \right] =
\begin{pmatrix}
$\,\,$ 1 $\,\,$ & $\,\,$1.2110$\,\,$ & $\,\,$\color{red} 5.1326\color{black} $\,\,$ & $\,\,$4.5992$\,\,$ \\
$\,\,$0.8258$\,\,$ & $\,\,$ 1 $\,\,$ & $\,\,$\color{red} 4.2384\color{black} $\,\,$ & $\,\,$3.7979  $\,\,$ \\
$\,\,$\color{red} 0.1948\color{black} $\,\,$ & $\,\,$\color{red} 0.2359\color{black} $\,\,$ & $\,\,$ 1 $\,\,$ & $\,\,$\color{red} 0.8961\color{black}  $\,\,$ \\
$\,\,$0.2174$\,\,$ & $\,\,$0.2633$\,\,$ & $\,\,$\color{red} 1.1160\color{black} $\,\,$ & $\,\,$ 1  $\,\,$ \\
\end{pmatrix},
\end{equation*}

\begin{equation*}
\mathbf{w}^{\prime} =
\begin{pmatrix}
0.445791\\
0.368123\\
0.089158\\
0.096928
\end{pmatrix} =
0.997696\cdot
\begin{pmatrix}
0.446820\\
0.368973\\
\color{gr} 0.089364\color{black} \\
0.097152
\end{pmatrix},
\end{equation*}
\begin{equation*}
\left[ \frac{{w}^{\prime}_i}{{w}^{\prime}_j} \right] =
\begin{pmatrix}
$\,\,$ 1 $\,\,$ & $\,\,$1.2110$\,\,$ & $\,\,$\color{gr} \color{blue} 5\color{black} $\,\,$ & $\,\,$4.5992$\,\,$ \\
$\,\,$0.8258$\,\,$ & $\,\,$ 1 $\,\,$ & $\,\,$\color{gr} 4.1289\color{black} $\,\,$ & $\,\,$3.7979  $\,\,$ \\
$\,\,$\color{gr} \color{blue}  1/5\color{black} $\,\,$ & $\,\,$\color{gr} 0.2422\color{black} $\,\,$ & $\,\,$ 1 $\,\,$ & $\,\,$\color{gr} 0.9198\color{black}  $\,\,$ \\
$\,\,$0.2174$\,\,$ & $\,\,$0.2633$\,\,$ & $\,\,$\color{gr} 1.0871\color{black} $\,\,$ & $\,\,$ 1  $\,\,$ \\
\end{pmatrix},
\end{equation*}
\end{example}
\newpage
\begin{example}
\begin{equation*}
\mathbf{A} =
\begin{pmatrix}
$\,\,$ 1 $\,\,$ & $\,\,$2$\,\,$ & $\,\,$5$\,\,$ & $\,\,$3 $\,\,$ \\
$\,\,$ 1/2$\,\,$ & $\,\,$ 1 $\,\,$ & $\,\,$4$\,\,$ & $\,\,$8 $\,\,$ \\
$\,\,$ 1/5$\,\,$ & $\,\,$ 1/4$\,\,$ & $\,\,$ 1 $\,\,$ & $\,\,$1 $\,\,$ \\
$\,\,$ 1/3$\,\,$ & $\,\,$ 1/8$\,\,$ & $\,\,$ 1 $\,\,$ & $\,\,$ 1  $\,\,$ \\
\end{pmatrix},
\qquad
\lambda_{\max} =
4.2460,
\qquad
CR = 0.0928
\end{equation*}

\begin{equation*}
\mathbf{w}^{cos} =
\begin{pmatrix}
0.444355\\
0.375335\\
\color{red} 0.085684\color{black} \\
0.094626
\end{pmatrix}\end{equation*}
\begin{equation*}
\left[ \frac{{w}^{cos}_i}{{w}^{cos}_j} \right] =
\begin{pmatrix}
$\,\,$ 1 $\,\,$ & $\,\,$1.1839$\,\,$ & $\,\,$\color{red} 5.1860\color{black} $\,\,$ & $\,\,$4.6959$\,\,$ \\
$\,\,$0.8447$\,\,$ & $\,\,$ 1 $\,\,$ & $\,\,$\color{red} 4.3804\color{black} $\,\,$ & $\,\,$3.9665  $\,\,$ \\
$\,\,$\color{red} 0.1928\color{black} $\,\,$ & $\,\,$\color{red} 0.2283\color{black} $\,\,$ & $\,\,$ 1 $\,\,$ & $\,\,$\color{red} 0.9055\color{black}  $\,\,$ \\
$\,\,$0.2130$\,\,$ & $\,\,$0.2521$\,\,$ & $\,\,$\color{red} 1.1044\color{black} $\,\,$ & $\,\,$ 1  $\,\,$ \\
\end{pmatrix},
\end{equation*}

\begin{equation*}
\mathbf{w}^{\prime} =
\begin{pmatrix}
0.442943\\
0.374143\\
0.088589\\
0.094325
\end{pmatrix} =
0.996823\cdot
\begin{pmatrix}
0.444355\\
0.375335\\
\color{gr} 0.088871\color{black} \\
0.094626
\end{pmatrix},
\end{equation*}
\begin{equation*}
\left[ \frac{{w}^{\prime}_i}{{w}^{\prime}_j} \right] =
\begin{pmatrix}
$\,\,$ 1 $\,\,$ & $\,\,$1.1839$\,\,$ & $\,\,$\color{gr} \color{blue} 5\color{black} $\,\,$ & $\,\,$4.6959$\,\,$ \\
$\,\,$0.8447$\,\,$ & $\,\,$ 1 $\,\,$ & $\,\,$\color{gr} 4.2234\color{black} $\,\,$ & $\,\,$3.9665  $\,\,$ \\
$\,\,$\color{gr} \color{blue}  1/5\color{black} $\,\,$ & $\,\,$\color{gr} 0.2368\color{black} $\,\,$ & $\,\,$ 1 $\,\,$ & $\,\,$\color{gr} 0.9392\color{black}  $\,\,$ \\
$\,\,$0.2130$\,\,$ & $\,\,$0.2521$\,\,$ & $\,\,$\color{gr} 1.0648\color{black} $\,\,$ & $\,\,$ 1  $\,\,$ \\
\end{pmatrix},
\end{equation*}
\end{example}
\newpage
\begin{example}
\begin{equation*}
\mathbf{A} =
\begin{pmatrix}
$\,\,$ 1 $\,\,$ & $\,\,$2$\,\,$ & $\,\,$5$\,\,$ & $\,\,$6 $\,\,$ \\
$\,\,$ 1/2$\,\,$ & $\,\,$ 1 $\,\,$ & $\,\,$4$\,\,$ & $\,\,$2 $\,\,$ \\
$\,\,$ 1/5$\,\,$ & $\,\,$ 1/4$\,\,$ & $\,\,$ 1 $\,\,$ & $\,\,$2 $\,\,$ \\
$\,\,$ 1/6$\,\,$ & $\,\,$ 1/2$\,\,$ & $\,\,$ 1/2$\,\,$ & $\,\,$ 1  $\,\,$ \\
\end{pmatrix},
\qquad
\lambda_{\max} =
4.1655,
\qquad
CR = 0.0624
\end{equation*}

\begin{equation*}
\mathbf{w}^{cos} =
\begin{pmatrix}
\color{red} 0.522799\color{black} \\
0.273953\\
0.112884\\
0.090364
\end{pmatrix}\end{equation*}
\begin{equation*}
\left[ \frac{{w}^{cos}_i}{{w}^{cos}_j} \right] =
\begin{pmatrix}
$\,\,$ 1 $\,\,$ & $\,\,$\color{red} 1.9084\color{black} $\,\,$ & $\,\,$\color{red} 4.6313\color{black} $\,\,$ & $\,\,$\color{red} 5.7855\color{black} $\,\,$ \\
$\,\,$\color{red} 0.5240\color{black} $\,\,$ & $\,\,$ 1 $\,\,$ & $\,\,$2.4268$\,\,$ & $\,\,$3.0316  $\,\,$ \\
$\,\,$\color{red} 0.2159\color{black} $\,\,$ & $\,\,$0.4121$\,\,$ & $\,\,$ 1 $\,\,$ & $\,\,$1.2492 $\,\,$ \\
$\,\,$\color{red} 0.1728\color{black} $\,\,$ & $\,\,$0.3299$\,\,$ & $\,\,$0.8005$\,\,$ & $\,\,$ 1  $\,\,$ \\
\end{pmatrix},
\end{equation*}

\begin{equation*}
\mathbf{w}^{\prime} =
\begin{pmatrix}
0.531874\\
0.268743\\
0.110738\\
0.088646
\end{pmatrix} =
0.980983\cdot
\begin{pmatrix}
\color{gr} 0.542185\color{black} \\
0.273953\\
0.112884\\
0.090364
\end{pmatrix},
\end{equation*}
\begin{equation*}
\left[ \frac{{w}^{\prime}_i}{{w}^{\prime}_j} \right] =
\begin{pmatrix}
$\,\,$ 1 $\,\,$ & $\,\,$\color{gr} 1.9791\color{black} $\,\,$ & $\,\,$\color{gr} 4.8030\color{black} $\,\,$ & $\,\,$\color{gr} \color{blue} 6\color{black} $\,\,$ \\
$\,\,$\color{gr} 0.5053\color{black} $\,\,$ & $\,\,$ 1 $\,\,$ & $\,\,$2.4268$\,\,$ & $\,\,$3.0316  $\,\,$ \\
$\,\,$\color{gr} 0.2082\color{black} $\,\,$ & $\,\,$0.4121$\,\,$ & $\,\,$ 1 $\,\,$ & $\,\,$1.2492 $\,\,$ \\
$\,\,$\color{gr} \color{blue}  1/6\color{black} $\,\,$ & $\,\,$0.3299$\,\,$ & $\,\,$0.8005$\,\,$ & $\,\,$ 1  $\,\,$ \\
\end{pmatrix},
\end{equation*}
\end{example}
\newpage
\begin{example}
\begin{equation*}
\mathbf{A} =
\begin{pmatrix}
$\,\,$ 1 $\,\,$ & $\,\,$2$\,\,$ & $\,\,$5$\,\,$ & $\,\,$7 $\,\,$ \\
$\,\,$ 1/2$\,\,$ & $\,\,$ 1 $\,\,$ & $\,\,$4$\,\,$ & $\,\,$2 $\,\,$ \\
$\,\,$ 1/5$\,\,$ & $\,\,$ 1/4$\,\,$ & $\,\,$ 1 $\,\,$ & $\,\,$2 $\,\,$ \\
$\,\,$ 1/7$\,\,$ & $\,\,$ 1/2$\,\,$ & $\,\,$ 1/2$\,\,$ & $\,\,$ 1  $\,\,$ \\
\end{pmatrix},
\qquad
\lambda_{\max} =
4.1665,
\qquad
CR = 0.0628
\end{equation*}

\begin{equation*}
\mathbf{w}^{cos} =
\begin{pmatrix}
\color{red} 0.533559\color{black} \\
0.272039\\
0.108844\\
0.085558
\end{pmatrix}\end{equation*}
\begin{equation*}
\left[ \frac{{w}^{cos}_i}{{w}^{cos}_j} \right] =
\begin{pmatrix}
$\,\,$ 1 $\,\,$ & $\,\,$\color{red} 1.9613\color{black} $\,\,$ & $\,\,$\color{red} 4.9021\color{black} $\,\,$ & $\,\,$\color{red} 6.2363\color{black} $\,\,$ \\
$\,\,$\color{red} 0.5099\color{black} $\,\,$ & $\,\,$ 1 $\,\,$ & $\,\,$2.4994$\,\,$ & $\,\,$3.1796  $\,\,$ \\
$\,\,$\color{red} 0.2040\color{black} $\,\,$ & $\,\,$0.4001$\,\,$ & $\,\,$ 1 $\,\,$ & $\,\,$1.2722 $\,\,$ \\
$\,\,$\color{red} 0.1604\color{black} $\,\,$ & $\,\,$0.3145$\,\,$ & $\,\,$0.7861$\,\,$ & $\,\,$ 1  $\,\,$ \\
\end{pmatrix},
\end{equation*}

\begin{equation*}
\mathbf{w}^{\prime} =
\begin{pmatrix}
0.538415\\
0.269207\\
0.107711\\
0.084667
\end{pmatrix} =
0.989590\cdot
\begin{pmatrix}
\color{gr} 0.544079\color{black} \\
0.272039\\
0.108844\\
0.085558
\end{pmatrix},
\end{equation*}
\begin{equation*}
\left[ \frac{{w}^{\prime}_i}{{w}^{\prime}_j} \right] =
\begin{pmatrix}
$\,\,$ 1 $\,\,$ & $\,\,$\color{gr} \color{blue} 2\color{black} $\,\,$ & $\,\,$\color{gr} 4.9987\color{black} $\,\,$ & $\,\,$\color{gr} 6.3592\color{black} $\,\,$ \\
$\,\,$\color{gr} \color{blue}  1/2\color{black} $\,\,$ & $\,\,$ 1 $\,\,$ & $\,\,$2.4994$\,\,$ & $\,\,$3.1796  $\,\,$ \\
$\,\,$\color{gr} 0.2001\color{black} $\,\,$ & $\,\,$0.4001$\,\,$ & $\,\,$ 1 $\,\,$ & $\,\,$1.2722 $\,\,$ \\
$\,\,$\color{gr} 0.1573\color{black} $\,\,$ & $\,\,$0.3145$\,\,$ & $\,\,$0.7861$\,\,$ & $\,\,$ 1  $\,\,$ \\
\end{pmatrix},
\end{equation*}
\end{example}
\newpage
\begin{example}
\begin{equation*}
\mathbf{A} =
\begin{pmatrix}
$\,\,$ 1 $\,\,$ & $\,\,$2$\,\,$ & $\,\,$5$\,\,$ & $\,\,$7 $\,\,$ \\
$\,\,$ 1/2$\,\,$ & $\,\,$ 1 $\,\,$ & $\,\,$8$\,\,$ & $\,\,$5 $\,\,$ \\
$\,\,$ 1/5$\,\,$ & $\,\,$ 1/8$\,\,$ & $\,\,$ 1 $\,\,$ & $\,\,$1 $\,\,$ \\
$\,\,$ 1/7$\,\,$ & $\,\,$ 1/5$\,\,$ & $\,\,$ 1 $\,\,$ & $\,\,$ 1  $\,\,$ \\
\end{pmatrix},
\qquad
\lambda_{\max} =
4.1159,
\qquad
CR = 0.0437
\end{equation*}

\begin{equation*}
\mathbf{w}^{cos} =
\begin{pmatrix}
0.493086\\
0.365997\\
0.071780\\
\color{red} 0.069137\color{black}
\end{pmatrix}\end{equation*}
\begin{equation*}
\left[ \frac{{w}^{cos}_i}{{w}^{cos}_j} \right] =
\begin{pmatrix}
$\,\,$ 1 $\,\,$ & $\,\,$1.3472$\,\,$ & $\,\,$6.8694$\,\,$ & $\,\,$\color{red} 7.1320\color{black} $\,\,$ \\
$\,\,$0.7423$\,\,$ & $\,\,$ 1 $\,\,$ & $\,\,$5.0989$\,\,$ & $\,\,$\color{red} 5.2938\color{black}   $\,\,$ \\
$\,\,$0.1456$\,\,$ & $\,\,$0.1961$\,\,$ & $\,\,$ 1 $\,\,$ & $\,\,$\color{red} 1.0382\color{black}  $\,\,$ \\
$\,\,$\color{red} 0.1402\color{black} $\,\,$ & $\,\,$\color{red} 0.1889\color{black} $\,\,$ & $\,\,$\color{red} 0.9632\color{black} $\,\,$ & $\,\,$ 1  $\,\,$ \\
\end{pmatrix},
\end{equation*}

\begin{equation*}
\mathbf{w}^{\prime} =
\begin{pmatrix}
0.492444\\
0.365521\\
0.071686\\
0.070349
\end{pmatrix} =
0.998698\cdot
\begin{pmatrix}
0.493086\\
0.365997\\
0.071780\\
\color{gr} 0.070441\color{black}
\end{pmatrix},
\end{equation*}
\begin{equation*}
\left[ \frac{{w}^{\prime}_i}{{w}^{\prime}_j} \right] =
\begin{pmatrix}
$\,\,$ 1 $\,\,$ & $\,\,$1.3472$\,\,$ & $\,\,$6.8694$\,\,$ & $\,\,$\color{gr} \color{blue} 7\color{black} $\,\,$ \\
$\,\,$0.7423$\,\,$ & $\,\,$ 1 $\,\,$ & $\,\,$5.0989$\,\,$ & $\,\,$\color{gr} 5.1958\color{black}   $\,\,$ \\
$\,\,$0.1456$\,\,$ & $\,\,$0.1961$\,\,$ & $\,\,$ 1 $\,\,$ & $\,\,$\color{gr} 1.0190\color{black}  $\,\,$ \\
$\,\,$\color{gr} \color{blue}  1/7\color{black} $\,\,$ & $\,\,$\color{gr} 0.1925\color{black} $\,\,$ & $\,\,$\color{gr} 0.9813\color{black} $\,\,$ & $\,\,$ 1  $\,\,$ \\
\end{pmatrix},
\end{equation*}
\end{example}
\newpage
\begin{example}
\begin{equation*}
\mathbf{A} =
\begin{pmatrix}
$\,\,$ 1 $\,\,$ & $\,\,$2$\,\,$ & $\,\,$5$\,\,$ & $\,\,$7 $\,\,$ \\
$\,\,$ 1/2$\,\,$ & $\,\,$ 1 $\,\,$ & $\,\,$9$\,\,$ & $\,\,$5 $\,\,$ \\
$\,\,$ 1/5$\,\,$ & $\,\,$ 1/9$\,\,$ & $\,\,$ 1 $\,\,$ & $\,\,$1 $\,\,$ \\
$\,\,$ 1/7$\,\,$ & $\,\,$ 1/5$\,\,$ & $\,\,$ 1 $\,\,$ & $\,\,$ 1  $\,\,$ \\
\end{pmatrix},
\qquad
\lambda_{\max} =
4.1429,
\qquad
CR = 0.0539
\end{equation*}

\begin{equation*}
\mathbf{w}^{cos} =
\begin{pmatrix}
0.489275\\
0.372683\\
0.069853\\
\color{red} 0.068188\color{black}
\end{pmatrix}\end{equation*}
\begin{equation*}
\left[ \frac{{w}^{cos}_i}{{w}^{cos}_j} \right] =
\begin{pmatrix}
$\,\,$ 1 $\,\,$ & $\,\,$1.3128$\,\,$ & $\,\,$7.0043$\,\,$ & $\,\,$\color{red} 7.1754\color{black} $\,\,$ \\
$\,\,$0.7617$\,\,$ & $\,\,$ 1 $\,\,$ & $\,\,$5.3352$\,\,$ & $\,\,$\color{red} 5.4655\color{black}   $\,\,$ \\
$\,\,$0.1428$\,\,$ & $\,\,$0.1874$\,\,$ & $\,\,$ 1 $\,\,$ & $\,\,$\color{red} 1.0244\color{black}  $\,\,$ \\
$\,\,$\color{red} 0.1394\color{black} $\,\,$ & $\,\,$\color{red} 0.1830\color{black} $\,\,$ & $\,\,$\color{red} 0.9762\color{black} $\,\,$ & $\,\,$ 1  $\,\,$ \\
\end{pmatrix},
\end{equation*}

\begin{equation*}
\mathbf{w}^{\prime} =
\begin{pmatrix}
0.488462\\
0.372063\\
0.069737\\
0.069737
\end{pmatrix} =
0.998337\cdot
\begin{pmatrix}
0.489275\\
0.372683\\
0.069853\\
\color{gr} 0.069853\color{black}
\end{pmatrix},
\end{equation*}
\begin{equation*}
\left[ \frac{{w}^{\prime}_i}{{w}^{\prime}_j} \right] =
\begin{pmatrix}
$\,\,$ 1 $\,\,$ & $\,\,$1.3128$\,\,$ & $\,\,$7.0043$\,\,$ & $\,\,$\color{gr} 7.0043\color{black} $\,\,$ \\
$\,\,$0.7617$\,\,$ & $\,\,$ 1 $\,\,$ & $\,\,$5.3352$\,\,$ & $\,\,$\color{gr} 5.3352\color{black}   $\,\,$ \\
$\,\,$0.1428$\,\,$ & $\,\,$0.1874$\,\,$ & $\,\,$ 1 $\,\,$ & $\,\,$\color{gr} \color{blue} 1\color{black}  $\,\,$ \\
$\,\,$\color{gr} 0.1428\color{black} $\,\,$ & $\,\,$\color{gr} 0.1874\color{black} $\,\,$ & $\,\,$\color{gr} \color{blue} 1\color{black} $\,\,$ & $\,\,$ 1  $\,\,$ \\
\end{pmatrix},
\end{equation*}
\end{example}
\newpage
\begin{example}
\begin{equation*}
\mathbf{A} =
\begin{pmatrix}
$\,\,$ 1 $\,\,$ & $\,\,$2$\,\,$ & $\,\,$6$\,\,$ & $\,\,$3 $\,\,$ \\
$\,\,$ 1/2$\,\,$ & $\,\,$ 1 $\,\,$ & $\,\,$5$\,\,$ & $\,\,$8 $\,\,$ \\
$\,\,$ 1/6$\,\,$ & $\,\,$ 1/5$\,\,$ & $\,\,$ 1 $\,\,$ & $\,\,$1 $\,\,$ \\
$\,\,$ 1/3$\,\,$ & $\,\,$ 1/8$\,\,$ & $\,\,$ 1 $\,\,$ & $\,\,$ 1  $\,\,$ \\
\end{pmatrix},
\qquad
\lambda_{\max} =
4.2460,
\qquad
CR = 0.0928
\end{equation*}

\begin{equation*}
\mathbf{w}^{cos} =
\begin{pmatrix}
0.449795\\
0.383656\\
\color{red} 0.074718\color{black} \\
0.091831
\end{pmatrix}\end{equation*}
\begin{equation*}
\left[ \frac{{w}^{cos}_i}{{w}^{cos}_j} \right] =
\begin{pmatrix}
$\,\,$ 1 $\,\,$ & $\,\,$1.1724$\,\,$ & $\,\,$\color{red} 6.0199\color{black} $\,\,$ & $\,\,$4.8981$\,\,$ \\
$\,\,$0.8530$\,\,$ & $\,\,$ 1 $\,\,$ & $\,\,$\color{red} 5.1347\color{black} $\,\,$ & $\,\,$4.1779  $\,\,$ \\
$\,\,$\color{red} 0.1661\color{black} $\,\,$ & $\,\,$\color{red} 0.1948\color{black} $\,\,$ & $\,\,$ 1 $\,\,$ & $\,\,$\color{red} 0.8136\color{black}  $\,\,$ \\
$\,\,$0.2042$\,\,$ & $\,\,$0.2394$\,\,$ & $\,\,$\color{red} 1.2290\color{black} $\,\,$ & $\,\,$ 1  $\,\,$ \\
\end{pmatrix},
\end{equation*}

\begin{equation*}
\mathbf{w}^{\prime} =
\begin{pmatrix}
0.449683\\
0.383561\\
0.074947\\
0.091808
\end{pmatrix} =
0.999752\cdot
\begin{pmatrix}
0.449795\\
0.383656\\
\color{gr} 0.074966\color{black} \\
0.091831
\end{pmatrix},
\end{equation*}
\begin{equation*}
\left[ \frac{{w}^{\prime}_i}{{w}^{\prime}_j} \right] =
\begin{pmatrix}
$\,\,$ 1 $\,\,$ & $\,\,$1.1724$\,\,$ & $\,\,$\color{gr} \color{blue} 6\color{black} $\,\,$ & $\,\,$4.8981$\,\,$ \\
$\,\,$0.8530$\,\,$ & $\,\,$ 1 $\,\,$ & $\,\,$\color{gr} 5.1178\color{black} $\,\,$ & $\,\,$4.1779  $\,\,$ \\
$\,\,$\color{gr} \color{blue}  1/6\color{black} $\,\,$ & $\,\,$\color{gr} 0.1954\color{black} $\,\,$ & $\,\,$ 1 $\,\,$ & $\,\,$\color{gr} 0.8163\color{black}  $\,\,$ \\
$\,\,$0.2042$\,\,$ & $\,\,$0.2394$\,\,$ & $\,\,$\color{gr} 1.2250\color{black} $\,\,$ & $\,\,$ 1  $\,\,$ \\
\end{pmatrix},
\end{equation*}
\end{example}
\newpage
\begin{example}
\begin{equation*}
\mathbf{A} =
\begin{pmatrix}
$\,\,$ 1 $\,\,$ & $\,\,$2$\,\,$ & $\,\,$6$\,\,$ & $\,\,$4 $\,\,$ \\
$\,\,$ 1/2$\,\,$ & $\,\,$ 1 $\,\,$ & $\,\,$5$\,\,$ & $\,\,$8 $\,\,$ \\
$\,\,$ 1/6$\,\,$ & $\,\,$ 1/5$\,\,$ & $\,\,$ 1 $\,\,$ & $\,\,$1 $\,\,$ \\
$\,\,$ 1/4$\,\,$ & $\,\,$ 1/8$\,\,$ & $\,\,$ 1 $\,\,$ & $\,\,$ 1  $\,\,$ \\
\end{pmatrix},
\qquad
\lambda_{\max} =
4.1655,
\qquad
CR = 0.0624
\end{equation*}

\begin{equation*}
\mathbf{w}^{cos} =
\begin{pmatrix}
0.467048\\
0.378274\\
\color{red} 0.074282\color{black} \\
0.080395
\end{pmatrix}\end{equation*}
\begin{equation*}
\left[ \frac{{w}^{cos}_i}{{w}^{cos}_j} \right] =
\begin{pmatrix}
$\,\,$ 1 $\,\,$ & $\,\,$1.2347$\,\,$ & $\,\,$\color{red} 6.2875\color{black} $\,\,$ & $\,\,$5.8094$\,\,$ \\
$\,\,$0.8099$\,\,$ & $\,\,$ 1 $\,\,$ & $\,\,$\color{red} 5.0924\color{black} $\,\,$ & $\,\,$4.7052  $\,\,$ \\
$\,\,$\color{red} 0.1590\color{black} $\,\,$ & $\,\,$\color{red} 0.1964\color{black} $\,\,$ & $\,\,$ 1 $\,\,$ & $\,\,$\color{red} 0.9240\color{black}  $\,\,$ \\
$\,\,$0.1721$\,\,$ & $\,\,$0.2125$\,\,$ & $\,\,$\color{red} 1.0823\color{black} $\,\,$ & $\,\,$ 1  $\,\,$ \\
\end{pmatrix},
\end{equation*}

\begin{equation*}
\mathbf{w}^{\prime} =
\begin{pmatrix}
0.466408\\
0.377756\\
0.075551\\
0.080285
\end{pmatrix} =
0.998629\cdot
\begin{pmatrix}
0.467048\\
0.378274\\
\color{gr} 0.075655\color{black} \\
0.080395
\end{pmatrix},
\end{equation*}
\begin{equation*}
\left[ \frac{{w}^{\prime}_i}{{w}^{\prime}_j} \right] =
\begin{pmatrix}
$\,\,$ 1 $\,\,$ & $\,\,$1.2347$\,\,$ & $\,\,$\color{gr} 6.1734\color{black} $\,\,$ & $\,\,$5.8094$\,\,$ \\
$\,\,$0.8099$\,\,$ & $\,\,$ 1 $\,\,$ & $\,\,$\color{gr} \color{blue} 5\color{black} $\,\,$ & $\,\,$4.7052  $\,\,$ \\
$\,\,$\color{gr} 0.1620\color{black} $\,\,$ & $\,\,$\color{gr} \color{blue}  1/5\color{black} $\,\,$ & $\,\,$ 1 $\,\,$ & $\,\,$\color{gr} 0.9410\color{black}  $\,\,$ \\
$\,\,$0.1721$\,\,$ & $\,\,$0.2125$\,\,$ & $\,\,$\color{gr} 1.0627\color{black} $\,\,$ & $\,\,$ 1  $\,\,$ \\
\end{pmatrix},
\end{equation*}
\end{example}
\newpage
\begin{example}
\begin{equation*}
\mathbf{A} =
\begin{pmatrix}
$\,\,$ 1 $\,\,$ & $\,\,$2$\,\,$ & $\,\,$6$\,\,$ & $\,\,$4 $\,\,$ \\
$\,\,$ 1/2$\,\,$ & $\,\,$ 1 $\,\,$ & $\,\,$5$\,\,$ & $\,\,$9 $\,\,$ \\
$\,\,$ 1/6$\,\,$ & $\,\,$ 1/5$\,\,$ & $\,\,$ 1 $\,\,$ & $\,\,$1 $\,\,$ \\
$\,\,$ 1/4$\,\,$ & $\,\,$ 1/9$\,\,$ & $\,\,$ 1 $\,\,$ & $\,\,$ 1  $\,\,$ \\
\end{pmatrix},
\qquad
\lambda_{\max} =
4.1966,
\qquad
CR = 0.0741
\end{equation*}

\begin{equation*}
\mathbf{w}^{cos} =
\begin{pmatrix}
0.464054\\
0.384321\\
\color{red} 0.073227\color{black} \\
0.078397
\end{pmatrix}\end{equation*}
\begin{equation*}
\left[ \frac{{w}^{cos}_i}{{w}^{cos}_j} \right] =
\begin{pmatrix}
$\,\,$ 1 $\,\,$ & $\,\,$1.2075$\,\,$ & $\,\,$\color{red} 6.3372\color{black} $\,\,$ & $\,\,$5.9193$\,\,$ \\
$\,\,$0.8282$\,\,$ & $\,\,$ 1 $\,\,$ & $\,\,$\color{red} 5.2483\color{black} $\,\,$ & $\,\,$4.9022  $\,\,$ \\
$\,\,$\color{red} 0.1578\color{black} $\,\,$ & $\,\,$\color{red} 0.1905\color{black} $\,\,$ & $\,\,$ 1 $\,\,$ & $\,\,$\color{red} 0.9341\color{black}  $\,\,$ \\
$\,\,$0.1689$\,\,$ & $\,\,$0.2040$\,\,$ & $\,\,$\color{red} 1.0706\color{black} $\,\,$ & $\,\,$ 1  $\,\,$ \\
\end{pmatrix},
\end{equation*}

\begin{equation*}
\mathbf{w}^{\prime} =
\begin{pmatrix}
0.462373\\
0.382929\\
0.076586\\
0.078113
\end{pmatrix} =
0.996376\cdot
\begin{pmatrix}
0.464054\\
0.384321\\
\color{gr} 0.076864\color{black} \\
0.078397
\end{pmatrix},
\end{equation*}
\begin{equation*}
\left[ \frac{{w}^{\prime}_i}{{w}^{\prime}_j} \right] =
\begin{pmatrix}
$\,\,$ 1 $\,\,$ & $\,\,$1.2075$\,\,$ & $\,\,$\color{gr} 6.0373\color{black} $\,\,$ & $\,\,$5.9193$\,\,$ \\
$\,\,$0.8282$\,\,$ & $\,\,$ 1 $\,\,$ & $\,\,$\color{gr} \color{blue} 5\color{black} $\,\,$ & $\,\,$4.9022  $\,\,$ \\
$\,\,$\color{gr} 0.1656\color{black} $\,\,$ & $\,\,$\color{gr} \color{blue}  1/5\color{black} $\,\,$ & $\,\,$ 1 $\,\,$ & $\,\,$\color{gr} 0.9804\color{black}  $\,\,$ \\
$\,\,$0.1689$\,\,$ & $\,\,$0.2040$\,\,$ & $\,\,$\color{gr} 1.0199\color{black} $\,\,$ & $\,\,$ 1  $\,\,$ \\
\end{pmatrix},
\end{equation*}
\end{example}
\newpage
\begin{example}
\begin{equation*}
\mathbf{A} =
\begin{pmatrix}
$\,\,$ 1 $\,\,$ & $\,\,$2$\,\,$ & $\,\,$6$\,\,$ & $\,\,$5 $\,\,$ \\
$\,\,$ 1/2$\,\,$ & $\,\,$ 1 $\,\,$ & $\,\,$2$\,\,$ & $\,\,$3 $\,\,$ \\
$\,\,$ 1/6$\,\,$ & $\,\,$ 1/2$\,\,$ & $\,\,$ 1 $\,\,$ & $\,\,$2 $\,\,$ \\
$\,\,$ 1/5$\,\,$ & $\,\,$ 1/3$\,\,$ & $\,\,$ 1/2$\,\,$ & $\,\,$ 1  $\,\,$ \\
\end{pmatrix},
\qquad
\lambda_{\max} =
4.0662,
\qquad
CR = 0.0250
\end{equation*}

\begin{equation*}
\mathbf{w}^{cos} =
\begin{pmatrix}
0.531987\\
\color{red} 0.254397\color{black} \\
0.128344\\
0.085272
\end{pmatrix}\end{equation*}
\begin{equation*}
\left[ \frac{{w}^{cos}_i}{{w}^{cos}_j} \right] =
\begin{pmatrix}
$\,\,$ 1 $\,\,$ & $\,\,$\color{red} 2.0912\color{black} $\,\,$ & $\,\,$4.1450$\,\,$ & $\,\,$6.2387$\,\,$ \\
$\,\,$\color{red} 0.4782\color{black} $\,\,$ & $\,\,$ 1 $\,\,$ & $\,\,$\color{red} 1.9822\color{black} $\,\,$ & $\,\,$\color{red} 2.9833\color{black}   $\,\,$ \\
$\,\,$0.2413$\,\,$ & $\,\,$\color{red} 0.5045\color{black} $\,\,$ & $\,\,$ 1 $\,\,$ & $\,\,$1.5051 $\,\,$ \\
$\,\,$0.1603$\,\,$ & $\,\,$\color{red} 0.3352\color{black} $\,\,$ & $\,\,$0.6644$\,\,$ & $\,\,$ 1  $\,\,$ \\
\end{pmatrix},
\end{equation*}

\begin{equation*}
\mathbf{w}^{\prime} =
\begin{pmatrix}
0.531233\\
0.255454\\
0.128162\\
0.085151
\end{pmatrix} =
0.998582\cdot
\begin{pmatrix}
0.531987\\
\color{gr} 0.255817\color{black} \\
0.128344\\
0.085272
\end{pmatrix},
\end{equation*}
\begin{equation*}
\left[ \frac{{w}^{\prime}_i}{{w}^{\prime}_j} \right] =
\begin{pmatrix}
$\,\,$ 1 $\,\,$ & $\,\,$\color{gr} 2.0796\color{black} $\,\,$ & $\,\,$4.1450$\,\,$ & $\,\,$6.2387$\,\,$ \\
$\,\,$\color{gr} 0.4809\color{black} $\,\,$ & $\,\,$ 1 $\,\,$ & $\,\,$\color{gr} 1.9932\color{black} $\,\,$ & $\,\,$\color{gr} \color{blue} 3\color{black}   $\,\,$ \\
$\,\,$0.2413$\,\,$ & $\,\,$\color{gr} 0.5017\color{black} $\,\,$ & $\,\,$ 1 $\,\,$ & $\,\,$1.5051 $\,\,$ \\
$\,\,$0.1603$\,\,$ & $\,\,$\color{gr} \color{blue}  1/3\color{black} $\,\,$ & $\,\,$0.6644$\,\,$ & $\,\,$ 1  $\,\,$ \\
\end{pmatrix},
\end{equation*}
\end{example}
\newpage
\begin{example}
\begin{equation*}
\mathbf{A} =
\begin{pmatrix}
$\,\,$ 1 $\,\,$ & $\,\,$2$\,\,$ & $\,\,$6$\,\,$ & $\,\,$6 $\,\,$ \\
$\,\,$ 1/2$\,\,$ & $\,\,$ 1 $\,\,$ & $\,\,$2$\,\,$ & $\,\,$4 $\,\,$ \\
$\,\,$ 1/6$\,\,$ & $\,\,$ 1/2$\,\,$ & $\,\,$ 1 $\,\,$ & $\,\,$3 $\,\,$ \\
$\,\,$ 1/6$\,\,$ & $\,\,$ 1/4$\,\,$ & $\,\,$ 1/3$\,\,$ & $\,\,$ 1  $\,\,$ \\
\end{pmatrix},
\qquad
\lambda_{\max} =
4.1031,
\qquad
CR = 0.0389
\end{equation*}

\begin{equation*}
\mathbf{w}^{cos} =
\begin{pmatrix}
0.532190\\
\color{red} 0.261615\color{black} \\
0.139171\\
0.067024
\end{pmatrix}\end{equation*}
\begin{equation*}
\left[ \frac{{w}^{cos}_i}{{w}^{cos}_j} \right] =
\begin{pmatrix}
$\,\,$ 1 $\,\,$ & $\,\,$\color{red} 2.0342\color{black} $\,\,$ & $\,\,$3.8240$\,\,$ & $\,\,$7.9403$\,\,$ \\
$\,\,$\color{red} 0.4916\color{black} $\,\,$ & $\,\,$ 1 $\,\,$ & $\,\,$\color{red} 1.8798\color{black} $\,\,$ & $\,\,$\color{red} 3.9033\color{black}   $\,\,$ \\
$\,\,$0.2615$\,\,$ & $\,\,$\color{red} 0.5320\color{black} $\,\,$ & $\,\,$ 1 $\,\,$ & $\,\,$2.0764 $\,\,$ \\
$\,\,$0.1259$\,\,$ & $\,\,$\color{red} 0.2562\color{black} $\,\,$ & $\,\,$0.4816$\,\,$ & $\,\,$ 1  $\,\,$ \\
\end{pmatrix},
\end{equation*}

\begin{equation*}
\mathbf{w}^{\prime} =
\begin{pmatrix}
0.529817\\
0.264908\\
0.138550\\
0.066725
\end{pmatrix} =
0.995540\cdot
\begin{pmatrix}
0.532190\\
\color{gr} 0.266095\color{black} \\
0.139171\\
0.067024
\end{pmatrix},
\end{equation*}
\begin{equation*}
\left[ \frac{{w}^{\prime}_i}{{w}^{\prime}_j} \right] =
\begin{pmatrix}
$\,\,$ 1 $\,\,$ & $\,\,$\color{gr} \color{blue} 2\color{black} $\,\,$ & $\,\,$3.8240$\,\,$ & $\,\,$7.9403$\,\,$ \\
$\,\,$\color{gr} \color{blue}  1/2\color{black} $\,\,$ & $\,\,$ 1 $\,\,$ & $\,\,$\color{gr} 1.9120\color{black} $\,\,$ & $\,\,$\color{gr} 3.9702\color{black}   $\,\,$ \\
$\,\,$0.2615$\,\,$ & $\,\,$\color{gr} 0.5230\color{black} $\,\,$ & $\,\,$ 1 $\,\,$ & $\,\,$2.0764 $\,\,$ \\
$\,\,$0.1259$\,\,$ & $\,\,$\color{gr} 0.2519\color{black} $\,\,$ & $\,\,$0.4816$\,\,$ & $\,\,$ 1  $\,\,$ \\
\end{pmatrix},
\end{equation*}
\end{example}
\newpage
\begin{example}
\begin{equation*}
\mathbf{A} =
\begin{pmatrix}
$\,\,$ 1 $\,\,$ & $\,\,$2$\,\,$ & $\,\,$6$\,\,$ & $\,\,$6 $\,\,$ \\
$\,\,$ 1/2$\,\,$ & $\,\,$ 1 $\,\,$ & $\,\,$5$\,\,$ & $\,\,$2 $\,\,$ \\
$\,\,$ 1/6$\,\,$ & $\,\,$ 1/5$\,\,$ & $\,\,$ 1 $\,\,$ & $\,\,$2 $\,\,$ \\
$\,\,$ 1/6$\,\,$ & $\,\,$ 1/2$\,\,$ & $\,\,$ 1/2$\,\,$ & $\,\,$ 1  $\,\,$ \\
\end{pmatrix},
\qquad
\lambda_{\max} =
4.2277,
\qquad
CR = 0.0859
\end{equation*}

\begin{equation*}
\mathbf{w}^{cos} =
\begin{pmatrix}
\color{red} 0.528117\color{black} \\
0.280279\\
0.102171\\
0.089432
\end{pmatrix}\end{equation*}
\begin{equation*}
\left[ \frac{{w}^{cos}_i}{{w}^{cos}_j} \right] =
\begin{pmatrix}
$\,\,$ 1 $\,\,$ & $\,\,$\color{red} 1.8843\color{black} $\,\,$ & $\,\,$\color{red} 5.1689\color{black} $\,\,$ & $\,\,$\color{red} 5.9052\color{black} $\,\,$ \\
$\,\,$\color{red} 0.5307\color{black} $\,\,$ & $\,\,$ 1 $\,\,$ & $\,\,$2.7432$\,\,$ & $\,\,$3.1340  $\,\,$ \\
$\,\,$\color{red} 0.1935\color{black} $\,\,$ & $\,\,$0.3645$\,\,$ & $\,\,$ 1 $\,\,$ & $\,\,$1.1425 $\,\,$ \\
$\,\,$\color{red} 0.1693\color{black} $\,\,$ & $\,\,$0.3191$\,\,$ & $\,\,$0.8753$\,\,$ & $\,\,$ 1  $\,\,$ \\
\end{pmatrix},
\end{equation*}

\begin{equation*}
\mathbf{w}^{\prime} =
\begin{pmatrix}
0.532082\\
0.277924\\
0.101313\\
0.088680
\end{pmatrix} =
0.991597\cdot
\begin{pmatrix}
\color{gr} 0.536591\color{black} \\
0.280279\\
0.102171\\
0.089432
\end{pmatrix},
\end{equation*}
\begin{equation*}
\left[ \frac{{w}^{\prime}_i}{{w}^{\prime}_j} \right] =
\begin{pmatrix}
$\,\,$ 1 $\,\,$ & $\,\,$\color{gr} 1.9145\color{black} $\,\,$ & $\,\,$\color{gr} 5.2519\color{black} $\,\,$ & $\,\,$\color{gr} \color{blue} 6\color{black} $\,\,$ \\
$\,\,$\color{gr} 0.5223\color{black} $\,\,$ & $\,\,$ 1 $\,\,$ & $\,\,$2.7432$\,\,$ & $\,\,$3.1340  $\,\,$ \\
$\,\,$\color{gr} 0.1904\color{black} $\,\,$ & $\,\,$0.3645$\,\,$ & $\,\,$ 1 $\,\,$ & $\,\,$1.1425 $\,\,$ \\
$\,\,$\color{gr} \color{blue}  1/6\color{black} $\,\,$ & $\,\,$0.3191$\,\,$ & $\,\,$0.8753$\,\,$ & $\,\,$ 1  $\,\,$ \\
\end{pmatrix},
\end{equation*}
\end{example}
\newpage
\begin{example}
\begin{equation*}
\mathbf{A} =
\begin{pmatrix}
$\,\,$ 1 $\,\,$ & $\,\,$2$\,\,$ & $\,\,$6$\,\,$ & $\,\,$7 $\,\,$ \\
$\,\,$ 1/2$\,\,$ & $\,\,$ 1 $\,\,$ & $\,\,$2$\,\,$ & $\,\,$5 $\,\,$ \\
$\,\,$ 1/6$\,\,$ & $\,\,$ 1/2$\,\,$ & $\,\,$ 1 $\,\,$ & $\,\,$5 $\,\,$ \\
$\,\,$ 1/7$\,\,$ & $\,\,$ 1/5$\,\,$ & $\,\,$ 1/5$\,\,$ & $\,\,$ 1  $\,\,$ \\
\end{pmatrix},
\qquad
\lambda_{\max} =
4.1868,
\qquad
CR = 0.0704
\end{equation*}

\begin{equation*}
\mathbf{w}^{cos} =
\begin{pmatrix}
0.526013\\
\color{red} 0.262059\color{black} \\
0.158522\\
0.053406
\end{pmatrix}\end{equation*}
\begin{equation*}
\left[ \frac{{w}^{cos}_i}{{w}^{cos}_j} \right] =
\begin{pmatrix}
$\,\,$ 1 $\,\,$ & $\,\,$\color{red} 2.0072\color{black} $\,\,$ & $\,\,$3.3182$\,\,$ & $\,\,$9.8494$\,\,$ \\
$\,\,$\color{red} 0.4982\color{black} $\,\,$ & $\,\,$ 1 $\,\,$ & $\,\,$\color{red} 1.6531\color{black} $\,\,$ & $\,\,$\color{red} 4.9069\color{black}   $\,\,$ \\
$\,\,$0.3014$\,\,$ & $\,\,$\color{red} 0.6049\color{black} $\,\,$ & $\,\,$ 1 $\,\,$ & $\,\,$2.9683 $\,\,$ \\
$\,\,$0.1015$\,\,$ & $\,\,$\color{red} 0.2038\color{black} $\,\,$ & $\,\,$0.3369$\,\,$ & $\,\,$ 1  $\,\,$ \\
\end{pmatrix},
\end{equation*}

\begin{equation*}
\mathbf{w}^{\prime} =
\begin{pmatrix}
0.525515\\
0.262758\\
0.158372\\
0.053355
\end{pmatrix} =
0.999053\cdot
\begin{pmatrix}
0.526013\\
\color{gr} 0.263007\color{black} \\
0.158522\\
0.053406
\end{pmatrix},
\end{equation*}
\begin{equation*}
\left[ \frac{{w}^{\prime}_i}{{w}^{\prime}_j} \right] =
\begin{pmatrix}
$\,\,$ 1 $\,\,$ & $\,\,$\color{gr} \color{blue} 2\color{black} $\,\,$ & $\,\,$3.3182$\,\,$ & $\,\,$9.8494$\,\,$ \\
$\,\,$\color{gr} \color{blue}  1/2\color{black} $\,\,$ & $\,\,$ 1 $\,\,$ & $\,\,$\color{gr} 1.6591\color{black} $\,\,$ & $\,\,$\color{gr} 4.9247\color{black}   $\,\,$ \\
$\,\,$0.3014$\,\,$ & $\,\,$\color{gr} 0.6027\color{black} $\,\,$ & $\,\,$ 1 $\,\,$ & $\,\,$2.9683 $\,\,$ \\
$\,\,$0.1015$\,\,$ & $\,\,$\color{gr} 0.2031\color{black} $\,\,$ & $\,\,$0.3369$\,\,$ & $\,\,$ 1  $\,\,$ \\
\end{pmatrix},
\end{equation*}
\end{example}
\newpage
\begin{example}
\begin{equation*}
\mathbf{A} =
\begin{pmatrix}
$\,\,$ 1 $\,\,$ & $\,\,$2$\,\,$ & $\,\,$6$\,\,$ & $\,\,$7 $\,\,$ \\
$\,\,$ 1/2$\,\,$ & $\,\,$ 1 $\,\,$ & $\,\,$2$\,\,$ & $\,\,$5 $\,\,$ \\
$\,\,$ 1/6$\,\,$ & $\,\,$ 1/2$\,\,$ & $\,\,$ 1 $\,\,$ & $\,\,$6 $\,\,$ \\
$\,\,$ 1/7$\,\,$ & $\,\,$ 1/5$\,\,$ & $\,\,$ 1/6$\,\,$ & $\,\,$ 1  $\,\,$ \\
\end{pmatrix},
\qquad
\lambda_{\max} =
4.2439,
\qquad
CR = 0.0920
\end{equation*}

\begin{equation*}
\mathbf{w}^{cos} =
\begin{pmatrix}
0.520631\\
\color{red} 0.258178\color{black} \\
0.169371\\
0.051820
\end{pmatrix}\end{equation*}
\begin{equation*}
\left[ \frac{{w}^{cos}_i}{{w}^{cos}_j} \right] =
\begin{pmatrix}
$\,\,$ 1 $\,\,$ & $\,\,$\color{red} 2.0166\color{black} $\,\,$ & $\,\,$3.0739$\,\,$ & $\,\,$10.0469$\,\,$ \\
$\,\,$\color{red} 0.4959\color{black} $\,\,$ & $\,\,$ 1 $\,\,$ & $\,\,$\color{red} 1.5243\color{black} $\,\,$ & $\,\,$\color{red} 4.9822\color{black}   $\,\,$ \\
$\,\,$0.3253$\,\,$ & $\,\,$\color{red} 0.6560\color{black} $\,\,$ & $\,\,$ 1 $\,\,$ & $\,\,$3.2684 $\,\,$ \\
$\,\,$0.0995$\,\,$ & $\,\,$\color{red} 0.2007\color{black} $\,\,$ & $\,\,$0.3060$\,\,$ & $\,\,$ 1  $\,\,$ \\
\end{pmatrix},
\end{equation*}

\begin{equation*}
\mathbf{w}^{\prime} =
\begin{pmatrix}
0.520151\\
0.258862\\
0.169215\\
0.051772
\end{pmatrix} =
0.999079\cdot
\begin{pmatrix}
0.520631\\
\color{gr} 0.259100\color{black} \\
0.169371\\
0.051820
\end{pmatrix},
\end{equation*}
\begin{equation*}
\left[ \frac{{w}^{\prime}_i}{{w}^{\prime}_j} \right] =
\begin{pmatrix}
$\,\,$ 1 $\,\,$ & $\,\,$\color{gr} 2.0094\color{black} $\,\,$ & $\,\,$3.0739$\,\,$ & $\,\,$10.0469$\,\,$ \\
$\,\,$\color{gr} 0.4977\color{black} $\,\,$ & $\,\,$ 1 $\,\,$ & $\,\,$\color{gr} 1.5298\color{black} $\,\,$ & $\,\,$\color{gr} \color{blue} 5\color{black}   $\,\,$ \\
$\,\,$0.3253$\,\,$ & $\,\,$\color{gr} 0.6537\color{black} $\,\,$ & $\,\,$ 1 $\,\,$ & $\,\,$3.2684 $\,\,$ \\
$\,\,$0.0995$\,\,$ & $\,\,$\color{gr} \color{blue}  1/5\color{black} $\,\,$ & $\,\,$0.3060$\,\,$ & $\,\,$ 1  $\,\,$ \\
\end{pmatrix},
\end{equation*}
\end{example}
\newpage
\begin{example}
\begin{equation*}
\mathbf{A} =
\begin{pmatrix}
$\,\,$ 1 $\,\,$ & $\,\,$2$\,\,$ & $\,\,$6$\,\,$ & $\,\,$7 $\,\,$ \\
$\,\,$ 1/2$\,\,$ & $\,\,$ 1 $\,\,$ & $\,\,$5$\,\,$ & $\,\,$2 $\,\,$ \\
$\,\,$ 1/6$\,\,$ & $\,\,$ 1/5$\,\,$ & $\,\,$ 1 $\,\,$ & $\,\,$2 $\,\,$ \\
$\,\,$ 1/7$\,\,$ & $\,\,$ 1/2$\,\,$ & $\,\,$ 1/2$\,\,$ & $\,\,$ 1  $\,\,$ \\
\end{pmatrix},
\qquad
\lambda_{\max} =
4.2251,
\qquad
CR = 0.0849
\end{equation*}

\begin{equation*}
\mathbf{w}^{cos} =
\begin{pmatrix}
\color{red} 0.539080\color{black} \\
0.278433\\
0.097940\\
0.084547
\end{pmatrix}\end{equation*}
\begin{equation*}
\left[ \frac{{w}^{cos}_i}{{w}^{cos}_j} \right] =
\begin{pmatrix}
$\,\,$ 1 $\,\,$ & $\,\,$\color{red} 1.9361\color{black} $\,\,$ & $\,\,$\color{red} 5.5042\color{black} $\,\,$ & $\,\,$\color{red} 6.3761\color{black} $\,\,$ \\
$\,\,$\color{red} 0.5165\color{black} $\,\,$ & $\,\,$ 1 $\,\,$ & $\,\,$2.8429$\,\,$ & $\,\,$3.2932  $\,\,$ \\
$\,\,$\color{red} 0.1817\color{black} $\,\,$ & $\,\,$0.3518$\,\,$ & $\,\,$ 1 $\,\,$ & $\,\,$1.1584 $\,\,$ \\
$\,\,$\color{red} 0.1568\color{black} $\,\,$ & $\,\,$0.3037$\,\,$ & $\,\,$0.8632$\,\,$ & $\,\,$ 1  $\,\,$ \\
\end{pmatrix},
\end{equation*}

\begin{equation*}
\mathbf{w}^{\prime} =
\begin{pmatrix}
0.547134\\
0.273567\\
0.096229\\
0.083069
\end{pmatrix} =
0.982525\cdot
\begin{pmatrix}
\color{gr} 0.556865\color{black} \\
0.278433\\
0.097940\\
0.084547
\end{pmatrix},
\end{equation*}
\begin{equation*}
\left[ \frac{{w}^{\prime}_i}{{w}^{\prime}_j} \right] =
\begin{pmatrix}
$\,\,$ 1 $\,\,$ & $\,\,$\color{gr} \color{blue} 2\color{black} $\,\,$ & $\,\,$\color{gr} 5.6858\color{black} $\,\,$ & $\,\,$\color{gr} 6.5865\color{black} $\,\,$ \\
$\,\,$\color{gr} \color{blue}  1/2\color{black} $\,\,$ & $\,\,$ 1 $\,\,$ & $\,\,$2.8429$\,\,$ & $\,\,$3.2932  $\,\,$ \\
$\,\,$\color{gr} 0.1759\color{black} $\,\,$ & $\,\,$0.3518$\,\,$ & $\,\,$ 1 $\,\,$ & $\,\,$1.1584 $\,\,$ \\
$\,\,$\color{gr} 0.1518\color{black} $\,\,$ & $\,\,$0.3037$\,\,$ & $\,\,$0.8632$\,\,$ & $\,\,$ 1  $\,\,$ \\
\end{pmatrix},
\end{equation*}
\end{example}
\newpage
\begin{example}
\begin{equation*}
\mathbf{A} =
\begin{pmatrix}
$\,\,$ 1 $\,\,$ & $\,\,$2$\,\,$ & $\,\,$6$\,\,$ & $\,\,$8 $\,\,$ \\
$\,\,$ 1/2$\,\,$ & $\,\,$ 1 $\,\,$ & $\,\,$1$\,\,$ & $\,\,$3 $\,\,$ \\
$\,\,$ 1/6$\,\,$ & $\,\,$ 1 $\,\,$ & $\,\,$ 1 $\,\,$ & $\,\,$2 $\,\,$ \\
$\,\,$ 1/8$\,\,$ & $\,\,$ 1/3$\,\,$ & $\,\,$ 1/2$\,\,$ & $\,\,$ 1  $\,\,$ \\
\end{pmatrix},
\qquad
\lambda_{\max} =
4.1031,
\qquad
CR = 0.0389
\end{equation*}

\begin{equation*}
\mathbf{w}^{cos} =
\begin{pmatrix}
0.566852\\
0.213990\\
0.149369\\
\color{red} 0.069789\color{black}
\end{pmatrix}\end{equation*}
\begin{equation*}
\left[ \frac{{w}^{cos}_i}{{w}^{cos}_j} \right] =
\begin{pmatrix}
$\,\,$ 1 $\,\,$ & $\,\,$2.6490$\,\,$ & $\,\,$3.7950$\,\,$ & $\,\,$\color{red} 8.1224\color{black} $\,\,$ \\
$\,\,$0.3775$\,\,$ & $\,\,$ 1 $\,\,$ & $\,\,$1.4326$\,\,$ & $\,\,$\color{red} 3.0662\color{black}   $\,\,$ \\
$\,\,$0.2635$\,\,$ & $\,\,$0.6980$\,\,$ & $\,\,$ 1 $\,\,$ & $\,\,$\color{red} 2.1403\color{black}  $\,\,$ \\
$\,\,$\color{red} 0.1231\color{black} $\,\,$ & $\,\,$\color{red} 0.3261\color{black} $\,\,$ & $\,\,$\color{red} 0.4672\color{black} $\,\,$ & $\,\,$ 1  $\,\,$ \\
\end{pmatrix},
\end{equation*}

\begin{equation*}
\mathbf{w}^{\prime} =
\begin{pmatrix}
0.566248\\
0.213761\\
0.149210\\
0.070781
\end{pmatrix} =
0.998934\cdot
\begin{pmatrix}
0.566852\\
0.213990\\
0.149369\\
\color{gr} 0.070857\color{black}
\end{pmatrix},
\end{equation*}
\begin{equation*}
\left[ \frac{{w}^{\prime}_i}{{w}^{\prime}_j} \right] =
\begin{pmatrix}
$\,\,$ 1 $\,\,$ & $\,\,$2.6490$\,\,$ & $\,\,$3.7950$\,\,$ & $\,\,$\color{gr} \color{blue} 8\color{black} $\,\,$ \\
$\,\,$0.3775$\,\,$ & $\,\,$ 1 $\,\,$ & $\,\,$1.4326$\,\,$ & $\,\,$\color{gr} 3.0200\color{black}   $\,\,$ \\
$\,\,$0.2635$\,\,$ & $\,\,$0.6980$\,\,$ & $\,\,$ 1 $\,\,$ & $\,\,$\color{gr} 2.1081\color{black}  $\,\,$ \\
$\,\,$\color{gr} \color{blue}  1/8\color{black} $\,\,$ & $\,\,$\color{gr} 0.3311\color{black} $\,\,$ & $\,\,$\color{gr} 0.4744\color{black} $\,\,$ & $\,\,$ 1  $\,\,$ \\
\end{pmatrix},
\end{equation*}
\end{example}
\newpage
\begin{example}
\begin{equation*}
\mathbf{A} =
\begin{pmatrix}
$\,\,$ 1 $\,\,$ & $\,\,$2$\,\,$ & $\,\,$6$\,\,$ & $\,\,$8 $\,\,$ \\
$\,\,$ 1/2$\,\,$ & $\,\,$ 1 $\,\,$ & $\,\,$5$\,\,$ & $\,\,$2 $\,\,$ \\
$\,\,$ 1/6$\,\,$ & $\,\,$ 1/5$\,\,$ & $\,\,$ 1 $\,\,$ & $\,\,$2 $\,\,$ \\
$\,\,$ 1/8$\,\,$ & $\,\,$ 1/2$\,\,$ & $\,\,$ 1/2$\,\,$ & $\,\,$ 1  $\,\,$ \\
\end{pmatrix},
\qquad
\lambda_{\max} =
4.2277,
\qquad
CR = 0.0859
\end{equation*}

\begin{equation*}
\mathbf{w}^{cos} =
\begin{pmatrix}
\color{red} 0.547804\color{black} \\
0.276988\\
0.094489\\
0.080719
\end{pmatrix}\end{equation*}
\begin{equation*}
\left[ \frac{{w}^{cos}_i}{{w}^{cos}_j} \right] =
\begin{pmatrix}
$\,\,$ 1 $\,\,$ & $\,\,$\color{red} 1.9777\color{black} $\,\,$ & $\,\,$\color{red} 5.7975\color{black} $\,\,$ & $\,\,$\color{red} 6.7866\color{black} $\,\,$ \\
$\,\,$\color{red} 0.5056\color{black} $\,\,$ & $\,\,$ 1 $\,\,$ & $\,\,$2.9314$\,\,$ & $\,\,$3.4315  $\,\,$ \\
$\,\,$\color{red} 0.1725\color{black} $\,\,$ & $\,\,$0.3411$\,\,$ & $\,\,$ 1 $\,\,$ & $\,\,$1.1706 $\,\,$ \\
$\,\,$\color{red} 0.1473\color{black} $\,\,$ & $\,\,$0.2914$\,\,$ & $\,\,$0.8543$\,\,$ & $\,\,$ 1  $\,\,$ \\
\end{pmatrix},
\end{equation*}

\begin{equation*}
\mathbf{w}^{\prime} =
\begin{pmatrix}
0.550578\\
0.275289\\
0.093910\\
0.080224
\end{pmatrix} =
0.993867\cdot
\begin{pmatrix}
\color{gr} 0.553975\color{black} \\
0.276988\\
0.094489\\
0.080719
\end{pmatrix},
\end{equation*}
\begin{equation*}
\left[ \frac{{w}^{\prime}_i}{{w}^{\prime}_j} \right] =
\begin{pmatrix}
$\,\,$ 1 $\,\,$ & $\,\,$\color{gr} \color{blue} 2\color{black} $\,\,$ & $\,\,$\color{gr} 5.8628\color{black} $\,\,$ & $\,\,$\color{gr} 6.8630\color{black} $\,\,$ \\
$\,\,$\color{gr} \color{blue}  1/2\color{black} $\,\,$ & $\,\,$ 1 $\,\,$ & $\,\,$2.9314$\,\,$ & $\,\,$3.4315  $\,\,$ \\
$\,\,$\color{gr} 0.1706\color{black} $\,\,$ & $\,\,$0.3411$\,\,$ & $\,\,$ 1 $\,\,$ & $\,\,$1.1706 $\,\,$ \\
$\,\,$\color{gr} 0.1457\color{black} $\,\,$ & $\,\,$0.2914$\,\,$ & $\,\,$0.8543$\,\,$ & $\,\,$ 1  $\,\,$ \\
\end{pmatrix},
\end{equation*}
\end{example}
\newpage
\begin{example}
\begin{equation*}
\mathbf{A} =
\begin{pmatrix}
$\,\,$ 1 $\,\,$ & $\,\,$2$\,\,$ & $\,\,$6$\,\,$ & $\,\,$8 $\,\,$ \\
$\,\,$ 1/2$\,\,$ & $\,\,$ 1 $\,\,$ & $\,\,$9$\,\,$ & $\,\,$6 $\,\,$ \\
$\,\,$ 1/6$\,\,$ & $\,\,$ 1/9$\,\,$ & $\,\,$ 1 $\,\,$ & $\,\,$1 $\,\,$ \\
$\,\,$ 1/8$\,\,$ & $\,\,$ 1/6$\,\,$ & $\,\,$ 1 $\,\,$ & $\,\,$ 1  $\,\,$ \\
\end{pmatrix},
\qquad
\lambda_{\max} =
4.1031,
\qquad
CR = 0.0389
\end{equation*}

\begin{equation*}
\mathbf{w}^{cos} =
\begin{pmatrix}
0.503766\\
0.372965\\
0.062597\\
\color{red} 0.060672\color{black}
\end{pmatrix}\end{equation*}
\begin{equation*}
\left[ \frac{{w}^{cos}_i}{{w}^{cos}_j} \right] =
\begin{pmatrix}
$\,\,$ 1 $\,\,$ & $\,\,$1.3507$\,\,$ & $\,\,$8.0478$\,\,$ & $\,\,$\color{red} 8.3031\color{black} $\,\,$ \\
$\,\,$0.7404$\,\,$ & $\,\,$ 1 $\,\,$ & $\,\,$5.9582$\,\,$ & $\,\,$\color{red} 6.1472\color{black}   $\,\,$ \\
$\,\,$0.1243$\,\,$ & $\,\,$0.1678$\,\,$ & $\,\,$ 1 $\,\,$ & $\,\,$\color{red} 1.0317\color{black}  $\,\,$ \\
$\,\,$\color{red} 0.1204\color{black} $\,\,$ & $\,\,$\color{red} 0.1627\color{black} $\,\,$ & $\,\,$\color{red} 0.9693\color{black} $\,\,$ & $\,\,$ 1  $\,\,$ \\
\end{pmatrix},
\end{equation*}

\begin{equation*}
\mathbf{w}^{\prime} =
\begin{pmatrix}
0.503017\\
0.372411\\
0.062504\\
0.062068
\end{pmatrix} =
0.998513\cdot
\begin{pmatrix}
0.503766\\
0.372965\\
0.062597\\
\color{gr} 0.062161\color{black}
\end{pmatrix},
\end{equation*}
\begin{equation*}
\left[ \frac{{w}^{\prime}_i}{{w}^{\prime}_j} \right] =
\begin{pmatrix}
$\,\,$ 1 $\,\,$ & $\,\,$1.3507$\,\,$ & $\,\,$8.0478$\,\,$ & $\,\,$\color{gr} 8.1042\color{black} $\,\,$ \\
$\,\,$0.7404$\,\,$ & $\,\,$ 1 $\,\,$ & $\,\,$5.9582$\,\,$ & $\,\,$\color{gr} \color{blue} 6\color{black}   $\,\,$ \\
$\,\,$0.1243$\,\,$ & $\,\,$0.1678$\,\,$ & $\,\,$ 1 $\,\,$ & $\,\,$\color{gr} 1.0070\color{black}  $\,\,$ \\
$\,\,$\color{gr} 0.1234\color{black} $\,\,$ & $\,\,$\color{gr} \color{blue}  1/6\color{black} $\,\,$ & $\,\,$\color{gr} 0.9930\color{black} $\,\,$ & $\,\,$ 1  $\,\,$ \\
\end{pmatrix},
\end{equation*}
\end{example}
\newpage
\begin{example}
\begin{equation*}
\mathbf{A} =
\begin{pmatrix}
$\,\,$ 1 $\,\,$ & $\,\,$2$\,\,$ & $\,\,$7$\,\,$ & $\,\,$3 $\,\,$ \\
$\,\,$ 1/2$\,\,$ & $\,\,$ 1 $\,\,$ & $\,\,$2$\,\,$ & $\,\,$2 $\,\,$ \\
$\,\,$ 1/7$\,\,$ & $\,\,$ 1/2$\,\,$ & $\,\,$ 1 $\,\,$ & $\,\,$2 $\,\,$ \\
$\,\,$ 1/3$\,\,$ & $\,\,$ 1/2$\,\,$ & $\,\,$ 1/2$\,\,$ & $\,\,$ 1  $\,\,$ \\
\end{pmatrix},
\qquad
\lambda_{\max} =
4.2109,
\qquad
CR = 0.0795
\end{equation*}

\begin{equation*}
\mathbf{w}^{cos} =
\begin{pmatrix}
0.501795\\
\color{red} 0.238087\color{black} \\
0.140624\\
0.119494
\end{pmatrix}\end{equation*}
\begin{equation*}
\left[ \frac{{w}^{cos}_i}{{w}^{cos}_j} \right] =
\begin{pmatrix}
$\,\,$ 1 $\,\,$ & $\,\,$\color{red} 2.1076\color{black} $\,\,$ & $\,\,$3.5683$\,\,$ & $\,\,$4.1993$\,\,$ \\
$\,\,$\color{red} 0.4745\color{black} $\,\,$ & $\,\,$ 1 $\,\,$ & $\,\,$\color{red} 1.6931\color{black} $\,\,$ & $\,\,$\color{red} 1.9925\color{black}   $\,\,$ \\
$\,\,$0.2802$\,\,$ & $\,\,$\color{red} 0.5906\color{black} $\,\,$ & $\,\,$ 1 $\,\,$ & $\,\,$1.1768 $\,\,$ \\
$\,\,$0.2381$\,\,$ & $\,\,$\color{red} 0.5019\color{black} $\,\,$ & $\,\,$0.8497$\,\,$ & $\,\,$ 1  $\,\,$ \\
\end{pmatrix},
\end{equation*}

\begin{equation*}
\mathbf{w}^{\prime} =
\begin{pmatrix}
0.501344\\
0.238773\\
0.140497\\
0.119386
\end{pmatrix} =
0.999101\cdot
\begin{pmatrix}
0.501795\\
\color{gr} 0.238988\color{black} \\
0.140624\\
0.119494
\end{pmatrix},
\end{equation*}
\begin{equation*}
\left[ \frac{{w}^{\prime}_i}{{w}^{\prime}_j} \right] =
\begin{pmatrix}
$\,\,$ 1 $\,\,$ & $\,\,$\color{gr} 2.0997\color{black} $\,\,$ & $\,\,$3.5683$\,\,$ & $\,\,$4.1993$\,\,$ \\
$\,\,$\color{gr} 0.4763\color{black} $\,\,$ & $\,\,$ 1 $\,\,$ & $\,\,$\color{gr} 1.6995\color{black} $\,\,$ & $\,\,$\color{gr} \color{blue} 2\color{black}   $\,\,$ \\
$\,\,$0.2802$\,\,$ & $\,\,$\color{gr} 0.5884\color{black} $\,\,$ & $\,\,$ 1 $\,\,$ & $\,\,$1.1768 $\,\,$ \\
$\,\,$0.2381$\,\,$ & $\,\,$\color{gr} \color{blue}  1/2\color{black} $\,\,$ & $\,\,$0.8497$\,\,$ & $\,\,$ 1  $\,\,$ \\
\end{pmatrix},
\end{equation*}
\end{example}
\newpage
\begin{example}
\begin{equation*}
\mathbf{A} =
\begin{pmatrix}
$\,\,$ 1 $\,\,$ & $\,\,$2$\,\,$ & $\,\,$7$\,\,$ & $\,\,$4 $\,\,$ \\
$\,\,$ 1/2$\,\,$ & $\,\,$ 1 $\,\,$ & $\,\,$2$\,\,$ & $\,\,$3 $\,\,$ \\
$\,\,$ 1/7$\,\,$ & $\,\,$ 1/2$\,\,$ & $\,\,$ 1 $\,\,$ & $\,\,$3 $\,\,$ \\
$\,\,$ 1/4$\,\,$ & $\,\,$ 1/3$\,\,$ & $\,\,$ 1/3$\,\,$ & $\,\,$ 1  $\,\,$ \\
\end{pmatrix},
\qquad
\lambda_{\max} =
4.2421,
\qquad
CR = 0.0913
\end{equation*}

\begin{equation*}
\mathbf{w}^{cos} =
\begin{pmatrix}
0.511903\\
\color{red} 0.250633\color{black} \\
0.149823\\
0.087640
\end{pmatrix}\end{equation*}
\begin{equation*}
\left[ \frac{{w}^{cos}_i}{{w}^{cos}_j} \right] =
\begin{pmatrix}
$\,\,$ 1 $\,\,$ & $\,\,$\color{red} 2.0424\color{black} $\,\,$ & $\,\,$3.4167$\,\,$ & $\,\,$5.8410$\,\,$ \\
$\,\,$\color{red} 0.4896\color{black} $\,\,$ & $\,\,$ 1 $\,\,$ & $\,\,$\color{red} 1.6729\color{black} $\,\,$ & $\,\,$\color{red} 2.8598\color{black}   $\,\,$ \\
$\,\,$0.2927$\,\,$ & $\,\,$\color{red} 0.5978\color{black} $\,\,$ & $\,\,$ 1 $\,\,$ & $\,\,$1.7095 $\,\,$ \\
$\,\,$0.1712$\,\,$ & $\,\,$\color{red} 0.3497\color{black} $\,\,$ & $\,\,$0.5850$\,\,$ & $\,\,$ 1  $\,\,$ \\
\end{pmatrix},
\end{equation*}

\begin{equation*}
\mathbf{w}^{\prime} =
\begin{pmatrix}
0.509195\\
0.254598\\
0.149031\\
0.087177
\end{pmatrix} =
0.994710\cdot
\begin{pmatrix}
0.511903\\
\color{gr} 0.255952\color{black} \\
0.149823\\
0.087640
\end{pmatrix},
\end{equation*}
\begin{equation*}
\left[ \frac{{w}^{\prime}_i}{{w}^{\prime}_j} \right] =
\begin{pmatrix}
$\,\,$ 1 $\,\,$ & $\,\,$\color{gr} \color{blue} 2\color{black} $\,\,$ & $\,\,$3.4167$\,\,$ & $\,\,$5.8410$\,\,$ \\
$\,\,$\color{gr} \color{blue}  1/2\color{black} $\,\,$ & $\,\,$ 1 $\,\,$ & $\,\,$\color{gr} 1.7084\color{black} $\,\,$ & $\,\,$\color{gr} 2.9205\color{black}   $\,\,$ \\
$\,\,$0.2927$\,\,$ & $\,\,$\color{gr} 0.5854\color{black} $\,\,$ & $\,\,$ 1 $\,\,$ & $\,\,$1.7095 $\,\,$ \\
$\,\,$0.1712$\,\,$ & $\,\,$\color{gr} 0.3424\color{black} $\,\,$ & $\,\,$0.5850$\,\,$ & $\,\,$ 1  $\,\,$ \\
\end{pmatrix},
\end{equation*}
\end{example}
\newpage
\begin{example}
\begin{equation*}
\mathbf{A} =
\begin{pmatrix}
$\,\,$ 1 $\,\,$ & $\,\,$2$\,\,$ & $\,\,$7$\,\,$ & $\,\,$5 $\,\,$ \\
$\,\,$ 1/2$\,\,$ & $\,\,$ 1 $\,\,$ & $\,\,$2$\,\,$ & $\,\,$4 $\,\,$ \\
$\,\,$ 1/7$\,\,$ & $\,\,$ 1/2$\,\,$ & $\,\,$ 1 $\,\,$ & $\,\,$4 $\,\,$ \\
$\,\,$ 1/5$\,\,$ & $\,\,$ 1/4$\,\,$ & $\,\,$ 1/4$\,\,$ & $\,\,$ 1  $\,\,$ \\
\end{pmatrix},
\qquad
\lambda_{\max} =
4.2610,
\qquad
CR = 0.0984
\end{equation*}

\begin{equation*}
\mathbf{w}^{cos} =
\begin{pmatrix}
0.517818\\
\color{red} 0.257820\color{black} \\
0.155036\\
0.069327
\end{pmatrix}\end{equation*}
\begin{equation*}
\left[ \frac{{w}^{cos}_i}{{w}^{cos}_j} \right] =
\begin{pmatrix}
$\,\,$ 1 $\,\,$ & $\,\,$\color{red} 2.0084\color{black} $\,\,$ & $\,\,$3.3400$\,\,$ & $\,\,$7.4692$\,\,$ \\
$\,\,$\color{red} 0.4979\color{black} $\,\,$ & $\,\,$ 1 $\,\,$ & $\,\,$\color{red} 1.6630\color{black} $\,\,$ & $\,\,$\color{red} 3.7189\color{black}   $\,\,$ \\
$\,\,$0.2994$\,\,$ & $\,\,$\color{red} 0.6013\color{black} $\,\,$ & $\,\,$ 1 $\,\,$ & $\,\,$2.2363 $\,\,$ \\
$\,\,$0.1339$\,\,$ & $\,\,$\color{red} 0.2689\color{black} $\,\,$ & $\,\,$0.4472$\,\,$ & $\,\,$ 1  $\,\,$ \\
\end{pmatrix},
\end{equation*}

\begin{equation*}
\mathbf{w}^{\prime} =
\begin{pmatrix}
0.517254\\
0.258627\\
0.154868\\
0.069251
\end{pmatrix} =
0.998912\cdot
\begin{pmatrix}
0.517818\\
\color{gr} 0.258909\color{black} \\
0.155036\\
0.069327
\end{pmatrix},
\end{equation*}
\begin{equation*}
\left[ \frac{{w}^{\prime}_i}{{w}^{\prime}_j} \right] =
\begin{pmatrix}
$\,\,$ 1 $\,\,$ & $\,\,$\color{gr} \color{blue} 2\color{black} $\,\,$ & $\,\,$3.3400$\,\,$ & $\,\,$7.4692$\,\,$ \\
$\,\,$\color{gr} \color{blue}  1/2\color{black} $\,\,$ & $\,\,$ 1 $\,\,$ & $\,\,$\color{gr} 1.6700\color{black} $\,\,$ & $\,\,$\color{gr} 3.7346\color{black}   $\,\,$ \\
$\,\,$0.2994$\,\,$ & $\,\,$\color{gr} 0.5988\color{black} $\,\,$ & $\,\,$ 1 $\,\,$ & $\,\,$2.2363 $\,\,$ \\
$\,\,$0.1339$\,\,$ & $\,\,$\color{gr} 0.2678\color{black} $\,\,$ & $\,\,$0.4472$\,\,$ & $\,\,$ 1  $\,\,$ \\
\end{pmatrix},
\end{equation*}
\end{example}
\newpage
\begin{example}
\begin{equation*}
\mathbf{A} =
\begin{pmatrix}
$\,\,$ 1 $\,\,$ & $\,\,$2$\,\,$ & $\,\,$7$\,\,$ & $\,\,$5 $\,\,$ \\
$\,\,$ 1/2$\,\,$ & $\,\,$ 1 $\,\,$ & $\,\,$5$\,\,$ & $\,\,$7 $\,\,$ \\
$\,\,$ 1/7$\,\,$ & $\,\,$ 1/5$\,\,$ & $\,\,$ 1 $\,\,$ & $\,\,$1 $\,\,$ \\
$\,\,$ 1/5$\,\,$ & $\,\,$ 1/7$\,\,$ & $\,\,$ 1 $\,\,$ & $\,\,$ 1  $\,\,$ \\
\end{pmatrix},
\qquad
\lambda_{\max} =
4.0899,
\qquad
CR = 0.0339
\end{equation*}

\begin{equation*}
\mathbf{w}^{cos} =
\begin{pmatrix}
0.497667\\
0.357946\\
\color{red} 0.070250\color{black} \\
0.074137
\end{pmatrix}\end{equation*}
\begin{equation*}
\left[ \frac{{w}^{cos}_i}{{w}^{cos}_j} \right] =
\begin{pmatrix}
$\,\,$ 1 $\,\,$ & $\,\,$1.3903$\,\,$ & $\,\,$\color{red} 7.0842\color{black} $\,\,$ & $\,\,$6.7128$\,\,$ \\
$\,\,$0.7192$\,\,$ & $\,\,$ 1 $\,\,$ & $\,\,$\color{red} 5.0953\color{black} $\,\,$ & $\,\,$4.8282  $\,\,$ \\
$\,\,$\color{red} 0.1412\color{black} $\,\,$ & $\,\,$\color{red} 0.1963\color{black} $\,\,$ & $\,\,$ 1 $\,\,$ & $\,\,$\color{red} 0.9476\color{black}  $\,\,$ \\
$\,\,$0.1490$\,\,$ & $\,\,$0.2071$\,\,$ & $\,\,$\color{red} 1.0553\color{black} $\,\,$ & $\,\,$ 1  $\,\,$ \\
\end{pmatrix},
\end{equation*}

\begin{equation*}
\mathbf{w}^{\prime} =
\begin{pmatrix}
0.497247\\
0.357643\\
0.071035\\
0.074074
\end{pmatrix} =
0.999156\cdot
\begin{pmatrix}
0.497667\\
0.357946\\
\color{gr} 0.071095\color{black} \\
0.074137
\end{pmatrix},
\end{equation*}
\begin{equation*}
\left[ \frac{{w}^{\prime}_i}{{w}^{\prime}_j} \right] =
\begin{pmatrix}
$\,\,$ 1 $\,\,$ & $\,\,$1.3903$\,\,$ & $\,\,$\color{gr} \color{blue} 7\color{black} $\,\,$ & $\,\,$6.7128$\,\,$ \\
$\,\,$0.7192$\,\,$ & $\,\,$ 1 $\,\,$ & $\,\,$\color{gr} 5.0347\color{black} $\,\,$ & $\,\,$4.8282  $\,\,$ \\
$\,\,$\color{gr} \color{blue}  1/7\color{black} $\,\,$ & $\,\,$\color{gr} 0.1986\color{black} $\,\,$ & $\,\,$ 1 $\,\,$ & $\,\,$\color{gr} 0.9590\color{black}  $\,\,$ \\
$\,\,$0.1490$\,\,$ & $\,\,$0.2071$\,\,$ & $\,\,$\color{gr} 1.0428\color{black} $\,\,$ & $\,\,$ 1  $\,\,$ \\
\end{pmatrix},
\end{equation*}
\end{example}
\newpage
\begin{example}
\begin{equation*}
\mathbf{A} =
\begin{pmatrix}
$\,\,$ 1 $\,\,$ & $\,\,$2$\,\,$ & $\,\,$7$\,\,$ & $\,\,$6 $\,\,$ \\
$\,\,$ 1/2$\,\,$ & $\,\,$ 1 $\,\,$ & $\,\,$2$\,\,$ & $\,\,$4 $\,\,$ \\
$\,\,$ 1/7$\,\,$ & $\,\,$ 1/2$\,\,$ & $\,\,$ 1 $\,\,$ & $\,\,$3 $\,\,$ \\
$\,\,$ 1/6$\,\,$ & $\,\,$ 1/4$\,\,$ & $\,\,$ 1/3$\,\,$ & $\,\,$ 1  $\,\,$ \\
\end{pmatrix},
\qquad
\lambda_{\max} =
4.1365,
\qquad
CR = 0.0515
\end{equation*}

\begin{equation*}
\mathbf{w}^{cos} =
\begin{pmatrix}
0.540646\\
\color{red} 0.258301\color{black} \\
0.134303\\
0.066750
\end{pmatrix}\end{equation*}
\begin{equation*}
\left[ \frac{{w}^{cos}_i}{{w}^{cos}_j} \right] =
\begin{pmatrix}
$\,\,$ 1 $\,\,$ & $\,\,$\color{red} 2.0931\color{black} $\,\,$ & $\,\,$4.0256$\,\,$ & $\,\,$8.0996$\,\,$ \\
$\,\,$\color{red} 0.4778\color{black} $\,\,$ & $\,\,$ 1 $\,\,$ & $\,\,$\color{red} 1.9233\color{black} $\,\,$ & $\,\,$\color{red} 3.8697\color{black}   $\,\,$ \\
$\,\,$0.2484$\,\,$ & $\,\,$\color{red} 0.5199\color{black} $\,\,$ & $\,\,$ 1 $\,\,$ & $\,\,$2.0120 $\,\,$ \\
$\,\,$0.1235$\,\,$ & $\,\,$\color{red} 0.2584\color{black} $\,\,$ & $\,\,$0.4970$\,\,$ & $\,\,$ 1  $\,\,$ \\
\end{pmatrix},
\end{equation*}

\begin{equation*}
\mathbf{w}^{\prime} =
\begin{pmatrix}
0.535984\\
0.264697\\
0.133145\\
0.066174
\end{pmatrix} =
0.991377\cdot
\begin{pmatrix}
0.540646\\
\color{gr} 0.266999\color{black} \\
0.134303\\
0.066750
\end{pmatrix},
\end{equation*}
\begin{equation*}
\left[ \frac{{w}^{\prime}_i}{{w}^{\prime}_j} \right] =
\begin{pmatrix}
$\,\,$ 1 $\,\,$ & $\,\,$\color{gr} 2.0249\color{black} $\,\,$ & $\,\,$4.0256$\,\,$ & $\,\,$8.0996$\,\,$ \\
$\,\,$\color{gr} 0.4939\color{black} $\,\,$ & $\,\,$ 1 $\,\,$ & $\,\,$\color{gr} 1.9880\color{black} $\,\,$ & $\,\,$\color{gr} \color{blue} 4\color{black}   $\,\,$ \\
$\,\,$0.2484$\,\,$ & $\,\,$\color{gr} 0.5030\color{black} $\,\,$ & $\,\,$ 1 $\,\,$ & $\,\,$2.0120 $\,\,$ \\
$\,\,$0.1235$\,\,$ & $\,\,$\color{gr} \color{blue}  1/4\color{black} $\,\,$ & $\,\,$0.4970$\,\,$ & $\,\,$ 1  $\,\,$ \\
\end{pmatrix},
\end{equation*}
\end{example}
\newpage
\begin{example}
\begin{equation*}
\mathbf{A} =
\begin{pmatrix}
$\,\,$ 1 $\,\,$ & $\,\,$2$\,\,$ & $\,\,$7$\,\,$ & $\,\,$6 $\,\,$ \\
$\,\,$ 1/2$\,\,$ & $\,\,$ 1 $\,\,$ & $\,\,$2$\,\,$ & $\,\,$4 $\,\,$ \\
$\,\,$ 1/7$\,\,$ & $\,\,$ 1/2$\,\,$ & $\,\,$ 1 $\,\,$ & $\,\,$4 $\,\,$ \\
$\,\,$ 1/6$\,\,$ & $\,\,$ 1/4$\,\,$ & $\,\,$ 1/4$\,\,$ & $\,\,$ 1  $\,\,$ \\
\end{pmatrix},
\qquad
\lambda_{\max} =
4.2109,
\qquad
CR = 0.0795
\end{equation*}

\begin{equation*}
\mathbf{w}^{cos} =
\begin{pmatrix}
0.533087\\
\color{red} 0.253508\color{black} \\
0.149628\\
0.063777
\end{pmatrix}\end{equation*}
\begin{equation*}
\left[ \frac{{w}^{cos}_i}{{w}^{cos}_j} \right] =
\begin{pmatrix}
$\,\,$ 1 $\,\,$ & $\,\,$\color{red} 2.1028\color{black} $\,\,$ & $\,\,$3.5627$\,\,$ & $\,\,$8.3585$\,\,$ \\
$\,\,$\color{red} 0.4755\color{black} $\,\,$ & $\,\,$ 1 $\,\,$ & $\,\,$\color{red} 1.6943\color{black} $\,\,$ & $\,\,$\color{red} 3.9749\color{black}   $\,\,$ \\
$\,\,$0.2807$\,\,$ & $\,\,$\color{red} 0.5902\color{black} $\,\,$ & $\,\,$ 1 $\,\,$ & $\,\,$2.3461 $\,\,$ \\
$\,\,$0.1196$\,\,$ & $\,\,$\color{red} 0.2516\color{black} $\,\,$ & $\,\,$0.4262$\,\,$ & $\,\,$ 1  $\,\,$ \\
\end{pmatrix},
\end{equation*}

\begin{equation*}
\mathbf{w}^{\prime} =
\begin{pmatrix}
0.532234\\
0.254702\\
0.149389\\
0.063675
\end{pmatrix} =
0.998401\cdot
\begin{pmatrix}
0.533087\\
\color{gr} 0.255110\color{black} \\
0.149628\\
0.063777
\end{pmatrix},
\end{equation*}
\begin{equation*}
\left[ \frac{{w}^{\prime}_i}{{w}^{\prime}_j} \right] =
\begin{pmatrix}
$\,\,$ 1 $\,\,$ & $\,\,$\color{gr} 2.0896\color{black} $\,\,$ & $\,\,$3.5627$\,\,$ & $\,\,$8.3585$\,\,$ \\
$\,\,$\color{gr} 0.4786\color{black} $\,\,$ & $\,\,$ 1 $\,\,$ & $\,\,$\color{gr} 1.7050\color{black} $\,\,$ & $\,\,$\color{gr} \color{blue} 4\color{black}   $\,\,$ \\
$\,\,$0.2807$\,\,$ & $\,\,$\color{gr} 0.5865\color{black} $\,\,$ & $\,\,$ 1 $\,\,$ & $\,\,$2.3461 $\,\,$ \\
$\,\,$0.1196$\,\,$ & $\,\,$\color{gr} \color{blue}  1/4\color{black} $\,\,$ & $\,\,$0.4262$\,\,$ & $\,\,$ 1  $\,\,$ \\
\end{pmatrix},
\end{equation*}
\end{example}
\newpage
\begin{example}
\begin{equation*}
\mathbf{A} =
\begin{pmatrix}
$\,\,$ 1 $\,\,$ & $\,\,$2$\,\,$ & $\,\,$7$\,\,$ & $\,\,$7 $\,\,$ \\
$\,\,$ 1/2$\,\,$ & $\,\,$ 1 $\,\,$ & $\,\,$2$\,\,$ & $\,\,$5 $\,\,$ \\
$\,\,$ 1/7$\,\,$ & $\,\,$ 1/2$\,\,$ & $\,\,$ 1 $\,\,$ & $\,\,$4 $\,\,$ \\
$\,\,$ 1/7$\,\,$ & $\,\,$ 1/5$\,\,$ & $\,\,$ 1/4$\,\,$ & $\,\,$ 1  $\,\,$ \\
\end{pmatrix},
\qquad
\lambda_{\max} =
4.1665,
\qquad
CR = 0.0628
\end{equation*}

\begin{equation*}
\mathbf{w}^{cos} =
\begin{pmatrix}
0.540476\\
\color{red} 0.262881\color{black} \\
0.141342\\
0.055301
\end{pmatrix}\end{equation*}
\begin{equation*}
\left[ \frac{{w}^{cos}_i}{{w}^{cos}_j} \right] =
\begin{pmatrix}
$\,\,$ 1 $\,\,$ & $\,\,$\color{red} 2.0560\color{black} $\,\,$ & $\,\,$3.8239$\,\,$ & $\,\,$9.7733$\,\,$ \\
$\,\,$\color{red} 0.4864\color{black} $\,\,$ & $\,\,$ 1 $\,\,$ & $\,\,$\color{red} 1.8599\color{black} $\,\,$ & $\,\,$\color{red} 4.7536\color{black}   $\,\,$ \\
$\,\,$0.2615$\,\,$ & $\,\,$\color{red} 0.5377\color{black} $\,\,$ & $\,\,$ 1 $\,\,$ & $\,\,$2.5559 $\,\,$ \\
$\,\,$0.1023$\,\,$ & $\,\,$\color{red} 0.2104\color{black} $\,\,$ & $\,\,$0.3913$\,\,$ & $\,\,$ 1  $\,\,$ \\
\end{pmatrix},
\end{equation*}

\begin{equation*}
\mathbf{w}^{\prime} =
\begin{pmatrix}
0.536529\\
0.268264\\
0.140310\\
0.054897
\end{pmatrix} =
0.992697\cdot
\begin{pmatrix}
0.540476\\
\color{gr} 0.270238\color{black} \\
0.141342\\
0.055301
\end{pmatrix},
\end{equation*}
\begin{equation*}
\left[ \frac{{w}^{\prime}_i}{{w}^{\prime}_j} \right] =
\begin{pmatrix}
$\,\,$ 1 $\,\,$ & $\,\,$\color{gr} \color{blue} 2\color{black} $\,\,$ & $\,\,$3.8239$\,\,$ & $\,\,$9.7733$\,\,$ \\
$\,\,$\color{gr} \color{blue}  1/2\color{black} $\,\,$ & $\,\,$ 1 $\,\,$ & $\,\,$\color{gr} 1.9119\color{black} $\,\,$ & $\,\,$\color{gr} 4.8867\color{black}   $\,\,$ \\
$\,\,$0.2615$\,\,$ & $\,\,$\color{gr} 0.5230\color{black} $\,\,$ & $\,\,$ 1 $\,\,$ & $\,\,$2.5559 $\,\,$ \\
$\,\,$0.1023$\,\,$ & $\,\,$\color{gr} 0.2046\color{black} $\,\,$ & $\,\,$0.3913$\,\,$ & $\,\,$ 1  $\,\,$ \\
\end{pmatrix},
\end{equation*}
\end{example}
\newpage
\begin{example}
\begin{equation*}
\mathbf{A} =
\begin{pmatrix}
$\,\,$ 1 $\,\,$ & $\,\,$2$\,\,$ & $\,\,$7$\,\,$ & $\,\,$7 $\,\,$ \\
$\,\,$ 1/2$\,\,$ & $\,\,$ 1 $\,\,$ & $\,\,$2$\,\,$ & $\,\,$5 $\,\,$ \\
$\,\,$ 1/7$\,\,$ & $\,\,$ 1/2$\,\,$ & $\,\,$ 1 $\,\,$ & $\,\,$5 $\,\,$ \\
$\,\,$ 1/7$\,\,$ & $\,\,$ 1/5$\,\,$ & $\,\,$ 1/5$\,\,$ & $\,\,$ 1  $\,\,$ \\
\end{pmatrix},
\qquad
\lambda_{\max} =
4.2287,
\qquad
CR = 0.0862
\end{equation*}

\begin{equation*}
\mathbf{w}^{cos} =
\begin{pmatrix}
0.534231\\
\color{red} 0.258628\color{black} \\
0.153764\\
0.053377
\end{pmatrix}\end{equation*}
\begin{equation*}
\left[ \frac{{w}^{cos}_i}{{w}^{cos}_j} \right] =
\begin{pmatrix}
$\,\,$ 1 $\,\,$ & $\,\,$\color{red} 2.0656\color{black} $\,\,$ & $\,\,$3.4744$\,\,$ & $\,\,$10.0087$\,\,$ \\
$\,\,$\color{red} 0.4841\color{black} $\,\,$ & $\,\,$ 1 $\,\,$ & $\,\,$\color{red} 1.6820\color{black} $\,\,$ & $\,\,$\color{red} 4.8453\color{black}   $\,\,$ \\
$\,\,$0.2878$\,\,$ & $\,\,$\color{red} 0.5945\color{black} $\,\,$ & $\,\,$ 1 $\,\,$ & $\,\,$2.8807 $\,\,$ \\
$\,\,$0.0999$\,\,$ & $\,\,$\color{red} 0.2064\color{black} $\,\,$ & $\,\,$0.3471$\,\,$ & $\,\,$ 1  $\,\,$ \\
\end{pmatrix},
\end{equation*}

\begin{equation*}
\mathbf{w}^{\prime} =
\begin{pmatrix}
0.529857\\
0.264699\\
0.152505\\
0.052940
\end{pmatrix} =
0.991811\cdot
\begin{pmatrix}
0.534231\\
\color{gr} 0.266884\color{black} \\
0.153764\\
0.053377
\end{pmatrix},
\end{equation*}
\begin{equation*}
\left[ \frac{{w}^{\prime}_i}{{w}^{\prime}_j} \right] =
\begin{pmatrix}
$\,\,$ 1 $\,\,$ & $\,\,$\color{gr} 2.0017\color{black} $\,\,$ & $\,\,$3.4744$\,\,$ & $\,\,$10.0087$\,\,$ \\
$\,\,$\color{gr} 0.4996\color{black} $\,\,$ & $\,\,$ 1 $\,\,$ & $\,\,$\color{gr} 1.7357\color{black} $\,\,$ & $\,\,$\color{gr} \color{blue} 5\color{black}   $\,\,$ \\
$\,\,$0.2878$\,\,$ & $\,\,$\color{gr} 0.5761\color{black} $\,\,$ & $\,\,$ 1 $\,\,$ & $\,\,$2.8807 $\,\,$ \\
$\,\,$0.0999$\,\,$ & $\,\,$\color{gr} \color{blue}  1/5\color{black} $\,\,$ & $\,\,$0.3471$\,\,$ & $\,\,$ 1  $\,\,$ \\
\end{pmatrix},
\end{equation*}
\end{example}
\newpage
\begin{example}
\begin{equation*}
\mathbf{A} =
\begin{pmatrix}
$\,\,$ 1 $\,\,$ & $\,\,$2$\,\,$ & $\,\,$7$\,\,$ & $\,\,$8 $\,\,$ \\
$\,\,$ 1/2$\,\,$ & $\,\,$ 1 $\,\,$ & $\,\,$1$\,\,$ & $\,\,$3 $\,\,$ \\
$\,\,$ 1/7$\,\,$ & $\,\,$ 1 $\,\,$ & $\,\,$ 1 $\,\,$ & $\,\,$2 $\,\,$ \\
$\,\,$ 1/8$\,\,$ & $\,\,$ 1/3$\,\,$ & $\,\,$ 1/2$\,\,$ & $\,\,$ 1  $\,\,$ \\
\end{pmatrix},
\qquad
\lambda_{\max} =
4.1365,
\qquad
CR = 0.0515
\end{equation*}

\begin{equation*}
\mathbf{w}^{cos} =
\begin{pmatrix}
0.574301\\
0.212851\\
0.144060\\
\color{red} 0.068788\color{black}
\end{pmatrix}\end{equation*}
\begin{equation*}
\left[ \frac{{w}^{cos}_i}{{w}^{cos}_j} \right] =
\begin{pmatrix}
$\,\,$ 1 $\,\,$ & $\,\,$2.6981$\,\,$ & $\,\,$3.9865$\,\,$ & $\,\,$\color{red} 8.3489\color{black} $\,\,$ \\
$\,\,$0.3706$\,\,$ & $\,\,$ 1 $\,\,$ & $\,\,$1.4775$\,\,$ & $\,\,$\color{red} 3.0943\color{black}   $\,\,$ \\
$\,\,$0.2508$\,\,$ & $\,\,$0.6768$\,\,$ & $\,\,$ 1 $\,\,$ & $\,\,$\color{red} 2.0943\color{black}  $\,\,$ \\
$\,\,$\color{red} 0.1198\color{black} $\,\,$ & $\,\,$\color{red} 0.3232\color{black} $\,\,$ & $\,\,$\color{red} 0.4775\color{black} $\,\,$ & $\,\,$ 1  $\,\,$ \\
\end{pmatrix},
\end{equation*}

\begin{equation*}
\mathbf{w}^{\prime} =
\begin{pmatrix}
0.573062\\
0.212392\\
0.143749\\
0.070797
\end{pmatrix} =
0.997842\cdot
\begin{pmatrix}
0.574301\\
0.212851\\
0.144060\\
\color{gr} 0.070950\color{black}
\end{pmatrix},
\end{equation*}
\begin{equation*}
\left[ \frac{{w}^{\prime}_i}{{w}^{\prime}_j} \right] =
\begin{pmatrix}
$\,\,$ 1 $\,\,$ & $\,\,$2.6981$\,\,$ & $\,\,$3.9865$\,\,$ & $\,\,$\color{gr} 8.0944\color{black} $\,\,$ \\
$\,\,$0.3706$\,\,$ & $\,\,$ 1 $\,\,$ & $\,\,$1.4775$\,\,$ & $\,\,$\color{gr} \color{blue} 3\color{black}   $\,\,$ \\
$\,\,$0.2508$\,\,$ & $\,\,$0.6768$\,\,$ & $\,\,$ 1 $\,\,$ & $\,\,$\color{gr} 2.0304\color{black}  $\,\,$ \\
$\,\,$\color{gr} 0.1235\color{black} $\,\,$ & $\,\,$\color{gr} \color{blue}  1/3\color{black} $\,\,$ & $\,\,$\color{gr} 0.4925\color{black} $\,\,$ & $\,\,$ 1  $\,\,$ \\
\end{pmatrix},
\end{equation*}
\end{example}
\newpage
\begin{example}
\begin{equation*}
\mathbf{A} =
\begin{pmatrix}
$\,\,$ 1 $\,\,$ & $\,\,$2$\,\,$ & $\,\,$7$\,\,$ & $\,\,$8 $\,\,$ \\
$\,\,$ 1/2$\,\,$ & $\,\,$ 1 $\,\,$ & $\,\,$2$\,\,$ & $\,\,$5 $\,\,$ \\
$\,\,$ 1/7$\,\,$ & $\,\,$ 1/2$\,\,$ & $\,\,$ 1 $\,\,$ & $\,\,$4 $\,\,$ \\
$\,\,$ 1/8$\,\,$ & $\,\,$ 1/5$\,\,$ & $\,\,$ 1/4$\,\,$ & $\,\,$ 1  $\,\,$ \\
\end{pmatrix},
\qquad
\lambda_{\max} =
4.1380,
\qquad
CR = 0.0520
\end{equation*}

\begin{equation*}
\mathbf{w}^{cos} =
\begin{pmatrix}
0.551577\\
\color{red} 0.258851\color{black} \\
0.137542\\
0.052031
\end{pmatrix}\end{equation*}
\begin{equation*}
\left[ \frac{{w}^{cos}_i}{{w}^{cos}_j} \right] =
\begin{pmatrix}
$\,\,$ 1 $\,\,$ & $\,\,$\color{red} 2.1309\color{black} $\,\,$ & $\,\,$4.0103$\,\,$ & $\,\,$10.6010$\,\,$ \\
$\,\,$\color{red} 0.4693\color{black} $\,\,$ & $\,\,$ 1 $\,\,$ & $\,\,$\color{red} 1.8820\color{black} $\,\,$ & $\,\,$\color{red} 4.9750\color{black}   $\,\,$ \\
$\,\,$0.2494$\,\,$ & $\,\,$\color{red} 0.5314\color{black} $\,\,$ & $\,\,$ 1 $\,\,$ & $\,\,$2.6435 $\,\,$ \\
$\,\,$0.0943$\,\,$ & $\,\,$\color{red} 0.2010\color{black} $\,\,$ & $\,\,$0.3783$\,\,$ & $\,\,$ 1  $\,\,$ \\
\end{pmatrix},
\end{equation*}

\begin{equation*}
\mathbf{w}^{\prime} =
\begin{pmatrix}
0.550859\\
0.259815\\
0.137363\\
0.051963
\end{pmatrix} =
0.998699\cdot
\begin{pmatrix}
0.551577\\
\color{gr} 0.260154\color{black} \\
0.137542\\
0.052031
\end{pmatrix},
\end{equation*}
\begin{equation*}
\left[ \frac{{w}^{\prime}_i}{{w}^{\prime}_j} \right] =
\begin{pmatrix}
$\,\,$ 1 $\,\,$ & $\,\,$\color{gr} 2.1202\color{black} $\,\,$ & $\,\,$4.0103$\,\,$ & $\,\,$10.6010$\,\,$ \\
$\,\,$\color{gr} 0.4717\color{black} $\,\,$ & $\,\,$ 1 $\,\,$ & $\,\,$\color{gr} 1.8915\color{black} $\,\,$ & $\,\,$\color{gr} \color{blue} 5\color{black}   $\,\,$ \\
$\,\,$0.2494$\,\,$ & $\,\,$\color{gr} 0.5287\color{black} $\,\,$ & $\,\,$ 1 $\,\,$ & $\,\,$2.6435 $\,\,$ \\
$\,\,$0.0943$\,\,$ & $\,\,$\color{gr} \color{blue}  1/5\color{black} $\,\,$ & $\,\,$0.3783$\,\,$ & $\,\,$ 1  $\,\,$ \\
\end{pmatrix},
\end{equation*}
\end{example}
\newpage
\begin{example}
\begin{equation*}
\mathbf{A} =
\begin{pmatrix}
$\,\,$ 1 $\,\,$ & $\,\,$2$\,\,$ & $\,\,$7$\,\,$ & $\,\,$8 $\,\,$ \\
$\,\,$ 1/2$\,\,$ & $\,\,$ 1 $\,\,$ & $\,\,$2$\,\,$ & $\,\,$6 $\,\,$ \\
$\,\,$ 1/7$\,\,$ & $\,\,$ 1/2$\,\,$ & $\,\,$ 1 $\,\,$ & $\,\,$5 $\,\,$ \\
$\,\,$ 1/8$\,\,$ & $\,\,$ 1/6$\,\,$ & $\,\,$ 1/5$\,\,$ & $\,\,$ 1  $\,\,$ \\
\end{pmatrix},
\qquad
\lambda_{\max} =
4.1888,
\qquad
CR = 0.0712
\end{equation*}

\begin{equation*}
\mathbf{w}^{cos} =
\begin{pmatrix}
0.540351\\
\color{red} 0.266084\color{black} \\
0.146276\\
0.047289
\end{pmatrix}\end{equation*}
\begin{equation*}
\left[ \frac{{w}^{cos}_i}{{w}^{cos}_j} \right] =
\begin{pmatrix}
$\,\,$ 1 $\,\,$ & $\,\,$\color{red} 2.0308\color{black} $\,\,$ & $\,\,$3.6940$\,\,$ & $\,\,$11.4265$\,\,$ \\
$\,\,$\color{red} 0.4924\color{black} $\,\,$ & $\,\,$ 1 $\,\,$ & $\,\,$\color{red} 1.8191\color{black} $\,\,$ & $\,\,$\color{red} 5.6268\color{black}   $\,\,$ \\
$\,\,$0.2707$\,\,$ & $\,\,$\color{red} 0.5497\color{black} $\,\,$ & $\,\,$ 1 $\,\,$ & $\,\,$3.0932 $\,\,$ \\
$\,\,$0.0875$\,\,$ & $\,\,$\color{red} 0.1777\color{black} $\,\,$ & $\,\,$0.3233$\,\,$ & $\,\,$ 1  $\,\,$ \\
\end{pmatrix},
\end{equation*}

\begin{equation*}
\mathbf{w}^{\prime} =
\begin{pmatrix}
0.538149\\
0.269075\\
0.145680\\
0.047096
\end{pmatrix} =
0.995925\cdot
\begin{pmatrix}
0.540351\\
\color{gr} 0.270175\color{black} \\
0.146276\\
0.047289
\end{pmatrix},
\end{equation*}
\begin{equation*}
\left[ \frac{{w}^{\prime}_i}{{w}^{\prime}_j} \right] =
\begin{pmatrix}
$\,\,$ 1 $\,\,$ & $\,\,$\color{gr} \color{blue} 2\color{black} $\,\,$ & $\,\,$3.6940$\,\,$ & $\,\,$11.4265$\,\,$ \\
$\,\,$\color{gr} \color{blue}  1/2\color{black} $\,\,$ & $\,\,$ 1 $\,\,$ & $\,\,$\color{gr} 1.8470\color{black} $\,\,$ & $\,\,$\color{gr} 5.7133\color{black}   $\,\,$ \\
$\,\,$0.2707$\,\,$ & $\,\,$\color{gr} 0.5414\color{black} $\,\,$ & $\,\,$ 1 $\,\,$ & $\,\,$3.0932 $\,\,$ \\
$\,\,$0.0875$\,\,$ & $\,\,$\color{gr} 0.1750\color{black} $\,\,$ & $\,\,$0.3233$\,\,$ & $\,\,$ 1  $\,\,$ \\
\end{pmatrix},
\end{equation*}
\end{example}
\newpage
\begin{example}
\begin{equation*}
\mathbf{A} =
\begin{pmatrix}
$\,\,$ 1 $\,\,$ & $\,\,$2$\,\,$ & $\,\,$7$\,\,$ & $\,\,$8 $\,\,$ \\
$\,\,$ 1/2$\,\,$ & $\,\,$ 1 $\,\,$ & $\,\,$2$\,\,$ & $\,\,$6 $\,\,$ \\
$\,\,$ 1/7$\,\,$ & $\,\,$ 1/2$\,\,$ & $\,\,$ 1 $\,\,$ & $\,\,$6 $\,\,$ \\
$\,\,$ 1/8$\,\,$ & $\,\,$ 1/6$\,\,$ & $\,\,$ 1/6$\,\,$ & $\,\,$ 1  $\,\,$ \\
\end{pmatrix},
\qquad
\lambda_{\max} =
4.2421,
\qquad
CR = 0.0913
\end{equation*}

\begin{equation*}
\mathbf{w}^{cos} =
\begin{pmatrix}
0.535069\\
\color{red} 0.262292\color{black} \\
0.156702\\
0.045937
\end{pmatrix}\end{equation*}
\begin{equation*}
\left[ \frac{{w}^{cos}_i}{{w}^{cos}_j} \right] =
\begin{pmatrix}
$\,\,$ 1 $\,\,$ & $\,\,$\color{red} 2.0400\color{black} $\,\,$ & $\,\,$3.4146$\,\,$ & $\,\,$11.6479$\,\,$ \\
$\,\,$\color{red} 0.4902\color{black} $\,\,$ & $\,\,$ 1 $\,\,$ & $\,\,$\color{red} 1.6738\color{black} $\,\,$ & $\,\,$\color{red} 5.7098\color{black}   $\,\,$ \\
$\,\,$0.2929$\,\,$ & $\,\,$\color{red} 0.5974\color{black} $\,\,$ & $\,\,$ 1 $\,\,$ & $\,\,$3.4112 $\,\,$ \\
$\,\,$0.0859$\,\,$ & $\,\,$\color{red} 0.1751\color{black} $\,\,$ & $\,\,$0.2931$\,\,$ & $\,\,$ 1  $\,\,$ \\
\end{pmatrix},
\end{equation*}

\begin{equation*}
\mathbf{w}^{\prime} =
\begin{pmatrix}
0.532279\\
0.266139\\
0.155885\\
0.045697
\end{pmatrix} =
0.994784\cdot
\begin{pmatrix}
0.535069\\
\color{gr} 0.267535\color{black} \\
0.156702\\
0.045937
\end{pmatrix},
\end{equation*}
\begin{equation*}
\left[ \frac{{w}^{\prime}_i}{{w}^{\prime}_j} \right] =
\begin{pmatrix}
$\,\,$ 1 $\,\,$ & $\,\,$\color{gr} \color{blue} 2\color{black} $\,\,$ & $\,\,$3.4146$\,\,$ & $\,\,$11.6479$\,\,$ \\
$\,\,$\color{gr} \color{blue}  1/2\color{black} $\,\,$ & $\,\,$ 1 $\,\,$ & $\,\,$\color{gr} 1.7073\color{black} $\,\,$ & $\,\,$\color{gr} 5.8240\color{black}   $\,\,$ \\
$\,\,$0.2929$\,\,$ & $\,\,$\color{gr} 0.5857\color{black} $\,\,$ & $\,\,$ 1 $\,\,$ & $\,\,$3.4112 $\,\,$ \\
$\,\,$0.0859$\,\,$ & $\,\,$\color{gr} 0.1717\color{black} $\,\,$ & $\,\,$0.2931$\,\,$ & $\,\,$ 1  $\,\,$ \\
\end{pmatrix},
\end{equation*}
\end{example}
\newpage
\begin{example}
\begin{equation*}
\mathbf{A} =
\begin{pmatrix}
$\,\,$ 1 $\,\,$ & $\,\,$2$\,\,$ & $\,\,$7$\,\,$ & $\,\,$9 $\,\,$ \\
$\,\,$ 1/2$\,\,$ & $\,\,$ 1 $\,\,$ & $\,\,$2$\,\,$ & $\,\,$6 $\,\,$ \\
$\,\,$ 1/7$\,\,$ & $\,\,$ 1/2$\,\,$ & $\,\,$ 1 $\,\,$ & $\,\,$5 $\,\,$ \\
$\,\,$ 1/9$\,\,$ & $\,\,$ 1/6$\,\,$ & $\,\,$ 1/5$\,\,$ & $\,\,$ 1  $\,\,$ \\
\end{pmatrix},
\qquad
\lambda_{\max} =
4.1610,
\qquad
CR = 0.0607
\end{equation*}

\begin{equation*}
\mathbf{w}^{cos} =
\begin{pmatrix}
0.549964\\
\color{red} 0.262481\color{black} \\
0.142793\\
0.044763
\end{pmatrix}\end{equation*}
\begin{equation*}
\left[ \frac{{w}^{cos}_i}{{w}^{cos}_j} \right] =
\begin{pmatrix}
$\,\,$ 1 $\,\,$ & $\,\,$\color{red} 2.0953\color{black} $\,\,$ & $\,\,$3.8515$\,\,$ & $\,\,$12.2862$\,\,$ \\
$\,\,$\color{red} 0.4773\color{black} $\,\,$ & $\,\,$ 1 $\,\,$ & $\,\,$\color{red} 1.8382\color{black} $\,\,$ & $\,\,$\color{red} 5.8638\color{black}   $\,\,$ \\
$\,\,$0.2596$\,\,$ & $\,\,$\color{red} 0.5440\color{black} $\,\,$ & $\,\,$ 1 $\,\,$ & $\,\,$3.1900 $\,\,$ \\
$\,\,$0.0814$\,\,$ & $\,\,$\color{red} 0.1705\color{black} $\,\,$ & $\,\,$0.3135$\,\,$ & $\,\,$ 1  $\,\,$ \\
\end{pmatrix},
\end{equation*}

\begin{equation*}
\mathbf{w}^{\prime} =
\begin{pmatrix}
0.546632\\
0.266949\\
0.141928\\
0.044491
\end{pmatrix} =
0.993942\cdot
\begin{pmatrix}
0.549964\\
\color{gr} 0.268576\color{black} \\
0.142793\\
0.044763
\end{pmatrix},
\end{equation*}
\begin{equation*}
\left[ \frac{{w}^{\prime}_i}{{w}^{\prime}_j} \right] =
\begin{pmatrix}
$\,\,$ 1 $\,\,$ & $\,\,$\color{gr} 2.0477\color{black} $\,\,$ & $\,\,$3.8515$\,\,$ & $\,\,$12.2862$\,\,$ \\
$\,\,$\color{gr} 0.4884\color{black} $\,\,$ & $\,\,$ 1 $\,\,$ & $\,\,$\color{gr} 1.8809\color{black} $\,\,$ & $\,\,$\color{gr} \color{blue} 6\color{black}   $\,\,$ \\
$\,\,$0.2596$\,\,$ & $\,\,$\color{gr} 0.5317\color{black} $\,\,$ & $\,\,$ 1 $\,\,$ & $\,\,$3.1900 $\,\,$ \\
$\,\,$0.0814$\,\,$ & $\,\,$\color{gr} \color{blue}  1/6\color{black} $\,\,$ & $\,\,$0.3135$\,\,$ & $\,\,$ 1  $\,\,$ \\
\end{pmatrix},
\end{equation*}
\end{example}
\newpage
\begin{example}
\begin{equation*}
\mathbf{A} =
\begin{pmatrix}
$\,\,$ 1 $\,\,$ & $\,\,$2$\,\,$ & $\,\,$7$\,\,$ & $\,\,$9 $\,\,$ \\
$\,\,$ 1/2$\,\,$ & $\,\,$ 1 $\,\,$ & $\,\,$2$\,\,$ & $\,\,$6 $\,\,$ \\
$\,\,$ 1/7$\,\,$ & $\,\,$ 1/2$\,\,$ & $\,\,$ 1 $\,\,$ & $\,\,$6 $\,\,$ \\
$\,\,$ 1/9$\,\,$ & $\,\,$ 1/6$\,\,$ & $\,\,$ 1/6$\,\,$ & $\,\,$ 1  $\,\,$ \\
\end{pmatrix},
\qquad
\lambda_{\max} =
4.2109,
\qquad
CR = 0.0795
\end{equation*}

\begin{equation*}
\mathbf{w}^{cos} =
\begin{pmatrix}
0.544560\\
\color{red} 0.259075\color{black} \\
0.152892\\
0.043473
\end{pmatrix}\end{equation*}
\begin{equation*}
\left[ \frac{{w}^{cos}_i}{{w}^{cos}_j} \right] =
\begin{pmatrix}
$\,\,$ 1 $\,\,$ & $\,\,$\color{red} 2.1019\color{black} $\,\,$ & $\,\,$3.5617$\,\,$ & $\,\,$12.5264$\,\,$ \\
$\,\,$\color{red} 0.4758\color{black} $\,\,$ & $\,\,$ 1 $\,\,$ & $\,\,$\color{red} 1.6945\color{black} $\,\,$ & $\,\,$\color{red} 5.9594\color{black}   $\,\,$ \\
$\,\,$0.2808$\,\,$ & $\,\,$\color{red} 0.5901\color{black} $\,\,$ & $\,\,$ 1 $\,\,$ & $\,\,$3.5169 $\,\,$ \\
$\,\,$0.0798$\,\,$ & $\,\,$\color{red} 0.1678\color{black} $\,\,$ & $\,\,$0.2843$\,\,$ & $\,\,$ 1  $\,\,$ \\
\end{pmatrix},
\end{equation*}

\begin{equation*}
\mathbf{w}^{\prime} =
\begin{pmatrix}
0.543602\\
0.260379\\
0.152623\\
0.043397
\end{pmatrix} =
0.998240\cdot
\begin{pmatrix}
0.544560\\
\color{gr} 0.260838\color{black} \\
0.152892\\
0.043473
\end{pmatrix},
\end{equation*}
\begin{equation*}
\left[ \frac{{w}^{\prime}_i}{{w}^{\prime}_j} \right] =
\begin{pmatrix}
$\,\,$ 1 $\,\,$ & $\,\,$\color{gr} 2.0877\color{black} $\,\,$ & $\,\,$3.5617$\,\,$ & $\,\,$12.5264$\,\,$ \\
$\,\,$\color{gr} 0.4790\color{black} $\,\,$ & $\,\,$ 1 $\,\,$ & $\,\,$\color{gr} 1.7060\color{black} $\,\,$ & $\,\,$\color{gr} \color{blue} 6\color{black}   $\,\,$ \\
$\,\,$0.2808$\,\,$ & $\,\,$\color{gr} 0.5862\color{black} $\,\,$ & $\,\,$ 1 $\,\,$ & $\,\,$3.5169 $\,\,$ \\
$\,\,$0.0798$\,\,$ & $\,\,$\color{gr} \color{blue}  1/6\color{black} $\,\,$ & $\,\,$0.2843$\,\,$ & $\,\,$ 1  $\,\,$ \\
\end{pmatrix},
\end{equation*}
\end{example}
\newpage
\begin{example}
\begin{equation*}
\mathbf{A} =
\begin{pmatrix}
$\,\,$ 1 $\,\,$ & $\,\,$2$\,\,$ & $\,\,$7$\,\,$ & $\,\,$9 $\,\,$ \\
$\,\,$ 1/2$\,\,$ & $\,\,$ 1 $\,\,$ & $\,\,$8$\,\,$ & $\,\,$6 $\,\,$ \\
$\,\,$ 1/7$\,\,$ & $\,\,$ 1/8$\,\,$ & $\,\,$ 1 $\,\,$ & $\,\,$1 $\,\,$ \\
$\,\,$ 1/9$\,\,$ & $\,\,$ 1/6$\,\,$ & $\,\,$ 1 $\,\,$ & $\,\,$ 1  $\,\,$ \\
\end{pmatrix},
\qquad
\lambda_{\max} =
4.0576,
\qquad
CR = 0.0217
\end{equation*}

\begin{equation*}
\mathbf{w}^{cos} =
\begin{pmatrix}
0.527976\\
0.354461\\
0.059551\\
\color{red} 0.058012\color{black}
\end{pmatrix}\end{equation*}
\begin{equation*}
\left[ \frac{{w}^{cos}_i}{{w}^{cos}_j} \right] =
\begin{pmatrix}
$\,\,$ 1 $\,\,$ & $\,\,$1.4895$\,\,$ & $\,\,$8.8660$\,\,$ & $\,\,$\color{red} 9.1012\color{black} $\,\,$ \\
$\,\,$0.6714$\,\,$ & $\,\,$ 1 $\,\,$ & $\,\,$5.9522$\,\,$ & $\,\,$\color{red} 6.1101\color{black}   $\,\,$ \\
$\,\,$0.1128$\,\,$ & $\,\,$0.1680$\,\,$ & $\,\,$ 1 $\,\,$ & $\,\,$\color{red} 1.0265\color{black}  $\,\,$ \\
$\,\,$\color{red} 0.1099\color{black} $\,\,$ & $\,\,$\color{red} 0.1637\color{black} $\,\,$ & $\,\,$\color{red} 0.9742\color{black} $\,\,$ & $\,\,$ 1  $\,\,$ \\
\end{pmatrix},
\end{equation*}

\begin{equation*}
\mathbf{w}^{\prime} =
\begin{pmatrix}
0.527632\\
0.354230\\
0.059512\\
0.058626
\end{pmatrix} =
0.999348\cdot
\begin{pmatrix}
0.527976\\
0.354461\\
0.059551\\
\color{gr} 0.058664\color{black}
\end{pmatrix},
\end{equation*}
\begin{equation*}
\left[ \frac{{w}^{\prime}_i}{{w}^{\prime}_j} \right] =
\begin{pmatrix}
$\,\,$ 1 $\,\,$ & $\,\,$1.4895$\,\,$ & $\,\,$8.8660$\,\,$ & $\,\,$\color{gr} \color{blue} 9\color{black} $\,\,$ \\
$\,\,$0.6714$\,\,$ & $\,\,$ 1 $\,\,$ & $\,\,$5.9522$\,\,$ & $\,\,$\color{gr} 6.0422\color{black}   $\,\,$ \\
$\,\,$0.1128$\,\,$ & $\,\,$0.1680$\,\,$ & $\,\,$ 1 $\,\,$ & $\,\,$\color{gr} 1.0151\color{black}  $\,\,$ \\
$\,\,$\color{gr} \color{blue}  1/9\color{black} $\,\,$ & $\,\,$\color{gr} 0.1655\color{black} $\,\,$ & $\,\,$\color{gr} 0.9851\color{black} $\,\,$ & $\,\,$ 1  $\,\,$ \\
\end{pmatrix},
\end{equation*}
\end{example}
\newpage
\begin{example}
\begin{equation*}
\mathbf{A} =
\begin{pmatrix}
$\,\,$ 1 $\,\,$ & $\,\,$2$\,\,$ & $\,\,$7$\,\,$ & $\,\,$9 $\,\,$ \\
$\,\,$ 1/2$\,\,$ & $\,\,$ 1 $\,\,$ & $\,\,$9$\,\,$ & $\,\,$6 $\,\,$ \\
$\,\,$ 1/7$\,\,$ & $\,\,$ 1/9$\,\,$ & $\,\,$ 1 $\,\,$ & $\,\,$1 $\,\,$ \\
$\,\,$ 1/9$\,\,$ & $\,\,$ 1/6$\,\,$ & $\,\,$ 1 $\,\,$ & $\,\,$ 1  $\,\,$ \\
\end{pmatrix},
\qquad
\lambda_{\max} =
4.0762,
\qquad
CR = 0.0287
\end{equation*}

\begin{equation*}
\mathbf{w}^{cos} =
\begin{pmatrix}
0.523024\\
0.361976\\
0.057768\\
\color{red} 0.057233\color{black}
\end{pmatrix}\end{equation*}
\begin{equation*}
\left[ \frac{{w}^{cos}_i}{{w}^{cos}_j} \right] =
\begin{pmatrix}
$\,\,$ 1 $\,\,$ & $\,\,$1.4449$\,\,$ & $\,\,$9.0539$\,\,$ & $\,\,$\color{red} 9.1385\color{black} $\,\,$ \\
$\,\,$0.6921$\,\,$ & $\,\,$ 1 $\,\,$ & $\,\,$6.2660$\,\,$ & $\,\,$\color{red} 6.3246\color{black}   $\,\,$ \\
$\,\,$0.1105$\,\,$ & $\,\,$0.1596$\,\,$ & $\,\,$ 1 $\,\,$ & $\,\,$\color{red} 1.0094\color{black}  $\,\,$ \\
$\,\,$\color{red} 0.1094\color{black} $\,\,$ & $\,\,$\color{red} 0.1581\color{black} $\,\,$ & $\,\,$\color{red} 0.9907\color{black} $\,\,$ & $\,\,$ 1  $\,\,$ \\
\end{pmatrix},
\end{equation*}

\begin{equation*}
\mathbf{w}^{\prime} =
\begin{pmatrix}
0.522744\\
0.361782\\
0.057737\\
0.057737
\end{pmatrix} =
0.999465\cdot
\begin{pmatrix}
0.523024\\
0.361976\\
0.057768\\
\color{gr} 0.057768\color{black}
\end{pmatrix},
\end{equation*}
\begin{equation*}
\left[ \frac{{w}^{\prime}_i}{{w}^{\prime}_j} \right] =
\begin{pmatrix}
$\,\,$ 1 $\,\,$ & $\,\,$1.4449$\,\,$ & $\,\,$9.0539$\,\,$ & $\,\,$\color{gr} 9.0539\color{black} $\,\,$ \\
$\,\,$0.6921$\,\,$ & $\,\,$ 1 $\,\,$ & $\,\,$6.2660$\,\,$ & $\,\,$\color{gr} 6.2660\color{black}   $\,\,$ \\
$\,\,$0.1105$\,\,$ & $\,\,$0.1596$\,\,$ & $\,\,$ 1 $\,\,$ & $\,\,$\color{gr} \color{blue} 1\color{black}  $\,\,$ \\
$\,\,$\color{gr} 0.1105\color{black} $\,\,$ & $\,\,$\color{gr} 0.1596\color{black} $\,\,$ & $\,\,$\color{gr} \color{blue} 1\color{black} $\,\,$ & $\,\,$ 1  $\,\,$ \\
\end{pmatrix},
\end{equation*}
\end{example}
\newpage
\begin{example}
\begin{equation*}
\mathbf{A} =
\begin{pmatrix}
$\,\,$ 1 $\,\,$ & $\,\,$2$\,\,$ & $\,\,$8$\,\,$ & $\,\,$3 $\,\,$ \\
$\,\,$ 1/2$\,\,$ & $\,\,$ 1 $\,\,$ & $\,\,$2$\,\,$ & $\,\,$2 $\,\,$ \\
$\,\,$ 1/8$\,\,$ & $\,\,$ 1/2$\,\,$ & $\,\,$ 1 $\,\,$ & $\,\,$2 $\,\,$ \\
$\,\,$ 1/3$\,\,$ & $\,\,$ 1/2$\,\,$ & $\,\,$ 1/2$\,\,$ & $\,\,$ 1  $\,\,$ \\
\end{pmatrix},
\qquad
\lambda_{\max} =
4.2512,
\qquad
CR = 0.0947
\end{equation*}

\begin{equation*}
\mathbf{w}^{cos} =
\begin{pmatrix}
0.507894\\
\color{red} 0.235506\color{black} \\
0.137200\\
0.119400
\end{pmatrix}\end{equation*}
\begin{equation*}
\left[ \frac{{w}^{cos}_i}{{w}^{cos}_j} \right] =
\begin{pmatrix}
$\,\,$ 1 $\,\,$ & $\,\,$\color{red} 2.1566\color{black} $\,\,$ & $\,\,$3.7019$\,\,$ & $\,\,$4.2537$\,\,$ \\
$\,\,$\color{red} 0.4637\color{black} $\,\,$ & $\,\,$ 1 $\,\,$ & $\,\,$\color{red} 1.7165\color{black} $\,\,$ & $\,\,$\color{red} 1.9724\color{black}   $\,\,$ \\
$\,\,$0.2701$\,\,$ & $\,\,$\color{red} 0.5826\color{black} $\,\,$ & $\,\,$ 1 $\,\,$ & $\,\,$1.1491 $\,\,$ \\
$\,\,$0.2351$\,\,$ & $\,\,$\color{red} 0.5070\color{black} $\,\,$ & $\,\,$0.8703$\,\,$ & $\,\,$ 1  $\,\,$ \\
\end{pmatrix},
\end{equation*}

\begin{equation*}
\mathbf{w}^{\prime} =
\begin{pmatrix}
0.506227\\
0.238016\\
0.136750\\
0.119008
\end{pmatrix} =
0.996717\cdot
\begin{pmatrix}
0.507894\\
\color{gr} 0.238800\color{black} \\
0.137200\\
0.119400
\end{pmatrix},
\end{equation*}
\begin{equation*}
\left[ \frac{{w}^{\prime}_i}{{w}^{\prime}_j} \right] =
\begin{pmatrix}
$\,\,$ 1 $\,\,$ & $\,\,$\color{gr} 2.1269\color{black} $\,\,$ & $\,\,$3.7019$\,\,$ & $\,\,$4.2537$\,\,$ \\
$\,\,$\color{gr} 0.4702\color{black} $\,\,$ & $\,\,$ 1 $\,\,$ & $\,\,$\color{gr} 1.7405\color{black} $\,\,$ & $\,\,$\color{gr} \color{blue} 2\color{black}   $\,\,$ \\
$\,\,$0.2701$\,\,$ & $\,\,$\color{gr} 0.5745\color{black} $\,\,$ & $\,\,$ 1 $\,\,$ & $\,\,$1.1491 $\,\,$ \\
$\,\,$0.2351$\,\,$ & $\,\,$\color{gr} \color{blue}  1/2\color{black} $\,\,$ & $\,\,$0.8703$\,\,$ & $\,\,$ 1  $\,\,$ \\
\end{pmatrix},
\end{equation*}
\end{example}
\newpage
\begin{example}
\begin{equation*}
\mathbf{A} =
\begin{pmatrix}
$\,\,$ 1 $\,\,$ & $\,\,$2$\,\,$ & $\,\,$8$\,\,$ & $\,\,$5 $\,\,$ \\
$\,\,$ 1/2$\,\,$ & $\,\,$ 1 $\,\,$ & $\,\,$2$\,\,$ & $\,\,$4 $\,\,$ \\
$\,\,$ 1/8$\,\,$ & $\,\,$ 1/2$\,\,$ & $\,\,$ 1 $\,\,$ & $\,\,$3 $\,\,$ \\
$\,\,$ 1/5$\,\,$ & $\,\,$ 1/4$\,\,$ & $\,\,$ 1/3$\,\,$ & $\,\,$ 1  $\,\,$ \\
\end{pmatrix},
\qquad
\lambda_{\max} =
4.2162,
\qquad
CR = 0.0815
\end{equation*}

\begin{equation*}
\mathbf{w}^{cos} =
\begin{pmatrix}
0.531437\\
\color{red} 0.260953\color{black} \\
0.135257\\
0.072353
\end{pmatrix}\end{equation*}
\begin{equation*}
\left[ \frac{{w}^{cos}_i}{{w}^{cos}_j} \right] =
\begin{pmatrix}
$\,\,$ 1 $\,\,$ & $\,\,$\color{red} 2.0365\color{black} $\,\,$ & $\,\,$3.9291$\,\,$ & $\,\,$7.3451$\,\,$ \\
$\,\,$\color{red} 0.4910\color{black} $\,\,$ & $\,\,$ 1 $\,\,$ & $\,\,$\color{red} 1.9293\color{black} $\,\,$ & $\,\,$\color{red} 3.6067\color{black}   $\,\,$ \\
$\,\,$0.2545$\,\,$ & $\,\,$\color{red} 0.5183\color{black} $\,\,$ & $\,\,$ 1 $\,\,$ & $\,\,$1.8694 $\,\,$ \\
$\,\,$0.1361$\,\,$ & $\,\,$\color{red} 0.2773\color{black} $\,\,$ & $\,\,$0.5349$\,\,$ & $\,\,$ 1  $\,\,$ \\
\end{pmatrix},
\end{equation*}

\begin{equation*}
\mathbf{w}^{\prime} =
\begin{pmatrix}
0.528917\\
0.264458\\
0.134615\\
0.072010
\end{pmatrix} =
0.995257\cdot
\begin{pmatrix}
0.531437\\
\color{gr} 0.265719\color{black} \\
0.135257\\
0.072353
\end{pmatrix},
\end{equation*}
\begin{equation*}
\left[ \frac{{w}^{\prime}_i}{{w}^{\prime}_j} \right] =
\begin{pmatrix}
$\,\,$ 1 $\,\,$ & $\,\,$\color{gr} \color{blue} 2\color{black} $\,\,$ & $\,\,$3.9291$\,\,$ & $\,\,$7.3451$\,\,$ \\
$\,\,$\color{gr} \color{blue}  1/2\color{black} $\,\,$ & $\,\,$ 1 $\,\,$ & $\,\,$\color{gr} 1.9645\color{black} $\,\,$ & $\,\,$\color{gr} 3.6725\color{black}   $\,\,$ \\
$\,\,$0.2545$\,\,$ & $\,\,$\color{gr} 0.5090\color{black} $\,\,$ & $\,\,$ 1 $\,\,$ & $\,\,$1.8694 $\,\,$ \\
$\,\,$0.1361$\,\,$ & $\,\,$\color{gr} 0.2723\color{black} $\,\,$ & $\,\,$0.5349$\,\,$ & $\,\,$ 1  $\,\,$ \\
\end{pmatrix},
\end{equation*}
\end{example}
\newpage
\begin{example}
\begin{equation*}
\mathbf{A} =
\begin{pmatrix}
$\,\,$ 1 $\,\,$ & $\,\,$2$\,\,$ & $\,\,$8$\,\,$ & $\,\,$5 $\,\,$ \\
$\,\,$ 1/2$\,\,$ & $\,\,$ 1 $\,\,$ & $\,\,$6$\,\,$ & $\,\,$9 $\,\,$ \\
$\,\,$ 1/8$\,\,$ & $\,\,$ 1/6$\,\,$ & $\,\,$ 1 $\,\,$ & $\,\,$1 $\,\,$ \\
$\,\,$ 1/5$\,\,$ & $\,\,$ 1/9$\,\,$ & $\,\,$ 1 $\,\,$ & $\,\,$ 1  $\,\,$ \\
\end{pmatrix},
\qquad
\lambda_{\max} =
4.1406,
\qquad
CR = 0.0530
\end{equation*}

\begin{equation*}
\mathbf{w}^{cos} =
\begin{pmatrix}
0.491709\\
0.379083\\
\color{red} 0.061296\color{black} \\
0.067912
\end{pmatrix}\end{equation*}
\begin{equation*}
\left[ \frac{{w}^{cos}_i}{{w}^{cos}_j} \right] =
\begin{pmatrix}
$\,\,$ 1 $\,\,$ & $\,\,$1.2971$\,\,$ & $\,\,$\color{red} 8.0218\color{black} $\,\,$ & $\,\,$7.2404$\,\,$ \\
$\,\,$0.7710$\,\,$ & $\,\,$ 1 $\,\,$ & $\,\,$\color{red} 6.1845\color{black} $\,\,$ & $\,\,$5.5820  $\,\,$ \\
$\,\,$\color{red} 0.1247\color{black} $\,\,$ & $\,\,$\color{red} 0.1617\color{black} $\,\,$ & $\,\,$ 1 $\,\,$ & $\,\,$\color{red} 0.9026\color{black}  $\,\,$ \\
$\,\,$0.1381$\,\,$ & $\,\,$0.1791$\,\,$ & $\,\,$\color{red} 1.1079\color{black} $\,\,$ & $\,\,$ 1  $\,\,$ \\
\end{pmatrix},
\end{equation*}

\begin{equation*}
\mathbf{w}^{\prime} =
\begin{pmatrix}
0.491626\\
0.379020\\
0.061453\\
0.067900
\end{pmatrix} =
0.999833\cdot
\begin{pmatrix}
0.491709\\
0.379083\\
\color{gr} 0.061464\color{black} \\
0.067912
\end{pmatrix},
\end{equation*}
\begin{equation*}
\left[ \frac{{w}^{\prime}_i}{{w}^{\prime}_j} \right] =
\begin{pmatrix}
$\,\,$ 1 $\,\,$ & $\,\,$1.2971$\,\,$ & $\,\,$\color{gr} \color{blue} 8\color{black} $\,\,$ & $\,\,$7.2404$\,\,$ \\
$\,\,$0.7710$\,\,$ & $\,\,$ 1 $\,\,$ & $\,\,$\color{gr} 6.1676\color{black} $\,\,$ & $\,\,$5.5820  $\,\,$ \\
$\,\,$\color{gr} \color{blue}  1/8\color{black} $\,\,$ & $\,\,$\color{gr} 0.1621\color{black} $\,\,$ & $\,\,$ 1 $\,\,$ & $\,\,$\color{gr} 0.9051\color{black}  $\,\,$ \\
$\,\,$0.1381$\,\,$ & $\,\,$0.1791$\,\,$ & $\,\,$\color{gr} 1.1049\color{black} $\,\,$ & $\,\,$ 1  $\,\,$ \\
\end{pmatrix},
\end{equation*}
\end{example}
\newpage
\begin{example}
\begin{equation*}
\mathbf{A} =
\begin{pmatrix}
$\,\,$ 1 $\,\,$ & $\,\,$2$\,\,$ & $\,\,$8$\,\,$ & $\,\,$6 $\,\,$ \\
$\,\,$ 1/2$\,\,$ & $\,\,$ 1 $\,\,$ & $\,\,$2$\,\,$ & $\,\,$4 $\,\,$ \\
$\,\,$ 1/8$\,\,$ & $\,\,$ 1/2$\,\,$ & $\,\,$ 1 $\,\,$ & $\,\,$3 $\,\,$ \\
$\,\,$ 1/6$\,\,$ & $\,\,$ 1/4$\,\,$ & $\,\,$ 1/3$\,\,$ & $\,\,$ 1  $\,\,$ \\
\end{pmatrix},
\qquad
\lambda_{\max} =
4.1707,
\qquad
CR = 0.0644
\end{equation*}

\begin{equation*}
\mathbf{w}^{cos} =
\begin{pmatrix}
0.547213\\
\color{red} 0.255702\color{black} \\
0.130545\\
0.066541
\end{pmatrix}\end{equation*}
\begin{equation*}
\left[ \frac{{w}^{cos}_i}{{w}^{cos}_j} \right] =
\begin{pmatrix}
$\,\,$ 1 $\,\,$ & $\,\,$\color{red} 2.1400\color{black} $\,\,$ & $\,\,$4.1918$\,\,$ & $\,\,$8.2237$\,\,$ \\
$\,\,$\color{red} 0.4673\color{black} $\,\,$ & $\,\,$ 1 $\,\,$ & $\,\,$\color{red} 1.9587\color{black} $\,\,$ & $\,\,$\color{red} 3.8428\color{black}   $\,\,$ \\
$\,\,$0.2386$\,\,$ & $\,\,$\color{red} 0.5105\color{black} $\,\,$ & $\,\,$ 1 $\,\,$ & $\,\,$1.9619 $\,\,$ \\
$\,\,$0.1216$\,\,$ & $\,\,$\color{red} 0.2602\color{black} $\,\,$ & $\,\,$0.5097$\,\,$ & $\,\,$ 1  $\,\,$ \\
\end{pmatrix},
\end{equation*}

\begin{equation*}
\mathbf{w}^{\prime} =
\begin{pmatrix}
0.544280\\
0.259691\\
0.129845\\
0.066184
\end{pmatrix} =
0.994641\cdot
\begin{pmatrix}
0.547213\\
\color{gr} 0.261090\color{black} \\
0.130545\\
0.066541
\end{pmatrix},
\end{equation*}
\begin{equation*}
\left[ \frac{{w}^{\prime}_i}{{w}^{\prime}_j} \right] =
\begin{pmatrix}
$\,\,$ 1 $\,\,$ & $\,\,$\color{gr} 2.0959\color{black} $\,\,$ & $\,\,$4.1918$\,\,$ & $\,\,$8.2237$\,\,$ \\
$\,\,$\color{gr} 0.4771\color{black} $\,\,$ & $\,\,$ 1 $\,\,$ & $\,\,$\color{gr} \color{blue} 2\color{black} $\,\,$ & $\,\,$\color{gr} 3.9238\color{black}   $\,\,$ \\
$\,\,$0.2386$\,\,$ & $\,\,$\color{gr} \color{blue}  1/2\color{black} $\,\,$ & $\,\,$ 1 $\,\,$ & $\,\,$1.9619 $\,\,$ \\
$\,\,$0.1216$\,\,$ & $\,\,$\color{gr} 0.2549\color{black} $\,\,$ & $\,\,$0.5097$\,\,$ & $\,\,$ 1  $\,\,$ \\
\end{pmatrix},
\end{equation*}
\end{example}
\newpage
\begin{example}
\begin{equation*}
\mathbf{A} =
\begin{pmatrix}
$\,\,$ 1 $\,\,$ & $\,\,$2$\,\,$ & $\,\,$8$\,\,$ & $\,\,$6 $\,\,$ \\
$\,\,$ 1/2$\,\,$ & $\,\,$ 1 $\,\,$ & $\,\,$2$\,\,$ & $\,\,$4 $\,\,$ \\
$\,\,$ 1/8$\,\,$ & $\,\,$ 1/2$\,\,$ & $\,\,$ 1 $\,\,$ & $\,\,$4 $\,\,$ \\
$\,\,$ 1/6$\,\,$ & $\,\,$ 1/4$\,\,$ & $\,\,$ 1/4$\,\,$ & $\,\,$ 1  $\,\,$ \\
\end{pmatrix},
\qquad
\lambda_{\max} =
4.2512,
\qquad
CR = 0.0947
\end{equation*}

\begin{equation*}
\mathbf{w}^{cos} =
\begin{pmatrix}
0.539459\\
\color{red} 0.250824\color{black} \\
0.145981\\
0.063736
\end{pmatrix}\end{equation*}
\begin{equation*}
\left[ \frac{{w}^{cos}_i}{{w}^{cos}_j} \right] =
\begin{pmatrix}
$\,\,$ 1 $\,\,$ & $\,\,$\color{red} 2.1508\color{black} $\,\,$ & $\,\,$3.6954$\,\,$ & $\,\,$8.4639$\,\,$ \\
$\,\,$\color{red} 0.4650\color{black} $\,\,$ & $\,\,$ 1 $\,\,$ & $\,\,$\color{red} 1.7182\color{black} $\,\,$ & $\,\,$\color{red} 3.9353\color{black}   $\,\,$ \\
$\,\,$0.2706$\,\,$ & $\,\,$\color{red} 0.5820\color{black} $\,\,$ & $\,\,$ 1 $\,\,$ & $\,\,$2.2904 $\,\,$ \\
$\,\,$0.1181$\,\,$ & $\,\,$\color{red} 0.2541\color{black} $\,\,$ & $\,\,$0.4366$\,\,$ & $\,\,$ 1  $\,\,$ \\
\end{pmatrix},
\end{equation*}

\begin{equation*}
\mathbf{w}^{\prime} =
\begin{pmatrix}
0.537245\\
0.253898\\
0.145382\\
0.063475
\end{pmatrix} =
0.995896\cdot
\begin{pmatrix}
0.539459\\
\color{gr} 0.254945\color{black} \\
0.145981\\
0.063736
\end{pmatrix},
\end{equation*}
\begin{equation*}
\left[ \frac{{w}^{\prime}_i}{{w}^{\prime}_j} \right] =
\begin{pmatrix}
$\,\,$ 1 $\,\,$ & $\,\,$\color{gr} 2.1160\color{black} $\,\,$ & $\,\,$3.6954$\,\,$ & $\,\,$8.4639$\,\,$ \\
$\,\,$\color{gr} 0.4726\color{black} $\,\,$ & $\,\,$ 1 $\,\,$ & $\,\,$\color{gr} 1.7464\color{black} $\,\,$ & $\,\,$\color{gr} \color{blue} 4\color{black}   $\,\,$ \\
$\,\,$0.2706$\,\,$ & $\,\,$\color{gr} 0.5726\color{black} $\,\,$ & $\,\,$ 1 $\,\,$ & $\,\,$2.2904 $\,\,$ \\
$\,\,$0.1181$\,\,$ & $\,\,$\color{gr} \color{blue}  1/4\color{black} $\,\,$ & $\,\,$0.4366$\,\,$ & $\,\,$ 1  $\,\,$ \\
\end{pmatrix},
\end{equation*}
\end{example}
\newpage
\begin{example}
\begin{equation*}
\mathbf{A} =
\begin{pmatrix}
$\,\,$ 1 $\,\,$ & $\,\,$2$\,\,$ & $\,\,$8$\,\,$ & $\,\,$6 $\,\,$ \\
$\,\,$ 1/2$\,\,$ & $\,\,$ 1 $\,\,$ & $\,\,$2$\,\,$ & $\,\,$5 $\,\,$ \\
$\,\,$ 1/8$\,\,$ & $\,\,$ 1/2$\,\,$ & $\,\,$ 1 $\,\,$ & $\,\,$4 $\,\,$ \\
$\,\,$ 1/6$\,\,$ & $\,\,$ 1/5$\,\,$ & $\,\,$ 1/4$\,\,$ & $\,\,$ 1  $\,\,$ \\
\end{pmatrix},
\qquad
\lambda_{\max} =
4.2460,
\qquad
CR = 0.0928
\end{equation*}

\begin{equation*}
\mathbf{w}^{cos} =
\begin{pmatrix}
0.534041\\
\color{red} 0.264783\color{black} \\
0.141830\\
0.059346
\end{pmatrix}\end{equation*}
\begin{equation*}
\left[ \frac{{w}^{cos}_i}{{w}^{cos}_j} \right] =
\begin{pmatrix}
$\,\,$ 1 $\,\,$ & $\,\,$\color{red} 2.0169\color{black} $\,\,$ & $\,\,$3.7654$\,\,$ & $\,\,$8.9988$\,\,$ \\
$\,\,$\color{red} 0.4958\color{black} $\,\,$ & $\,\,$ 1 $\,\,$ & $\,\,$\color{red} 1.8669\color{black} $\,\,$ & $\,\,$\color{red} 4.4617\color{black}   $\,\,$ \\
$\,\,$0.2656$\,\,$ & $\,\,$\color{red} 0.5356\color{black} $\,\,$ & $\,\,$ 1 $\,\,$ & $\,\,$2.3899 $\,\,$ \\
$\,\,$0.1111$\,\,$ & $\,\,$\color{red} 0.2241\color{black} $\,\,$ & $\,\,$0.4184$\,\,$ & $\,\,$ 1  $\,\,$ \\
\end{pmatrix},
\end{equation*}

\begin{equation*}
\mathbf{w}^{\prime} =
\begin{pmatrix}
0.532849\\
0.266424\\
0.141513\\
0.059214
\end{pmatrix} =
0.997767\cdot
\begin{pmatrix}
0.534041\\
\color{gr} 0.267020\color{black} \\
0.141830\\
0.059346
\end{pmatrix},
\end{equation*}
\begin{equation*}
\left[ \frac{{w}^{\prime}_i}{{w}^{\prime}_j} \right] =
\begin{pmatrix}
$\,\,$ 1 $\,\,$ & $\,\,$\color{gr} \color{blue} 2\color{black} $\,\,$ & $\,\,$3.7654$\,\,$ & $\,\,$8.9988$\,\,$ \\
$\,\,$\color{gr} \color{blue}  1/2\color{black} $\,\,$ & $\,\,$ 1 $\,\,$ & $\,\,$\color{gr} 1.8827\color{black} $\,\,$ & $\,\,$\color{gr} 4.4994\color{black}   $\,\,$ \\
$\,\,$0.2656$\,\,$ & $\,\,$\color{gr} 0.5312\color{black} $\,\,$ & $\,\,$ 1 $\,\,$ & $\,\,$2.3899 $\,\,$ \\
$\,\,$0.1111$\,\,$ & $\,\,$\color{gr} 0.2223\color{black} $\,\,$ & $\,\,$0.4184$\,\,$ & $\,\,$ 1  $\,\,$ \\
\end{pmatrix},
\end{equation*}
\end{example}
\newpage
\begin{example}
\begin{equation*}
\mathbf{A} =
\begin{pmatrix}
$\,\,$ 1 $\,\,$ & $\,\,$2$\,\,$ & $\,\,$8$\,\,$ & $\,\,$6 $\,\,$ \\
$\,\,$ 1/2$\,\,$ & $\,\,$ 1 $\,\,$ & $\,\,$6$\,\,$ & $\,\,$2 $\,\,$ \\
$\,\,$ 1/8$\,\,$ & $\,\,$ 1/6$\,\,$ & $\,\,$ 1 $\,\,$ & $\,\,$1 $\,\,$ \\
$\,\,$ 1/6$\,\,$ & $\,\,$ 1/2$\,\,$ & $\,\,$ 1 $\,\,$ & $\,\,$ 1  $\,\,$ \\
\end{pmatrix},
\qquad
\lambda_{\max} =
4.1031,
\qquad
CR = 0.0389
\end{equation*}

\begin{equation*}
\mathbf{w}^{cos} =
\begin{pmatrix}
\color{red} 0.550557\color{black} \\
0.282221\\
0.069185\\
0.098037
\end{pmatrix}\end{equation*}
\begin{equation*}
\left[ \frac{{w}^{cos}_i}{{w}^{cos}_j} \right] =
\begin{pmatrix}
$\,\,$ 1 $\,\,$ & $\,\,$\color{red} 1.9508\color{black} $\,\,$ & $\,\,$\color{red} 7.9577\color{black} $\,\,$ & $\,\,$\color{red} 5.6158\color{black} $\,\,$ \\
$\,\,$\color{red} 0.5126\color{black} $\,\,$ & $\,\,$ 1 $\,\,$ & $\,\,$4.0792$\,\,$ & $\,\,$2.8787  $\,\,$ \\
$\,\,$\color{red} 0.1257\color{black} $\,\,$ & $\,\,$0.2451$\,\,$ & $\,\,$ 1 $\,\,$ & $\,\,$0.7057 $\,\,$ \\
$\,\,$\color{red} 0.1781\color{black} $\,\,$ & $\,\,$0.3474$\,\,$ & $\,\,$1.4170$\,\,$ & $\,\,$ 1  $\,\,$ \\
\end{pmatrix},
\end{equation*}

\begin{equation*}
\mathbf{w}^{\prime} =
\begin{pmatrix}
0.551867\\
0.281398\\
0.068983\\
0.097752
\end{pmatrix} =
0.997085\cdot
\begin{pmatrix}
\color{gr} 0.553481\color{black} \\
0.282221\\
0.069185\\
0.098037
\end{pmatrix},
\end{equation*}
\begin{equation*}
\left[ \frac{{w}^{\prime}_i}{{w}^{\prime}_j} \right] =
\begin{pmatrix}
$\,\,$ 1 $\,\,$ & $\,\,$\color{gr} 1.9612\color{black} $\,\,$ & $\,\,$\color{gr} \color{blue} 8\color{black} $\,\,$ & $\,\,$\color{gr} 5.6456\color{black} $\,\,$ \\
$\,\,$\color{gr} 0.5099\color{black} $\,\,$ & $\,\,$ 1 $\,\,$ & $\,\,$4.0792$\,\,$ & $\,\,$2.8787  $\,\,$ \\
$\,\,$\color{gr} \color{blue}  1/8\color{black} $\,\,$ & $\,\,$0.2451$\,\,$ & $\,\,$ 1 $\,\,$ & $\,\,$0.7057 $\,\,$ \\
$\,\,$\color{gr} 0.1771\color{black} $\,\,$ & $\,\,$0.3474$\,\,$ & $\,\,$1.4170$\,\,$ & $\,\,$ 1  $\,\,$ \\
\end{pmatrix},
\end{equation*}
\end{example}
\newpage
\begin{example}
\begin{equation*}
\mathbf{A} =
\begin{pmatrix}
$\,\,$ 1 $\,\,$ & $\,\,$2$\,\,$ & $\,\,$8$\,\,$ & $\,\,$7 $\,\,$ \\
$\,\,$ 1/2$\,\,$ & $\,\,$ 1 $\,\,$ & $\,\,$2$\,\,$ & $\,\,$5 $\,\,$ \\
$\,\,$ 1/8$\,\,$ & $\,\,$ 1/2$\,\,$ & $\,\,$ 1 $\,\,$ & $\,\,$4 $\,\,$ \\
$\,\,$ 1/7$\,\,$ & $\,\,$ 1/5$\,\,$ & $\,\,$ 1/4$\,\,$ & $\,\,$ 1  $\,\,$ \\
\end{pmatrix},
\qquad
\lambda_{\max} =
4.2035,
\qquad
CR = 0.0767
\end{equation*}

\begin{equation*}
\mathbf{w}^{cos} =
\begin{pmatrix}
0.546977\\
\color{red} 0.260254\color{black} \\
0.137586\\
0.055183
\end{pmatrix}\end{equation*}
\begin{equation*}
\left[ \frac{{w}^{cos}_i}{{w}^{cos}_j} \right] =
\begin{pmatrix}
$\,\,$ 1 $\,\,$ & $\,\,$\color{red} 2.1017\color{black} $\,\,$ & $\,\,$3.9755$\,\,$ & $\,\,$9.9121$\,\,$ \\
$\,\,$\color{red} 0.4758\color{black} $\,\,$ & $\,\,$ 1 $\,\,$ & $\,\,$\color{red} 1.8916\color{black} $\,\,$ & $\,\,$\color{red} 4.7162\color{black}   $\,\,$ \\
$\,\,$0.2515$\,\,$ & $\,\,$\color{red} 0.5287\color{black} $\,\,$ & $\,\,$ 1 $\,\,$ & $\,\,$2.4933 $\,\,$ \\
$\,\,$0.1009$\,\,$ & $\,\,$\color{red} 0.2120\color{black} $\,\,$ & $\,\,$0.4011$\,\,$ & $\,\,$ 1  $\,\,$ \\
\end{pmatrix},
\end{equation*}

\begin{equation*}
\mathbf{w}^{\prime} =
\begin{pmatrix}
0.539833\\
0.269916\\
0.135789\\
0.054462
\end{pmatrix} =
0.986939\cdot
\begin{pmatrix}
0.546977\\
\color{gr} 0.273488\color{black} \\
0.137586\\
0.055183
\end{pmatrix},
\end{equation*}
\begin{equation*}
\left[ \frac{{w}^{\prime}_i}{{w}^{\prime}_j} \right] =
\begin{pmatrix}
$\,\,$ 1 $\,\,$ & $\,\,$\color{gr} \color{blue} 2\color{black} $\,\,$ & $\,\,$3.9755$\,\,$ & $\,\,$9.9121$\,\,$ \\
$\,\,$\color{gr} \color{blue}  1/2\color{black} $\,\,$ & $\,\,$ 1 $\,\,$ & $\,\,$\color{gr} 1.9878\color{black} $\,\,$ & $\,\,$\color{gr} 4.9560\color{black}   $\,\,$ \\
$\,\,$0.2515$\,\,$ & $\,\,$\color{gr} 0.5031\color{black} $\,\,$ & $\,\,$ 1 $\,\,$ & $\,\,$2.4933 $\,\,$ \\
$\,\,$0.1009$\,\,$ & $\,\,$\color{gr} 0.2018\color{black} $\,\,$ & $\,\,$0.4011$\,\,$ & $\,\,$ 1  $\,\,$ \\
\end{pmatrix},
\end{equation*}
\end{example}
\newpage
\begin{example}
\begin{equation*}
\mathbf{A} =
\begin{pmatrix}
$\,\,$ 1 $\,\,$ & $\,\,$2$\,\,$ & $\,\,$8$\,\,$ & $\,\,$8 $\,\,$ \\
$\,\,$ 1/2$\,\,$ & $\,\,$ 1 $\,\,$ & $\,\,$1$\,\,$ & $\,\,$3 $\,\,$ \\
$\,\,$ 1/8$\,\,$ & $\,\,$ 1 $\,\,$ & $\,\,$ 1 $\,\,$ & $\,\,$2 $\,\,$ \\
$\,\,$ 1/8$\,\,$ & $\,\,$ 1/3$\,\,$ & $\,\,$ 1/2$\,\,$ & $\,\,$ 1  $\,\,$ \\
\end{pmatrix},
\qquad
\lambda_{\max} =
4.1707,
\qquad
CR = 0.0644
\end{equation*}

\begin{equation*}
\mathbf{w}^{cos} =
\begin{pmatrix}
0.580027\\
0.211978\\
0.139981\\
\color{red} 0.068014\color{black}
\end{pmatrix}\end{equation*}
\begin{equation*}
\left[ \frac{{w}^{cos}_i}{{w}^{cos}_j} \right] =
\begin{pmatrix}
$\,\,$ 1 $\,\,$ & $\,\,$2.7363$\,\,$ & $\,\,$4.1436$\,\,$ & $\,\,$\color{red} 8.5281\color{black} $\,\,$ \\
$\,\,$0.3655$\,\,$ & $\,\,$ 1 $\,\,$ & $\,\,$1.5143$\,\,$ & $\,\,$\color{red} 3.1167\color{black}   $\,\,$ \\
$\,\,$0.2413$\,\,$ & $\,\,$0.6604$\,\,$ & $\,\,$ 1 $\,\,$ & $\,\,$\color{red} 2.0581\color{black}  $\,\,$ \\
$\,\,$\color{red} 0.1173\color{black} $\,\,$ & $\,\,$\color{red} 0.3209\color{black} $\,\,$ & $\,\,$\color{red} 0.4859\color{black} $\,\,$ & $\,\,$ 1  $\,\,$ \\
\end{pmatrix},
\end{equation*}

\begin{equation*}
\mathbf{w}^{\prime} =
\begin{pmatrix}
0.578883\\
0.211560\\
0.139705\\
0.069852
\end{pmatrix} =
0.998027\cdot
\begin{pmatrix}
0.580027\\
0.211978\\
0.139981\\
\color{gr} 0.069990\color{black}
\end{pmatrix},
\end{equation*}
\begin{equation*}
\left[ \frac{{w}^{\prime}_i}{{w}^{\prime}_j} \right] =
\begin{pmatrix}
$\,\,$ 1 $\,\,$ & $\,\,$2.7363$\,\,$ & $\,\,$4.1436$\,\,$ & $\,\,$\color{gr} 8.2872\color{black} $\,\,$ \\
$\,\,$0.3655$\,\,$ & $\,\,$ 1 $\,\,$ & $\,\,$1.5143$\,\,$ & $\,\,$\color{gr} 3.0287\color{black}   $\,\,$ \\
$\,\,$0.2413$\,\,$ & $\,\,$0.6604$\,\,$ & $\,\,$ 1 $\,\,$ & $\,\,$\color{gr} \color{blue} 2\color{black}  $\,\,$ \\
$\,\,$\color{gr} 0.1207\color{black} $\,\,$ & $\,\,$\color{gr} 0.3302\color{black} $\,\,$ & $\,\,$\color{gr} \color{blue}  1/2\color{black} $\,\,$ & $\,\,$ 1  $\,\,$ \\
\end{pmatrix},
\end{equation*}
\end{example}
\newpage
\begin{example}
\begin{equation*}
\mathbf{A} =
\begin{pmatrix}
$\,\,$ 1 $\,\,$ & $\,\,$2$\,\,$ & $\,\,$8$\,\,$ & $\,\,$8 $\,\,$ \\
$\,\,$ 1/2$\,\,$ & $\,\,$ 1 $\,\,$ & $\,\,$2$\,\,$ & $\,\,$5 $\,\,$ \\
$\,\,$ 1/8$\,\,$ & $\,\,$ 1/2$\,\,$ & $\,\,$ 1 $\,\,$ & $\,\,$4 $\,\,$ \\
$\,\,$ 1/8$\,\,$ & $\,\,$ 1/5$\,\,$ & $\,\,$ 1/4$\,\,$ & $\,\,$ 1  $\,\,$ \\
\end{pmatrix},
\qquad
\lambda_{\max} =
4.1722,
\qquad
CR = 0.0649
\end{equation*}

\begin{equation*}
\mathbf{w}^{cos} =
\begin{pmatrix}
0.558226\\
\color{red} 0.256172\color{black} \\
0.133722\\
0.051880
\end{pmatrix}\end{equation*}
\begin{equation*}
\left[ \frac{{w}^{cos}_i}{{w}^{cos}_j} \right] =
\begin{pmatrix}
$\,\,$ 1 $\,\,$ & $\,\,$\color{red} 2.1791\color{black} $\,\,$ & $\,\,$4.1745$\,\,$ & $\,\,$10.7600$\,\,$ \\
$\,\,$\color{red} 0.4589\color{black} $\,\,$ & $\,\,$ 1 $\,\,$ & $\,\,$\color{red} 1.9157\color{black} $\,\,$ & $\,\,$\color{red} 4.9378\color{black}   $\,\,$ \\
$\,\,$0.2395$\,\,$ & $\,\,$\color{red} 0.5220\color{black} $\,\,$ & $\,\,$ 1 $\,\,$ & $\,\,$2.5775 $\,\,$ \\
$\,\,$0.0929$\,\,$ & $\,\,$\color{red} 0.2025\color{black} $\,\,$ & $\,\,$0.3880$\,\,$ & $\,\,$ 1  $\,\,$ \\
\end{pmatrix},
\end{equation*}

\begin{equation*}
\mathbf{w}^{\prime} =
\begin{pmatrix}
0.556431\\
0.258564\\
0.133292\\
0.051713
\end{pmatrix} =
0.996785\cdot
\begin{pmatrix}
0.558226\\
\color{gr} 0.259398\color{black} \\
0.133722\\
0.051880
\end{pmatrix},
\end{equation*}
\begin{equation*}
\left[ \frac{{w}^{\prime}_i}{{w}^{\prime}_j} \right] =
\begin{pmatrix}
$\,\,$ 1 $\,\,$ & $\,\,$\color{gr} 2.1520\color{black} $\,\,$ & $\,\,$4.1745$\,\,$ & $\,\,$10.7600$\,\,$ \\
$\,\,$\color{gr} 0.4647\color{black} $\,\,$ & $\,\,$ 1 $\,\,$ & $\,\,$\color{gr} 1.9398\color{black} $\,\,$ & $\,\,$\color{gr} \color{blue} 5\color{black}   $\,\,$ \\
$\,\,$0.2395$\,\,$ & $\,\,$\color{gr} 0.5155\color{black} $\,\,$ & $\,\,$ 1 $\,\,$ & $\,\,$2.5775 $\,\,$ \\
$\,\,$0.0929$\,\,$ & $\,\,$\color{gr} \color{blue}  1/5\color{black} $\,\,$ & $\,\,$0.3880$\,\,$ & $\,\,$ 1  $\,\,$ \\
\end{pmatrix},
\end{equation*}
\end{example}
\newpage
\begin{example}
\begin{equation*}
\mathbf{A} =
\begin{pmatrix}
$\,\,$ 1 $\,\,$ & $\,\,$2$\,\,$ & $\,\,$8$\,\,$ & $\,\,$8 $\,\,$ \\
$\,\,$ 1/2$\,\,$ & $\,\,$ 1 $\,\,$ & $\,\,$2$\,\,$ & $\,\,$6 $\,\,$ \\
$\,\,$ 1/8$\,\,$ & $\,\,$ 1/2$\,\,$ & $\,\,$ 1 $\,\,$ & $\,\,$5 $\,\,$ \\
$\,\,$ 1/8$\,\,$ & $\,\,$ 1/6$\,\,$ & $\,\,$ 1/5$\,\,$ & $\,\,$ 1  $\,\,$ \\
\end{pmatrix},
\qquad
\lambda_{\max} =
4.2277,
\qquad
CR = 0.0859
\end{equation*}

\begin{equation*}
\mathbf{w}^{cos} =
\begin{pmatrix}
0.546821\\
\color{red} 0.263441\color{black} \\
0.142521\\
0.047217
\end{pmatrix}\end{equation*}
\begin{equation*}
\left[ \frac{{w}^{cos}_i}{{w}^{cos}_j} \right] =
\begin{pmatrix}
$\,\,$ 1 $\,\,$ & $\,\,$\color{red} 2.0757\color{black} $\,\,$ & $\,\,$3.8368$\,\,$ & $\,\,$11.5809$\,\,$ \\
$\,\,$\color{red} 0.4818\color{black} $\,\,$ & $\,\,$ 1 $\,\,$ & $\,\,$\color{red} 1.8484\color{black} $\,\,$ & $\,\,$\color{red} 5.5793\color{black}   $\,\,$ \\
$\,\,$0.2606$\,\,$ & $\,\,$\color{red} 0.5410\color{black} $\,\,$ & $\,\,$ 1 $\,\,$ & $\,\,$3.0184 $\,\,$ \\
$\,\,$0.0863$\,\,$ & $\,\,$\color{red} 0.1792\color{black} $\,\,$ & $\,\,$0.3313$\,\,$ & $\,\,$ 1  $\,\,$ \\
\end{pmatrix},
\end{equation*}

\begin{equation*}
\mathbf{w}^{\prime} =
\begin{pmatrix}
0.541423\\
0.270711\\
0.141115\\
0.046751
\end{pmatrix} =
0.990129\cdot
\begin{pmatrix}
0.546821\\
\color{gr} 0.273410\color{black} \\
0.142521\\
0.047217
\end{pmatrix},
\end{equation*}
\begin{equation*}
\left[ \frac{{w}^{\prime}_i}{{w}^{\prime}_j} \right] =
\begin{pmatrix}
$\,\,$ 1 $\,\,$ & $\,\,$\color{gr} \color{blue} 2\color{black} $\,\,$ & $\,\,$3.8368$\,\,$ & $\,\,$11.5809$\,\,$ \\
$\,\,$\color{gr} \color{blue}  1/2\color{black} $\,\,$ & $\,\,$ 1 $\,\,$ & $\,\,$\color{gr} 1.9184\color{black} $\,\,$ & $\,\,$\color{gr} 5.7905\color{black}   $\,\,$ \\
$\,\,$0.2606$\,\,$ & $\,\,$\color{gr} 0.5213\color{black} $\,\,$ & $\,\,$ 1 $\,\,$ & $\,\,$3.0184 $\,\,$ \\
$\,\,$0.0863$\,\,$ & $\,\,$\color{gr} 0.1727\color{black} $\,\,$ & $\,\,$0.3313$\,\,$ & $\,\,$ 1  $\,\,$ \\
\end{pmatrix},
\end{equation*}
\end{example}
\newpage
\begin{example}
\begin{equation*}
\mathbf{A} =
\begin{pmatrix}
$\,\,$ 1 $\,\,$ & $\,\,$2$\,\,$ & $\,\,$8$\,\,$ & $\,\,$9 $\,\,$ \\
$\,\,$ 1/2$\,\,$ & $\,\,$ 1 $\,\,$ & $\,\,$1$\,\,$ & $\,\,$3 $\,\,$ \\
$\,\,$ 1/8$\,\,$ & $\,\,$ 1 $\,\,$ & $\,\,$ 1 $\,\,$ & $\,\,$2 $\,\,$ \\
$\,\,$ 1/9$\,\,$ & $\,\,$ 1/3$\,\,$ & $\,\,$ 1/2$\,\,$ & $\,\,$ 1  $\,\,$ \\
\end{pmatrix},
\qquad
\lambda_{\max} =
4.1664,
\qquad
CR = 0.0627
\end{equation*}

\begin{equation*}
\mathbf{w}^{cos} =
\begin{pmatrix}
0.588442\\
0.208816\\
0.137822\\
\color{red} 0.064920\color{black}
\end{pmatrix}\end{equation*}
\begin{equation*}
\left[ \frac{{w}^{cos}_i}{{w}^{cos}_j} \right] =
\begin{pmatrix}
$\,\,$ 1 $\,\,$ & $\,\,$2.8180$\,\,$ & $\,\,$4.2696$\,\,$ & $\,\,$\color{red} 9.0642\color{black} $\,\,$ \\
$\,\,$0.3549$\,\,$ & $\,\,$ 1 $\,\,$ & $\,\,$1.5151$\,\,$ & $\,\,$\color{red} 3.2165\color{black}   $\,\,$ \\
$\,\,$0.2342$\,\,$ & $\,\,$0.6600$\,\,$ & $\,\,$ 1 $\,\,$ & $\,\,$\color{red} 2.1230\color{black}  $\,\,$ \\
$\,\,$\color{red} 0.1103\color{black} $\,\,$ & $\,\,$\color{red} 0.3109\color{black} $\,\,$ & $\,\,$\color{red} 0.4710\color{black} $\,\,$ & $\,\,$ 1  $\,\,$ \\
\end{pmatrix},
\end{equation*}

\begin{equation*}
\mathbf{w}^{\prime} =
\begin{pmatrix}
0.588169\\
0.208720\\
0.137759\\
0.065352
\end{pmatrix} =
0.999537\cdot
\begin{pmatrix}
0.588442\\
0.208816\\
0.137822\\
\color{gr} 0.065382\color{black}
\end{pmatrix},
\end{equation*}
\begin{equation*}
\left[ \frac{{w}^{\prime}_i}{{w}^{\prime}_j} \right] =
\begin{pmatrix}
$\,\,$ 1 $\,\,$ & $\,\,$2.8180$\,\,$ & $\,\,$4.2696$\,\,$ & $\,\,$\color{gr} \color{blue} 9\color{black} $\,\,$ \\
$\,\,$0.3549$\,\,$ & $\,\,$ 1 $\,\,$ & $\,\,$1.5151$\,\,$ & $\,\,$\color{gr} 3.1938\color{black}   $\,\,$ \\
$\,\,$0.2342$\,\,$ & $\,\,$0.6600$\,\,$ & $\,\,$ 1 $\,\,$ & $\,\,$\color{gr} 2.1079\color{black}  $\,\,$ \\
$\,\,$\color{gr} \color{blue}  1/9\color{black} $\,\,$ & $\,\,$\color{gr} 0.3131\color{black} $\,\,$ & $\,\,$\color{gr} 0.4744\color{black} $\,\,$ & $\,\,$ 1  $\,\,$ \\
\end{pmatrix},
\end{equation*}
\end{example}
\newpage
\begin{example}
\begin{equation*}
\mathbf{A} =
\begin{pmatrix}
$\,\,$ 1 $\,\,$ & $\,\,$2$\,\,$ & $\,\,$8$\,\,$ & $\,\,$9 $\,\,$ \\
$\,\,$ 1/2$\,\,$ & $\,\,$ 1 $\,\,$ & $\,\,$2$\,\,$ & $\,\,$6 $\,\,$ \\
$\,\,$ 1/8$\,\,$ & $\,\,$ 1/2$\,\,$ & $\,\,$ 1 $\,\,$ & $\,\,$5 $\,\,$ \\
$\,\,$ 1/9$\,\,$ & $\,\,$ 1/6$\,\,$ & $\,\,$ 1/5$\,\,$ & $\,\,$ 1  $\,\,$ \\
\end{pmatrix},
\qquad
\lambda_{\max} =
4.1974,
\qquad
CR = 0.0744
\end{equation*}

\begin{equation*}
\mathbf{w}^{cos} =
\begin{pmatrix}
0.556555\\
\color{red} 0.259793\color{black} \\
0.138985\\
0.044666
\end{pmatrix}\end{equation*}
\begin{equation*}
\left[ \frac{{w}^{cos}_i}{{w}^{cos}_j} \right] =
\begin{pmatrix}
$\,\,$ 1 $\,\,$ & $\,\,$\color{red} 2.1423\color{black} $\,\,$ & $\,\,$4.0044$\,\,$ & $\,\,$12.4604$\,\,$ \\
$\,\,$\color{red} 0.4668\color{black} $\,\,$ & $\,\,$ 1 $\,\,$ & $\,\,$\color{red} 1.8692\color{black} $\,\,$ & $\,\,$\color{red} 5.8164\color{black}   $\,\,$ \\
$\,\,$0.2497$\,\,$ & $\,\,$\color{red} 0.5350\color{black} $\,\,$ & $\,\,$ 1 $\,\,$ & $\,\,$3.1117 $\,\,$ \\
$\,\,$0.0803$\,\,$ & $\,\,$\color{red} 0.1719\color{black} $\,\,$ & $\,\,$0.3214$\,\,$ & $\,\,$ 1  $\,\,$ \\
\end{pmatrix},
\end{equation*}

\begin{equation*}
\mathbf{w}^{\prime} =
\begin{pmatrix}
0.552027\\
0.265816\\
0.137855\\
0.044303
\end{pmatrix} =
0.991864\cdot
\begin{pmatrix}
0.556555\\
\color{gr} 0.267996\color{black} \\
0.138985\\
0.044666
\end{pmatrix},
\end{equation*}
\begin{equation*}
\left[ \frac{{w}^{\prime}_i}{{w}^{\prime}_j} \right] =
\begin{pmatrix}
$\,\,$ 1 $\,\,$ & $\,\,$\color{gr} 2.0767\color{black} $\,\,$ & $\,\,$4.0044$\,\,$ & $\,\,$12.4604$\,\,$ \\
$\,\,$\color{gr} 0.4815\color{black} $\,\,$ & $\,\,$ 1 $\,\,$ & $\,\,$\color{gr} 1.9282\color{black} $\,\,$ & $\,\,$\color{gr} \color{blue} 6\color{black}   $\,\,$ \\
$\,\,$0.2497$\,\,$ & $\,\,$\color{gr} 0.5186\color{black} $\,\,$ & $\,\,$ 1 $\,\,$ & $\,\,$3.1117 $\,\,$ \\
$\,\,$0.0803$\,\,$ & $\,\,$\color{gr} \color{blue}  1/6\color{black} $\,\,$ & $\,\,$0.3214$\,\,$ & $\,\,$ 1  $\,\,$ \\
\end{pmatrix},
\end{equation*}
\end{example}
\newpage
\begin{example}
\begin{equation*}
\mathbf{A} =
\begin{pmatrix}
$\,\,$ 1 $\,\,$ & $\,\,$2$\,\,$ & $\,\,$8$\,\,$ & $\,\,$9 $\,\,$ \\
$\,\,$ 1/2$\,\,$ & $\,\,$ 1 $\,\,$ & $\,\,$2$\,\,$ & $\,\,$6 $\,\,$ \\
$\,\,$ 1/8$\,\,$ & $\,\,$ 1/2$\,\,$ & $\,\,$ 1 $\,\,$ & $\,\,$6 $\,\,$ \\
$\,\,$ 1/9$\,\,$ & $\,\,$ 1/6$\,\,$ & $\,\,$ 1/6$\,\,$ & $\,\,$ 1  $\,\,$ \\
\end{pmatrix},
\qquad
\lambda_{\max} =
4.2512,
\qquad
CR = 0.0947
\end{equation*}

\begin{equation*}
\mathbf{w}^{cos} =
\begin{pmatrix}
0.551047\\
\color{red} 0.256343\color{black} \\
0.149164\\
0.043446
\end{pmatrix}\end{equation*}
\begin{equation*}
\left[ \frac{{w}^{cos}_i}{{w}^{cos}_j} \right] =
\begin{pmatrix}
$\,\,$ 1 $\,\,$ & $\,\,$\color{red} 2.1496\color{black} $\,\,$ & $\,\,$3.6942$\,\,$ & $\,\,$12.6835$\,\,$ \\
$\,\,$\color{red} 0.4652\color{black} $\,\,$ & $\,\,$ 1 $\,\,$ & $\,\,$\color{red} 1.7185\color{black} $\,\,$ & $\,\,$\color{red} 5.9003\color{black}   $\,\,$ \\
$\,\,$0.2707$\,\,$ & $\,\,$\color{red} 0.5819\color{black} $\,\,$ & $\,\,$ 1 $\,\,$ & $\,\,$3.4333 $\,\,$ \\
$\,\,$0.0788$\,\,$ & $\,\,$\color{red} 0.1695\color{black} $\,\,$ & $\,\,$0.2913$\,\,$ & $\,\,$ 1  $\,\,$ \\
\end{pmatrix},
\end{equation*}

\begin{equation*}
\mathbf{w}^{\prime} =
\begin{pmatrix}
0.548670\\
0.259551\\
0.148521\\
0.043258
\end{pmatrix} =
0.995687\cdot
\begin{pmatrix}
0.551047\\
\color{gr} 0.260675\color{black} \\
0.149164\\
0.043446
\end{pmatrix},
\end{equation*}
\begin{equation*}
\left[ \frac{{w}^{\prime}_i}{{w}^{\prime}_j} \right] =
\begin{pmatrix}
$\,\,$ 1 $\,\,$ & $\,\,$\color{gr} 2.1139\color{black} $\,\,$ & $\,\,$3.6942$\,\,$ & $\,\,$12.6835$\,\,$ \\
$\,\,$\color{gr} 0.4731\color{black} $\,\,$ & $\,\,$ 1 $\,\,$ & $\,\,$\color{gr} 1.7476\color{black} $\,\,$ & $\,\,$\color{gr} \color{blue} 6\color{black}   $\,\,$ \\
$\,\,$0.2707$\,\,$ & $\,\,$\color{gr} 0.5722\color{black} $\,\,$ & $\,\,$ 1 $\,\,$ & $\,\,$3.4333 $\,\,$ \\
$\,\,$0.0788$\,\,$ & $\,\,$\color{gr} \color{blue}  1/6\color{black} $\,\,$ & $\,\,$0.2913$\,\,$ & $\,\,$ 1  $\,\,$ \\
\end{pmatrix},
\end{equation*}
\end{example}
\newpage
\begin{example}
\begin{equation*}
\mathbf{A} =
\begin{pmatrix}
$\,\,$ 1 $\,\,$ & $\,\,$2$\,\,$ & $\,\,$8$\,\,$ & $\,\,$9 $\,\,$ \\
$\,\,$ 1/2$\,\,$ & $\,\,$ 1 $\,\,$ & $\,\,$2$\,\,$ & $\,\,$7 $\,\,$ \\
$\,\,$ 1/8$\,\,$ & $\,\,$ 1/2$\,\,$ & $\,\,$ 1 $\,\,$ & $\,\,$5 $\,\,$ \\
$\,\,$ 1/9$\,\,$ & $\,\,$ 1/7$\,\,$ & $\,\,$ 1/5$\,\,$ & $\,\,$ 1  $\,\,$ \\
\end{pmatrix},
\qquad
\lambda_{\max} =
4.1961,
\qquad
CR = 0.0739
\end{equation*}

\begin{equation*}
\mathbf{w}^{cos} =
\begin{pmatrix}
0.551865\\
\color{red} 0.269427\color{black} \\
0.136242\\
0.042467
\end{pmatrix}\end{equation*}
\begin{equation*}
\left[ \frac{{w}^{cos}_i}{{w}^{cos}_j} \right] =
\begin{pmatrix}
$\,\,$ 1 $\,\,$ & $\,\,$\color{red} 2.0483\color{black} $\,\,$ & $\,\,$4.0506$\,\,$ & $\,\,$12.9952$\,\,$ \\
$\,\,$\color{red} 0.4882\color{black} $\,\,$ & $\,\,$ 1 $\,\,$ & $\,\,$\color{red} 1.9776\color{black} $\,\,$ & $\,\,$\color{red} 6.3444\color{black}   $\,\,$ \\
$\,\,$0.2469$\,\,$ & $\,\,$\color{red} 0.5057\color{black} $\,\,$ & $\,\,$ 1 $\,\,$ & $\,\,$3.2082 $\,\,$ \\
$\,\,$0.0770$\,\,$ & $\,\,$\color{red} 0.1576\color{black} $\,\,$ & $\,\,$0.3117$\,\,$ & $\,\,$ 1  $\,\,$ \\
\end{pmatrix},
\end{equation*}

\begin{equation*}
\mathbf{w}^{\prime} =
\begin{pmatrix}
0.550183\\
0.271653\\
0.135826\\
0.042338
\end{pmatrix} =
0.996953\cdot
\begin{pmatrix}
0.551865\\
\color{gr} 0.272483\color{black} \\
0.136242\\
0.042467
\end{pmatrix},
\end{equation*}
\begin{equation*}
\left[ \frac{{w}^{\prime}_i}{{w}^{\prime}_j} \right] =
\begin{pmatrix}
$\,\,$ 1 $\,\,$ & $\,\,$\color{gr} 2.0253\color{black} $\,\,$ & $\,\,$4.0506$\,\,$ & $\,\,$12.9952$\,\,$ \\
$\,\,$\color{gr} 0.4937\color{black} $\,\,$ & $\,\,$ 1 $\,\,$ & $\,\,$\color{gr} \color{blue} 2\color{black} $\,\,$ & $\,\,$\color{gr} 6.4164\color{black}   $\,\,$ \\
$\,\,$0.2469$\,\,$ & $\,\,$\color{gr} \color{blue}  1/2\color{black} $\,\,$ & $\,\,$ 1 $\,\,$ & $\,\,$3.2082 $\,\,$ \\
$\,\,$0.0770$\,\,$ & $\,\,$\color{gr} 0.1559\color{black} $\,\,$ & $\,\,$0.3117$\,\,$ & $\,\,$ 1  $\,\,$ \\
\end{pmatrix},
\end{equation*}
\end{example}
\newpage
\begin{example}
\begin{equation*}
\mathbf{A} =
\begin{pmatrix}
$\,\,$ 1 $\,\,$ & $\,\,$2$\,\,$ & $\,\,$8$\,\,$ & $\,\,$9 $\,\,$ \\
$\,\,$ 1/2$\,\,$ & $\,\,$ 1 $\,\,$ & $\,\,$2$\,\,$ & $\,\,$7 $\,\,$ \\
$\,\,$ 1/8$\,\,$ & $\,\,$ 1/2$\,\,$ & $\,\,$ 1 $\,\,$ & $\,\,$6 $\,\,$ \\
$\,\,$ 1/9$\,\,$ & $\,\,$ 1/7$\,\,$ & $\,\,$ 1/6$\,\,$ & $\,\,$ 1  $\,\,$ \\
\end{pmatrix},
\qquad
\lambda_{\max} =
4.2463,
\qquad
CR = 0.0929
\end{equation*}

\begin{equation*}
\mathbf{w}^{cos} =
\begin{pmatrix}
0.546734\\
\color{red} 0.265800\color{black} \\
0.146169\\
0.041297
\end{pmatrix}\end{equation*}
\begin{equation*}
\left[ \frac{{w}^{cos}_i}{{w}^{cos}_j} \right] =
\begin{pmatrix}
$\,\,$ 1 $\,\,$ & $\,\,$\color{red} 2.0569\color{black} $\,\,$ & $\,\,$3.7404$\,\,$ & $\,\,$13.2390$\,\,$ \\
$\,\,$\color{red} 0.4862\color{black} $\,\,$ & $\,\,$ 1 $\,\,$ & $\,\,$\color{red} 1.8184\color{black} $\,\,$ & $\,\,$\color{red} 6.4363\color{black}   $\,\,$ \\
$\,\,$0.2673$\,\,$ & $\,\,$\color{red} 0.5499\color{black} $\,\,$ & $\,\,$ 1 $\,\,$ & $\,\,$3.5394 $\,\,$ \\
$\,\,$0.0755$\,\,$ & $\,\,$\color{red} 0.1554\color{black} $\,\,$ & $\,\,$0.2825$\,\,$ & $\,\,$ 1  $\,\,$ \\
\end{pmatrix},
\end{equation*}

\begin{equation*}
\mathbf{w}^{\prime} =
\begin{pmatrix}
0.542628\\
0.271314\\
0.145071\\
0.040987
\end{pmatrix} =
0.992490\cdot
\begin{pmatrix}
0.546734\\
\color{gr} 0.273367\color{black} \\
0.146169\\
0.041297
\end{pmatrix},
\end{equation*}
\begin{equation*}
\left[ \frac{{w}^{\prime}_i}{{w}^{\prime}_j} \right] =
\begin{pmatrix}
$\,\,$ 1 $\,\,$ & $\,\,$\color{gr} \color{blue} 2\color{black} $\,\,$ & $\,\,$3.7404$\,\,$ & $\,\,$13.2390$\,\,$ \\
$\,\,$\color{gr} \color{blue}  1/2\color{black} $\,\,$ & $\,\,$ 1 $\,\,$ & $\,\,$\color{gr} 1.8702\color{black} $\,\,$ & $\,\,$\color{gr} 6.6195\color{black}   $\,\,$ \\
$\,\,$0.2673$\,\,$ & $\,\,$\color{gr} 0.5347\color{black} $\,\,$ & $\,\,$ 1 $\,\,$ & $\,\,$3.5394 $\,\,$ \\
$\,\,$0.0755$\,\,$ & $\,\,$\color{gr} 0.1511\color{black} $\,\,$ & $\,\,$0.2825$\,\,$ & $\,\,$ 1  $\,\,$ \\
\end{pmatrix},
\end{equation*}
\end{example}
\newpage
\begin{example}
\begin{equation*}
\mathbf{A} =
\begin{pmatrix}
$\,\,$ 1 $\,\,$ & $\,\,$2$\,\,$ & $\,\,$9$\,\,$ & $\,\,$3 $\,\,$ \\
$\,\,$ 1/2$\,\,$ & $\,\,$ 1 $\,\,$ & $\,\,$3$\,\,$ & $\,\,$2 $\,\,$ \\
$\,\,$ 1/9$\,\,$ & $\,\,$ 1/3$\,\,$ & $\,\,$ 1 $\,\,$ & $\,\,$1 $\,\,$ \\
$\,\,$ 1/3$\,\,$ & $\,\,$ 1/2$\,\,$ & $\,\,$ 1 $\,\,$ & $\,\,$ 1  $\,\,$ \\
\end{pmatrix},
\qquad
\lambda_{\max} =
4.1031,
\qquad
CR = 0.0389
\end{equation*}

\begin{equation*}
\mathbf{w}^{cos} =
\begin{pmatrix}
0.521135\\
\color{red} 0.256422\color{black} \\
0.091348\\
0.131094
\end{pmatrix}\end{equation*}
\begin{equation*}
\left[ \frac{{w}^{cos}_i}{{w}^{cos}_j} \right] =
\begin{pmatrix}
$\,\,$ 1 $\,\,$ & $\,\,$\color{red} 2.0323\color{black} $\,\,$ & $\,\,$5.7049$\,\,$ & $\,\,$3.9753$\,\,$ \\
$\,\,$\color{red} 0.4920\color{black} $\,\,$ & $\,\,$ 1 $\,\,$ & $\,\,$\color{red} 2.8071\color{black} $\,\,$ & $\,\,$\color{red} 1.9560\color{black}   $\,\,$ \\
$\,\,$0.1753$\,\,$ & $\,\,$\color{red} 0.3562\color{black} $\,\,$ & $\,\,$ 1 $\,\,$ & $\,\,$0.6968 $\,\,$ \\
$\,\,$0.2516$\,\,$ & $\,\,$\color{red} 0.5112\color{black} $\,\,$ & $\,\,$1.4351$\,\,$ & $\,\,$ 1  $\,\,$ \\
\end{pmatrix},
\end{equation*}

\begin{equation*}
\mathbf{w}^{\prime} =
\begin{pmatrix}
0.518984\\
0.259492\\
0.090971\\
0.130553
\end{pmatrix} =
0.995871\cdot
\begin{pmatrix}
0.521135\\
\color{gr} 0.260568\color{black} \\
0.091348\\
0.131094
\end{pmatrix},
\end{equation*}
\begin{equation*}
\left[ \frac{{w}^{\prime}_i}{{w}^{\prime}_j} \right] =
\begin{pmatrix}
$\,\,$ 1 $\,\,$ & $\,\,$\color{gr} \color{blue} 2\color{black} $\,\,$ & $\,\,$5.7049$\,\,$ & $\,\,$3.9753$\,\,$ \\
$\,\,$\color{gr} \color{blue}  1/2\color{black} $\,\,$ & $\,\,$ 1 $\,\,$ & $\,\,$\color{gr} 2.8525\color{black} $\,\,$ & $\,\,$\color{gr} 1.9876\color{black}   $\,\,$ \\
$\,\,$0.1753$\,\,$ & $\,\,$\color{gr} 0.3506\color{black} $\,\,$ & $\,\,$ 1 $\,\,$ & $\,\,$0.6968 $\,\,$ \\
$\,\,$0.2516$\,\,$ & $\,\,$\color{gr} 0.5031\color{black} $\,\,$ & $\,\,$1.4351$\,\,$ & $\,\,$ 1  $\,\,$ \\
\end{pmatrix},
\end{equation*}
\end{example}
\newpage
\begin{example}
\begin{equation*}
\mathbf{A} =
\begin{pmatrix}
$\,\,$ 1 $\,\,$ & $\,\,$2$\,\,$ & $\,\,$9$\,\,$ & $\,\,$5 $\,\,$ \\
$\,\,$ 1/2$\,\,$ & $\,\,$ 1 $\,\,$ & $\,\,$2$\,\,$ & $\,\,$4 $\,\,$ \\
$\,\,$ 1/9$\,\,$ & $\,\,$ 1/2$\,\,$ & $\,\,$ 1 $\,\,$ & $\,\,$3 $\,\,$ \\
$\,\,$ 1/5$\,\,$ & $\,\,$ 1/4$\,\,$ & $\,\,$ 1/3$\,\,$ & $\,\,$ 1  $\,\,$ \\
\end{pmatrix},
\qquad
\lambda_{\max} =
4.2541,
\qquad
CR = 0.0958
\end{equation*}

\begin{equation*}
\mathbf{w}^{cos} =
\begin{pmatrix}
0.536502\\
\color{red} 0.258922\color{black} \\
0.132340\\
0.072236
\end{pmatrix}\end{equation*}
\begin{equation*}
\left[ \frac{{w}^{cos}_i}{{w}^{cos}_j} \right] =
\begin{pmatrix}
$\,\,$ 1 $\,\,$ & $\,\,$\color{red} 2.0721\color{black} $\,\,$ & $\,\,$4.0540$\,\,$ & $\,\,$7.4271$\,\,$ \\
$\,\,$\color{red} 0.4826\color{black} $\,\,$ & $\,\,$ 1 $\,\,$ & $\,\,$\color{red} 1.9565\color{black} $\,\,$ & $\,\,$\color{red} 3.5844\color{black}   $\,\,$ \\
$\,\,$0.2467$\,\,$ & $\,\,$\color{red} 0.5111\color{black} $\,\,$ & $\,\,$ 1 $\,\,$ & $\,\,$1.8321 $\,\,$ \\
$\,\,$0.1346$\,\,$ & $\,\,$\color{red} 0.2790\color{black} $\,\,$ & $\,\,$0.5458$\,\,$ & $\,\,$ 1  $\,\,$ \\
\end{pmatrix},
\end{equation*}

\begin{equation*}
\mathbf{w}^{\prime} =
\begin{pmatrix}
0.533430\\
0.263165\\
0.131583\\
0.071822
\end{pmatrix} =
0.994274\cdot
\begin{pmatrix}
0.536502\\
\color{gr} 0.264681\color{black} \\
0.132340\\
0.072236
\end{pmatrix},
\end{equation*}
\begin{equation*}
\left[ \frac{{w}^{\prime}_i}{{w}^{\prime}_j} \right] =
\begin{pmatrix}
$\,\,$ 1 $\,\,$ & $\,\,$\color{gr} 2.0270\color{black} $\,\,$ & $\,\,$4.0540$\,\,$ & $\,\,$7.4271$\,\,$ \\
$\,\,$\color{gr} 0.4933\color{black} $\,\,$ & $\,\,$ 1 $\,\,$ & $\,\,$\color{gr} \color{blue} 2\color{black} $\,\,$ & $\,\,$\color{gr} 3.6641\color{black}   $\,\,$ \\
$\,\,$0.2467$\,\,$ & $\,\,$\color{gr} \color{blue}  1/2\color{black} $\,\,$ & $\,\,$ 1 $\,\,$ & $\,\,$1.8321 $\,\,$ \\
$\,\,$0.1346$\,\,$ & $\,\,$\color{gr} 0.2729\color{black} $\,\,$ & $\,\,$0.5458$\,\,$ & $\,\,$ 1  $\,\,$ \\
\end{pmatrix},
\end{equation*}
\end{example}
\newpage
\begin{example}
\begin{equation*}
\mathbf{A} =
\begin{pmatrix}
$\,\,$ 1 $\,\,$ & $\,\,$2$\,\,$ & $\,\,$9$\,\,$ & $\,\,$6 $\,\,$ \\
$\,\,$ 1/2$\,\,$ & $\,\,$ 1 $\,\,$ & $\,\,$2$\,\,$ & $\,\,$4 $\,\,$ \\
$\,\,$ 1/9$\,\,$ & $\,\,$ 1/2$\,\,$ & $\,\,$ 1 $\,\,$ & $\,\,$3 $\,\,$ \\
$\,\,$ 1/6$\,\,$ & $\,\,$ 1/4$\,\,$ & $\,\,$ 1/3$\,\,$ & $\,\,$ 1  $\,\,$ \\
\end{pmatrix},
\qquad
\lambda_{\max} =
4.2052,
\qquad
CR = 0.0774
\end{equation*}

\begin{equation*}
\mathbf{w}^{cos} =
\begin{pmatrix}
0.552448\\
\color{red} 0.253617\color{black} \\
0.127558\\
0.066377
\end{pmatrix}\end{equation*}
\begin{equation*}
\left[ \frac{{w}^{cos}_i}{{w}^{cos}_j} \right] =
\begin{pmatrix}
$\,\,$ 1 $\,\,$ & $\,\,$\color{red} 2.1783\color{black} $\,\,$ & $\,\,$4.3310$\,\,$ & $\,\,$8.3229$\,\,$ \\
$\,\,$\color{red} 0.4591\color{black} $\,\,$ & $\,\,$ 1 $\,\,$ & $\,\,$\color{red} 1.9882\color{black} $\,\,$ & $\,\,$\color{red} 3.8209\color{black}   $\,\,$ \\
$\,\,$0.2309$\,\,$ & $\,\,$\color{red} 0.5030\color{black} $\,\,$ & $\,\,$ 1 $\,\,$ & $\,\,$1.9217 $\,\,$ \\
$\,\,$0.1201$\,\,$ & $\,\,$\color{red} 0.2617\color{black} $\,\,$ & $\,\,$0.5204$\,\,$ & $\,\,$ 1  $\,\,$ \\
\end{pmatrix},
\end{equation*}

\begin{equation*}
\mathbf{w}^{\prime} =
\begin{pmatrix}
0.551622\\
0.254734\\
0.127367\\
0.066277
\end{pmatrix} =
0.998503\cdot
\begin{pmatrix}
0.552448\\
\color{gr} 0.255116\color{black} \\
0.127558\\
0.066377
\end{pmatrix},
\end{equation*}
\begin{equation*}
\left[ \frac{{w}^{\prime}_i}{{w}^{\prime}_j} \right] =
\begin{pmatrix}
$\,\,$ 1 $\,\,$ & $\,\,$\color{gr} 2.1655\color{black} $\,\,$ & $\,\,$4.3310$\,\,$ & $\,\,$8.3229$\,\,$ \\
$\,\,$\color{gr} 0.4618\color{black} $\,\,$ & $\,\,$ 1 $\,\,$ & $\,\,$\color{gr} \color{blue} 2\color{black} $\,\,$ & $\,\,$\color{gr} 3.8435\color{black}   $\,\,$ \\
$\,\,$0.2309$\,\,$ & $\,\,$\color{gr} \color{blue}  1/2\color{black} $\,\,$ & $\,\,$ 1 $\,\,$ & $\,\,$1.9217 $\,\,$ \\
$\,\,$0.1201$\,\,$ & $\,\,$\color{gr} 0.2602\color{black} $\,\,$ & $\,\,$0.5204$\,\,$ & $\,\,$ 1  $\,\,$ \\
\end{pmatrix},
\end{equation*}
\end{example}
\newpage
\begin{example}
\begin{equation*}
\mathbf{A} =
\begin{pmatrix}
$\,\,$ 1 $\,\,$ & $\,\,$2$\,\,$ & $\,\,$9$\,\,$ & $\,\,$6 $\,\,$ \\
$\,\,$ 1/2$\,\,$ & $\,\,$ 1 $\,\,$ & $\,\,$3$\,\,$ & $\,\,$4 $\,\,$ \\
$\,\,$ 1/9$\,\,$ & $\,\,$ 1/3$\,\,$ & $\,\,$ 1 $\,\,$ & $\,\,$2 $\,\,$ \\
$\,\,$ 1/6$\,\,$ & $\,\,$ 1/4$\,\,$ & $\,\,$ 1/2$\,\,$ & $\,\,$ 1  $\,\,$ \\
\end{pmatrix},
\qquad
\lambda_{\max} =
4.1031,
\qquad
CR = 0.0389
\end{equation*}

\begin{equation*}
\mathbf{w}^{cos} =
\begin{pmatrix}
0.557162\\
\color{red} 0.274665\color{black} \\
0.097832\\
0.070341
\end{pmatrix}\end{equation*}
\begin{equation*}
\left[ \frac{{w}^{cos}_i}{{w}^{cos}_j} \right] =
\begin{pmatrix}
$\,\,$ 1 $\,\,$ & $\,\,$\color{red} 2.0285\color{black} $\,\,$ & $\,\,$5.6951$\,\,$ & $\,\,$7.9208$\,\,$ \\
$\,\,$\color{red} 0.4930\color{black} $\,\,$ & $\,\,$ 1 $\,\,$ & $\,\,$\color{red} 2.8075\color{black} $\,\,$ & $\,\,$\color{red} 3.9047\color{black}   $\,\,$ \\
$\,\,$0.1756$\,\,$ & $\,\,$\color{red} 0.3562\color{black} $\,\,$ & $\,\,$ 1 $\,\,$ & $\,\,$1.3908 $\,\,$ \\
$\,\,$0.1262$\,\,$ & $\,\,$\color{red} 0.2561\color{black} $\,\,$ & $\,\,$0.7190$\,\,$ & $\,\,$ 1  $\,\,$ \\
\end{pmatrix},
\end{equation*}

\begin{equation*}
\mathbf{w}^{\prime} =
\begin{pmatrix}
0.554989\\
0.277494\\
0.097450\\
0.070067
\end{pmatrix} =
0.996100\cdot
\begin{pmatrix}
0.557162\\
\color{gr} 0.278581\color{black} \\
0.097832\\
0.070341
\end{pmatrix},
\end{equation*}
\begin{equation*}
\left[ \frac{{w}^{\prime}_i}{{w}^{\prime}_j} \right] =
\begin{pmatrix}
$\,\,$ 1 $\,\,$ & $\,\,$\color{gr} \color{blue} 2\color{black} $\,\,$ & $\,\,$5.6951$\,\,$ & $\,\,$7.9208$\,\,$ \\
$\,\,$\color{gr} \color{blue}  1/2\color{black} $\,\,$ & $\,\,$ 1 $\,\,$ & $\,\,$\color{gr} 2.8476\color{black} $\,\,$ & $\,\,$\color{gr} 3.9604\color{black}   $\,\,$ \\
$\,\,$0.1756$\,\,$ & $\,\,$\color{gr} 0.3512\color{black} $\,\,$ & $\,\,$ 1 $\,\,$ & $\,\,$1.3908 $\,\,$ \\
$\,\,$0.1262$\,\,$ & $\,\,$\color{gr} 0.2525\color{black} $\,\,$ & $\,\,$0.7190$\,\,$ & $\,\,$ 1  $\,\,$ \\
\end{pmatrix},
\end{equation*}
\end{example}
\newpage
\begin{example}
\begin{equation*}
\mathbf{A} =
\begin{pmatrix}
$\,\,$ 1 $\,\,$ & $\,\,$2$\,\,$ & $\,\,$9$\,\,$ & $\,\,$6 $\,\,$ \\
$\,\,$ 1/2$\,\,$ & $\,\,$ 1 $\,\,$ & $\,\,$7$\,\,$ & $\,\,$2 $\,\,$ \\
$\,\,$ 1/9$\,\,$ & $\,\,$ 1/7$\,\,$ & $\,\,$ 1 $\,\,$ & $\,\,$1 $\,\,$ \\
$\,\,$ 1/6$\,\,$ & $\,\,$ 1/2$\,\,$ & $\,\,$ 1 $\,\,$ & $\,\,$ 1  $\,\,$ \\
\end{pmatrix},
\qquad
\lambda_{\max} =
4.1342,
\qquad
CR = 0.0506
\end{equation*}

\begin{equation*}
\mathbf{w}^{cos} =
\begin{pmatrix}
\color{red} 0.552687\color{black} \\
0.286452\\
0.064103\\
0.096758
\end{pmatrix}\end{equation*}
\begin{equation*}
\left[ \frac{{w}^{cos}_i}{{w}^{cos}_j} \right] =
\begin{pmatrix}
$\,\,$ 1 $\,\,$ & $\,\,$\color{red} 1.9294\color{black} $\,\,$ & $\,\,$\color{red} 8.6219\color{black} $\,\,$ & $\,\,$\color{red} 5.7120\color{black} $\,\,$ \\
$\,\,$\color{red} 0.5183\color{black} $\,\,$ & $\,\,$ 1 $\,\,$ & $\,\,$4.4686$\,\,$ & $\,\,$2.9605  $\,\,$ \\
$\,\,$\color{red} 0.1160\color{black} $\,\,$ & $\,\,$0.2238$\,\,$ & $\,\,$ 1 $\,\,$ & $\,\,$0.6625 $\,\,$ \\
$\,\,$\color{red} 0.1751\color{black} $\,\,$ & $\,\,$0.3378$\,\,$ & $\,\,$1.5094$\,\,$ & $\,\,$ 1  $\,\,$ \\
\end{pmatrix},
\end{equation*}

\begin{equation*}
\mathbf{w}^{\prime} =
\begin{pmatrix}
0.561551\\
0.280775\\
0.062833\\
0.094841
\end{pmatrix} =
0.980184\cdot
\begin{pmatrix}
\color{gr} 0.572903\color{black} \\
0.286452\\
0.064103\\
0.096758
\end{pmatrix},
\end{equation*}
\begin{equation*}
\left[ \frac{{w}^{\prime}_i}{{w}^{\prime}_j} \right] =
\begin{pmatrix}
$\,\,$ 1 $\,\,$ & $\,\,$\color{gr} \color{blue} 2\color{black} $\,\,$ & $\,\,$\color{gr} 8.9373\color{black} $\,\,$ & $\,\,$\color{gr} 5.9210\color{black} $\,\,$ \\
$\,\,$\color{gr} \color{blue}  1/2\color{black} $\,\,$ & $\,\,$ 1 $\,\,$ & $\,\,$4.4686$\,\,$ & $\,\,$2.9605  $\,\,$ \\
$\,\,$\color{gr} 0.1119\color{black} $\,\,$ & $\,\,$0.2238$\,\,$ & $\,\,$ 1 $\,\,$ & $\,\,$0.6625 $\,\,$ \\
$\,\,$\color{gr} 0.1689\color{black} $\,\,$ & $\,\,$0.3378$\,\,$ & $\,\,$1.5094$\,\,$ & $\,\,$ 1  $\,\,$ \\
\end{pmatrix},
\end{equation*}
\end{example}
\newpage
\begin{example}
\begin{equation*}
\mathbf{A} =
\begin{pmatrix}
$\,\,$ 1 $\,\,$ & $\,\,$2$\,\,$ & $\,\,$9$\,\,$ & $\,\,$6 $\,\,$ \\
$\,\,$ 1/2$\,\,$ & $\,\,$ 1 $\,\,$ & $\,\,$8$\,\,$ & $\,\,$2 $\,\,$ \\
$\,\,$ 1/9$\,\,$ & $\,\,$ 1/8$\,\,$ & $\,\,$ 1 $\,\,$ & $\,\,$1 $\,\,$ \\
$\,\,$ 1/6$\,\,$ & $\,\,$ 1/2$\,\,$ & $\,\,$ 1 $\,\,$ & $\,\,$ 1  $\,\,$ \\
\end{pmatrix},
\qquad
\lambda_{\max} =
4.1664,
\qquad
CR = 0.0627
\end{equation*}

\begin{equation*}
\mathbf{w}^{cos} =
\begin{pmatrix}
\color{red} 0.546778\color{black} \\
0.294871\\
0.062203\\
0.096148
\end{pmatrix}\end{equation*}
\begin{equation*}
\left[ \frac{{w}^{cos}_i}{{w}^{cos}_j} \right] =
\begin{pmatrix}
$\,\,$ 1 $\,\,$ & $\,\,$\color{red} 1.8543\color{black} $\,\,$ & $\,\,$\color{red} 8.7903\color{black} $\,\,$ & $\,\,$\color{red} 5.6868\color{black} $\,\,$ \\
$\,\,$\color{red} 0.5393\color{black} $\,\,$ & $\,\,$ 1 $\,\,$ & $\,\,$4.7405$\,\,$ & $\,\,$3.0668  $\,\,$ \\
$\,\,$\color{red} 0.1138\color{black} $\,\,$ & $\,\,$0.2109$\,\,$ & $\,\,$ 1 $\,\,$ & $\,\,$0.6469 $\,\,$ \\
$\,\,$\color{red} 0.1758\color{black} $\,\,$ & $\,\,$0.3261$\,\,$ & $\,\,$1.5457$\,\,$ & $\,\,$ 1  $\,\,$ \\
\end{pmatrix},
\end{equation*}

\begin{equation*}
\mathbf{w}^{\prime} =
\begin{pmatrix}
0.552614\\
0.291074\\
0.061402\\
0.094910
\end{pmatrix} =
0.987123\cdot
\begin{pmatrix}
\color{gr} 0.559823\color{black} \\
0.294871\\
0.062203\\
0.096148
\end{pmatrix},
\end{equation*}
\begin{equation*}
\left[ \frac{{w}^{\prime}_i}{{w}^{\prime}_j} \right] =
\begin{pmatrix}
$\,\,$ 1 $\,\,$ & $\,\,$\color{gr} 1.8985\color{black} $\,\,$ & $\,\,$\color{gr} \color{blue} 9\color{black} $\,\,$ & $\,\,$\color{gr} 5.8225\color{black} $\,\,$ \\
$\,\,$\color{gr} 0.5267\color{black} $\,\,$ & $\,\,$ 1 $\,\,$ & $\,\,$4.7405$\,\,$ & $\,\,$3.0668  $\,\,$ \\
$\,\,$\color{gr} \color{blue}  1/9\color{black} $\,\,$ & $\,\,$0.2109$\,\,$ & $\,\,$ 1 $\,\,$ & $\,\,$0.6469 $\,\,$ \\
$\,\,$\color{gr} 0.1717\color{black} $\,\,$ & $\,\,$0.3261$\,\,$ & $\,\,$1.5457$\,\,$ & $\,\,$ 1  $\,\,$ \\
\end{pmatrix},
\end{equation*}
\end{example}
\newpage
\begin{example}
\begin{equation*}
\mathbf{A} =
\begin{pmatrix}
$\,\,$ 1 $\,\,$ & $\,\,$2$\,\,$ & $\,\,$9$\,\,$ & $\,\,$6 $\,\,$ \\
$\,\,$ 1/2$\,\,$ & $\,\,$ 1 $\,\,$ & $\,\,$9$\,\,$ & $\,\,$2 $\,\,$ \\
$\,\,$ 1/9$\,\,$ & $\,\,$ 1/9$\,\,$ & $\,\,$ 1 $\,\,$ & $\,\,$1 $\,\,$ \\
$\,\,$ 1/6$\,\,$ & $\,\,$ 1/2$\,\,$ & $\,\,$ 1 $\,\,$ & $\,\,$ 1  $\,\,$ \\
\end{pmatrix},
\qquad
\lambda_{\max} =
4.1990,
\qquad
CR = 0.0750
\end{equation*}

\begin{equation*}
\mathbf{w}^{cos} =
\begin{pmatrix}
\color{red} 0.541447\color{black} \\
0.302287\\
0.060641\\
0.095624
\end{pmatrix}\end{equation*}
\begin{equation*}
\left[ \frac{{w}^{cos}_i}{{w}^{cos}_j} \right] =
\begin{pmatrix}
$\,\,$ 1 $\,\,$ & $\,\,$\color{red} 1.7912\color{black} $\,\,$ & $\,\,$\color{red} 8.9287\color{black} $\,\,$ & $\,\,$\color{red} 5.6623\color{black} $\,\,$ \\
$\,\,$\color{red} 0.5583\color{black} $\,\,$ & $\,\,$ 1 $\,\,$ & $\,\,$4.9849$\,\,$ & $\,\,$3.1612  $\,\,$ \\
$\,\,$\color{red} 0.1120\color{black} $\,\,$ & $\,\,$0.2006$\,\,$ & $\,\,$ 1 $\,\,$ & $\,\,$0.6342 $\,\,$ \\
$\,\,$\color{red} 0.1766\color{black} $\,\,$ & $\,\,$0.3163$\,\,$ & $\,\,$1.5769$\,\,$ & $\,\,$ 1  $\,\,$ \\
\end{pmatrix},
\end{equation*}

\begin{equation*}
\mathbf{w}^{\prime} =
\begin{pmatrix}
0.543421\\
0.300987\\
0.060380\\
0.095212
\end{pmatrix} =
0.995697\cdot
\begin{pmatrix}
\color{gr} 0.545769\color{black} \\
0.302287\\
0.060641\\
0.095624
\end{pmatrix},
\end{equation*}
\begin{equation*}
\left[ \frac{{w}^{\prime}_i}{{w}^{\prime}_j} \right] =
\begin{pmatrix}
$\,\,$ 1 $\,\,$ & $\,\,$\color{gr} 1.8055\color{black} $\,\,$ & $\,\,$\color{gr} \color{blue} 9\color{black} $\,\,$ & $\,\,$\color{gr} 5.7075\color{black} $\,\,$ \\
$\,\,$\color{gr} 0.5539\color{black} $\,\,$ & $\,\,$ 1 $\,\,$ & $\,\,$4.9849$\,\,$ & $\,\,$3.1612  $\,\,$ \\
$\,\,$\color{gr} \color{blue}  1/9\color{black} $\,\,$ & $\,\,$0.2006$\,\,$ & $\,\,$ 1 $\,\,$ & $\,\,$0.6342 $\,\,$ \\
$\,\,$\color{gr} 0.1752\color{black} $\,\,$ & $\,\,$0.3163$\,\,$ & $\,\,$1.5769$\,\,$ & $\,\,$ 1  $\,\,$ \\
\end{pmatrix},
\end{equation*}
\end{example}
\newpage
\begin{example}
\begin{equation*}
\mathbf{A} =
\begin{pmatrix}
$\,\,$ 1 $\,\,$ & $\,\,$2$\,\,$ & $\,\,$9$\,\,$ & $\,\,$7 $\,\,$ \\
$\,\,$ 1/2$\,\,$ & $\,\,$ 1 $\,\,$ & $\,\,$2$\,\,$ & $\,\,$5 $\,\,$ \\
$\,\,$ 1/9$\,\,$ & $\,\,$ 1/2$\,\,$ & $\,\,$ 1 $\,\,$ & $\,\,$4 $\,\,$ \\
$\,\,$ 1/7$\,\,$ & $\,\,$ 1/5$\,\,$ & $\,\,$ 1/4$\,\,$ & $\,\,$ 1  $\,\,$ \\
\end{pmatrix},
\qquad
\lambda_{\max} =
4.2405,
\qquad
CR = 0.0907
\end{equation*}

\begin{equation*}
\mathbf{w}^{cos} =
\begin{pmatrix}
0.552157\\
\color{red} 0.258148\color{black} \\
0.134603\\
0.055092
\end{pmatrix}\end{equation*}
\begin{equation*}
\left[ \frac{{w}^{cos}_i}{{w}^{cos}_j} \right] =
\begin{pmatrix}
$\,\,$ 1 $\,\,$ & $\,\,$\color{red} 2.1389\color{black} $\,\,$ & $\,\,$4.1021$\,\,$ & $\,\,$10.0225$\,\,$ \\
$\,\,$\color{red} 0.4675\color{black} $\,\,$ & $\,\,$ 1 $\,\,$ & $\,\,$\color{red} 1.9178\color{black} $\,\,$ & $\,\,$\color{red} 4.6858\color{black}   $\,\,$ \\
$\,\,$0.2438$\,\,$ & $\,\,$\color{red} 0.5214\color{black} $\,\,$ & $\,\,$ 1 $\,\,$ & $\,\,$2.4433 $\,\,$ \\
$\,\,$0.0998$\,\,$ & $\,\,$\color{red} 0.2134\color{black} $\,\,$ & $\,\,$0.4093$\,\,$ & $\,\,$ 1  $\,\,$ \\
\end{pmatrix},
\end{equation*}

\begin{equation*}
\mathbf{w}^{\prime} =
\begin{pmatrix}
0.546118\\
0.266262\\
0.133131\\
0.054489
\end{pmatrix} =
0.989062\cdot
\begin{pmatrix}
0.552157\\
\color{gr} 0.269207\color{black} \\
0.134603\\
0.055092
\end{pmatrix},
\end{equation*}
\begin{equation*}
\left[ \frac{{w}^{\prime}_i}{{w}^{\prime}_j} \right] =
\begin{pmatrix}
$\,\,$ 1 $\,\,$ & $\,\,$\color{gr} 2.0511\color{black} $\,\,$ & $\,\,$4.1021$\,\,$ & $\,\,$10.0225$\,\,$ \\
$\,\,$\color{gr} 0.4876\color{black} $\,\,$ & $\,\,$ 1 $\,\,$ & $\,\,$\color{gr} \color{blue} 2\color{black} $\,\,$ & $\,\,$\color{gr} 4.8865\color{black}   $\,\,$ \\
$\,\,$0.2438$\,\,$ & $\,\,$\color{gr} \color{blue}  1/2\color{black} $\,\,$ & $\,\,$ 1 $\,\,$ & $\,\,$2.4433 $\,\,$ \\
$\,\,$0.0998$\,\,$ & $\,\,$\color{gr} 0.2046\color{black} $\,\,$ & $\,\,$0.4093$\,\,$ & $\,\,$ 1  $\,\,$ \\
\end{pmatrix},
\end{equation*}
\end{example}
\newpage
\begin{example}
\begin{equation*}
\mathbf{A} =
\begin{pmatrix}
$\,\,$ 1 $\,\,$ & $\,\,$2$\,\,$ & $\,\,$9$\,\,$ & $\,\,$8 $\,\,$ \\
$\,\,$ 1/2$\,\,$ & $\,\,$ 1 $\,\,$ & $\,\,$1$\,\,$ & $\,\,$3 $\,\,$ \\
$\,\,$ 1/9$\,\,$ & $\,\,$ 1 $\,\,$ & $\,\,$ 1 $\,\,$ & $\,\,$2 $\,\,$ \\
$\,\,$ 1/8$\,\,$ & $\,\,$ 1/3$\,\,$ & $\,\,$ 1/2$\,\,$ & $\,\,$ 1  $\,\,$ \\
\end{pmatrix},
\qquad
\lambda_{\max} =
4.2052,
\qquad
CR = 0.0774
\end{equation*}

\begin{equation*}
\mathbf{w}^{cos} =
\begin{pmatrix}
0.584562\\
0.211289\\
0.136751\\
\color{red} 0.067399\color{black}
\end{pmatrix}\end{equation*}
\begin{equation*}
\left[ \frac{{w}^{cos}_i}{{w}^{cos}_j} \right] =
\begin{pmatrix}
$\,\,$ 1 $\,\,$ & $\,\,$2.7666$\,\,$ & $\,\,$4.2746$\,\,$ & $\,\,$\color{red} 8.6732\color{black} $\,\,$ \\
$\,\,$0.3614$\,\,$ & $\,\,$ 1 $\,\,$ & $\,\,$1.5451$\,\,$ & $\,\,$\color{red} 3.1349\color{black}   $\,\,$ \\
$\,\,$0.2339$\,\,$ & $\,\,$0.6472$\,\,$ & $\,\,$ 1 $\,\,$ & $\,\,$\color{red} 2.0290\color{black}  $\,\,$ \\
$\,\,$\color{red} 0.1153\color{black} $\,\,$ & $\,\,$\color{red} 0.3190\color{black} $\,\,$ & $\,\,$\color{red} 0.4929\color{black} $\,\,$ & $\,\,$ 1  $\,\,$ \\
\end{pmatrix},
\end{equation*}

\begin{equation*}
\mathbf{w}^{\prime} =
\begin{pmatrix}
0.583991\\
0.211083\\
0.136617\\
0.068309
\end{pmatrix} =
0.999024\cdot
\begin{pmatrix}
0.584562\\
0.211289\\
0.136751\\
\color{gr} 0.068375\color{black}
\end{pmatrix},
\end{equation*}
\begin{equation*}
\left[ \frac{{w}^{\prime}_i}{{w}^{\prime}_j} \right] =
\begin{pmatrix}
$\,\,$ 1 $\,\,$ & $\,\,$2.7666$\,\,$ & $\,\,$4.2746$\,\,$ & $\,\,$\color{gr} 8.5493\color{black} $\,\,$ \\
$\,\,$0.3614$\,\,$ & $\,\,$ 1 $\,\,$ & $\,\,$1.5451$\,\,$ & $\,\,$\color{gr} 3.0901\color{black}   $\,\,$ \\
$\,\,$0.2339$\,\,$ & $\,\,$0.6472$\,\,$ & $\,\,$ 1 $\,\,$ & $\,\,$\color{gr} \color{blue} 2\color{black}  $\,\,$ \\
$\,\,$\color{gr} 0.1170\color{black} $\,\,$ & $\,\,$\color{gr} 0.3236\color{black} $\,\,$ & $\,\,$\color{gr} \color{blue}  1/2\color{black} $\,\,$ & $\,\,$ 1  $\,\,$ \\
\end{pmatrix},
\end{equation*}
\end{example}
\newpage
\begin{example}
\begin{equation*}
\mathbf{A} =
\begin{pmatrix}
$\,\,$ 1 $\,\,$ & $\,\,$2$\,\,$ & $\,\,$9$\,\,$ & $\,\,$8 $\,\,$ \\
$\,\,$ 1/2$\,\,$ & $\,\,$ 1 $\,\,$ & $\,\,$2$\,\,$ & $\,\,$5 $\,\,$ \\
$\,\,$ 1/9$\,\,$ & $\,\,$ 1/2$\,\,$ & $\,\,$ 1 $\,\,$ & $\,\,$4 $\,\,$ \\
$\,\,$ 1/8$\,\,$ & $\,\,$ 1/5$\,\,$ & $\,\,$ 1/4$\,\,$ & $\,\,$ 1  $\,\,$ \\
\end{pmatrix},
\qquad
\lambda_{\max} =
4.2067,
\qquad
CR = 0.0779
\end{equation*}

\begin{equation*}
\mathbf{w}^{cos} =
\begin{pmatrix}
0.563528\\
\color{red} 0.254024\color{black} \\
0.130687\\
0.051762
\end{pmatrix}\end{equation*}
\begin{equation*}
\left[ \frac{{w}^{cos}_i}{{w}^{cos}_j} \right] =
\begin{pmatrix}
$\,\,$ 1 $\,\,$ & $\,\,$\color{red} 2.2184\color{black} $\,\,$ & $\,\,$4.3120$\,\,$ & $\,\,$10.8870$\,\,$ \\
$\,\,$\color{red} 0.4508\color{black} $\,\,$ & $\,\,$ 1 $\,\,$ & $\,\,$\color{red} 1.9438\color{black} $\,\,$ & $\,\,$\color{red} 4.9076\color{black}   $\,\,$ \\
$\,\,$0.2319$\,\,$ & $\,\,$\color{red} 0.5145\color{black} $\,\,$ & $\,\,$ 1 $\,\,$ & $\,\,$2.5248 $\,\,$ \\
$\,\,$0.0919$\,\,$ & $\,\,$\color{red} 0.2038\color{black} $\,\,$ & $\,\,$0.3961$\,\,$ & $\,\,$ 1  $\,\,$ \\
\end{pmatrix},
\end{equation*}

\begin{equation*}
\mathbf{w}^{\prime} =
\begin{pmatrix}
0.560844\\
0.257576\\
0.130065\\
0.051515
\end{pmatrix} =
0.995238\cdot
\begin{pmatrix}
0.563528\\
\color{gr} 0.258808\color{black} \\
0.130687\\
0.051762
\end{pmatrix},
\end{equation*}
\begin{equation*}
\left[ \frac{{w}^{\prime}_i}{{w}^{\prime}_j} \right] =
\begin{pmatrix}
$\,\,$ 1 $\,\,$ & $\,\,$\color{gr} 2.1774\color{black} $\,\,$ & $\,\,$4.3120$\,\,$ & $\,\,$10.8870$\,\,$ \\
$\,\,$\color{gr} 0.4593\color{black} $\,\,$ & $\,\,$ 1 $\,\,$ & $\,\,$\color{gr} 1.9804\color{black} $\,\,$ & $\,\,$\color{gr} \color{blue} 5\color{black}   $\,\,$ \\
$\,\,$0.2319$\,\,$ & $\,\,$\color{gr} 0.5050\color{black} $\,\,$ & $\,\,$ 1 $\,\,$ & $\,\,$2.5248 $\,\,$ \\
$\,\,$0.0919$\,\,$ & $\,\,$\color{gr} \color{blue}  1/5\color{black} $\,\,$ & $\,\,$0.3961$\,\,$ & $\,\,$ 1  $\,\,$ \\
\end{pmatrix},
\end{equation*}
\end{example}
\newpage
\begin{example}
\begin{equation*}
\mathbf{A} =
\begin{pmatrix}
$\,\,$ 1 $\,\,$ & $\,\,$2$\,\,$ & $\,\,$9$\,\,$ & $\,\,$9 $\,\,$ \\
$\,\,$ 1/2$\,\,$ & $\,\,$ 1 $\,\,$ & $\,\,$1$\,\,$ & $\,\,$3 $\,\,$ \\
$\,\,$ 1/9$\,\,$ & $\,\,$ 1 $\,\,$ & $\,\,$ 1 $\,\,$ & $\,\,$2 $\,\,$ \\
$\,\,$ 1/9$\,\,$ & $\,\,$ 1/3$\,\,$ & $\,\,$ 1/2$\,\,$ & $\,\,$ 1  $\,\,$ \\
\end{pmatrix},
\qquad
\lambda_{\max} =
4.1990,
\qquad
CR = 0.0750
\end{equation*}

\begin{equation*}
\mathbf{w}^{cos} =
\begin{pmatrix}
0.593077\\
0.208101\\
0.134546\\
\color{red} 0.064276\color{black}
\end{pmatrix}\end{equation*}
\begin{equation*}
\left[ \frac{{w}^{cos}_i}{{w}^{cos}_j} \right] =
\begin{pmatrix}
$\,\,$ 1 $\,\,$ & $\,\,$2.8499$\,\,$ & $\,\,$4.4080$\,\,$ & $\,\,$\color{red} 9.2270\color{black} $\,\,$ \\
$\,\,$0.3509$\,\,$ & $\,\,$ 1 $\,\,$ & $\,\,$1.5467$\,\,$ & $\,\,$\color{red} 3.2376\color{black}   $\,\,$ \\
$\,\,$0.2269$\,\,$ & $\,\,$0.6465$\,\,$ & $\,\,$ 1 $\,\,$ & $\,\,$\color{red} 2.0932\color{black}  $\,\,$ \\
$\,\,$\color{red} 0.1084\color{black} $\,\,$ & $\,\,$\color{red} 0.3089\color{black} $\,\,$ & $\,\,$\color{red} 0.4777\color{black} $\,\,$ & $\,\,$ 1  $\,\,$ \\
\end{pmatrix},
\end{equation*}

\begin{equation*}
\mathbf{w}^{\prime} =
\begin{pmatrix}
0.592117\\
0.207764\\
0.134328\\
0.065791
\end{pmatrix} =
0.998381\cdot
\begin{pmatrix}
0.593077\\
0.208101\\
0.134546\\
\color{gr} 0.065897\color{black}
\end{pmatrix},
\end{equation*}
\begin{equation*}
\left[ \frac{{w}^{\prime}_i}{{w}^{\prime}_j} \right] =
\begin{pmatrix}
$\,\,$ 1 $\,\,$ & $\,\,$2.8499$\,\,$ & $\,\,$4.4080$\,\,$ & $\,\,$\color{gr} \color{blue} 9\color{black} $\,\,$ \\
$\,\,$0.3509$\,\,$ & $\,\,$ 1 $\,\,$ & $\,\,$1.5467$\,\,$ & $\,\,$\color{gr} 3.1580\color{black}   $\,\,$ \\
$\,\,$0.2269$\,\,$ & $\,\,$0.6465$\,\,$ & $\,\,$ 1 $\,\,$ & $\,\,$\color{gr} 2.0417\color{black}  $\,\,$ \\
$\,\,$\color{gr} \color{blue}  1/9\color{black} $\,\,$ & $\,\,$\color{gr} 0.3167\color{black} $\,\,$ & $\,\,$\color{gr} 0.4898\color{black} $\,\,$ & $\,\,$ 1  $\,\,$ \\
\end{pmatrix},
\end{equation*}
\end{example}
\newpage
\begin{example}
\begin{equation*}
\mathbf{A} =
\begin{pmatrix}
$\,\,$ 1 $\,\,$ & $\,\,$2$\,\,$ & $\,\,$9$\,\,$ & $\,\,$9 $\,\,$ \\
$\,\,$ 1/2$\,\,$ & $\,\,$ 1 $\,\,$ & $\,\,$2$\,\,$ & $\,\,$6 $\,\,$ \\
$\,\,$ 1/9$\,\,$ & $\,\,$ 1/2$\,\,$ & $\,\,$ 1 $\,\,$ & $\,\,$5 $\,\,$ \\
$\,\,$ 1/9$\,\,$ & $\,\,$ 1/6$\,\,$ & $\,\,$ 1/5$\,\,$ & $\,\,$ 1  $\,\,$ \\
\end{pmatrix},
\qquad
\lambda_{\max} =
4.2339,
\qquad
CR = 0.0882
\end{equation*}

\begin{equation*}
\mathbf{w}^{cos} =
\begin{pmatrix}
0.561809\\
\color{red} 0.257638\color{black} \\
0.135962\\
0.044591
\end{pmatrix}\end{equation*}
\begin{equation*}
\left[ \frac{{w}^{cos}_i}{{w}^{cos}_j} \right] =
\begin{pmatrix}
$\,\,$ 1 $\,\,$ & $\,\,$\color{red} 2.1806\color{black} $\,\,$ & $\,\,$4.1321$\,\,$ & $\,\,$12.5990$\,\,$ \\
$\,\,$\color{red} 0.4586\color{black} $\,\,$ & $\,\,$ 1 $\,\,$ & $\,\,$\color{red} 1.8949\color{black} $\,\,$ & $\,\,$\color{red} 5.7777\color{black}   $\,\,$ \\
$\,\,$0.2420$\,\,$ & $\,\,$\color{red} 0.5277\color{black} $\,\,$ & $\,\,$ 1 $\,\,$ & $\,\,$3.0490 $\,\,$ \\
$\,\,$0.0794$\,\,$ & $\,\,$\color{red} 0.1731\color{black} $\,\,$ & $\,\,$0.3280$\,\,$ & $\,\,$ 1  $\,\,$ \\
\end{pmatrix},
\end{equation*}

\begin{equation*}
\mathbf{w}^{\prime} =
\begin{pmatrix}
0.556296\\
0.264923\\
0.134627\\
0.044154
\end{pmatrix} =
0.990187\cdot
\begin{pmatrix}
0.561809\\
\color{gr} 0.267549\color{black} \\
0.135962\\
0.044591
\end{pmatrix},
\end{equation*}
\begin{equation*}
\left[ \frac{{w}^{\prime}_i}{{w}^{\prime}_j} \right] =
\begin{pmatrix}
$\,\,$ 1 $\,\,$ & $\,\,$\color{gr} 2.0998\color{black} $\,\,$ & $\,\,$4.1321$\,\,$ & $\,\,$12.5990$\,\,$ \\
$\,\,$\color{gr} 0.4762\color{black} $\,\,$ & $\,\,$ 1 $\,\,$ & $\,\,$\color{gr} 1.9678\color{black} $\,\,$ & $\,\,$\color{gr} \color{blue} 6\color{black}   $\,\,$ \\
$\,\,$0.2420$\,\,$ & $\,\,$\color{gr} 0.5082\color{black} $\,\,$ & $\,\,$ 1 $\,\,$ & $\,\,$3.0490 $\,\,$ \\
$\,\,$0.0794$\,\,$ & $\,\,$\color{gr} \color{blue}  1/6\color{black} $\,\,$ & $\,\,$0.3280$\,\,$ & $\,\,$ 1  $\,\,$ \\
\end{pmatrix},
\end{equation*}
\end{example}
\newpage
\begin{example}
\begin{equation*}
\mathbf{A} =
\begin{pmatrix}
$\,\,$ 1 $\,\,$ & $\,\,$2$\,\,$ & $\,\,$9$\,\,$ & $\,\,$9 $\,\,$ \\
$\,\,$ 1/2$\,\,$ & $\,\,$ 1 $\,\,$ & $\,\,$7$\,\,$ & $\,\,$3 $\,\,$ \\
$\,\,$ 1/9$\,\,$ & $\,\,$ 1/7$\,\,$ & $\,\,$ 1 $\,\,$ & $\,\,$2 $\,\,$ \\
$\,\,$ 1/9$\,\,$ & $\,\,$ 1/3$\,\,$ & $\,\,$ 1/2$\,\,$ & $\,\,$ 1  $\,\,$ \\
\end{pmatrix},
\qquad
\lambda_{\max} =
4.2086,
\qquad
CR = 0.0786
\end{equation*}

\begin{equation*}
\mathbf{w}^{cos} =
\begin{pmatrix}
\color{red} 0.567633\color{black} \\
0.294316\\
0.074098\\
0.063953
\end{pmatrix}\end{equation*}
\begin{equation*}
\left[ \frac{{w}^{cos}_i}{{w}^{cos}_j} \right] =
\begin{pmatrix}
$\,\,$ 1 $\,\,$ & $\,\,$\color{red} 1.9287\color{black} $\,\,$ & $\,\,$\color{red} 7.6605\color{black} $\,\,$ & $\,\,$\color{red} 8.8758\color{black} $\,\,$ \\
$\,\,$\color{red} 0.5185\color{black} $\,\,$ & $\,\,$ 1 $\,\,$ & $\,\,$3.9720$\,\,$ & $\,\,$4.6021  $\,\,$ \\
$\,\,$\color{red} 0.1305\color{black} $\,\,$ & $\,\,$0.2518$\,\,$ & $\,\,$ 1 $\,\,$ & $\,\,$1.1586 $\,\,$ \\
$\,\,$\color{red} 0.1127\color{black} $\,\,$ & $\,\,$0.2173$\,\,$ & $\,\,$0.8631$\,\,$ & $\,\,$ 1  $\,\,$ \\
\end{pmatrix},
\end{equation*}

\begin{equation*}
\mathbf{w}^{\prime} =
\begin{pmatrix}
0.571040\\
0.291997\\
0.073515\\
0.063449
\end{pmatrix} =
0.992121\cdot
\begin{pmatrix}
\color{gr} 0.575575\color{black} \\
0.294316\\
0.074098\\
0.063953
\end{pmatrix},
\end{equation*}
\begin{equation*}
\left[ \frac{{w}^{\prime}_i}{{w}^{\prime}_j} \right] =
\begin{pmatrix}
$\,\,$ 1 $\,\,$ & $\,\,$\color{gr} 1.9556\color{black} $\,\,$ & $\,\,$\color{gr} 7.7677\color{black} $\,\,$ & $\,\,$\color{gr} \color{blue} 9\color{black} $\,\,$ \\
$\,\,$\color{gr} 0.5113\color{black} $\,\,$ & $\,\,$ 1 $\,\,$ & $\,\,$3.9720$\,\,$ & $\,\,$4.6021  $\,\,$ \\
$\,\,$\color{gr} 0.1287\color{black} $\,\,$ & $\,\,$0.2518$\,\,$ & $\,\,$ 1 $\,\,$ & $\,\,$1.1586 $\,\,$ \\
$\,\,$\color{gr} \color{blue}  1/9\color{black} $\,\,$ & $\,\,$0.2173$\,\,$ & $\,\,$0.8631$\,\,$ & $\,\,$ 1  $\,\,$ \\
\end{pmatrix},
\end{equation*}
\end{example}
\newpage
\begin{example}
\begin{equation*}
\mathbf{A} =
\begin{pmatrix}
$\,\,$ 1 $\,\,$ & $\,\,$2$\,\,$ & $\,\,$9$\,\,$ & $\,\,$9 $\,\,$ \\
$\,\,$ 1/2$\,\,$ & $\,\,$ 1 $\,\,$ & $\,\,$8$\,\,$ & $\,\,$3 $\,\,$ \\
$\,\,$ 1/9$\,\,$ & $\,\,$ 1/8$\,\,$ & $\,\,$ 1 $\,\,$ & $\,\,$2 $\,\,$ \\
$\,\,$ 1/9$\,\,$ & $\,\,$ 1/3$\,\,$ & $\,\,$ 1/2$\,\,$ & $\,\,$ 1  $\,\,$ \\
\end{pmatrix},
\qquad
\lambda_{\max} =
4.2469,
\qquad
CR = 0.0931
\end{equation*}

\begin{equation*}
\mathbf{w}^{cos} =
\begin{pmatrix}
\color{red} 0.561399\color{black} \\
0.302837\\
0.072124\\
0.063640
\end{pmatrix}\end{equation*}
\begin{equation*}
\left[ \frac{{w}^{cos}_i}{{w}^{cos}_j} \right] =
\begin{pmatrix}
$\,\,$ 1 $\,\,$ & $\,\,$\color{red} 1.8538\color{black} $\,\,$ & $\,\,$\color{red} 7.7839\color{black} $\,\,$ & $\,\,$\color{red} 8.8215\color{black} $\,\,$ \\
$\,\,$\color{red} 0.5394\color{black} $\,\,$ & $\,\,$ 1 $\,\,$ & $\,\,$4.1989$\,\,$ & $\,\,$4.7586  $\,\,$ \\
$\,\,$\color{red} 0.1285\color{black} $\,\,$ & $\,\,$0.2382$\,\,$ & $\,\,$ 1 $\,\,$ & $\,\,$1.1333 $\,\,$ \\
$\,\,$\color{red} 0.1134\color{black} $\,\,$ & $\,\,$0.2101$\,\,$ & $\,\,$0.8824$\,\,$ & $\,\,$ 1  $\,\,$ \\
\end{pmatrix},
\end{equation*}

\begin{equation*}
\mathbf{w}^{\prime} =
\begin{pmatrix}
0.566326\\
0.299436\\
0.071313\\
0.062925
\end{pmatrix} =
0.988767\cdot
\begin{pmatrix}
\color{gr} 0.572760\color{black} \\
0.302837\\
0.072124\\
0.063640
\end{pmatrix},
\end{equation*}
\begin{equation*}
\left[ \frac{{w}^{\prime}_i}{{w}^{\prime}_j} \right] =
\begin{pmatrix}
$\,\,$ 1 $\,\,$ & $\,\,$\color{gr} 1.8913\color{black} $\,\,$ & $\,\,$\color{gr} 7.9414\color{black} $\,\,$ & $\,\,$\color{gr} \color{blue} 9\color{black} $\,\,$ \\
$\,\,$\color{gr} 0.5287\color{black} $\,\,$ & $\,\,$ 1 $\,\,$ & $\,\,$4.1989$\,\,$ & $\,\,$4.7586  $\,\,$ \\
$\,\,$\color{gr} 0.1259\color{black} $\,\,$ & $\,\,$0.2382$\,\,$ & $\,\,$ 1 $\,\,$ & $\,\,$1.1333 $\,\,$ \\
$\,\,$\color{gr} \color{blue}  1/9\color{black} $\,\,$ & $\,\,$0.2101$\,\,$ & $\,\,$0.8824$\,\,$ & $\,\,$ 1  $\,\,$ \\
\end{pmatrix},
\end{equation*}
\end{example}
\newpage
\begin{example}
\begin{equation*}
\mathbf{A} =
\begin{pmatrix}
$\,\,$ 1 $\,\,$ & $\,\,$3$\,\,$ & $\,\,$2$\,\,$ & $\,\,$3 $\,\,$ \\
$\,\,$ 1/3$\,\,$ & $\,\,$ 1 $\,\,$ & $\,\,$1$\,\,$ & $\,\,$3 $\,\,$ \\
$\,\,$ 1/2$\,\,$ & $\,\,$ 1 $\,\,$ & $\,\,$ 1 $\,\,$ & $\,\,$2 $\,\,$ \\
$\,\,$ 1/3$\,\,$ & $\,\,$ 1/3$\,\,$ & $\,\,$ 1/2$\,\,$ & $\,\,$ 1  $\,\,$ \\
\end{pmatrix},
\qquad
\lambda_{\max} =
4.1031,
\qquad
CR = 0.0389
\end{equation*}

\begin{equation*}
\mathbf{w}^{cos} =
\begin{pmatrix}
0.445988\\
0.226592\\
\color{red} 0.216517\color{black} \\
0.110903
\end{pmatrix}\end{equation*}
\begin{equation*}
\left[ \frac{{w}^{cos}_i}{{w}^{cos}_j} \right] =
\begin{pmatrix}
$\,\,$ 1 $\,\,$ & $\,\,$1.9682$\,\,$ & $\,\,$\color{red} 2.0598\color{black} $\,\,$ & $\,\,$4.0214$\,\,$ \\
$\,\,$0.5081$\,\,$ & $\,\,$ 1 $\,\,$ & $\,\,$\color{red} 1.0465\color{black} $\,\,$ & $\,\,$2.0431  $\,\,$ \\
$\,\,$\color{red} 0.4855\color{black} $\,\,$ & $\,\,$\color{red} 0.9555\color{black} $\,\,$ & $\,\,$ 1 $\,\,$ & $\,\,$\color{red} 1.9523\color{black}  $\,\,$ \\
$\,\,$0.2487$\,\,$ & $\,\,$0.4894$\,\,$ & $\,\,$\color{red} 0.5122\color{black} $\,\,$ & $\,\,$ 1  $\,\,$ \\
\end{pmatrix},
\end{equation*}

\begin{equation*}
\mathbf{w}^{\prime} =
\begin{pmatrix}
0.443641\\
0.225400\\
0.220640\\
0.110320
\end{pmatrix} =
0.994738\cdot
\begin{pmatrix}
0.445988\\
0.226592\\
\color{gr} 0.221807\color{black} \\
0.110903
\end{pmatrix},
\end{equation*}
\begin{equation*}
\left[ \frac{{w}^{\prime}_i}{{w}^{\prime}_j} \right] =
\begin{pmatrix}
$\,\,$ 1 $\,\,$ & $\,\,$1.9682$\,\,$ & $\,\,$\color{gr} 2.0107\color{black} $\,\,$ & $\,\,$4.0214$\,\,$ \\
$\,\,$0.5081$\,\,$ & $\,\,$ 1 $\,\,$ & $\,\,$\color{gr} 1.0216\color{black} $\,\,$ & $\,\,$2.0431  $\,\,$ \\
$\,\,$\color{gr} 0.4973\color{black} $\,\,$ & $\,\,$\color{gr} 0.9789\color{black} $\,\,$ & $\,\,$ 1 $\,\,$ & $\,\,$\color{gr} \color{blue} 2\color{black}  $\,\,$ \\
$\,\,$0.2487$\,\,$ & $\,\,$0.4894$\,\,$ & $\,\,$\color{gr} \color{blue}  1/2\color{black} $\,\,$ & $\,\,$ 1  $\,\,$ \\
\end{pmatrix},
\end{equation*}
\end{example}
\newpage
\begin{example}
\begin{equation*}
\mathbf{A} =
\begin{pmatrix}
$\,\,$ 1 $\,\,$ & $\,\,$3$\,\,$ & $\,\,$2$\,\,$ & $\,\,$4 $\,\,$ \\
$\,\,$ 1/3$\,\,$ & $\,\,$ 1 $\,\,$ & $\,\,$1$\,\,$ & $\,\,$5 $\,\,$ \\
$\,\,$ 1/2$\,\,$ & $\,\,$ 1 $\,\,$ & $\,\,$ 1 $\,\,$ & $\,\,$3 $\,\,$ \\
$\,\,$ 1/4$\,\,$ & $\,\,$ 1/5$\,\,$ & $\,\,$ 1/3$\,\,$ & $\,\,$ 1  $\,\,$ \\
\end{pmatrix},
\qquad
\lambda_{\max} =
4.1502,
\qquad
CR = 0.0566
\end{equation*}

\begin{equation*}
\mathbf{w}^{cos} =
\begin{pmatrix}
0.451771\\
0.244757\\
\color{red} 0.224367\color{black} \\
0.079105
\end{pmatrix}\end{equation*}
\begin{equation*}
\left[ \frac{{w}^{cos}_i}{{w}^{cos}_j} \right] =
\begin{pmatrix}
$\,\,$ 1 $\,\,$ & $\,\,$1.8458$\,\,$ & $\,\,$\color{red} 2.0135\color{black} $\,\,$ & $\,\,$5.7110$\,\,$ \\
$\,\,$0.5418$\,\,$ & $\,\,$ 1 $\,\,$ & $\,\,$\color{red} 1.0909\color{black} $\,\,$ & $\,\,$3.0941  $\,\,$ \\
$\,\,$\color{red} 0.4966\color{black} $\,\,$ & $\,\,$\color{red} 0.9167\color{black} $\,\,$ & $\,\,$ 1 $\,\,$ & $\,\,$\color{red} 2.8363\color{black}  $\,\,$ \\
$\,\,$0.1751$\,\,$ & $\,\,$0.3232$\,\,$ & $\,\,$\color{red} 0.3526\color{black} $\,\,$ & $\,\,$ 1  $\,\,$ \\
\end{pmatrix},
\end{equation*}

\begin{equation*}
\mathbf{w}^{\prime} =
\begin{pmatrix}
0.451086\\
0.244386\\
0.225543\\
0.078985
\end{pmatrix} =
0.998484\cdot
\begin{pmatrix}
0.451771\\
0.244757\\
\color{gr} 0.225886\color{black} \\
0.079105
\end{pmatrix},
\end{equation*}
\begin{equation*}
\left[ \frac{{w}^{\prime}_i}{{w}^{\prime}_j} \right] =
\begin{pmatrix}
$\,\,$ 1 $\,\,$ & $\,\,$1.8458$\,\,$ & $\,\,$\color{gr} \color{blue} 2\color{black} $\,\,$ & $\,\,$5.7110$\,\,$ \\
$\,\,$0.5418$\,\,$ & $\,\,$ 1 $\,\,$ & $\,\,$\color{gr} 1.0835\color{black} $\,\,$ & $\,\,$3.0941  $\,\,$ \\
$\,\,$\color{gr} \color{blue}  1/2\color{black} $\,\,$ & $\,\,$\color{gr} 0.9229\color{black} $\,\,$ & $\,\,$ 1 $\,\,$ & $\,\,$\color{gr} 2.8555\color{black}  $\,\,$ \\
$\,\,$0.1751$\,\,$ & $\,\,$0.3232$\,\,$ & $\,\,$\color{gr} 0.3502\color{black} $\,\,$ & $\,\,$ 1  $\,\,$ \\
\end{pmatrix},
\end{equation*}
\end{example}
\newpage
\begin{example}
\begin{equation*}
\mathbf{A} =
\begin{pmatrix}
$\,\,$ 1 $\,\,$ & $\,\,$3$\,\,$ & $\,\,$2$\,\,$ & $\,\,$4 $\,\,$ \\
$\,\,$ 1/3$\,\,$ & $\,\,$ 1 $\,\,$ & $\,\,$1$\,\,$ & $\,\,$6 $\,\,$ \\
$\,\,$ 1/2$\,\,$ & $\,\,$ 1 $\,\,$ & $\,\,$ 1 $\,\,$ & $\,\,$3 $\,\,$ \\
$\,\,$ 1/4$\,\,$ & $\,\,$ 1/6$\,\,$ & $\,\,$ 1/3$\,\,$ & $\,\,$ 1  $\,\,$ \\
\end{pmatrix},
\qquad
\lambda_{\max} =
4.1990,
\qquad
CR = 0.0750
\end{equation*}

\begin{equation*}
\mathbf{w}^{cos} =
\begin{pmatrix}
0.447625\\
0.255680\\
\color{red} 0.220364\color{black} \\
0.076331
\end{pmatrix}\end{equation*}
\begin{equation*}
\left[ \frac{{w}^{cos}_i}{{w}^{cos}_j} \right] =
\begin{pmatrix}
$\,\,$ 1 $\,\,$ & $\,\,$1.7507$\,\,$ & $\,\,$\color{red} 2.0313\color{black} $\,\,$ & $\,\,$5.8643$\,\,$ \\
$\,\,$0.5712$\,\,$ & $\,\,$ 1 $\,\,$ & $\,\,$\color{red} 1.1603\color{black} $\,\,$ & $\,\,$3.3496  $\,\,$ \\
$\,\,$\color{red} 0.4923\color{black} $\,\,$ & $\,\,$\color{red} 0.8619\color{black} $\,\,$ & $\,\,$ 1 $\,\,$ & $\,\,$\color{red} 2.8869\color{black}  $\,\,$ \\
$\,\,$0.1705$\,\,$ & $\,\,$0.2985$\,\,$ & $\,\,$\color{red} 0.3464\color{black} $\,\,$ & $\,\,$ 1  $\,\,$ \\
\end{pmatrix},
\end{equation*}

\begin{equation*}
\mathbf{w}^{\prime} =
\begin{pmatrix}
0.446086\\
0.254802\\
0.223043\\
0.076069
\end{pmatrix} =
0.996563\cdot
\begin{pmatrix}
0.447625\\
0.255680\\
\color{gr} 0.223812\color{black} \\
0.076331
\end{pmatrix},
\end{equation*}
\begin{equation*}
\left[ \frac{{w}^{\prime}_i}{{w}^{\prime}_j} \right] =
\begin{pmatrix}
$\,\,$ 1 $\,\,$ & $\,\,$1.7507$\,\,$ & $\,\,$\color{gr} \color{blue} 2\color{black} $\,\,$ & $\,\,$5.8643$\,\,$ \\
$\,\,$0.5712$\,\,$ & $\,\,$ 1 $\,\,$ & $\,\,$\color{gr} 1.1424\color{black} $\,\,$ & $\,\,$3.3496  $\,\,$ \\
$\,\,$\color{gr} \color{blue}  1/2\color{black} $\,\,$ & $\,\,$\color{gr} 0.8754\color{black} $\,\,$ & $\,\,$ 1 $\,\,$ & $\,\,$\color{gr} 2.9321\color{black}  $\,\,$ \\
$\,\,$0.1705$\,\,$ & $\,\,$0.2985$\,\,$ & $\,\,$\color{gr} 0.3410\color{black} $\,\,$ & $\,\,$ 1  $\,\,$ \\
\end{pmatrix},
\end{equation*}
\end{example}
\newpage
\begin{example}
\begin{equation*}
\mathbf{A} =
\begin{pmatrix}
$\,\,$ 1 $\,\,$ & $\,\,$3$\,\,$ & $\,\,$2$\,\,$ & $\,\,$4 $\,\,$ \\
$\,\,$ 1/3$\,\,$ & $\,\,$ 1 $\,\,$ & $\,\,$1$\,\,$ & $\,\,$7 $\,\,$ \\
$\,\,$ 1/2$\,\,$ & $\,\,$ 1 $\,\,$ & $\,\,$ 1 $\,\,$ & $\,\,$3 $\,\,$ \\
$\,\,$ 1/4$\,\,$ & $\,\,$ 1/7$\,\,$ & $\,\,$ 1/3$\,\,$ & $\,\,$ 1  $\,\,$ \\
\end{pmatrix},
\qquad
\lambda_{\max} =
4.2478,
\qquad
CR = 0.0935
\end{equation*}

\begin{equation*}
\mathbf{w}^{cos} =
\begin{pmatrix}
0.444340\\
0.264506\\
\color{red} 0.216986\color{black} \\
0.074167
\end{pmatrix}\end{equation*}
\begin{equation*}
\left[ \frac{{w}^{cos}_i}{{w}^{cos}_j} \right] =
\begin{pmatrix}
$\,\,$ 1 $\,\,$ & $\,\,$1.6799$\,\,$ & $\,\,$\color{red} 2.0478\color{black} $\,\,$ & $\,\,$5.9910$\,\,$ \\
$\,\,$0.5953$\,\,$ & $\,\,$ 1 $\,\,$ & $\,\,$\color{red} 1.2190\color{black} $\,\,$ & $\,\,$3.5663  $\,\,$ \\
$\,\,$\color{red} 0.4883\color{black} $\,\,$ & $\,\,$\color{red} 0.8203\color{black} $\,\,$ & $\,\,$ 1 $\,\,$ & $\,\,$\color{red} 2.9256\color{black}  $\,\,$ \\
$\,\,$0.1669$\,\,$ & $\,\,$0.2804$\,\,$ & $\,\,$\color{red} 0.3418\color{black} $\,\,$ & $\,\,$ 1  $\,\,$ \\
\end{pmatrix},
\end{equation*}

\begin{equation*}
\mathbf{w}^{\prime} =
\begin{pmatrix}
0.442049\\
0.263142\\
0.221024\\
0.073785
\end{pmatrix} =
0.994843\cdot
\begin{pmatrix}
0.444340\\
0.264506\\
\color{gr} 0.222170\color{black} \\
0.074167
\end{pmatrix},
\end{equation*}
\begin{equation*}
\left[ \frac{{w}^{\prime}_i}{{w}^{\prime}_j} \right] =
\begin{pmatrix}
$\,\,$ 1 $\,\,$ & $\,\,$1.6799$\,\,$ & $\,\,$\color{gr} \color{blue} 2\color{black} $\,\,$ & $\,\,$5.9910$\,\,$ \\
$\,\,$0.5953$\,\,$ & $\,\,$ 1 $\,\,$ & $\,\,$\color{gr} 1.1906\color{black} $\,\,$ & $\,\,$3.5663  $\,\,$ \\
$\,\,$\color{gr} \color{blue}  1/2\color{black} $\,\,$ & $\,\,$\color{gr} 0.8399\color{black} $\,\,$ & $\,\,$ 1 $\,\,$ & $\,\,$\color{gr} 2.9955\color{black}  $\,\,$ \\
$\,\,$0.1669$\,\,$ & $\,\,$0.2804$\,\,$ & $\,\,$\color{gr} 0.3338\color{black} $\,\,$ & $\,\,$ 1  $\,\,$ \\
\end{pmatrix},
\end{equation*}
\end{example}
\newpage
\begin{example}
\begin{equation*}
\mathbf{A} =
\begin{pmatrix}
$\,\,$ 1 $\,\,$ & $\,\,$3$\,\,$ & $\,\,$2$\,\,$ & $\,\,$5 $\,\,$ \\
$\,\,$ 1/3$\,\,$ & $\,\,$ 1 $\,\,$ & $\,\,$1$\,\,$ & $\,\,$8 $\,\,$ \\
$\,\,$ 1/2$\,\,$ & $\,\,$ 1 $\,\,$ & $\,\,$ 1 $\,\,$ & $\,\,$4 $\,\,$ \\
$\,\,$ 1/5$\,\,$ & $\,\,$ 1/8$\,\,$ & $\,\,$ 1/4$\,\,$ & $\,\,$ 1  $\,\,$ \\
\end{pmatrix},
\qquad
\lambda_{\max} =
4.2162,
\qquad
CR = 0.0815
\end{equation*}

\begin{equation*}
\mathbf{w}^{cos} =
\begin{pmatrix}
0.452267\\
0.262170\\
\color{red} 0.225398\color{black} \\
0.060164
\end{pmatrix}\end{equation*}
\begin{equation*}
\left[ \frac{{w}^{cos}_i}{{w}^{cos}_j} \right] =
\begin{pmatrix}
$\,\,$ 1 $\,\,$ & $\,\,$1.7251$\,\,$ & $\,\,$\color{red} 2.0065\color{black} $\,\,$ & $\,\,$7.5172$\,\,$ \\
$\,\,$0.5797$\,\,$ & $\,\,$ 1 $\,\,$ & $\,\,$\color{red} 1.1631\color{black} $\,\,$ & $\,\,$4.3576  $\,\,$ \\
$\,\,$\color{red} 0.4984\color{black} $\,\,$ & $\,\,$\color{red} 0.8597\color{black} $\,\,$ & $\,\,$ 1 $\,\,$ & $\,\,$\color{red} 3.7464\color{black}  $\,\,$ \\
$\,\,$0.1330$\,\,$ & $\,\,$0.2295$\,\,$ & $\,\,$\color{red} 0.2669\color{black} $\,\,$ & $\,\,$ 1  $\,\,$ \\
\end{pmatrix},
\end{equation*}

\begin{equation*}
\mathbf{w}^{\prime} =
\begin{pmatrix}
0.451935\\
0.261978\\
0.225967\\
0.060120
\end{pmatrix} =
0.999265\cdot
\begin{pmatrix}
0.452267\\
0.262170\\
\color{gr} 0.226134\color{black} \\
0.060164
\end{pmatrix},
\end{equation*}
\begin{equation*}
\left[ \frac{{w}^{\prime}_i}{{w}^{\prime}_j} \right] =
\begin{pmatrix}
$\,\,$ 1 $\,\,$ & $\,\,$1.7251$\,\,$ & $\,\,$\color{gr} \color{blue} 2\color{black} $\,\,$ & $\,\,$7.5172$\,\,$ \\
$\,\,$0.5797$\,\,$ & $\,\,$ 1 $\,\,$ & $\,\,$\color{gr} 1.1594\color{black} $\,\,$ & $\,\,$4.3576  $\,\,$ \\
$\,\,$\color{gr} \color{blue}  1/2\color{black} $\,\,$ & $\,\,$\color{gr} 0.8625\color{black} $\,\,$ & $\,\,$ 1 $\,\,$ & $\,\,$\color{gr} 3.7586\color{black}  $\,\,$ \\
$\,\,$0.1330$\,\,$ & $\,\,$0.2295$\,\,$ & $\,\,$\color{gr} 0.2661\color{black} $\,\,$ & $\,\,$ 1  $\,\,$ \\
\end{pmatrix},
\end{equation*}
\end{example}
\newpage
\begin{example}
\begin{equation*}
\mathbf{A} =
\begin{pmatrix}
$\,\,$ 1 $\,\,$ & $\,\,$3$\,\,$ & $\,\,$2$\,\,$ & $\,\,$5 $\,\,$ \\
$\,\,$ 1/3$\,\,$ & $\,\,$ 1 $\,\,$ & $\,\,$1$\,\,$ & $\,\,$9 $\,\,$ \\
$\,\,$ 1/2$\,\,$ & $\,\,$ 1 $\,\,$ & $\,\,$ 1 $\,\,$ & $\,\,$4 $\,\,$ \\
$\,\,$ 1/5$\,\,$ & $\,\,$ 1/9$\,\,$ & $\,\,$ 1/4$\,\,$ & $\,\,$ 1  $\,\,$ \\
\end{pmatrix},
\qquad
\lambda_{\max} =
4.2541,
\qquad
CR = 0.0958
\end{equation*}

\begin{equation*}
\mathbf{w}^{cos} =
\begin{pmatrix}
0.449730\\
0.268855\\
\color{red} 0.222563\color{black} \\
0.058852
\end{pmatrix}\end{equation*}
\begin{equation*}
\left[ \frac{{w}^{cos}_i}{{w}^{cos}_j} \right] =
\begin{pmatrix}
$\,\,$ 1 $\,\,$ & $\,\,$1.6728$\,\,$ & $\,\,$\color{red} 2.0207\color{black} $\,\,$ & $\,\,$7.6417$\,\,$ \\
$\,\,$0.5978$\,\,$ & $\,\,$ 1 $\,\,$ & $\,\,$\color{red} 1.2080\color{black} $\,\,$ & $\,\,$4.5683  $\,\,$ \\
$\,\,$\color{red} 0.4949\color{black} $\,\,$ & $\,\,$\color{red} 0.8278\color{black} $\,\,$ & $\,\,$ 1 $\,\,$ & $\,\,$\color{red} 3.7817\color{black}  $\,\,$ \\
$\,\,$0.1309$\,\,$ & $\,\,$0.2189$\,\,$ & $\,\,$\color{red} 0.2644\color{black} $\,\,$ & $\,\,$ 1  $\,\,$ \\
\end{pmatrix},
\end{equation*}

\begin{equation*}
\mathbf{w}^{\prime} =
\begin{pmatrix}
0.448697\\
0.268237\\
0.224349\\
0.058717
\end{pmatrix} =
0.997703\cdot
\begin{pmatrix}
0.449730\\
0.268855\\
\color{gr} 0.224865\color{black} \\
0.058852
\end{pmatrix},
\end{equation*}
\begin{equation*}
\left[ \frac{{w}^{\prime}_i}{{w}^{\prime}_j} \right] =
\begin{pmatrix}
$\,\,$ 1 $\,\,$ & $\,\,$1.6728$\,\,$ & $\,\,$\color{gr} \color{blue} 2\color{black} $\,\,$ & $\,\,$7.6417$\,\,$ \\
$\,\,$0.5978$\,\,$ & $\,\,$ 1 $\,\,$ & $\,\,$\color{gr} 1.1956\color{black} $\,\,$ & $\,\,$4.5683  $\,\,$ \\
$\,\,$\color{gr} \color{blue}  1/2\color{black} $\,\,$ & $\,\,$\color{gr} 0.8364\color{black} $\,\,$ & $\,\,$ 1 $\,\,$ & $\,\,$\color{gr} 3.8208\color{black}  $\,\,$ \\
$\,\,$0.1309$\,\,$ & $\,\,$0.2189$\,\,$ & $\,\,$\color{gr} 0.2617\color{black} $\,\,$ & $\,\,$ 1  $\,\,$ \\
\end{pmatrix},
\end{equation*}
\end{example}
\newpage
\begin{example}
\begin{equation*}
\mathbf{A} =
\begin{pmatrix}
$\,\,$ 1 $\,\,$ & $\,\,$3$\,\,$ & $\,\,$2$\,\,$ & $\,\,$6 $\,\,$ \\
$\,\,$ 1/3$\,\,$ & $\,\,$ 1 $\,\,$ & $\,\,$1$\,\,$ & $\,\,$6 $\,\,$ \\
$\,\,$ 1/2$\,\,$ & $\,\,$ 1 $\,\,$ & $\,\,$ 1 $\,\,$ & $\,\,$4 $\,\,$ \\
$\,\,$ 1/6$\,\,$ & $\,\,$ 1/6$\,\,$ & $\,\,$ 1/4$\,\,$ & $\,\,$ 1  $\,\,$ \\
\end{pmatrix},
\qquad
\lambda_{\max} =
4.1031,
\qquad
CR = 0.0389
\end{equation*}

\begin{equation*}
\mathbf{w}^{cos} =
\begin{pmatrix}
0.471937\\
0.239805\\
\color{red} 0.229396\color{black} \\
0.058862
\end{pmatrix}\end{equation*}
\begin{equation*}
\left[ \frac{{w}^{cos}_i}{{w}^{cos}_j} \right] =
\begin{pmatrix}
$\,\,$ 1 $\,\,$ & $\,\,$1.9680$\,\,$ & $\,\,$\color{red} 2.0573\color{black} $\,\,$ & $\,\,$8.0176$\,\,$ \\
$\,\,$0.5081$\,\,$ & $\,\,$ 1 $\,\,$ & $\,\,$\color{red} 1.0454\color{black} $\,\,$ & $\,\,$4.0740  $\,\,$ \\
$\,\,$\color{red} 0.4861\color{black} $\,\,$ & $\,\,$\color{red} 0.9566\color{black} $\,\,$ & $\,\,$ 1 $\,\,$ & $\,\,$\color{red} 3.8972\color{black}  $\,\,$ \\
$\,\,$0.1247$\,\,$ & $\,\,$0.2455$\,\,$ & $\,\,$\color{red} 0.2566\color{black} $\,\,$ & $\,\,$ 1  $\,\,$ \\
\end{pmatrix},
\end{equation*}

\begin{equation*}
\mathbf{w}^{\prime} =
\begin{pmatrix}
0.469098\\
0.238362\\
0.234033\\
0.058508
\end{pmatrix} =
0.993984\cdot
\begin{pmatrix}
0.471937\\
0.239805\\
\color{gr} 0.235449\color{black} \\
0.058862
\end{pmatrix},
\end{equation*}
\begin{equation*}
\left[ \frac{{w}^{\prime}_i}{{w}^{\prime}_j} \right] =
\begin{pmatrix}
$\,\,$ 1 $\,\,$ & $\,\,$1.9680$\,\,$ & $\,\,$\color{gr} 2.0044\color{black} $\,\,$ & $\,\,$8.0176$\,\,$ \\
$\,\,$0.5081$\,\,$ & $\,\,$ 1 $\,\,$ & $\,\,$\color{gr} 1.0185\color{black} $\,\,$ & $\,\,$4.0740  $\,\,$ \\
$\,\,$\color{gr} 0.4989\color{black} $\,\,$ & $\,\,$\color{gr} 0.9818\color{black} $\,\,$ & $\,\,$ 1 $\,\,$ & $\,\,$\color{gr} \color{blue} 4\color{black}  $\,\,$ \\
$\,\,$0.1247$\,\,$ & $\,\,$0.2455$\,\,$ & $\,\,$\color{gr} \color{blue}  1/4\color{black} $\,\,$ & $\,\,$ 1  $\,\,$ \\
\end{pmatrix},
\end{equation*}
\end{example}
\newpage
\begin{example}
\begin{equation*}
\mathbf{A} =
\begin{pmatrix}
$\,\,$ 1 $\,\,$ & $\,\,$3$\,\,$ & $\,\,$2$\,\,$ & $\,\,$6 $\,\,$ \\
$\,\,$ 1/3$\,\,$ & $\,\,$ 1 $\,\,$ & $\,\,$1$\,\,$ & $\,\,$7 $\,\,$ \\
$\,\,$ 1/2$\,\,$ & $\,\,$ 1 $\,\,$ & $\,\,$ 1 $\,\,$ & $\,\,$4 $\,\,$ \\
$\,\,$ 1/6$\,\,$ & $\,\,$ 1/7$\,\,$ & $\,\,$ 1/4$\,\,$ & $\,\,$ 1  $\,\,$ \\
\end{pmatrix},
\qquad
\lambda_{\max} =
4.1365,
\qquad
CR = 0.0515
\end{equation*}

\begin{equation*}
\mathbf{w}^{cos} =
\begin{pmatrix}
0.467620\\
0.249210\\
\color{red} 0.226171\color{black} \\
0.056999
\end{pmatrix}\end{equation*}
\begin{equation*}
\left[ \frac{{w}^{cos}_i}{{w}^{cos}_j} \right] =
\begin{pmatrix}
$\,\,$ 1 $\,\,$ & $\,\,$1.8764$\,\,$ & $\,\,$\color{red} 2.0676\color{black} $\,\,$ & $\,\,$8.2040$\,\,$ \\
$\,\,$0.5329$\,\,$ & $\,\,$ 1 $\,\,$ & $\,\,$\color{red} 1.1019\color{black} $\,\,$ & $\,\,$4.3722  $\,\,$ \\
$\,\,$\color{red} 0.4837\color{black} $\,\,$ & $\,\,$\color{red} 0.9075\color{black} $\,\,$ & $\,\,$ 1 $\,\,$ & $\,\,$\color{red} 3.9680\color{black}  $\,\,$ \\
$\,\,$0.1219$\,\,$ & $\,\,$0.2287$\,\,$ & $\,\,$\color{red} 0.2520\color{black} $\,\,$ & $\,\,$ 1  $\,\,$ \\
\end{pmatrix},
\end{equation*}

\begin{equation*}
\mathbf{w}^{\prime} =
\begin{pmatrix}
0.466769\\
0.248756\\
0.227580\\
0.056895
\end{pmatrix} =
0.998179\cdot
\begin{pmatrix}
0.467620\\
0.249210\\
\color{gr} 0.227995\color{black} \\
0.056999
\end{pmatrix},
\end{equation*}
\begin{equation*}
\left[ \frac{{w}^{\prime}_i}{{w}^{\prime}_j} \right] =
\begin{pmatrix}
$\,\,$ 1 $\,\,$ & $\,\,$1.8764$\,\,$ & $\,\,$\color{gr} 2.0510\color{black} $\,\,$ & $\,\,$8.2040$\,\,$ \\
$\,\,$0.5329$\,\,$ & $\,\,$ 1 $\,\,$ & $\,\,$\color{gr} 1.0931\color{black} $\,\,$ & $\,\,$4.3722  $\,\,$ \\
$\,\,$\color{gr} 0.4876\color{black} $\,\,$ & $\,\,$\color{gr} 0.9149\color{black} $\,\,$ & $\,\,$ 1 $\,\,$ & $\,\,$\color{gr} \color{blue} 4\color{black}  $\,\,$ \\
$\,\,$0.1219$\,\,$ & $\,\,$0.2287$\,\,$ & $\,\,$\color{gr} \color{blue}  1/4\color{black} $\,\,$ & $\,\,$ 1  $\,\,$ \\
\end{pmatrix},
\end{equation*}
\end{example}
\newpage
\begin{example}
\begin{equation*}
\mathbf{A} =
\begin{pmatrix}
$\,\,$ 1 $\,\,$ & $\,\,$3$\,\,$ & $\,\,$2$\,\,$ & $\,\,$6 $\,\,$ \\
$\,\,$ 1/3$\,\,$ & $\,\,$ 1 $\,\,$ & $\,\,$3$\,\,$ & $\,\,$3 $\,\,$ \\
$\,\,$ 1/2$\,\,$ & $\,\,$ 1/3$\,\,$ & $\,\,$ 1 $\,\,$ & $\,\,$2 $\,\,$ \\
$\,\,$ 1/6$\,\,$ & $\,\,$ 1/3$\,\,$ & $\,\,$ 1/2$\,\,$ & $\,\,$ 1  $\,\,$ \\
\end{pmatrix},
\qquad
\lambda_{\max} =
4.1990,
\qquad
CR = 0.0750
\end{equation*}

\begin{equation*}
\mathbf{w}^{cos} =
\begin{pmatrix}
0.481352\\
0.275959\\
0.163676\\
\color{red} 0.079012\color{black}
\end{pmatrix}\end{equation*}
\begin{equation*}
\left[ \frac{{w}^{cos}_i}{{w}^{cos}_j} \right] =
\begin{pmatrix}
$\,\,$ 1 $\,\,$ & $\,\,$1.7443$\,\,$ & $\,\,$2.9409$\,\,$ & $\,\,$\color{red} 6.0921\color{black} $\,\,$ \\
$\,\,$0.5733$\,\,$ & $\,\,$ 1 $\,\,$ & $\,\,$1.6860$\,\,$ & $\,\,$\color{red} 3.4926\color{black}   $\,\,$ \\
$\,\,$0.3400$\,\,$ & $\,\,$0.5931$\,\,$ & $\,\,$ 1 $\,\,$ & $\,\,$\color{red} 2.0715\color{black}  $\,\,$ \\
$\,\,$\color{red} 0.1641\color{black} $\,\,$ & $\,\,$\color{red} 0.2863\color{black} $\,\,$ & $\,\,$\color{red} 0.4827\color{black} $\,\,$ & $\,\,$ 1  $\,\,$ \\
\end{pmatrix},
\end{equation*}

\begin{equation*}
\mathbf{w}^{\prime} =
\begin{pmatrix}
0.480769\\
0.275625\\
0.163478\\
0.080128
\end{pmatrix} =
0.998788\cdot
\begin{pmatrix}
0.481352\\
0.275959\\
0.163676\\
\color{gr} 0.080225\color{black}
\end{pmatrix},
\end{equation*}
\begin{equation*}
\left[ \frac{{w}^{\prime}_i}{{w}^{\prime}_j} \right] =
\begin{pmatrix}
$\,\,$ 1 $\,\,$ & $\,\,$1.7443$\,\,$ & $\,\,$2.9409$\,\,$ & $\,\,$\color{gr} \color{blue} 6\color{black} $\,\,$ \\
$\,\,$0.5733$\,\,$ & $\,\,$ 1 $\,\,$ & $\,\,$1.6860$\,\,$ & $\,\,$\color{gr} 3.4398\color{black}   $\,\,$ \\
$\,\,$0.3400$\,\,$ & $\,\,$0.5931$\,\,$ & $\,\,$ 1 $\,\,$ & $\,\,$\color{gr} 2.0402\color{black}  $\,\,$ \\
$\,\,$\color{gr} \color{blue}  1/6\color{black} $\,\,$ & $\,\,$\color{gr} 0.2907\color{black} $\,\,$ & $\,\,$\color{gr} 0.4901\color{black} $\,\,$ & $\,\,$ 1  $\,\,$ \\
\end{pmatrix},
\end{equation*}
\end{example}
\newpage
\begin{example}
\begin{equation*}
\mathbf{A} =
\begin{pmatrix}
$\,\,$ 1 $\,\,$ & $\,\,$3$\,\,$ & $\,\,$2$\,\,$ & $\,\,$7 $\,\,$ \\
$\,\,$ 1/3$\,\,$ & $\,\,$ 1 $\,\,$ & $\,\,$1$\,\,$ & $\,\,$7 $\,\,$ \\
$\,\,$ 1/2$\,\,$ & $\,\,$ 1 $\,\,$ & $\,\,$ 1 $\,\,$ & $\,\,$5 $\,\,$ \\
$\,\,$ 1/7$\,\,$ & $\,\,$ 1/7$\,\,$ & $\,\,$ 1/5$\,\,$ & $\,\,$ 1  $\,\,$ \\
\end{pmatrix},
\qquad
\lambda_{\max} =
4.1027,
\qquad
CR = 0.0387
\end{equation*}

\begin{equation*}
\mathbf{w}^{cos} =
\begin{pmatrix}
0.474632\\
0.240567\\
\color{red} 0.234953\color{black} \\
0.049849
\end{pmatrix}\end{equation*}
\begin{equation*}
\left[ \frac{{w}^{cos}_i}{{w}^{cos}_j} \right] =
\begin{pmatrix}
$\,\,$ 1 $\,\,$ & $\,\,$1.9730$\,\,$ & $\,\,$\color{red} 2.0201\color{black} $\,\,$ & $\,\,$9.5215$\,\,$ \\
$\,\,$0.5068$\,\,$ & $\,\,$ 1 $\,\,$ & $\,\,$\color{red} 1.0239\color{black} $\,\,$ & $\,\,$4.8260  $\,\,$ \\
$\,\,$\color{red} 0.4950\color{black} $\,\,$ & $\,\,$\color{red} 0.9767\color{black} $\,\,$ & $\,\,$ 1 $\,\,$ & $\,\,$\color{red} 4.7133\color{black}  $\,\,$ \\
$\,\,$0.1050$\,\,$ & $\,\,$0.2072$\,\,$ & $\,\,$\color{red} 0.2122\color{black} $\,\,$ & $\,\,$ 1  $\,\,$ \\
\end{pmatrix},
\end{equation*}

\begin{equation*}
\mathbf{w}^{\prime} =
\begin{pmatrix}
0.473513\\
0.240000\\
0.236756\\
0.049731
\end{pmatrix} =
0.997643\cdot
\begin{pmatrix}
0.474632\\
0.240567\\
\color{gr} 0.237316\color{black} \\
0.049849
\end{pmatrix},
\end{equation*}
\begin{equation*}
\left[ \frac{{w}^{\prime}_i}{{w}^{\prime}_j} \right] =
\begin{pmatrix}
$\,\,$ 1 $\,\,$ & $\,\,$1.9730$\,\,$ & $\,\,$\color{gr} \color{blue} 2\color{black} $\,\,$ & $\,\,$9.5215$\,\,$ \\
$\,\,$0.5068$\,\,$ & $\,\,$ 1 $\,\,$ & $\,\,$\color{gr} 1.0137\color{black} $\,\,$ & $\,\,$4.8260  $\,\,$ \\
$\,\,$\color{gr} \color{blue}  1/2\color{black} $\,\,$ & $\,\,$\color{gr} 0.9865\color{black} $\,\,$ & $\,\,$ 1 $\,\,$ & $\,\,$\color{gr} 4.7607\color{black}  $\,\,$ \\
$\,\,$0.1050$\,\,$ & $\,\,$0.2072$\,\,$ & $\,\,$\color{gr} 0.2101\color{black} $\,\,$ & $\,\,$ 1  $\,\,$ \\
\end{pmatrix},
\end{equation*}
\end{example}
\newpage
\begin{example}
\begin{equation*}
\mathbf{A} =
\begin{pmatrix}
$\,\,$ 1 $\,\,$ & $\,\,$3$\,\,$ & $\,\,$2$\,\,$ & $\,\,$7 $\,\,$ \\
$\,\,$ 1/3$\,\,$ & $\,\,$ 1 $\,\,$ & $\,\,$1$\,\,$ & $\,\,$8 $\,\,$ \\
$\,\,$ 1/2$\,\,$ & $\,\,$ 1 $\,\,$ & $\,\,$ 1 $\,\,$ & $\,\,$5 $\,\,$ \\
$\,\,$ 1/7$\,\,$ & $\,\,$ 1/8$\,\,$ & $\,\,$ 1/5$\,\,$ & $\,\,$ 1  $\,\,$ \\
\end{pmatrix},
\qquad
\lambda_{\max} =
4.1301,
\qquad
CR = 0.0490
\end{equation*}

\begin{equation*}
\mathbf{w}^{cos} =
\begin{pmatrix}
0.470872\\
0.248724\\
\color{red} 0.231949\color{black} \\
0.048454
\end{pmatrix}\end{equation*}
\begin{equation*}
\left[ \frac{{w}^{cos}_i}{{w}^{cos}_j} \right] =
\begin{pmatrix}
$\,\,$ 1 $\,\,$ & $\,\,$1.8931$\,\,$ & $\,\,$\color{red} 2.0301\color{black} $\,\,$ & $\,\,$9.7179$\,\,$ \\
$\,\,$0.5282$\,\,$ & $\,\,$ 1 $\,\,$ & $\,\,$\color{red} 1.0723\color{black} $\,\,$ & $\,\,$5.1332  $\,\,$ \\
$\,\,$\color{red} 0.4926\color{black} $\,\,$ & $\,\,$\color{red} 0.9326\color{black} $\,\,$ & $\,\,$ 1 $\,\,$ & $\,\,$\color{red} 4.7870\color{black}  $\,\,$ \\
$\,\,$0.1029$\,\,$ & $\,\,$0.1948$\,\,$ & $\,\,$\color{red} 0.2089\color{black} $\,\,$ & $\,\,$ 1  $\,\,$ \\
\end{pmatrix},
\end{equation*}

\begin{equation*}
\mathbf{w}^{\prime} =
\begin{pmatrix}
0.469236\\
0.247860\\
0.234618\\
0.048286
\end{pmatrix} =
0.996525\cdot
\begin{pmatrix}
0.470872\\
0.248724\\
\color{gr} 0.235436\color{black} \\
0.048454
\end{pmatrix},
\end{equation*}
\begin{equation*}
\left[ \frac{{w}^{\prime}_i}{{w}^{\prime}_j} \right] =
\begin{pmatrix}
$\,\,$ 1 $\,\,$ & $\,\,$1.8931$\,\,$ & $\,\,$\color{gr} \color{blue} 2\color{black} $\,\,$ & $\,\,$9.7179$\,\,$ \\
$\,\,$0.5282$\,\,$ & $\,\,$ 1 $\,\,$ & $\,\,$\color{gr} 1.0564\color{black} $\,\,$ & $\,\,$5.1332  $\,\,$ \\
$\,\,$\color{gr} \color{blue}  1/2\color{black} $\,\,$ & $\,\,$\color{gr} 0.9466\color{black} $\,\,$ & $\,\,$ 1 $\,\,$ & $\,\,$\color{gr} 4.8589\color{black}  $\,\,$ \\
$\,\,$0.1029$\,\,$ & $\,\,$0.1948$\,\,$ & $\,\,$\color{gr} 0.2058\color{black} $\,\,$ & $\,\,$ 1  $\,\,$ \\
\end{pmatrix},
\end{equation*}
\end{example}
\newpage
\begin{example}
\begin{equation*}
\mathbf{A} =
\begin{pmatrix}
$\,\,$ 1 $\,\,$ & $\,\,$3$\,\,$ & $\,\,$2$\,\,$ & $\,\,$7 $\,\,$ \\
$\,\,$ 1/3$\,\,$ & $\,\,$ 1 $\,\,$ & $\,\,$1$\,\,$ & $\,\,$9 $\,\,$ \\
$\,\,$ 1/2$\,\,$ & $\,\,$ 1 $\,\,$ & $\,\,$ 1 $\,\,$ & $\,\,$5 $\,\,$ \\
$\,\,$ 1/7$\,\,$ & $\,\,$ 1/9$\,\,$ & $\,\,$ 1/5$\,\,$ & $\,\,$ 1  $\,\,$ \\
\end{pmatrix},
\qquad
\lambda_{\max} =
4.1583,
\qquad
CR = 0.0597
\end{equation*}

\begin{equation*}
\mathbf{w}^{cos} =
\begin{pmatrix}
0.467589\\
0.255894\\
\color{red} 0.229226\color{black} \\
0.047291
\end{pmatrix}\end{equation*}
\begin{equation*}
\left[ \frac{{w}^{cos}_i}{{w}^{cos}_j} \right] =
\begin{pmatrix}
$\,\,$ 1 $\,\,$ & $\,\,$1.8273$\,\,$ & $\,\,$\color{red} 2.0399\color{black} $\,\,$ & $\,\,$9.8875$\,\,$ \\
$\,\,$0.5473$\,\,$ & $\,\,$ 1 $\,\,$ & $\,\,$\color{red} 1.1163\color{black} $\,\,$ & $\,\,$5.4110  $\,\,$ \\
$\,\,$\color{red} 0.4902\color{black} $\,\,$ & $\,\,$\color{red} 0.8958\color{black} $\,\,$ & $\,\,$ 1 $\,\,$ & $\,\,$\color{red} 4.8471\color{black}  $\,\,$ \\
$\,\,$0.1011$\,\,$ & $\,\,$0.1848$\,\,$ & $\,\,$\color{red} 0.2063\color{black} $\,\,$ & $\,\,$ 1  $\,\,$ \\
\end{pmatrix},
\end{equation*}

\begin{equation*}
\mathbf{w}^{\prime} =
\begin{pmatrix}
0.465463\\
0.254730\\
0.232731\\
0.047076
\end{pmatrix} =
0.995452\cdot
\begin{pmatrix}
0.467589\\
0.255894\\
\color{gr} 0.233795\color{black} \\
0.047291
\end{pmatrix},
\end{equation*}
\begin{equation*}
\left[ \frac{{w}^{\prime}_i}{{w}^{\prime}_j} \right] =
\begin{pmatrix}
$\,\,$ 1 $\,\,$ & $\,\,$1.8273$\,\,$ & $\,\,$\color{gr} \color{blue} 2\color{black} $\,\,$ & $\,\,$9.8875$\,\,$ \\
$\,\,$0.5473$\,\,$ & $\,\,$ 1 $\,\,$ & $\,\,$\color{gr} 1.0945\color{black} $\,\,$ & $\,\,$5.4110  $\,\,$ \\
$\,\,$\color{gr} \color{blue}  1/2\color{black} $\,\,$ & $\,\,$\color{gr} 0.9136\color{black} $\,\,$ & $\,\,$ 1 $\,\,$ & $\,\,$\color{gr} 4.9437\color{black}  $\,\,$ \\
$\,\,$0.1011$\,\,$ & $\,\,$0.1848$\,\,$ & $\,\,$\color{gr} 0.2023\color{black} $\,\,$ & $\,\,$ 1  $\,\,$ \\
\end{pmatrix},
\end{equation*}
\end{example}
\newpage
\begin{example}
\begin{equation*}
\mathbf{A} =
\begin{pmatrix}
$\,\,$ 1 $\,\,$ & $\,\,$3$\,\,$ & $\,\,$2$\,\,$ & $\,\,$8 $\,\,$ \\
$\,\,$ 1/3$\,\,$ & $\,\,$ 1 $\,\,$ & $\,\,$1$\,\,$ & $\,\,$7 $\,\,$ \\
$\,\,$ 1/2$\,\,$ & $\,\,$ 1 $\,\,$ & $\,\,$ 1 $\,\,$ & $\,\,$5 $\,\,$ \\
$\,\,$ 1/8$\,\,$ & $\,\,$ 1/7$\,\,$ & $\,\,$ 1/5$\,\,$ & $\,\,$ 1  $\,\,$ \\
\end{pmatrix},
\qquad
\lambda_{\max} =
4.0799,
\qquad
CR = 0.0301
\end{equation*}

\begin{equation*}
\mathbf{w}^{cos} =
\begin{pmatrix}
0.484216\\
0.236413\\
\color{red} 0.232300\color{black} \\
0.047070
\end{pmatrix}\end{equation*}
\begin{equation*}
\left[ \frac{{w}^{cos}_i}{{w}^{cos}_j} \right] =
\begin{pmatrix}
$\,\,$ 1 $\,\,$ & $\,\,$2.0482$\,\,$ & $\,\,$\color{red} 2.0844\color{black} $\,\,$ & $\,\,$10.2871$\,\,$ \\
$\,\,$0.4882$\,\,$ & $\,\,$ 1 $\,\,$ & $\,\,$\color{red} 1.0177\color{black} $\,\,$ & $\,\,$5.0226  $\,\,$ \\
$\,\,$\color{red} 0.4797\color{black} $\,\,$ & $\,\,$\color{red} 0.9826\color{black} $\,\,$ & $\,\,$ 1 $\,\,$ & $\,\,$\color{red} 4.9352\color{black}  $\,\,$ \\
$\,\,$0.0972$\,\,$ & $\,\,$0.1991$\,\,$ & $\,\,$\color{red} 0.2026\color{black} $\,\,$ & $\,\,$ 1  $\,\,$ \\
\end{pmatrix},
\end{equation*}

\begin{equation*}
\mathbf{w}^{\prime} =
\begin{pmatrix}
0.482744\\
0.235694\\
0.234635\\
0.046927
\end{pmatrix} =
0.996958\cdot
\begin{pmatrix}
0.484216\\
0.236413\\
\color{gr} 0.235351\color{black} \\
0.047070
\end{pmatrix},
\end{equation*}
\begin{equation*}
\left[ \frac{{w}^{\prime}_i}{{w}^{\prime}_j} \right] =
\begin{pmatrix}
$\,\,$ 1 $\,\,$ & $\,\,$2.0482$\,\,$ & $\,\,$\color{gr} 2.0574\color{black} $\,\,$ & $\,\,$10.2871$\,\,$ \\
$\,\,$0.4882$\,\,$ & $\,\,$ 1 $\,\,$ & $\,\,$\color{gr} 1.0045\color{black} $\,\,$ & $\,\,$5.0226  $\,\,$ \\
$\,\,$\color{gr} 0.4860\color{black} $\,\,$ & $\,\,$\color{gr} 0.9955\color{black} $\,\,$ & $\,\,$ 1 $\,\,$ & $\,\,$\color{gr} \color{blue} 5\color{black}  $\,\,$ \\
$\,\,$0.0972$\,\,$ & $\,\,$0.1991$\,\,$ & $\,\,$\color{gr} \color{blue}  1/5\color{black} $\,\,$ & $\,\,$ 1  $\,\,$ \\
\end{pmatrix},
\end{equation*}
\end{example}
\newpage
\begin{example}
\begin{equation*}
\mathbf{A} =
\begin{pmatrix}
$\,\,$ 1 $\,\,$ & $\,\,$3$\,\,$ & $\,\,$2$\,\,$ & $\,\,$8 $\,\,$ \\
$\,\,$ 1/3$\,\,$ & $\,\,$ 1 $\,\,$ & $\,\,$1$\,\,$ & $\,\,$9 $\,\,$ \\
$\,\,$ 1/2$\,\,$ & $\,\,$ 1 $\,\,$ & $\,\,$ 1 $\,\,$ & $\,\,$6 $\,\,$ \\
$\,\,$ 1/8$\,\,$ & $\,\,$ 1/9$\,\,$ & $\,\,$ 1/6$\,\,$ & $\,\,$ 1  $\,\,$ \\
\end{pmatrix},
\qquad
\lambda_{\max} =
4.1263,
\qquad
CR = 0.0476
\end{equation*}

\begin{equation*}
\mathbf{w}^{cos} =
\begin{pmatrix}
0.473261\\
0.248284\\
\color{red} 0.236268\color{black} \\
0.042186
\end{pmatrix}\end{equation*}
\begin{equation*}
\left[ \frac{{w}^{cos}_i}{{w}^{cos}_j} \right] =
\begin{pmatrix}
$\,\,$ 1 $\,\,$ & $\,\,$1.9061$\,\,$ & $\,\,$\color{red} 2.0031\color{black} $\,\,$ & $\,\,$11.2185$\,\,$ \\
$\,\,$0.5246$\,\,$ & $\,\,$ 1 $\,\,$ & $\,\,$\color{red} 1.0509\color{black} $\,\,$ & $\,\,$5.8855  $\,\,$ \\
$\,\,$\color{red} 0.4992\color{black} $\,\,$ & $\,\,$\color{red} 0.9516\color{black} $\,\,$ & $\,\,$ 1 $\,\,$ & $\,\,$\color{red} 5.6007\color{black}  $\,\,$ \\
$\,\,$0.0891$\,\,$ & $\,\,$0.1699$\,\,$ & $\,\,$\color{red} 0.1785\color{black} $\,\,$ & $\,\,$ 1  $\,\,$ \\
\end{pmatrix},
\end{equation*}

\begin{equation*}
\mathbf{w}^{\prime} =
\begin{pmatrix}
0.473090\\
0.248195\\
0.236545\\
0.042170
\end{pmatrix} =
0.999638\cdot
\begin{pmatrix}
0.473261\\
0.248284\\
\color{gr} 0.236631\color{black} \\
0.042186
\end{pmatrix},
\end{equation*}
\begin{equation*}
\left[ \frac{{w}^{\prime}_i}{{w}^{\prime}_j} \right] =
\begin{pmatrix}
$\,\,$ 1 $\,\,$ & $\,\,$1.9061$\,\,$ & $\,\,$\color{gr} \color{blue} 2\color{black} $\,\,$ & $\,\,$11.2185$\,\,$ \\
$\,\,$0.5246$\,\,$ & $\,\,$ 1 $\,\,$ & $\,\,$\color{gr} 1.0492\color{black} $\,\,$ & $\,\,$5.8855  $\,\,$ \\
$\,\,$\color{gr} \color{blue}  1/2\color{black} $\,\,$ & $\,\,$\color{gr} 0.9531\color{black} $\,\,$ & $\,\,$ 1 $\,\,$ & $\,\,$\color{gr} 5.6093\color{black}  $\,\,$ \\
$\,\,$0.0891$\,\,$ & $\,\,$0.1699$\,\,$ & $\,\,$\color{gr} 0.1783\color{black} $\,\,$ & $\,\,$ 1  $\,\,$ \\
\end{pmatrix},
\end{equation*}
\end{example}
\newpage
\begin{example}
\begin{equation*}
\mathbf{A} =
\begin{pmatrix}
$\,\,$ 1 $\,\,$ & $\,\,$3$\,\,$ & $\,\,$2$\,\,$ & $\,\,$8 $\,\,$ \\
$\,\,$ 1/3$\,\,$ & $\,\,$ 1 $\,\,$ & $\,\,$2$\,\,$ & $\,\,$4 $\,\,$ \\
$\,\,$ 1/2$\,\,$ & $\,\,$ 1/2$\,\,$ & $\,\,$ 1 $\,\,$ & $\,\,$3 $\,\,$ \\
$\,\,$ 1/8$\,\,$ & $\,\,$ 1/4$\,\,$ & $\,\,$ 1/3$\,\,$ & $\,\,$ 1  $\,\,$ \\
\end{pmatrix},
\qquad
\lambda_{\max} =
4.1031,
\qquad
CR = 0.0389
\end{equation*}

\begin{equation*}
\mathbf{w}^{cos} =
\begin{pmatrix}
0.498715\\
0.254440\\
0.186220\\
\color{red} 0.060625\color{black}
\end{pmatrix}\end{equation*}
\begin{equation*}
\left[ \frac{{w}^{cos}_i}{{w}^{cos}_j} \right] =
\begin{pmatrix}
$\,\,$ 1 $\,\,$ & $\,\,$1.9601$\,\,$ & $\,\,$2.6781$\,\,$ & $\,\,$\color{red} 8.2262\color{black} $\,\,$ \\
$\,\,$0.5102$\,\,$ & $\,\,$ 1 $\,\,$ & $\,\,$1.3663$\,\,$ & $\,\,$\color{red} 4.1969\color{black}   $\,\,$ \\
$\,\,$0.3734$\,\,$ & $\,\,$0.7319$\,\,$ & $\,\,$ 1 $\,\,$ & $\,\,$\color{red} 3.0717\color{black}  $\,\,$ \\
$\,\,$\color{red} 0.1216\color{black} $\,\,$ & $\,\,$\color{red} 0.2383\color{black} $\,\,$ & $\,\,$\color{red} 0.3256\color{black} $\,\,$ & $\,\,$ 1  $\,\,$ \\
\end{pmatrix},
\end{equation*}

\begin{equation*}
\mathbf{w}^{\prime} =
\begin{pmatrix}
0.497994\\
0.254072\\
0.185951\\
0.061984
\end{pmatrix} =
0.998554\cdot
\begin{pmatrix}
0.498715\\
0.254440\\
0.186220\\
\color{gr} 0.062073\color{black}
\end{pmatrix},
\end{equation*}
\begin{equation*}
\left[ \frac{{w}^{\prime}_i}{{w}^{\prime}_j} \right] =
\begin{pmatrix}
$\,\,$ 1 $\,\,$ & $\,\,$1.9601$\,\,$ & $\,\,$2.6781$\,\,$ & $\,\,$\color{gr} 8.0343\color{black} $\,\,$ \\
$\,\,$0.5102$\,\,$ & $\,\,$ 1 $\,\,$ & $\,\,$1.3663$\,\,$ & $\,\,$\color{gr} 4.0990\color{black}   $\,\,$ \\
$\,\,$0.3734$\,\,$ & $\,\,$0.7319$\,\,$ & $\,\,$ 1 $\,\,$ & $\,\,$\color{gr} \color{blue} 3\color{black}  $\,\,$ \\
$\,\,$\color{gr} 0.1245\color{black} $\,\,$ & $\,\,$\color{gr} 0.2440\color{black} $\,\,$ & $\,\,$\color{gr} \color{blue}  1/3\color{black} $\,\,$ & $\,\,$ 1  $\,\,$ \\
\end{pmatrix},
\end{equation*}
\end{example}
\newpage
\begin{example}
\begin{equation*}
\mathbf{A} =
\begin{pmatrix}
$\,\,$ 1 $\,\,$ & $\,\,$3$\,\,$ & $\,\,$2$\,\,$ & $\,\,$8 $\,\,$ \\
$\,\,$ 1/3$\,\,$ & $\,\,$ 1 $\,\,$ & $\,\,$3$\,\,$ & $\,\,$5 $\,\,$ \\
$\,\,$ 1/2$\,\,$ & $\,\,$ 1/3$\,\,$ & $\,\,$ 1 $\,\,$ & $\,\,$3 $\,\,$ \\
$\,\,$ 1/8$\,\,$ & $\,\,$ 1/5$\,\,$ & $\,\,$ 1/3$\,\,$ & $\,\,$ 1  $\,\,$ \\
\end{pmatrix},
\qquad
\lambda_{\max} =
4.1999,
\qquad
CR = 0.0754
\end{equation*}

\begin{equation*}
\mathbf{w}^{cos} =
\begin{pmatrix}
0.482594\\
0.292780\\
0.169350\\
\color{red} 0.055276\color{black}
\end{pmatrix}\end{equation*}
\begin{equation*}
\left[ \frac{{w}^{cos}_i}{{w}^{cos}_j} \right] =
\begin{pmatrix}
$\,\,$ 1 $\,\,$ & $\,\,$1.6483$\,\,$ & $\,\,$2.8497$\,\,$ & $\,\,$\color{red} 8.7306\color{black} $\,\,$ \\
$\,\,$0.6067$\,\,$ & $\,\,$ 1 $\,\,$ & $\,\,$1.7288$\,\,$ & $\,\,$\color{red} 5.2967\color{black}   $\,\,$ \\
$\,\,$0.3509$\,\,$ & $\,\,$0.5784$\,\,$ & $\,\,$ 1 $\,\,$ & $\,\,$\color{red} 3.0637\color{black}  $\,\,$ \\
$\,\,$\color{red} 0.1145\color{black} $\,\,$ & $\,\,$\color{red} 0.1888\color{black} $\,\,$ & $\,\,$\color{red} 0.3264\color{black} $\,\,$ & $\,\,$ 1  $\,\,$ \\
\end{pmatrix},
\end{equation*}

\begin{equation*}
\mathbf{w}^{\prime} =
\begin{pmatrix}
0.482028\\
0.292437\\
0.169151\\
0.056384
\end{pmatrix} =
0.998828\cdot
\begin{pmatrix}
0.482594\\
0.292780\\
0.169350\\
\color{gr} 0.056450\color{black}
\end{pmatrix},
\end{equation*}
\begin{equation*}
\left[ \frac{{w}^{\prime}_i}{{w}^{\prime}_j} \right] =
\begin{pmatrix}
$\,\,$ 1 $\,\,$ & $\,\,$1.6483$\,\,$ & $\,\,$2.8497$\,\,$ & $\,\,$\color{gr} 8.5491\color{black} $\,\,$ \\
$\,\,$0.6067$\,\,$ & $\,\,$ 1 $\,\,$ & $\,\,$1.7288$\,\,$ & $\,\,$\color{gr} 5.1865\color{black}   $\,\,$ \\
$\,\,$0.3509$\,\,$ & $\,\,$0.5784$\,\,$ & $\,\,$ 1 $\,\,$ & $\,\,$\color{gr} \color{blue} 3\color{black}  $\,\,$ \\
$\,\,$\color{gr} 0.1170\color{black} $\,\,$ & $\,\,$\color{gr} 0.1928\color{black} $\,\,$ & $\,\,$\color{gr} \color{blue}  1/3\color{black} $\,\,$ & $\,\,$ 1  $\,\,$ \\
\end{pmatrix},
\end{equation*}
\end{example}
\newpage
\begin{example}
\begin{equation*}
\mathbf{A} =
\begin{pmatrix}
$\,\,$ 1 $\,\,$ & $\,\,$3$\,\,$ & $\,\,$2$\,\,$ & $\,\,$9 $\,\,$ \\
$\,\,$ 1/3$\,\,$ & $\,\,$ 1 $\,\,$ & $\,\,$1$\,\,$ & $\,\,$8 $\,\,$ \\
$\,\,$ 1/2$\,\,$ & $\,\,$ 1 $\,\,$ & $\,\,$ 1 $\,\,$ & $\,\,$6 $\,\,$ \\
$\,\,$ 1/9$\,\,$ & $\,\,$ 1/8$\,\,$ & $\,\,$ 1/6$\,\,$ & $\,\,$ 1  $\,\,$ \\
\end{pmatrix},
\qquad
\lambda_{\max} =
4.0820,
\qquad
CR = 0.0309
\end{equation*}

\begin{equation*}
\mathbf{w}^{cos} =
\begin{pmatrix}
0.484909\\
0.237450\\
\color{red} 0.236547\color{black} \\
0.041093
\end{pmatrix}\end{equation*}
\begin{equation*}
\left[ \frac{{w}^{cos}_i}{{w}^{cos}_j} \right] =
\begin{pmatrix}
$\,\,$ 1 $\,\,$ & $\,\,$2.0421$\,\,$ & $\,\,$\color{red} 2.0500\color{black} $\,\,$ & $\,\,$11.8001$\,\,$ \\
$\,\,$0.4897$\,\,$ & $\,\,$ 1 $\,\,$ & $\,\,$\color{red} 1.0038\color{black} $\,\,$ & $\,\,$5.7783  $\,\,$ \\
$\,\,$\color{red} 0.4878\color{black} $\,\,$ & $\,\,$\color{red} 0.9962\color{black} $\,\,$ & $\,\,$ 1 $\,\,$ & $\,\,$\color{red} 5.7563\color{black}  $\,\,$ \\
$\,\,$0.0847$\,\,$ & $\,\,$0.1731$\,\,$ & $\,\,$\color{red} 0.1737\color{black} $\,\,$ & $\,\,$ 1  $\,\,$ \\
\end{pmatrix},
\end{equation*}

\begin{equation*}
\mathbf{w}^{\prime} =
\begin{pmatrix}
0.484471\\
0.237236\\
0.237236\\
0.041056
\end{pmatrix} =
0.999097\cdot
\begin{pmatrix}
0.484909\\
0.237450\\
\color{gr} 0.237450\color{black} \\
0.041093
\end{pmatrix},
\end{equation*}
\begin{equation*}
\left[ \frac{{w}^{\prime}_i}{{w}^{\prime}_j} \right] =
\begin{pmatrix}
$\,\,$ 1 $\,\,$ & $\,\,$2.0421$\,\,$ & $\,\,$\color{gr} 2.0421\color{black} $\,\,$ & $\,\,$11.8001$\,\,$ \\
$\,\,$0.4897$\,\,$ & $\,\,$ 1 $\,\,$ & $\,\,$\color{gr} \color{blue} 1\color{black} $\,\,$ & $\,\,$5.7783  $\,\,$ \\
$\,\,$\color{gr} 0.4897\color{black} $\,\,$ & $\,\,$\color{gr} \color{blue} 1\color{black} $\,\,$ & $\,\,$ 1 $\,\,$ & $\,\,$\color{gr} 5.7783\color{black}  $\,\,$ \\
$\,\,$0.0847$\,\,$ & $\,\,$0.1731$\,\,$ & $\,\,$\color{gr} 0.1731\color{black} $\,\,$ & $\,\,$ 1  $\,\,$ \\
\end{pmatrix},
\end{equation*}
\end{example}
\newpage
\begin{example}
\begin{equation*}
\mathbf{A} =
\begin{pmatrix}
$\,\,$ 1 $\,\,$ & $\,\,$3$\,\,$ & $\,\,$2$\,\,$ & $\,\,$9 $\,\,$ \\
$\,\,$ 1/3$\,\,$ & $\,\,$ 1 $\,\,$ & $\,\,$1$\,\,$ & $\,\,$9 $\,\,$ \\
$\,\,$ 1/2$\,\,$ & $\,\,$ 1 $\,\,$ & $\,\,$ 1 $\,\,$ & $\,\,$6 $\,\,$ \\
$\,\,$ 1/9$\,\,$ & $\,\,$ 1/9$\,\,$ & $\,\,$ 1/6$\,\,$ & $\,\,$ 1  $\,\,$ \\
\end{pmatrix},
\qquad
\lambda_{\max} =
4.1031,
\qquad
CR = 0.0389
\end{equation*}

\begin{equation*}
\mathbf{w}^{cos} =
\begin{pmatrix}
0.481343\\
0.244587\\
\color{red} 0.234023\color{black} \\
0.040048
\end{pmatrix}\end{equation*}
\begin{equation*}
\left[ \frac{{w}^{cos}_i}{{w}^{cos}_j} \right] =
\begin{pmatrix}
$\,\,$ 1 $\,\,$ & $\,\,$1.9680$\,\,$ & $\,\,$\color{red} 2.0568\color{black} $\,\,$ & $\,\,$12.0193$\,\,$ \\
$\,\,$0.5081$\,\,$ & $\,\,$ 1 $\,\,$ & $\,\,$\color{red} 1.0451\color{black} $\,\,$ & $\,\,$6.1074  $\,\,$ \\
$\,\,$\color{red} 0.4862\color{black} $\,\,$ & $\,\,$\color{red} 0.9568\color{black} $\,\,$ & $\,\,$ 1 $\,\,$ & $\,\,$\color{red} 5.8436\color{black}  $\,\,$ \\
$\,\,$0.0832$\,\,$ & $\,\,$0.1637$\,\,$ & $\,\,$\color{red} 0.1711\color{black} $\,\,$ & $\,\,$ 1  $\,\,$ \\
\end{pmatrix},
\end{equation*}

\begin{equation*}
\mathbf{w}^{\prime} =
\begin{pmatrix}
0.478347\\
0.243065\\
0.238790\\
0.039798
\end{pmatrix} =
0.993776\cdot
\begin{pmatrix}
0.481343\\
0.244587\\
\color{gr} 0.240286\color{black} \\
0.040048
\end{pmatrix},
\end{equation*}
\begin{equation*}
\left[ \frac{{w}^{\prime}_i}{{w}^{\prime}_j} \right] =
\begin{pmatrix}
$\,\,$ 1 $\,\,$ & $\,\,$1.9680$\,\,$ & $\,\,$\color{gr} 2.0032\color{black} $\,\,$ & $\,\,$12.0193$\,\,$ \\
$\,\,$0.5081$\,\,$ & $\,\,$ 1 $\,\,$ & $\,\,$\color{gr} 1.0179\color{black} $\,\,$ & $\,\,$6.1074  $\,\,$ \\
$\,\,$\color{gr} 0.4992\color{black} $\,\,$ & $\,\,$\color{gr} 0.9824\color{black} $\,\,$ & $\,\,$ 1 $\,\,$ & $\,\,$\color{gr} \color{blue} 6\color{black}  $\,\,$ \\
$\,\,$0.0832$\,\,$ & $\,\,$0.1637$\,\,$ & $\,\,$\color{gr} \color{blue}  1/6\color{black} $\,\,$ & $\,\,$ 1  $\,\,$ \\
\end{pmatrix},
\end{equation*}
\end{example}
\newpage
\begin{example}
\begin{equation*}
\mathbf{A} =
\begin{pmatrix}
$\,\,$ 1 $\,\,$ & $\,\,$3$\,\,$ & $\,\,$2$\,\,$ & $\,\,$9 $\,\,$ \\
$\,\,$ 1/3$\,\,$ & $\,\,$ 1 $\,\,$ & $\,\,$3$\,\,$ & $\,\,$5 $\,\,$ \\
$\,\,$ 1/2$\,\,$ & $\,\,$ 1/3$\,\,$ & $\,\,$ 1 $\,\,$ & $\,\,$3 $\,\,$ \\
$\,\,$ 1/9$\,\,$ & $\,\,$ 1/5$\,\,$ & $\,\,$ 1/3$\,\,$ & $\,\,$ 1  $\,\,$ \\
\end{pmatrix},
\qquad
\lambda_{\max} =
4.1966,
\qquad
CR = 0.0741
\end{equation*}

\begin{equation*}
\mathbf{w}^{cos} =
\begin{pmatrix}
0.491194\\
0.289032\\
0.167102\\
\color{red} 0.052672\color{black}
\end{pmatrix}\end{equation*}
\begin{equation*}
\left[ \frac{{w}^{cos}_i}{{w}^{cos}_j} \right] =
\begin{pmatrix}
$\,\,$ 1 $\,\,$ & $\,\,$1.6994$\,\,$ & $\,\,$2.9395$\,\,$ & $\,\,$\color{red} 9.3256\color{black} $\,\,$ \\
$\,\,$0.5884$\,\,$ & $\,\,$ 1 $\,\,$ & $\,\,$1.7297$\,\,$ & $\,\,$\color{red} 5.4874\color{black}   $\,\,$ \\
$\,\,$0.3402$\,\,$ & $\,\,$0.5781$\,\,$ & $\,\,$ 1 $\,\,$ & $\,\,$\color{red} 3.1725\color{black}  $\,\,$ \\
$\,\,$\color{red} 0.1072\color{black} $\,\,$ & $\,\,$\color{red} 0.1822\color{black} $\,\,$ & $\,\,$\color{red} 0.3152\color{black} $\,\,$ & $\,\,$ 1  $\,\,$ \\
\end{pmatrix},
\end{equation*}

\begin{equation*}
\mathbf{w}^{\prime} =
\begin{pmatrix}
0.490260\\
0.288483\\
0.166784\\
0.054473
\end{pmatrix} =
0.998098\cdot
\begin{pmatrix}
0.491194\\
0.289032\\
0.167102\\
\color{gr} 0.054577\color{black}
\end{pmatrix},
\end{equation*}
\begin{equation*}
\left[ \frac{{w}^{\prime}_i}{{w}^{\prime}_j} \right] =
\begin{pmatrix}
$\,\,$ 1 $\,\,$ & $\,\,$1.6994$\,\,$ & $\,\,$2.9395$\,\,$ & $\,\,$\color{gr} \color{blue} 9\color{black} $\,\,$ \\
$\,\,$0.5884$\,\,$ & $\,\,$ 1 $\,\,$ & $\,\,$1.7297$\,\,$ & $\,\,$\color{gr} 5.2959\color{black}   $\,\,$ \\
$\,\,$0.3402$\,\,$ & $\,\,$0.5781$\,\,$ & $\,\,$ 1 $\,\,$ & $\,\,$\color{gr} 3.0618\color{black}  $\,\,$ \\
$\,\,$\color{gr} \color{blue}  1/9\color{black} $\,\,$ & $\,\,$\color{gr} 0.1888\color{black} $\,\,$ & $\,\,$\color{gr} 0.3266\color{black} $\,\,$ & $\,\,$ 1  $\,\,$ \\
\end{pmatrix},
\end{equation*}
\end{example}
\newpage
\begin{example}
\begin{equation*}
\mathbf{A} =
\begin{pmatrix}
$\,\,$ 1 $\,\,$ & $\,\,$3$\,\,$ & $\,\,$3$\,\,$ & $\,\,$4 $\,\,$ \\
$\,\,$ 1/3$\,\,$ & $\,\,$ 1 $\,\,$ & $\,\,$2$\,\,$ & $\,\,$7 $\,\,$ \\
$\,\,$ 1/3$\,\,$ & $\,\,$ 1/2$\,\,$ & $\,\,$ 1 $\,\,$ & $\,\,$2 $\,\,$ \\
$\,\,$ 1/4$\,\,$ & $\,\,$ 1/7$\,\,$ & $\,\,$ 1/2$\,\,$ & $\,\,$ 1  $\,\,$ \\
\end{pmatrix},
\qquad
\lambda_{\max} =
4.2421,
\qquad
CR = 0.0913
\end{equation*}

\begin{equation*}
\mathbf{w}^{cos} =
\begin{pmatrix}
0.473023\\
0.301853\\
\color{red} 0.145993\color{black} \\
0.079131
\end{pmatrix}\end{equation*}
\begin{equation*}
\left[ \frac{{w}^{cos}_i}{{w}^{cos}_j} \right] =
\begin{pmatrix}
$\,\,$ 1 $\,\,$ & $\,\,$1.5671$\,\,$ & $\,\,$\color{red} 3.2400\color{black} $\,\,$ & $\,\,$5.9777$\,\,$ \\
$\,\,$0.6381$\,\,$ & $\,\,$ 1 $\,\,$ & $\,\,$\color{red} 2.0676\color{black} $\,\,$ & $\,\,$3.8146  $\,\,$ \\
$\,\,$\color{red} 0.3086\color{black} $\,\,$ & $\,\,$\color{red} 0.4837\color{black} $\,\,$ & $\,\,$ 1 $\,\,$ & $\,\,$\color{red} 1.8450\color{black}  $\,\,$ \\
$\,\,$0.1673$\,\,$ & $\,\,$0.2622$\,\,$ & $\,\,$\color{red} 0.5420\color{black} $\,\,$ & $\,\,$ 1  $\,\,$ \\
\end{pmatrix},
\end{equation*}

\begin{equation*}
\mathbf{w}^{\prime} =
\begin{pmatrix}
0.470700\\
0.300371\\
0.150186\\
0.078743
\end{pmatrix} =
0.995091\cdot
\begin{pmatrix}
0.473023\\
0.301853\\
\color{gr} 0.150927\color{black} \\
0.079131
\end{pmatrix},
\end{equation*}
\begin{equation*}
\left[ \frac{{w}^{\prime}_i}{{w}^{\prime}_j} \right] =
\begin{pmatrix}
$\,\,$ 1 $\,\,$ & $\,\,$1.5671$\,\,$ & $\,\,$\color{gr} 3.1341\color{black} $\,\,$ & $\,\,$5.9777$\,\,$ \\
$\,\,$0.6381$\,\,$ & $\,\,$ 1 $\,\,$ & $\,\,$\color{gr} \color{blue} 2\color{black} $\,\,$ & $\,\,$3.8146  $\,\,$ \\
$\,\,$\color{gr} 0.3191\color{black} $\,\,$ & $\,\,$\color{gr} \color{blue}  1/2\color{black} $\,\,$ & $\,\,$ 1 $\,\,$ & $\,\,$\color{gr} 1.9073\color{black}  $\,\,$ \\
$\,\,$0.1673$\,\,$ & $\,\,$0.2622$\,\,$ & $\,\,$\color{gr} 0.5243\color{black} $\,\,$ & $\,\,$ 1  $\,\,$ \\
\end{pmatrix},
\end{equation*}
\end{example}
\newpage
\begin{example}
\begin{equation*}
\mathbf{A} =
\begin{pmatrix}
$\,\,$ 1 $\,\,$ & $\,\,$3$\,\,$ & $\,\,$3$\,\,$ & $\,\,$4 $\,\,$ \\
$\,\,$ 1/3$\,\,$ & $\,\,$ 1 $\,\,$ & $\,\,$3$\,\,$ & $\,\,$2 $\,\,$ \\
$\,\,$ 1/3$\,\,$ & $\,\,$ 1/3$\,\,$ & $\,\,$ 1 $\,\,$ & $\,\,$1 $\,\,$ \\
$\,\,$ 1/4$\,\,$ & $\,\,$ 1/2$\,\,$ & $\,\,$ 1 $\,\,$ & $\,\,$ 1  $\,\,$ \\
\end{pmatrix},
\qquad
\lambda_{\max} =
4.1031,
\qquad
CR = 0.0389
\end{equation*}

\begin{equation*}
\mathbf{w}^{cos} =
\begin{pmatrix}
0.499159\\
0.254519\\
0.124787\\
\color{red} 0.121535\color{black}
\end{pmatrix}\end{equation*}
\begin{equation*}
\left[ \frac{{w}^{cos}_i}{{w}^{cos}_j} \right] =
\begin{pmatrix}
$\,\,$ 1 $\,\,$ & $\,\,$1.9612$\,\,$ & $\,\,$4.0001$\,\,$ & $\,\,$\color{red} 4.1071\color{black} $\,\,$ \\
$\,\,$0.5099$\,\,$ & $\,\,$ 1 $\,\,$ & $\,\,$2.0396$\,\,$ & $\,\,$\color{red} 2.0942\color{black}   $\,\,$ \\
$\,\,$0.2500$\,\,$ & $\,\,$0.4903$\,\,$ & $\,\,$ 1 $\,\,$ & $\,\,$\color{red} 1.0268\color{black}  $\,\,$ \\
$\,\,$\color{red} 0.2435\color{black} $\,\,$ & $\,\,$\color{red} 0.4775\color{black} $\,\,$ & $\,\,$\color{red} 0.9739\color{black} $\,\,$ & $\,\,$ 1  $\,\,$ \\
\end{pmatrix},
\end{equation*}

\begin{equation*}
\mathbf{w}^{\prime} =
\begin{pmatrix}
0.497541\\
0.253695\\
0.124382\\
0.124382
\end{pmatrix} =
0.996759\cdot
\begin{pmatrix}
0.499159\\
0.254519\\
0.124787\\
\color{gr} 0.124787\color{black}
\end{pmatrix},
\end{equation*}
\begin{equation*}
\left[ \frac{{w}^{\prime}_i}{{w}^{\prime}_j} \right] =
\begin{pmatrix}
$\,\,$ 1 $\,\,$ & $\,\,$1.9612$\,\,$ & $\,\,$4.0001$\,\,$ & $\,\,$\color{gr} 4.0001\color{black} $\,\,$ \\
$\,\,$0.5099$\,\,$ & $\,\,$ 1 $\,\,$ & $\,\,$2.0396$\,\,$ & $\,\,$\color{gr} 2.0396\color{black}   $\,\,$ \\
$\,\,$0.2500$\,\,$ & $\,\,$0.4903$\,\,$ & $\,\,$ 1 $\,\,$ & $\,\,$\color{gr} \color{blue} 1\color{black}  $\,\,$ \\
$\,\,$\color{gr} 0.2500\color{black} $\,\,$ & $\,\,$\color{gr} 0.4903\color{black} $\,\,$ & $\,\,$\color{gr} \color{blue} 1\color{black} $\,\,$ & $\,\,$ 1  $\,\,$ \\
\end{pmatrix},
\end{equation*}
\end{example}
\newpage
\begin{example}
\begin{equation*}
\mathbf{A} =
\begin{pmatrix}
$\,\,$ 1 $\,\,$ & $\,\,$3$\,\,$ & $\,\,$3$\,\,$ & $\,\,$8 $\,\,$ \\
$\,\,$ 1/3$\,\,$ & $\,\,$ 1 $\,\,$ & $\,\,$4$\,\,$ & $\,\,$5 $\,\,$ \\
$\,\,$ 1/3$\,\,$ & $\,\,$ 1/4$\,\,$ & $\,\,$ 1 $\,\,$ & $\,\,$2 $\,\,$ \\
$\,\,$ 1/8$\,\,$ & $\,\,$ 1/5$\,\,$ & $\,\,$ 1/2$\,\,$ & $\,\,$ 1  $\,\,$ \\
\end{pmatrix},
\qquad
\lambda_{\max} =
4.1689,
\qquad
CR = 0.0637
\end{equation*}

\begin{equation*}
\mathbf{w}^{cos} =
\begin{pmatrix}
0.514792\\
0.302332\\
0.123379\\
\color{red} 0.059497\color{black}
\end{pmatrix}\end{equation*}
\begin{equation*}
\left[ \frac{{w}^{cos}_i}{{w}^{cos}_j} \right] =
\begin{pmatrix}
$\,\,$ 1 $\,\,$ & $\,\,$1.7027$\,\,$ & $\,\,$4.1724$\,\,$ & $\,\,$\color{red} 8.6523\color{black} $\,\,$ \\
$\,\,$0.5873$\,\,$ & $\,\,$ 1 $\,\,$ & $\,\,$2.4504$\,\,$ & $\,\,$\color{red} 5.0814\color{black}   $\,\,$ \\
$\,\,$0.2397$\,\,$ & $\,\,$0.4081$\,\,$ & $\,\,$ 1 $\,\,$ & $\,\,$\color{red} 2.0737\color{black}  $\,\,$ \\
$\,\,$\color{red} 0.1156\color{black} $\,\,$ & $\,\,$\color{red} 0.1968\color{black} $\,\,$ & $\,\,$\color{red} 0.4822\color{black} $\,\,$ & $\,\,$ 1  $\,\,$ \\
\end{pmatrix},
\end{equation*}

\begin{equation*}
\mathbf{w}^{\prime} =
\begin{pmatrix}
0.514294\\
0.302039\\
0.123260\\
0.060408
\end{pmatrix} =
0.999032\cdot
\begin{pmatrix}
0.514792\\
0.302332\\
0.123379\\
\color{gr} 0.060466\color{black}
\end{pmatrix},
\end{equation*}
\begin{equation*}
\left[ \frac{{w}^{\prime}_i}{{w}^{\prime}_j} \right] =
\begin{pmatrix}
$\,\,$ 1 $\,\,$ & $\,\,$1.7027$\,\,$ & $\,\,$4.1724$\,\,$ & $\,\,$\color{gr} 8.5137\color{black} $\,\,$ \\
$\,\,$0.5873$\,\,$ & $\,\,$ 1 $\,\,$ & $\,\,$2.4504$\,\,$ & $\,\,$\color{gr} \color{blue} 5\color{black}   $\,\,$ \\
$\,\,$0.2397$\,\,$ & $\,\,$0.4081$\,\,$ & $\,\,$ 1 $\,\,$ & $\,\,$\color{gr} 2.0405\color{black}  $\,\,$ \\
$\,\,$\color{gr} 0.1175\color{black} $\,\,$ & $\,\,$\color{gr} \color{blue}  1/5\color{black} $\,\,$ & $\,\,$\color{gr} 0.4901\color{black} $\,\,$ & $\,\,$ 1  $\,\,$ \\
\end{pmatrix},
\end{equation*}
\end{example}
\newpage
\begin{example}
\begin{equation*}
\mathbf{A} =
\begin{pmatrix}
$\,\,$ 1 $\,\,$ & $\,\,$3$\,\,$ & $\,\,$3$\,\,$ & $\,\,$8 $\,\,$ \\
$\,\,$ 1/3$\,\,$ & $\,\,$ 1 $\,\,$ & $\,\,$5$\,\,$ & $\,\,$5 $\,\,$ \\
$\,\,$ 1/3$\,\,$ & $\,\,$ 1/5$\,\,$ & $\,\,$ 1 $\,\,$ & $\,\,$2 $\,\,$ \\
$\,\,$ 1/8$\,\,$ & $\,\,$ 1/5$\,\,$ & $\,\,$ 1/2$\,\,$ & $\,\,$ 1  $\,\,$ \\
\end{pmatrix},
\qquad
\lambda_{\max} =
4.2311,
\qquad
CR = 0.0871
\end{equation*}

\begin{equation*}
\mathbf{w}^{cos} =
\begin{pmatrix}
0.508140\\
0.315762\\
0.118034\\
\color{red} 0.058064\color{black}
\end{pmatrix}\end{equation*}
\begin{equation*}
\left[ \frac{{w}^{cos}_i}{{w}^{cos}_j} \right] =
\begin{pmatrix}
$\,\,$ 1 $\,\,$ & $\,\,$1.6093$\,\,$ & $\,\,$4.3050$\,\,$ & $\,\,$\color{red} 8.7514\color{black} $\,\,$ \\
$\,\,$0.6214$\,\,$ & $\,\,$ 1 $\,\,$ & $\,\,$2.6752$\,\,$ & $\,\,$\color{red} 5.4382\color{black}   $\,\,$ \\
$\,\,$0.2323$\,\,$ & $\,\,$0.3738$\,\,$ & $\,\,$ 1 $\,\,$ & $\,\,$\color{red} 2.0328\color{black}  $\,\,$ \\
$\,\,$\color{red} 0.1143\color{black} $\,\,$ & $\,\,$\color{red} 0.1839\color{black} $\,\,$ & $\,\,$\color{red} 0.4919\color{black} $\,\,$ & $\,\,$ 1  $\,\,$ \\
\end{pmatrix},
\end{equation*}

\begin{equation*}
\mathbf{w}^{\prime} =
\begin{pmatrix}
0.507656\\
0.315461\\
0.117922\\
0.058961
\end{pmatrix} =
0.999048\cdot
\begin{pmatrix}
0.508140\\
0.315762\\
0.118034\\
\color{gr} 0.059017\color{black}
\end{pmatrix},
\end{equation*}
\begin{equation*}
\left[ \frac{{w}^{\prime}_i}{{w}^{\prime}_j} \right] =
\begin{pmatrix}
$\,\,$ 1 $\,\,$ & $\,\,$1.6093$\,\,$ & $\,\,$4.3050$\,\,$ & $\,\,$\color{gr} 8.6101\color{black} $\,\,$ \\
$\,\,$0.6214$\,\,$ & $\,\,$ 1 $\,\,$ & $\,\,$2.6752$\,\,$ & $\,\,$\color{gr} 5.3503\color{black}   $\,\,$ \\
$\,\,$0.2323$\,\,$ & $\,\,$0.3738$\,\,$ & $\,\,$ 1 $\,\,$ & $\,\,$\color{gr} \color{blue} 2\color{black}  $\,\,$ \\
$\,\,$\color{gr} 0.1161\color{black} $\,\,$ & $\,\,$\color{gr} 0.1869\color{black} $\,\,$ & $\,\,$\color{gr} \color{blue}  1/2\color{black} $\,\,$ & $\,\,$ 1  $\,\,$ \\
\end{pmatrix},
\end{equation*}
\end{example}
\newpage
\begin{example}
\begin{equation*}
\mathbf{A} =
\begin{pmatrix}
$\,\,$ 1 $\,\,$ & $\,\,$3$\,\,$ & $\,\,$3$\,\,$ & $\,\,$9 $\,\,$ \\
$\,\,$ 1/3$\,\,$ & $\,\,$ 1 $\,\,$ & $\,\,$2$\,\,$ & $\,\,$2 $\,\,$ \\
$\,\,$ 1/3$\,\,$ & $\,\,$ 1/2$\,\,$ & $\,\,$ 1 $\,\,$ & $\,\,$5 $\,\,$ \\
$\,\,$ 1/9$\,\,$ & $\,\,$ 1/2$\,\,$ & $\,\,$ 1/5$\,\,$ & $\,\,$ 1  $\,\,$ \\
\end{pmatrix},
\qquad
\lambda_{\max} =
4.2277,
\qquad
CR = 0.0859
\end{equation*}

\begin{equation*}
\mathbf{w}^{cos} =
\begin{pmatrix}
\color{red} 0.542905\color{black} \\
0.207790\\
0.186510\\
0.062795
\end{pmatrix}\end{equation*}
\begin{equation*}
\left[ \frac{{w}^{cos}_i}{{w}^{cos}_j} \right] =
\begin{pmatrix}
$\,\,$ 1 $\,\,$ & $\,\,$\color{red} 2.6128\color{black} $\,\,$ & $\,\,$\color{red} 2.9109\color{black} $\,\,$ & $\,\,$\color{red} 8.6457\color{black} $\,\,$ \\
$\,\,$\color{red} 0.3827\color{black} $\,\,$ & $\,\,$ 1 $\,\,$ & $\,\,$1.1141$\,\,$ & $\,\,$3.3090  $\,\,$ \\
$\,\,$\color{red} 0.3435\color{black} $\,\,$ & $\,\,$0.8976$\,\,$ & $\,\,$ 1 $\,\,$ & $\,\,$2.9701 $\,\,$ \\
$\,\,$\color{red} 0.1157\color{black} $\,\,$ & $\,\,$0.3022$\,\,$ & $\,\,$0.3367$\,\,$ & $\,\,$ 1  $\,\,$ \\
\end{pmatrix},
\end{equation*}

\begin{equation*}
\mathbf{w}^{\prime} =
\begin{pmatrix}
0.550379\\
0.204393\\
0.183460\\
0.061768
\end{pmatrix} =
0.983647\cdot
\begin{pmatrix}
\color{gr} 0.559529\color{black} \\
0.207790\\
0.186510\\
0.062795
\end{pmatrix},
\end{equation*}
\begin{equation*}
\left[ \frac{{w}^{\prime}_i}{{w}^{\prime}_j} \right] =
\begin{pmatrix}
$\,\,$ 1 $\,\,$ & $\,\,$\color{gr} 2.6928\color{black} $\,\,$ & $\,\,$\color{gr} \color{blue} 3\color{black} $\,\,$ & $\,\,$\color{gr} 8.9104\color{black} $\,\,$ \\
$\,\,$\color{gr} 0.3714\color{black} $\,\,$ & $\,\,$ 1 $\,\,$ & $\,\,$1.1141$\,\,$ & $\,\,$3.3090  $\,\,$ \\
$\,\,$\color{gr} \color{blue}  1/3\color{black} $\,\,$ & $\,\,$0.8976$\,\,$ & $\,\,$ 1 $\,\,$ & $\,\,$2.9701 $\,\,$ \\
$\,\,$\color{gr} 0.1122\color{black} $\,\,$ & $\,\,$0.3022$\,\,$ & $\,\,$0.3367$\,\,$ & $\,\,$ 1  $\,\,$ \\
\end{pmatrix},
\end{equation*}
\end{example}
\newpage
\begin{example}
\begin{equation*}
\mathbf{A} =
\begin{pmatrix}
$\,\,$ 1 $\,\,$ & $\,\,$3$\,\,$ & $\,\,$3$\,\,$ & $\,\,$9 $\,\,$ \\
$\,\,$ 1/3$\,\,$ & $\,\,$ 1 $\,\,$ & $\,\,$4$\,\,$ & $\,\,$5 $\,\,$ \\
$\,\,$ 1/3$\,\,$ & $\,\,$ 1/4$\,\,$ & $\,\,$ 1 $\,\,$ & $\,\,$2 $\,\,$ \\
$\,\,$ 1/9$\,\,$ & $\,\,$ 1/5$\,\,$ & $\,\,$ 1/2$\,\,$ & $\,\,$ 1  $\,\,$ \\
\end{pmatrix},
\qquad
\lambda_{\max} =
4.1655,
\qquad
CR = 0.0624
\end{equation*}

\begin{equation*}
\mathbf{w}^{cos} =
\begin{pmatrix}
0.523485\\
0.298106\\
0.121745\\
\color{red} 0.056663\color{black}
\end{pmatrix}\end{equation*}
\begin{equation*}
\left[ \frac{{w}^{cos}_i}{{w}^{cos}_j} \right] =
\begin{pmatrix}
$\,\,$ 1 $\,\,$ & $\,\,$1.7560$\,\,$ & $\,\,$4.2998$\,\,$ & $\,\,$\color{red} 9.2385\color{black} $\,\,$ \\
$\,\,$0.5695$\,\,$ & $\,\,$ 1 $\,\,$ & $\,\,$2.4486$\,\,$ & $\,\,$\color{red} 5.2610\color{black}   $\,\,$ \\
$\,\,$0.2326$\,\,$ & $\,\,$0.4084$\,\,$ & $\,\,$ 1 $\,\,$ & $\,\,$\color{red} 2.1486\color{black}  $\,\,$ \\
$\,\,$\color{red} 0.1082\color{black} $\,\,$ & $\,\,$\color{red} 0.1901\color{black} $\,\,$ & $\,\,$\color{red} 0.4654\color{black} $\,\,$ & $\,\,$ 1  $\,\,$ \\
\end{pmatrix},
\end{equation*}

\begin{equation*}
\mathbf{w}^{\prime} =
\begin{pmatrix}
0.522700\\
0.297659\\
0.121563\\
0.058078
\end{pmatrix} =
0.998501\cdot
\begin{pmatrix}
0.523485\\
0.298106\\
0.121745\\
\color{gr} 0.058165\color{black}
\end{pmatrix},
\end{equation*}
\begin{equation*}
\left[ \frac{{w}^{\prime}_i}{{w}^{\prime}_j} \right] =
\begin{pmatrix}
$\,\,$ 1 $\,\,$ & $\,\,$1.7560$\,\,$ & $\,\,$4.2998$\,\,$ & $\,\,$\color{gr} \color{blue} 9\color{black} $\,\,$ \\
$\,\,$0.5695$\,\,$ & $\,\,$ 1 $\,\,$ & $\,\,$2.4486$\,\,$ & $\,\,$\color{gr} 5.1252\color{black}   $\,\,$ \\
$\,\,$0.2326$\,\,$ & $\,\,$0.4084$\,\,$ & $\,\,$ 1 $\,\,$ & $\,\,$\color{gr} 2.0931\color{black}  $\,\,$ \\
$\,\,$\color{gr} \color{blue}  1/9\color{black} $\,\,$ & $\,\,$\color{gr} 0.1951\color{black} $\,\,$ & $\,\,$\color{gr} 0.4778\color{black} $\,\,$ & $\,\,$ 1  $\,\,$ \\
\end{pmatrix},
\end{equation*}
\end{example}
\newpage
\begin{example}
\begin{equation*}
\mathbf{A} =
\begin{pmatrix}
$\,\,$ 1 $\,\,$ & $\,\,$3$\,\,$ & $\,\,$3$\,\,$ & $\,\,$9 $\,\,$ \\
$\,\,$ 1/3$\,\,$ & $\,\,$ 1 $\,\,$ & $\,\,$5$\,\,$ & $\,\,$5 $\,\,$ \\
$\,\,$ 1/3$\,\,$ & $\,\,$ 1/5$\,\,$ & $\,\,$ 1 $\,\,$ & $\,\,$2 $\,\,$ \\
$\,\,$ 1/9$\,\,$ & $\,\,$ 1/5$\,\,$ & $\,\,$ 1/2$\,\,$ & $\,\,$ 1  $\,\,$ \\
\end{pmatrix},
\qquad
\lambda_{\max} =
4.2277,
\qquad
CR = 0.0859
\end{equation*}

\begin{equation*}
\mathbf{w}^{cos} =
\begin{pmatrix}
0.516885\\
0.311589\\
0.116340\\
\color{red} 0.055186\color{black}
\end{pmatrix}\end{equation*}
\begin{equation*}
\left[ \frac{{w}^{cos}_i}{{w}^{cos}_j} \right] =
\begin{pmatrix}
$\,\,$ 1 $\,\,$ & $\,\,$1.6589$\,\,$ & $\,\,$4.4429$\,\,$ & $\,\,$\color{red} 9.3663\color{black} $\,\,$ \\
$\,\,$0.6028$\,\,$ & $\,\,$ 1 $\,\,$ & $\,\,$2.6782$\,\,$ & $\,\,$\color{red} 5.6462\color{black}   $\,\,$ \\
$\,\,$0.2251$\,\,$ & $\,\,$0.3734$\,\,$ & $\,\,$ 1 $\,\,$ & $\,\,$\color{red} 2.1082\color{black}  $\,\,$ \\
$\,\,$\color{red} 0.1068\color{black} $\,\,$ & $\,\,$\color{red} 0.1771\color{black} $\,\,$ & $\,\,$\color{red} 0.4743\color{black} $\,\,$ & $\,\,$ 1  $\,\,$ \\
\end{pmatrix},
\end{equation*}

\begin{equation*}
\mathbf{w}^{\prime} =
\begin{pmatrix}
0.515727\\
0.310890\\
0.116080\\
0.057303
\end{pmatrix} =
0.997759\cdot
\begin{pmatrix}
0.516885\\
0.311589\\
0.116340\\
\color{gr} 0.057432\color{black}
\end{pmatrix},
\end{equation*}
\begin{equation*}
\left[ \frac{{w}^{\prime}_i}{{w}^{\prime}_j} \right] =
\begin{pmatrix}
$\,\,$ 1 $\,\,$ & $\,\,$1.6589$\,\,$ & $\,\,$4.4429$\,\,$ & $\,\,$\color{gr} \color{blue} 9\color{black} $\,\,$ \\
$\,\,$0.6028$\,\,$ & $\,\,$ 1 $\,\,$ & $\,\,$2.6782$\,\,$ & $\,\,$\color{gr} 5.4254\color{black}   $\,\,$ \\
$\,\,$0.2251$\,\,$ & $\,\,$0.3734$\,\,$ & $\,\,$ 1 $\,\,$ & $\,\,$\color{gr} 2.0257\color{black}  $\,\,$ \\
$\,\,$\color{gr} \color{blue}  1/9\color{black} $\,\,$ & $\,\,$\color{gr} 0.1843\color{black} $\,\,$ & $\,\,$\color{gr} 0.4937\color{black} $\,\,$ & $\,\,$ 1  $\,\,$ \\
\end{pmatrix},
\end{equation*}
\end{example}
\newpage
\begin{example}
\begin{equation*}
\mathbf{A} =
\begin{pmatrix}
$\,\,$ 1 $\,\,$ & $\,\,$3$\,\,$ & $\,\,$3$\,\,$ & $\,\,$9 $\,\,$ \\
$\,\,$ 1/3$\,\,$ & $\,\,$ 1 $\,\,$ & $\,\,$5$\,\,$ & $\,\,$6 $\,\,$ \\
$\,\,$ 1/3$\,\,$ & $\,\,$ 1/5$\,\,$ & $\,\,$ 1 $\,\,$ & $\,\,$2 $\,\,$ \\
$\,\,$ 1/9$\,\,$ & $\,\,$ 1/6$\,\,$ & $\,\,$ 1/2$\,\,$ & $\,\,$ 1  $\,\,$ \\
\end{pmatrix},
\qquad
\lambda_{\max} =
4.2277,
\qquad
CR = 0.0859
\end{equation*}

\begin{equation*}
\mathbf{w}^{cos} =
\begin{pmatrix}
0.510092\\
0.322357\\
0.114815\\
\color{red} 0.052736\color{black}
\end{pmatrix}\end{equation*}
\begin{equation*}
\left[ \frac{{w}^{cos}_i}{{w}^{cos}_j} \right] =
\begin{pmatrix}
$\,\,$ 1 $\,\,$ & $\,\,$1.5824$\,\,$ & $\,\,$4.4427$\,\,$ & $\,\,$\color{red} 9.6725\color{black} $\,\,$ \\
$\,\,$0.6320$\,\,$ & $\,\,$ 1 $\,\,$ & $\,\,$2.8076$\,\,$ & $\,\,$\color{red} 6.1126\color{black}   $\,\,$ \\
$\,\,$0.2251$\,\,$ & $\,\,$0.3562$\,\,$ & $\,\,$ 1 $\,\,$ & $\,\,$\color{red} 2.1772\color{black}  $\,\,$ \\
$\,\,$\color{red} 0.1034\color{black} $\,\,$ & $\,\,$\color{red} 0.1636\color{black} $\,\,$ & $\,\,$\color{red} 0.4593\color{black} $\,\,$ & $\,\,$ 1  $\,\,$ \\
\end{pmatrix},
\end{equation*}

\begin{equation*}
\mathbf{w}^{\prime} =
\begin{pmatrix}
0.509587\\
0.322038\\
0.114701\\
0.053673
\end{pmatrix} =
0.999011\cdot
\begin{pmatrix}
0.510092\\
0.322357\\
0.114815\\
\color{gr} 0.053726\color{black}
\end{pmatrix},
\end{equation*}
\begin{equation*}
\left[ \frac{{w}^{\prime}_i}{{w}^{\prime}_j} \right] =
\begin{pmatrix}
$\,\,$ 1 $\,\,$ & $\,\,$1.5824$\,\,$ & $\,\,$4.4427$\,\,$ & $\,\,$\color{gr} 9.4943\color{black} $\,\,$ \\
$\,\,$0.6320$\,\,$ & $\,\,$ 1 $\,\,$ & $\,\,$2.8076$\,\,$ & $\,\,$\color{gr} \color{blue} 6\color{black}   $\,\,$ \\
$\,\,$0.2251$\,\,$ & $\,\,$0.3562$\,\,$ & $\,\,$ 1 $\,\,$ & $\,\,$\color{gr} 2.1370\color{black}  $\,\,$ \\
$\,\,$\color{gr} 0.1053\color{black} $\,\,$ & $\,\,$\color{gr} \color{blue}  1/6\color{black} $\,\,$ & $\,\,$\color{gr} 0.4679\color{black} $\,\,$ & $\,\,$ 1  $\,\,$ \\
\end{pmatrix},
\end{equation*}
\end{example}
\newpage
\begin{example}
\begin{equation*}
\mathbf{A} =
\begin{pmatrix}
$\,\,$ 1 $\,\,$ & $\,\,$3$\,\,$ & $\,\,$4$\,\,$ & $\,\,$5 $\,\,$ \\
$\,\,$ 1/3$\,\,$ & $\,\,$ 1 $\,\,$ & $\,\,$2$\,\,$ & $\,\,$8 $\,\,$ \\
$\,\,$ 1/4$\,\,$ & $\,\,$ 1/2$\,\,$ & $\,\,$ 1 $\,\,$ & $\,\,$2 $\,\,$ \\
$\,\,$ 1/5$\,\,$ & $\,\,$ 1/8$\,\,$ & $\,\,$ 1/2$\,\,$ & $\,\,$ 1  $\,\,$ \\
\end{pmatrix},
\qquad
\lambda_{\max} =
4.2162,
\qquad
CR = 0.0815
\end{equation*}

\begin{equation*}
\mathbf{w}^{cos} =
\begin{pmatrix}
0.509105\\
0.295555\\
\color{red} 0.127170\color{black} \\
0.068170
\end{pmatrix}\end{equation*}
\begin{equation*}
\left[ \frac{{w}^{cos}_i}{{w}^{cos}_j} \right] =
\begin{pmatrix}
$\,\,$ 1 $\,\,$ & $\,\,$1.7225$\,\,$ & $\,\,$\color{red} 4.0033\color{black} $\,\,$ & $\,\,$7.4681$\,\,$ \\
$\,\,$0.5805$\,\,$ & $\,\,$ 1 $\,\,$ & $\,\,$\color{red} 2.3241\color{black} $\,\,$ & $\,\,$4.3355  $\,\,$ \\
$\,\,$\color{red} 0.2498\color{black} $\,\,$ & $\,\,$\color{red} 0.4303\color{black} $\,\,$ & $\,\,$ 1 $\,\,$ & $\,\,$\color{red} 1.8655\color{black}  $\,\,$ \\
$\,\,$0.1339$\,\,$ & $\,\,$0.2307$\,\,$ & $\,\,$\color{red} 0.5361\color{black} $\,\,$ & $\,\,$ 1  $\,\,$ \\
\end{pmatrix},
\end{equation*}

\begin{equation*}
\mathbf{w}^{\prime} =
\begin{pmatrix}
0.509051\\
0.295524\\
0.127263\\
0.068163
\end{pmatrix} =
0.999894\cdot
\begin{pmatrix}
0.509105\\
0.295555\\
\color{gr} 0.127276\color{black} \\
0.068170
\end{pmatrix},
\end{equation*}
\begin{equation*}
\left[ \frac{{w}^{\prime}_i}{{w}^{\prime}_j} \right] =
\begin{pmatrix}
$\,\,$ 1 $\,\,$ & $\,\,$1.7225$\,\,$ & $\,\,$\color{gr} \color{blue} 4\color{black} $\,\,$ & $\,\,$7.4681$\,\,$ \\
$\,\,$0.5805$\,\,$ & $\,\,$ 1 $\,\,$ & $\,\,$\color{gr} 2.3222\color{black} $\,\,$ & $\,\,$4.3355  $\,\,$ \\
$\,\,$\color{gr} \color{blue}  1/4\color{black} $\,\,$ & $\,\,$\color{gr} 0.4306\color{black} $\,\,$ & $\,\,$ 1 $\,\,$ & $\,\,$\color{gr} 1.8670\color{black}  $\,\,$ \\
$\,\,$0.1339$\,\,$ & $\,\,$0.2307$\,\,$ & $\,\,$\color{gr} 0.5356\color{black} $\,\,$ & $\,\,$ 1  $\,\,$ \\
\end{pmatrix},
\end{equation*}
\end{example}
\newpage
\begin{example}
\begin{equation*}
\mathbf{A} =
\begin{pmatrix}
$\,\,$ 1 $\,\,$ & $\,\,$3$\,\,$ & $\,\,$4$\,\,$ & $\,\,$5 $\,\,$ \\
$\,\,$ 1/3$\,\,$ & $\,\,$ 1 $\,\,$ & $\,\,$2$\,\,$ & $\,\,$9 $\,\,$ \\
$\,\,$ 1/4$\,\,$ & $\,\,$ 1/2$\,\,$ & $\,\,$ 1 $\,\,$ & $\,\,$2 $\,\,$ \\
$\,\,$ 1/5$\,\,$ & $\,\,$ 1/9$\,\,$ & $\,\,$ 1/2$\,\,$ & $\,\,$ 1  $\,\,$ \\
\end{pmatrix},
\qquad
\lambda_{\max} =
4.2541,
\qquad
CR = 0.0958
\end{equation*}

\begin{equation*}
\mathbf{w}^{cos} =
\begin{pmatrix}
0.505897\\
0.302098\\
\color{red} 0.125386\color{black} \\
0.066618
\end{pmatrix}\end{equation*}
\begin{equation*}
\left[ \frac{{w}^{cos}_i}{{w}^{cos}_j} \right] =
\begin{pmatrix}
$\,\,$ 1 $\,\,$ & $\,\,$1.6746$\,\,$ & $\,\,$\color{red} 4.0347\color{black} $\,\,$ & $\,\,$7.5939$\,\,$ \\
$\,\,$0.5972$\,\,$ & $\,\,$ 1 $\,\,$ & $\,\,$\color{red} 2.4093\color{black} $\,\,$ & $\,\,$4.5348  $\,\,$ \\
$\,\,$\color{red} 0.2478\color{black} $\,\,$ & $\,\,$\color{red} 0.4151\color{black} $\,\,$ & $\,\,$ 1 $\,\,$ & $\,\,$\color{red} 1.8822\color{black}  $\,\,$ \\
$\,\,$0.1317$\,\,$ & $\,\,$0.2205$\,\,$ & $\,\,$\color{red} 0.5313\color{black} $\,\,$ & $\,\,$ 1  $\,\,$ \\
\end{pmatrix},
\end{equation*}

\begin{equation*}
\mathbf{w}^{\prime} =
\begin{pmatrix}
0.505347\\
0.301770\\
0.126337\\
0.066546
\end{pmatrix} =
0.998913\cdot
\begin{pmatrix}
0.505897\\
0.302098\\
\color{gr} 0.126474\color{black} \\
0.066618
\end{pmatrix},
\end{equation*}
\begin{equation*}
\left[ \frac{{w}^{\prime}_i}{{w}^{\prime}_j} \right] =
\begin{pmatrix}
$\,\,$ 1 $\,\,$ & $\,\,$1.6746$\,\,$ & $\,\,$\color{gr} \color{blue} 4\color{black} $\,\,$ & $\,\,$7.5939$\,\,$ \\
$\,\,$0.5972$\,\,$ & $\,\,$ 1 $\,\,$ & $\,\,$\color{gr} 2.3886\color{black} $\,\,$ & $\,\,$4.5348  $\,\,$ \\
$\,\,$\color{gr} \color{blue}  1/4\color{black} $\,\,$ & $\,\,$\color{gr} 0.4187\color{black} $\,\,$ & $\,\,$ 1 $\,\,$ & $\,\,$\color{gr} 1.8985\color{black}  $\,\,$ \\
$\,\,$0.1317$\,\,$ & $\,\,$0.2205$\,\,$ & $\,\,$\color{gr} 0.5267\color{black} $\,\,$ & $\,\,$ 1  $\,\,$ \\
\end{pmatrix},
\end{equation*}
\end{example}
\newpage
\begin{example}
\begin{equation*}
\mathbf{A} =
\begin{pmatrix}
$\,\,$ 1 $\,\,$ & $\,\,$3$\,\,$ & $\,\,$4$\,\,$ & $\,\,$5 $\,\,$ \\
$\,\,$ 1/3$\,\,$ & $\,\,$ 1 $\,\,$ & $\,\,$5$\,\,$ & $\,\,$3 $\,\,$ \\
$\,\,$ 1/4$\,\,$ & $\,\,$ 1/5$\,\,$ & $\,\,$ 1 $\,\,$ & $\,\,$1 $\,\,$ \\
$\,\,$ 1/5$\,\,$ & $\,\,$ 1/3$\,\,$ & $\,\,$ 1 $\,\,$ & $\,\,$ 1  $\,\,$ \\
\end{pmatrix},
\qquad
\lambda_{\max} =
4.1544,
\qquad
CR = 0.0582
\end{equation*}

\begin{equation*}
\mathbf{w}^{cos} =
\begin{pmatrix}
0.515622\\
0.294237\\
0.095348\\
\color{red} 0.094794\color{black}
\end{pmatrix}\end{equation*}
\begin{equation*}
\left[ \frac{{w}^{cos}_i}{{w}^{cos}_j} \right] =
\begin{pmatrix}
$\,\,$ 1 $\,\,$ & $\,\,$1.7524$\,\,$ & $\,\,$5.4078$\,\,$ & $\,\,$\color{red} 5.4394\color{black} $\,\,$ \\
$\,\,$0.5706$\,\,$ & $\,\,$ 1 $\,\,$ & $\,\,$3.0859$\,\,$ & $\,\,$\color{red} 3.1040\color{black}   $\,\,$ \\
$\,\,$0.1849$\,\,$ & $\,\,$0.3241$\,\,$ & $\,\,$ 1 $\,\,$ & $\,\,$\color{red} 1.0058\color{black}  $\,\,$ \\
$\,\,$\color{red} 0.1838\color{black} $\,\,$ & $\,\,$\color{red} 0.3222\color{black} $\,\,$ & $\,\,$\color{red} 0.9942\color{black} $\,\,$ & $\,\,$ 1  $\,\,$ \\
\end{pmatrix},
\end{equation*}

\begin{equation*}
\mathbf{w}^{\prime} =
\begin{pmatrix}
0.515336\\
0.294074\\
0.095295\\
0.095295
\end{pmatrix} =
0.999446\cdot
\begin{pmatrix}
0.515622\\
0.294237\\
0.095348\\
\color{gr} 0.095348\color{black}
\end{pmatrix},
\end{equation*}
\begin{equation*}
\left[ \frac{{w}^{\prime}_i}{{w}^{\prime}_j} \right] =
\begin{pmatrix}
$\,\,$ 1 $\,\,$ & $\,\,$1.7524$\,\,$ & $\,\,$5.4078$\,\,$ & $\,\,$\color{gr} 5.4078\color{black} $\,\,$ \\
$\,\,$0.5706$\,\,$ & $\,\,$ 1 $\,\,$ & $\,\,$3.0859$\,\,$ & $\,\,$\color{gr} 3.0859\color{black}   $\,\,$ \\
$\,\,$0.1849$\,\,$ & $\,\,$0.3241$\,\,$ & $\,\,$ 1 $\,\,$ & $\,\,$\color{gr} \color{blue} 1\color{black}  $\,\,$ \\
$\,\,$\color{gr} 0.1849\color{black} $\,\,$ & $\,\,$\color{gr} 0.3241\color{black} $\,\,$ & $\,\,$\color{gr} \color{blue} 1\color{black} $\,\,$ & $\,\,$ 1  $\,\,$ \\
\end{pmatrix},
\end{equation*}
\end{example}
\newpage
\begin{example}
\begin{equation*}
\mathbf{A} =
\begin{pmatrix}
$\,\,$ 1 $\,\,$ & $\,\,$3$\,\,$ & $\,\,$4$\,\,$ & $\,\,$6 $\,\,$ \\
$\,\,$ 1/3$\,\,$ & $\,\,$ 1 $\,\,$ & $\,\,$2$\,\,$ & $\,\,$6 $\,\,$ \\
$\,\,$ 1/4$\,\,$ & $\,\,$ 1/2$\,\,$ & $\,\,$ 1 $\,\,$ & $\,\,$2 $\,\,$ \\
$\,\,$ 1/6$\,\,$ & $\,\,$ 1/6$\,\,$ & $\,\,$ 1/2$\,\,$ & $\,\,$ 1  $\,\,$ \\
\end{pmatrix},
\qquad
\lambda_{\max} =
4.1031,
\qquad
CR = 0.0389
\end{equation*}

\begin{equation*}
\mathbf{w}^{cos} =
\begin{pmatrix}
0.532133\\
0.271367\\
\color{red} 0.129746\color{black} \\
0.066755
\end{pmatrix}\end{equation*}
\begin{equation*}
\left[ \frac{{w}^{cos}_i}{{w}^{cos}_j} \right] =
\begin{pmatrix}
$\,\,$ 1 $\,\,$ & $\,\,$1.9609$\,\,$ & $\,\,$\color{red} 4.1013\color{black} $\,\,$ & $\,\,$7.9715$\,\,$ \\
$\,\,$0.5100$\,\,$ & $\,\,$ 1 $\,\,$ & $\,\,$\color{red} 2.0915\color{black} $\,\,$ & $\,\,$4.0651  $\,\,$ \\
$\,\,$\color{red} 0.2438\color{black} $\,\,$ & $\,\,$\color{red} 0.4781\color{black} $\,\,$ & $\,\,$ 1 $\,\,$ & $\,\,$\color{red} 1.9436\color{black}  $\,\,$ \\
$\,\,$0.1254$\,\,$ & $\,\,$0.2460$\,\,$ & $\,\,$\color{red} 0.5145\color{black} $\,\,$ & $\,\,$ 1  $\,\,$ \\
\end{pmatrix},
\end{equation*}

\begin{equation*}
\mathbf{w}^{\prime} =
\begin{pmatrix}
0.530389\\
0.270478\\
0.132597\\
0.066536
\end{pmatrix} =
0.996724\cdot
\begin{pmatrix}
0.532133\\
0.271367\\
\color{gr} 0.133033\color{black} \\
0.066755
\end{pmatrix},
\end{equation*}
\begin{equation*}
\left[ \frac{{w}^{\prime}_i}{{w}^{\prime}_j} \right] =
\begin{pmatrix}
$\,\,$ 1 $\,\,$ & $\,\,$1.9609$\,\,$ & $\,\,$\color{gr} \color{blue} 4\color{black} $\,\,$ & $\,\,$7.9715$\,\,$ \\
$\,\,$0.5100$\,\,$ & $\,\,$ 1 $\,\,$ & $\,\,$\color{gr} 2.0398\color{black} $\,\,$ & $\,\,$4.0651  $\,\,$ \\
$\,\,$\color{gr} \color{blue}  1/4\color{black} $\,\,$ & $\,\,$\color{gr} 0.4902\color{black} $\,\,$ & $\,\,$ 1 $\,\,$ & $\,\,$\color{gr} 1.9929\color{black}  $\,\,$ \\
$\,\,$0.1254$\,\,$ & $\,\,$0.2460$\,\,$ & $\,\,$\color{gr} 0.5018\color{black} $\,\,$ & $\,\,$ 1  $\,\,$ \\
\end{pmatrix},
\end{equation*}
\end{example}
\newpage
\begin{example}
\begin{equation*}
\mathbf{A} =
\begin{pmatrix}
$\,\,$ 1 $\,\,$ & $\,\,$3$\,\,$ & $\,\,$4$\,\,$ & $\,\,$6 $\,\,$ \\
$\,\,$ 1/3$\,\,$ & $\,\,$ 1 $\,\,$ & $\,\,$2$\,\,$ & $\,\,$7 $\,\,$ \\
$\,\,$ 1/4$\,\,$ & $\,\,$ 1/2$\,\,$ & $\,\,$ 1 $\,\,$ & $\,\,$2 $\,\,$ \\
$\,\,$ 1/6$\,\,$ & $\,\,$ 1/7$\,\,$ & $\,\,$ 1/2$\,\,$ & $\,\,$ 1  $\,\,$ \\
\end{pmatrix},
\qquad
\lambda_{\max} =
4.1365,
\qquad
CR = 0.0515
\end{equation*}

\begin{equation*}
\mathbf{w}^{cos} =
\begin{pmatrix}
0.526608\\
0.281163\\
\color{red} 0.127673\color{black} \\
0.064557
\end{pmatrix}\end{equation*}
\begin{equation*}
\left[ \frac{{w}^{cos}_i}{{w}^{cos}_j} \right] =
\begin{pmatrix}
$\,\,$ 1 $\,\,$ & $\,\,$1.8730$\,\,$ & $\,\,$\color{red} 4.1247\color{black} $\,\,$ & $\,\,$8.1573$\,\,$ \\
$\,\,$0.5339$\,\,$ & $\,\,$ 1 $\,\,$ & $\,\,$\color{red} 2.2022\color{black} $\,\,$ & $\,\,$4.3553  $\,\,$ \\
$\,\,$\color{red} 0.2424\color{black} $\,\,$ & $\,\,$\color{red} 0.4541\color{black} $\,\,$ & $\,\,$ 1 $\,\,$ & $\,\,$\color{red} 1.9777\color{black}  $\,\,$ \\
$\,\,$0.1226$\,\,$ & $\,\,$0.2296$\,\,$ & $\,\,$\color{red} 0.5056\color{black} $\,\,$ & $\,\,$ 1  $\,\,$ \\
\end{pmatrix},
\end{equation*}

\begin{equation*}
\mathbf{w}^{\prime} =
\begin{pmatrix}
0.525850\\
0.280758\\
0.128928\\
0.064464
\end{pmatrix} =
0.998561\cdot
\begin{pmatrix}
0.526608\\
0.281163\\
\color{gr} 0.129114\color{black} \\
0.064557
\end{pmatrix},
\end{equation*}
\begin{equation*}
\left[ \frac{{w}^{\prime}_i}{{w}^{\prime}_j} \right] =
\begin{pmatrix}
$\,\,$ 1 $\,\,$ & $\,\,$1.8730$\,\,$ & $\,\,$\color{gr} 4.0786\color{black} $\,\,$ & $\,\,$8.1573$\,\,$ \\
$\,\,$0.5339$\,\,$ & $\,\,$ 1 $\,\,$ & $\,\,$\color{gr} 2.1776\color{black} $\,\,$ & $\,\,$4.3553  $\,\,$ \\
$\,\,$\color{gr} 0.2452\color{black} $\,\,$ & $\,\,$\color{gr} 0.4592\color{black} $\,\,$ & $\,\,$ 1 $\,\,$ & $\,\,$\color{gr} \color{blue} 2\color{black}  $\,\,$ \\
$\,\,$0.1226$\,\,$ & $\,\,$0.2296$\,\,$ & $\,\,$\color{gr} \color{blue}  1/2\color{black} $\,\,$ & $\,\,$ 1  $\,\,$ \\
\end{pmatrix},
\end{equation*}
\end{example}
\newpage
\begin{example}
\begin{equation*}
\mathbf{A} =
\begin{pmatrix}
$\,\,$ 1 $\,\,$ & $\,\,$3$\,\,$ & $\,\,$4$\,\,$ & $\,\,$6 $\,\,$ \\
$\,\,$ 1/3$\,\,$ & $\,\,$ 1 $\,\,$ & $\,\,$5$\,\,$ & $\,\,$3 $\,\,$ \\
$\,\,$ 1/4$\,\,$ & $\,\,$ 1/5$\,\,$ & $\,\,$ 1 $\,\,$ & $\,\,$1 $\,\,$ \\
$\,\,$ 1/6$\,\,$ & $\,\,$ 1/3$\,\,$ & $\,\,$ 1 $\,\,$ & $\,\,$ 1  $\,\,$ \\
\end{pmatrix},
\qquad
\lambda_{\max} =
4.1502,
\qquad
CR = 0.0566
\end{equation*}

\begin{equation*}
\mathbf{w}^{cos} =
\begin{pmatrix}
0.529897\\
0.288515\\
0.093506\\
\color{red} 0.088082\color{black}
\end{pmatrix}\end{equation*}
\begin{equation*}
\left[ \frac{{w}^{cos}_i}{{w}^{cos}_j} \right] =
\begin{pmatrix}
$\,\,$ 1 $\,\,$ & $\,\,$1.8366$\,\,$ & $\,\,$5.6670$\,\,$ & $\,\,$\color{red} 6.0160\color{black} $\,\,$ \\
$\,\,$0.5445$\,\,$ & $\,\,$ 1 $\,\,$ & $\,\,$3.0855$\,\,$ & $\,\,$\color{red} 3.2755\color{black}   $\,\,$ \\
$\,\,$0.1765$\,\,$ & $\,\,$0.3241$\,\,$ & $\,\,$ 1 $\,\,$ & $\,\,$\color{red} 1.0616\color{black}  $\,\,$ \\
$\,\,$\color{red} 0.1662\color{black} $\,\,$ & $\,\,$\color{red} 0.3053\color{black} $\,\,$ & $\,\,$\color{red} 0.9420\color{black} $\,\,$ & $\,\,$ 1  $\,\,$ \\
\end{pmatrix},
\end{equation*}

\begin{equation*}
\mathbf{w}^{\prime} =
\begin{pmatrix}
0.529773\\
0.288448\\
0.093484\\
0.088295
\end{pmatrix} =
0.999766\cdot
\begin{pmatrix}
0.529897\\
0.288515\\
0.093506\\
\color{gr} 0.088316\color{black}
\end{pmatrix},
\end{equation*}
\begin{equation*}
\left[ \frac{{w}^{\prime}_i}{{w}^{\prime}_j} \right] =
\begin{pmatrix}
$\,\,$ 1 $\,\,$ & $\,\,$1.8366$\,\,$ & $\,\,$5.6670$\,\,$ & $\,\,$\color{gr} \color{blue} 6\color{black} $\,\,$ \\
$\,\,$0.5445$\,\,$ & $\,\,$ 1 $\,\,$ & $\,\,$3.0855$\,\,$ & $\,\,$\color{gr} 3.2668\color{black}   $\,\,$ \\
$\,\,$0.1765$\,\,$ & $\,\,$0.3241$\,\,$ & $\,\,$ 1 $\,\,$ & $\,\,$\color{gr} 1.0588\color{black}  $\,\,$ \\
$\,\,$\color{gr} \color{blue}  1/6\color{black} $\,\,$ & $\,\,$\color{gr} 0.3061\color{black} $\,\,$ & $\,\,$\color{gr} 0.9445\color{black} $\,\,$ & $\,\,$ 1  $\,\,$ \\
\end{pmatrix},
\end{equation*}
\end{example}
\newpage
\begin{example}
\begin{equation*}
\mathbf{A} =
\begin{pmatrix}
$\,\,$ 1 $\,\,$ & $\,\,$3$\,\,$ & $\,\,$4$\,\,$ & $\,\,$6 $\,\,$ \\
$\,\,$ 1/3$\,\,$ & $\,\,$ 1 $\,\,$ & $\,\,$6$\,\,$ & $\,\,$3 $\,\,$ \\
$\,\,$ 1/4$\,\,$ & $\,\,$ 1/6$\,\,$ & $\,\,$ 1 $\,\,$ & $\,\,$1 $\,\,$ \\
$\,\,$ 1/6$\,\,$ & $\,\,$ 1/3$\,\,$ & $\,\,$ 1 $\,\,$ & $\,\,$ 1  $\,\,$ \\
\end{pmatrix},
\qquad
\lambda_{\max} =
4.1990,
\qquad
CR = 0.0750
\end{equation*}

\begin{equation*}
\mathbf{w}^{cos} =
\begin{pmatrix}
0.524066\\
0.299689\\
0.090018\\
\color{red} 0.086227\color{black}
\end{pmatrix}\end{equation*}
\begin{equation*}
\left[ \frac{{w}^{cos}_i}{{w}^{cos}_j} \right] =
\begin{pmatrix}
$\,\,$ 1 $\,\,$ & $\,\,$1.7487$\,\,$ & $\,\,$5.8218$\,\,$ & $\,\,$\color{red} 6.0778\color{black} $\,\,$ \\
$\,\,$0.5719$\,\,$ & $\,\,$ 1 $\,\,$ & $\,\,$3.3292$\,\,$ & $\,\,$\color{red} 3.4756\color{black}   $\,\,$ \\
$\,\,$0.1718$\,\,$ & $\,\,$0.3004$\,\,$ & $\,\,$ 1 $\,\,$ & $\,\,$\color{red} 1.0440\color{black}  $\,\,$ \\
$\,\,$\color{red} 0.1645\color{black} $\,\,$ & $\,\,$\color{red} 0.2877\color{black} $\,\,$ & $\,\,$\color{red} 0.9579\color{black} $\,\,$ & $\,\,$ 1  $\,\,$ \\
\end{pmatrix},
\end{equation*}

\begin{equation*}
\mathbf{w}^{\prime} =
\begin{pmatrix}
0.523481\\
0.299354\\
0.089918\\
0.087247
\end{pmatrix} =
0.998884\cdot
\begin{pmatrix}
0.524066\\
0.299689\\
0.090018\\
\color{gr} 0.087344\color{black}
\end{pmatrix},
\end{equation*}
\begin{equation*}
\left[ \frac{{w}^{\prime}_i}{{w}^{\prime}_j} \right] =
\begin{pmatrix}
$\,\,$ 1 $\,\,$ & $\,\,$1.7487$\,\,$ & $\,\,$5.8218$\,\,$ & $\,\,$\color{gr} \color{blue} 6\color{black} $\,\,$ \\
$\,\,$0.5719$\,\,$ & $\,\,$ 1 $\,\,$ & $\,\,$3.3292$\,\,$ & $\,\,$\color{gr} 3.4311\color{black}   $\,\,$ \\
$\,\,$0.1718$\,\,$ & $\,\,$0.3004$\,\,$ & $\,\,$ 1 $\,\,$ & $\,\,$\color{gr} 1.0306\color{black}  $\,\,$ \\
$\,\,$\color{gr} \color{blue}  1/6\color{black} $\,\,$ & $\,\,$\color{gr} 0.2915\color{black} $\,\,$ & $\,\,$\color{gr} 0.9703\color{black} $\,\,$ & $\,\,$ 1  $\,\,$ \\
\end{pmatrix},
\end{equation*}
\end{example}
\newpage
\begin{example}
\begin{equation*}
\mathbf{A} =
\begin{pmatrix}
$\,\,$ 1 $\,\,$ & $\,\,$3$\,\,$ & $\,\,$4$\,\,$ & $\,\,$6 $\,\,$ \\
$\,\,$ 1/3$\,\,$ & $\,\,$ 1 $\,\,$ & $\,\,$7$\,\,$ & $\,\,$3 $\,\,$ \\
$\,\,$ 1/4$\,\,$ & $\,\,$ 1/7$\,\,$ & $\,\,$ 1 $\,\,$ & $\,\,$1 $\,\,$ \\
$\,\,$ 1/6$\,\,$ & $\,\,$ 1/3$\,\,$ & $\,\,$ 1 $\,\,$ & $\,\,$ 1  $\,\,$ \\
\end{pmatrix},
\qquad
\lambda_{\max} =
4.2478,
\qquad
CR = 0.0935
\end{equation*}

\begin{equation*}
\mathbf{w}^{cos} =
\begin{pmatrix}
0.519517\\
0.308455\\
0.087320\\
\color{red} 0.084708\color{black}
\end{pmatrix}\end{equation*}
\begin{equation*}
\left[ \frac{{w}^{cos}_i}{{w}^{cos}_j} \right] =
\begin{pmatrix}
$\,\,$ 1 $\,\,$ & $\,\,$1.6843$\,\,$ & $\,\,$5.9496$\,\,$ & $\,\,$\color{red} 6.1331\color{black} $\,\,$ \\
$\,\,$0.5937$\,\,$ & $\,\,$ 1 $\,\,$ & $\,\,$3.5325$\,\,$ & $\,\,$\color{red} 3.6414\color{black}   $\,\,$ \\
$\,\,$0.1681$\,\,$ & $\,\,$0.2831$\,\,$ & $\,\,$ 1 $\,\,$ & $\,\,$\color{red} 1.0308\color{black}  $\,\,$ \\
$\,\,$\color{red} 0.1631\color{black} $\,\,$ & $\,\,$\color{red} 0.2746\color{black} $\,\,$ & $\,\,$\color{red} 0.9701\color{black} $\,\,$ & $\,\,$ 1  $\,\,$ \\
\end{pmatrix},
\end{equation*}

\begin{equation*}
\mathbf{w}^{\prime} =
\begin{pmatrix}
0.518543\\
0.307877\\
0.087157\\
0.086424
\end{pmatrix} =
0.998125\cdot
\begin{pmatrix}
0.519517\\
0.308455\\
0.087320\\
\color{gr} 0.086586\color{black}
\end{pmatrix},
\end{equation*}
\begin{equation*}
\left[ \frac{{w}^{\prime}_i}{{w}^{\prime}_j} \right] =
\begin{pmatrix}
$\,\,$ 1 $\,\,$ & $\,\,$1.6843$\,\,$ & $\,\,$5.9496$\,\,$ & $\,\,$\color{gr} \color{blue} 6\color{black} $\,\,$ \\
$\,\,$0.5937$\,\,$ & $\,\,$ 1 $\,\,$ & $\,\,$3.5325$\,\,$ & $\,\,$\color{gr} 3.5624\color{black}   $\,\,$ \\
$\,\,$0.1681$\,\,$ & $\,\,$0.2831$\,\,$ & $\,\,$ 1 $\,\,$ & $\,\,$\color{gr} 1.0085\color{black}  $\,\,$ \\
$\,\,$\color{gr} \color{blue}  1/6\color{black} $\,\,$ & $\,\,$\color{gr} 0.2807\color{black} $\,\,$ & $\,\,$\color{gr} 0.9916\color{black} $\,\,$ & $\,\,$ 1  $\,\,$ \\
\end{pmatrix},
\end{equation*}
\end{example}
\newpage
\begin{example}
\begin{equation*}
\mathbf{A} =
\begin{pmatrix}
$\,\,$ 1 $\,\,$ & $\,\,$3$\,\,$ & $\,\,$4$\,\,$ & $\,\,$6 $\,\,$ \\
$\,\,$ 1/3$\,\,$ & $\,\,$ 1 $\,\,$ & $\,\,$7$\,\,$ & $\,\,$4 $\,\,$ \\
$\,\,$ 1/4$\,\,$ & $\,\,$ 1/7$\,\,$ & $\,\,$ 1 $\,\,$ & $\,\,$1 $\,\,$ \\
$\,\,$ 1/6$\,\,$ & $\,\,$ 1/4$\,\,$ & $\,\,$ 1 $\,\,$ & $\,\,$ 1  $\,\,$ \\
\end{pmatrix},
\qquad
\lambda_{\max} =
4.2421,
\qquad
CR = 0.0913
\end{equation*}

\begin{equation*}
\mathbf{w}^{cos} =
\begin{pmatrix}
0.509947\\
0.325510\\
0.085681\\
\color{red} 0.078862\color{black}
\end{pmatrix}\end{equation*}
\begin{equation*}
\left[ \frac{{w}^{cos}_i}{{w}^{cos}_j} \right] =
\begin{pmatrix}
$\,\,$ 1 $\,\,$ & $\,\,$1.5666$\,\,$ & $\,\,$5.9517$\,\,$ & $\,\,$\color{red} 6.4663\color{black} $\,\,$ \\
$\,\,$0.6383$\,\,$ & $\,\,$ 1 $\,\,$ & $\,\,$3.7991$\,\,$ & $\,\,$\color{red} 4.1276\color{black}   $\,\,$ \\
$\,\,$0.1680$\,\,$ & $\,\,$0.2632$\,\,$ & $\,\,$ 1 $\,\,$ & $\,\,$\color{red} 1.0865\color{black}  $\,\,$ \\
$\,\,$\color{red} 0.1546\color{black} $\,\,$ & $\,\,$\color{red} 0.2423\color{black} $\,\,$ & $\,\,$\color{red} 0.9204\color{black} $\,\,$ & $\,\,$ 1  $\,\,$ \\
\end{pmatrix},
\end{equation*}

\begin{equation*}
\mathbf{w}^{\prime} =
\begin{pmatrix}
0.508668\\
0.324693\\
0.085466\\
0.081173
\end{pmatrix} =
0.997491\cdot
\begin{pmatrix}
0.509947\\
0.325510\\
0.085681\\
\color{gr} 0.081377\color{black}
\end{pmatrix},
\end{equation*}
\begin{equation*}
\left[ \frac{{w}^{\prime}_i}{{w}^{\prime}_j} \right] =
\begin{pmatrix}
$\,\,$ 1 $\,\,$ & $\,\,$1.5666$\,\,$ & $\,\,$5.9517$\,\,$ & $\,\,$\color{gr} 6.2664\color{black} $\,\,$ \\
$\,\,$0.6383$\,\,$ & $\,\,$ 1 $\,\,$ & $\,\,$3.7991$\,\,$ & $\,\,$\color{gr} \color{blue} 4\color{black}   $\,\,$ \\
$\,\,$0.1680$\,\,$ & $\,\,$0.2632$\,\,$ & $\,\,$ 1 $\,\,$ & $\,\,$\color{gr} 1.0529\color{black}  $\,\,$ \\
$\,\,$\color{gr} 0.1596\color{black} $\,\,$ & $\,\,$\color{gr} \color{blue}  1/4\color{black} $\,\,$ & $\,\,$\color{gr} 0.9498\color{black} $\,\,$ & $\,\,$ 1  $\,\,$ \\
\end{pmatrix},
\end{equation*}
\end{example}
\newpage
\begin{example}
\begin{equation*}
\mathbf{A} =
\begin{pmatrix}
$\,\,$ 1 $\,\,$ & $\,\,$3$\,\,$ & $\,\,$4$\,\,$ & $\,\,$8 $\,\,$ \\
$\,\,$ 1/3$\,\,$ & $\,\,$ 1 $\,\,$ & $\,\,$2$\,\,$ & $\,\,$2 $\,\,$ \\
$\,\,$ 1/4$\,\,$ & $\,\,$ 1/2$\,\,$ & $\,\,$ 1 $\,\,$ & $\,\,$3 $\,\,$ \\
$\,\,$ 1/8$\,\,$ & $\,\,$ 1/2$\,\,$ & $\,\,$ 1/3$\,\,$ & $\,\,$ 1  $\,\,$ \\
\end{pmatrix},
\qquad
\lambda_{\max} =
4.1031,
\qquad
CR = 0.0389
\end{equation*}

\begin{equation*}
\mathbf{w}^{cos} =
\begin{pmatrix}
\color{red} 0.575317\color{black} \\
0.202838\\
0.149564\\
0.072281
\end{pmatrix}\end{equation*}
\begin{equation*}
\left[ \frac{{w}^{cos}_i}{{w}^{cos}_j} \right] =
\begin{pmatrix}
$\,\,$ 1 $\,\,$ & $\,\,$\color{red} 2.8363\color{black} $\,\,$ & $\,\,$\color{red} 3.8466\color{black} $\,\,$ & $\,\,$\color{red} 7.9595\color{black} $\,\,$ \\
$\,\,$\color{red} 0.3526\color{black} $\,\,$ & $\,\,$ 1 $\,\,$ & $\,\,$1.3562$\,\,$ & $\,\,$2.8063  $\,\,$ \\
$\,\,$\color{red} 0.2600\color{black} $\,\,$ & $\,\,$0.7374$\,\,$ & $\,\,$ 1 $\,\,$ & $\,\,$2.0692 $\,\,$ \\
$\,\,$\color{red} 0.1256\color{black} $\,\,$ & $\,\,$0.3563$\,\,$ & $\,\,$0.4833$\,\,$ & $\,\,$ 1  $\,\,$ \\
\end{pmatrix},
\end{equation*}

\begin{equation*}
\mathbf{w}^{\prime} =
\begin{pmatrix}
0.576557\\
0.202246\\
0.149128\\
0.072070
\end{pmatrix} =
0.997081\cdot
\begin{pmatrix}
\color{gr} 0.578245\color{black} \\
0.202838\\
0.149564\\
0.072281
\end{pmatrix},
\end{equation*}
\begin{equation*}
\left[ \frac{{w}^{\prime}_i}{{w}^{\prime}_j} \right] =
\begin{pmatrix}
$\,\,$ 1 $\,\,$ & $\,\,$\color{gr} 2.8508\color{black} $\,\,$ & $\,\,$\color{gr} 3.8662\color{black} $\,\,$ & $\,\,$\color{gr} \color{blue} 8\color{black} $\,\,$ \\
$\,\,$\color{gr} 0.3508\color{black} $\,\,$ & $\,\,$ 1 $\,\,$ & $\,\,$1.3562$\,\,$ & $\,\,$2.8063  $\,\,$ \\
$\,\,$\color{gr} 0.2587\color{black} $\,\,$ & $\,\,$0.7374$\,\,$ & $\,\,$ 1 $\,\,$ & $\,\,$2.0692 $\,\,$ \\
$\,\,$\color{gr} \color{blue}  1/8\color{black} $\,\,$ & $\,\,$0.3563$\,\,$ & $\,\,$0.4833$\,\,$ & $\,\,$ 1  $\,\,$ \\
\end{pmatrix},
\end{equation*}
\end{example}
\newpage
\begin{example}
\begin{equation*}
\mathbf{A} =
\begin{pmatrix}
$\,\,$ 1 $\,\,$ & $\,\,$3$\,\,$ & $\,\,$4$\,\,$ & $\,\,$9 $\,\,$ \\
$\,\,$ 1/3$\,\,$ & $\,\,$ 1 $\,\,$ & $\,\,$2$\,\,$ & $\,\,$2 $\,\,$ \\
$\,\,$ 1/4$\,\,$ & $\,\,$ 1/2$\,\,$ & $\,\,$ 1 $\,\,$ & $\,\,$4 $\,\,$ \\
$\,\,$ 1/9$\,\,$ & $\,\,$ 1/2$\,\,$ & $\,\,$ 1/4$\,\,$ & $\,\,$ 1  $\,\,$ \\
\end{pmatrix},
\qquad
\lambda_{\max} =
4.1664,
\qquad
CR = 0.0627
\end{equation*}

\begin{equation*}
\mathbf{w}^{cos} =
\begin{pmatrix}
\color{red} 0.575894\color{black} \\
0.199349\\
0.159322\\
0.065436
\end{pmatrix}\end{equation*}
\begin{equation*}
\left[ \frac{{w}^{cos}_i}{{w}^{cos}_j} \right] =
\begin{pmatrix}
$\,\,$ 1 $\,\,$ & $\,\,$\color{red} 2.8889\color{black} $\,\,$ & $\,\,$\color{red} 3.6147\color{black} $\,\,$ & $\,\,$\color{red} 8.8009\color{black} $\,\,$ \\
$\,\,$\color{red} 0.3462\color{black} $\,\,$ & $\,\,$ 1 $\,\,$ & $\,\,$1.2512$\,\,$ & $\,\,$3.0465  $\,\,$ \\
$\,\,$\color{red} 0.2767\color{black} $\,\,$ & $\,\,$0.7992$\,\,$ & $\,\,$ 1 $\,\,$ & $\,\,$2.4348 $\,\,$ \\
$\,\,$\color{red} 0.1136\color{black} $\,\,$ & $\,\,$0.3282$\,\,$ & $\,\,$0.4107$\,\,$ & $\,\,$ 1  $\,\,$ \\
\end{pmatrix},
\end{equation*}

\begin{equation*}
\mathbf{w}^{\prime} =
\begin{pmatrix}
0.581348\\
0.196785\\
0.157273\\
0.064594
\end{pmatrix} =
0.987139\cdot
\begin{pmatrix}
\color{gr} 0.588922\color{black} \\
0.199349\\
0.159322\\
0.065436
\end{pmatrix},
\end{equation*}
\begin{equation*}
\left[ \frac{{w}^{\prime}_i}{{w}^{\prime}_j} \right] =
\begin{pmatrix}
$\,\,$ 1 $\,\,$ & $\,\,$\color{gr} 2.9542\color{black} $\,\,$ & $\,\,$\color{gr} 3.6964\color{black} $\,\,$ & $\,\,$\color{gr} \color{blue} 9\color{black} $\,\,$ \\
$\,\,$\color{gr} 0.3385\color{black} $\,\,$ & $\,\,$ 1 $\,\,$ & $\,\,$1.2512$\,\,$ & $\,\,$3.0465  $\,\,$ \\
$\,\,$\color{gr} 0.2705\color{black} $\,\,$ & $\,\,$0.7992$\,\,$ & $\,\,$ 1 $\,\,$ & $\,\,$2.4348 $\,\,$ \\
$\,\,$\color{gr} \color{blue}  1/9\color{black} $\,\,$ & $\,\,$0.3282$\,\,$ & $\,\,$0.4107$\,\,$ & $\,\,$ 1  $\,\,$ \\
\end{pmatrix},
\end{equation*}
\end{example}
\newpage
\begin{example}
\begin{equation*}
\mathbf{A} =
\begin{pmatrix}
$\,\,$ 1 $\,\,$ & $\,\,$3$\,\,$ & $\,\,$4$\,\,$ & $\,\,$9 $\,\,$ \\
$\,\,$ 1/3$\,\,$ & $\,\,$ 1 $\,\,$ & $\,\,$2$\,\,$ & $\,\,$8 $\,\,$ \\
$\,\,$ 1/4$\,\,$ & $\,\,$ 1/2$\,\,$ & $\,\,$ 1 $\,\,$ & $\,\,$3 $\,\,$ \\
$\,\,$ 1/9$\,\,$ & $\,\,$ 1/8$\,\,$ & $\,\,$ 1/3$\,\,$ & $\,\,$ 1  $\,\,$ \\
\end{pmatrix},
\qquad
\lambda_{\max} =
4.0820,
\qquad
CR = 0.0309
\end{equation*}

\begin{equation*}
\mathbf{w}^{cos} =
\begin{pmatrix}
0.548857\\
0.269976\\
\color{red} 0.134383\color{black} \\
0.046784
\end{pmatrix}\end{equation*}
\begin{equation*}
\left[ \frac{{w}^{cos}_i}{{w}^{cos}_j} \right] =
\begin{pmatrix}
$\,\,$ 1 $\,\,$ & $\,\,$2.0330$\,\,$ & $\,\,$\color{red} 4.0843\color{black} $\,\,$ & $\,\,$11.7316$\,\,$ \\
$\,\,$0.4919$\,\,$ & $\,\,$ 1 $\,\,$ & $\,\,$\color{red} 2.0090\color{black} $\,\,$ & $\,\,$5.7706  $\,\,$ \\
$\,\,$\color{red} 0.2448\color{black} $\,\,$ & $\,\,$\color{red} 0.4978\color{black} $\,\,$ & $\,\,$ 1 $\,\,$ & $\,\,$\color{red} 2.8724\color{black}  $\,\,$ \\
$\,\,$0.0852$\,\,$ & $\,\,$0.1733$\,\,$ & $\,\,$\color{red} 0.3481\color{black} $\,\,$ & $\,\,$ 1  $\,\,$ \\
\end{pmatrix},
\end{equation*}

\begin{equation*}
\mathbf{w}^{\prime} =
\begin{pmatrix}
0.548525\\
0.269813\\
0.134906\\
0.046756
\end{pmatrix} =
0.999395\cdot
\begin{pmatrix}
0.548857\\
0.269976\\
\color{gr} 0.134988\color{black} \\
0.046784
\end{pmatrix},
\end{equation*}
\begin{equation*}
\left[ \frac{{w}^{\prime}_i}{{w}^{\prime}_j} \right] =
\begin{pmatrix}
$\,\,$ 1 $\,\,$ & $\,\,$2.0330$\,\,$ & $\,\,$\color{gr} 4.0660\color{black} $\,\,$ & $\,\,$11.7316$\,\,$ \\
$\,\,$0.4919$\,\,$ & $\,\,$ 1 $\,\,$ & $\,\,$\color{gr} \color{blue} 2\color{black} $\,\,$ & $\,\,$5.7706  $\,\,$ \\
$\,\,$\color{gr} 0.2459\color{black} $\,\,$ & $\,\,$\color{gr} \color{blue}  1/2\color{black} $\,\,$ & $\,\,$ 1 $\,\,$ & $\,\,$\color{gr} 2.8853\color{black}  $\,\,$ \\
$\,\,$0.0852$\,\,$ & $\,\,$0.1733$\,\,$ & $\,\,$\color{gr} 0.3466\color{black} $\,\,$ & $\,\,$ 1  $\,\,$ \\
\end{pmatrix},
\end{equation*}
\end{example}
\newpage
\begin{example}
\begin{equation*}
\mathbf{A} =
\begin{pmatrix}
$\,\,$ 1 $\,\,$ & $\,\,$3$\,\,$ & $\,\,$4$\,\,$ & $\,\,$9 $\,\,$ \\
$\,\,$ 1/3$\,\,$ & $\,\,$ 1 $\,\,$ & $\,\,$2$\,\,$ & $\,\,$9 $\,\,$ \\
$\,\,$ 1/4$\,\,$ & $\,\,$ 1/2$\,\,$ & $\,\,$ 1 $\,\,$ & $\,\,$3 $\,\,$ \\
$\,\,$ 1/9$\,\,$ & $\,\,$ 1/9$\,\,$ & $\,\,$ 1/3$\,\,$ & $\,\,$ 1  $\,\,$ \\
\end{pmatrix},
\qquad
\lambda_{\max} =
4.1031,
\qquad
CR = 0.0389
\end{equation*}

\begin{equation*}
\mathbf{w}^{cos} =
\begin{pmatrix}
0.544205\\
0.277526\\
\color{red} 0.132725\color{black} \\
0.045544
\end{pmatrix}\end{equation*}
\begin{equation*}
\left[ \frac{{w}^{cos}_i}{{w}^{cos}_j} \right] =
\begin{pmatrix}
$\,\,$ 1 $\,\,$ & $\,\,$1.9609$\,\,$ & $\,\,$\color{red} 4.1003\color{black} $\,\,$ & $\,\,$11.9490$\,\,$ \\
$\,\,$0.5100$\,\,$ & $\,\,$ 1 $\,\,$ & $\,\,$\color{red} 2.0910\color{black} $\,\,$ & $\,\,$6.0936  $\,\,$ \\
$\,\,$\color{red} 0.2439\color{black} $\,\,$ & $\,\,$\color{red} 0.4782\color{black} $\,\,$ & $\,\,$ 1 $\,\,$ & $\,\,$\color{red} 2.9142\color{black}  $\,\,$ \\
$\,\,$0.0837$\,\,$ & $\,\,$0.1641$\,\,$ & $\,\,$\color{red} 0.3431\color{black} $\,\,$ & $\,\,$ 1  $\,\,$ \\
\end{pmatrix},
\end{equation*}

\begin{equation*}
\mathbf{w}^{\prime} =
\begin{pmatrix}
0.542401\\
0.276606\\
0.135600\\
0.045393
\end{pmatrix} =
0.996685\cdot
\begin{pmatrix}
0.544205\\
0.277526\\
\color{gr} 0.136051\color{black} \\
0.045544
\end{pmatrix},
\end{equation*}
\begin{equation*}
\left[ \frac{{w}^{\prime}_i}{{w}^{\prime}_j} \right] =
\begin{pmatrix}
$\,\,$ 1 $\,\,$ & $\,\,$1.9609$\,\,$ & $\,\,$\color{gr} \color{blue} 4\color{black} $\,\,$ & $\,\,$11.9490$\,\,$ \\
$\,\,$0.5100$\,\,$ & $\,\,$ 1 $\,\,$ & $\,\,$\color{gr} 2.0399\color{black} $\,\,$ & $\,\,$6.0936  $\,\,$ \\
$\,\,$\color{gr} \color{blue}  1/4\color{black} $\,\,$ & $\,\,$\color{gr} 0.4902\color{black} $\,\,$ & $\,\,$ 1 $\,\,$ & $\,\,$\color{gr} 2.9872\color{black}  $\,\,$ \\
$\,\,$0.0837$\,\,$ & $\,\,$0.1641$\,\,$ & $\,\,$\color{gr} 0.3348\color{black} $\,\,$ & $\,\,$ 1  $\,\,$ \\
\end{pmatrix},
\end{equation*}
\end{example}
\newpage
\begin{example}
\begin{equation*}
\mathbf{A} =
\begin{pmatrix}
$\,\,$ 1 $\,\,$ & $\,\,$3$\,\,$ & $\,\,$5$\,\,$ & $\,\,$3 $\,\,$ \\
$\,\,$ 1/3$\,\,$ & $\,\,$ 1 $\,\,$ & $\,\,$3$\,\,$ & $\,\,$5 $\,\,$ \\
$\,\,$ 1/5$\,\,$ & $\,\,$ 1/3$\,\,$ & $\,\,$ 1 $\,\,$ & $\,\,$1 $\,\,$ \\
$\,\,$ 1/3$\,\,$ & $\,\,$ 1/5$\,\,$ & $\,\,$ 1 $\,\,$ & $\,\,$ 1  $\,\,$ \\
\end{pmatrix},
\qquad
\lambda_{\max} =
4.2253,
\qquad
CR = 0.0849
\end{equation*}

\begin{equation*}
\mathbf{w}^{cos} =
\begin{pmatrix}
0.493171\\
0.302897\\
\color{red} 0.095980\color{black} \\
0.107952
\end{pmatrix}\end{equation*}
\begin{equation*}
\left[ \frac{{w}^{cos}_i}{{w}^{cos}_j} \right] =
\begin{pmatrix}
$\,\,$ 1 $\,\,$ & $\,\,$1.6282$\,\,$ & $\,\,$\color{red} 5.1382\color{black} $\,\,$ & $\,\,$4.5684$\,\,$ \\
$\,\,$0.6142$\,\,$ & $\,\,$ 1 $\,\,$ & $\,\,$\color{red} 3.1558\color{black} $\,\,$ & $\,\,$2.8058  $\,\,$ \\
$\,\,$\color{red} 0.1946\color{black} $\,\,$ & $\,\,$\color{red} 0.3169\color{black} $\,\,$ & $\,\,$ 1 $\,\,$ & $\,\,$\color{red} 0.8891\color{black}  $\,\,$ \\
$\,\,$0.2189$\,\,$ & $\,\,$0.3564$\,\,$ & $\,\,$\color{red} 1.1247\color{black} $\,\,$ & $\,\,$ 1  $\,\,$ \\
\end{pmatrix},
\end{equation*}

\begin{equation*}
\mathbf{w}^{\prime} =
\begin{pmatrix}
0.491866\\
0.302095\\
0.098373\\
0.107666
\end{pmatrix} =
0.997353\cdot
\begin{pmatrix}
0.493171\\
0.302897\\
\color{gr} 0.098634\color{black} \\
0.107952
\end{pmatrix},
\end{equation*}
\begin{equation*}
\left[ \frac{{w}^{\prime}_i}{{w}^{\prime}_j} \right] =
\begin{pmatrix}
$\,\,$ 1 $\,\,$ & $\,\,$1.6282$\,\,$ & $\,\,$\color{gr} \color{blue} 5\color{black} $\,\,$ & $\,\,$4.5684$\,\,$ \\
$\,\,$0.6142$\,\,$ & $\,\,$ 1 $\,\,$ & $\,\,$\color{gr} 3.0709\color{black} $\,\,$ & $\,\,$2.8058  $\,\,$ \\
$\,\,$\color{gr} \color{blue}  1/5\color{black} $\,\,$ & $\,\,$\color{gr} 0.3256\color{black} $\,\,$ & $\,\,$ 1 $\,\,$ & $\,\,$\color{gr} 0.9137\color{black}  $\,\,$ \\
$\,\,$0.2189$\,\,$ & $\,\,$0.3564$\,\,$ & $\,\,$\color{gr} 1.0945\color{black} $\,\,$ & $\,\,$ 1  $\,\,$ \\
\end{pmatrix},
\end{equation*}
\end{example}
\newpage
\begin{example}
\begin{equation*}
\mathbf{A} =
\begin{pmatrix}
$\,\,$ 1 $\,\,$ & $\,\,$3$\,\,$ & $\,\,$5$\,\,$ & $\,\,$7 $\,\,$ \\
$\,\,$ 1/3$\,\,$ & $\,\,$ 1 $\,\,$ & $\,\,$6$\,\,$ & $\,\,$4 $\,\,$ \\
$\,\,$ 1/5$\,\,$ & $\,\,$ 1/6$\,\,$ & $\,\,$ 1 $\,\,$ & $\,\,$1 $\,\,$ \\
$\,\,$ 1/7$\,\,$ & $\,\,$ 1/4$\,\,$ & $\,\,$ 1 $\,\,$ & $\,\,$ 1  $\,\,$ \\
\end{pmatrix},
\qquad
\lambda_{\max} =
4.1417,
\qquad
CR = 0.0534
\end{equation*}

\begin{equation*}
\mathbf{w}^{cos} =
\begin{pmatrix}
0.543943\\
0.302896\\
0.078735\\
\color{red} 0.074426\color{black}
\end{pmatrix}\end{equation*}
\begin{equation*}
\left[ \frac{{w}^{cos}_i}{{w}^{cos}_j} \right] =
\begin{pmatrix}
$\,\,$ 1 $\,\,$ & $\,\,$1.7958$\,\,$ & $\,\,$6.9086$\,\,$ & $\,\,$\color{red} 7.3085\color{black} $\,\,$ \\
$\,\,$0.5569$\,\,$ & $\,\,$ 1 $\,\,$ & $\,\,$3.8471$\,\,$ & $\,\,$\color{red} 4.0697\color{black}   $\,\,$ \\
$\,\,$0.1447$\,\,$ & $\,\,$0.2599$\,\,$ & $\,\,$ 1 $\,\,$ & $\,\,$\color{red} 1.0579\color{black}  $\,\,$ \\
$\,\,$\color{red} 0.1368\color{black} $\,\,$ & $\,\,$\color{red} 0.2457\color{black} $\,\,$ & $\,\,$\color{red} 0.9453\color{black} $\,\,$ & $\,\,$ 1  $\,\,$ \\
\end{pmatrix},
\end{equation*}

\begin{equation*}
\mathbf{w}^{\prime} =
\begin{pmatrix}
0.543238\\
0.302504\\
0.078633\\
0.075626
\end{pmatrix} =
0.998704\cdot
\begin{pmatrix}
0.543943\\
0.302896\\
0.078735\\
\color{gr} 0.075724\color{black}
\end{pmatrix},
\end{equation*}
\begin{equation*}
\left[ \frac{{w}^{\prime}_i}{{w}^{\prime}_j} \right] =
\begin{pmatrix}
$\,\,$ 1 $\,\,$ & $\,\,$1.7958$\,\,$ & $\,\,$6.9086$\,\,$ & $\,\,$\color{gr} 7.1832\color{black} $\,\,$ \\
$\,\,$0.5569$\,\,$ & $\,\,$ 1 $\,\,$ & $\,\,$3.8471$\,\,$ & $\,\,$\color{gr} \color{blue} 4\color{black}   $\,\,$ \\
$\,\,$0.1447$\,\,$ & $\,\,$0.2599$\,\,$ & $\,\,$ 1 $\,\,$ & $\,\,$\color{gr} 1.0398\color{black}  $\,\,$ \\
$\,\,$\color{gr} 0.1392\color{black} $\,\,$ & $\,\,$\color{gr} \color{blue}  1/4\color{black} $\,\,$ & $\,\,$\color{gr} 0.9618\color{black} $\,\,$ & $\,\,$ 1  $\,\,$ \\
\end{pmatrix},
\end{equation*}
\end{example}
\newpage
\begin{example}
\begin{equation*}
\mathbf{A} =
\begin{pmatrix}
$\,\,$ 1 $\,\,$ & $\,\,$3$\,\,$ & $\,\,$5$\,\,$ & $\,\,$7 $\,\,$ \\
$\,\,$ 1/3$\,\,$ & $\,\,$ 1 $\,\,$ & $\,\,$7$\,\,$ & $\,\,$4 $\,\,$ \\
$\,\,$ 1/5$\,\,$ & $\,\,$ 1/7$\,\,$ & $\,\,$ 1 $\,\,$ & $\,\,$1 $\,\,$ \\
$\,\,$ 1/7$\,\,$ & $\,\,$ 1/4$\,\,$ & $\,\,$ 1 $\,\,$ & $\,\,$ 1  $\,\,$ \\
\end{pmatrix},
\qquad
\lambda_{\max} =
4.1793,
\qquad
CR = 0.0676
\end{equation*}

\begin{equation*}
\mathbf{w}^{cos} =
\begin{pmatrix}
0.538249\\
0.312516\\
0.076173\\
\color{red} 0.073063\color{black}
\end{pmatrix}\end{equation*}
\begin{equation*}
\left[ \frac{{w}^{cos}_i}{{w}^{cos}_j} \right] =
\begin{pmatrix}
$\,\,$ 1 $\,\,$ & $\,\,$1.7223$\,\,$ & $\,\,$7.0662$\,\,$ & $\,\,$\color{red} 7.3669\color{black} $\,\,$ \\
$\,\,$0.5806$\,\,$ & $\,\,$ 1 $\,\,$ & $\,\,$4.1027$\,\,$ & $\,\,$\color{red} 4.2774\color{black}   $\,\,$ \\
$\,\,$0.1415$\,\,$ & $\,\,$0.2437$\,\,$ & $\,\,$ 1 $\,\,$ & $\,\,$\color{red} 1.0426\color{black}  $\,\,$ \\
$\,\,$\color{red} 0.1357\color{black} $\,\,$ & $\,\,$\color{red} 0.2338\color{black} $\,\,$ & $\,\,$\color{red} 0.9592\color{black} $\,\,$ & $\,\,$ 1  $\,\,$ \\
\end{pmatrix},
\end{equation*}

\begin{equation*}
\mathbf{w}^{\prime} =
\begin{pmatrix}
0.536580\\
0.311547\\
0.075937\\
0.075937
\end{pmatrix} =
0.996900\cdot
\begin{pmatrix}
0.538249\\
0.312516\\
0.076173\\
\color{gr} 0.076173\color{black}
\end{pmatrix},
\end{equation*}
\begin{equation*}
\left[ \frac{{w}^{\prime}_i}{{w}^{\prime}_j} \right] =
\begin{pmatrix}
$\,\,$ 1 $\,\,$ & $\,\,$1.7223$\,\,$ & $\,\,$7.0662$\,\,$ & $\,\,$\color{gr} 7.0662\color{black} $\,\,$ \\
$\,\,$0.5806$\,\,$ & $\,\,$ 1 $\,\,$ & $\,\,$4.1027$\,\,$ & $\,\,$\color{gr} 4.1027\color{black}   $\,\,$ \\
$\,\,$0.1415$\,\,$ & $\,\,$0.2437$\,\,$ & $\,\,$ 1 $\,\,$ & $\,\,$\color{gr} \color{blue} 1\color{black}  $\,\,$ \\
$\,\,$\color{gr} 0.1415\color{black} $\,\,$ & $\,\,$\color{gr} 0.2437\color{black} $\,\,$ & $\,\,$\color{gr} \color{blue} 1\color{black} $\,\,$ & $\,\,$ 1  $\,\,$ \\
\end{pmatrix},
\end{equation*}
\end{example}
\newpage
\begin{example}
\begin{equation*}
\mathbf{A} =
\begin{pmatrix}
$\,\,$ 1 $\,\,$ & $\,\,$3$\,\,$ & $\,\,$5$\,\,$ & $\,\,$7 $\,\,$ \\
$\,\,$ 1/3$\,\,$ & $\,\,$ 1 $\,\,$ & $\,\,$8$\,\,$ & $\,\,$4 $\,\,$ \\
$\,\,$ 1/5$\,\,$ & $\,\,$ 1/8$\,\,$ & $\,\,$ 1 $\,\,$ & $\,\,$1 $\,\,$ \\
$\,\,$ 1/7$\,\,$ & $\,\,$ 1/4$\,\,$ & $\,\,$ 1 $\,\,$ & $\,\,$ 1  $\,\,$ \\
\end{pmatrix},
\qquad
\lambda_{\max} =
4.2174,
\qquad
CR = 0.0820
\end{equation*}

\begin{equation*}
\mathbf{w}^{cos} =
\begin{pmatrix}
0.533600\\
0.320379\\
0.074113\\
\color{red} 0.071908\color{black}
\end{pmatrix}\end{equation*}
\begin{equation*}
\left[ \frac{{w}^{cos}_i}{{w}^{cos}_j} \right] =
\begin{pmatrix}
$\,\,$ 1 $\,\,$ & $\,\,$1.6655$\,\,$ & $\,\,$7.1998$\,\,$ & $\,\,$\color{red} 7.4206\color{black} $\,\,$ \\
$\,\,$0.6004$\,\,$ & $\,\,$ 1 $\,\,$ & $\,\,$4.3229$\,\,$ & $\,\,$\color{red} 4.4554\color{black}   $\,\,$ \\
$\,\,$0.1389$\,\,$ & $\,\,$0.2313$\,\,$ & $\,\,$ 1 $\,\,$ & $\,\,$\color{red} 1.0307\color{black}  $\,\,$ \\
$\,\,$\color{red} 0.1348\color{black} $\,\,$ & $\,\,$\color{red} 0.2244\color{black} $\,\,$ & $\,\,$\color{red} 0.9703\color{black} $\,\,$ & $\,\,$ 1  $\,\,$ \\
\end{pmatrix},
\end{equation*}

\begin{equation*}
\mathbf{w}^{\prime} =
\begin{pmatrix}
0.532426\\
0.319674\\
0.073950\\
0.073950
\end{pmatrix} =
0.997800\cdot
\begin{pmatrix}
0.533600\\
0.320379\\
0.074113\\
\color{gr} 0.074113\color{black}
\end{pmatrix},
\end{equation*}
\begin{equation*}
\left[ \frac{{w}^{\prime}_i}{{w}^{\prime}_j} \right] =
\begin{pmatrix}
$\,\,$ 1 $\,\,$ & $\,\,$1.6655$\,\,$ & $\,\,$7.1998$\,\,$ & $\,\,$\color{gr} 7.1998\color{black} $\,\,$ \\
$\,\,$0.6004$\,\,$ & $\,\,$ 1 $\,\,$ & $\,\,$4.3229$\,\,$ & $\,\,$\color{gr} 4.3229\color{black}   $\,\,$ \\
$\,\,$0.1389$\,\,$ & $\,\,$0.2313$\,\,$ & $\,\,$ 1 $\,\,$ & $\,\,$\color{gr} \color{blue} 1\color{black}  $\,\,$ \\
$\,\,$\color{gr} 0.1389\color{black} $\,\,$ & $\,\,$\color{gr} 0.2313\color{black} $\,\,$ & $\,\,$\color{gr} \color{blue} 1\color{black} $\,\,$ & $\,\,$ 1  $\,\,$ \\
\end{pmatrix},
\end{equation*}
\end{example}
\newpage
\begin{example}
\begin{equation*}
\mathbf{A} =
\begin{pmatrix}
$\,\,$ 1 $\,\,$ & $\,\,$3$\,\,$ & $\,\,$5$\,\,$ & $\,\,$7 $\,\,$ \\
$\,\,$ 1/3$\,\,$ & $\,\,$ 1 $\,\,$ & $\,\,$9$\,\,$ & $\,\,$4 $\,\,$ \\
$\,\,$ 1/5$\,\,$ & $\,\,$ 1/9$\,\,$ & $\,\,$ 1 $\,\,$ & $\,\,$1 $\,\,$ \\
$\,\,$ 1/7$\,\,$ & $\,\,$ 1/4$\,\,$ & $\,\,$ 1 $\,\,$ & $\,\,$ 1  $\,\,$ \\
\end{pmatrix},
\qquad
\lambda_{\max} =
4.2553,
\qquad
CR = 0.0963
\end{equation*}

\begin{equation*}
\mathbf{w}^{cos} =
\begin{pmatrix}
0.529777\\
0.326874\\
0.072422\\
\color{red} 0.070928\color{black}
\end{pmatrix}\end{equation*}
\begin{equation*}
\left[ \frac{{w}^{cos}_i}{{w}^{cos}_j} \right] =
\begin{pmatrix}
$\,\,$ 1 $\,\,$ & $\,\,$1.6207$\,\,$ & $\,\,$7.3152$\,\,$ & $\,\,$\color{red} 7.4692\color{black} $\,\,$ \\
$\,\,$0.6170$\,\,$ & $\,\,$ 1 $\,\,$ & $\,\,$4.5135$\,\,$ & $\,\,$\color{red} 4.6085\color{black}   $\,\,$ \\
$\,\,$0.1367$\,\,$ & $\,\,$0.2216$\,\,$ & $\,\,$ 1 $\,\,$ & $\,\,$\color{red} 1.0211\color{black}  $\,\,$ \\
$\,\,$\color{red} 0.1339\color{black} $\,\,$ & $\,\,$\color{red} 0.2170\color{black} $\,\,$ & $\,\,$\color{red} 0.9794\color{black} $\,\,$ & $\,\,$ 1  $\,\,$ \\
\end{pmatrix},
\end{equation*}

\begin{equation*}
\mathbf{w}^{\prime} =
\begin{pmatrix}
0.528986\\
0.326386\\
0.072314\\
0.072314
\end{pmatrix} =
0.998509\cdot
\begin{pmatrix}
0.529777\\
0.326874\\
0.072422\\
\color{gr} 0.072422\color{black}
\end{pmatrix},
\end{equation*}
\begin{equation*}
\left[ \frac{{w}^{\prime}_i}{{w}^{\prime}_j} \right] =
\begin{pmatrix}
$\,\,$ 1 $\,\,$ & $\,\,$1.6207$\,\,$ & $\,\,$7.3152$\,\,$ & $\,\,$\color{gr} 7.3152\color{black} $\,\,$ \\
$\,\,$0.6170$\,\,$ & $\,\,$ 1 $\,\,$ & $\,\,$4.5135$\,\,$ & $\,\,$\color{gr} 4.5135\color{black}   $\,\,$ \\
$\,\,$0.1367$\,\,$ & $\,\,$0.2216$\,\,$ & $\,\,$ 1 $\,\,$ & $\,\,$\color{gr} \color{blue} 1\color{black}  $\,\,$ \\
$\,\,$\color{gr} 0.1367\color{black} $\,\,$ & $\,\,$\color{gr} 0.2216\color{black} $\,\,$ & $\,\,$\color{gr} \color{blue} 1\color{black} $\,\,$ & $\,\,$ 1  $\,\,$ \\
\end{pmatrix},
\end{equation*}
\end{example}
\newpage
\begin{example}
\begin{equation*}
\mathbf{A} =
\begin{pmatrix}
$\,\,$ 1 $\,\,$ & $\,\,$3$\,\,$ & $\,\,$5$\,\,$ & $\,\,$7 $\,\,$ \\
$\,\,$ 1/3$\,\,$ & $\,\,$ 1 $\,\,$ & $\,\,$9$\,\,$ & $\,\,$5 $\,\,$ \\
$\,\,$ 1/5$\,\,$ & $\,\,$ 1/9$\,\,$ & $\,\,$ 1 $\,\,$ & $\,\,$1 $\,\,$ \\
$\,\,$ 1/7$\,\,$ & $\,\,$ 1/5$\,\,$ & $\,\,$ 1 $\,\,$ & $\,\,$ 1  $\,\,$ \\
\end{pmatrix},
\qquad
\lambda_{\max} =
4.2539,
\qquad
CR = 0.0957
\end{equation*}

\begin{equation*}
\mathbf{w}^{cos} =
\begin{pmatrix}
0.521019\\
0.340762\\
0.071154\\
\color{red} 0.067064\color{black}
\end{pmatrix}\end{equation*}
\begin{equation*}
\left[ \frac{{w}^{cos}_i}{{w}^{cos}_j} \right] =
\begin{pmatrix}
$\,\,$ 1 $\,\,$ & $\,\,$1.5290$\,\,$ & $\,\,$7.3224$\,\,$ & $\,\,$\color{red} 7.7690\color{black} $\,\,$ \\
$\,\,$0.6540$\,\,$ & $\,\,$ 1 $\,\,$ & $\,\,$4.7891$\,\,$ & $\,\,$\color{red} 5.0811\color{black}   $\,\,$ \\
$\,\,$0.1366$\,\,$ & $\,\,$0.2088$\,\,$ & $\,\,$ 1 $\,\,$ & $\,\,$\color{red} 1.0610\color{black}  $\,\,$ \\
$\,\,$\color{red} 0.1287\color{black} $\,\,$ & $\,\,$\color{red} 0.1968\color{black} $\,\,$ & $\,\,$\color{red} 0.9425\color{black} $\,\,$ & $\,\,$ 1  $\,\,$ \\
\end{pmatrix},
\end{equation*}

\begin{equation*}
\mathbf{w}^{\prime} =
\begin{pmatrix}
0.520453\\
0.340392\\
0.071077\\
0.068078
\end{pmatrix} =
0.998913\cdot
\begin{pmatrix}
0.521019\\
0.340762\\
0.071154\\
\color{gr} 0.068152\color{black}
\end{pmatrix},
\end{equation*}
\begin{equation*}
\left[ \frac{{w}^{\prime}_i}{{w}^{\prime}_j} \right] =
\begin{pmatrix}
$\,\,$ 1 $\,\,$ & $\,\,$1.5290$\,\,$ & $\,\,$7.3224$\,\,$ & $\,\,$\color{gr} 7.6449\color{black} $\,\,$ \\
$\,\,$0.6540$\,\,$ & $\,\,$ 1 $\,\,$ & $\,\,$4.7891$\,\,$ & $\,\,$\color{gr} \color{blue} 5\color{black}   $\,\,$ \\
$\,\,$0.1366$\,\,$ & $\,\,$0.2088$\,\,$ & $\,\,$ 1 $\,\,$ & $\,\,$\color{gr} 1.0440\color{black}  $\,\,$ \\
$\,\,$\color{gr} 0.1308\color{black} $\,\,$ & $\,\,$\color{gr} \color{blue}  1/5\color{black} $\,\,$ & $\,\,$\color{gr} 0.9578\color{black} $\,\,$ & $\,\,$ 1  $\,\,$ \\
\end{pmatrix},
\end{equation*}
\end{example}
\newpage
\begin{example}
\begin{equation*}
\mathbf{A} =
\begin{pmatrix}
$\,\,$ 1 $\,\,$ & $\,\,$3$\,\,$ & $\,\,$5$\,\,$ & $\,\,$8 $\,\,$ \\
$\,\,$ 1/3$\,\,$ & $\,\,$ 1 $\,\,$ & $\,\,$8$\,\,$ & $\,\,$4 $\,\,$ \\
$\,\,$ 1/5$\,\,$ & $\,\,$ 1/8$\,\,$ & $\,\,$ 1 $\,\,$ & $\,\,$1 $\,\,$ \\
$\,\,$ 1/8$\,\,$ & $\,\,$ 1/4$\,\,$ & $\,\,$ 1 $\,\,$ & $\,\,$ 1  $\,\,$ \\
\end{pmatrix},
\qquad
\lambda_{\max} =
4.2162,
\qquad
CR = 0.0815
\end{equation*}

\begin{equation*}
\mathbf{w}^{cos} =
\begin{pmatrix}
0.543511\\
0.315649\\
0.072912\\
\color{red} 0.067928\color{black}
\end{pmatrix}\end{equation*}
\begin{equation*}
\left[ \frac{{w}^{cos}_i}{{w}^{cos}_j} \right] =
\begin{pmatrix}
$\,\,$ 1 $\,\,$ & $\,\,$1.7219$\,\,$ & $\,\,$7.4544$\,\,$ & $\,\,$\color{red} 8.0013\color{black} $\,\,$ \\
$\,\,$0.5808$\,\,$ & $\,\,$ 1 $\,\,$ & $\,\,$4.3292$\,\,$ & $\,\,$\color{red} 4.6468\color{black}   $\,\,$ \\
$\,\,$0.1341$\,\,$ & $\,\,$0.2310$\,\,$ & $\,\,$ 1 $\,\,$ & $\,\,$\color{red} 1.0734\color{black}  $\,\,$ \\
$\,\,$\color{red} 0.1250\color{black} $\,\,$ & $\,\,$\color{red} 0.2152\color{black} $\,\,$ & $\,\,$\color{red} 0.9317\color{black} $\,\,$ & $\,\,$ 1  $\,\,$ \\
\end{pmatrix},
\end{equation*}

\begin{equation*}
\mathbf{w}^{\prime} =
\begin{pmatrix}
0.543505\\
0.315645\\
0.072911\\
0.067938
\end{pmatrix} =
0.999989\cdot
\begin{pmatrix}
0.543511\\
0.315649\\
0.072912\\
\color{gr} 0.067939\color{black}
\end{pmatrix},
\end{equation*}
\begin{equation*}
\left[ \frac{{w}^{\prime}_i}{{w}^{\prime}_j} \right] =
\begin{pmatrix}
$\,\,$ 1 $\,\,$ & $\,\,$1.7219$\,\,$ & $\,\,$7.4544$\,\,$ & $\,\,$\color{gr} \color{blue} 8\color{black} $\,\,$ \\
$\,\,$0.5808$\,\,$ & $\,\,$ 1 $\,\,$ & $\,\,$4.3292$\,\,$ & $\,\,$\color{gr} 4.6461\color{black}   $\,\,$ \\
$\,\,$0.1341$\,\,$ & $\,\,$0.2310$\,\,$ & $\,\,$ 1 $\,\,$ & $\,\,$\color{gr} 1.0732\color{black}  $\,\,$ \\
$\,\,$\color{gr} \color{blue}  1/8\color{black} $\,\,$ & $\,\,$\color{gr} 0.2152\color{black} $\,\,$ & $\,\,$\color{gr} 0.9318\color{black} $\,\,$ & $\,\,$ 1  $\,\,$ \\
\end{pmatrix},
\end{equation*}
\end{example}
\newpage
\begin{example}
\begin{equation*}
\mathbf{A} =
\begin{pmatrix}
$\,\,$ 1 $\,\,$ & $\,\,$3$\,\,$ & $\,\,$5$\,\,$ & $\,\,$8 $\,\,$ \\
$\,\,$ 1/3$\,\,$ & $\,\,$ 1 $\,\,$ & $\,\,$9$\,\,$ & $\,\,$4 $\,\,$ \\
$\,\,$ 1/5$\,\,$ & $\,\,$ 1/9$\,\,$ & $\,\,$ 1 $\,\,$ & $\,\,$1 $\,\,$ \\
$\,\,$ 1/8$\,\,$ & $\,\,$ 1/4$\,\,$ & $\,\,$ 1 $\,\,$ & $\,\,$ 1  $\,\,$ \\
\end{pmatrix},
\qquad
\lambda_{\max} =
4.2541,
\qquad
CR = 0.0958
\end{equation*}

\begin{equation*}
\mathbf{w}^{cos} =
\begin{pmatrix}
0.539714\\
0.322175\\
0.071197\\
\color{red} 0.066914\color{black}
\end{pmatrix}\end{equation*}
\begin{equation*}
\left[ \frac{{w}^{cos}_i}{{w}^{cos}_j} \right] =
\begin{pmatrix}
$\,\,$ 1 $\,\,$ & $\,\,$1.6752$\,\,$ & $\,\,$7.5805$\,\,$ & $\,\,$\color{red} 8.0658\color{black} $\,\,$ \\
$\,\,$0.5969$\,\,$ & $\,\,$ 1 $\,\,$ & $\,\,$4.5251$\,\,$ & $\,\,$\color{red} 4.8147\color{black}   $\,\,$ \\
$\,\,$0.1319$\,\,$ & $\,\,$0.2210$\,\,$ & $\,\,$ 1 $\,\,$ & $\,\,$\color{red} 1.0640\color{black}  $\,\,$ \\
$\,\,$\color{red} 0.1240\color{black} $\,\,$ & $\,\,$\color{red} 0.2077\color{black} $\,\,$ & $\,\,$\color{red} 0.9398\color{black} $\,\,$ & $\,\,$ 1  $\,\,$ \\
\end{pmatrix},
\end{equation*}

\begin{equation*}
\mathbf{w}^{\prime} =
\begin{pmatrix}
0.539417\\
0.321997\\
0.071158\\
0.067427
\end{pmatrix} =
0.999450\cdot
\begin{pmatrix}
0.539714\\
0.322175\\
0.071197\\
\color{gr} 0.067464\color{black}
\end{pmatrix},
\end{equation*}
\begin{equation*}
\left[ \frac{{w}^{\prime}_i}{{w}^{\prime}_j} \right] =
\begin{pmatrix}
$\,\,$ 1 $\,\,$ & $\,\,$1.6752$\,\,$ & $\,\,$7.5805$\,\,$ & $\,\,$\color{gr} \color{blue} 8\color{black} $\,\,$ \\
$\,\,$0.5969$\,\,$ & $\,\,$ 1 $\,\,$ & $\,\,$4.5251$\,\,$ & $\,\,$\color{gr} 4.7755\color{black}   $\,\,$ \\
$\,\,$0.1319$\,\,$ & $\,\,$0.2210$\,\,$ & $\,\,$ 1 $\,\,$ & $\,\,$\color{gr} 1.0553\color{black}  $\,\,$ \\
$\,\,$\color{gr} \color{blue}  1/8\color{black} $\,\,$ & $\,\,$\color{gr} 0.2094\color{black} $\,\,$ & $\,\,$\color{gr} 0.9476\color{black} $\,\,$ & $\,\,$ 1  $\,\,$ \\
\end{pmatrix},
\end{equation*}
\end{example}
\newpage
\begin{example}
\begin{equation*}
\mathbf{A} =
\begin{pmatrix}
$\,\,$ 1 $\,\,$ & $\,\,$3$\,\,$ & $\,\,$5$\,\,$ & $\,\,$8 $\,\,$ \\
$\,\,$ 1/3$\,\,$ & $\,\,$ 1 $\,\,$ & $\,\,$9$\,\,$ & $\,\,$5 $\,\,$ \\
$\,\,$ 1/5$\,\,$ & $\,\,$ 1/9$\,\,$ & $\,\,$ 1 $\,\,$ & $\,\,$1 $\,\,$ \\
$\,\,$ 1/8$\,\,$ & $\,\,$ 1/5$\,\,$ & $\,\,$ 1 $\,\,$ & $\,\,$ 1  $\,\,$ \\
\end{pmatrix},
\qquad
\lambda_{\max} =
4.2489,
\qquad
CR = 0.0939
\end{equation*}

\begin{equation*}
\mathbf{w}^{cos} =
\begin{pmatrix}
0.531336\\
0.335301\\
0.070127\\
\color{red} 0.063236\color{black}
\end{pmatrix}\end{equation*}
\begin{equation*}
\left[ \frac{{w}^{cos}_i}{{w}^{cos}_j} \right] =
\begin{pmatrix}
$\,\,$ 1 $\,\,$ & $\,\,$1.5847$\,\,$ & $\,\,$7.5767$\,\,$ & $\,\,$\color{red} 8.4025\color{black} $\,\,$ \\
$\,\,$0.6311$\,\,$ & $\,\,$ 1 $\,\,$ & $\,\,$4.7813$\,\,$ & $\,\,$\color{red} 5.3024\color{black}   $\,\,$ \\
$\,\,$0.1320$\,\,$ & $\,\,$0.2091$\,\,$ & $\,\,$ 1 $\,\,$ & $\,\,$\color{red} 1.1090\color{black}  $\,\,$ \\
$\,\,$\color{red} 0.1190\color{black} $\,\,$ & $\,\,$\color{red} 0.1886\color{black} $\,\,$ & $\,\,$\color{red} 0.9017\color{black} $\,\,$ & $\,\,$ 1  $\,\,$ \\
\end{pmatrix},
\end{equation*}

\begin{equation*}
\mathbf{w}^{\prime} =
\begin{pmatrix}
0.529651\\
0.334238\\
0.069905\\
0.066206
\end{pmatrix} =
0.996829\cdot
\begin{pmatrix}
0.531336\\
0.335301\\
0.070127\\
\color{gr} 0.066417\color{black}
\end{pmatrix},
\end{equation*}
\begin{equation*}
\left[ \frac{{w}^{\prime}_i}{{w}^{\prime}_j} \right] =
\begin{pmatrix}
$\,\,$ 1 $\,\,$ & $\,\,$1.5847$\,\,$ & $\,\,$7.5767$\,\,$ & $\,\,$\color{gr} \color{blue} 8\color{black} $\,\,$ \\
$\,\,$0.6311$\,\,$ & $\,\,$ 1 $\,\,$ & $\,\,$4.7813$\,\,$ & $\,\,$\color{gr} 5.0484\color{black}   $\,\,$ \\
$\,\,$0.1320$\,\,$ & $\,\,$0.2091$\,\,$ & $\,\,$ 1 $\,\,$ & $\,\,$\color{gr} 1.0559\color{black}  $\,\,$ \\
$\,\,$\color{gr} \color{blue}  1/8\color{black} $\,\,$ & $\,\,$\color{gr} 0.1981\color{black} $\,\,$ & $\,\,$\color{gr} 0.9471\color{black} $\,\,$ & $\,\,$ 1  $\,\,$ \\
\end{pmatrix},
\end{equation*}
\end{example}
\newpage
\begin{example}
\begin{equation*}
\mathbf{A} =
\begin{pmatrix}
$\,\,$ 1 $\,\,$ & $\,\,$3$\,\,$ & $\,\,$5$\,\,$ & $\,\,$9 $\,\,$ \\
$\,\,$ 1/3$\,\,$ & $\,\,$ 1 $\,\,$ & $\,\,$3$\,\,$ & $\,\,$2 $\,\,$ \\
$\,\,$ 1/5$\,\,$ & $\,\,$ 1/3$\,\,$ & $\,\,$ 1 $\,\,$ & $\,\,$3 $\,\,$ \\
$\,\,$ 1/9$\,\,$ & $\,\,$ 1/2$\,\,$ & $\,\,$ 1/3$\,\,$ & $\,\,$ 1  $\,\,$ \\
\end{pmatrix},
\qquad
\lambda_{\max} =
4.1966,
\qquad
CR = 0.0741
\end{equation*}

\begin{equation*}
\mathbf{w}^{cos} =
\begin{pmatrix}
\color{red} 0.590582\color{black} \\
0.216798\\
0.124675\\
0.067945
\end{pmatrix}\end{equation*}
\begin{equation*}
\left[ \frac{{w}^{cos}_i}{{w}^{cos}_j} \right] =
\begin{pmatrix}
$\,\,$ 1 $\,\,$ & $\,\,$\color{red} 2.7241\color{black} $\,\,$ & $\,\,$\color{red} 4.7370\color{black} $\,\,$ & $\,\,$\color{red} 8.6920\color{black} $\,\,$ \\
$\,\,$\color{red} 0.3671\color{black} $\,\,$ & $\,\,$ 1 $\,\,$ & $\,\,$1.7389$\,\,$ & $\,\,$3.1908  $\,\,$ \\
$\,\,$\color{red} 0.2111\color{black} $\,\,$ & $\,\,$0.5751$\,\,$ & $\,\,$ 1 $\,\,$ & $\,\,$1.8349 $\,\,$ \\
$\,\,$\color{red} 0.1150\color{black} $\,\,$ & $\,\,$0.3134$\,\,$ & $\,\,$0.5450$\,\,$ & $\,\,$ 1  $\,\,$ \\
\end{pmatrix},
\end{equation*}

\begin{equation*}
\mathbf{w}^{\prime} =
\begin{pmatrix}
0.598973\\
0.212355\\
0.122120\\
0.066553
\end{pmatrix} =
0.979505\cdot
\begin{pmatrix}
\color{gr} 0.611505\color{black} \\
0.216798\\
0.124675\\
0.067945
\end{pmatrix},
\end{equation*}
\begin{equation*}
\left[ \frac{{w}^{\prime}_i}{{w}^{\prime}_j} \right] =
\begin{pmatrix}
$\,\,$ 1 $\,\,$ & $\,\,$\color{gr} 2.8206\color{black} $\,\,$ & $\,\,$\color{gr} 4.9048\color{black} $\,\,$ & $\,\,$\color{gr} \color{blue} 9\color{black} $\,\,$ \\
$\,\,$\color{gr} 0.3545\color{black} $\,\,$ & $\,\,$ 1 $\,\,$ & $\,\,$1.7389$\,\,$ & $\,\,$3.1908  $\,\,$ \\
$\,\,$\color{gr} 0.2039\color{black} $\,\,$ & $\,\,$0.5751$\,\,$ & $\,\,$ 1 $\,\,$ & $\,\,$1.8349 $\,\,$ \\
$\,\,$\color{gr} \color{blue}  1/9\color{black} $\,\,$ & $\,\,$0.3134$\,\,$ & $\,\,$0.5450$\,\,$ & $\,\,$ 1  $\,\,$ \\
\end{pmatrix},
\end{equation*}
\end{example}
\newpage
\begin{example}
\begin{equation*}
\mathbf{A} =
\begin{pmatrix}
$\,\,$ 1 $\,\,$ & $\,\,$3$\,\,$ & $\,\,$6$\,\,$ & $\,\,$8 $\,\,$ \\
$\,\,$ 1/3$\,\,$ & $\,\,$ 1 $\,\,$ & $\,\,$3$\,\,$ & $\,\,$2 $\,\,$ \\
$\,\,$ 1/6$\,\,$ & $\,\,$ 1/3$\,\,$ & $\,\,$ 1 $\,\,$ & $\,\,$2 $\,\,$ \\
$\,\,$ 1/8$\,\,$ & $\,\,$ 1/2$\,\,$ & $\,\,$ 1/2$\,\,$ & $\,\,$ 1  $\,\,$ \\
\end{pmatrix},
\qquad
\lambda_{\max} =
4.1031,
\qquad
CR = 0.0389
\end{equation*}

\begin{equation*}
\mathbf{w}^{cos} =
\begin{pmatrix}
\color{red} 0.605532\color{black} \\
0.213196\\
0.105240\\
0.076032
\end{pmatrix}\end{equation*}
\begin{equation*}
\left[ \frac{{w}^{cos}_i}{{w}^{cos}_j} \right] =
\begin{pmatrix}
$\,\,$ 1 $\,\,$ & $\,\,$\color{red} 2.8403\color{black} $\,\,$ & $\,\,$\color{red} 5.7538\color{black} $\,\,$ & $\,\,$\color{red} 7.9641\color{black} $\,\,$ \\
$\,\,$\color{red} 0.3521\color{black} $\,\,$ & $\,\,$ 1 $\,\,$ & $\,\,$2.0258$\,\,$ & $\,\,$2.8040  $\,\,$ \\
$\,\,$\color{red} 0.1738\color{black} $\,\,$ & $\,\,$0.4936$\,\,$ & $\,\,$ 1 $\,\,$ & $\,\,$1.3841 $\,\,$ \\
$\,\,$\color{red} 0.1256\color{black} $\,\,$ & $\,\,$0.3566$\,\,$ & $\,\,$0.7225$\,\,$ & $\,\,$ 1  $\,\,$ \\
\end{pmatrix},
\end{equation*}

\begin{equation*}
\mathbf{w}^{\prime} =
\begin{pmatrix}
0.606605\\
0.212616\\
0.104953\\
0.075826
\end{pmatrix} =
0.997280\cdot
\begin{pmatrix}
\color{gr} 0.608259\color{black} \\
0.213196\\
0.105240\\
0.076032
\end{pmatrix},
\end{equation*}
\begin{equation*}
\left[ \frac{{w}^{\prime}_i}{{w}^{\prime}_j} \right] =
\begin{pmatrix}
$\,\,$ 1 $\,\,$ & $\,\,$\color{gr} 2.8530\color{black} $\,\,$ & $\,\,$\color{gr} 5.7798\color{black} $\,\,$ & $\,\,$\color{gr} \color{blue} 8\color{black} $\,\,$ \\
$\,\,$\color{gr} 0.3505\color{black} $\,\,$ & $\,\,$ 1 $\,\,$ & $\,\,$2.0258$\,\,$ & $\,\,$2.8040  $\,\,$ \\
$\,\,$\color{gr} 0.1730\color{black} $\,\,$ & $\,\,$0.4936$\,\,$ & $\,\,$ 1 $\,\,$ & $\,\,$1.3841 $\,\,$ \\
$\,\,$\color{gr} \color{blue}  1/8\color{black} $\,\,$ & $\,\,$0.3566$\,\,$ & $\,\,$0.7225$\,\,$ & $\,\,$ 1  $\,\,$ \\
\end{pmatrix},
\end{equation*}
\end{example}
\newpage
\begin{example}
\begin{equation*}
\mathbf{A} =
\begin{pmatrix}
$\,\,$ 1 $\,\,$ & $\,\,$3$\,\,$ & $\,\,$6$\,\,$ & $\,\,$8 $\,\,$ \\
$\,\,$ 1/3$\,\,$ & $\,\,$ 1 $\,\,$ & $\,\,$4$\,\,$ & $\,\,$2 $\,\,$ \\
$\,\,$ 1/6$\,\,$ & $\,\,$ 1/4$\,\,$ & $\,\,$ 1 $\,\,$ & $\,\,$2 $\,\,$ \\
$\,\,$ 1/8$\,\,$ & $\,\,$ 1/2$\,\,$ & $\,\,$ 1/2$\,\,$ & $\,\,$ 1  $\,\,$ \\
\end{pmatrix},
\qquad
\lambda_{\max} =
4.1707,
\qquad
CR = 0.0644
\end{equation*}

\begin{equation*}
\mathbf{w}^{cos} =
\begin{pmatrix}
\color{red} 0.595081\color{black} \\
0.230558\\
0.099209\\
0.075151
\end{pmatrix}\end{equation*}
\begin{equation*}
\left[ \frac{{w}^{cos}_i}{{w}^{cos}_j} \right] =
\begin{pmatrix}
$\,\,$ 1 $\,\,$ & $\,\,$\color{red} 2.5811\color{black} $\,\,$ & $\,\,$\color{red} 5.9982\color{black} $\,\,$ & $\,\,$\color{red} 7.9184\color{black} $\,\,$ \\
$\,\,$\color{red} 0.3874\color{black} $\,\,$ & $\,\,$ 1 $\,\,$ & $\,\,$2.3240$\,\,$ & $\,\,$3.0679  $\,\,$ \\
$\,\,$\color{red} 0.1667\color{black} $\,\,$ & $\,\,$0.4303$\,\,$ & $\,\,$ 1 $\,\,$ & $\,\,$1.3201 $\,\,$ \\
$\,\,$\color{red} 0.1263\color{black} $\,\,$ & $\,\,$0.3260$\,\,$ & $\,\,$0.7575$\,\,$ & $\,\,$ 1  $\,\,$ \\
\end{pmatrix},
\end{equation*}

\begin{equation*}
\mathbf{w}^{\prime} =
\begin{pmatrix}
0.595152\\
0.230517\\
0.099192\\
0.075138
\end{pmatrix} =
0.999825\cdot
\begin{pmatrix}
\color{gr} 0.595256\color{black} \\
0.230558\\
0.099209\\
0.075151
\end{pmatrix},
\end{equation*}
\begin{equation*}
\left[ \frac{{w}^{\prime}_i}{{w}^{\prime}_j} \right] =
\begin{pmatrix}
$\,\,$ 1 $\,\,$ & $\,\,$\color{gr} 2.5818\color{black} $\,\,$ & $\,\,$\color{gr} \color{blue} 6\color{black} $\,\,$ & $\,\,$\color{gr} 7.9208\color{black} $\,\,$ \\
$\,\,$\color{gr} 0.3873\color{black} $\,\,$ & $\,\,$ 1 $\,\,$ & $\,\,$2.3240$\,\,$ & $\,\,$3.0679  $\,\,$ \\
$\,\,$\color{gr} \color{blue}  1/6\color{black} $\,\,$ & $\,\,$0.4303$\,\,$ & $\,\,$ 1 $\,\,$ & $\,\,$1.3201 $\,\,$ \\
$\,\,$\color{gr} 0.1263\color{black} $\,\,$ & $\,\,$0.3260$\,\,$ & $\,\,$0.7575$\,\,$ & $\,\,$ 1  $\,\,$ \\
\end{pmatrix},
\end{equation*}
\end{example}
\newpage
\begin{example}
\begin{equation*}
\mathbf{A} =
\begin{pmatrix}
$\,\,$ 1 $\,\,$ & $\,\,$3$\,\,$ & $\,\,$6$\,\,$ & $\,\,$8 $\,\,$ \\
$\,\,$ 1/3$\,\,$ & $\,\,$ 1 $\,\,$ & $\,\,$6$\,\,$ & $\,\,$4 $\,\,$ \\
$\,\,$ 1/6$\,\,$ & $\,\,$ 1/6$\,\,$ & $\,\,$ 1 $\,\,$ & $\,\,$1 $\,\,$ \\
$\,\,$ 1/8$\,\,$ & $\,\,$ 1/4$\,\,$ & $\,\,$ 1 $\,\,$ & $\,\,$ 1  $\,\,$ \\
\end{pmatrix},
\qquad
\lambda_{\max} =
4.1031,
\qquad
CR = 0.0389
\end{equation*}

\begin{equation*}
\mathbf{w}^{cos} =
\begin{pmatrix}
0.568778\\
0.290350\\
0.071468\\
\color{red} 0.069404\color{black}
\end{pmatrix}\end{equation*}
\begin{equation*}
\left[ \frac{{w}^{cos}_i}{{w}^{cos}_j} \right] =
\begin{pmatrix}
$\,\,$ 1 $\,\,$ & $\,\,$1.9589$\,\,$ & $\,\,$7.9585$\,\,$ & $\,\,$\color{red} 8.1952\color{black} $\,\,$ \\
$\,\,$0.5105$\,\,$ & $\,\,$ 1 $\,\,$ & $\,\,$4.0626$\,\,$ & $\,\,$\color{red} 4.1835\color{black}   $\,\,$ \\
$\,\,$0.1257$\,\,$ & $\,\,$0.2461$\,\,$ & $\,\,$ 1 $\,\,$ & $\,\,$\color{red} 1.0297\color{black}  $\,\,$ \\
$\,\,$\color{red} 0.1220\color{black} $\,\,$ & $\,\,$\color{red} 0.2390\color{black} $\,\,$ & $\,\,$\color{red} 0.9711\color{black} $\,\,$ & $\,\,$ 1  $\,\,$ \\
\end{pmatrix},
\end{equation*}

\begin{equation*}
\mathbf{w}^{\prime} =
\begin{pmatrix}
0.567816\\
0.289860\\
0.071347\\
0.070977
\end{pmatrix} =
0.998310\cdot
\begin{pmatrix}
0.568778\\
0.290350\\
0.071468\\
\color{gr} 0.071097\color{black}
\end{pmatrix},
\end{equation*}
\begin{equation*}
\left[ \frac{{w}^{\prime}_i}{{w}^{\prime}_j} \right] =
\begin{pmatrix}
$\,\,$ 1 $\,\,$ & $\,\,$1.9589$\,\,$ & $\,\,$7.9585$\,\,$ & $\,\,$\color{gr} \color{blue} 8\color{black} $\,\,$ \\
$\,\,$0.5105$\,\,$ & $\,\,$ 1 $\,\,$ & $\,\,$4.0626$\,\,$ & $\,\,$\color{gr} 4.0839\color{black}   $\,\,$ \\
$\,\,$0.1257$\,\,$ & $\,\,$0.2461$\,\,$ & $\,\,$ 1 $\,\,$ & $\,\,$\color{gr} 1.0052\color{black}  $\,\,$ \\
$\,\,$\color{gr} \color{blue}  1/8\color{black} $\,\,$ & $\,\,$\color{gr} 0.2449\color{black} $\,\,$ & $\,\,$\color{gr} 0.9948\color{black} $\,\,$ & $\,\,$ 1  $\,\,$ \\
\end{pmatrix},
\end{equation*}
\end{example}
\newpage
\begin{example}
\begin{equation*}
\mathbf{A} =
\begin{pmatrix}
$\,\,$ 1 $\,\,$ & $\,\,$3$\,\,$ & $\,\,$6$\,\,$ & $\,\,$8 $\,\,$ \\
$\,\,$ 1/3$\,\,$ & $\,\,$ 1 $\,\,$ & $\,\,$7$\,\,$ & $\,\,$4 $\,\,$ \\
$\,\,$ 1/6$\,\,$ & $\,\,$ 1/7$\,\,$ & $\,\,$ 1 $\,\,$ & $\,\,$1 $\,\,$ \\
$\,\,$ 1/8$\,\,$ & $\,\,$ 1/4$\,\,$ & $\,\,$ 1 $\,\,$ & $\,\,$ 1  $\,\,$ \\
\end{pmatrix},
\qquad
\lambda_{\max} =
4.1365,
\qquad
CR = 0.0515
\end{equation*}

\begin{equation*}
\mathbf{w}^{cos} =
\begin{pmatrix}
0.562344\\
0.300390\\
0.069049\\
\color{red} 0.068217\color{black}
\end{pmatrix}\end{equation*}
\begin{equation*}
\left[ \frac{{w}^{cos}_i}{{w}^{cos}_j} \right] =
\begin{pmatrix}
$\,\,$ 1 $\,\,$ & $\,\,$1.8720$\,\,$ & $\,\,$8.1441$\,\,$ & $\,\,$\color{red} 8.2435\color{black} $\,\,$ \\
$\,\,$0.5342$\,\,$ & $\,\,$ 1 $\,\,$ & $\,\,$4.3504$\,\,$ & $\,\,$\color{red} 4.4035\color{black}   $\,\,$ \\
$\,\,$0.1228$\,\,$ & $\,\,$0.2299$\,\,$ & $\,\,$ 1 $\,\,$ & $\,\,$\color{red} 1.0122\color{black}  $\,\,$ \\
$\,\,$\color{red} 0.1213\color{black} $\,\,$ & $\,\,$\color{red} 0.2271\color{black} $\,\,$ & $\,\,$\color{red} 0.9879\color{black} $\,\,$ & $\,\,$ 1  $\,\,$ \\
\end{pmatrix},
\end{equation*}

\begin{equation*}
\mathbf{w}^{\prime} =
\begin{pmatrix}
0.561877\\
0.300140\\
0.068992\\
0.068992
\end{pmatrix} =
0.999168\cdot
\begin{pmatrix}
0.562344\\
0.300390\\
0.069049\\
\color{gr} 0.069049\color{black}
\end{pmatrix},
\end{equation*}
\begin{equation*}
\left[ \frac{{w}^{\prime}_i}{{w}^{\prime}_j} \right] =
\begin{pmatrix}
$\,\,$ 1 $\,\,$ & $\,\,$1.8720$\,\,$ & $\,\,$8.1441$\,\,$ & $\,\,$\color{gr} 8.1441\color{black} $\,\,$ \\
$\,\,$0.5342$\,\,$ & $\,\,$ 1 $\,\,$ & $\,\,$4.3504$\,\,$ & $\,\,$\color{gr} 4.3504\color{black}   $\,\,$ \\
$\,\,$0.1228$\,\,$ & $\,\,$0.2299$\,\,$ & $\,\,$ 1 $\,\,$ & $\,\,$\color{gr} \color{blue} 1\color{black}  $\,\,$ \\
$\,\,$\color{gr} 0.1228\color{black} $\,\,$ & $\,\,$\color{gr} 0.2299\color{black} $\,\,$ & $\,\,$\color{gr} \color{blue} 1\color{black} $\,\,$ & $\,\,$ 1  $\,\,$ \\
\end{pmatrix},
\end{equation*}
\end{example}
\newpage
\begin{example}
\begin{equation*}
\mathbf{A} =
\begin{pmatrix}
$\,\,$ 1 $\,\,$ & $\,\,$3$\,\,$ & $\,\,$6$\,\,$ & $\,\,$8 $\,\,$ \\
$\,\,$ 1/3$\,\,$ & $\,\,$ 1 $\,\,$ & $\,\,$8$\,\,$ & $\,\,$5 $\,\,$ \\
$\,\,$ 1/6$\,\,$ & $\,\,$ 1/8$\,\,$ & $\,\,$ 1 $\,\,$ & $\,\,$1 $\,\,$ \\
$\,\,$ 1/8$\,\,$ & $\,\,$ 1/5$\,\,$ & $\,\,$ 1 $\,\,$ & $\,\,$ 1  $\,\,$ \\
\end{pmatrix},
\qquad
\lambda_{\max} =
4.1689,
\qquad
CR = 0.0637
\end{equation*}

\begin{equation*}
\mathbf{w}^{cos} =
\begin{pmatrix}
0.548566\\
0.321874\\
0.066034\\
\color{red} 0.063526\color{black}
\end{pmatrix}\end{equation*}
\begin{equation*}
\left[ \frac{{w}^{cos}_i}{{w}^{cos}_j} \right] =
\begin{pmatrix}
$\,\,$ 1 $\,\,$ & $\,\,$1.7043$\,\,$ & $\,\,$8.3074$\,\,$ & $\,\,$\color{red} 8.6353\color{black} $\,\,$ \\
$\,\,$0.5868$\,\,$ & $\,\,$ 1 $\,\,$ & $\,\,$4.8744$\,\,$ & $\,\,$\color{red} 5.0668\color{black}   $\,\,$ \\
$\,\,$0.1204$\,\,$ & $\,\,$0.2052$\,\,$ & $\,\,$ 1 $\,\,$ & $\,\,$\color{red} 1.0395\color{black}  $\,\,$ \\
$\,\,$\color{red} 0.1158\color{black} $\,\,$ & $\,\,$\color{red} 0.1974\color{black} $\,\,$ & $\,\,$\color{red} 0.9620\color{black} $\,\,$ & $\,\,$ 1  $\,\,$ \\
\end{pmatrix},
\end{equation*}

\begin{equation*}
\mathbf{w}^{\prime} =
\begin{pmatrix}
0.548101\\
0.321601\\
0.065978\\
0.064320
\end{pmatrix} =
0.999152\cdot
\begin{pmatrix}
0.548566\\
0.321874\\
0.066034\\
\color{gr} 0.064375\color{black}
\end{pmatrix},
\end{equation*}
\begin{equation*}
\left[ \frac{{w}^{\prime}_i}{{w}^{\prime}_j} \right] =
\begin{pmatrix}
$\,\,$ 1 $\,\,$ & $\,\,$1.7043$\,\,$ & $\,\,$8.3074$\,\,$ & $\,\,$\color{gr} 8.5214\color{black} $\,\,$ \\
$\,\,$0.5868$\,\,$ & $\,\,$ 1 $\,\,$ & $\,\,$4.8744$\,\,$ & $\,\,$\color{gr} \color{blue} 5\color{black}   $\,\,$ \\
$\,\,$0.1204$\,\,$ & $\,\,$0.2052$\,\,$ & $\,\,$ 1 $\,\,$ & $\,\,$\color{gr} 1.0258\color{black}  $\,\,$ \\
$\,\,$\color{gr} 0.1174\color{black} $\,\,$ & $\,\,$\color{gr} \color{blue}  1/5\color{black} $\,\,$ & $\,\,$\color{gr} 0.9749\color{black} $\,\,$ & $\,\,$ 1  $\,\,$ \\
\end{pmatrix},
\end{equation*}
\end{example}
\newpage
\begin{example}
\begin{equation*}
\mathbf{A} =
\begin{pmatrix}
$\,\,$ 1 $\,\,$ & $\,\,$3$\,\,$ & $\,\,$6$\,\,$ & $\,\,$8 $\,\,$ \\
$\,\,$ 1/3$\,\,$ & $\,\,$ 1 $\,\,$ & $\,\,$9$\,\,$ & $\,\,$5 $\,\,$ \\
$\,\,$ 1/6$\,\,$ & $\,\,$ 1/9$\,\,$ & $\,\,$ 1 $\,\,$ & $\,\,$1 $\,\,$ \\
$\,\,$ 1/8$\,\,$ & $\,\,$ 1/5$\,\,$ & $\,\,$ 1 $\,\,$ & $\,\,$ 1  $\,\,$ \\
\end{pmatrix},
\qquad
\lambda_{\max} =
4.1999,
\qquad
CR = 0.0754
\end{equation*}

\begin{equation*}
\mathbf{w}^{cos} =
\begin{pmatrix}
0.543992\\
0.328974\\
0.064412\\
\color{red} 0.062622\color{black}
\end{pmatrix}\end{equation*}
\begin{equation*}
\left[ \frac{{w}^{cos}_i}{{w}^{cos}_j} \right] =
\begin{pmatrix}
$\,\,$ 1 $\,\,$ & $\,\,$1.6536$\,\,$ & $\,\,$8.4456$\,\,$ & $\,\,$\color{red} 8.6869\color{black} $\,\,$ \\
$\,\,$0.6047$\,\,$ & $\,\,$ 1 $\,\,$ & $\,\,$5.1074$\,\,$ & $\,\,$\color{red} 5.2533\color{black}   $\,\,$ \\
$\,\,$0.1184$\,\,$ & $\,\,$0.1958$\,\,$ & $\,\,$ 1 $\,\,$ & $\,\,$\color{red} 1.0286\color{black}  $\,\,$ \\
$\,\,$\color{red} 0.1151\color{black} $\,\,$ & $\,\,$\color{red} 0.1904\color{black} $\,\,$ & $\,\,$\color{red} 0.9722\color{black} $\,\,$ & $\,\,$ 1  $\,\,$ \\
\end{pmatrix},
\end{equation*}

\begin{equation*}
\mathbf{w}^{\prime} =
\begin{pmatrix}
0.543021\\
0.328386\\
0.064296\\
0.064296
\end{pmatrix} =
0.998214\cdot
\begin{pmatrix}
0.543992\\
0.328974\\
0.064412\\
\color{gr} 0.064412\color{black}
\end{pmatrix},
\end{equation*}
\begin{equation*}
\left[ \frac{{w}^{\prime}_i}{{w}^{\prime}_j} \right] =
\begin{pmatrix}
$\,\,$ 1 $\,\,$ & $\,\,$1.6536$\,\,$ & $\,\,$8.4456$\,\,$ & $\,\,$\color{gr} 8.4456\color{black} $\,\,$ \\
$\,\,$0.6047$\,\,$ & $\,\,$ 1 $\,\,$ & $\,\,$5.1074$\,\,$ & $\,\,$\color{gr} 5.1074\color{black}   $\,\,$ \\
$\,\,$0.1184$\,\,$ & $\,\,$0.1958$\,\,$ & $\,\,$ 1 $\,\,$ & $\,\,$\color{gr} \color{blue} 1\color{black}  $\,\,$ \\
$\,\,$\color{gr} 0.1184\color{black} $\,\,$ & $\,\,$\color{gr} 0.1958\color{black} $\,\,$ & $\,\,$\color{gr} \color{blue} 1\color{black} $\,\,$ & $\,\,$ 1  $\,\,$ \\
\end{pmatrix},
\end{equation*}
\end{example}
\newpage
\begin{example}
\begin{equation*}
\mathbf{A} =
\begin{pmatrix}
$\,\,$ 1 $\,\,$ & $\,\,$3$\,\,$ & $\,\,$6$\,\,$ & $\,\,$9 $\,\,$ \\
$\,\,$ 1/3$\,\,$ & $\,\,$ 1 $\,\,$ & $\,\,$3$\,\,$ & $\,\,$2 $\,\,$ \\
$\,\,$ 1/6$\,\,$ & $\,\,$ 1/3$\,\,$ & $\,\,$ 1 $\,\,$ & $\,\,$2 $\,\,$ \\
$\,\,$ 1/9$\,\,$ & $\,\,$ 1/2$\,\,$ & $\,\,$ 1/2$\,\,$ & $\,\,$ 1  $\,\,$ \\
\end{pmatrix},
\qquad
\lambda_{\max} =
4.1031,
\qquad
CR = 0.0389
\end{equation*}

\begin{equation*}
\mathbf{w}^{cos} =
\begin{pmatrix}
\color{red} 0.613493\color{black} \\
0.211410\\
0.102403\\
0.072694
\end{pmatrix}\end{equation*}
\begin{equation*}
\left[ \frac{{w}^{cos}_i}{{w}^{cos}_j} \right] =
\begin{pmatrix}
$\,\,$ 1 $\,\,$ & $\,\,$\color{red} 2.9019\color{black} $\,\,$ & $\,\,$\color{red} 5.9910\color{black} $\,\,$ & $\,\,$\color{red} 8.4394\color{black} $\,\,$ \\
$\,\,$\color{red} 0.3446\color{black} $\,\,$ & $\,\,$ 1 $\,\,$ & $\,\,$2.0645$\,\,$ & $\,\,$2.9082  $\,\,$ \\
$\,\,$\color{red} 0.1669\color{black} $\,\,$ & $\,\,$0.4844$\,\,$ & $\,\,$ 1 $\,\,$ & $\,\,$1.4087 $\,\,$ \\
$\,\,$\color{red} 0.1185\color{black} $\,\,$ & $\,\,$0.3439$\,\,$ & $\,\,$0.7099$\,\,$ & $\,\,$ 1  $\,\,$ \\
\end{pmatrix},
\end{equation*}

\begin{equation*}
\mathbf{w}^{\prime} =
\begin{pmatrix}
0.613850\\
0.211215\\
0.102308\\
0.072627
\end{pmatrix} =
0.999078\cdot
\begin{pmatrix}
\color{gr} 0.614416\color{black} \\
0.211410\\
0.102403\\
0.072694
\end{pmatrix},
\end{equation*}
\begin{equation*}
\left[ \frac{{w}^{\prime}_i}{{w}^{\prime}_j} \right] =
\begin{pmatrix}
$\,\,$ 1 $\,\,$ & $\,\,$\color{gr} 2.9063\color{black} $\,\,$ & $\,\,$\color{gr} \color{blue} 6\color{black} $\,\,$ & $\,\,$\color{gr} 8.4521\color{black} $\,\,$ \\
$\,\,$\color{gr} 0.3441\color{black} $\,\,$ & $\,\,$ 1 $\,\,$ & $\,\,$2.0645$\,\,$ & $\,\,$2.9082  $\,\,$ \\
$\,\,$\color{gr} \color{blue}  1/6\color{black} $\,\,$ & $\,\,$0.4844$\,\,$ & $\,\,$ 1 $\,\,$ & $\,\,$1.4087 $\,\,$ \\
$\,\,$\color{gr} 0.1183\color{black} $\,\,$ & $\,\,$0.3439$\,\,$ & $\,\,$0.7099$\,\,$ & $\,\,$ 1  $\,\,$ \\
\end{pmatrix},
\end{equation*}
\end{example}
\newpage
\begin{example}
\begin{equation*}
\mathbf{A} =
\begin{pmatrix}
$\,\,$ 1 $\,\,$ & $\,\,$3$\,\,$ & $\,\,$6$\,\,$ & $\,\,$9 $\,\,$ \\
$\,\,$ 1/3$\,\,$ & $\,\,$ 1 $\,\,$ & $\,\,$3$\,\,$ & $\,\,$2 $\,\,$ \\
$\,\,$ 1/6$\,\,$ & $\,\,$ 1/3$\,\,$ & $\,\,$ 1 $\,\,$ & $\,\,$3 $\,\,$ \\
$\,\,$ 1/9$\,\,$ & $\,\,$ 1/2$\,\,$ & $\,\,$ 1/3$\,\,$ & $\,\,$ 1  $\,\,$ \\
\end{pmatrix},
\qquad
\lambda_{\max} =
4.1990,
\qquad
CR = 0.0750
\end{equation*}

\begin{equation*}
\mathbf{w}^{cos} =
\begin{pmatrix}
\color{red} 0.605389\color{black} \\
0.209163\\
0.117739\\
0.067709
\end{pmatrix}\end{equation*}
\begin{equation*}
\left[ \frac{{w}^{cos}_i}{{w}^{cos}_j} \right] =
\begin{pmatrix}
$\,\,$ 1 $\,\,$ & $\,\,$\color{red} 2.8943\color{black} $\,\,$ & $\,\,$\color{red} 5.1418\color{black} $\,\,$ & $\,\,$\color{red} 8.9410\color{black} $\,\,$ \\
$\,\,$\color{red} 0.3455\color{black} $\,\,$ & $\,\,$ 1 $\,\,$ & $\,\,$1.7765$\,\,$ & $\,\,$3.0891  $\,\,$ \\
$\,\,$\color{red} 0.1945\color{black} $\,\,$ & $\,\,$0.5629$\,\,$ & $\,\,$ 1 $\,\,$ & $\,\,$1.7389 $\,\,$ \\
$\,\,$\color{red} 0.1118\color{black} $\,\,$ & $\,\,$0.3237$\,\,$ & $\,\,$0.5751$\,\,$ & $\,\,$ 1  $\,\,$ \\
\end{pmatrix},
\end{equation*}

\begin{equation*}
\mathbf{w}^{\prime} =
\begin{pmatrix}
0.606959\\
0.208331\\
0.117271\\
0.067440
\end{pmatrix} =
0.996023\cdot
\begin{pmatrix}
\color{gr} 0.609382\color{black} \\
0.209163\\
0.117739\\
0.067709
\end{pmatrix},
\end{equation*}
\begin{equation*}
\left[ \frac{{w}^{\prime}_i}{{w}^{\prime}_j} \right] =
\begin{pmatrix}
$\,\,$ 1 $\,\,$ & $\,\,$\color{gr} 2.9134\color{black} $\,\,$ & $\,\,$\color{gr} 5.1757\color{black} $\,\,$ & $\,\,$\color{gr} \color{blue} 9\color{black} $\,\,$ \\
$\,\,$\color{gr} 0.3432\color{black} $\,\,$ & $\,\,$ 1 $\,\,$ & $\,\,$1.7765$\,\,$ & $\,\,$3.0891  $\,\,$ \\
$\,\,$\color{gr} 0.1932\color{black} $\,\,$ & $\,\,$0.5629$\,\,$ & $\,\,$ 1 $\,\,$ & $\,\,$1.7389 $\,\,$ \\
$\,\,$\color{gr} \color{blue}  1/9\color{black} $\,\,$ & $\,\,$0.3237$\,\,$ & $\,\,$0.5751$\,\,$ & $\,\,$ 1  $\,\,$ \\
\end{pmatrix},
\end{equation*}
\end{example}
\newpage
\begin{example}
\begin{equation*}
\mathbf{A} =
\begin{pmatrix}
$\,\,$ 1 $\,\,$ & $\,\,$3$\,\,$ & $\,\,$6$\,\,$ & $\,\,$9 $\,\,$ \\
$\,\,$ 1/3$\,\,$ & $\,\,$ 1 $\,\,$ & $\,\,$7$\,\,$ & $\,\,$5 $\,\,$ \\
$\,\,$ 1/6$\,\,$ & $\,\,$ 1/7$\,\,$ & $\,\,$ 1 $\,\,$ & $\,\,$1 $\,\,$ \\
$\,\,$ 1/9$\,\,$ & $\,\,$ 1/5$\,\,$ & $\,\,$ 1 $\,\,$ & $\,\,$ 1  $\,\,$ \\
\end{pmatrix},
\qquad
\lambda_{\max} =
4.1351,
\qquad
CR = 0.0509
\end{equation*}

\begin{equation*}
\mathbf{w}^{cos} =
\begin{pmatrix}
0.562783\\
0.308638\\
0.067094\\
\color{red} 0.061485\color{black}
\end{pmatrix}\end{equation*}
\begin{equation*}
\left[ \frac{{w}^{cos}_i}{{w}^{cos}_j} \right] =
\begin{pmatrix}
$\,\,$ 1 $\,\,$ & $\,\,$1.8234$\,\,$ & $\,\,$8.3879$\,\,$ & $\,\,$\color{red} 9.1532\color{black} $\,\,$ \\
$\,\,$0.5484$\,\,$ & $\,\,$ 1 $\,\,$ & $\,\,$4.6001$\,\,$ & $\,\,$\color{red} 5.0198\color{black}   $\,\,$ \\
$\,\,$0.1192$\,\,$ & $\,\,$0.2174$\,\,$ & $\,\,$ 1 $\,\,$ & $\,\,$\color{red} 1.0912\color{black}  $\,\,$ \\
$\,\,$\color{red} 0.1093\color{black} $\,\,$ & $\,\,$\color{red} 0.1992\color{black} $\,\,$ & $\,\,$\color{red} 0.9164\color{black} $\,\,$ & $\,\,$ 1  $\,\,$ \\
\end{pmatrix},
\end{equation*}

\begin{equation*}
\mathbf{w}^{\prime} =
\begin{pmatrix}
0.562646\\
0.308563\\
0.067078\\
0.061713
\end{pmatrix} =
0.999757\cdot
\begin{pmatrix}
0.562783\\
0.308638\\
0.067094\\
\color{gr} 0.061728\color{black}
\end{pmatrix},
\end{equation*}
\begin{equation*}
\left[ \frac{{w}^{\prime}_i}{{w}^{\prime}_j} \right] =
\begin{pmatrix}
$\,\,$ 1 $\,\,$ & $\,\,$1.8234$\,\,$ & $\,\,$8.3879$\,\,$ & $\,\,$\color{gr} 9.1172\color{black} $\,\,$ \\
$\,\,$0.5484$\,\,$ & $\,\,$ 1 $\,\,$ & $\,\,$4.6001$\,\,$ & $\,\,$\color{gr} \color{blue} 5\color{black}   $\,\,$ \\
$\,\,$0.1192$\,\,$ & $\,\,$0.2174$\,\,$ & $\,\,$ 1 $\,\,$ & $\,\,$\color{gr} 1.0869\color{black}  $\,\,$ \\
$\,\,$\color{gr} 0.1097\color{black} $\,\,$ & $\,\,$\color{gr} \color{blue}  1/5\color{black} $\,\,$ & $\,\,$\color{gr} 0.9200\color{black} $\,\,$ & $\,\,$ 1  $\,\,$ \\
\end{pmatrix},
\end{equation*}
\end{example}
\newpage
\begin{example}
\begin{equation*}
\mathbf{A} =
\begin{pmatrix}
$\,\,$ 1 $\,\,$ & $\,\,$3$\,\,$ & $\,\,$6$\,\,$ & $\,\,$9 $\,\,$ \\
$\,\,$ 1/3$\,\,$ & $\,\,$ 1 $\,\,$ & $\,\,$8$\,\,$ & $\,\,$5 $\,\,$ \\
$\,\,$ 1/6$\,\,$ & $\,\,$ 1/8$\,\,$ & $\,\,$ 1 $\,\,$ & $\,\,$1 $\,\,$ \\
$\,\,$ 1/9$\,\,$ & $\,\,$ 1/5$\,\,$ & $\,\,$ 1 $\,\,$ & $\,\,$ 1  $\,\,$ \\
\end{pmatrix},
\qquad
\lambda_{\max} =
4.1655,
\qquad
CR = 0.0624
\end{equation*}

\begin{equation*}
\mathbf{w}^{cos} =
\begin{pmatrix}
0.557357\\
0.317110\\
0.065115\\
\color{red} 0.060419\color{black}
\end{pmatrix}\end{equation*}
\begin{equation*}
\left[ \frac{{w}^{cos}_i}{{w}^{cos}_j} \right] =
\begin{pmatrix}
$\,\,$ 1 $\,\,$ & $\,\,$1.7576$\,\,$ & $\,\,$8.5596$\,\,$ & $\,\,$\color{red} 9.2249\color{black} $\,\,$ \\
$\,\,$0.5690$\,\,$ & $\,\,$ 1 $\,\,$ & $\,\,$4.8700$\,\,$ & $\,\,$\color{red} 5.2485\color{black}   $\,\,$ \\
$\,\,$0.1168$\,\,$ & $\,\,$0.2053$\,\,$ & $\,\,$ 1 $\,\,$ & $\,\,$\color{red} 1.0777\color{black}  $\,\,$ \\
$\,\,$\color{red} 0.1084\color{black} $\,\,$ & $\,\,$\color{red} 0.1905\color{black} $\,\,$ & $\,\,$\color{red} 0.9279\color{black} $\,\,$ & $\,\,$ 1  $\,\,$ \\
\end{pmatrix},
\end{equation*}

\begin{equation*}
\mathbf{w}^{\prime} =
\begin{pmatrix}
0.556516\\
0.316632\\
0.065017\\
0.061835
\end{pmatrix} =
0.998493\cdot
\begin{pmatrix}
0.557357\\
0.317110\\
0.065115\\
\color{gr} 0.061929\color{black}
\end{pmatrix},
\end{equation*}
\begin{equation*}
\left[ \frac{{w}^{\prime}_i}{{w}^{\prime}_j} \right] =
\begin{pmatrix}
$\,\,$ 1 $\,\,$ & $\,\,$1.7576$\,\,$ & $\,\,$8.5596$\,\,$ & $\,\,$\color{gr} \color{blue} 9\color{black} $\,\,$ \\
$\,\,$0.5690$\,\,$ & $\,\,$ 1 $\,\,$ & $\,\,$4.8700$\,\,$ & $\,\,$\color{gr} 5.1206\color{black}   $\,\,$ \\
$\,\,$0.1168$\,\,$ & $\,\,$0.2053$\,\,$ & $\,\,$ 1 $\,\,$ & $\,\,$\color{gr} 1.0515\color{black}  $\,\,$ \\
$\,\,$\color{gr} \color{blue}  1/9\color{black} $\,\,$ & $\,\,$\color{gr} 0.1953\color{black} $\,\,$ & $\,\,$\color{gr} 0.9511\color{black} $\,\,$ & $\,\,$ 1  $\,\,$ \\
\end{pmatrix},
\end{equation*}
\end{example}
\newpage
\begin{example}
\begin{equation*}
\mathbf{A} =
\begin{pmatrix}
$\,\,$ 1 $\,\,$ & $\,\,$3$\,\,$ & $\,\,$6$\,\,$ & $\,\,$9 $\,\,$ \\
$\,\,$ 1/3$\,\,$ & $\,\,$ 1 $\,\,$ & $\,\,$9$\,\,$ & $\,\,$5 $\,\,$ \\
$\,\,$ 1/6$\,\,$ & $\,\,$ 1/9$\,\,$ & $\,\,$ 1 $\,\,$ & $\,\,$1 $\,\,$ \\
$\,\,$ 1/9$\,\,$ & $\,\,$ 1/5$\,\,$ & $\,\,$ 1 $\,\,$ & $\,\,$ 1  $\,\,$ \\
\end{pmatrix},
\qquad
\lambda_{\max} =
4.1966,
\qquad
CR = 0.0741
\end{equation*}

\begin{equation*}
\mathbf{w}^{cos} =
\begin{pmatrix}
0.552794\\
0.324238\\
0.063476\\
\color{red} 0.059492\color{black}
\end{pmatrix}\end{equation*}
\begin{equation*}
\left[ \frac{{w}^{cos}_i}{{w}^{cos}_j} \right] =
\begin{pmatrix}
$\,\,$ 1 $\,\,$ & $\,\,$1.7049$\,\,$ & $\,\,$8.7087$\,\,$ & $\,\,$\color{red} 9.2919\color{black} $\,\,$ \\
$\,\,$0.5865$\,\,$ & $\,\,$ 1 $\,\,$ & $\,\,$5.1080$\,\,$ & $\,\,$\color{red} 5.4501\color{black}   $\,\,$ \\
$\,\,$0.1148$\,\,$ & $\,\,$0.1958$\,\,$ & $\,\,$ 1 $\,\,$ & $\,\,$\color{red} 1.0670\color{black}  $\,\,$ \\
$\,\,$\color{red} 0.1076\color{black} $\,\,$ & $\,\,$\color{red} 0.1835\color{black} $\,\,$ & $\,\,$\color{red} 0.9372\color{black} $\,\,$ & $\,\,$ 1  $\,\,$ \\
\end{pmatrix},
\end{equation*}

\begin{equation*}
\mathbf{w}^{\prime} =
\begin{pmatrix}
0.551729\\
0.323613\\
0.063354\\
0.061303
\end{pmatrix} =
0.998074\cdot
\begin{pmatrix}
0.552794\\
0.324238\\
0.063476\\
\color{gr} 0.061422\color{black}
\end{pmatrix},
\end{equation*}
\begin{equation*}
\left[ \frac{{w}^{\prime}_i}{{w}^{\prime}_j} \right] =
\begin{pmatrix}
$\,\,$ 1 $\,\,$ & $\,\,$1.7049$\,\,$ & $\,\,$8.7087$\,\,$ & $\,\,$\color{gr} \color{blue} 9\color{black} $\,\,$ \\
$\,\,$0.5865$\,\,$ & $\,\,$ 1 $\,\,$ & $\,\,$5.1080$\,\,$ & $\,\,$\color{gr} 5.2789\color{black}   $\,\,$ \\
$\,\,$0.1148$\,\,$ & $\,\,$0.1958$\,\,$ & $\,\,$ 1 $\,\,$ & $\,\,$\color{gr} 1.0335\color{black}  $\,\,$ \\
$\,\,$\color{gr} \color{blue}  1/9\color{black} $\,\,$ & $\,\,$\color{gr} 0.1894\color{black} $\,\,$ & $\,\,$\color{gr} 0.9676\color{black} $\,\,$ & $\,\,$ 1  $\,\,$ \\
\end{pmatrix},
\end{equation*}
\end{example}
\newpage
\begin{example}
\begin{equation*}
\mathbf{A} =
\begin{pmatrix}
$\,\,$ 1 $\,\,$ & $\,\,$3$\,\,$ & $\,\,$7$\,\,$ & $\,\,$4 $\,\,$ \\
$\,\,$ 1/3$\,\,$ & $\,\,$ 1 $\,\,$ & $\,\,$4$\,\,$ & $\,\,$7 $\,\,$ \\
$\,\,$ 1/7$\,\,$ & $\,\,$ 1/4$\,\,$ & $\,\,$ 1 $\,\,$ & $\,\,$1 $\,\,$ \\
$\,\,$ 1/4$\,\,$ & $\,\,$ 1/7$\,\,$ & $\,\,$ 1 $\,\,$ & $\,\,$ 1  $\,\,$ \\
\end{pmatrix},
\qquad
\lambda_{\max} =
4.2395,
\qquad
CR = 0.0903
\end{equation*}

\begin{equation*}
\mathbf{w}^{cos} =
\begin{pmatrix}
0.521828\\
0.319867\\
\color{red} 0.073915\color{black} \\
0.084390
\end{pmatrix}\end{equation*}
\begin{equation*}
\left[ \frac{{w}^{cos}_i}{{w}^{cos}_j} \right] =
\begin{pmatrix}
$\,\,$ 1 $\,\,$ & $\,\,$1.6314$\,\,$ & $\,\,$\color{red} 7.0598\color{black} $\,\,$ & $\,\,$6.1835$\,\,$ \\
$\,\,$0.6130$\,\,$ & $\,\,$ 1 $\,\,$ & $\,\,$\color{red} 4.3275\color{black} $\,\,$ & $\,\,$3.7903  $\,\,$ \\
$\,\,$\color{red} 0.1416\color{black} $\,\,$ & $\,\,$\color{red} 0.2311\color{black} $\,\,$ & $\,\,$ 1 $\,\,$ & $\,\,$\color{red} 0.8759\color{black}  $\,\,$ \\
$\,\,$0.1617$\,\,$ & $\,\,$0.2638$\,\,$ & $\,\,$\color{red} 1.1417\color{black} $\,\,$ & $\,\,$ 1  $\,\,$ \\
\end{pmatrix},
\end{equation*}

\begin{equation*}
\mathbf{w}^{\prime} =
\begin{pmatrix}
0.521499\\
0.319665\\
0.074500\\
0.084337
\end{pmatrix} =
0.999369\cdot
\begin{pmatrix}
0.521828\\
0.319867\\
\color{gr} 0.074547\color{black} \\
0.084390
\end{pmatrix},
\end{equation*}
\begin{equation*}
\left[ \frac{{w}^{\prime}_i}{{w}^{\prime}_j} \right] =
\begin{pmatrix}
$\,\,$ 1 $\,\,$ & $\,\,$1.6314$\,\,$ & $\,\,$\color{gr} \color{blue} 7\color{black} $\,\,$ & $\,\,$6.1835$\,\,$ \\
$\,\,$0.6130$\,\,$ & $\,\,$ 1 $\,\,$ & $\,\,$\color{gr} 4.2908\color{black} $\,\,$ & $\,\,$3.7903  $\,\,$ \\
$\,\,$\color{gr} \color{blue}  1/7\color{black} $\,\,$ & $\,\,$\color{gr} 0.2331\color{black} $\,\,$ & $\,\,$ 1 $\,\,$ & $\,\,$\color{gr} 0.8834\color{black}  $\,\,$ \\
$\,\,$0.1617$\,\,$ & $\,\,$0.2638$\,\,$ & $\,\,$\color{gr} 1.1320\color{black} $\,\,$ & $\,\,$ 1  $\,\,$ \\
\end{pmatrix},
\end{equation*}
\end{example}
\newpage
\begin{example}
\begin{equation*}
\mathbf{A} =
\begin{pmatrix}
$\,\,$ 1 $\,\,$ & $\,\,$3$\,\,$ & $\,\,$7$\,\,$ & $\,\,$9 $\,\,$ \\
$\,\,$ 1/3$\,\,$ & $\,\,$ 1 $\,\,$ & $\,\,$4$\,\,$ & $\,\,$2 $\,\,$ \\
$\,\,$ 1/7$\,\,$ & $\,\,$ 1/4$\,\,$ & $\,\,$ 1 $\,\,$ & $\,\,$2 $\,\,$ \\
$\,\,$ 1/9$\,\,$ & $\,\,$ 1/2$\,\,$ & $\,\,$ 1/2$\,\,$ & $\,\,$ 1  $\,\,$ \\
\end{pmatrix},
\qquad
\lambda_{\max} =
4.1658,
\qquad
CR = 0.0625
\end{equation*}

\begin{equation*}
\mathbf{w}^{cos} =
\begin{pmatrix}
\color{red} 0.615574\color{black} \\
0.221821\\
0.091234\\
0.071372
\end{pmatrix}\end{equation*}
\begin{equation*}
\left[ \frac{{w}^{cos}_i}{{w}^{cos}_j} \right] =
\begin{pmatrix}
$\,\,$ 1 $\,\,$ & $\,\,$\color{red} 2.7751\color{black} $\,\,$ & $\,\,$\color{red} 6.7472\color{black} $\,\,$ & $\,\,$\color{red} 8.6249\color{black} $\,\,$ \\
$\,\,$\color{red} 0.3603\color{black} $\,\,$ & $\,\,$ 1 $\,\,$ & $\,\,$2.4313$\,\,$ & $\,\,$3.1080  $\,\,$ \\
$\,\,$\color{red} 0.1482\color{black} $\,\,$ & $\,\,$0.4113$\,\,$ & $\,\,$ 1 $\,\,$ & $\,\,$1.2783 $\,\,$ \\
$\,\,$\color{red} 0.1159\color{black} $\,\,$ & $\,\,$0.3218$\,\,$ & $\,\,$0.7823$\,\,$ & $\,\,$ 1  $\,\,$ \\
\end{pmatrix},
\end{equation*}

\begin{equation*}
\mathbf{w}^{\prime} =
\begin{pmatrix}
0.624240\\
0.216820\\
0.089177\\
0.069763
\end{pmatrix} =
0.977457\cdot
\begin{pmatrix}
\color{gr} 0.638637\color{black} \\
0.221821\\
0.091234\\
0.071372
\end{pmatrix},
\end{equation*}
\begin{equation*}
\left[ \frac{{w}^{\prime}_i}{{w}^{\prime}_j} \right] =
\begin{pmatrix}
$\,\,$ 1 $\,\,$ & $\,\,$\color{gr} 2.8791\color{black} $\,\,$ & $\,\,$\color{gr} \color{blue} 7\color{black} $\,\,$ & $\,\,$\color{gr} 8.9480\color{black} $\,\,$ \\
$\,\,$\color{gr} 0.3473\color{black} $\,\,$ & $\,\,$ 1 $\,\,$ & $\,\,$2.4313$\,\,$ & $\,\,$3.1080  $\,\,$ \\
$\,\,$\color{gr} \color{blue}  1/7\color{black} $\,\,$ & $\,\,$0.4113$\,\,$ & $\,\,$ 1 $\,\,$ & $\,\,$1.2783 $\,\,$ \\
$\,\,$\color{gr} 0.1118\color{black} $\,\,$ & $\,\,$0.3218$\,\,$ & $\,\,$0.7823$\,\,$ & $\,\,$ 1  $\,\,$ \\
\end{pmatrix},
\end{equation*}
\end{example}
\newpage
\begin{example}
\begin{equation*}
\mathbf{A} =
\begin{pmatrix}
$\,\,$ 1 $\,\,$ & $\,\,$3$\,\,$ & $\,\,$7$\,\,$ & $\,\,$9 $\,\,$ \\
$\,\,$ 1/3$\,\,$ & $\,\,$ 1 $\,\,$ & $\,\,$5$\,\,$ & $\,\,$2 $\,\,$ \\
$\,\,$ 1/7$\,\,$ & $\,\,$ 1/5$\,\,$ & $\,\,$ 1 $\,\,$ & $\,\,$2 $\,\,$ \\
$\,\,$ 1/9$\,\,$ & $\,\,$ 1/2$\,\,$ & $\,\,$ 1/2$\,\,$ & $\,\,$ 1  $\,\,$ \\
\end{pmatrix},
\qquad
\lambda_{\max} =
4.2300,
\qquad
CR = 0.0867
\end{equation*}

\begin{equation*}
\mathbf{w}^{cos} =
\begin{pmatrix}
\color{red} 0.606199\color{black} \\
0.235838\\
0.087275\\
0.070688
\end{pmatrix}\end{equation*}
\begin{equation*}
\left[ \frac{{w}^{cos}_i}{{w}^{cos}_j} \right] =
\begin{pmatrix}
$\,\,$ 1 $\,\,$ & $\,\,$\color{red} 2.5704\color{black} $\,\,$ & $\,\,$\color{red} 6.9459\color{black} $\,\,$ & $\,\,$\color{red} 8.5757\color{black} $\,\,$ \\
$\,\,$\color{red} 0.3890\color{black} $\,\,$ & $\,\,$ 1 $\,\,$ & $\,\,$2.7022$\,\,$ & $\,\,$3.3363  $\,\,$ \\
$\,\,$\color{red} 0.1440\color{black} $\,\,$ & $\,\,$0.3701$\,\,$ & $\,\,$ 1 $\,\,$ & $\,\,$1.2346 $\,\,$ \\
$\,\,$\color{red} 0.1166\color{black} $\,\,$ & $\,\,$0.2997$\,\,$ & $\,\,$0.8100$\,\,$ & $\,\,$ 1  $\,\,$ \\
\end{pmatrix},
\end{equation*}

\begin{equation*}
\mathbf{w}^{\prime} =
\begin{pmatrix}
0.608051\\
0.234729\\
0.086864\\
0.070356
\end{pmatrix} =
0.995297\cdot
\begin{pmatrix}
\color{gr} 0.610924\color{black} \\
0.235838\\
0.087275\\
0.070688
\end{pmatrix},
\end{equation*}
\begin{equation*}
\left[ \frac{{w}^{\prime}_i}{{w}^{\prime}_j} \right] =
\begin{pmatrix}
$\,\,$ 1 $\,\,$ & $\,\,$\color{gr} 2.5904\color{black} $\,\,$ & $\,\,$\color{gr} \color{blue} 7\color{black} $\,\,$ & $\,\,$\color{gr} 8.6425\color{black} $\,\,$ \\
$\,\,$\color{gr} 0.3860\color{black} $\,\,$ & $\,\,$ 1 $\,\,$ & $\,\,$2.7022$\,\,$ & $\,\,$3.3363  $\,\,$ \\
$\,\,$\color{gr} \color{blue}  1/7\color{black} $\,\,$ & $\,\,$0.3701$\,\,$ & $\,\,$ 1 $\,\,$ & $\,\,$1.2346 $\,\,$ \\
$\,\,$\color{gr} 0.1157\color{black} $\,\,$ & $\,\,$0.2997$\,\,$ & $\,\,$0.8100$\,\,$ & $\,\,$ 1  $\,\,$ \\
\end{pmatrix},
\end{equation*}
\end{example}
\newpage
\begin{example}
\begin{equation*}
\mathbf{A} =
\begin{pmatrix}
$\,\,$ 1 $\,\,$ & $\,\,$3$\,\,$ & $\,\,$7$\,\,$ & $\,\,$9 $\,\,$ \\
$\,\,$ 1/3$\,\,$ & $\,\,$ 1 $\,\,$ & $\,\,$5$\,\,$ & $\,\,$4 $\,\,$ \\
$\,\,$ 1/7$\,\,$ & $\,\,$ 1/5$\,\,$ & $\,\,$ 1 $\,\,$ & $\,\,$1 $\,\,$ \\
$\,\,$ 1/9$\,\,$ & $\,\,$ 1/4$\,\,$ & $\,\,$ 1 $\,\,$ & $\,\,$ 1  $\,\,$ \\
\end{pmatrix},
\qquad
\lambda_{\max} =
4.0490,
\qquad
CR = 0.0185
\end{equation*}

\begin{equation*}
\mathbf{w}^{cos} =
\begin{pmatrix}
0.597959\\
0.267078\\
0.068645\\
\color{red} 0.066319\color{black}
\end{pmatrix}\end{equation*}
\begin{equation*}
\left[ \frac{{w}^{cos}_i}{{w}^{cos}_j} \right] =
\begin{pmatrix}
$\,\,$ 1 $\,\,$ & $\,\,$2.2389$\,\,$ & $\,\,$8.7109$\,\,$ & $\,\,$\color{red} 9.0165\color{black} $\,\,$ \\
$\,\,$0.4466$\,\,$ & $\,\,$ 1 $\,\,$ & $\,\,$3.8907$\,\,$ & $\,\,$\color{red} 4.0272\color{black}   $\,\,$ \\
$\,\,$0.1148$\,\,$ & $\,\,$0.2570$\,\,$ & $\,\,$ 1 $\,\,$ & $\,\,$\color{red} 1.0351\color{black}  $\,\,$ \\
$\,\,$\color{red} 0.1109\color{black} $\,\,$ & $\,\,$\color{red} 0.2483\color{black} $\,\,$ & $\,\,$\color{red} 0.9661\color{black} $\,\,$ & $\,\,$ 1  $\,\,$ \\
\end{pmatrix},
\end{equation*}

\begin{equation*}
\mathbf{w}^{\prime} =
\begin{pmatrix}
0.597886\\
0.267046\\
0.068636\\
0.066432
\end{pmatrix} =
0.999879\cdot
\begin{pmatrix}
0.597959\\
0.267078\\
0.068645\\
\color{gr} 0.066440\color{black}
\end{pmatrix},
\end{equation*}
\begin{equation*}
\left[ \frac{{w}^{\prime}_i}{{w}^{\prime}_j} \right] =
\begin{pmatrix}
$\,\,$ 1 $\,\,$ & $\,\,$2.2389$\,\,$ & $\,\,$8.7109$\,\,$ & $\,\,$\color{gr} \color{blue} 9\color{black} $\,\,$ \\
$\,\,$0.4466$\,\,$ & $\,\,$ 1 $\,\,$ & $\,\,$3.8907$\,\,$ & $\,\,$\color{gr} 4.0198\color{black}   $\,\,$ \\
$\,\,$0.1148$\,\,$ & $\,\,$0.2570$\,\,$ & $\,\,$ 1 $\,\,$ & $\,\,$\color{gr} 1.0332\color{black}  $\,\,$ \\
$\,\,$\color{gr} \color{blue}  1/9\color{black} $\,\,$ & $\,\,$\color{gr} 0.2488\color{black} $\,\,$ & $\,\,$\color{gr} 0.9679\color{black} $\,\,$ & $\,\,$ 1  $\,\,$ \\
\end{pmatrix},
\end{equation*}
\end{example}
\newpage
\begin{example}
\begin{equation*}
\mathbf{A} =
\begin{pmatrix}
$\,\,$ 1 $\,\,$ & $\,\,$3$\,\,$ & $\,\,$7$\,\,$ & $\,\,$9 $\,\,$ \\
$\,\,$ 1/3$\,\,$ & $\,\,$ 1 $\,\,$ & $\,\,$6$\,\,$ & $\,\,$4 $\,\,$ \\
$\,\,$ 1/7$\,\,$ & $\,\,$ 1/6$\,\,$ & $\,\,$ 1 $\,\,$ & $\,\,$1 $\,\,$ \\
$\,\,$ 1/9$\,\,$ & $\,\,$ 1/4$\,\,$ & $\,\,$ 1 $\,\,$ & $\,\,$ 1  $\,\,$ \\
\end{pmatrix},
\qquad
\lambda_{\max} =
4.0762,
\qquad
CR = 0.0287
\end{equation*}

\begin{equation*}
\mathbf{w}^{cos} =
\begin{pmatrix}
0.590031\\
0.279079\\
0.065732\\
\color{red} 0.065157\color{black}
\end{pmatrix}\end{equation*}
\begin{equation*}
\left[ \frac{{w}^{cos}_i}{{w}^{cos}_j} \right] =
\begin{pmatrix}
$\,\,$ 1 $\,\,$ & $\,\,$2.1142$\,\,$ & $\,\,$8.9763$\,\,$ & $\,\,$\color{red} 9.0556\color{black} $\,\,$ \\
$\,\,$0.4730$\,\,$ & $\,\,$ 1 $\,\,$ & $\,\,$4.2457$\,\,$ & $\,\,$\color{red} 4.2832\color{black}   $\,\,$ \\
$\,\,$0.1114$\,\,$ & $\,\,$0.2355$\,\,$ & $\,\,$ 1 $\,\,$ & $\,\,$\color{red} 1.0088\color{black}  $\,\,$ \\
$\,\,$\color{red} 0.1104\color{black} $\,\,$ & $\,\,$\color{red} 0.2335\color{black} $\,\,$ & $\,\,$\color{red} 0.9912\color{black} $\,\,$ & $\,\,$ 1  $\,\,$ \\
\end{pmatrix},
\end{equation*}

\begin{equation*}
\mathbf{w}^{\prime} =
\begin{pmatrix}
0.589794\\
0.278967\\
0.065706\\
0.065533
\end{pmatrix} =
0.999598\cdot
\begin{pmatrix}
0.590031\\
0.279079\\
0.065732\\
\color{gr} 0.065559\color{black}
\end{pmatrix},
\end{equation*}
\begin{equation*}
\left[ \frac{{w}^{\prime}_i}{{w}^{\prime}_j} \right] =
\begin{pmatrix}
$\,\,$ 1 $\,\,$ & $\,\,$2.1142$\,\,$ & $\,\,$8.9763$\,\,$ & $\,\,$\color{gr} \color{blue} 9\color{black} $\,\,$ \\
$\,\,$0.4730$\,\,$ & $\,\,$ 1 $\,\,$ & $\,\,$4.2457$\,\,$ & $\,\,$\color{gr} 4.2569\color{black}   $\,\,$ \\
$\,\,$0.1114$\,\,$ & $\,\,$0.2355$\,\,$ & $\,\,$ 1 $\,\,$ & $\,\,$\color{gr} 1.0026\color{black}  $\,\,$ \\
$\,\,$\color{gr} \color{blue}  1/9\color{black} $\,\,$ & $\,\,$\color{gr} 0.2349\color{black} $\,\,$ & $\,\,$\color{gr} 0.9974\color{black} $\,\,$ & $\,\,$ 1  $\,\,$ \\
\end{pmatrix},
\end{equation*}
\end{example}
\newpage
\begin{example}
\begin{equation*}
\mathbf{A} =
\begin{pmatrix}
$\,\,$ 1 $\,\,$ & $\,\,$3$\,\,$ & $\,\,$7$\,\,$ & $\,\,$9 $\,\,$ \\
$\,\,$ 1/3$\,\,$ & $\,\,$ 1 $\,\,$ & $\,\,$8$\,\,$ & $\,\,$5 $\,\,$ \\
$\,\,$ 1/7$\,\,$ & $\,\,$ 1/8$\,\,$ & $\,\,$ 1 $\,\,$ & $\,\,$1 $\,\,$ \\
$\,\,$ 1/9$\,\,$ & $\,\,$ 1/5$\,\,$ & $\,\,$ 1 $\,\,$ & $\,\,$ 1  $\,\,$ \\
\end{pmatrix},
\qquad
\lambda_{\max} =
4.1321,
\qquad
CR = 0.0498
\end{equation*}

\begin{equation*}
\mathbf{w}^{cos} =
\begin{pmatrix}
0.569212\\
0.310428\\
0.060685\\
\color{red} 0.059676\color{black}
\end{pmatrix}\end{equation*}
\begin{equation*}
\left[ \frac{{w}^{cos}_i}{{w}^{cos}_j} \right] =
\begin{pmatrix}
$\,\,$ 1 $\,\,$ & $\,\,$1.8336$\,\,$ & $\,\,$9.3797$\,\,$ & $\,\,$\color{red} 9.5384\color{black} $\,\,$ \\
$\,\,$0.5454$\,\,$ & $\,\,$ 1 $\,\,$ & $\,\,$5.1154$\,\,$ & $\,\,$\color{red} 5.2019\color{black}   $\,\,$ \\
$\,\,$0.1066$\,\,$ & $\,\,$0.1955$\,\,$ & $\,\,$ 1 $\,\,$ & $\,\,$\color{red} 1.0169\color{black}  $\,\,$ \\
$\,\,$\color{red} 0.1048\color{black} $\,\,$ & $\,\,$\color{red} 0.1922\color{black} $\,\,$ & $\,\,$\color{red} 0.9834\color{black} $\,\,$ & $\,\,$ 1  $\,\,$ \\
\end{pmatrix},
\end{equation*}

\begin{equation*}
\mathbf{w}^{\prime} =
\begin{pmatrix}
0.568637\\
0.310114\\
0.060624\\
0.060624
\end{pmatrix} =
0.998991\cdot
\begin{pmatrix}
0.569212\\
0.310428\\
0.060685\\
\color{gr} 0.060685\color{black}
\end{pmatrix},
\end{equation*}
\begin{equation*}
\left[ \frac{{w}^{\prime}_i}{{w}^{\prime}_j} \right] =
\begin{pmatrix}
$\,\,$ 1 $\,\,$ & $\,\,$1.8336$\,\,$ & $\,\,$9.3797$\,\,$ & $\,\,$\color{gr} 9.3797\color{black} $\,\,$ \\
$\,\,$0.5454$\,\,$ & $\,\,$ 1 $\,\,$ & $\,\,$5.1154$\,\,$ & $\,\,$\color{gr} 5.1154\color{black}   $\,\,$ \\
$\,\,$0.1066$\,\,$ & $\,\,$0.1955$\,\,$ & $\,\,$ 1 $\,\,$ & $\,\,$\color{gr} \color{blue} 1\color{black}  $\,\,$ \\
$\,\,$\color{gr} 0.1066\color{black} $\,\,$ & $\,\,$\color{gr} 0.1955\color{black} $\,\,$ & $\,\,$\color{gr} \color{blue} 1\color{black} $\,\,$ & $\,\,$ 1  $\,\,$ \\
\end{pmatrix},
\end{equation*}
\end{example}
\newpage
\begin{example}
\begin{equation*}
\mathbf{A} =
\begin{pmatrix}
$\,\,$ 1 $\,\,$ & $\,\,$3$\,\,$ & $\,\,$7$\,\,$ & $\,\,$9 $\,\,$ \\
$\,\,$ 1/3$\,\,$ & $\,\,$ 1 $\,\,$ & $\,\,$9$\,\,$ & $\,\,$5 $\,\,$ \\
$\,\,$ 1/7$\,\,$ & $\,\,$ 1/9$\,\,$ & $\,\,$ 1 $\,\,$ & $\,\,$1 $\,\,$ \\
$\,\,$ 1/9$\,\,$ & $\,\,$ 1/5$\,\,$ & $\,\,$ 1 $\,\,$ & $\,\,$ 1  $\,\,$ \\
\end{pmatrix},
\qquad
\lambda_{\max} =
4.1603,
\qquad
CR = 0.0605
\end{equation*}

\begin{equation*}
\mathbf{w}^{cos} =
\begin{pmatrix}
0.564103\\
0.317898\\
0.059138\\
\color{red} 0.058861\color{black}
\end{pmatrix}\end{equation*}
\begin{equation*}
\left[ \frac{{w}^{cos}_i}{{w}^{cos}_j} \right] =
\begin{pmatrix}
$\,\,$ 1 $\,\,$ & $\,\,$1.7745$\,\,$ & $\,\,$9.5388$\,\,$ & $\,\,$\color{red} 9.5836\color{black} $\,\,$ \\
$\,\,$0.5635$\,\,$ & $\,\,$ 1 $\,\,$ & $\,\,$5.3755$\,\,$ & $\,\,$\color{red} 5.4008\color{black}   $\,\,$ \\
$\,\,$0.1048$\,\,$ & $\,\,$0.1860$\,\,$ & $\,\,$ 1 $\,\,$ & $\,\,$\color{red} 1.0047\color{black}  $\,\,$ \\
$\,\,$\color{red} 0.1043\color{black} $\,\,$ & $\,\,$\color{red} 0.1852\color{black} $\,\,$ & $\,\,$\color{red} 0.9953\color{black} $\,\,$ & $\,\,$ 1  $\,\,$ \\
\end{pmatrix},
\end{equation*}

\begin{equation*}
\mathbf{w}^{\prime} =
\begin{pmatrix}
0.563947\\
0.317810\\
0.059122\\
0.059122
\end{pmatrix} =
0.999723\cdot
\begin{pmatrix}
0.564103\\
0.317898\\
0.059138\\
\color{gr} 0.059138\color{black}
\end{pmatrix},
\end{equation*}
\begin{equation*}
\left[ \frac{{w}^{\prime}_i}{{w}^{\prime}_j} \right] =
\begin{pmatrix}
$\,\,$ 1 $\,\,$ & $\,\,$1.7745$\,\,$ & $\,\,$9.5388$\,\,$ & $\,\,$\color{gr} 9.5388\color{black} $\,\,$ \\
$\,\,$0.5635$\,\,$ & $\,\,$ 1 $\,\,$ & $\,\,$5.3755$\,\,$ & $\,\,$\color{gr} 5.3755\color{black}   $\,\,$ \\
$\,\,$0.1048$\,\,$ & $\,\,$0.1860$\,\,$ & $\,\,$ 1 $\,\,$ & $\,\,$\color{gr} \color{blue} 1\color{black}  $\,\,$ \\
$\,\,$\color{gr} 0.1048\color{black} $\,\,$ & $\,\,$\color{gr} 0.1860\color{black} $\,\,$ & $\,\,$\color{gr} \color{blue} 1\color{black} $\,\,$ & $\,\,$ 1  $\,\,$ \\
\end{pmatrix},
\end{equation*}
\end{example}
\newpage
\begin{example}
\begin{equation*}
\mathbf{A} =
\begin{pmatrix}
$\,\,$ 1 $\,\,$ & $\,\,$3$\,\,$ & $\,\,$8$\,\,$ & $\,\,$5 $\,\,$ \\
$\,\,$ 1/3$\,\,$ & $\,\,$ 1 $\,\,$ & $\,\,$5$\,\,$ & $\,\,$8 $\,\,$ \\
$\,\,$ 1/8$\,\,$ & $\,\,$ 1/5$\,\,$ & $\,\,$ 1 $\,\,$ & $\,\,$1 $\,\,$ \\
$\,\,$ 1/5$\,\,$ & $\,\,$ 1/8$\,\,$ & $\,\,$ 1 $\,\,$ & $\,\,$ 1  $\,\,$ \\
\end{pmatrix},
\qquad
\lambda_{\max} =
4.2144,
\qquad
CR = 0.0808
\end{equation*}

\begin{equation*}
\mathbf{w}^{cos} =
\begin{pmatrix}
0.535176\\
0.328708\\
\color{red} 0.064270\color{black} \\
0.071845
\end{pmatrix}\end{equation*}
\begin{equation*}
\left[ \frac{{w}^{cos}_i}{{w}^{cos}_j} \right] =
\begin{pmatrix}
$\,\,$ 1 $\,\,$ & $\,\,$1.6281$\,\,$ & $\,\,$\color{red} 8.3270\color{black} $\,\,$ & $\,\,$7.4490$\,\,$ \\
$\,\,$0.6142$\,\,$ & $\,\,$ 1 $\,\,$ & $\,\,$\color{red} 5.1145\color{black} $\,\,$ & $\,\,$4.5752  $\,\,$ \\
$\,\,$\color{red} 0.1201\color{black} $\,\,$ & $\,\,$\color{red} 0.1955\color{black} $\,\,$ & $\,\,$ 1 $\,\,$ & $\,\,$\color{red} 0.8946\color{black}  $\,\,$ \\
$\,\,$0.1342$\,\,$ & $\,\,$0.2186$\,\,$ & $\,\,$\color{red} 1.1179\color{black} $\,\,$ & $\,\,$ 1  $\,\,$ \\
\end{pmatrix},
\end{equation*}

\begin{equation*}
\mathbf{w}^{\prime} =
\begin{pmatrix}
0.534390\\
0.328225\\
0.065645\\
0.071740
\end{pmatrix} =
0.998531\cdot
\begin{pmatrix}
0.535176\\
0.328708\\
\color{gr} 0.065742\color{black} \\
0.071845
\end{pmatrix},
\end{equation*}
\begin{equation*}
\left[ \frac{{w}^{\prime}_i}{{w}^{\prime}_j} \right] =
\begin{pmatrix}
$\,\,$ 1 $\,\,$ & $\,\,$1.6281$\,\,$ & $\,\,$\color{gr} 8.1406\color{black} $\,\,$ & $\,\,$7.4490$\,\,$ \\
$\,\,$0.6142$\,\,$ & $\,\,$ 1 $\,\,$ & $\,\,$\color{gr} \color{blue} 5\color{black} $\,\,$ & $\,\,$4.5752  $\,\,$ \\
$\,\,$\color{gr} 0.1228\color{black} $\,\,$ & $\,\,$\color{gr} \color{blue}  1/5\color{black} $\,\,$ & $\,\,$ 1 $\,\,$ & $\,\,$\color{gr} 0.9150\color{black}  $\,\,$ \\
$\,\,$0.1342$\,\,$ & $\,\,$0.2186$\,\,$ & $\,\,$\color{gr} 1.0928\color{black} $\,\,$ & $\,\,$ 1  $\,\,$ \\
\end{pmatrix},
\end{equation*}
\end{example}
\newpage
\begin{example}
\begin{equation*}
\mathbf{A} =
\begin{pmatrix}
$\,\,$ 1 $\,\,$ & $\,\,$3$\,\,$ & $\,\,$8$\,\,$ & $\,\,$9 $\,\,$ \\
$\,\,$ 1/3$\,\,$ & $\,\,$ 1 $\,\,$ & $\,\,$4$\,\,$ & $\,\,$2 $\,\,$ \\
$\,\,$ 1/8$\,\,$ & $\,\,$ 1/4$\,\,$ & $\,\,$ 1 $\,\,$ & $\,\,$2 $\,\,$ \\
$\,\,$ 1/9$\,\,$ & $\,\,$ 1/2$\,\,$ & $\,\,$ 1/2$\,\,$ & $\,\,$ 1  $\,\,$ \\
\end{pmatrix},
\qquad
\lambda_{\max} =
4.1664,
\qquad
CR = 0.0627
\end{equation*}

\begin{equation*}
\mathbf{w}^{cos} =
\begin{pmatrix}
\color{red} 0.625919\color{black} \\
0.215817\\
0.087240\\
0.071025
\end{pmatrix}\end{equation*}
\begin{equation*}
\left[ \frac{{w}^{cos}_i}{{w}^{cos}_j} \right] =
\begin{pmatrix}
$\,\,$ 1 $\,\,$ & $\,\,$\color{red} 2.9002\color{black} $\,\,$ & $\,\,$\color{red} 7.1747\color{black} $\,\,$ & $\,\,$\color{red} 8.8127\color{black} $\,\,$ \\
$\,\,$\color{red} 0.3448\color{black} $\,\,$ & $\,\,$ 1 $\,\,$ & $\,\,$2.4738$\,\,$ & $\,\,$3.0386  $\,\,$ \\
$\,\,$\color{red} 0.1394\color{black} $\,\,$ & $\,\,$0.4042$\,\,$ & $\,\,$ 1 $\,\,$ & $\,\,$1.2283 $\,\,$ \\
$\,\,$\color{red} 0.1135\color{black} $\,\,$ & $\,\,$0.3291$\,\,$ & $\,\,$0.8141$\,\,$ & $\,\,$ 1  $\,\,$ \\
\end{pmatrix},
\end{equation*}

\begin{equation*}
\mathbf{w}^{\prime} =
\begin{pmatrix}
0.630831\\
0.212983\\
0.086094\\
0.070092
\end{pmatrix} =
0.986869\cdot
\begin{pmatrix}
\color{gr} 0.639224\color{black} \\
0.215817\\
0.087240\\
0.071025
\end{pmatrix},
\end{equation*}
\begin{equation*}
\left[ \frac{{w}^{\prime}_i}{{w}^{\prime}_j} \right] =
\begin{pmatrix}
$\,\,$ 1 $\,\,$ & $\,\,$\color{gr} 2.9619\color{black} $\,\,$ & $\,\,$\color{gr} 7.3272\color{black} $\,\,$ & $\,\,$\color{gr} \color{blue} 9\color{black} $\,\,$ \\
$\,\,$\color{gr} 0.3376\color{black} $\,\,$ & $\,\,$ 1 $\,\,$ & $\,\,$2.4738$\,\,$ & $\,\,$3.0386  $\,\,$ \\
$\,\,$\color{gr} 0.1365\color{black} $\,\,$ & $\,\,$0.4042$\,\,$ & $\,\,$ 1 $\,\,$ & $\,\,$1.2283 $\,\,$ \\
$\,\,$\color{gr} \color{blue}  1/9\color{black} $\,\,$ & $\,\,$0.3291$\,\,$ & $\,\,$0.8141$\,\,$ & $\,\,$ 1  $\,\,$ \\
\end{pmatrix},
\end{equation*}
\end{example}
\newpage
\begin{example}
\begin{equation*}
\mathbf{A} =
\begin{pmatrix}
$\,\,$ 1 $\,\,$ & $\,\,$3$\,\,$ & $\,\,$8$\,\,$ & $\,\,$9 $\,\,$ \\
$\,\,$ 1/3$\,\,$ & $\,\,$ 1 $\,\,$ & $\,\,$5$\,\,$ & $\,\,$2 $\,\,$ \\
$\,\,$ 1/8$\,\,$ & $\,\,$ 1/5$\,\,$ & $\,\,$ 1 $\,\,$ & $\,\,$2 $\,\,$ \\
$\,\,$ 1/9$\,\,$ & $\,\,$ 1/2$\,\,$ & $\,\,$ 1/2$\,\,$ & $\,\,$ 1  $\,\,$ \\
\end{pmatrix},
\qquad
\lambda_{\max} =
4.2267,
\qquad
CR = 0.0855
\end{equation*}

\begin{equation*}
\mathbf{w}^{cos} =
\begin{pmatrix}
\color{red} 0.616897\color{black} \\
0.229245\\
0.083458\\
0.070400
\end{pmatrix}\end{equation*}
\begin{equation*}
\left[ \frac{{w}^{cos}_i}{{w}^{cos}_j} \right] =
\begin{pmatrix}
$\,\,$ 1 $\,\,$ & $\,\,$\color{red} 2.6910\color{black} $\,\,$ & $\,\,$\color{red} 7.3917\color{black} $\,\,$ & $\,\,$\color{red} 8.7627\color{black} $\,\,$ \\
$\,\,$\color{red} 0.3716\color{black} $\,\,$ & $\,\,$ 1 $\,\,$ & $\,\,$2.7468$\,\,$ & $\,\,$3.2563  $\,\,$ \\
$\,\,$\color{red} 0.1353\color{black} $\,\,$ & $\,\,$0.3641$\,\,$ & $\,\,$ 1 $\,\,$ & $\,\,$1.1855 $\,\,$ \\
$\,\,$\color{red} 0.1141\color{black} $\,\,$ & $\,\,$0.3071$\,\,$ & $\,\,$0.8435$\,\,$ & $\,\,$ 1  $\,\,$ \\
\end{pmatrix},
\end{equation*}

\begin{equation*}
\mathbf{w}^{\prime} =
\begin{pmatrix}
0.623192\\
0.225478\\
0.082087\\
0.069244
\end{pmatrix} =
0.983570\cdot
\begin{pmatrix}
\color{gr} 0.633602\color{black} \\
0.229245\\
0.083458\\
0.070400
\end{pmatrix},
\end{equation*}
\begin{equation*}
\left[ \frac{{w}^{\prime}_i}{{w}^{\prime}_j} \right] =
\begin{pmatrix}
$\,\,$ 1 $\,\,$ & $\,\,$\color{gr} 2.7639\color{black} $\,\,$ & $\,\,$\color{gr} 7.5919\color{black} $\,\,$ & $\,\,$\color{gr} \color{blue} 9\color{black} $\,\,$ \\
$\,\,$\color{gr} 0.3618\color{black} $\,\,$ & $\,\,$ 1 $\,\,$ & $\,\,$2.7468$\,\,$ & $\,\,$3.2563  $\,\,$ \\
$\,\,$\color{gr} 0.1317\color{black} $\,\,$ & $\,\,$0.3641$\,\,$ & $\,\,$ 1 $\,\,$ & $\,\,$1.1855 $\,\,$ \\
$\,\,$\color{gr} \color{blue}  1/9\color{black} $\,\,$ & $\,\,$0.3071$\,\,$ & $\,\,$0.8435$\,\,$ & $\,\,$ 1  $\,\,$ \\
\end{pmatrix},
\end{equation*}
\end{example}
\newpage
\begin{example}
\begin{equation*}
\mathbf{A} =
\begin{pmatrix}
$\,\,$ 1 $\,\,$ & $\,\,$3$\,\,$ & $\,\,$9$\,\,$ & $\,\,$7 $\,\,$ \\
$\,\,$ 1/3$\,\,$ & $\,\,$ 1 $\,\,$ & $\,\,$2$\,\,$ & $\,\,$3 $\,\,$ \\
$\,\,$ 1/9$\,\,$ & $\,\,$ 1/2$\,\,$ & $\,\,$ 1 $\,\,$ & $\,\,$2 $\,\,$ \\
$\,\,$ 1/7$\,\,$ & $\,\,$ 1/3$\,\,$ & $\,\,$ 1/2$\,\,$ & $\,\,$ 1  $\,\,$ \\
\end{pmatrix},
\qquad
\lambda_{\max} =
4.0762,
\qquad
CR = 0.0287
\end{equation*}

\begin{equation*}
\mathbf{w}^{cos} =
\begin{pmatrix}
0.622835\\
\color{red} 0.203644\color{black} \\
0.103673\\
0.069847
\end{pmatrix}\end{equation*}
\begin{equation*}
\left[ \frac{{w}^{cos}_i}{{w}^{cos}_j} \right] =
\begin{pmatrix}
$\,\,$ 1 $\,\,$ & $\,\,$\color{red} 3.0584\color{black} $\,\,$ & $\,\,$6.0077$\,\,$ & $\,\,$8.9171$\,\,$ \\
$\,\,$\color{red} 0.3270\color{black} $\,\,$ & $\,\,$ 1 $\,\,$ & $\,\,$\color{red} 1.9643\color{black} $\,\,$ & $\,\,$\color{red} 2.9156\color{black}   $\,\,$ \\
$\,\,$0.1665$\,\,$ & $\,\,$\color{red} 0.5091\color{black} $\,\,$ & $\,\,$ 1 $\,\,$ & $\,\,$1.4843 $\,\,$ \\
$\,\,$0.1121$\,\,$ & $\,\,$\color{red} 0.3430\color{black} $\,\,$ & $\,\,$0.6737$\,\,$ & $\,\,$ 1  $\,\,$ \\
\end{pmatrix},
\end{equation*}

\begin{equation*}
\mathbf{w}^{\prime} =
\begin{pmatrix}
0.620538\\
0.206581\\
0.103291\\
0.069590
\end{pmatrix} =
0.996312\cdot
\begin{pmatrix}
0.622835\\
\color{gr} 0.207346\color{black} \\
0.103673\\
0.069847
\end{pmatrix},
\end{equation*}
\begin{equation*}
\left[ \frac{{w}^{\prime}_i}{{w}^{\prime}_j} \right] =
\begin{pmatrix}
$\,\,$ 1 $\,\,$ & $\,\,$\color{gr} 3.0038\color{black} $\,\,$ & $\,\,$6.0077$\,\,$ & $\,\,$8.9171$\,\,$ \\
$\,\,$\color{gr} 0.3329\color{black} $\,\,$ & $\,\,$ 1 $\,\,$ & $\,\,$\color{gr} \color{blue} 2\color{black} $\,\,$ & $\,\,$\color{gr} 2.9686\color{black}   $\,\,$ \\
$\,\,$0.1665$\,\,$ & $\,\,$\color{gr} \color{blue}  1/2\color{black} $\,\,$ & $\,\,$ 1 $\,\,$ & $\,\,$1.4843 $\,\,$ \\
$\,\,$0.1121$\,\,$ & $\,\,$\color{gr} 0.3369\color{black} $\,\,$ & $\,\,$0.6737$\,\,$ & $\,\,$ 1  $\,\,$ \\
\end{pmatrix},
\end{equation*}
\end{example}
\newpage
\begin{example}
\begin{equation*}
\mathbf{A} =
\begin{pmatrix}
$\,\,$ 1 $\,\,$ & $\,\,$3$\,\,$ & $\,\,$9$\,\,$ & $\,\,$9 $\,\,$ \\
$\,\,$ 1/3$\,\,$ & $\,\,$ 1 $\,\,$ & $\,\,$2$\,\,$ & $\,\,$4 $\,\,$ \\
$\,\,$ 1/9$\,\,$ & $\,\,$ 1/2$\,\,$ & $\,\,$ 1 $\,\,$ & $\,\,$3 $\,\,$ \\
$\,\,$ 1/9$\,\,$ & $\,\,$ 1/4$\,\,$ & $\,\,$ 1/3$\,\,$ & $\,\,$ 1  $\,\,$ \\
\end{pmatrix},
\qquad
\lambda_{\max} =
4.1031,
\qquad
CR = 0.0389
\end{equation*}

\begin{equation*}
\mathbf{w}^{cos} =
\begin{pmatrix}
0.628027\\
\color{red} 0.207466\color{black} \\
0.111269\\
0.053238
\end{pmatrix}\end{equation*}
\begin{equation*}
\left[ \frac{{w}^{cos}_i}{{w}^{cos}_j} \right] =
\begin{pmatrix}
$\,\,$ 1 $\,\,$ & $\,\,$\color{red} 3.0271\color{black} $\,\,$ & $\,\,$5.6442$\,\,$ & $\,\,$11.7967$\,\,$ \\
$\,\,$\color{red} 0.3303\color{black} $\,\,$ & $\,\,$ 1 $\,\,$ & $\,\,$\color{red} 1.8646\color{black} $\,\,$ & $\,\,$\color{red} 3.8970\color{black}   $\,\,$ \\
$\,\,$0.1772$\,\,$ & $\,\,$\color{red} 0.5363\color{black} $\,\,$ & $\,\,$ 1 $\,\,$ & $\,\,$2.0900 $\,\,$ \\
$\,\,$0.0848$\,\,$ & $\,\,$\color{red} 0.2566\color{black} $\,\,$ & $\,\,$0.4785$\,\,$ & $\,\,$ 1  $\,\,$ \\
\end{pmatrix},
\end{equation*}

\begin{equation*}
\mathbf{w}^{\prime} =
\begin{pmatrix}
0.626851\\
0.208950\\
0.111060\\
0.053138
\end{pmatrix} =
0.998128\cdot
\begin{pmatrix}
0.628027\\
\color{gr} 0.209342\color{black} \\
0.111269\\
0.053238
\end{pmatrix},
\end{equation*}
\begin{equation*}
\left[ \frac{{w}^{\prime}_i}{{w}^{\prime}_j} \right] =
\begin{pmatrix}
$\,\,$ 1 $\,\,$ & $\,\,$\color{gr} \color{blue} 3\color{black} $\,\,$ & $\,\,$5.6442$\,\,$ & $\,\,$11.7967$\,\,$ \\
$\,\,$\color{gr} \color{blue}  1/3\color{black} $\,\,$ & $\,\,$ 1 $\,\,$ & $\,\,$\color{gr} 1.8814\color{black} $\,\,$ & $\,\,$\color{gr} 3.9322\color{black}   $\,\,$ \\
$\,\,$0.1772$\,\,$ & $\,\,$\color{gr} 0.5315\color{black} $\,\,$ & $\,\,$ 1 $\,\,$ & $\,\,$2.0900 $\,\,$ \\
$\,\,$0.0848$\,\,$ & $\,\,$\color{gr} 0.2543\color{black} $\,\,$ & $\,\,$0.4785$\,\,$ & $\,\,$ 1  $\,\,$ \\
\end{pmatrix},
\end{equation*}
\end{example}
\newpage
\begin{example}
\begin{equation*}
\mathbf{A} =
\begin{pmatrix}
$\,\,$ 1 $\,\,$ & $\,\,$3$\,\,$ & $\,\,$9$\,\,$ & $\,\,$9 $\,\,$ \\
$\,\,$ 1/3$\,\,$ & $\,\,$ 1 $\,\,$ & $\,\,$5$\,\,$ & $\,\,$2 $\,\,$ \\
$\,\,$ 1/9$\,\,$ & $\,\,$ 1/5$\,\,$ & $\,\,$ 1 $\,\,$ & $\,\,$2 $\,\,$ \\
$\,\,$ 1/9$\,\,$ & $\,\,$ 1/2$\,\,$ & $\,\,$ 1/2$\,\,$ & $\,\,$ 1  $\,\,$ \\
\end{pmatrix},
\qquad
\lambda_{\max} =
4.2277,
\qquad
CR = 0.0859
\end{equation*}

\begin{equation*}
\mathbf{w}^{cos} =
\begin{pmatrix}
\color{red} 0.625935\color{black} \\
0.223519\\
0.080374\\
0.070172
\end{pmatrix}\end{equation*}
\begin{equation*}
\left[ \frac{{w}^{cos}_i}{{w}^{cos}_j} \right] =
\begin{pmatrix}
$\,\,$ 1 $\,\,$ & $\,\,$\color{red} 2.8004\color{black} $\,\,$ & $\,\,$\color{red} 7.7877\color{black} $\,\,$ & $\,\,$\color{red} 8.9200\color{black} $\,\,$ \\
$\,\,$\color{red} 0.3571\color{black} $\,\,$ & $\,\,$ 1 $\,\,$ & $\,\,$2.7810$\,\,$ & $\,\,$3.1853  $\,\,$ \\
$\,\,$\color{red} 0.1284\color{black} $\,\,$ & $\,\,$0.3596$\,\,$ & $\,\,$ 1 $\,\,$ & $\,\,$1.1454 $\,\,$ \\
$\,\,$\color{red} 0.1121\color{black} $\,\,$ & $\,\,$0.3139$\,\,$ & $\,\,$0.8731$\,\,$ & $\,\,$ 1  $\,\,$ \\
\end{pmatrix},
\end{equation*}

\begin{equation*}
\mathbf{w}^{\prime} =
\begin{pmatrix}
0.628022\\
0.222272\\
0.079926\\
0.069780
\end{pmatrix} =
0.994421\cdot
\begin{pmatrix}
\color{gr} 0.631546\color{black} \\
0.223519\\
0.080374\\
0.070172
\end{pmatrix},
\end{equation*}
\begin{equation*}
\left[ \frac{{w}^{\prime}_i}{{w}^{\prime}_j} \right] =
\begin{pmatrix}
$\,\,$ 1 $\,\,$ & $\,\,$\color{gr} 2.8255\color{black} $\,\,$ & $\,\,$\color{gr} 7.8575\color{black} $\,\,$ & $\,\,$\color{gr} \color{blue} 9\color{black} $\,\,$ \\
$\,\,$\color{gr} 0.3539\color{black} $\,\,$ & $\,\,$ 1 $\,\,$ & $\,\,$2.7810$\,\,$ & $\,\,$3.1853  $\,\,$ \\
$\,\,$\color{gr} 0.1273\color{black} $\,\,$ & $\,\,$0.3596$\,\,$ & $\,\,$ 1 $\,\,$ & $\,\,$1.1454 $\,\,$ \\
$\,\,$\color{gr} \color{blue}  1/9\color{black} $\,\,$ & $\,\,$0.3139$\,\,$ & $\,\,$0.8731$\,\,$ & $\,\,$ 1  $\,\,$ \\
\end{pmatrix},
\end{equation*}
\end{example}
\newpage
\begin{example}
\begin{equation*}
\mathbf{A} =
\begin{pmatrix}
$\,\,$ 1 $\,\,$ & $\,\,$4$\,\,$ & $\,\,$2$\,\,$ & $\,\,$3 $\,\,$ \\
$\,\,$ 1/4$\,\,$ & $\,\,$ 1 $\,\,$ & $\,\,$1$\,\,$ & $\,\,$4 $\,\,$ \\
$\,\,$ 1/2$\,\,$ & $\,\,$ 1 $\,\,$ & $\,\,$ 1 $\,\,$ & $\,\,$2 $\,\,$ \\
$\,\,$ 1/3$\,\,$ & $\,\,$ 1/4$\,\,$ & $\,\,$ 1/2$\,\,$ & $\,\,$ 1  $\,\,$ \\
\end{pmatrix},
\qquad
\lambda_{\max} =
4.2512,
\qquad
CR = 0.0947
\end{equation*}

\begin{equation*}
\mathbf{w}^{cos} =
\begin{pmatrix}
0.457005\\
0.230200\\
\color{red} 0.207456\color{black} \\
0.105339
\end{pmatrix}\end{equation*}
\begin{equation*}
\left[ \frac{{w}^{cos}_i}{{w}^{cos}_j} \right] =
\begin{pmatrix}
$\,\,$ 1 $\,\,$ & $\,\,$1.9853$\,\,$ & $\,\,$\color{red} 2.2029\color{black} $\,\,$ & $\,\,$4.3384$\,\,$ \\
$\,\,$0.5037$\,\,$ & $\,\,$ 1 $\,\,$ & $\,\,$\color{red} 1.1096\color{black} $\,\,$ & $\,\,$2.1853  $\,\,$ \\
$\,\,$\color{red} 0.4539\color{black} $\,\,$ & $\,\,$\color{red} 0.9012\color{black} $\,\,$ & $\,\,$ 1 $\,\,$ & $\,\,$\color{red} 1.9694\color{black}  $\,\,$ \\
$\,\,$0.2305$\,\,$ & $\,\,$0.4576$\,\,$ & $\,\,$\color{red} 0.5078\color{black} $\,\,$ & $\,\,$ 1  $\,\,$ \\
\end{pmatrix},
\end{equation*}

\begin{equation*}
\mathbf{w}^{\prime} =
\begin{pmatrix}
0.455538\\
0.229461\\
0.210001\\
0.105000
\end{pmatrix} =
0.996789\cdot
\begin{pmatrix}
0.457005\\
0.230200\\
\color{gr} 0.210677\color{black} \\
0.105339
\end{pmatrix},
\end{equation*}
\begin{equation*}
\left[ \frac{{w}^{\prime}_i}{{w}^{\prime}_j} \right] =
\begin{pmatrix}
$\,\,$ 1 $\,\,$ & $\,\,$1.9853$\,\,$ & $\,\,$\color{gr} 2.1692\color{black} $\,\,$ & $\,\,$4.3384$\,\,$ \\
$\,\,$0.5037$\,\,$ & $\,\,$ 1 $\,\,$ & $\,\,$\color{gr} 1.0927\color{black} $\,\,$ & $\,\,$2.1853  $\,\,$ \\
$\,\,$\color{gr} 0.4610\color{black} $\,\,$ & $\,\,$\color{gr} 0.9152\color{black} $\,\,$ & $\,\,$ 1 $\,\,$ & $\,\,$\color{gr} \color{blue} 2\color{black}  $\,\,$ \\
$\,\,$0.2305$\,\,$ & $\,\,$0.4576$\,\,$ & $\,\,$\color{gr} \color{blue}  1/2\color{black} $\,\,$ & $\,\,$ 1  $\,\,$ \\
\end{pmatrix},
\end{equation*}
\end{example}
\newpage
\begin{example}
\begin{equation*}
\mathbf{A} =
\begin{pmatrix}
$\,\,$ 1 $\,\,$ & $\,\,$4$\,\,$ & $\,\,$2$\,\,$ & $\,\,$4 $\,\,$ \\
$\,\,$ 1/4$\,\,$ & $\,\,$ 1 $\,\,$ & $\,\,$1$\,\,$ & $\,\,$5 $\,\,$ \\
$\,\,$ 1/2$\,\,$ & $\,\,$ 1 $\,\,$ & $\,\,$ 1 $\,\,$ & $\,\,$3 $\,\,$ \\
$\,\,$ 1/4$\,\,$ & $\,\,$ 1/5$\,\,$ & $\,\,$ 1/3$\,\,$ & $\,\,$ 1  $\,\,$ \\
\end{pmatrix},
\qquad
\lambda_{\max} =
4.2277,
\qquad
CR = 0.0859
\end{equation*}

\begin{equation*}
\mathbf{w}^{cos} =
\begin{pmatrix}
0.469490\\
0.230507\\
\color{red} 0.220586\color{black} \\
0.079416
\end{pmatrix}\end{equation*}
\begin{equation*}
\left[ \frac{{w}^{cos}_i}{{w}^{cos}_j} \right] =
\begin{pmatrix}
$\,\,$ 1 $\,\,$ & $\,\,$2.0368$\,\,$ & $\,\,$\color{red} 2.1284\color{black} $\,\,$ & $\,\,$5.9118$\,\,$ \\
$\,\,$0.4910$\,\,$ & $\,\,$ 1 $\,\,$ & $\,\,$\color{red} 1.0450\color{black} $\,\,$ & $\,\,$2.9025  $\,\,$ \\
$\,\,$\color{red} 0.4698\color{black} $\,\,$ & $\,\,$\color{red} 0.9570\color{black} $\,\,$ & $\,\,$ 1 $\,\,$ & $\,\,$\color{red} 2.7776\color{black}  $\,\,$ \\
$\,\,$0.1692$\,\,$ & $\,\,$0.3445$\,\,$ & $\,\,$\color{red} 0.3600\color{black} $\,\,$ & $\,\,$ 1  $\,\,$ \\
\end{pmatrix},
\end{equation*}

\begin{equation*}
\mathbf{w}^{\prime} =
\begin{pmatrix}
0.464878\\
0.228243\\
0.228243\\
0.078636
\end{pmatrix} =
0.990176\cdot
\begin{pmatrix}
0.469490\\
0.230507\\
\color{gr} 0.230507\color{black} \\
0.079416
\end{pmatrix},
\end{equation*}
\begin{equation*}
\left[ \frac{{w}^{\prime}_i}{{w}^{\prime}_j} \right] =
\begin{pmatrix}
$\,\,$ 1 $\,\,$ & $\,\,$2.0368$\,\,$ & $\,\,$\color{gr} 2.0368\color{black} $\,\,$ & $\,\,$5.9118$\,\,$ \\
$\,\,$0.4910$\,\,$ & $\,\,$ 1 $\,\,$ & $\,\,$\color{gr} \color{blue} 1\color{black} $\,\,$ & $\,\,$2.9025  $\,\,$ \\
$\,\,$\color{gr} 0.4910\color{black} $\,\,$ & $\,\,$\color{gr} \color{blue} 1\color{black} $\,\,$ & $\,\,$ 1 $\,\,$ & $\,\,$\color{gr} 2.9025\color{black}  $\,\,$ \\
$\,\,$0.1692$\,\,$ & $\,\,$0.3445$\,\,$ & $\,\,$\color{gr} 0.3445\color{black} $\,\,$ & $\,\,$ 1  $\,\,$ \\
\end{pmatrix},
\end{equation*}
\end{example}
\newpage
\begin{example}
\begin{equation*}
\mathbf{A} =
\begin{pmatrix}
$\,\,$ 1 $\,\,$ & $\,\,$4$\,\,$ & $\,\,$2$\,\,$ & $\,\,$5 $\,\,$ \\
$\,\,$ 1/4$\,\,$ & $\,\,$ 1 $\,\,$ & $\,\,$1$\,\,$ & $\,\,$6 $\,\,$ \\
$\,\,$ 1/2$\,\,$ & $\,\,$ 1 $\,\,$ & $\,\,$ 1 $\,\,$ & $\,\,$4 $\,\,$ \\
$\,\,$ 1/5$\,\,$ & $\,\,$ 1/6$\,\,$ & $\,\,$ 1/4$\,\,$ & $\,\,$ 1  $\,\,$ \\
\end{pmatrix},
\qquad
\lambda_{\max} =
4.2162,
\qquad
CR = 0.0815
\end{equation*}

\begin{equation*}
\mathbf{w}^{cos} =
\begin{pmatrix}
0.477042\\
0.230392\\
\color{red} 0.228669\color{black} \\
0.063897
\end{pmatrix}\end{equation*}
\begin{equation*}
\left[ \frac{{w}^{cos}_i}{{w}^{cos}_j} \right] =
\begin{pmatrix}
$\,\,$ 1 $\,\,$ & $\,\,$2.0706$\,\,$ & $\,\,$\color{red} 2.0862\color{black} $\,\,$ & $\,\,$7.4658$\,\,$ \\
$\,\,$0.4830$\,\,$ & $\,\,$ 1 $\,\,$ & $\,\,$\color{red} 1.0075\color{black} $\,\,$ & $\,\,$3.6057  $\,\,$ \\
$\,\,$\color{red} 0.4793\color{black} $\,\,$ & $\,\,$\color{red} 0.9925\color{black} $\,\,$ & $\,\,$ 1 $\,\,$ & $\,\,$\color{red} 3.5787\color{black}  $\,\,$ \\
$\,\,$0.1339$\,\,$ & $\,\,$0.2773$\,\,$ & $\,\,$\color{red} 0.2794\color{black} $\,\,$ & $\,\,$ 1  $\,\,$ \\
\end{pmatrix},
\end{equation*}

\begin{equation*}
\mathbf{w}^{\prime} =
\begin{pmatrix}
0.476221\\
0.229996\\
0.229996\\
0.063787
\end{pmatrix} =
0.998280\cdot
\begin{pmatrix}
0.477042\\
0.230392\\
\color{gr} 0.230392\color{black} \\
0.063897
\end{pmatrix},
\end{equation*}
\begin{equation*}
\left[ \frac{{w}^{\prime}_i}{{w}^{\prime}_j} \right] =
\begin{pmatrix}
$\,\,$ 1 $\,\,$ & $\,\,$2.0706$\,\,$ & $\,\,$\color{gr} 2.0706\color{black} $\,\,$ & $\,\,$7.4658$\,\,$ \\
$\,\,$0.4830$\,\,$ & $\,\,$ 1 $\,\,$ & $\,\,$\color{gr} \color{blue} 1\color{black} $\,\,$ & $\,\,$3.6057  $\,\,$ \\
$\,\,$\color{gr} 0.4830\color{black} $\,\,$ & $\,\,$\color{gr} \color{blue} 1\color{black} $\,\,$ & $\,\,$ 1 $\,\,$ & $\,\,$\color{gr} 3.6057\color{black}  $\,\,$ \\
$\,\,$0.1339$\,\,$ & $\,\,$0.2773$\,\,$ & $\,\,$\color{gr} 0.2773\color{black} $\,\,$ & $\,\,$ 1  $\,\,$ \\
\end{pmatrix},
\end{equation*}
\end{example}
\newpage
\begin{example}
\begin{equation*}
\mathbf{A} =
\begin{pmatrix}
$\,\,$ 1 $\,\,$ & $\,\,$4$\,\,$ & $\,\,$2$\,\,$ & $\,\,$5 $\,\,$ \\
$\,\,$ 1/4$\,\,$ & $\,\,$ 1 $\,\,$ & $\,\,$1$\,\,$ & $\,\,$7 $\,\,$ \\
$\,\,$ 1/2$\,\,$ & $\,\,$ 1 $\,\,$ & $\,\,$ 1 $\,\,$ & $\,\,$4 $\,\,$ \\
$\,\,$ 1/5$\,\,$ & $\,\,$ 1/7$\,\,$ & $\,\,$ 1/4$\,\,$ & $\,\,$ 1  $\,\,$ \\
\end{pmatrix},
\qquad
\lambda_{\max} =
4.2610,
\qquad
CR = 0.0984
\end{equation*}

\begin{equation*}
\mathbf{w}^{cos} =
\begin{pmatrix}
0.473247\\
0.239819\\
\color{red} 0.224813\color{black} \\
0.062122
\end{pmatrix}\end{equation*}
\begin{equation*}
\left[ \frac{{w}^{cos}_i}{{w}^{cos}_j} \right] =
\begin{pmatrix}
$\,\,$ 1 $\,\,$ & $\,\,$1.9734$\,\,$ & $\,\,$\color{red} 2.1051\color{black} $\,\,$ & $\,\,$7.6181$\,\,$ \\
$\,\,$0.5068$\,\,$ & $\,\,$ 1 $\,\,$ & $\,\,$\color{red} 1.0667\color{black} $\,\,$ & $\,\,$3.8605  $\,\,$ \\
$\,\,$\color{red} 0.4750\color{black} $\,\,$ & $\,\,$\color{red} 0.9374\color{black} $\,\,$ & $\,\,$ 1 $\,\,$ & $\,\,$\color{red} 3.6189\color{black}  $\,\,$ \\
$\,\,$0.1313$\,\,$ & $\,\,$0.2590$\,\,$ & $\,\,$\color{red} 0.2763\color{black} $\,\,$ & $\,\,$ 1  $\,\,$ \\
\end{pmatrix},
\end{equation*}

\begin{equation*}
\mathbf{w}^{\prime} =
\begin{pmatrix}
0.467723\\
0.237020\\
0.233861\\
0.061396
\end{pmatrix} =
0.988328\cdot
\begin{pmatrix}
0.473247\\
0.239819\\
\color{gr} 0.236623\color{black} \\
0.062122
\end{pmatrix},
\end{equation*}
\begin{equation*}
\left[ \frac{{w}^{\prime}_i}{{w}^{\prime}_j} \right] =
\begin{pmatrix}
$\,\,$ 1 $\,\,$ & $\,\,$1.9734$\,\,$ & $\,\,$\color{gr} \color{blue} 2\color{black} $\,\,$ & $\,\,$7.6181$\,\,$ \\
$\,\,$0.5068$\,\,$ & $\,\,$ 1 $\,\,$ & $\,\,$\color{gr} 1.0135\color{black} $\,\,$ & $\,\,$3.8605  $\,\,$ \\
$\,\,$\color{gr} \color{blue}  1/2\color{black} $\,\,$ & $\,\,$\color{gr} 0.9867\color{black} $\,\,$ & $\,\,$ 1 $\,\,$ & $\,\,$\color{gr} 3.8090\color{black}  $\,\,$ \\
$\,\,$0.1313$\,\,$ & $\,\,$0.2590$\,\,$ & $\,\,$\color{gr} 0.2625\color{black} $\,\,$ & $\,\,$ 1  $\,\,$ \\
\end{pmatrix},
\end{equation*}
\end{example}
\newpage
\begin{example}
\begin{equation*}
\mathbf{A} =
\begin{pmatrix}
$\,\,$ 1 $\,\,$ & $\,\,$4$\,\,$ & $\,\,$2$\,\,$ & $\,\,$6 $\,\,$ \\
$\,\,$ 1/4$\,\,$ & $\,\,$ 1 $\,\,$ & $\,\,$1$\,\,$ & $\,\,$7 $\,\,$ \\
$\,\,$ 1/2$\,\,$ & $\,\,$ 1 $\,\,$ & $\,\,$ 1 $\,\,$ & $\,\,$4 $\,\,$ \\
$\,\,$ 1/6$\,\,$ & $\,\,$ 1/7$\,\,$ & $\,\,$ 1/4$\,\,$ & $\,\,$ 1  $\,\,$ \\
\end{pmatrix},
\qquad
\lambda_{\max} =
4.2109,
\qquad
CR = 0.0795
\end{equation*}

\begin{equation*}
\mathbf{w}^{cos} =
\begin{pmatrix}
0.485921\\
0.234572\\
\color{red} 0.222305\color{black} \\
0.057202
\end{pmatrix}\end{equation*}
\begin{equation*}
\left[ \frac{{w}^{cos}_i}{{w}^{cos}_j} \right] =
\begin{pmatrix}
$\,\,$ 1 $\,\,$ & $\,\,$2.0715$\,\,$ & $\,\,$\color{red} 2.1858\color{black} $\,\,$ & $\,\,$8.4948$\,\,$ \\
$\,\,$0.4827$\,\,$ & $\,\,$ 1 $\,\,$ & $\,\,$\color{red} 1.0552\color{black} $\,\,$ & $\,\,$4.1007  $\,\,$ \\
$\,\,$\color{red} 0.4575\color{black} $\,\,$ & $\,\,$\color{red} 0.9477\color{black} $\,\,$ & $\,\,$ 1 $\,\,$ & $\,\,$\color{red} 3.8863\color{black}  $\,\,$ \\
$\,\,$0.1177$\,\,$ & $\,\,$0.2439$\,\,$ & $\,\,$\color{red} 0.2573\color{black} $\,\,$ & $\,\,$ 1  $\,\,$ \\
\end{pmatrix},
\end{equation*}

\begin{equation*}
\mathbf{w}^{\prime} =
\begin{pmatrix}
0.482780\\
0.233056\\
0.227331\\
0.056833
\end{pmatrix} =
0.993537\cdot
\begin{pmatrix}
0.485921\\
0.234572\\
\color{gr} 0.228810\color{black} \\
0.057202
\end{pmatrix},
\end{equation*}
\begin{equation*}
\left[ \frac{{w}^{\prime}_i}{{w}^{\prime}_j} \right] =
\begin{pmatrix}
$\,\,$ 1 $\,\,$ & $\,\,$2.0715$\,\,$ & $\,\,$\color{gr} 2.1237\color{black} $\,\,$ & $\,\,$8.4948$\,\,$ \\
$\,\,$0.4827$\,\,$ & $\,\,$ 1 $\,\,$ & $\,\,$\color{gr} 1.0252\color{black} $\,\,$ & $\,\,$4.1007  $\,\,$ \\
$\,\,$\color{gr} 0.4709\color{black} $\,\,$ & $\,\,$\color{gr} 0.9754\color{black} $\,\,$ & $\,\,$ 1 $\,\,$ & $\,\,$\color{gr} \color{blue} 4\color{black}  $\,\,$ \\
$\,\,$0.1177$\,\,$ & $\,\,$0.2439$\,\,$ & $\,\,$\color{gr} \color{blue}  1/4\color{black} $\,\,$ & $\,\,$ 1  $\,\,$ \\
\end{pmatrix},
\end{equation*}
\end{example}
\newpage
\begin{example}
\begin{equation*}
\mathbf{A} =
\begin{pmatrix}
$\,\,$ 1 $\,\,$ & $\,\,$4$\,\,$ & $\,\,$2$\,\,$ & $\,\,$6 $\,\,$ \\
$\,\,$ 1/4$\,\,$ & $\,\,$ 1 $\,\,$ & $\,\,$1$\,\,$ & $\,\,$8 $\,\,$ \\
$\,\,$ 1/2$\,\,$ & $\,\,$ 1 $\,\,$ & $\,\,$ 1 $\,\,$ & $\,\,$4 $\,\,$ \\
$\,\,$ 1/6$\,\,$ & $\,\,$ 1/8$\,\,$ & $\,\,$ 1/4$\,\,$ & $\,\,$ 1  $\,\,$ \\
\end{pmatrix},
\qquad
\lambda_{\max} =
4.2512,
\qquad
CR = 0.0947
\end{equation*}

\begin{equation*}
\mathbf{w}^{cos} =
\begin{pmatrix}
0.482150\\
0.242702\\
\color{red} 0.219324\color{black} \\
0.055825
\end{pmatrix}\end{equation*}
\begin{equation*}
\left[ \frac{{w}^{cos}_i}{{w}^{cos}_j} \right] =
\begin{pmatrix}
$\,\,$ 1 $\,\,$ & $\,\,$1.9866$\,\,$ & $\,\,$\color{red} 2.1983\color{black} $\,\,$ & $\,\,$8.6368$\,\,$ \\
$\,\,$0.5034$\,\,$ & $\,\,$ 1 $\,\,$ & $\,\,$\color{red} 1.1066\color{black} $\,\,$ & $\,\,$4.3476  $\,\,$ \\
$\,\,$\color{red} 0.4549\color{black} $\,\,$ & $\,\,$\color{red} 0.9037\color{black} $\,\,$ & $\,\,$ 1 $\,\,$ & $\,\,$\color{red} 3.9288\color{black}  $\,\,$ \\
$\,\,$0.1158$\,\,$ & $\,\,$0.2300$\,\,$ & $\,\,$\color{red} 0.2545\color{black} $\,\,$ & $\,\,$ 1  $\,\,$ \\
\end{pmatrix},
\end{equation*}

\begin{equation*}
\mathbf{w}^{\prime} =
\begin{pmatrix}
0.480241\\
0.241741\\
0.222415\\
0.055604
\end{pmatrix} =
0.996040\cdot
\begin{pmatrix}
0.482150\\
0.242702\\
\color{gr} 0.223299\color{black} \\
0.055825
\end{pmatrix},
\end{equation*}
\begin{equation*}
\left[ \frac{{w}^{\prime}_i}{{w}^{\prime}_j} \right] =
\begin{pmatrix}
$\,\,$ 1 $\,\,$ & $\,\,$1.9866$\,\,$ & $\,\,$\color{gr} 2.1592\color{black} $\,\,$ & $\,\,$8.6368$\,\,$ \\
$\,\,$0.5034$\,\,$ & $\,\,$ 1 $\,\,$ & $\,\,$\color{gr} 1.0869\color{black} $\,\,$ & $\,\,$4.3476  $\,\,$ \\
$\,\,$\color{gr} 0.4631\color{black} $\,\,$ & $\,\,$\color{gr} 0.9201\color{black} $\,\,$ & $\,\,$ 1 $\,\,$ & $\,\,$\color{gr} \color{blue} 4\color{black}  $\,\,$ \\
$\,\,$0.1158$\,\,$ & $\,\,$0.2300$\,\,$ & $\,\,$\color{gr} \color{blue}  1/4\color{black} $\,\,$ & $\,\,$ 1  $\,\,$ \\
\end{pmatrix},
\end{equation*}
\end{example}
\newpage
\begin{example}
\begin{equation*}
\mathbf{A} =
\begin{pmatrix}
$\,\,$ 1 $\,\,$ & $\,\,$4$\,\,$ & $\,\,$2$\,\,$ & $\,\,$6 $\,\,$ \\
$\,\,$ 1/4$\,\,$ & $\,\,$ 1 $\,\,$ & $\,\,$1$\,\,$ & $\,\,$8 $\,\,$ \\
$\,\,$ 1/2$\,\,$ & $\,\,$ 1 $\,\,$ & $\,\,$ 1 $\,\,$ & $\,\,$5 $\,\,$ \\
$\,\,$ 1/6$\,\,$ & $\,\,$ 1/8$\,\,$ & $\,\,$ 1/5$\,\,$ & $\,\,$ 1  $\,\,$ \\
\end{pmatrix},
\qquad
\lambda_{\max} =
4.2460,
\qquad
CR = 0.0928
\end{equation*}

\begin{equation*}
\mathbf{w}^{cos} =
\begin{pmatrix}
0.478727\\
0.238437\\
\color{red} 0.230649\color{black} \\
0.052187
\end{pmatrix}\end{equation*}
\begin{equation*}
\left[ \frac{{w}^{cos}_i}{{w}^{cos}_j} \right] =
\begin{pmatrix}
$\,\,$ 1 $\,\,$ & $\,\,$2.0078$\,\,$ & $\,\,$\color{red} 2.0756\color{black} $\,\,$ & $\,\,$9.1733$\,\,$ \\
$\,\,$0.4981$\,\,$ & $\,\,$ 1 $\,\,$ & $\,\,$\color{red} 1.0338\color{black} $\,\,$ & $\,\,$4.5689  $\,\,$ \\
$\,\,$\color{red} 0.4818\color{black} $\,\,$ & $\,\,$\color{red} 0.9673\color{black} $\,\,$ & $\,\,$ 1 $\,\,$ & $\,\,$\color{red} 4.4196\color{black}  $\,\,$ \\
$\,\,$0.1090$\,\,$ & $\,\,$0.2189$\,\,$ & $\,\,$\color{red} 0.2263\color{black} $\,\,$ & $\,\,$ 1  $\,\,$ \\
\end{pmatrix},
\end{equation*}

\begin{equation*}
\mathbf{w}^{\prime} =
\begin{pmatrix}
0.475028\\
0.236594\\
0.236594\\
0.051784
\end{pmatrix} =
0.992272\cdot
\begin{pmatrix}
0.478727\\
0.238437\\
\color{gr} 0.238437\color{black} \\
0.052187
\end{pmatrix},
\end{equation*}
\begin{equation*}
\left[ \frac{{w}^{\prime}_i}{{w}^{\prime}_j} \right] =
\begin{pmatrix}
$\,\,$ 1 $\,\,$ & $\,\,$2.0078$\,\,$ & $\,\,$\color{gr} 2.0078\color{black} $\,\,$ & $\,\,$9.1733$\,\,$ \\
$\,\,$0.4981$\,\,$ & $\,\,$ 1 $\,\,$ & $\,\,$\color{gr} \color{blue} 1\color{black} $\,\,$ & $\,\,$4.5689  $\,\,$ \\
$\,\,$\color{gr} 0.4981\color{black} $\,\,$ & $\,\,$\color{gr} \color{blue} 1\color{black} $\,\,$ & $\,\,$ 1 $\,\,$ & $\,\,$\color{gr} 4.5689\color{black}  $\,\,$ \\
$\,\,$0.1090$\,\,$ & $\,\,$0.2189$\,\,$ & $\,\,$\color{gr} 0.2189\color{black} $\,\,$ & $\,\,$ 1  $\,\,$ \\
\end{pmatrix},
\end{equation*}
\end{example}
\newpage
\begin{example}
\begin{equation*}
\mathbf{A} =
\begin{pmatrix}
$\,\,$ 1 $\,\,$ & $\,\,$4$\,\,$ & $\,\,$2$\,\,$ & $\,\,$7 $\,\,$ \\
$\,\,$ 1/4$\,\,$ & $\,\,$ 1 $\,\,$ & $\,\,$1$\,\,$ & $\,\,$8 $\,\,$ \\
$\,\,$ 1/2$\,\,$ & $\,\,$ 1 $\,\,$ & $\,\,$ 1 $\,\,$ & $\,\,$5 $\,\,$ \\
$\,\,$ 1/7$\,\,$ & $\,\,$ 1/8$\,\,$ & $\,\,$ 1/5$\,\,$ & $\,\,$ 1  $\,\,$ \\
\end{pmatrix},
\qquad
\lambda_{\max} =
4.2035,
\qquad
CR = 0.0767
\end{equation*}

\begin{equation*}
\mathbf{w}^{cos} =
\begin{pmatrix}
0.489307\\
0.233934\\
\color{red} 0.228166\color{black} \\
0.048593
\end{pmatrix}\end{equation*}
\begin{equation*}
\left[ \frac{{w}^{cos}_i}{{w}^{cos}_j} \right] =
\begin{pmatrix}
$\,\,$ 1 $\,\,$ & $\,\,$2.0916$\,\,$ & $\,\,$\color{red} 2.1445\color{black} $\,\,$ & $\,\,$10.0696$\,\,$ \\
$\,\,$0.4781$\,\,$ & $\,\,$ 1 $\,\,$ & $\,\,$\color{red} 1.0253\color{black} $\,\,$ & $\,\,$4.8142  $\,\,$ \\
$\,\,$\color{red} 0.4663\color{black} $\,\,$ & $\,\,$\color{red} 0.9753\color{black} $\,\,$ & $\,\,$ 1 $\,\,$ & $\,\,$\color{red} 4.6955\color{black}  $\,\,$ \\
$\,\,$0.0993$\,\,$ & $\,\,$0.2077$\,\,$ & $\,\,$\color{red} 0.2130\color{black} $\,\,$ & $\,\,$ 1  $\,\,$ \\
\end{pmatrix},
\end{equation*}

\begin{equation*}
\mathbf{w}^{\prime} =
\begin{pmatrix}
0.486501\\
0.232593\\
0.232593\\
0.048314
\end{pmatrix} =
0.994264\cdot
\begin{pmatrix}
0.489307\\
0.233934\\
\color{gr} 0.233934\color{black} \\
0.048593
\end{pmatrix},
\end{equation*}
\begin{equation*}
\left[ \frac{{w}^{\prime}_i}{{w}^{\prime}_j} \right] =
\begin{pmatrix}
$\,\,$ 1 $\,\,$ & $\,\,$2.0916$\,\,$ & $\,\,$\color{gr} 2.0916\color{black} $\,\,$ & $\,\,$10.0696$\,\,$ \\
$\,\,$0.4781$\,\,$ & $\,\,$ 1 $\,\,$ & $\,\,$\color{gr} \color{blue} 1\color{black} $\,\,$ & $\,\,$4.8142  $\,\,$ \\
$\,\,$\color{gr} 0.4781\color{black} $\,\,$ & $\,\,$\color{gr} \color{blue} 1\color{black} $\,\,$ & $\,\,$ 1 $\,\,$ & $\,\,$\color{gr} 4.8142\color{black}  $\,\,$ \\
$\,\,$0.0993$\,\,$ & $\,\,$0.2077$\,\,$ & $\,\,$\color{gr} 0.2077\color{black} $\,\,$ & $\,\,$ 1  $\,\,$ \\
\end{pmatrix},
\end{equation*}
\end{example}
\newpage
\begin{example}
\begin{equation*}
\mathbf{A} =
\begin{pmatrix}
$\,\,$ 1 $\,\,$ & $\,\,$4$\,\,$ & $\,\,$2$\,\,$ & $\,\,$7 $\,\,$ \\
$\,\,$ 1/4$\,\,$ & $\,\,$ 1 $\,\,$ & $\,\,$1$\,\,$ & $\,\,$9 $\,\,$ \\
$\,\,$ 1/2$\,\,$ & $\,\,$ 1 $\,\,$ & $\,\,$ 1 $\,\,$ & $\,\,$5 $\,\,$ \\
$\,\,$ 1/7$\,\,$ & $\,\,$ 1/9$\,\,$ & $\,\,$ 1/5$\,\,$ & $\,\,$ 1  $\,\,$ \\
\end{pmatrix},
\qquad
\lambda_{\max} =
4.2371,
\qquad
CR = 0.0894
\end{equation*}

\begin{equation*}
\mathbf{w}^{cos} =
\begin{pmatrix}
0.485952\\
0.241173\\
\color{red} 0.225338\color{black} \\
0.047537
\end{pmatrix}\end{equation*}
\begin{equation*}
\left[ \frac{{w}^{cos}_i}{{w}^{cos}_j} \right] =
\begin{pmatrix}
$\,\,$ 1 $\,\,$ & $\,\,$2.0150$\,\,$ & $\,\,$\color{red} 2.1565\color{black} $\,\,$ & $\,\,$10.2225$\,\,$ \\
$\,\,$0.4963$\,\,$ & $\,\,$ 1 $\,\,$ & $\,\,$\color{red} 1.0703\color{black} $\,\,$ & $\,\,$5.0733  $\,\,$ \\
$\,\,$\color{red} 0.4637\color{black} $\,\,$ & $\,\,$\color{red} 0.9343\color{black} $\,\,$ & $\,\,$ 1 $\,\,$ & $\,\,$\color{red} 4.7402\color{black}  $\,\,$ \\
$\,\,$0.0978$\,\,$ & $\,\,$0.1971$\,\,$ & $\,\,$\color{red} 0.2110\color{black} $\,\,$ & $\,\,$ 1  $\,\,$ \\
\end{pmatrix},
\end{equation*}

\begin{equation*}
\mathbf{w}^{\prime} =
\begin{pmatrix}
0.480024\\
0.238231\\
0.234788\\
0.046958
\end{pmatrix} =
0.987801\cdot
\begin{pmatrix}
0.485952\\
0.241173\\
\color{gr} 0.237687\color{black} \\
0.047537
\end{pmatrix},
\end{equation*}
\begin{equation*}
\left[ \frac{{w}^{\prime}_i}{{w}^{\prime}_j} \right] =
\begin{pmatrix}
$\,\,$ 1 $\,\,$ & $\,\,$2.0150$\,\,$ & $\,\,$\color{gr} 2.0445\color{black} $\,\,$ & $\,\,$10.2225$\,\,$ \\
$\,\,$0.4963$\,\,$ & $\,\,$ 1 $\,\,$ & $\,\,$\color{gr} 1.0147\color{black} $\,\,$ & $\,\,$5.0733  $\,\,$ \\
$\,\,$\color{gr} 0.4891\color{black} $\,\,$ & $\,\,$\color{gr} 0.9855\color{black} $\,\,$ & $\,\,$ 1 $\,\,$ & $\,\,$\color{gr} \color{blue} 5\color{black}  $\,\,$ \\
$\,\,$0.0978$\,\,$ & $\,\,$0.1971$\,\,$ & $\,\,$\color{gr} \color{blue}  1/5\color{black} $\,\,$ & $\,\,$ 1  $\,\,$ \\
\end{pmatrix},
\end{equation*}
\end{example}
\newpage
\begin{example}
\begin{equation*}
\mathbf{A} =
\begin{pmatrix}
$\,\,$ 1 $\,\,$ & $\,\,$4$\,\,$ & $\,\,$2$\,\,$ & $\,\,$7 $\,\,$ \\
$\,\,$ 1/4$\,\,$ & $\,\,$ 1 $\,\,$ & $\,\,$1$\,\,$ & $\,\,$9 $\,\,$ \\
$\,\,$ 1/2$\,\,$ & $\,\,$ 1 $\,\,$ & $\,\,$ 1 $\,\,$ & $\,\,$6 $\,\,$ \\
$\,\,$ 1/7$\,\,$ & $\,\,$ 1/9$\,\,$ & $\,\,$ 1/6$\,\,$ & $\,\,$ 1  $\,\,$ \\
\end{pmatrix},
\qquad
\lambda_{\max} =
4.2359,
\qquad
CR = 0.0890
\end{equation*}

\begin{equation*}
\mathbf{w}^{cos} =
\begin{pmatrix}
0.482730\\
0.237314\\
\color{red} 0.234934\color{black} \\
0.045023
\end{pmatrix}\end{equation*}
\begin{equation*}
\left[ \frac{{w}^{cos}_i}{{w}^{cos}_j} \right] =
\begin{pmatrix}
$\,\,$ 1 $\,\,$ & $\,\,$2.0341$\,\,$ & $\,\,$\color{red} 2.0547\color{black} $\,\,$ & $\,\,$10.7219$\,\,$ \\
$\,\,$0.4916$\,\,$ & $\,\,$ 1 $\,\,$ & $\,\,$\color{red} 1.0101\color{black} $\,\,$ & $\,\,$5.2710  $\,\,$ \\
$\,\,$\color{red} 0.4867\color{black} $\,\,$ & $\,\,$\color{red} 0.9900\color{black} $\,\,$ & $\,\,$ 1 $\,\,$ & $\,\,$\color{red} 5.2181\color{black}  $\,\,$ \\
$\,\,$0.0933$\,\,$ & $\,\,$0.1897$\,\,$ & $\,\,$\color{red} 0.1916\color{black} $\,\,$ & $\,\,$ 1  $\,\,$ \\
\end{pmatrix},
\end{equation*}

\begin{equation*}
\mathbf{w}^{\prime} =
\begin{pmatrix}
0.481583\\
0.236750\\
0.236750\\
0.044916
\end{pmatrix} =
0.997625\cdot
\begin{pmatrix}
0.482730\\
0.237314\\
\color{gr} 0.237314\color{black} \\
0.045023
\end{pmatrix},
\end{equation*}
\begin{equation*}
\left[ \frac{{w}^{\prime}_i}{{w}^{\prime}_j} \right] =
\begin{pmatrix}
$\,\,$ 1 $\,\,$ & $\,\,$2.0341$\,\,$ & $\,\,$\color{gr} 2.0341\color{black} $\,\,$ & $\,\,$10.7219$\,\,$ \\
$\,\,$0.4916$\,\,$ & $\,\,$ 1 $\,\,$ & $\,\,$\color{gr} \color{blue} 1\color{black} $\,\,$ & $\,\,$5.2710  $\,\,$ \\
$\,\,$\color{gr} 0.4916\color{black} $\,\,$ & $\,\,$\color{gr} \color{blue} 1\color{black} $\,\,$ & $\,\,$ 1 $\,\,$ & $\,\,$\color{gr} 5.2710\color{black}  $\,\,$ \\
$\,\,$0.0933$\,\,$ & $\,\,$0.1897$\,\,$ & $\,\,$\color{gr} 0.1897\color{black} $\,\,$ & $\,\,$ 1  $\,\,$ \\
\end{pmatrix},
\end{equation*}
\end{example}
\newpage
\begin{example}
\begin{equation*}
\mathbf{A} =
\begin{pmatrix}
$\,\,$ 1 $\,\,$ & $\,\,$4$\,\,$ & $\,\,$2$\,\,$ & $\,\,$8 $\,\,$ \\
$\,\,$ 1/4$\,\,$ & $\,\,$ 1 $\,\,$ & $\,\,$1$\,\,$ & $\,\,$8 $\,\,$ \\
$\,\,$ 1/2$\,\,$ & $\,\,$ 1 $\,\,$ & $\,\,$ 1 $\,\,$ & $\,\,$5 $\,\,$ \\
$\,\,$ 1/8$\,\,$ & $\,\,$ 1/8$\,\,$ & $\,\,$ 1/5$\,\,$ & $\,\,$ 1  $\,\,$ \\
\end{pmatrix},
\qquad
\lambda_{\max} =
4.1722,
\qquad
CR = 0.0649
\end{equation*}

\begin{equation*}
\mathbf{w}^{cos} =
\begin{pmatrix}
0.498881\\
0.229568\\
\color{red} 0.225780\color{black} \\
0.045770
\end{pmatrix}\end{equation*}
\begin{equation*}
\left[ \frac{{w}^{cos}_i}{{w}^{cos}_j} \right] =
\begin{pmatrix}
$\,\,$ 1 $\,\,$ & $\,\,$2.1731$\,\,$ & $\,\,$\color{red} 2.2096\color{black} $\,\,$ & $\,\,$10.8997$\,\,$ \\
$\,\,$0.4602$\,\,$ & $\,\,$ 1 $\,\,$ & $\,\,$\color{red} 1.0168\color{black} $\,\,$ & $\,\,$5.0157  $\,\,$ \\
$\,\,$\color{red} 0.4526\color{black} $\,\,$ & $\,\,$\color{red} 0.9835\color{black} $\,\,$ & $\,\,$ 1 $\,\,$ & $\,\,$\color{red} 4.9329\color{black}  $\,\,$ \\
$\,\,$0.0917$\,\,$ & $\,\,$0.1994$\,\,$ & $\,\,$\color{red} 0.2027\color{black} $\,\,$ & $\,\,$ 1  $\,\,$ \\
\end{pmatrix},
\end{equation*}

\begin{equation*}
\mathbf{w}^{\prime} =
\begin{pmatrix}
0.497354\\
0.228865\\
0.228150\\
0.045630
\end{pmatrix} =
0.996939\cdot
\begin{pmatrix}
0.498881\\
0.229568\\
\color{gr} 0.228851\color{black} \\
0.045770
\end{pmatrix},
\end{equation*}
\begin{equation*}
\left[ \frac{{w}^{\prime}_i}{{w}^{\prime}_j} \right] =
\begin{pmatrix}
$\,\,$ 1 $\,\,$ & $\,\,$2.1731$\,\,$ & $\,\,$\color{gr} 2.1799\color{black} $\,\,$ & $\,\,$10.8997$\,\,$ \\
$\,\,$0.4602$\,\,$ & $\,\,$ 1 $\,\,$ & $\,\,$\color{gr} 1.0031\color{black} $\,\,$ & $\,\,$5.0157  $\,\,$ \\
$\,\,$\color{gr} 0.4587\color{black} $\,\,$ & $\,\,$\color{gr} 0.9969\color{black} $\,\,$ & $\,\,$ 1 $\,\,$ & $\,\,$\color{gr} \color{blue} 5\color{black}  $\,\,$ \\
$\,\,$0.0917$\,\,$ & $\,\,$0.1994$\,\,$ & $\,\,$\color{gr} \color{blue}  1/5\color{black} $\,\,$ & $\,\,$ 1  $\,\,$ \\
\end{pmatrix},
\end{equation*}
\end{example}
\newpage
\begin{example}
\begin{equation*}
\mathbf{A} =
\begin{pmatrix}
$\,\,$ 1 $\,\,$ & $\,\,$4$\,\,$ & $\,\,$2$\,\,$ & $\,\,$8 $\,\,$ \\
$\,\,$ 1/4$\,\,$ & $\,\,$ 1 $\,\,$ & $\,\,$1$\,\,$ & $\,\,$9 $\,\,$ \\
$\,\,$ 1/2$\,\,$ & $\,\,$ 1 $\,\,$ & $\,\,$ 1 $\,\,$ & $\,\,$5 $\,\,$ \\
$\,\,$ 1/8$\,\,$ & $\,\,$ 1/9$\,\,$ & $\,\,$ 1/5$\,\,$ & $\,\,$ 1  $\,\,$ \\
\end{pmatrix},
\qquad
\lambda_{\max} =
4.2032,
\qquad
CR = 0.0766
\end{equation*}

\begin{equation*}
\mathbf{w}^{cos} =
\begin{pmatrix}
0.495168\\
0.236842\\
\color{red} 0.223229\color{black} \\
0.044760
\end{pmatrix}\end{equation*}
\begin{equation*}
\left[ \frac{{w}^{cos}_i}{{w}^{cos}_j} \right] =
\begin{pmatrix}
$\,\,$ 1 $\,\,$ & $\,\,$2.0907$\,\,$ & $\,\,$\color{red} 2.2182\color{black} $\,\,$ & $\,\,$11.0626$\,\,$ \\
$\,\,$0.4783$\,\,$ & $\,\,$ 1 $\,\,$ & $\,\,$\color{red} 1.0610\color{black} $\,\,$ & $\,\,$5.2913  $\,\,$ \\
$\,\,$\color{red} 0.4508\color{black} $\,\,$ & $\,\,$\color{red} 0.9425\color{black} $\,\,$ & $\,\,$ 1 $\,\,$ & $\,\,$\color{red} 4.9872\color{black}  $\,\,$ \\
$\,\,$0.0904$\,\,$ & $\,\,$0.1890$\,\,$ & $\,\,$\color{red} 0.2005\color{black} $\,\,$ & $\,\,$ 1  $\,\,$ \\
\end{pmatrix},
\end{equation*}

\begin{equation*}
\mathbf{w}^{\prime} =
\begin{pmatrix}
0.494884\\
0.236707\\
0.223674\\
0.044735
\end{pmatrix} =
0.999427\cdot
\begin{pmatrix}
0.495168\\
0.236842\\
\color{gr} 0.223802\color{black} \\
0.044760
\end{pmatrix},
\end{equation*}
\begin{equation*}
\left[ \frac{{w}^{\prime}_i}{{w}^{\prime}_j} \right] =
\begin{pmatrix}
$\,\,$ 1 $\,\,$ & $\,\,$2.0907$\,\,$ & $\,\,$\color{gr} 2.2125\color{black} $\,\,$ & $\,\,$11.0626$\,\,$ \\
$\,\,$0.4783$\,\,$ & $\,\,$ 1 $\,\,$ & $\,\,$\color{gr} 1.0583\color{black} $\,\,$ & $\,\,$5.2913  $\,\,$ \\
$\,\,$\color{gr} 0.4520\color{black} $\,\,$ & $\,\,$\color{gr} 0.9449\color{black} $\,\,$ & $\,\,$ 1 $\,\,$ & $\,\,$\color{gr} \color{blue} 5\color{black}  $\,\,$ \\
$\,\,$0.0904$\,\,$ & $\,\,$0.1890$\,\,$ & $\,\,$\color{gr} \color{blue}  1/5\color{black} $\,\,$ & $\,\,$ 1  $\,\,$ \\
\end{pmatrix},
\end{equation*}
\end{example}
\newpage
\begin{example}
\begin{equation*}
\mathbf{A} =
\begin{pmatrix}
$\,\,$ 1 $\,\,$ & $\,\,$4$\,\,$ & $\,\,$2$\,\,$ & $\,\,$8 $\,\,$ \\
$\,\,$ 1/4$\,\,$ & $\,\,$ 1 $\,\,$ & $\,\,$1$\,\,$ & $\,\,$9 $\,\,$ \\
$\,\,$ 1/2$\,\,$ & $\,\,$ 1 $\,\,$ & $\,\,$ 1 $\,\,$ & $\,\,$6 $\,\,$ \\
$\,\,$ 1/8$\,\,$ & $\,\,$ 1/9$\,\,$ & $\,\,$ 1/6$\,\,$ & $\,\,$ 1  $\,\,$ \\
\end{pmatrix},
\qquad
\lambda_{\max} =
4.1990,
\qquad
CR = 0.0750
\end{equation*}

\begin{equation*}
\mathbf{w}^{cos} =
\begin{pmatrix}
0.491796\\
0.233377\\
\color{red} 0.232545\color{black} \\
0.042283
\end{pmatrix}\end{equation*}
\begin{equation*}
\left[ \frac{{w}^{cos}_i}{{w}^{cos}_j} \right] =
\begin{pmatrix}
$\,\,$ 1 $\,\,$ & $\,\,$2.1073$\,\,$ & $\,\,$\color{red} 2.1148\color{black} $\,\,$ & $\,\,$11.6312$\,\,$ \\
$\,\,$0.4745$\,\,$ & $\,\,$ 1 $\,\,$ & $\,\,$\color{red} 1.0036\color{black} $\,\,$ & $\,\,$5.5195  $\,\,$ \\
$\,\,$\color{red} 0.4728\color{black} $\,\,$ & $\,\,$\color{red} 0.9964\color{black} $\,\,$ & $\,\,$ 1 $\,\,$ & $\,\,$\color{red} 5.4998\color{black}  $\,\,$ \\
$\,\,$0.0860$\,\,$ & $\,\,$0.1812$\,\,$ & $\,\,$\color{red} 0.1818\color{black} $\,\,$ & $\,\,$ 1  $\,\,$ \\
\end{pmatrix},
\end{equation*}

\begin{equation*}
\mathbf{w}^{\prime} =
\begin{pmatrix}
0.491387\\
0.233183\\
0.233183\\
0.042247
\end{pmatrix} =
0.999169\cdot
\begin{pmatrix}
0.491796\\
0.233377\\
\color{gr} 0.233377\color{black} \\
0.042283
\end{pmatrix},
\end{equation*}
\begin{equation*}
\left[ \frac{{w}^{\prime}_i}{{w}^{\prime}_j} \right] =
\begin{pmatrix}
$\,\,$ 1 $\,\,$ & $\,\,$2.1073$\,\,$ & $\,\,$\color{gr} 2.1073\color{black} $\,\,$ & $\,\,$11.6312$\,\,$ \\
$\,\,$0.4745$\,\,$ & $\,\,$ 1 $\,\,$ & $\,\,$\color{gr} \color{blue} 1\color{black} $\,\,$ & $\,\,$5.5195  $\,\,$ \\
$\,\,$\color{gr} 0.4745\color{black} $\,\,$ & $\,\,$\color{gr} \color{blue} 1\color{black} $\,\,$ & $\,\,$ 1 $\,\,$ & $\,\,$\color{gr} 5.5195\color{black}  $\,\,$ \\
$\,\,$0.0860$\,\,$ & $\,\,$0.1812$\,\,$ & $\,\,$\color{gr} 0.1812\color{black} $\,\,$ & $\,\,$ 1  $\,\,$ \\
\end{pmatrix},
\end{equation*}
\end{example}
\newpage
\begin{example}
\begin{equation*}
\mathbf{A} =
\begin{pmatrix}
$\,\,$ 1 $\,\,$ & $\,\,$4$\,\,$ & $\,\,$2$\,\,$ & $\,\,$8 $\,\,$ \\
$\,\,$ 1/4$\,\,$ & $\,\,$ 1 $\,\,$ & $\,\,$2$\,\,$ & $\,\,$4 $\,\,$ \\
$\,\,$ 1/2$\,\,$ & $\,\,$ 1/2$\,\,$ & $\,\,$ 1 $\,\,$ & $\,\,$3 $\,\,$ \\
$\,\,$ 1/8$\,\,$ & $\,\,$ 1/4$\,\,$ & $\,\,$ 1/3$\,\,$ & $\,\,$ 1  $\,\,$ \\
\end{pmatrix},
\qquad
\lambda_{\max} =
4.1707,
\qquad
CR = 0.0644
\end{equation*}

\begin{equation*}
\mathbf{w}^{cos} =
\begin{pmatrix}
0.516054\\
0.238534\\
0.185931\\
\color{red} 0.059481\color{black}
\end{pmatrix}\end{equation*}
\begin{equation*}
\left[ \frac{{w}^{cos}_i}{{w}^{cos}_j} \right] =
\begin{pmatrix}
$\,\,$ 1 $\,\,$ & $\,\,$2.1634$\,\,$ & $\,\,$2.7755$\,\,$ & $\,\,$\color{red} 8.6759\color{black} $\,\,$ \\
$\,\,$0.4622$\,\,$ & $\,\,$ 1 $\,\,$ & $\,\,$1.2829$\,\,$ & $\,\,$\color{red} 4.0102\color{black}   $\,\,$ \\
$\,\,$0.3603$\,\,$ & $\,\,$0.7795$\,\,$ & $\,\,$ 1 $\,\,$ & $\,\,$\color{red} 3.1259\color{black}  $\,\,$ \\
$\,\,$\color{red} 0.1153\color{black} $\,\,$ & $\,\,$\color{red} 0.2494\color{black} $\,\,$ & $\,\,$\color{red} 0.3199\color{black} $\,\,$ & $\,\,$ 1  $\,\,$ \\
\end{pmatrix},
\end{equation*}

\begin{equation*}
\mathbf{w}^{\prime} =
\begin{pmatrix}
0.515976\\
0.238498\\
0.185902\\
0.059624
\end{pmatrix} =
0.999848\cdot
\begin{pmatrix}
0.516054\\
0.238534\\
0.185931\\
\color{gr} 0.059633\color{black}
\end{pmatrix},
\end{equation*}
\begin{equation*}
\left[ \frac{{w}^{\prime}_i}{{w}^{\prime}_j} \right] =
\begin{pmatrix}
$\,\,$ 1 $\,\,$ & $\,\,$2.1634$\,\,$ & $\,\,$2.7755$\,\,$ & $\,\,$\color{gr} 8.6538\color{black} $\,\,$ \\
$\,\,$0.4622$\,\,$ & $\,\,$ 1 $\,\,$ & $\,\,$1.2829$\,\,$ & $\,\,$\color{gr} \color{blue} 4\color{black}   $\,\,$ \\
$\,\,$0.3603$\,\,$ & $\,\,$0.7795$\,\,$ & $\,\,$ 1 $\,\,$ & $\,\,$\color{gr} 3.1179\color{black}  $\,\,$ \\
$\,\,$\color{gr} 0.1156\color{black} $\,\,$ & $\,\,$\color{gr} \color{blue}  1/4\color{black} $\,\,$ & $\,\,$\color{gr} 0.3207\color{black} $\,\,$ & $\,\,$ 1  $\,\,$ \\
\end{pmatrix},
\end{equation*}
\end{example}
\newpage
\begin{example}
\begin{equation*}
\mathbf{A} =
\begin{pmatrix}
$\,\,$ 1 $\,\,$ & $\,\,$4$\,\,$ & $\,\,$2$\,\,$ & $\,\,$9 $\,\,$ \\
$\,\,$ 1/4$\,\,$ & $\,\,$ 1 $\,\,$ & $\,\,$2$\,\,$ & $\,\,$4 $\,\,$ \\
$\,\,$ 1/2$\,\,$ & $\,\,$ 1/2$\,\,$ & $\,\,$ 1 $\,\,$ & $\,\,$3 $\,\,$ \\
$\,\,$ 1/9$\,\,$ & $\,\,$ 1/4$\,\,$ & $\,\,$ 1/3$\,\,$ & $\,\,$ 1  $\,\,$ \\
\end{pmatrix},
\qquad
\lambda_{\max} =
4.1664,
\qquad
CR = 0.0627
\end{equation*}

\begin{equation*}
\mathbf{w}^{cos} =
\begin{pmatrix}
0.524681\\
0.235102\\
0.183467\\
\color{red} 0.056750\color{black}
\end{pmatrix}\end{equation*}
\begin{equation*}
\left[ \frac{{w}^{cos}_i}{{w}^{cos}_j} \right] =
\begin{pmatrix}
$\,\,$ 1 $\,\,$ & $\,\,$2.2317$\,\,$ & $\,\,$2.8598$\,\,$ & $\,\,$\color{red} 9.2455\color{black} $\,\,$ \\
$\,\,$0.4481$\,\,$ & $\,\,$ 1 $\,\,$ & $\,\,$1.2814$\,\,$ & $\,\,$\color{red} 4.1428\color{black}   $\,\,$ \\
$\,\,$0.3497$\,\,$ & $\,\,$0.7804$\,\,$ & $\,\,$ 1 $\,\,$ & $\,\,$\color{red} 3.2329\color{black}  $\,\,$ \\
$\,\,$\color{red} 0.1082\color{black} $\,\,$ & $\,\,$\color{red} 0.2414\color{black} $\,\,$ & $\,\,$\color{red} 0.3093\color{black} $\,\,$ & $\,\,$ 1  $\,\,$ \\
\end{pmatrix},
\end{equation*}

\begin{equation*}
\mathbf{w}^{\prime} =
\begin{pmatrix}
0.523870\\
0.234739\\
0.183184\\
0.058208
\end{pmatrix} =
0.998454\cdot
\begin{pmatrix}
0.524681\\
0.235102\\
0.183467\\
\color{gr} 0.058298\color{black}
\end{pmatrix},
\end{equation*}
\begin{equation*}
\left[ \frac{{w}^{\prime}_i}{{w}^{\prime}_j} \right] =
\begin{pmatrix}
$\,\,$ 1 $\,\,$ & $\,\,$2.2317$\,\,$ & $\,\,$2.8598$\,\,$ & $\,\,$\color{gr} \color{blue} 9\color{black} $\,\,$ \\
$\,\,$0.4481$\,\,$ & $\,\,$ 1 $\,\,$ & $\,\,$1.2814$\,\,$ & $\,\,$\color{gr} 4.0328\color{black}   $\,\,$ \\
$\,\,$0.3497$\,\,$ & $\,\,$0.7804$\,\,$ & $\,\,$ 1 $\,\,$ & $\,\,$\color{gr} 3.1471\color{black}  $\,\,$ \\
$\,\,$\color{gr} \color{blue}  1/9\color{black} $\,\,$ & $\,\,$\color{gr} 0.2480\color{black} $\,\,$ & $\,\,$\color{gr} 0.3178\color{black} $\,\,$ & $\,\,$ 1  $\,\,$ \\
\end{pmatrix},
\end{equation*}
\end{example}
\newpage
\begin{example}
\begin{equation*}
\mathbf{A} =
\begin{pmatrix}
$\,\,$ 1 $\,\,$ & $\,\,$4$\,\,$ & $\,\,$3$\,\,$ & $\,\,$4 $\,\,$ \\
$\,\,$ 1/4$\,\,$ & $\,\,$ 1 $\,\,$ & $\,\,$4$\,\,$ & $\,\,$2 $\,\,$ \\
$\,\,$ 1/3$\,\,$ & $\,\,$ 1/4$\,\,$ & $\,\,$ 1 $\,\,$ & $\,\,$1 $\,\,$ \\
$\,\,$ 1/4$\,\,$ & $\,\,$ 1/2$\,\,$ & $\,\,$ 1 $\,\,$ & $\,\,$ 1  $\,\,$ \\
\end{pmatrix},
\qquad
\lambda_{\max} =
4.2512,
\qquad
CR = 0.0947
\end{equation*}

\begin{equation*}
\mathbf{w}^{cos} =
\begin{pmatrix}
0.508759\\
0.256842\\
0.118349\\
\color{red} 0.116050\color{black}
\end{pmatrix}\end{equation*}
\begin{equation*}
\left[ \frac{{w}^{cos}_i}{{w}^{cos}_j} \right] =
\begin{pmatrix}
$\,\,$ 1 $\,\,$ & $\,\,$1.9808$\,\,$ & $\,\,$4.2988$\,\,$ & $\,\,$\color{red} 4.3840\color{black} $\,\,$ \\
$\,\,$0.5048$\,\,$ & $\,\,$ 1 $\,\,$ & $\,\,$2.1702$\,\,$ & $\,\,$\color{red} 2.2132\color{black}   $\,\,$ \\
$\,\,$0.2326$\,\,$ & $\,\,$0.4608$\,\,$ & $\,\,$ 1 $\,\,$ & $\,\,$\color{red} 1.0198\color{black}  $\,\,$ \\
$\,\,$\color{red} 0.2281\color{black} $\,\,$ & $\,\,$\color{red} 0.4518\color{black} $\,\,$ & $\,\,$\color{red} 0.9806\color{black} $\,\,$ & $\,\,$ 1  $\,\,$ \\
\end{pmatrix},
\end{equation*}

\begin{equation*}
\mathbf{w}^{\prime} =
\begin{pmatrix}
0.507592\\
0.256253\\
0.118078\\
0.118078
\end{pmatrix} =
0.997707\cdot
\begin{pmatrix}
0.508759\\
0.256842\\
0.118349\\
\color{gr} 0.118349\color{black}
\end{pmatrix},
\end{equation*}
\begin{equation*}
\left[ \frac{{w}^{\prime}_i}{{w}^{\prime}_j} \right] =
\begin{pmatrix}
$\,\,$ 1 $\,\,$ & $\,\,$1.9808$\,\,$ & $\,\,$4.2988$\,\,$ & $\,\,$\color{gr} 4.2988\color{black} $\,\,$ \\
$\,\,$0.5048$\,\,$ & $\,\,$ 1 $\,\,$ & $\,\,$2.1702$\,\,$ & $\,\,$\color{gr} 2.1702\color{black}   $\,\,$ \\
$\,\,$0.2326$\,\,$ & $\,\,$0.4608$\,\,$ & $\,\,$ 1 $\,\,$ & $\,\,$\color{gr} \color{blue} 1\color{black}  $\,\,$ \\
$\,\,$\color{gr} 0.2326\color{black} $\,\,$ & $\,\,$\color{gr} 0.4608\color{black} $\,\,$ & $\,\,$\color{gr} \color{blue} 1\color{black} $\,\,$ & $\,\,$ 1  $\,\,$ \\
\end{pmatrix},
\end{equation*}
\end{example}
\newpage
\begin{example}
\begin{equation*}
\mathbf{A} =
\begin{pmatrix}
$\,\,$ 1 $\,\,$ & $\,\,$4$\,\,$ & $\,\,$3$\,\,$ & $\,\,$7 $\,\,$ \\
$\,\,$ 1/4$\,\,$ & $\,\,$ 1 $\,\,$ & $\,\,$1$\,\,$ & $\,\,$4 $\,\,$ \\
$\,\,$ 1/3$\,\,$ & $\,\,$ 1 $\,\,$ & $\,\,$ 1 $\,\,$ & $\,\,$3 $\,\,$ \\
$\,\,$ 1/7$\,\,$ & $\,\,$ 1/4$\,\,$ & $\,\,$ 1/3$\,\,$ & $\,\,$ 1  $\,\,$ \\
\end{pmatrix},
\qquad
\lambda_{\max} =
4.0576,
\qquad
CR = 0.0217
\end{equation*}

\begin{equation*}
\mathbf{w}^{cos} =
\begin{pmatrix}
0.558511\\
0.192086\\
\color{red} 0.185943\color{black} \\
0.063461
\end{pmatrix}\end{equation*}
\begin{equation*}
\left[ \frac{{w}^{cos}_i}{{w}^{cos}_j} \right] =
\begin{pmatrix}
$\,\,$ 1 $\,\,$ & $\,\,$2.9076$\,\,$ & $\,\,$\color{red} 3.0037\color{black} $\,\,$ & $\,\,$8.8009$\,\,$ \\
$\,\,$0.3439$\,\,$ & $\,\,$ 1 $\,\,$ & $\,\,$\color{red} 1.0330\color{black} $\,\,$ & $\,\,$3.0268  $\,\,$ \\
$\,\,$\color{red} 0.3329\color{black} $\,\,$ & $\,\,$\color{red} 0.9680\color{black} $\,\,$ & $\,\,$ 1 $\,\,$ & $\,\,$\color{red} 2.9300\color{black}  $\,\,$ \\
$\,\,$0.1136$\,\,$ & $\,\,$0.3304$\,\,$ & $\,\,$\color{red} 0.3413\color{black} $\,\,$ & $\,\,$ 1  $\,\,$ \\
\end{pmatrix},
\end{equation*}

\begin{equation*}
\mathbf{w}^{\prime} =
\begin{pmatrix}
0.558384\\
0.192042\\
0.186128\\
0.063446
\end{pmatrix} =
0.999773\cdot
\begin{pmatrix}
0.558511\\
0.192086\\
\color{gr} 0.186170\color{black} \\
0.063461
\end{pmatrix},
\end{equation*}
\begin{equation*}
\left[ \frac{{w}^{\prime}_i}{{w}^{\prime}_j} \right] =
\begin{pmatrix}
$\,\,$ 1 $\,\,$ & $\,\,$2.9076$\,\,$ & $\,\,$\color{gr} \color{blue} 3\color{black} $\,\,$ & $\,\,$8.8009$\,\,$ \\
$\,\,$0.3439$\,\,$ & $\,\,$ 1 $\,\,$ & $\,\,$\color{gr} 1.0318\color{black} $\,\,$ & $\,\,$3.0268  $\,\,$ \\
$\,\,$\color{gr} \color{blue}  1/3\color{black} $\,\,$ & $\,\,$\color{gr} 0.9692\color{black} $\,\,$ & $\,\,$ 1 $\,\,$ & $\,\,$\color{gr} 2.9336\color{black}  $\,\,$ \\
$\,\,$0.1136$\,\,$ & $\,\,$0.3304$\,\,$ & $\,\,$\color{gr} 0.3409\color{black} $\,\,$ & $\,\,$ 1  $\,\,$ \\
\end{pmatrix},
\end{equation*}
\end{example}
\newpage
\begin{example}
\begin{equation*}
\mathbf{A} =
\begin{pmatrix}
$\,\,$ 1 $\,\,$ & $\,\,$4$\,\,$ & $\,\,$3$\,\,$ & $\,\,$8 $\,\,$ \\
$\,\,$ 1/4$\,\,$ & $\,\,$ 1 $\,\,$ & $\,\,$2$\,\,$ & $\,\,$3 $\,\,$ \\
$\,\,$ 1/3$\,\,$ & $\,\,$ 1/2$\,\,$ & $\,\,$ 1 $\,\,$ & $\,\,$2 $\,\,$ \\
$\,\,$ 1/8$\,\,$ & $\,\,$ 1/3$\,\,$ & $\,\,$ 1/2$\,\,$ & $\,\,$ 1  $\,\,$ \\
\end{pmatrix},
\qquad
\lambda_{\max} =
4.0820,
\qquad
CR = 0.0309
\end{equation*}

\begin{equation*}
\mathbf{w}^{cos} =
\begin{pmatrix}
0.570251\\
0.213393\\
0.146172\\
\color{red} 0.070184\color{black}
\end{pmatrix}\end{equation*}
\begin{equation*}
\left[ \frac{{w}^{cos}_i}{{w}^{cos}_j} \right] =
\begin{pmatrix}
$\,\,$ 1 $\,\,$ & $\,\,$2.6723$\,\,$ & $\,\,$3.9012$\,\,$ & $\,\,$\color{red} 8.1251\color{black} $\,\,$ \\
$\,\,$0.3742$\,\,$ & $\,\,$ 1 $\,\,$ & $\,\,$1.4599$\,\,$ & $\,\,$\color{red} 3.0405\color{black}   $\,\,$ \\
$\,\,$0.2563$\,\,$ & $\,\,$0.6850$\,\,$ & $\,\,$ 1 $\,\,$ & $\,\,$\color{red} 2.0827\color{black}  $\,\,$ \\
$\,\,$\color{red} 0.1231\color{black} $\,\,$ & $\,\,$\color{red} 0.3289\color{black} $\,\,$ & $\,\,$\color{red} 0.4801\color{black} $\,\,$ & $\,\,$ 1  $\,\,$ \\
\end{pmatrix},
\end{equation*}

\begin{equation*}
\mathbf{w}^{\prime} =
\begin{pmatrix}
0.569712\\
0.213191\\
0.146034\\
0.071064
\end{pmatrix} =
0.999054\cdot
\begin{pmatrix}
0.570251\\
0.213393\\
0.146172\\
\color{gr} 0.071131\color{black}
\end{pmatrix},
\end{equation*}
\begin{equation*}
\left[ \frac{{w}^{\prime}_i}{{w}^{\prime}_j} \right] =
\begin{pmatrix}
$\,\,$ 1 $\,\,$ & $\,\,$2.6723$\,\,$ & $\,\,$3.9012$\,\,$ & $\,\,$\color{gr} 8.0169\color{black} $\,\,$ \\
$\,\,$0.3742$\,\,$ & $\,\,$ 1 $\,\,$ & $\,\,$1.4599$\,\,$ & $\,\,$\color{gr} \color{blue} 3\color{black}   $\,\,$ \\
$\,\,$0.2563$\,\,$ & $\,\,$0.6850$\,\,$ & $\,\,$ 1 $\,\,$ & $\,\,$\color{gr} 2.0550\color{black}  $\,\,$ \\
$\,\,$\color{gr} 0.1247\color{black} $\,\,$ & $\,\,$\color{gr} \color{blue}  1/3\color{black} $\,\,$ & $\,\,$\color{gr} 0.4866\color{black} $\,\,$ & $\,\,$ 1  $\,\,$ \\
\end{pmatrix},
\end{equation*}
\end{example}
\newpage
\begin{example}
\begin{equation*}
\mathbf{A} =
\begin{pmatrix}
$\,\,$ 1 $\,\,$ & $\,\,$4$\,\,$ & $\,\,$3$\,\,$ & $\,\,$8 $\,\,$ \\
$\,\,$ 1/4$\,\,$ & $\,\,$ 1 $\,\,$ & $\,\,$4$\,\,$ & $\,\,$4 $\,\,$ \\
$\,\,$ 1/3$\,\,$ & $\,\,$ 1/4$\,\,$ & $\,\,$ 1 $\,\,$ & $\,\,$2 $\,\,$ \\
$\,\,$ 1/8$\,\,$ & $\,\,$ 1/4$\,\,$ & $\,\,$ 1/2$\,\,$ & $\,\,$ 1  $\,\,$ \\
\end{pmatrix},
\qquad
\lambda_{\max} =
4.2512,
\qquad
CR = 0.0947
\end{equation*}

\begin{equation*}
\mathbf{w}^{cos} =
\begin{pmatrix}
0.539803\\
0.272660\\
0.125887\\
\color{red} 0.061649\color{black}
\end{pmatrix}\end{equation*}
\begin{equation*}
\left[ \frac{{w}^{cos}_i}{{w}^{cos}_j} \right] =
\begin{pmatrix}
$\,\,$ 1 $\,\,$ & $\,\,$1.9798$\,\,$ & $\,\,$4.2880$\,\,$ & $\,\,$\color{red} 8.7560\color{black} $\,\,$ \\
$\,\,$0.5051$\,\,$ & $\,\,$ 1 $\,\,$ & $\,\,$2.1659$\,\,$ & $\,\,$\color{red} 4.4228\color{black}   $\,\,$ \\
$\,\,$0.2332$\,\,$ & $\,\,$0.4617$\,\,$ & $\,\,$ 1 $\,\,$ & $\,\,$\color{red} 2.0420\color{black}  $\,\,$ \\
$\,\,$\color{red} 0.1142\color{black} $\,\,$ & $\,\,$\color{red} 0.2261\color{black} $\,\,$ & $\,\,$\color{red} 0.4897\color{black} $\,\,$ & $\,\,$ 1  $\,\,$ \\
\end{pmatrix},
\end{equation*}

\begin{equation*}
\mathbf{w}^{\prime} =
\begin{pmatrix}
0.539105\\
0.272308\\
0.125725\\
0.062862
\end{pmatrix} =
0.998707\cdot
\begin{pmatrix}
0.539803\\
0.272660\\
0.125887\\
\color{gr} 0.062944\color{black}
\end{pmatrix},
\end{equation*}
\begin{equation*}
\left[ \frac{{w}^{\prime}_i}{{w}^{\prime}_j} \right] =
\begin{pmatrix}
$\,\,$ 1 $\,\,$ & $\,\,$1.9798$\,\,$ & $\,\,$4.2880$\,\,$ & $\,\,$\color{gr} 8.5760\color{black} $\,\,$ \\
$\,\,$0.5051$\,\,$ & $\,\,$ 1 $\,\,$ & $\,\,$2.1659$\,\,$ & $\,\,$\color{gr} 4.3318\color{black}   $\,\,$ \\
$\,\,$0.2332$\,\,$ & $\,\,$0.4617$\,\,$ & $\,\,$ 1 $\,\,$ & $\,\,$\color{gr} \color{blue} 2\color{black}  $\,\,$ \\
$\,\,$\color{gr} 0.1166\color{black} $\,\,$ & $\,\,$\color{gr} 0.2309\color{black} $\,\,$ & $\,\,$\color{gr} \color{blue}  1/2\color{black} $\,\,$ & $\,\,$ 1  $\,\,$ \\
\end{pmatrix},
\end{equation*}
\end{example}
\newpage
\begin{example}
\begin{equation*}
\mathbf{A} =
\begin{pmatrix}
$\,\,$ 1 $\,\,$ & $\,\,$4$\,\,$ & $\,\,$3$\,\,$ & $\,\,$9 $\,\,$ \\
$\,\,$ 1/4$\,\,$ & $\,\,$ 1 $\,\,$ & $\,\,$4$\,\,$ & $\,\,$4 $\,\,$ \\
$\,\,$ 1/3$\,\,$ & $\,\,$ 1/4$\,\,$ & $\,\,$ 1 $\,\,$ & $\,\,$2 $\,\,$ \\
$\,\,$ 1/9$\,\,$ & $\,\,$ 1/4$\,\,$ & $\,\,$ 1/2$\,\,$ & $\,\,$ 1  $\,\,$ \\
\end{pmatrix},
\qquad
\lambda_{\max} =
4.2469,
\qquad
CR = 0.0931
\end{equation*}

\begin{equation*}
\mathbf{w}^{cos} =
\begin{pmatrix}
0.548528\\
0.268916\\
0.123968\\
\color{red} 0.058588\color{black}
\end{pmatrix}\end{equation*}
\begin{equation*}
\left[ \frac{{w}^{cos}_i}{{w}^{cos}_j} \right] =
\begin{pmatrix}
$\,\,$ 1 $\,\,$ & $\,\,$2.0398$\,\,$ & $\,\,$4.4248$\,\,$ & $\,\,$\color{red} 9.3624\color{black} $\,\,$ \\
$\,\,$0.4903$\,\,$ & $\,\,$ 1 $\,\,$ & $\,\,$2.1692$\,\,$ & $\,\,$\color{red} 4.5899\color{black}   $\,\,$ \\
$\,\,$0.2260$\,\,$ & $\,\,$0.4610$\,\,$ & $\,\,$ 1 $\,\,$ & $\,\,$\color{red} 2.1159\color{black}  $\,\,$ \\
$\,\,$\color{red} 0.1068\color{black} $\,\,$ & $\,\,$\color{red} 0.2179\color{black} $\,\,$ & $\,\,$\color{red} 0.4726\color{black} $\,\,$ & $\,\,$ 1  $\,\,$ \\
\end{pmatrix},
\end{equation*}

\begin{equation*}
\mathbf{w}^{\prime} =
\begin{pmatrix}
0.547237\\
0.268283\\
0.123676\\
0.060804
\end{pmatrix} =
0.997646\cdot
\begin{pmatrix}
0.548528\\
0.268916\\
0.123968\\
\color{gr} 0.060948\color{black}
\end{pmatrix},
\end{equation*}
\begin{equation*}
\left[ \frac{{w}^{\prime}_i}{{w}^{\prime}_j} \right] =
\begin{pmatrix}
$\,\,$ 1 $\,\,$ & $\,\,$2.0398$\,\,$ & $\,\,$4.4248$\,\,$ & $\,\,$\color{gr} \color{blue} 9\color{black} $\,\,$ \\
$\,\,$0.4903$\,\,$ & $\,\,$ 1 $\,\,$ & $\,\,$2.1692$\,\,$ & $\,\,$\color{gr} 4.4123\color{black}   $\,\,$ \\
$\,\,$0.2260$\,\,$ & $\,\,$0.4610$\,\,$ & $\,\,$ 1 $\,\,$ & $\,\,$\color{gr} 2.0340\color{black}  $\,\,$ \\
$\,\,$\color{gr} \color{blue}  1/9\color{black} $\,\,$ & $\,\,$\color{gr} 0.2266\color{black} $\,\,$ & $\,\,$\color{gr} 0.4916\color{black} $\,\,$ & $\,\,$ 1  $\,\,$ \\
\end{pmatrix},
\end{equation*}
\end{example}
\newpage
\begin{example}
\begin{equation*}
\mathbf{A} =
\begin{pmatrix}
$\,\,$ 1 $\,\,$ & $\,\,$4$\,\,$ & $\,\,$3$\,\,$ & $\,\,$9 $\,\,$ \\
$\,\,$ 1/4$\,\,$ & $\,\,$ 1 $\,\,$ & $\,\,$4$\,\,$ & $\,\,$5 $\,\,$ \\
$\,\,$ 1/3$\,\,$ & $\,\,$ 1/4$\,\,$ & $\,\,$ 1 $\,\,$ & $\,\,$2 $\,\,$ \\
$\,\,$ 1/9$\,\,$ & $\,\,$ 1/5$\,\,$ & $\,\,$ 1/2$\,\,$ & $\,\,$ 1  $\,\,$ \\
\end{pmatrix},
\qquad
\lambda_{\max} =
4.2500,
\qquad
CR = 0.0942
\end{equation*}

\begin{equation*}
\mathbf{w}^{cos} =
\begin{pmatrix}
0.541012\\
0.280924\\
0.122275\\
\color{red} 0.055788\color{black}
\end{pmatrix}\end{equation*}
\begin{equation*}
\left[ \frac{{w}^{cos}_i}{{w}^{cos}_j} \right] =
\begin{pmatrix}
$\,\,$ 1 $\,\,$ & $\,\,$1.9258$\,\,$ & $\,\,$4.4246$\,\,$ & $\,\,$\color{red} 9.6976\color{black} $\,\,$ \\
$\,\,$0.5193$\,\,$ & $\,\,$ 1 $\,\,$ & $\,\,$2.2975$\,\,$ & $\,\,$\color{red} 5.0356\color{black}   $\,\,$ \\
$\,\,$0.2260$\,\,$ & $\,\,$0.4353$\,\,$ & $\,\,$ 1 $\,\,$ & $\,\,$\color{red} 2.1918\color{black}  $\,\,$ \\
$\,\,$\color{red} 0.1031\color{black} $\,\,$ & $\,\,$\color{red} 0.1986\color{black} $\,\,$ & $\,\,$\color{red} 0.4563\color{black} $\,\,$ & $\,\,$ 1  $\,\,$ \\
\end{pmatrix},
\end{equation*}

\begin{equation*}
\mathbf{w}^{\prime} =
\begin{pmatrix}
0.540798\\
0.280813\\
0.122227\\
0.056163
\end{pmatrix} =
0.999603\cdot
\begin{pmatrix}
0.541012\\
0.280924\\
0.122275\\
\color{gr} 0.056185\color{black}
\end{pmatrix},
\end{equation*}
\begin{equation*}
\left[ \frac{{w}^{\prime}_i}{{w}^{\prime}_j} \right] =
\begin{pmatrix}
$\,\,$ 1 $\,\,$ & $\,\,$1.9258$\,\,$ & $\,\,$4.4246$\,\,$ & $\,\,$\color{gr} 9.6291\color{black} $\,\,$ \\
$\,\,$0.5193$\,\,$ & $\,\,$ 1 $\,\,$ & $\,\,$2.2975$\,\,$ & $\,\,$\color{gr} \color{blue} 5\color{black}   $\,\,$ \\
$\,\,$0.2260$\,\,$ & $\,\,$0.4353$\,\,$ & $\,\,$ 1 $\,\,$ & $\,\,$\color{gr} 2.1763\color{black}  $\,\,$ \\
$\,\,$\color{gr} 0.1039\color{black} $\,\,$ & $\,\,$\color{gr} \color{blue}  1/5\color{black} $\,\,$ & $\,\,$\color{gr} 0.4595\color{black} $\,\,$ & $\,\,$ 1  $\,\,$ \\
\end{pmatrix},
\end{equation*}
\end{example}
\newpage
\begin{example}
\begin{equation*}
\mathbf{A} =
\begin{pmatrix}
$\,\,$ 1 $\,\,$ & $\,\,$4$\,\,$ & $\,\,$4$\,\,$ & $\,\,$5 $\,\,$ \\
$\,\,$ 1/4$\,\,$ & $\,\,$ 1 $\,\,$ & $\,\,$2$\,\,$ & $\,\,$6 $\,\,$ \\
$\,\,$ 1/4$\,\,$ & $\,\,$ 1/2$\,\,$ & $\,\,$ 1 $\,\,$ & $\,\,$2 $\,\,$ \\
$\,\,$ 1/5$\,\,$ & $\,\,$ 1/6$\,\,$ & $\,\,$ 1/2$\,\,$ & $\,\,$ 1  $\,\,$ \\
\end{pmatrix},
\qquad
\lambda_{\max} =
4.2162,
\qquad
CR = 0.0815
\end{equation*}

\begin{equation*}
\mathbf{w}^{cos} =
\begin{pmatrix}
0.536208\\
0.261524\\
\color{red} 0.129604\color{black} \\
0.072664
\end{pmatrix}\end{equation*}
\begin{equation*}
\left[ \frac{{w}^{cos}_i}{{w}^{cos}_j} \right] =
\begin{pmatrix}
$\,\,$ 1 $\,\,$ & $\,\,$2.0503$\,\,$ & $\,\,$\color{red} 4.1373\color{black} $\,\,$ & $\,\,$7.3793$\,\,$ \\
$\,\,$0.4877$\,\,$ & $\,\,$ 1 $\,\,$ & $\,\,$\color{red} 2.0179\color{black} $\,\,$ & $\,\,$3.5991  $\,\,$ \\
$\,\,$\color{red} 0.2417\color{black} $\,\,$ & $\,\,$\color{red} 0.4956\color{black} $\,\,$ & $\,\,$ 1 $\,\,$ & $\,\,$\color{red} 1.7836\color{black}  $\,\,$ \\
$\,\,$0.1355$\,\,$ & $\,\,$0.2778$\,\,$ & $\,\,$\color{red} 0.5607\color{black} $\,\,$ & $\,\,$ 1  $\,\,$ \\
\end{pmatrix},
\end{equation*}

\begin{equation*}
\mathbf{w}^{\prime} =
\begin{pmatrix}
0.535588\\
0.261222\\
0.130611\\
0.072580
\end{pmatrix} =
0.998843\cdot
\begin{pmatrix}
0.536208\\
0.261524\\
\color{gr} 0.130762\color{black} \\
0.072664
\end{pmatrix},
\end{equation*}
\begin{equation*}
\left[ \frac{{w}^{\prime}_i}{{w}^{\prime}_j} \right] =
\begin{pmatrix}
$\,\,$ 1 $\,\,$ & $\,\,$2.0503$\,\,$ & $\,\,$\color{gr} 4.1006\color{black} $\,\,$ & $\,\,$7.3793$\,\,$ \\
$\,\,$0.4877$\,\,$ & $\,\,$ 1 $\,\,$ & $\,\,$\color{gr} \color{blue} 2\color{black} $\,\,$ & $\,\,$3.5991  $\,\,$ \\
$\,\,$\color{gr} 0.2439\color{black} $\,\,$ & $\,\,$\color{gr} \color{blue}  1/2\color{black} $\,\,$ & $\,\,$ 1 $\,\,$ & $\,\,$\color{gr} 1.7996\color{black}  $\,\,$ \\
$\,\,$0.1355$\,\,$ & $\,\,$0.2778$\,\,$ & $\,\,$\color{gr} 0.5557\color{black} $\,\,$ & $\,\,$ 1  $\,\,$ \\
\end{pmatrix},
\end{equation*}
\end{example}
\newpage
\begin{example}
\begin{equation*}
\mathbf{A} =
\begin{pmatrix}
$\,\,$ 1 $\,\,$ & $\,\,$4$\,\,$ & $\,\,$4$\,\,$ & $\,\,$5 $\,\,$ \\
$\,\,$ 1/4$\,\,$ & $\,\,$ 1 $\,\,$ & $\,\,$2$\,\,$ & $\,\,$7 $\,\,$ \\
$\,\,$ 1/4$\,\,$ & $\,\,$ 1/2$\,\,$ & $\,\,$ 1 $\,\,$ & $\,\,$2 $\,\,$ \\
$\,\,$ 1/5$\,\,$ & $\,\,$ 1/7$\,\,$ & $\,\,$ 1/2$\,\,$ & $\,\,$ 1  $\,\,$ \\
\end{pmatrix},
\qquad
\lambda_{\max} =
4.2610,
\qquad
CR = 0.0984
\end{equation*}

\begin{equation*}
\mathbf{w}^{cos} =
\begin{pmatrix}
0.531298\\
0.271097\\
\color{red} 0.127073\color{black} \\
0.070532
\end{pmatrix}\end{equation*}
\begin{equation*}
\left[ \frac{{w}^{cos}_i}{{w}^{cos}_j} \right] =
\begin{pmatrix}
$\,\,$ 1 $\,\,$ & $\,\,$1.9598$\,\,$ & $\,\,$\color{red} 4.1810\color{black} $\,\,$ & $\,\,$7.5327$\,\,$ \\
$\,\,$0.5103$\,\,$ & $\,\,$ 1 $\,\,$ & $\,\,$\color{red} 2.1334\color{black} $\,\,$ & $\,\,$3.8436  $\,\,$ \\
$\,\,$\color{red} 0.2392\color{black} $\,\,$ & $\,\,$\color{red} 0.4687\color{black} $\,\,$ & $\,\,$ 1 $\,\,$ & $\,\,$\color{red} 1.8016\color{black}  $\,\,$ \\
$\,\,$0.1328$\,\,$ & $\,\,$0.2602$\,\,$ & $\,\,$\color{red} 0.5551\color{black} $\,\,$ & $\,\,$ 1  $\,\,$ \\
\end{pmatrix},
\end{equation*}

\begin{equation*}
\mathbf{w}^{\prime} =
\begin{pmatrix}
0.528259\\
0.269547\\
0.132065\\
0.070129
\end{pmatrix} =
0.994281\cdot
\begin{pmatrix}
0.531298\\
0.271097\\
\color{gr} 0.132824\color{black} \\
0.070532
\end{pmatrix},
\end{equation*}
\begin{equation*}
\left[ \frac{{w}^{\prime}_i}{{w}^{\prime}_j} \right] =
\begin{pmatrix}
$\,\,$ 1 $\,\,$ & $\,\,$1.9598$\,\,$ & $\,\,$\color{gr} \color{blue} 4\color{black} $\,\,$ & $\,\,$7.5327$\,\,$ \\
$\,\,$0.5103$\,\,$ & $\,\,$ 1 $\,\,$ & $\,\,$\color{gr} 2.0410\color{black} $\,\,$ & $\,\,$3.8436  $\,\,$ \\
$\,\,$\color{gr} \color{blue}  1/4\color{black} $\,\,$ & $\,\,$\color{gr} 0.4900\color{black} $\,\,$ & $\,\,$ 1 $\,\,$ & $\,\,$\color{gr} 1.8832\color{black}  $\,\,$ \\
$\,\,$0.1328$\,\,$ & $\,\,$0.2602$\,\,$ & $\,\,$\color{gr} 0.5310\color{black} $\,\,$ & $\,\,$ 1  $\,\,$ \\
\end{pmatrix},
\end{equation*}
\end{example}
\newpage
\begin{example}
\begin{equation*}
\mathbf{A} =
\begin{pmatrix}
$\,\,$ 1 $\,\,$ & $\,\,$4$\,\,$ & $\,\,$4$\,\,$ & $\,\,$5 $\,\,$ \\
$\,\,$ 1/4$\,\,$ & $\,\,$ 1 $\,\,$ & $\,\,$3$\,\,$ & $\,\,$2 $\,\,$ \\
$\,\,$ 1/4$\,\,$ & $\,\,$ 1/3$\,\,$ & $\,\,$ 1 $\,\,$ & $\,\,$1 $\,\,$ \\
$\,\,$ 1/5$\,\,$ & $\,\,$ 1/2$\,\,$ & $\,\,$ 1 $\,\,$ & $\,\,$ 1  $\,\,$ \\
\end{pmatrix},
\qquad
\lambda_{\max} =
4.1046,
\qquad
CR = 0.0395
\end{equation*}

\begin{equation*}
\mathbf{w}^{cos} =
\begin{pmatrix}
0.562078\\
0.222690\\
0.108116\\
\color{red} 0.107116\color{black}
\end{pmatrix}\end{equation*}
\begin{equation*}
\left[ \frac{{w}^{cos}_i}{{w}^{cos}_j} \right] =
\begin{pmatrix}
$\,\,$ 1 $\,\,$ & $\,\,$2.5240$\,\,$ & $\,\,$5.1988$\,\,$ & $\,\,$\color{red} 5.2474\color{black} $\,\,$ \\
$\,\,$0.3962$\,\,$ & $\,\,$ 1 $\,\,$ & $\,\,$2.0597$\,\,$ & $\,\,$\color{red} 2.0790\color{black}   $\,\,$ \\
$\,\,$0.1924$\,\,$ & $\,\,$0.4855$\,\,$ & $\,\,$ 1 $\,\,$ & $\,\,$\color{red} 1.0093\color{black}  $\,\,$ \\
$\,\,$\color{red} 0.1906\color{black} $\,\,$ & $\,\,$\color{red} 0.4810\color{black} $\,\,$ & $\,\,$\color{red} 0.9907\color{black} $\,\,$ & $\,\,$ 1  $\,\,$ \\
\end{pmatrix},
\end{equation*}

\begin{equation*}
\mathbf{w}^{\prime} =
\begin{pmatrix}
0.561516\\
0.222467\\
0.108008\\
0.108008
\end{pmatrix} =
0.999001\cdot
\begin{pmatrix}
0.562078\\
0.222690\\
0.108116\\
\color{gr} 0.108116\color{black}
\end{pmatrix},
\end{equation*}
\begin{equation*}
\left[ \frac{{w}^{\prime}_i}{{w}^{\prime}_j} \right] =
\begin{pmatrix}
$\,\,$ 1 $\,\,$ & $\,\,$2.5240$\,\,$ & $\,\,$5.1988$\,\,$ & $\,\,$\color{gr} 5.1988\color{black} $\,\,$ \\
$\,\,$0.3962$\,\,$ & $\,\,$ 1 $\,\,$ & $\,\,$2.0597$\,\,$ & $\,\,$\color{gr} 2.0597\color{black}   $\,\,$ \\
$\,\,$0.1924$\,\,$ & $\,\,$0.4855$\,\,$ & $\,\,$ 1 $\,\,$ & $\,\,$\color{gr} \color{blue} 1\color{black}  $\,\,$ \\
$\,\,$\color{gr} 0.1924\color{black} $\,\,$ & $\,\,$\color{gr} 0.4855\color{black} $\,\,$ & $\,\,$\color{gr} \color{blue} 1\color{black} $\,\,$ & $\,\,$ 1  $\,\,$ \\
\end{pmatrix},
\end{equation*}
\end{example}
\newpage
\begin{example}
\begin{equation*}
\mathbf{A} =
\begin{pmatrix}
$\,\,$ 1 $\,\,$ & $\,\,$4$\,\,$ & $\,\,$4$\,\,$ & $\,\,$6 $\,\,$ \\
$\,\,$ 1/4$\,\,$ & $\,\,$ 1 $\,\,$ & $\,\,$2$\,\,$ & $\,\,$7 $\,\,$ \\
$\,\,$ 1/4$\,\,$ & $\,\,$ 1/2$\,\,$ & $\,\,$ 1 $\,\,$ & $\,\,$2 $\,\,$ \\
$\,\,$ 1/6$\,\,$ & $\,\,$ 1/7$\,\,$ & $\,\,$ 1/2$\,\,$ & $\,\,$ 1  $\,\,$ \\
\end{pmatrix},
\qquad
\lambda_{\max} =
4.2109,
\qquad
CR = 0.0795
\end{equation*}

\begin{equation*}
\mathbf{w}^{cos} =
\begin{pmatrix}
0.545350\\
0.264346\\
\color{red} 0.125474\color{black} \\
0.064830
\end{pmatrix}\end{equation*}
\begin{equation*}
\left[ \frac{{w}^{cos}_i}{{w}^{cos}_j} \right] =
\begin{pmatrix}
$\,\,$ 1 $\,\,$ & $\,\,$2.0630$\,\,$ & $\,\,$\color{red} 4.3463\color{black} $\,\,$ & $\,\,$8.4120$\,\,$ \\
$\,\,$0.4847$\,\,$ & $\,\,$ 1 $\,\,$ & $\,\,$\color{red} 2.1068\color{black} $\,\,$ & $\,\,$4.0775  $\,\,$ \\
$\,\,$\color{red} 0.2301\color{black} $\,\,$ & $\,\,$\color{red} 0.4747\color{black} $\,\,$ & $\,\,$ 1 $\,\,$ & $\,\,$\color{red} 1.9354\color{black}  $\,\,$ \\
$\,\,$0.1189$\,\,$ & $\,\,$0.2452$\,\,$ & $\,\,$\color{red} 0.5167\color{black} $\,\,$ & $\,\,$ 1  $\,\,$ \\
\end{pmatrix},
\end{equation*}

\begin{equation*}
\mathbf{w}^{\prime} =
\begin{pmatrix}
0.543077\\
0.263244\\
0.129120\\
0.064560
\end{pmatrix} =
0.995832\cdot
\begin{pmatrix}
0.545350\\
0.264346\\
\color{gr} 0.129660\color{black} \\
0.064830
\end{pmatrix},
\end{equation*}
\begin{equation*}
\left[ \frac{{w}^{\prime}_i}{{w}^{\prime}_j} \right] =
\begin{pmatrix}
$\,\,$ 1 $\,\,$ & $\,\,$2.0630$\,\,$ & $\,\,$\color{gr} 4.2060\color{black} $\,\,$ & $\,\,$8.4120$\,\,$ \\
$\,\,$0.4847$\,\,$ & $\,\,$ 1 $\,\,$ & $\,\,$\color{gr} 2.0388\color{black} $\,\,$ & $\,\,$4.0775  $\,\,$ \\
$\,\,$\color{gr} 0.2378\color{black} $\,\,$ & $\,\,$\color{gr} 0.4905\color{black} $\,\,$ & $\,\,$ 1 $\,\,$ & $\,\,$\color{gr} \color{blue} 2\color{black}  $\,\,$ \\
$\,\,$0.1189$\,\,$ & $\,\,$0.2452$\,\,$ & $\,\,$\color{gr} \color{blue}  1/2\color{black} $\,\,$ & $\,\,$ 1  $\,\,$ \\
\end{pmatrix},
\end{equation*}
\end{example}
\newpage
\begin{example}
\begin{equation*}
\mathbf{A} =
\begin{pmatrix}
$\,\,$ 1 $\,\,$ & $\,\,$4$\,\,$ & $\,\,$4$\,\,$ & $\,\,$6 $\,\,$ \\
$\,\,$ 1/4$\,\,$ & $\,\,$ 1 $\,\,$ & $\,\,$2$\,\,$ & $\,\,$8 $\,\,$ \\
$\,\,$ 1/4$\,\,$ & $\,\,$ 1/2$\,\,$ & $\,\,$ 1 $\,\,$ & $\,\,$2 $\,\,$ \\
$\,\,$ 1/6$\,\,$ & $\,\,$ 1/8$\,\,$ & $\,\,$ 1/2$\,\,$ & $\,\,$ 1  $\,\,$ \\
\end{pmatrix},
\qquad
\lambda_{\max} =
4.2512,
\qquad
CR = 0.0947
\end{equation*}

\begin{equation*}
\mathbf{w}^{cos} =
\begin{pmatrix}
0.540580\\
0.272623\\
\color{red} 0.123592\color{black} \\
0.063205
\end{pmatrix}\end{equation*}
\begin{equation*}
\left[ \frac{{w}^{cos}_i}{{w}^{cos}_j} \right] =
\begin{pmatrix}
$\,\,$ 1 $\,\,$ & $\,\,$1.9829$\,\,$ & $\,\,$\color{red} 4.3739\color{black} $\,\,$ & $\,\,$8.5528$\,\,$ \\
$\,\,$0.5043$\,\,$ & $\,\,$ 1 $\,\,$ & $\,\,$\color{red} 2.2058\color{black} $\,\,$ & $\,\,$4.3133  $\,\,$ \\
$\,\,$\color{red} 0.2286\color{black} $\,\,$ & $\,\,$\color{red} 0.4533\color{black} $\,\,$ & $\,\,$ 1 $\,\,$ & $\,\,$\color{red} 1.9554\color{black}  $\,\,$ \\
$\,\,$0.1169$\,\,$ & $\,\,$0.2318$\,\,$ & $\,\,$\color{red} 0.5114\color{black} $\,\,$ & $\,\,$ 1  $\,\,$ \\
\end{pmatrix},
\end{equation*}

\begin{equation*}
\mathbf{w}^{\prime} =
\begin{pmatrix}
0.539061\\
0.271856\\
0.126055\\
0.063027
\end{pmatrix} =
0.997190\cdot
\begin{pmatrix}
0.540580\\
0.272623\\
\color{gr} 0.126410\color{black} \\
0.063205
\end{pmatrix},
\end{equation*}
\begin{equation*}
\left[ \frac{{w}^{\prime}_i}{{w}^{\prime}_j} \right] =
\begin{pmatrix}
$\,\,$ 1 $\,\,$ & $\,\,$1.9829$\,\,$ & $\,\,$\color{gr} 4.2764\color{black} $\,\,$ & $\,\,$8.5528$\,\,$ \\
$\,\,$0.5043$\,\,$ & $\,\,$ 1 $\,\,$ & $\,\,$\color{gr} 2.1567\color{black} $\,\,$ & $\,\,$4.3133  $\,\,$ \\
$\,\,$\color{gr} 0.2338\color{black} $\,\,$ & $\,\,$\color{gr} 0.4637\color{black} $\,\,$ & $\,\,$ 1 $\,\,$ & $\,\,$\color{gr} \color{blue} 2\color{black}  $\,\,$ \\
$\,\,$0.1169$\,\,$ & $\,\,$0.2318$\,\,$ & $\,\,$\color{gr} \color{blue}  1/2\color{black} $\,\,$ & $\,\,$ 1  $\,\,$ \\
\end{pmatrix},
\end{equation*}
\end{example}
\newpage
\begin{example}
\begin{equation*}
\mathbf{A} =
\begin{pmatrix}
$\,\,$ 1 $\,\,$ & $\,\,$4$\,\,$ & $\,\,$4$\,\,$ & $\,\,$6 $\,\,$ \\
$\,\,$ 1/4$\,\,$ & $\,\,$ 1 $\,\,$ & $\,\,$5$\,\,$ & $\,\,$3 $\,\,$ \\
$\,\,$ 1/4$\,\,$ & $\,\,$ 1/5$\,\,$ & $\,\,$ 1 $\,\,$ & $\,\,$1 $\,\,$ \\
$\,\,$ 1/6$\,\,$ & $\,\,$ 1/3$\,\,$ & $\,\,$ 1 $\,\,$ & $\,\,$ 1  $\,\,$ \\
\end{pmatrix},
\qquad
\lambda_{\max} =
4.2277,
\qquad
CR = 0.0859
\end{equation*}

\begin{equation*}
\mathbf{w}^{cos} =
\begin{pmatrix}
0.548169\\
0.271359\\
0.093916\\
\color{red} 0.086555\color{black}
\end{pmatrix}\end{equation*}
\begin{equation*}
\left[ \frac{{w}^{cos}_i}{{w}^{cos}_j} \right] =
\begin{pmatrix}
$\,\,$ 1 $\,\,$ & $\,\,$2.0201$\,\,$ & $\,\,$5.8368$\,\,$ & $\,\,$\color{red} 6.3332\color{black} $\,\,$ \\
$\,\,$0.4950$\,\,$ & $\,\,$ 1 $\,\,$ & $\,\,$2.8894$\,\,$ & $\,\,$\color{red} 3.1351\color{black}   $\,\,$ \\
$\,\,$0.1713$\,\,$ & $\,\,$0.3461$\,\,$ & $\,\,$ 1 $\,\,$ & $\,\,$\color{red} 1.0850\color{black}  $\,\,$ \\
$\,\,$\color{red} 0.1579\color{black} $\,\,$ & $\,\,$\color{red} 0.3190\color{black} $\,\,$ & $\,\,$\color{red} 0.9216\color{black} $\,\,$ & $\,\,$ 1  $\,\,$ \\
\end{pmatrix},
\end{equation*}

\begin{equation*}
\mathbf{w}^{\prime} =
\begin{pmatrix}
0.546041\\
0.270306\\
0.093552\\
0.090102
\end{pmatrix} =
0.996117\cdot
\begin{pmatrix}
0.548169\\
0.271359\\
0.093916\\
\color{gr} 0.090453\color{black}
\end{pmatrix},
\end{equation*}
\begin{equation*}
\left[ \frac{{w}^{\prime}_i}{{w}^{\prime}_j} \right] =
\begin{pmatrix}
$\,\,$ 1 $\,\,$ & $\,\,$2.0201$\,\,$ & $\,\,$5.8368$\,\,$ & $\,\,$\color{gr} 6.0603\color{black} $\,\,$ \\
$\,\,$0.4950$\,\,$ & $\,\,$ 1 $\,\,$ & $\,\,$2.8894$\,\,$ & $\,\,$\color{gr} \color{blue} 3\color{black}   $\,\,$ \\
$\,\,$0.1713$\,\,$ & $\,\,$0.3461$\,\,$ & $\,\,$ 1 $\,\,$ & $\,\,$\color{gr} 1.0383\color{black}  $\,\,$ \\
$\,\,$\color{gr} 0.1650\color{black} $\,\,$ & $\,\,$\color{gr} \color{blue}  1/3\color{black} $\,\,$ & $\,\,$\color{gr} 0.9631\color{black} $\,\,$ & $\,\,$ 1  $\,\,$ \\
\end{pmatrix},
\end{equation*}
\end{example}
\newpage
\begin{example}
\begin{equation*}
\mathbf{A} =
\begin{pmatrix}
$\,\,$ 1 $\,\,$ & $\,\,$4$\,\,$ & $\,\,$4$\,\,$ & $\,\,$7 $\,\,$ \\
$\,\,$ 1/4$\,\,$ & $\,\,$ 1 $\,\,$ & $\,\,$2$\,\,$ & $\,\,$9 $\,\,$ \\
$\,\,$ 1/4$\,\,$ & $\,\,$ 1/2$\,\,$ & $\,\,$ 1 $\,\,$ & $\,\,$3 $\,\,$ \\
$\,\,$ 1/7$\,\,$ & $\,\,$ 1/9$\,\,$ & $\,\,$ 1/3$\,\,$ & $\,\,$ 1  $\,\,$ \\
\end{pmatrix},
\qquad
\lambda_{\max} =
4.2359,
\qquad
CR = 0.0890
\end{equation*}

\begin{equation*}
\mathbf{w}^{cos} =
\begin{pmatrix}
0.544288\\
0.270614\\
\color{red} 0.133690\color{black} \\
0.051408
\end{pmatrix}\end{equation*}
\begin{equation*}
\left[ \frac{{w}^{cos}_i}{{w}^{cos}_j} \right] =
\begin{pmatrix}
$\,\,$ 1 $\,\,$ & $\,\,$2.0113$\,\,$ & $\,\,$\color{red} 4.0713\color{black} $\,\,$ & $\,\,$10.5875$\,\,$ \\
$\,\,$0.4972$\,\,$ & $\,\,$ 1 $\,\,$ & $\,\,$\color{red} 2.0242\color{black} $\,\,$ & $\,\,$5.2640  $\,\,$ \\
$\,\,$\color{red} 0.2456\color{black} $\,\,$ & $\,\,$\color{red} 0.4940\color{black} $\,\,$ & $\,\,$ 1 $\,\,$ & $\,\,$\color{red} 2.6006\color{black}  $\,\,$ \\
$\,\,$0.0945$\,\,$ & $\,\,$0.1900$\,\,$ & $\,\,$\color{red} 0.3845\color{black} $\,\,$ & $\,\,$ 1  $\,\,$ \\
\end{pmatrix},
\end{equation*}

\begin{equation*}
\mathbf{w}^{\prime} =
\begin{pmatrix}
0.543409\\
0.270177\\
0.135088\\
0.051325
\end{pmatrix} =
0.998386\cdot
\begin{pmatrix}
0.544288\\
0.270614\\
\color{gr} 0.135307\color{black} \\
0.051408
\end{pmatrix},
\end{equation*}
\begin{equation*}
\left[ \frac{{w}^{\prime}_i}{{w}^{\prime}_j} \right] =
\begin{pmatrix}
$\,\,$ 1 $\,\,$ & $\,\,$2.0113$\,\,$ & $\,\,$\color{gr} 4.0226\color{black} $\,\,$ & $\,\,$10.5875$\,\,$ \\
$\,\,$0.4972$\,\,$ & $\,\,$ 1 $\,\,$ & $\,\,$\color{gr} \color{blue} 2\color{black} $\,\,$ & $\,\,$5.2640  $\,\,$ \\
$\,\,$\color{gr} 0.2486\color{black} $\,\,$ & $\,\,$\color{gr} \color{blue}  1/2\color{black} $\,\,$ & $\,\,$ 1 $\,\,$ & $\,\,$\color{gr} 2.6320\color{black}  $\,\,$ \\
$\,\,$0.0945$\,\,$ & $\,\,$0.1900$\,\,$ & $\,\,$\color{gr} 0.3799\color{black} $\,\,$ & $\,\,$ 1  $\,\,$ \\
\end{pmatrix},
\end{equation*}
\end{example}
\newpage
\begin{example}
\begin{equation*}
\mathbf{A} =
\begin{pmatrix}
$\,\,$ 1 $\,\,$ & $\,\,$4$\,\,$ & $\,\,$4$\,\,$ & $\,\,$8 $\,\,$ \\
$\,\,$ 1/4$\,\,$ & $\,\,$ 1 $\,\,$ & $\,\,$2$\,\,$ & $\,\,$9 $\,\,$ \\
$\,\,$ 1/4$\,\,$ & $\,\,$ 1/2$\,\,$ & $\,\,$ 1 $\,\,$ & $\,\,$3 $\,\,$ \\
$\,\,$ 1/8$\,\,$ & $\,\,$ 1/9$\,\,$ & $\,\,$ 1/3$\,\,$ & $\,\,$ 1  $\,\,$ \\
\end{pmatrix},
\qquad
\lambda_{\max} =
4.1990,
\qquad
CR = 0.0750
\end{equation*}

\begin{equation*}
\mathbf{w}^{cos} =
\begin{pmatrix}
0.554394\\
0.265311\\
\color{red} 0.132096\color{black} \\
0.048198
\end{pmatrix}\end{equation*}
\begin{equation*}
\left[ \frac{{w}^{cos}_i}{{w}^{cos}_j} \right] =
\begin{pmatrix}
$\,\,$ 1 $\,\,$ & $\,\,$2.0896$\,\,$ & $\,\,$\color{red} 4.1969\color{black} $\,\,$ & $\,\,$11.5024$\,\,$ \\
$\,\,$0.4786$\,\,$ & $\,\,$ 1 $\,\,$ & $\,\,$\color{red} 2.0085\color{black} $\,\,$ & $\,\,$5.5046  $\,\,$ \\
$\,\,$\color{red} 0.2383\color{black} $\,\,$ & $\,\,$\color{red} 0.4979\color{black} $\,\,$ & $\,\,$ 1 $\,\,$ & $\,\,$\color{red} 2.7407\color{black}  $\,\,$ \\
$\,\,$0.0869$\,\,$ & $\,\,$0.1817$\,\,$ & $\,\,$\color{red} 0.3649\color{black} $\,\,$ & $\,\,$ 1  $\,\,$ \\
\end{pmatrix},
\end{equation*}

\begin{equation*}
\mathbf{w}^{\prime} =
\begin{pmatrix}
0.554085\\
0.265163\\
0.132581\\
0.048171
\end{pmatrix} =
0.999441\cdot
\begin{pmatrix}
0.554394\\
0.265311\\
\color{gr} 0.132655\color{black} \\
0.048198
\end{pmatrix},
\end{equation*}
\begin{equation*}
\left[ \frac{{w}^{\prime}_i}{{w}^{\prime}_j} \right] =
\begin{pmatrix}
$\,\,$ 1 $\,\,$ & $\,\,$2.0896$\,\,$ & $\,\,$\color{gr} 4.1792\color{black} $\,\,$ & $\,\,$11.5024$\,\,$ \\
$\,\,$0.4786$\,\,$ & $\,\,$ 1 $\,\,$ & $\,\,$\color{gr} \color{blue} 2\color{black} $\,\,$ & $\,\,$5.5046  $\,\,$ \\
$\,\,$\color{gr} 0.2393\color{black} $\,\,$ & $\,\,$\color{gr} \color{blue}  1/2\color{black} $\,\,$ & $\,\,$ 1 $\,\,$ & $\,\,$\color{gr} 2.7523\color{black}  $\,\,$ \\
$\,\,$0.0869$\,\,$ & $\,\,$0.1817$\,\,$ & $\,\,$\color{gr} 0.3633\color{black} $\,\,$ & $\,\,$ 1  $\,\,$ \\
\end{pmatrix},
\end{equation*}
\end{example}
\newpage
\begin{example}
\begin{equation*}
\mathbf{A} =
\begin{pmatrix}
$\,\,$ 1 $\,\,$ & $\,\,$4$\,\,$ & $\,\,$5$\,\,$ & $\,\,$7 $\,\,$ \\
$\,\,$ 1/4$\,\,$ & $\,\,$ 1 $\,\,$ & $\,\,$2$\,\,$ & $\,\,$6 $\,\,$ \\
$\,\,$ 1/5$\,\,$ & $\,\,$ 1/2$\,\,$ & $\,\,$ 1 $\,\,$ & $\,\,$2 $\,\,$ \\
$\,\,$ 1/7$\,\,$ & $\,\,$ 1/6$\,\,$ & $\,\,$ 1/2$\,\,$ & $\,\,$ 1  $\,\,$ \\
\end{pmatrix},
\qquad
\lambda_{\max} =
4.1301,
\qquad
CR = 0.0490
\end{equation*}

\begin{equation*}
\mathbf{w}^{cos} =
\begin{pmatrix}
0.582601\\
0.241470\\
\color{red} 0.115056\color{black} \\
0.060873
\end{pmatrix}\end{equation*}
\begin{equation*}
\left[ \frac{{w}^{cos}_i}{{w}^{cos}_j} \right] =
\begin{pmatrix}
$\,\,$ 1 $\,\,$ & $\,\,$2.4127$\,\,$ & $\,\,$\color{red} 5.0636\color{black} $\,\,$ & $\,\,$9.5708$\,\,$ \\
$\,\,$0.4145$\,\,$ & $\,\,$ 1 $\,\,$ & $\,\,$\color{red} 2.0987\color{black} $\,\,$ & $\,\,$3.9668  $\,\,$ \\
$\,\,$\color{red} 0.1975\color{black} $\,\,$ & $\,\,$\color{red} 0.4765\color{black} $\,\,$ & $\,\,$ 1 $\,\,$ & $\,\,$\color{red} 1.8901\color{black}  $\,\,$ \\
$\,\,$0.1045$\,\,$ & $\,\,$0.2521$\,\,$ & $\,\,$\color{red} 0.5291\color{black} $\,\,$ & $\,\,$ 1  $\,\,$ \\
\end{pmatrix},
\end{equation*}

\begin{equation*}
\mathbf{w}^{\prime} =
\begin{pmatrix}
0.581749\\
0.241117\\
0.116350\\
0.060784
\end{pmatrix} =
0.998538\cdot
\begin{pmatrix}
0.582601\\
0.241470\\
\color{gr} 0.116520\color{black} \\
0.060873
\end{pmatrix},
\end{equation*}
\begin{equation*}
\left[ \frac{{w}^{\prime}_i}{{w}^{\prime}_j} \right] =
\begin{pmatrix}
$\,\,$ 1 $\,\,$ & $\,\,$2.4127$\,\,$ & $\,\,$\color{gr} \color{blue} 5\color{black} $\,\,$ & $\,\,$9.5708$\,\,$ \\
$\,\,$0.4145$\,\,$ & $\,\,$ 1 $\,\,$ & $\,\,$\color{gr} 2.0723\color{black} $\,\,$ & $\,\,$3.9668  $\,\,$ \\
$\,\,$\color{gr} \color{blue}  1/5\color{black} $\,\,$ & $\,\,$\color{gr} 0.4825\color{black} $\,\,$ & $\,\,$ 1 $\,\,$ & $\,\,$\color{gr} 1.9142\color{black}  $\,\,$ \\
$\,\,$0.1045$\,\,$ & $\,\,$0.2521$\,\,$ & $\,\,$\color{gr} 0.5224\color{black} $\,\,$ & $\,\,$ 1  $\,\,$ \\
\end{pmatrix},
\end{equation*}
\end{example}
\newpage
\begin{example}
\begin{equation*}
\mathbf{A} =
\begin{pmatrix}
$\,\,$ 1 $\,\,$ & $\,\,$4$\,\,$ & $\,\,$5$\,\,$ & $\,\,$7 $\,\,$ \\
$\,\,$ 1/4$\,\,$ & $\,\,$ 1 $\,\,$ & $\,\,$2$\,\,$ & $\,\,$7 $\,\,$ \\
$\,\,$ 1/5$\,\,$ & $\,\,$ 1/2$\,\,$ & $\,\,$ 1 $\,\,$ & $\,\,$2 $\,\,$ \\
$\,\,$ 1/7$\,\,$ & $\,\,$ 1/7$\,\,$ & $\,\,$ 1/2$\,\,$ & $\,\,$ 1  $\,\,$ \\
\end{pmatrix},
\qquad
\lambda_{\max} =
4.1665,
\qquad
CR = 0.0628
\end{equation*}

\begin{equation*}
\mathbf{w}^{cos} =
\begin{pmatrix}
0.576341\\
0.251569\\
\color{red} 0.113119\color{black} \\
0.058971
\end{pmatrix}\end{equation*}
\begin{equation*}
\left[ \frac{{w}^{cos}_i}{{w}^{cos}_j} \right] =
\begin{pmatrix}
$\,\,$ 1 $\,\,$ & $\,\,$2.2910$\,\,$ & $\,\,$\color{red} 5.0950\color{black} $\,\,$ & $\,\,$9.7734$\,\,$ \\
$\,\,$0.4365$\,\,$ & $\,\,$ 1 $\,\,$ & $\,\,$\color{red} 2.2239\color{black} $\,\,$ & $\,\,$4.2660  $\,\,$ \\
$\,\,$\color{red} 0.1963\color{black} $\,\,$ & $\,\,$\color{red} 0.4497\color{black} $\,\,$ & $\,\,$ 1 $\,\,$ & $\,\,$\color{red} 1.9182\color{black}  $\,\,$ \\
$\,\,$0.1023$\,\,$ & $\,\,$0.2344$\,\,$ & $\,\,$\color{red} 0.5213\color{black} $\,\,$ & $\,\,$ 1  $\,\,$ \\
\end{pmatrix},
\end{equation*}

\begin{equation*}
\mathbf{w}^{\prime} =
\begin{pmatrix}
0.575105\\
0.251029\\
0.115021\\
0.058844
\end{pmatrix} =
0.997856\cdot
\begin{pmatrix}
0.576341\\
0.251569\\
\color{gr} 0.115268\color{black} \\
0.058971
\end{pmatrix},
\end{equation*}
\begin{equation*}
\left[ \frac{{w}^{\prime}_i}{{w}^{\prime}_j} \right] =
\begin{pmatrix}
$\,\,$ 1 $\,\,$ & $\,\,$2.2910$\,\,$ & $\,\,$\color{gr} \color{blue} 5\color{black} $\,\,$ & $\,\,$9.7734$\,\,$ \\
$\,\,$0.4365$\,\,$ & $\,\,$ 1 $\,\,$ & $\,\,$\color{gr} 2.1825\color{black} $\,\,$ & $\,\,$4.2660  $\,\,$ \\
$\,\,$\color{gr} \color{blue}  1/5\color{black} $\,\,$ & $\,\,$\color{gr} 0.4582\color{black} $\,\,$ & $\,\,$ 1 $\,\,$ & $\,\,$\color{gr} 1.9547\color{black}  $\,\,$ \\
$\,\,$0.1023$\,\,$ & $\,\,$0.2344$\,\,$ & $\,\,$\color{gr} 0.5116\color{black} $\,\,$ & $\,\,$ 1  $\,\,$ \\
\end{pmatrix},
\end{equation*}
\end{example}
\newpage
\begin{example}
\begin{equation*}
\mathbf{A} =
\begin{pmatrix}
$\,\,$ 1 $\,\,$ & $\,\,$4$\,\,$ & $\,\,$5$\,\,$ & $\,\,$7 $\,\,$ \\
$\,\,$ 1/4$\,\,$ & $\,\,$ 1 $\,\,$ & $\,\,$2$\,\,$ & $\,\,$8 $\,\,$ \\
$\,\,$ 1/5$\,\,$ & $\,\,$ 1/2$\,\,$ & $\,\,$ 1 $\,\,$ & $\,\,$2 $\,\,$ \\
$\,\,$ 1/7$\,\,$ & $\,\,$ 1/8$\,\,$ & $\,\,$ 1/2$\,\,$ & $\,\,$ 1  $\,\,$ \\
\end{pmatrix},
\qquad
\lambda_{\max} =
4.2035,
\qquad
CR = 0.0767
\end{equation*}

\begin{equation*}
\mathbf{w}^{cos} =
\begin{pmatrix}
0.570988\\
0.260205\\
\color{red} 0.111398\color{black} \\
0.057409
\end{pmatrix}\end{equation*}
\begin{equation*}
\left[ \frac{{w}^{cos}_i}{{w}^{cos}_j} \right] =
\begin{pmatrix}
$\,\,$ 1 $\,\,$ & $\,\,$2.1944$\,\,$ & $\,\,$\color{red} 5.1257\color{black} $\,\,$ & $\,\,$9.9460$\,\,$ \\
$\,\,$0.4557$\,\,$ & $\,\,$ 1 $\,\,$ & $\,\,$\color{red} 2.3358\color{black} $\,\,$ & $\,\,$4.5325  $\,\,$ \\
$\,\,$\color{red} 0.1951\color{black} $\,\,$ & $\,\,$\color{red} 0.4281\color{black} $\,\,$ & $\,\,$ 1 $\,\,$ & $\,\,$\color{red} 1.9404\color{black}  $\,\,$ \\
$\,\,$0.1005$\,\,$ & $\,\,$0.2206$\,\,$ & $\,\,$\color{red} 0.5153\color{black} $\,\,$ & $\,\,$ 1  $\,\,$ \\
\end{pmatrix},
\end{equation*}

\begin{equation*}
\mathbf{w}^{\prime} =
\begin{pmatrix}
0.569394\\
0.259479\\
0.113879\\
0.057249
\end{pmatrix} =
0.997208\cdot
\begin{pmatrix}
0.570988\\
0.260205\\
\color{gr} 0.114198\color{black} \\
0.057409
\end{pmatrix},
\end{equation*}
\begin{equation*}
\left[ \frac{{w}^{\prime}_i}{{w}^{\prime}_j} \right] =
\begin{pmatrix}
$\,\,$ 1 $\,\,$ & $\,\,$2.1944$\,\,$ & $\,\,$\color{gr} \color{blue} 5\color{black} $\,\,$ & $\,\,$9.9460$\,\,$ \\
$\,\,$0.4557$\,\,$ & $\,\,$ 1 $\,\,$ & $\,\,$\color{gr} 2.2786\color{black} $\,\,$ & $\,\,$4.5325  $\,\,$ \\
$\,\,$\color{gr} \color{blue}  1/5\color{black} $\,\,$ & $\,\,$\color{gr} 0.4389\color{black} $\,\,$ & $\,\,$ 1 $\,\,$ & $\,\,$\color{gr} 1.9892\color{black}  $\,\,$ \\
$\,\,$0.1005$\,\,$ & $\,\,$0.2206$\,\,$ & $\,\,$\color{gr} 0.5027\color{black} $\,\,$ & $\,\,$ 1  $\,\,$ \\
\end{pmatrix},
\end{equation*}
\end{example}
\newpage
\begin{example}
\begin{equation*}
\mathbf{A} =
\begin{pmatrix}
$\,\,$ 1 $\,\,$ & $\,\,$4$\,\,$ & $\,\,$5$\,\,$ & $\,\,$7 $\,\,$ \\
$\,\,$ 1/4$\,\,$ & $\,\,$ 1 $\,\,$ & $\,\,$2$\,\,$ & $\,\,$9 $\,\,$ \\
$\,\,$ 1/5$\,\,$ & $\,\,$ 1/2$\,\,$ & $\,\,$ 1 $\,\,$ & $\,\,$2 $\,\,$ \\
$\,\,$ 1/7$\,\,$ & $\,\,$ 1/9$\,\,$ & $\,\,$ 1/2$\,\,$ & $\,\,$ 1  $\,\,$ \\
\end{pmatrix},
\qquad
\lambda_{\max} =
4.2405,
\qquad
CR = 0.0907
\end{equation*}

\begin{equation*}
\mathbf{w}^{cos} =
\begin{pmatrix}
0.566418\\
0.267606\\
\color{red} 0.109875\color{black} \\
0.056100
\end{pmatrix}\end{equation*}
\begin{equation*}
\left[ \frac{{w}^{cos}_i}{{w}^{cos}_j} \right] =
\begin{pmatrix}
$\,\,$ 1 $\,\,$ & $\,\,$2.1166$\,\,$ & $\,\,$\color{red} 5.1551\color{black} $\,\,$ & $\,\,$10.0965$\,\,$ \\
$\,\,$0.4725$\,\,$ & $\,\,$ 1 $\,\,$ & $\,\,$\color{red} 2.4355\color{black} $\,\,$ & $\,\,$4.7701  $\,\,$ \\
$\,\,$\color{red} 0.1940\color{black} $\,\,$ & $\,\,$\color{red} 0.4106\color{black} $\,\,$ & $\,\,$ 1 $\,\,$ & $\,\,$\color{red} 1.9586\color{black}  $\,\,$ \\
$\,\,$0.0990$\,\,$ & $\,\,$0.2096$\,\,$ & $\,\,$\color{red} 0.5106\color{black} $\,\,$ & $\,\,$ 1  $\,\,$ \\
\end{pmatrix},
\end{equation*}

\begin{equation*}
\mathbf{w}^{\prime} =
\begin{pmatrix}
0.565104\\
0.266986\\
0.111940\\
0.055970
\end{pmatrix} =
0.997680\cdot
\begin{pmatrix}
0.566418\\
0.267606\\
\color{gr} 0.112201\color{black} \\
0.056100
\end{pmatrix},
\end{equation*}
\begin{equation*}
\left[ \frac{{w}^{\prime}_i}{{w}^{\prime}_j} \right] =
\begin{pmatrix}
$\,\,$ 1 $\,\,$ & $\,\,$2.1166$\,\,$ & $\,\,$\color{gr} 5.0483\color{black} $\,\,$ & $\,\,$10.0965$\,\,$ \\
$\,\,$0.4725$\,\,$ & $\,\,$ 1 $\,\,$ & $\,\,$\color{gr} 2.3851\color{black} $\,\,$ & $\,\,$4.7701  $\,\,$ \\
$\,\,$\color{gr} 0.1981\color{black} $\,\,$ & $\,\,$\color{gr} 0.4193\color{black} $\,\,$ & $\,\,$ 1 $\,\,$ & $\,\,$\color{gr} \color{blue} 2\color{black}  $\,\,$ \\
$\,\,$0.0990$\,\,$ & $\,\,$0.2096$\,\,$ & $\,\,$\color{gr} \color{blue}  1/2\color{black} $\,\,$ & $\,\,$ 1  $\,\,$ \\
\end{pmatrix},
\end{equation*}
\end{example}
\newpage
\begin{example}
\begin{equation*}
\mathbf{A} =
\begin{pmatrix}
$\,\,$ 1 $\,\,$ & $\,\,$4$\,\,$ & $\,\,$5$\,\,$ & $\,\,$7 $\,\,$ \\
$\,\,$ 1/4$\,\,$ & $\,\,$ 1 $\,\,$ & $\,\,$5$\,\,$ & $\,\,$3 $\,\,$ \\
$\,\,$ 1/5$\,\,$ & $\,\,$ 1/5$\,\,$ & $\,\,$ 1 $\,\,$ & $\,\,$1 $\,\,$ \\
$\,\,$ 1/7$\,\,$ & $\,\,$ 1/3$\,\,$ & $\,\,$ 1 $\,\,$ & $\,\,$ 1  $\,\,$ \\
\end{pmatrix},
\qquad
\lambda_{\max} =
4.1667,
\qquad
CR = 0.0629
\end{equation*}

\begin{equation*}
\mathbf{w}^{cos} =
\begin{pmatrix}
0.579361\\
0.257422\\
0.083404\\
\color{red} 0.079813\color{black}
\end{pmatrix}\end{equation*}
\begin{equation*}
\left[ \frac{{w}^{cos}_i}{{w}^{cos}_j} \right] =
\begin{pmatrix}
$\,\,$ 1 $\,\,$ & $\,\,$2.2506$\,\,$ & $\,\,$6.9465$\,\,$ & $\,\,$\color{red} 7.2589\color{black} $\,\,$ \\
$\,\,$0.4443$\,\,$ & $\,\,$ 1 $\,\,$ & $\,\,$3.0865$\,\,$ & $\,\,$\color{red} 3.2253\color{black}   $\,\,$ \\
$\,\,$0.1440$\,\,$ & $\,\,$0.3240$\,\,$ & $\,\,$ 1 $\,\,$ & $\,\,$\color{red} 1.0450\color{black}  $\,\,$ \\
$\,\,$\color{red} 0.1378\color{black} $\,\,$ & $\,\,$\color{red} 0.3100\color{black} $\,\,$ & $\,\,$\color{red} 0.9570\color{black} $\,\,$ & $\,\,$ 1  $\,\,$ \\
\end{pmatrix},
\end{equation*}

\begin{equation*}
\mathbf{w}^{\prime} =
\begin{pmatrix}
0.577656\\
0.256664\\
0.083158\\
0.082522
\end{pmatrix} =
0.997056\cdot
\begin{pmatrix}
0.579361\\
0.257422\\
0.083404\\
\color{gr} 0.082766\color{black}
\end{pmatrix},
\end{equation*}
\begin{equation*}
\left[ \frac{{w}^{\prime}_i}{{w}^{\prime}_j} \right] =
\begin{pmatrix}
$\,\,$ 1 $\,\,$ & $\,\,$2.2506$\,\,$ & $\,\,$6.9465$\,\,$ & $\,\,$\color{gr} \color{blue} 7\color{black} $\,\,$ \\
$\,\,$0.4443$\,\,$ & $\,\,$ 1 $\,\,$ & $\,\,$3.0865$\,\,$ & $\,\,$\color{gr} 3.1102\color{black}   $\,\,$ \\
$\,\,$0.1440$\,\,$ & $\,\,$0.3240$\,\,$ & $\,\,$ 1 $\,\,$ & $\,\,$\color{gr} 1.0077\color{black}  $\,\,$ \\
$\,\,$\color{gr} \color{blue}  1/7\color{black} $\,\,$ & $\,\,$\color{gr} 0.3215\color{black} $\,\,$ & $\,\,$\color{gr} 0.9924\color{black} $\,\,$ & $\,\,$ 1  $\,\,$ \\
\end{pmatrix},
\end{equation*}
\end{example}
\newpage
\begin{example}
\begin{equation*}
\mathbf{A} =
\begin{pmatrix}
$\,\,$ 1 $\,\,$ & $\,\,$4$\,\,$ & $\,\,$5$\,\,$ & $\,\,$7 $\,\,$ \\
$\,\,$ 1/4$\,\,$ & $\,\,$ 1 $\,\,$ & $\,\,$6$\,\,$ & $\,\,$3 $\,\,$ \\
$\,\,$ 1/5$\,\,$ & $\,\,$ 1/6$\,\,$ & $\,\,$ 1 $\,\,$ & $\,\,$1 $\,\,$ \\
$\,\,$ 1/7$\,\,$ & $\,\,$ 1/3$\,\,$ & $\,\,$ 1 $\,\,$ & $\,\,$ 1  $\,\,$ \\
\end{pmatrix},
\qquad
\lambda_{\max} =
4.2174,
\qquad
CR = 0.0820
\end{equation*}

\begin{equation*}
\mathbf{w}^{cos} =
\begin{pmatrix}
0.572135\\
0.269276\\
0.080430\\
\color{red} 0.078159\color{black}
\end{pmatrix}\end{equation*}
\begin{equation*}
\left[ \frac{{w}^{cos}_i}{{w}^{cos}_j} \right] =
\begin{pmatrix}
$\,\,$ 1 $\,\,$ & $\,\,$2.1247$\,\,$ & $\,\,$7.1135$\,\,$ & $\,\,$\color{red} 7.3202\color{black} $\,\,$ \\
$\,\,$0.4707$\,\,$ & $\,\,$ 1 $\,\,$ & $\,\,$3.3480$\,\,$ & $\,\,$\color{red} 3.4452\color{black}   $\,\,$ \\
$\,\,$0.1406$\,\,$ & $\,\,$0.2987$\,\,$ & $\,\,$ 1 $\,\,$ & $\,\,$\color{red} 1.0291\color{black}  $\,\,$ \\
$\,\,$\color{red} 0.1366\color{black} $\,\,$ & $\,\,$\color{red} 0.2903\color{black} $\,\,$ & $\,\,$\color{red} 0.9718\color{black} $\,\,$ & $\,\,$ 1  $\,\,$ \\
\end{pmatrix},
\end{equation*}

\begin{equation*}
\mathbf{w}^{\prime} =
\begin{pmatrix}
0.570839\\
0.268666\\
0.080247\\
0.080247
\end{pmatrix} =
0.997734\cdot
\begin{pmatrix}
0.572135\\
0.269276\\
0.080430\\
\color{gr} 0.080430\color{black}
\end{pmatrix},
\end{equation*}
\begin{equation*}
\left[ \frac{{w}^{\prime}_i}{{w}^{\prime}_j} \right] =
\begin{pmatrix}
$\,\,$ 1 $\,\,$ & $\,\,$2.1247$\,\,$ & $\,\,$7.1135$\,\,$ & $\,\,$\color{gr} 7.1135\color{black} $\,\,$ \\
$\,\,$0.4707$\,\,$ & $\,\,$ 1 $\,\,$ & $\,\,$3.3480$\,\,$ & $\,\,$\color{gr} 3.3480\color{black}   $\,\,$ \\
$\,\,$0.1406$\,\,$ & $\,\,$0.2987$\,\,$ & $\,\,$ 1 $\,\,$ & $\,\,$\color{gr} \color{blue} 1\color{black}  $\,\,$ \\
$\,\,$\color{gr} 0.1406\color{black} $\,\,$ & $\,\,$\color{gr} 0.2987\color{black} $\,\,$ & $\,\,$\color{gr} \color{blue} 1\color{black} $\,\,$ & $\,\,$ 1  $\,\,$ \\
\end{pmatrix},
\end{equation*}
\end{example}
\newpage
\begin{example}
\begin{equation*}
\mathbf{A} =
\begin{pmatrix}
$\,\,$ 1 $\,\,$ & $\,\,$4$\,\,$ & $\,\,$5$\,\,$ & $\,\,$8 $\,\,$ \\
$\,\,$ 1/4$\,\,$ & $\,\,$ 1 $\,\,$ & $\,\,$2$\,\,$ & $\,\,$6 $\,\,$ \\
$\,\,$ 1/5$\,\,$ & $\,\,$ 1/2$\,\,$ & $\,\,$ 1 $\,\,$ & $\,\,$2 $\,\,$ \\
$\,\,$ 1/8$\,\,$ & $\,\,$ 1/6$\,\,$ & $\,\,$ 1/2$\,\,$ & $\,\,$ 1  $\,\,$ \\
\end{pmatrix},
\qquad
\lambda_{\max} =
4.1046,
\qquad
CR = 0.0395
\end{equation*}

\begin{equation*}
\mathbf{w}^{cos} =
\begin{pmatrix}
0.593971\\
0.235450\\
\color{red} 0.113323\color{black} \\
0.057255
\end{pmatrix}\end{equation*}
\begin{equation*}
\left[ \frac{{w}^{cos}_i}{{w}^{cos}_j} \right] =
\begin{pmatrix}
$\,\,$ 1 $\,\,$ & $\,\,$2.5227$\,\,$ & $\,\,$\color{red} 5.2414\color{black} $\,\,$ & $\,\,$10.3741$\,\,$ \\
$\,\,$0.3964$\,\,$ & $\,\,$ 1 $\,\,$ & $\,\,$\color{red} 2.0777\color{black} $\,\,$ & $\,\,$4.1123  $\,\,$ \\
$\,\,$\color{red} 0.1908\color{black} $\,\,$ & $\,\,$\color{red} 0.4813\color{black} $\,\,$ & $\,\,$ 1 $\,\,$ & $\,\,$\color{red} 1.9793\color{black}  $\,\,$ \\
$\,\,$0.0964$\,\,$ & $\,\,$0.2432$\,\,$ & $\,\,$\color{red} 0.5052\color{black} $\,\,$ & $\,\,$ 1  $\,\,$ \\
\end{pmatrix},
\end{equation*}

\begin{equation*}
\mathbf{w}^{\prime} =
\begin{pmatrix}
0.593267\\
0.235171\\
0.114375\\
0.057187
\end{pmatrix} =
0.998815\cdot
\begin{pmatrix}
0.593971\\
0.235450\\
\color{gr} 0.114510\color{black} \\
0.057255
\end{pmatrix},
\end{equation*}
\begin{equation*}
\left[ \frac{{w}^{\prime}_i}{{w}^{\prime}_j} \right] =
\begin{pmatrix}
$\,\,$ 1 $\,\,$ & $\,\,$2.5227$\,\,$ & $\,\,$\color{gr} 5.1871\color{black} $\,\,$ & $\,\,$10.3741$\,\,$ \\
$\,\,$0.3964$\,\,$ & $\,\,$ 1 $\,\,$ & $\,\,$\color{gr} 2.0562\color{black} $\,\,$ & $\,\,$4.1123  $\,\,$ \\
$\,\,$\color{gr} 0.1928\color{black} $\,\,$ & $\,\,$\color{gr} 0.4863\color{black} $\,\,$ & $\,\,$ 1 $\,\,$ & $\,\,$\color{gr} \color{blue} 2\color{black}  $\,\,$ \\
$\,\,$0.0964$\,\,$ & $\,\,$0.2432$\,\,$ & $\,\,$\color{gr} \color{blue}  1/2\color{black} $\,\,$ & $\,\,$ 1  $\,\,$ \\
\end{pmatrix},
\end{equation*}
\end{example}
\newpage
\begin{example}
\begin{equation*}
\mathbf{A} =
\begin{pmatrix}
$\,\,$ 1 $\,\,$ & $\,\,$4$\,\,$ & $\,\,$5$\,\,$ & $\,\,$8 $\,\,$ \\
$\,\,$ 1/4$\,\,$ & $\,\,$ 1 $\,\,$ & $\,\,$6$\,\,$ & $\,\,$4 $\,\,$ \\
$\,\,$ 1/5$\,\,$ & $\,\,$ 1/6$\,\,$ & $\,\,$ 1 $\,\,$ & $\,\,$1 $\,\,$ \\
$\,\,$ 1/8$\,\,$ & $\,\,$ 1/4$\,\,$ & $\,\,$ 1 $\,\,$ & $\,\,$ 1  $\,\,$ \\
\end{pmatrix},
\qquad
\lambda_{\max} =
4.2162,
\qquad
CR = 0.0815
\end{equation*}

\begin{equation*}
\mathbf{w}^{cos} =
\begin{pmatrix}
0.572675\\
0.280093\\
0.077858\\
\color{red} 0.069374\color{black}
\end{pmatrix}\end{equation*}
\begin{equation*}
\left[ \frac{{w}^{cos}_i}{{w}^{cos}_j} \right] =
\begin{pmatrix}
$\,\,$ 1 $\,\,$ & $\,\,$2.0446$\,\,$ & $\,\,$7.3553$\,\,$ & $\,\,$\color{red} 8.2549\color{black} $\,\,$ \\
$\,\,$0.4891$\,\,$ & $\,\,$ 1 $\,\,$ & $\,\,$3.5975$\,\,$ & $\,\,$\color{red} 4.0375\color{black}   $\,\,$ \\
$\,\,$0.1360$\,\,$ & $\,\,$0.2780$\,\,$ & $\,\,$ 1 $\,\,$ & $\,\,$\color{red} 1.1223\color{black}  $\,\,$ \\
$\,\,$\color{red} 0.1211\color{black} $\,\,$ & $\,\,$\color{red} 0.2477\color{black} $\,\,$ & $\,\,$\color{red} 0.8910\color{black} $\,\,$ & $\,\,$ 1  $\,\,$ \\
\end{pmatrix},
\end{equation*}

\begin{equation*}
\mathbf{w}^{\prime} =
\begin{pmatrix}
0.572303\\
0.279911\\
0.077808\\
0.069978
\end{pmatrix} =
0.999351\cdot
\begin{pmatrix}
0.572675\\
0.280093\\
0.077858\\
\color{gr} 0.070023\color{black}
\end{pmatrix},
\end{equation*}
\begin{equation*}
\left[ \frac{{w}^{\prime}_i}{{w}^{\prime}_j} \right] =
\begin{pmatrix}
$\,\,$ 1 $\,\,$ & $\,\,$2.0446$\,\,$ & $\,\,$7.3553$\,\,$ & $\,\,$\color{gr} 8.1783\color{black} $\,\,$ \\
$\,\,$0.4891$\,\,$ & $\,\,$ 1 $\,\,$ & $\,\,$3.5975$\,\,$ & $\,\,$\color{gr} \color{blue} 4\color{black}   $\,\,$ \\
$\,\,$0.1360$\,\,$ & $\,\,$0.2780$\,\,$ & $\,\,$ 1 $\,\,$ & $\,\,$\color{gr} 1.1119\color{black}  $\,\,$ \\
$\,\,$\color{gr} 0.1223\color{black} $\,\,$ & $\,\,$\color{gr} \color{blue}  1/4\color{black} $\,\,$ & $\,\,$\color{gr} 0.8994\color{black} $\,\,$ & $\,\,$ 1  $\,\,$ \\
\end{pmatrix},
\end{equation*}
\end{example}
\newpage
\begin{example}
\begin{equation*}
\mathbf{A} =
\begin{pmatrix}
$\,\,$ 1 $\,\,$ & $\,\,$4$\,\,$ & $\,\,$5$\,\,$ & $\,\,$8 $\,\,$ \\
$\,\,$ 1/4$\,\,$ & $\,\,$ 1 $\,\,$ & $\,\,$7$\,\,$ & $\,\,$4 $\,\,$ \\
$\,\,$ 1/5$\,\,$ & $\,\,$ 1/7$\,\,$ & $\,\,$ 1 $\,\,$ & $\,\,$1 $\,\,$ \\
$\,\,$ 1/8$\,\,$ & $\,\,$ 1/4$\,\,$ & $\,\,$ 1 $\,\,$ & $\,\,$ 1  $\,\,$ \\
\end{pmatrix},
\qquad
\lambda_{\max} =
4.2610,
\qquad
CR = 0.0984
\end{equation*}

\begin{equation*}
\mathbf{w}^{cos} =
\begin{pmatrix}
0.566828\\
0.289773\\
0.075484\\
\color{red} 0.067915\color{black}
\end{pmatrix}\end{equation*}
\begin{equation*}
\left[ \frac{{w}^{cos}_i}{{w}^{cos}_j} \right] =
\begin{pmatrix}
$\,\,$ 1 $\,\,$ & $\,\,$1.9561$\,\,$ & $\,\,$7.5092$\,\,$ & $\,\,$\color{red} 8.3461\color{black} $\,\,$ \\
$\,\,$0.5112$\,\,$ & $\,\,$ 1 $\,\,$ & $\,\,$3.8389$\,\,$ & $\,\,$\color{red} 4.2667\color{black}   $\,\,$ \\
$\,\,$0.1332$\,\,$ & $\,\,$0.2605$\,\,$ & $\,\,$ 1 $\,\,$ & $\,\,$\color{red} 1.1114\color{black}  $\,\,$ \\
$\,\,$\color{red} 0.1198\color{black} $\,\,$ & $\,\,$\color{red} 0.2344\color{black} $\,\,$ & $\,\,$\color{red} 0.8997\color{black} $\,\,$ & $\,\,$ 1  $\,\,$ \\
\end{pmatrix},
\end{equation*}

\begin{equation*}
\mathbf{w}^{\prime} =
\begin{pmatrix}
0.565167\\
0.288924\\
0.075263\\
0.070646
\end{pmatrix} =
0.997070\cdot
\begin{pmatrix}
0.566828\\
0.289773\\
0.075484\\
\color{gr} 0.070853\color{black}
\end{pmatrix},
\end{equation*}
\begin{equation*}
\left[ \frac{{w}^{\prime}_i}{{w}^{\prime}_j} \right] =
\begin{pmatrix}
$\,\,$ 1 $\,\,$ & $\,\,$1.9561$\,\,$ & $\,\,$7.5092$\,\,$ & $\,\,$\color{gr} \color{blue} 8\color{black} $\,\,$ \\
$\,\,$0.5112$\,\,$ & $\,\,$ 1 $\,\,$ & $\,\,$3.8389$\,\,$ & $\,\,$\color{gr} 4.0898\color{black}   $\,\,$ \\
$\,\,$0.1332$\,\,$ & $\,\,$0.2605$\,\,$ & $\,\,$ 1 $\,\,$ & $\,\,$\color{gr} 1.0654\color{black}  $\,\,$ \\
$\,\,$\color{gr} \color{blue}  1/8\color{black} $\,\,$ & $\,\,$\color{gr} 0.2445\color{black} $\,\,$ & $\,\,$\color{gr} 0.9387\color{black} $\,\,$ & $\,\,$ 1  $\,\,$ \\
\end{pmatrix},
\end{equation*}
\end{example}
\newpage
\begin{example}
\begin{equation*}
\mathbf{A} =
\begin{pmatrix}
$\,\,$ 1 $\,\,$ & $\,\,$4$\,\,$ & $\,\,$6$\,\,$ & $\,\,$8 $\,\,$ \\
$\,\,$ 1/4$\,\,$ & $\,\,$ 1 $\,\,$ & $\,\,$4$\,\,$ & $\,\,$3 $\,\,$ \\
$\,\,$ 1/6$\,\,$ & $\,\,$ 1/4$\,\,$ & $\,\,$ 1 $\,\,$ & $\,\,$1 $\,\,$ \\
$\,\,$ 1/8$\,\,$ & $\,\,$ 1/3$\,\,$ & $\,\,$ 1 $\,\,$ & $\,\,$ 1  $\,\,$ \\
\end{pmatrix},
\qquad
\lambda_{\max} =
4.0820,
\qquad
CR = 0.0309
\end{equation*}

\begin{equation*}
\mathbf{w}^{cos} =
\begin{pmatrix}
0.614790\\
0.230315\\
0.079088\\
\color{red} 0.075807\color{black}
\end{pmatrix}\end{equation*}
\begin{equation*}
\left[ \frac{{w}^{cos}_i}{{w}^{cos}_j} \right] =
\begin{pmatrix}
$\,\,$ 1 $\,\,$ & $\,\,$2.6693$\,\,$ & $\,\,$7.7735$\,\,$ & $\,\,$\color{red} 8.1100\color{black} $\,\,$ \\
$\,\,$0.3746$\,\,$ & $\,\,$ 1 $\,\,$ & $\,\,$2.9122$\,\,$ & $\,\,$\color{red} 3.0382\color{black}   $\,\,$ \\
$\,\,$0.1286$\,\,$ & $\,\,$0.3434$\,\,$ & $\,\,$ 1 $\,\,$ & $\,\,$\color{red} 1.0433\color{black}  $\,\,$ \\
$\,\,$\color{red} 0.1233\color{black} $\,\,$ & $\,\,$\color{red} 0.3291\color{black} $\,\,$ & $\,\,$\color{red} 0.9585\color{black} $\,\,$ & $\,\,$ 1  $\,\,$ \\
\end{pmatrix},
\end{equation*}

\begin{equation*}
\mathbf{w}^{\prime} =
\begin{pmatrix}
0.614197\\
0.230093\\
0.079011\\
0.076698
\end{pmatrix} =
0.999036\cdot
\begin{pmatrix}
0.614790\\
0.230315\\
0.079088\\
\color{gr} 0.076772\color{black}
\end{pmatrix},
\end{equation*}
\begin{equation*}
\left[ \frac{{w}^{\prime}_i}{{w}^{\prime}_j} \right] =
\begin{pmatrix}
$\,\,$ 1 $\,\,$ & $\,\,$2.6693$\,\,$ & $\,\,$7.7735$\,\,$ & $\,\,$\color{gr} 8.0080\color{black} $\,\,$ \\
$\,\,$0.3746$\,\,$ & $\,\,$ 1 $\,\,$ & $\,\,$2.9122$\,\,$ & $\,\,$\color{gr} \color{blue} 3\color{black}   $\,\,$ \\
$\,\,$0.1286$\,\,$ & $\,\,$0.3434$\,\,$ & $\,\,$ 1 $\,\,$ & $\,\,$\color{gr} 1.0302\color{black}  $\,\,$ \\
$\,\,$\color{gr} 0.1249\color{black} $\,\,$ & $\,\,$\color{gr} \color{blue}  1/3\color{black} $\,\,$ & $\,\,$\color{gr} 0.9707\color{black} $\,\,$ & $\,\,$ 1  $\,\,$ \\
\end{pmatrix},
\end{equation*}
\end{example}
\newpage
\begin{example}
\begin{equation*}
\mathbf{A} =
\begin{pmatrix}
$\,\,$ 1 $\,\,$ & $\,\,$4$\,\,$ & $\,\,$6$\,\,$ & $\,\,$8 $\,\,$ \\
$\,\,$ 1/4$\,\,$ & $\,\,$ 1 $\,\,$ & $\,\,$5$\,\,$ & $\,\,$3 $\,\,$ \\
$\,\,$ 1/6$\,\,$ & $\,\,$ 1/5$\,\,$ & $\,\,$ 1 $\,\,$ & $\,\,$1 $\,\,$ \\
$\,\,$ 1/8$\,\,$ & $\,\,$ 1/3$\,\,$ & $\,\,$ 1 $\,\,$ & $\,\,$ 1  $\,\,$ \\
\end{pmatrix},
\qquad
\lambda_{\max} =
4.1252,
\qquad
CR = 0.0472
\end{equation*}

\begin{equation*}
\mathbf{w}^{cos} =
\begin{pmatrix}
0.605358\\
0.244998\\
0.075447\\
\color{red} 0.074197\color{black}
\end{pmatrix}\end{equation*}
\begin{equation*}
\left[ \frac{{w}^{cos}_i}{{w}^{cos}_j} \right] =
\begin{pmatrix}
$\,\,$ 1 $\,\,$ & $\,\,$2.4709$\,\,$ & $\,\,$8.0236$\,\,$ & $\,\,$\color{red} 8.1587\color{black} $\,\,$ \\
$\,\,$0.4047$\,\,$ & $\,\,$ 1 $\,\,$ & $\,\,$3.2473$\,\,$ & $\,\,$\color{red} 3.3020\color{black}   $\,\,$ \\
$\,\,$0.1246$\,\,$ & $\,\,$0.3080$\,\,$ & $\,\,$ 1 $\,\,$ & $\,\,$\color{red} 1.0168\color{black}  $\,\,$ \\
$\,\,$\color{red} 0.1226\color{black} $\,\,$ & $\,\,$\color{red} 0.3028\color{black} $\,\,$ & $\,\,$\color{red} 0.9834\color{black} $\,\,$ & $\,\,$ 1  $\,\,$ \\
\end{pmatrix},
\end{equation*}

\begin{equation*}
\mathbf{w}^{\prime} =
\begin{pmatrix}
0.604602\\
0.244692\\
0.075353\\
0.075353
\end{pmatrix} =
0.998752\cdot
\begin{pmatrix}
0.605358\\
0.244998\\
0.075447\\
\color{gr} 0.075447\color{black}
\end{pmatrix},
\end{equation*}
\begin{equation*}
\left[ \frac{{w}^{\prime}_i}{{w}^{\prime}_j} \right] =
\begin{pmatrix}
$\,\,$ 1 $\,\,$ & $\,\,$2.4709$\,\,$ & $\,\,$8.0236$\,\,$ & $\,\,$\color{gr} 8.0236\color{black} $\,\,$ \\
$\,\,$0.4047$\,\,$ & $\,\,$ 1 $\,\,$ & $\,\,$3.2473$\,\,$ & $\,\,$\color{gr} 3.2473\color{black}   $\,\,$ \\
$\,\,$0.1246$\,\,$ & $\,\,$0.3080$\,\,$ & $\,\,$ 1 $\,\,$ & $\,\,$\color{gr} \color{blue} 1\color{black}  $\,\,$ \\
$\,\,$\color{gr} 0.1246\color{black} $\,\,$ & $\,\,$\color{gr} 0.3080\color{black} $\,\,$ & $\,\,$\color{gr} \color{blue} 1\color{black} $\,\,$ & $\,\,$ 1  $\,\,$ \\
\end{pmatrix},
\end{equation*}
\end{example}
\newpage
\begin{example}
\begin{equation*}
\mathbf{A} =
\begin{pmatrix}
$\,\,$ 1 $\,\,$ & $\,\,$4$\,\,$ & $\,\,$6$\,\,$ & $\,\,$8 $\,\,$ \\
$\,\,$ 1/4$\,\,$ & $\,\,$ 1 $\,\,$ & $\,\,$7$\,\,$ & $\,\,$4 $\,\,$ \\
$\,\,$ 1/6$\,\,$ & $\,\,$ 1/7$\,\,$ & $\,\,$ 1 $\,\,$ & $\,\,$1 $\,\,$ \\
$\,\,$ 1/8$\,\,$ & $\,\,$ 1/4$\,\,$ & $\,\,$ 1 $\,\,$ & $\,\,$ 1  $\,\,$ \\
\end{pmatrix},
\qquad
\lambda_{\max} =
4.2109,
\qquad
CR = 0.0795
\end{equation*}

\begin{equation*}
\mathbf{w}^{cos} =
\begin{pmatrix}
0.581507\\
0.282175\\
0.069315\\
\color{red} 0.067003\color{black}
\end{pmatrix}\end{equation*}
\begin{equation*}
\left[ \frac{{w}^{cos}_i}{{w}^{cos}_j} \right] =
\begin{pmatrix}
$\,\,$ 1 $\,\,$ & $\,\,$2.0608$\,\,$ & $\,\,$8.3893$\,\,$ & $\,\,$\color{red} 8.6788\color{black} $\,\,$ \\
$\,\,$0.4852$\,\,$ & $\,\,$ 1 $\,\,$ & $\,\,$4.0709$\,\,$ & $\,\,$\color{red} 4.2114\color{black}   $\,\,$ \\
$\,\,$0.1192$\,\,$ & $\,\,$0.2456$\,\,$ & $\,\,$ 1 $\,\,$ & $\,\,$\color{red} 1.0345\color{black}  $\,\,$ \\
$\,\,$\color{red} 0.1152\color{black} $\,\,$ & $\,\,$\color{red} 0.2375\color{black} $\,\,$ & $\,\,$\color{red} 0.9666\color{black} $\,\,$ & $\,\,$ 1  $\,\,$ \\
\end{pmatrix},
\end{equation*}

\begin{equation*}
\mathbf{w}^{\prime} =
\begin{pmatrix}
0.580166\\
0.281524\\
0.069155\\
0.069155
\end{pmatrix} =
0.997693\cdot
\begin{pmatrix}
0.581507\\
0.282175\\
0.069315\\
\color{gr} 0.069315\color{black}
\end{pmatrix},
\end{equation*}
\begin{equation*}
\left[ \frac{{w}^{\prime}_i}{{w}^{\prime}_j} \right] =
\begin{pmatrix}
$\,\,$ 1 $\,\,$ & $\,\,$2.0608$\,\,$ & $\,\,$8.3893$\,\,$ & $\,\,$\color{gr} 8.3893\color{black} $\,\,$ \\
$\,\,$0.4852$\,\,$ & $\,\,$ 1 $\,\,$ & $\,\,$4.0709$\,\,$ & $\,\,$\color{gr} 4.0709\color{black}   $\,\,$ \\
$\,\,$0.1192$\,\,$ & $\,\,$0.2456$\,\,$ & $\,\,$ 1 $\,\,$ & $\,\,$\color{gr} \color{blue} 1\color{black}  $\,\,$ \\
$\,\,$\color{gr} 0.1192\color{black} $\,\,$ & $\,\,$\color{gr} 0.2456\color{black} $\,\,$ & $\,\,$\color{gr} \color{blue} 1\color{black} $\,\,$ & $\,\,$ 1  $\,\,$ \\
\end{pmatrix},
\end{equation*}
\end{example}
\newpage
\begin{example}
\begin{equation*}
\mathbf{A} =
\begin{pmatrix}
$\,\,$ 1 $\,\,$ & $\,\,$4$\,\,$ & $\,\,$6$\,\,$ & $\,\,$8 $\,\,$ \\
$\,\,$ 1/4$\,\,$ & $\,\,$ 1 $\,\,$ & $\,\,$8$\,\,$ & $\,\,$4 $\,\,$ \\
$\,\,$ 1/6$\,\,$ & $\,\,$ 1/8$\,\,$ & $\,\,$ 1 $\,\,$ & $\,\,$1 $\,\,$ \\
$\,\,$ 1/8$\,\,$ & $\,\,$ 1/4$\,\,$ & $\,\,$ 1 $\,\,$ & $\,\,$ 1  $\,\,$ \\
\end{pmatrix},
\qquad
\lambda_{\max} =
4.2512,
\qquad
CR = 0.0947
\end{equation*}

\begin{equation*}
\mathbf{w}^{cos} =
\begin{pmatrix}
0.575958\\
0.290585\\
0.067523\\
\color{red} 0.065934\color{black}
\end{pmatrix}\end{equation*}
\begin{equation*}
\left[ \frac{{w}^{cos}_i}{{w}^{cos}_j} \right] =
\begin{pmatrix}
$\,\,$ 1 $\,\,$ & $\,\,$1.9821$\,\,$ & $\,\,$8.5298$\,\,$ & $\,\,$\color{red} 8.7354\color{black} $\,\,$ \\
$\,\,$0.5045$\,\,$ & $\,\,$ 1 $\,\,$ & $\,\,$4.3035$\,\,$ & $\,\,$\color{red} 4.4072\color{black}   $\,\,$ \\
$\,\,$0.1172$\,\,$ & $\,\,$0.2324$\,\,$ & $\,\,$ 1 $\,\,$ & $\,\,$\color{red} 1.0241\color{black}  $\,\,$ \\
$\,\,$\color{red} 0.1145\color{black} $\,\,$ & $\,\,$\color{red} 0.2269\color{black} $\,\,$ & $\,\,$\color{red} 0.9765\color{black} $\,\,$ & $\,\,$ 1  $\,\,$ \\
\end{pmatrix},
\end{equation*}

\begin{equation*}
\mathbf{w}^{\prime} =
\begin{pmatrix}
0.575044\\
0.290123\\
0.067416\\
0.067416
\end{pmatrix} =
0.998413\cdot
\begin{pmatrix}
0.575958\\
0.290585\\
0.067523\\
\color{gr} 0.067523\color{black}
\end{pmatrix},
\end{equation*}
\begin{equation*}
\left[ \frac{{w}^{\prime}_i}{{w}^{\prime}_j} \right] =
\begin{pmatrix}
$\,\,$ 1 $\,\,$ & $\,\,$1.9821$\,\,$ & $\,\,$8.5298$\,\,$ & $\,\,$\color{gr} 8.5298\color{black} $\,\,$ \\
$\,\,$0.5045$\,\,$ & $\,\,$ 1 $\,\,$ & $\,\,$4.3035$\,\,$ & $\,\,$\color{gr} 4.3035\color{black}   $\,\,$ \\
$\,\,$0.1172$\,\,$ & $\,\,$0.2324$\,\,$ & $\,\,$ 1 $\,\,$ & $\,\,$\color{gr} \color{blue} 1\color{black}  $\,\,$ \\
$\,\,$\color{gr} 0.1172\color{black} $\,\,$ & $\,\,$\color{gr} 0.2324\color{black} $\,\,$ & $\,\,$\color{gr} \color{blue} 1\color{black} $\,\,$ & $\,\,$ 1  $\,\,$ \\
\end{pmatrix},
\end{equation*}
\end{example}
\newpage
\begin{example}
\begin{equation*}
\mathbf{A} =
\begin{pmatrix}
$\,\,$ 1 $\,\,$ & $\,\,$4$\,\,$ & $\,\,$6$\,\,$ & $\,\,$9 $\,\,$ \\
$\,\,$ 1/4$\,\,$ & $\,\,$ 1 $\,\,$ & $\,\,$6$\,\,$ & $\,\,$4 $\,\,$ \\
$\,\,$ 1/6$\,\,$ & $\,\,$ 1/6$\,\,$ & $\,\,$ 1 $\,\,$ & $\,\,$1 $\,\,$ \\
$\,\,$ 1/9$\,\,$ & $\,\,$ 1/4$\,\,$ & $\,\,$ 1 $\,\,$ & $\,\,$ 1  $\,\,$ \\
\end{pmatrix},
\qquad
\lambda_{\max} =
4.1664,
\qquad
CR = 0.0627
\end{equation*}

\begin{equation*}
\mathbf{w}^{cos} =
\begin{pmatrix}
0.596913\\
0.267698\\
0.070470\\
\color{red} 0.064919\color{black}
\end{pmatrix}\end{equation*}
\begin{equation*}
\left[ \frac{{w}^{cos}_i}{{w}^{cos}_j} \right] =
\begin{pmatrix}
$\,\,$ 1 $\,\,$ & $\,\,$2.2298$\,\,$ & $\,\,$8.4705$\,\,$ & $\,\,$\color{red} 9.1947\color{black} $\,\,$ \\
$\,\,$0.4485$\,\,$ & $\,\,$ 1 $\,\,$ & $\,\,$3.7988$\,\,$ & $\,\,$\color{red} 4.1235\color{black}   $\,\,$ \\
$\,\,$0.1181$\,\,$ & $\,\,$0.2632$\,\,$ & $\,\,$ 1 $\,\,$ & $\,\,$\color{red} 1.0855\color{black}  $\,\,$ \\
$\,\,$\color{red} 0.1088\color{black} $\,\,$ & $\,\,$\color{red} 0.2425\color{black} $\,\,$ & $\,\,$\color{red} 0.9212\color{black} $\,\,$ & $\,\,$ 1  $\,\,$ \\
\end{pmatrix},
\end{equation*}

\begin{equation*}
\mathbf{w}^{\prime} =
\begin{pmatrix}
0.596076\\
0.267322\\
0.070371\\
0.066231
\end{pmatrix} =
0.998598\cdot
\begin{pmatrix}
0.596913\\
0.267698\\
0.070470\\
\color{gr} 0.066324\color{black}
\end{pmatrix},
\end{equation*}
\begin{equation*}
\left[ \frac{{w}^{\prime}_i}{{w}^{\prime}_j} \right] =
\begin{pmatrix}
$\,\,$ 1 $\,\,$ & $\,\,$2.2298$\,\,$ & $\,\,$8.4705$\,\,$ & $\,\,$\color{gr} \color{blue} 9\color{black} $\,\,$ \\
$\,\,$0.4485$\,\,$ & $\,\,$ 1 $\,\,$ & $\,\,$3.7988$\,\,$ & $\,\,$\color{gr} 4.0362\color{black}   $\,\,$ \\
$\,\,$0.1181$\,\,$ & $\,\,$0.2632$\,\,$ & $\,\,$ 1 $\,\,$ & $\,\,$\color{gr} 1.0625\color{black}  $\,\,$ \\
$\,\,$\color{gr} \color{blue}  1/9\color{black} $\,\,$ & $\,\,$\color{gr} 0.2478\color{black} $\,\,$ & $\,\,$\color{gr} 0.9412\color{black} $\,\,$ & $\,\,$ 1  $\,\,$ \\
\end{pmatrix},
\end{equation*}
\end{example}
\newpage
\begin{example}
\begin{equation*}
\mathbf{A} =
\begin{pmatrix}
$\,\,$ 1 $\,\,$ & $\,\,$4$\,\,$ & $\,\,$6$\,\,$ & $\,\,$9 $\,\,$ \\
$\,\,$ 1/4$\,\,$ & $\,\,$ 1 $\,\,$ & $\,\,$7$\,\,$ & $\,\,$4 $\,\,$ \\
$\,\,$ 1/6$\,\,$ & $\,\,$ 1/7$\,\,$ & $\,\,$ 1 $\,\,$ & $\,\,$1 $\,\,$ \\
$\,\,$ 1/9$\,\,$ & $\,\,$ 1/4$\,\,$ & $\,\,$ 1 $\,\,$ & $\,\,$ 1  $\,\,$ \\
\end{pmatrix},
\qquad
\lambda_{\max} =
4.2065,
\qquad
CR = 0.0779
\end{equation*}

\begin{equation*}
\mathbf{w}^{cos} =
\begin{pmatrix}
0.590238\\
0.277855\\
0.068250\\
\color{red} 0.063657\color{black}
\end{pmatrix}\end{equation*}
\begin{equation*}
\left[ \frac{{w}^{cos}_i}{{w}^{cos}_j} \right] =
\begin{pmatrix}
$\,\,$ 1 $\,\,$ & $\,\,$2.1243$\,\,$ & $\,\,$8.6482$\,\,$ & $\,\,$\color{red} 9.2721\color{black} $\,\,$ \\
$\,\,$0.4708$\,\,$ & $\,\,$ 1 $\,\,$ & $\,\,$4.0711$\,\,$ & $\,\,$\color{red} 4.3648\color{black}   $\,\,$ \\
$\,\,$0.1156$\,\,$ & $\,\,$0.2456$\,\,$ & $\,\,$ 1 $\,\,$ & $\,\,$\color{red} 1.0721\color{black}  $\,\,$ \\
$\,\,$\color{red} 0.1079\color{black} $\,\,$ & $\,\,$\color{red} 0.2291\color{black} $\,\,$ & $\,\,$\color{red} 0.9327\color{black} $\,\,$ & $\,\,$ 1  $\,\,$ \\
\end{pmatrix},
\end{equation*}

\begin{equation*}
\mathbf{w}^{\prime} =
\begin{pmatrix}
0.589104\\
0.277321\\
0.068119\\
0.065456
\end{pmatrix} =
0.998079\cdot
\begin{pmatrix}
0.590238\\
0.277855\\
0.068250\\
\color{gr} 0.065582\color{black}
\end{pmatrix},
\end{equation*}
\begin{equation*}
\left[ \frac{{w}^{\prime}_i}{{w}^{\prime}_j} \right] =
\begin{pmatrix}
$\,\,$ 1 $\,\,$ & $\,\,$2.1243$\,\,$ & $\,\,$8.6482$\,\,$ & $\,\,$\color{gr} \color{blue} 9\color{black} $\,\,$ \\
$\,\,$0.4708$\,\,$ & $\,\,$ 1 $\,\,$ & $\,\,$4.0711$\,\,$ & $\,\,$\color{gr} 4.2368\color{black}   $\,\,$ \\
$\,\,$0.1156$\,\,$ & $\,\,$0.2456$\,\,$ & $\,\,$ 1 $\,\,$ & $\,\,$\color{gr} 1.0407\color{black}  $\,\,$ \\
$\,\,$\color{gr} \color{blue}  1/9\color{black} $\,\,$ & $\,\,$\color{gr} 0.2360\color{black} $\,\,$ & $\,\,$\color{gr} 0.9609\color{black} $\,\,$ & $\,\,$ 1  $\,\,$ \\
\end{pmatrix},
\end{equation*}
\end{example}
\newpage
\begin{example}
\begin{equation*}
\mathbf{A} =
\begin{pmatrix}
$\,\,$ 1 $\,\,$ & $\,\,$4$\,\,$ & $\,\,$6$\,\,$ & $\,\,$9 $\,\,$ \\
$\,\,$ 1/4$\,\,$ & $\,\,$ 1 $\,\,$ & $\,\,$8$\,\,$ & $\,\,$4 $\,\,$ \\
$\,\,$ 1/6$\,\,$ & $\,\,$ 1/8$\,\,$ & $\,\,$ 1 $\,\,$ & $\,\,$1 $\,\,$ \\
$\,\,$ 1/9$\,\,$ & $\,\,$ 1/4$\,\,$ & $\,\,$ 1 $\,\,$ & $\,\,$ 1  $\,\,$ \\
\end{pmatrix},
\qquad
\lambda_{\max} =
4.2469,
\qquad
CR = 0.0931
\end{equation*}

\begin{equation*}
\mathbf{w}^{cos} =
\begin{pmatrix}
0.584684\\
0.286315\\
0.066438\\
\color{red} 0.062563\color{black}
\end{pmatrix}\end{equation*}
\begin{equation*}
\left[ \frac{{w}^{cos}_i}{{w}^{cos}_j} \right] =
\begin{pmatrix}
$\,\,$ 1 $\,\,$ & $\,\,$2.0421$\,\,$ & $\,\,$8.8005$\,\,$ & $\,\,$\color{red} 9.3456\color{black} $\,\,$ \\
$\,\,$0.4897$\,\,$ & $\,\,$ 1 $\,\,$ & $\,\,$4.3095$\,\,$ & $\,\,$\color{red} 4.5765\color{black}   $\,\,$ \\
$\,\,$0.1136$\,\,$ & $\,\,$0.2320$\,\,$ & $\,\,$ 1 $\,\,$ & $\,\,$\color{red} 1.0619\color{black}  $\,\,$ \\
$\,\,$\color{red} 0.1070\color{black} $\,\,$ & $\,\,$\color{red} 0.2185\color{black} $\,\,$ & $\,\,$\color{red} 0.9417\color{black} $\,\,$ & $\,\,$ 1  $\,\,$ \\
\end{pmatrix},
\end{equation*}

\begin{equation*}
\mathbf{w}^{\prime} =
\begin{pmatrix}
0.583283\\
0.285629\\
0.066279\\
0.064809
\end{pmatrix} =
0.997604\cdot
\begin{pmatrix}
0.584684\\
0.286315\\
0.066438\\
\color{gr} 0.064965\color{black}
\end{pmatrix},
\end{equation*}
\begin{equation*}
\left[ \frac{{w}^{\prime}_i}{{w}^{\prime}_j} \right] =
\begin{pmatrix}
$\,\,$ 1 $\,\,$ & $\,\,$2.0421$\,\,$ & $\,\,$8.8005$\,\,$ & $\,\,$\color{gr} \color{blue} 9\color{black} $\,\,$ \\
$\,\,$0.4897$\,\,$ & $\,\,$ 1 $\,\,$ & $\,\,$4.3095$\,\,$ & $\,\,$\color{gr} 4.4072\color{black}   $\,\,$ \\
$\,\,$0.1136$\,\,$ & $\,\,$0.2320$\,\,$ & $\,\,$ 1 $\,\,$ & $\,\,$\color{gr} 1.0227\color{black}  $\,\,$ \\
$\,\,$\color{gr} \color{blue}  1/9\color{black} $\,\,$ & $\,\,$\color{gr} 0.2269\color{black} $\,\,$ & $\,\,$\color{gr} 0.9778\color{black} $\,\,$ & $\,\,$ 1  $\,\,$ \\
\end{pmatrix},
\end{equation*}
\end{example}
\newpage
\begin{example}
\begin{equation*}
\mathbf{A} =
\begin{pmatrix}
$\,\,$ 1 $\,\,$ & $\,\,$4$\,\,$ & $\,\,$6$\,\,$ & $\,\,$9 $\,\,$ \\
$\,\,$ 1/4$\,\,$ & $\,\,$ 1 $\,\,$ & $\,\,$8$\,\,$ & $\,\,$5 $\,\,$ \\
$\,\,$ 1/6$\,\,$ & $\,\,$ 1/8$\,\,$ & $\,\,$ 1 $\,\,$ & $\,\,$1 $\,\,$ \\
$\,\,$ 1/9$\,\,$ & $\,\,$ 1/5$\,\,$ & $\,\,$ 1 $\,\,$ & $\,\,$ 1  $\,\,$ \\
\end{pmatrix},
\qquad
\lambda_{\max} =
4.2500,
\qquad
CR = 0.0942
\end{equation*}

\begin{equation*}
\mathbf{w}^{cos} =
\begin{pmatrix}
0.576159\\
0.298809\\
0.065477\\
\color{red} 0.059555\color{black}
\end{pmatrix}\end{equation*}
\begin{equation*}
\left[ \frac{{w}^{cos}_i}{{w}^{cos}_j} \right] =
\begin{pmatrix}
$\,\,$ 1 $\,\,$ & $\,\,$1.9282$\,\,$ & $\,\,$8.7995$\,\,$ & $\,\,$\color{red} 9.6743\color{black} $\,\,$ \\
$\,\,$0.5186$\,\,$ & $\,\,$ 1 $\,\,$ & $\,\,$4.5636$\,\,$ & $\,\,$\color{red} 5.0173\color{black}   $\,\,$ \\
$\,\,$0.1136$\,\,$ & $\,\,$0.2191$\,\,$ & $\,\,$ 1 $\,\,$ & $\,\,$\color{red} 1.0994\color{black}  $\,\,$ \\
$\,\,$\color{red} 0.1034\color{black} $\,\,$ & $\,\,$\color{red} 0.1993\color{black} $\,\,$ & $\,\,$\color{red} 0.9096\color{black} $\,\,$ & $\,\,$ 1  $\,\,$ \\
\end{pmatrix},
\end{equation*}

\begin{equation*}
\mathbf{w}^{\prime} =
\begin{pmatrix}
0.576040\\
0.298748\\
0.065463\\
0.059750
\end{pmatrix} =
0.999794\cdot
\begin{pmatrix}
0.576159\\
0.298809\\
0.065477\\
\color{gr} 0.059762\color{black}
\end{pmatrix},
\end{equation*}
\begin{equation*}
\left[ \frac{{w}^{\prime}_i}{{w}^{\prime}_j} \right] =
\begin{pmatrix}
$\,\,$ 1 $\,\,$ & $\,\,$1.9282$\,\,$ & $\,\,$8.7995$\,\,$ & $\,\,$\color{gr} 9.6409\color{black} $\,\,$ \\
$\,\,$0.5186$\,\,$ & $\,\,$ 1 $\,\,$ & $\,\,$4.5636$\,\,$ & $\,\,$\color{gr} \color{blue} 5\color{black}   $\,\,$ \\
$\,\,$0.1136$\,\,$ & $\,\,$0.2191$\,\,$ & $\,\,$ 1 $\,\,$ & $\,\,$\color{gr} 1.0956\color{black}  $\,\,$ \\
$\,\,$\color{gr} 0.1037\color{black} $\,\,$ & $\,\,$\color{gr} \color{blue}  1/5\color{black} $\,\,$ & $\,\,$\color{gr} 0.9127\color{black} $\,\,$ & $\,\,$ 1  $\,\,$ \\
\end{pmatrix},
\end{equation*}
\end{example}
\newpage
\begin{example}
\begin{equation*}
\mathbf{A} =
\begin{pmatrix}
$\,\,$ 1 $\,\,$ & $\,\,$4$\,\,$ & $\,\,$7$\,\,$ & $\,\,$9 $\,\,$ \\
$\,\,$ 1/4$\,\,$ & $\,\,$ 1 $\,\,$ & $\,\,$6$\,\,$ & $\,\,$4 $\,\,$ \\
$\,\,$ 1/7$\,\,$ & $\,\,$ 1/6$\,\,$ & $\,\,$ 1 $\,\,$ & $\,\,$1 $\,\,$ \\
$\,\,$ 1/9$\,\,$ & $\,\,$ 1/4$\,\,$ & $\,\,$ 1 $\,\,$ & $\,\,$ 1  $\,\,$ \\
\end{pmatrix},
\qquad
\lambda_{\max} =
4.1330,
\qquad
CR = 0.0501
\end{equation*}

\begin{equation*}
\mathbf{w}^{cos} =
\begin{pmatrix}
0.610281\\
0.260283\\
0.065586\\
\color{red} 0.063850\color{black}
\end{pmatrix}\end{equation*}
\begin{equation*}
\left[ \frac{{w}^{cos}_i}{{w}^{cos}_j} \right] =
\begin{pmatrix}
$\,\,$ 1 $\,\,$ & $\,\,$2.3447$\,\,$ & $\,\,$9.3051$\,\,$ & $\,\,$\color{red} 9.5580\color{black} $\,\,$ \\
$\,\,$0.4265$\,\,$ & $\,\,$ 1 $\,\,$ & $\,\,$3.9686$\,\,$ & $\,\,$\color{red} 4.0765\color{black}   $\,\,$ \\
$\,\,$0.1075$\,\,$ & $\,\,$0.2520$\,\,$ & $\,\,$ 1 $\,\,$ & $\,\,$\color{red} 1.0272\color{black}  $\,\,$ \\
$\,\,$\color{red} 0.1046\color{black} $\,\,$ & $\,\,$\color{red} 0.2453\color{black} $\,\,$ & $\,\,$\color{red} 0.9735\color{black} $\,\,$ & $\,\,$ 1  $\,\,$ \\
\end{pmatrix},
\end{equation*}

\begin{equation*}
\mathbf{w}^{\prime} =
\begin{pmatrix}
0.609537\\
0.259966\\
0.065506\\
0.064991
\end{pmatrix} =
0.998781\cdot
\begin{pmatrix}
0.610281\\
0.260283\\
0.065586\\
\color{gr} 0.065071\color{black}
\end{pmatrix},
\end{equation*}
\begin{equation*}
\left[ \frac{{w}^{\prime}_i}{{w}^{\prime}_j} \right] =
\begin{pmatrix}
$\,\,$ 1 $\,\,$ & $\,\,$2.3447$\,\,$ & $\,\,$9.3051$\,\,$ & $\,\,$\color{gr} 9.3787\color{black} $\,\,$ \\
$\,\,$0.4265$\,\,$ & $\,\,$ 1 $\,\,$ & $\,\,$3.9686$\,\,$ & $\,\,$\color{gr} \color{blue} 4\color{black}   $\,\,$ \\
$\,\,$0.1075$\,\,$ & $\,\,$0.2520$\,\,$ & $\,\,$ 1 $\,\,$ & $\,\,$\color{gr} 1.0079\color{black}  $\,\,$ \\
$\,\,$\color{gr} 0.1066\color{black} $\,\,$ & $\,\,$\color{gr} \color{blue}  1/4\color{black} $\,\,$ & $\,\,$\color{gr} 0.9921\color{black} $\,\,$ & $\,\,$ 1  $\,\,$ \\
\end{pmatrix},
\end{equation*}
\end{example}
\newpage
\begin{example}
\begin{equation*}
\mathbf{A} =
\begin{pmatrix}
$\,\,$ 1 $\,\,$ & $\,\,$4$\,\,$ & $\,\,$7$\,\,$ & $\,\,$9 $\,\,$ \\
$\,\,$ 1/4$\,\,$ & $\,\,$ 1 $\,\,$ & $\,\,$7$\,\,$ & $\,\,$4 $\,\,$ \\
$\,\,$ 1/7$\,\,$ & $\,\,$ 1/7$\,\,$ & $\,\,$ 1 $\,\,$ & $\,\,$1 $\,\,$ \\
$\,\,$ 1/9$\,\,$ & $\,\,$ 1/4$\,\,$ & $\,\,$ 1 $\,\,$ & $\,\,$ 1  $\,\,$ \\
\end{pmatrix},
\qquad
\lambda_{\max} =
4.1694,
\qquad
CR = 0.0639
\end{equation*}

\begin{equation*}
\mathbf{w}^{cos} =
\begin{pmatrix}
0.603110\\
0.270598\\
0.063523\\
\color{red} 0.062769\color{black}
\end{pmatrix}\end{equation*}
\begin{equation*}
\left[ \frac{{w}^{cos}_i}{{w}^{cos}_j} \right] =
\begin{pmatrix}
$\,\,$ 1 $\,\,$ & $\,\,$2.2288$\,\,$ & $\,\,$9.4944$\,\,$ & $\,\,$\color{red} 9.6085\color{black} $\,\,$ \\
$\,\,$0.4487$\,\,$ & $\,\,$ 1 $\,\,$ & $\,\,$4.2599$\,\,$ & $\,\,$\color{red} 4.3110\color{black}   $\,\,$ \\
$\,\,$0.1053$\,\,$ & $\,\,$0.2347$\,\,$ & $\,\,$ 1 $\,\,$ & $\,\,$\color{red} 1.0120\color{black}  $\,\,$ \\
$\,\,$\color{red} 0.1041\color{black} $\,\,$ & $\,\,$\color{red} 0.2320\color{black} $\,\,$ & $\,\,$\color{red} 0.9881\color{black} $\,\,$ & $\,\,$ 1  $\,\,$ \\
\end{pmatrix},
\end{equation*}

\begin{equation*}
\mathbf{w}^{\prime} =
\begin{pmatrix}
0.602656\\
0.270394\\
0.063475\\
0.063475
\end{pmatrix} =
0.999246\cdot
\begin{pmatrix}
0.603110\\
0.270598\\
0.063523\\
\color{gr} 0.063523\color{black}
\end{pmatrix},
\end{equation*}
\begin{equation*}
\left[ \frac{{w}^{\prime}_i}{{w}^{\prime}_j} \right] =
\begin{pmatrix}
$\,\,$ 1 $\,\,$ & $\,\,$2.2288$\,\,$ & $\,\,$9.4944$\,\,$ & $\,\,$\color{gr} 9.4944\color{black} $\,\,$ \\
$\,\,$0.4487$\,\,$ & $\,\,$ 1 $\,\,$ & $\,\,$4.2599$\,\,$ & $\,\,$\color{gr} 4.2599\color{black}   $\,\,$ \\
$\,\,$0.1053$\,\,$ & $\,\,$0.2347$\,\,$ & $\,\,$ 1 $\,\,$ & $\,\,$\color{gr} \color{blue} 1\color{black}  $\,\,$ \\
$\,\,$\color{gr} 0.1053\color{black} $\,\,$ & $\,\,$\color{gr} 0.2347\color{black} $\,\,$ & $\,\,$\color{gr} \color{blue} 1\color{black} $\,\,$ & $\,\,$ 1  $\,\,$ \\
\end{pmatrix},
\end{equation*}
\end{example}
\newpage
\begin{example}
\begin{equation*}
\mathbf{A} =
\begin{pmatrix}
$\,\,$ 1 $\,\,$ & $\,\,$4$\,\,$ & $\,\,$7$\,\,$ & $\,\,$9 $\,\,$ \\
$\,\,$ 1/4$\,\,$ & $\,\,$ 1 $\,\,$ & $\,\,$8$\,\,$ & $\,\,$4 $\,\,$ \\
$\,\,$ 1/7$\,\,$ & $\,\,$ 1/8$\,\,$ & $\,\,$ 1 $\,\,$ & $\,\,$1 $\,\,$ \\
$\,\,$ 1/9$\,\,$ & $\,\,$ 1/4$\,\,$ & $\,\,$ 1 $\,\,$ & $\,\,$ 1  $\,\,$ \\
\end{pmatrix},
\qquad
\lambda_{\max} =
4.2064,
\qquad
CR = 0.0778
\end{equation*}

\begin{equation*}
\mathbf{w}^{cos} =
\begin{pmatrix}
0.596983\\
0.279370\\
0.061831\\
\color{red} 0.061816\color{black}
\end{pmatrix}\end{equation*}
\begin{equation*}
\left[ \frac{{w}^{cos}_i}{{w}^{cos}_j} \right] =
\begin{pmatrix}
$\,\,$ 1 $\,\,$ & $\,\,$2.1369$\,\,$ & $\,\,$9.6551$\,\,$ & $\,\,$\color{red} 9.6575\color{black} $\,\,$ \\
$\,\,$0.4680$\,\,$ & $\,\,$ 1 $\,\,$ & $\,\,$4.5183$\,\,$ & $\,\,$\color{red} 4.5194\color{black}   $\,\,$ \\
$\,\,$0.1036$\,\,$ & $\,\,$0.2213$\,\,$ & $\,\,$ 1 $\,\,$ & $\,\,$\color{red} 1.0002\color{black}  $\,\,$ \\
$\,\,$\color{red} 0.1035\color{black} $\,\,$ & $\,\,$\color{red} 0.2213\color{black} $\,\,$ & $\,\,$\color{red} 0.9998\color{black} $\,\,$ & $\,\,$ 1  $\,\,$ \\
\end{pmatrix},
\end{equation*}

\begin{equation*}
\mathbf{w}^{\prime} =
\begin{pmatrix}
0.596974\\
0.279366\\
0.061830\\
0.061830
\end{pmatrix} =
0.999985\cdot
\begin{pmatrix}
0.596983\\
0.279370\\
0.061831\\
\color{gr} 0.061831\color{black}
\end{pmatrix},
\end{equation*}
\begin{equation*}
\left[ \frac{{w}^{\prime}_i}{{w}^{\prime}_j} \right] =
\begin{pmatrix}
$\,\,$ 1 $\,\,$ & $\,\,$2.1369$\,\,$ & $\,\,$9.6551$\,\,$ & $\,\,$\color{gr} 9.6551\color{black} $\,\,$ \\
$\,\,$0.4680$\,\,$ & $\,\,$ 1 $\,\,$ & $\,\,$4.5183$\,\,$ & $\,\,$\color{gr} 4.5183\color{black}   $\,\,$ \\
$\,\,$0.1036$\,\,$ & $\,\,$0.2213$\,\,$ & $\,\,$ 1 $\,\,$ & $\,\,$\color{gr} \color{blue} 1\color{black}  $\,\,$ \\
$\,\,$\color{gr} 0.1036\color{black} $\,\,$ & $\,\,$\color{gr} 0.2213\color{black} $\,\,$ & $\,\,$\color{gr} \color{blue} 1\color{black} $\,\,$ & $\,\,$ 1  $\,\,$ \\
\end{pmatrix},
\end{equation*}
\end{example}
\newpage
\begin{example}
\begin{equation*}
\mathbf{A} =
\begin{pmatrix}
$\,\,$ 1 $\,\,$ & $\,\,$4$\,\,$ & $\,\,$7$\,\,$ & $\,\,$9 $\,\,$ \\
$\,\,$ 1/4$\,\,$ & $\,\,$ 1 $\,\,$ & $\,\,$9$\,\,$ & $\,\,$5 $\,\,$ \\
$\,\,$ 1/7$\,\,$ & $\,\,$ 1/9$\,\,$ & $\,\,$ 1 $\,\,$ & $\,\,$1 $\,\,$ \\
$\,\,$ 1/9$\,\,$ & $\,\,$ 1/5$\,\,$ & $\,\,$ 1 $\,\,$ & $\,\,$ 1  $\,\,$ \\
\end{pmatrix},
\qquad
\lambda_{\max} =
4.2430,
\qquad
CR = 0.0916
\end{equation*}

\begin{equation*}
\mathbf{w}^{cos} =
\begin{pmatrix}
0.583147\\
0.299427\\
0.059469\\
\color{red} 0.057958\color{black}
\end{pmatrix}\end{equation*}
\begin{equation*}
\left[ \frac{{w}^{cos}_i}{{w}^{cos}_j} \right] =
\begin{pmatrix}
$\,\,$ 1 $\,\,$ & $\,\,$1.9475$\,\,$ & $\,\,$9.8060$\,\,$ & $\,\,$\color{red} 10.0616\color{black} $\,\,$ \\
$\,\,$0.5135$\,\,$ & $\,\,$ 1 $\,\,$ & $\,\,$5.0350$\,\,$ & $\,\,$\color{red} 5.1663\color{black}   $\,\,$ \\
$\,\,$0.1020$\,\,$ & $\,\,$0.1986$\,\,$ & $\,\,$ 1 $\,\,$ & $\,\,$\color{red} 1.0261\color{black}  $\,\,$ \\
$\,\,$\color{red} 0.0994\color{black} $\,\,$ & $\,\,$\color{red} 0.1936\color{black} $\,\,$ & $\,\,$\color{red} 0.9746\color{black} $\,\,$ & $\,\,$ 1  $\,\,$ \\
\end{pmatrix},
\end{equation*}

\begin{equation*}
\mathbf{w}^{\prime} =
\begin{pmatrix}
0.582267\\
0.298975\\
0.059379\\
0.059379
\end{pmatrix} =
0.998491\cdot
\begin{pmatrix}
0.583147\\
0.299427\\
0.059469\\
\color{gr} 0.059469\color{black}
\end{pmatrix},
\end{equation*}
\begin{equation*}
\left[ \frac{{w}^{\prime}_i}{{w}^{\prime}_j} \right] =
\begin{pmatrix}
$\,\,$ 1 $\,\,$ & $\,\,$1.9475$\,\,$ & $\,\,$9.8060$\,\,$ & $\,\,$\color{gr} 9.8060\color{black} $\,\,$ \\
$\,\,$0.5135$\,\,$ & $\,\,$ 1 $\,\,$ & $\,\,$5.0350$\,\,$ & $\,\,$\color{gr} 5.0350\color{black}   $\,\,$ \\
$\,\,$0.1020$\,\,$ & $\,\,$0.1986$\,\,$ & $\,\,$ 1 $\,\,$ & $\,\,$\color{gr} \color{blue} 1\color{black}  $\,\,$ \\
$\,\,$\color{gr} 0.1020\color{black} $\,\,$ & $\,\,$\color{gr} 0.1986\color{black} $\,\,$ & $\,\,$\color{gr} \color{blue} 1\color{black} $\,\,$ & $\,\,$ 1  $\,\,$ \\
\end{pmatrix},
\end{equation*}
\end{example}
\newpage
\begin{example}
\begin{equation*}
\mathbf{A} =
\begin{pmatrix}
$\,\,$ 1 $\,\,$ & $\,\,$5$\,\,$ & $\,\,$2$\,\,$ & $\,\,$6 $\,\,$ \\
$\,\,$ 1/5$\,\,$ & $\,\,$ 1 $\,\,$ & $\,\,$2$\,\,$ & $\,\,$2 $\,\,$ \\
$\,\,$ 1/2$\,\,$ & $\,\,$ 1/2$\,\,$ & $\,\,$ 1 $\,\,$ & $\,\,$2 $\,\,$ \\
$\,\,$ 1/6$\,\,$ & $\,\,$ 1/2$\,\,$ & $\,\,$ 1/2$\,\,$ & $\,\,$ 1  $\,\,$ \\
\end{pmatrix},
\qquad
\lambda_{\max} =
4.2277,
\qquad
CR = 0.0859
\end{equation*}

\begin{equation*}
\mathbf{w}^{cos} =
\begin{pmatrix}
0.527572\\
0.205814\\
0.180233\\
\color{red} 0.086382\color{black}
\end{pmatrix}\end{equation*}
\begin{equation*}
\left[ \frac{{w}^{cos}_i}{{w}^{cos}_j} \right] =
\begin{pmatrix}
$\,\,$ 1 $\,\,$ & $\,\,$2.5633$\,\,$ & $\,\,$2.9272$\,\,$ & $\,\,$\color{red} 6.1074\color{black} $\,\,$ \\
$\,\,$0.3901$\,\,$ & $\,\,$ 1 $\,\,$ & $\,\,$1.1419$\,\,$ & $\,\,$\color{red} 2.3826\color{black}   $\,\,$ \\
$\,\,$0.3416$\,\,$ & $\,\,$0.8757$\,\,$ & $\,\,$ 1 $\,\,$ & $\,\,$\color{red} 2.0865\color{black}  $\,\,$ \\
$\,\,$\color{red} 0.1637\color{black} $\,\,$ & $\,\,$\color{red} 0.4197\color{black} $\,\,$ & $\,\,$\color{red} 0.4793\color{black} $\,\,$ & $\,\,$ 1  $\,\,$ \\
\end{pmatrix},
\end{equation*}

\begin{equation*}
\mathbf{w}^{\prime} =
\begin{pmatrix}
0.526757\\
0.205496\\
0.179954\\
0.087793
\end{pmatrix} =
0.998456\cdot
\begin{pmatrix}
0.527572\\
0.205814\\
0.180233\\
\color{gr} 0.087929\color{black}
\end{pmatrix},
\end{equation*}
\begin{equation*}
\left[ \frac{{w}^{\prime}_i}{{w}^{\prime}_j} \right] =
\begin{pmatrix}
$\,\,$ 1 $\,\,$ & $\,\,$2.5633$\,\,$ & $\,\,$2.9272$\,\,$ & $\,\,$\color{gr} \color{blue} 6\color{black} $\,\,$ \\
$\,\,$0.3901$\,\,$ & $\,\,$ 1 $\,\,$ & $\,\,$1.1419$\,\,$ & $\,\,$\color{gr} 2.3407\color{black}   $\,\,$ \\
$\,\,$0.3416$\,\,$ & $\,\,$0.8757$\,\,$ & $\,\,$ 1 $\,\,$ & $\,\,$\color{gr} 2.0498\color{black}  $\,\,$ \\
$\,\,$\color{gr} \color{blue}  1/6\color{black} $\,\,$ & $\,\,$\color{gr} 0.4272\color{black} $\,\,$ & $\,\,$\color{gr} 0.4879\color{black} $\,\,$ & $\,\,$ 1  $\,\,$ \\
\end{pmatrix},
\end{equation*}
\end{example}
\newpage
\begin{example}
\begin{equation*}
\mathbf{A} =
\begin{pmatrix}
$\,\,$ 1 $\,\,$ & $\,\,$5$\,\,$ & $\,\,$2$\,\,$ & $\,\,$8 $\,\,$ \\
$\,\,$ 1/5$\,\,$ & $\,\,$ 1 $\,\,$ & $\,\,$2$\,\,$ & $\,\,$3 $\,\,$ \\
$\,\,$ 1/2$\,\,$ & $\,\,$ 1/2$\,\,$ & $\,\,$ 1 $\,\,$ & $\,\,$3 $\,\,$ \\
$\,\,$ 1/8$\,\,$ & $\,\,$ 1/3$\,\,$ & $\,\,$ 1/3$\,\,$ & $\,\,$ 1  $\,\,$ \\
\end{pmatrix},
\qquad
\lambda_{\max} =
4.2311,
\qquad
CR = 0.0871
\end{equation*}

\begin{equation*}
\mathbf{w}^{cos} =
\begin{pmatrix}
0.534531\\
0.214727\\
0.188530\\
\color{red} 0.062212\color{black}
\end{pmatrix}\end{equation*}
\begin{equation*}
\left[ \frac{{w}^{cos}_i}{{w}^{cos}_j} \right] =
\begin{pmatrix}
$\,\,$ 1 $\,\,$ & $\,\,$2.4894$\,\,$ & $\,\,$2.8353$\,\,$ & $\,\,$\color{red} 8.5921\color{black} $\,\,$ \\
$\,\,$0.4017$\,\,$ & $\,\,$ 1 $\,\,$ & $\,\,$1.1390$\,\,$ & $\,\,$\color{red} 3.4515\color{black}   $\,\,$ \\
$\,\,$0.3527$\,\,$ & $\,\,$0.8780$\,\,$ & $\,\,$ 1 $\,\,$ & $\,\,$\color{red} 3.0304\color{black}  $\,\,$ \\
$\,\,$\color{red} 0.1164\color{black} $\,\,$ & $\,\,$\color{red} 0.2897\color{black} $\,\,$ & $\,\,$\color{red} 0.3300\color{black} $\,\,$ & $\,\,$ 1  $\,\,$ \\
\end{pmatrix},
\end{equation*}

\begin{equation*}
\mathbf{w}^{\prime} =
\begin{pmatrix}
0.534194\\
0.214592\\
0.188411\\
0.062804
\end{pmatrix} =
0.999369\cdot
\begin{pmatrix}
0.534531\\
0.214727\\
0.188530\\
\color{gr} 0.062843\color{black}
\end{pmatrix},
\end{equation*}
\begin{equation*}
\left[ \frac{{w}^{\prime}_i}{{w}^{\prime}_j} \right] =
\begin{pmatrix}
$\,\,$ 1 $\,\,$ & $\,\,$2.4894$\,\,$ & $\,\,$2.8353$\,\,$ & $\,\,$\color{gr} 8.5058\color{black} $\,\,$ \\
$\,\,$0.4017$\,\,$ & $\,\,$ 1 $\,\,$ & $\,\,$1.1390$\,\,$ & $\,\,$\color{gr} 3.4169\color{black}   $\,\,$ \\
$\,\,$0.3527$\,\,$ & $\,\,$0.8780$\,\,$ & $\,\,$ 1 $\,\,$ & $\,\,$\color{gr} \color{blue} 3\color{black}  $\,\,$ \\
$\,\,$\color{gr} 0.1176\color{black} $\,\,$ & $\,\,$\color{gr} 0.2927\color{black} $\,\,$ & $\,\,$\color{gr} \color{blue}  1/3\color{black} $\,\,$ & $\,\,$ 1  $\,\,$ \\
\end{pmatrix},
\end{equation*}
\end{example}
\newpage
\begin{example}
\begin{equation*}
\mathbf{A} =
\begin{pmatrix}
$\,\,$ 1 $\,\,$ & $\,\,$5$\,\,$ & $\,\,$2$\,\,$ & $\,\,$9 $\,\,$ \\
$\,\,$ 1/5$\,\,$ & $\,\,$ 1 $\,\,$ & $\,\,$2$\,\,$ & $\,\,$3 $\,\,$ \\
$\,\,$ 1/2$\,\,$ & $\,\,$ 1/2$\,\,$ & $\,\,$ 1 $\,\,$ & $\,\,$3 $\,\,$ \\
$\,\,$ 1/9$\,\,$ & $\,\,$ 1/3$\,\,$ & $\,\,$ 1/3$\,\,$ & $\,\,$ 1  $\,\,$ \\
\end{pmatrix},
\qquad
\lambda_{\max} =
4.2277,
\qquad
CR = 0.0859
\end{equation*}

\begin{equation*}
\mathbf{w}^{cos} =
\begin{pmatrix}
0.543005\\
0.212036\\
0.185655\\
\color{red} 0.059304\color{black}
\end{pmatrix}\end{equation*}
\begin{equation*}
\left[ \frac{{w}^{cos}_i}{{w}^{cos}_j} \right] =
\begin{pmatrix}
$\,\,$ 1 $\,\,$ & $\,\,$2.5609$\,\,$ & $\,\,$2.9248$\,\,$ & $\,\,$\color{red} 9.1563\color{black} $\,\,$ \\
$\,\,$0.3905$\,\,$ & $\,\,$ 1 $\,\,$ & $\,\,$1.1421$\,\,$ & $\,\,$\color{red} 3.5754\color{black}   $\,\,$ \\
$\,\,$0.3419$\,\,$ & $\,\,$0.8756$\,\,$ & $\,\,$ 1 $\,\,$ & $\,\,$\color{red} 3.1306\color{black}  $\,\,$ \\
$\,\,$\color{red} 0.1092\color{black} $\,\,$ & $\,\,$\color{red} 0.2797\color{black} $\,\,$ & $\,\,$\color{red} 0.3194\color{black} $\,\,$ & $\,\,$ 1  $\,\,$ \\
\end{pmatrix},
\end{equation*}

\begin{equation*}
\mathbf{w}^{\prime} =
\begin{pmatrix}
0.542446\\
0.211818\\
0.185464\\
0.060272
\end{pmatrix} =
0.998971\cdot
\begin{pmatrix}
0.543005\\
0.212036\\
0.185655\\
\color{gr} 0.060334\color{black}
\end{pmatrix},
\end{equation*}
\begin{equation*}
\left[ \frac{{w}^{\prime}_i}{{w}^{\prime}_j} \right] =
\begin{pmatrix}
$\,\,$ 1 $\,\,$ & $\,\,$2.5609$\,\,$ & $\,\,$2.9248$\,\,$ & $\,\,$\color{gr} \color{blue} 9\color{black} $\,\,$ \\
$\,\,$0.3905$\,\,$ & $\,\,$ 1 $\,\,$ & $\,\,$1.1421$\,\,$ & $\,\,$\color{gr} 3.5144\color{black}   $\,\,$ \\
$\,\,$0.3419$\,\,$ & $\,\,$0.8756$\,\,$ & $\,\,$ 1 $\,\,$ & $\,\,$\color{gr} 3.0771\color{black}  $\,\,$ \\
$\,\,$\color{gr} \color{blue}  1/9\color{black} $\,\,$ & $\,\,$\color{gr} 0.2845\color{black} $\,\,$ & $\,\,$\color{gr} 0.3250\color{black} $\,\,$ & $\,\,$ 1  $\,\,$ \\
\end{pmatrix},
\end{equation*}
\end{example}
\newpage
\begin{example}
\begin{equation*}
\mathbf{A} =
\begin{pmatrix}
$\,\,$ 1 $\,\,$ & $\,\,$5$\,\,$ & $\,\,$2$\,\,$ & $\,\,$9 $\,\,$ \\
$\,\,$ 1/5$\,\,$ & $\,\,$ 1 $\,\,$ & $\,\,$2$\,\,$ & $\,\,$4 $\,\,$ \\
$\,\,$ 1/2$\,\,$ & $\,\,$ 1/2$\,\,$ & $\,\,$ 1 $\,\,$ & $\,\,$3 $\,\,$ \\
$\,\,$ 1/9$\,\,$ & $\,\,$ 1/4$\,\,$ & $\,\,$ 1/3$\,\,$ & $\,\,$ 1  $\,\,$ \\
\end{pmatrix},
\qquad
\lambda_{\max} =
4.2316,
\qquad
CR = 0.0873
\end{equation*}

\begin{equation*}
\mathbf{w}^{cos} =
\begin{pmatrix}
0.535960\\
0.224841\\
0.183253\\
\color{red} 0.055946\color{black}
\end{pmatrix}\end{equation*}
\begin{equation*}
\left[ \frac{{w}^{cos}_i}{{w}^{cos}_j} \right] =
\begin{pmatrix}
$\,\,$ 1 $\,\,$ & $\,\,$2.3837$\,\,$ & $\,\,$2.9247$\,\,$ & $\,\,$\color{red} 9.5799\color{black} $\,\,$ \\
$\,\,$0.4195$\,\,$ & $\,\,$ 1 $\,\,$ & $\,\,$1.2269$\,\,$ & $\,\,$\color{red} 4.0189\color{black}   $\,\,$ \\
$\,\,$0.3419$\,\,$ & $\,\,$0.8150$\,\,$ & $\,\,$ 1 $\,\,$ & $\,\,$\color{red} 3.2755\color{black}  $\,\,$ \\
$\,\,$\color{red} 0.1044\color{black} $\,\,$ & $\,\,$\color{red} 0.2488\color{black} $\,\,$ & $\,\,$\color{red} 0.3053\color{black} $\,\,$ & $\,\,$ 1  $\,\,$ \\
\end{pmatrix},
\end{equation*}

\begin{equation*}
\mathbf{w}^{\prime} =
\begin{pmatrix}
0.535818\\
0.224782\\
0.183204\\
0.056196
\end{pmatrix} =
0.999736\cdot
\begin{pmatrix}
0.535960\\
0.224841\\
0.183253\\
\color{gr} 0.056210\color{black}
\end{pmatrix},
\end{equation*}
\begin{equation*}
\left[ \frac{{w}^{\prime}_i}{{w}^{\prime}_j} \right] =
\begin{pmatrix}
$\,\,$ 1 $\,\,$ & $\,\,$2.3837$\,\,$ & $\,\,$2.9247$\,\,$ & $\,\,$\color{gr} 9.5349\color{black} $\,\,$ \\
$\,\,$0.4195$\,\,$ & $\,\,$ 1 $\,\,$ & $\,\,$1.2269$\,\,$ & $\,\,$\color{gr} \color{blue} 4\color{black}   $\,\,$ \\
$\,\,$0.3419$\,\,$ & $\,\,$0.8150$\,\,$ & $\,\,$ 1 $\,\,$ & $\,\,$\color{gr} 3.2601\color{black}  $\,\,$ \\
$\,\,$\color{gr} 0.1049\color{black} $\,\,$ & $\,\,$\color{gr} \color{blue}  1/4\color{black} $\,\,$ & $\,\,$\color{gr} 0.3067\color{black} $\,\,$ & $\,\,$ 1  $\,\,$ \\
\end{pmatrix},
\end{equation*}
\end{example}
\newpage
\begin{example}
\begin{equation*}
\mathbf{A} =
\begin{pmatrix}
$\,\,$ 1 $\,\,$ & $\,\,$5$\,\,$ & $\,\,$3$\,\,$ & $\,\,$4 $\,\,$ \\
$\,\,$ 1/5$\,\,$ & $\,\,$ 1 $\,\,$ & $\,\,$1$\,\,$ & $\,\,$3 $\,\,$ \\
$\,\,$ 1/3$\,\,$ & $\,\,$ 1 $\,\,$ & $\,\,$ 1 $\,\,$ & $\,\,$2 $\,\,$ \\
$\,\,$ 1/4$\,\,$ & $\,\,$ 1/3$\,\,$ & $\,\,$ 1/2$\,\,$ & $\,\,$ 1  $\,\,$ \\
\end{pmatrix},
\qquad
\lambda_{\max} =
4.1502,
\qquad
CR = 0.0566
\end{equation*}

\begin{equation*}
\mathbf{w}^{cos} =
\begin{pmatrix}
0.537946\\
0.187821\\
\color{red} 0.178342\color{black} \\
0.095892
\end{pmatrix}\end{equation*}
\begin{equation*}
\left[ \frac{{w}^{cos}_i}{{w}^{cos}_j} \right] =
\begin{pmatrix}
$\,\,$ 1 $\,\,$ & $\,\,$2.8641$\,\,$ & $\,\,$\color{red} 3.0164\color{black} $\,\,$ & $\,\,$5.6099$\,\,$ \\
$\,\,$0.3491$\,\,$ & $\,\,$ 1 $\,\,$ & $\,\,$\color{red} 1.0531\color{black} $\,\,$ & $\,\,$1.9587  $\,\,$ \\
$\,\,$\color{red} 0.3315\color{black} $\,\,$ & $\,\,$\color{red} 0.9495\color{black} $\,\,$ & $\,\,$ 1 $\,\,$ & $\,\,$\color{red} 1.8598\color{black}  $\,\,$ \\
$\,\,$0.1783$\,\,$ & $\,\,$0.5105$\,\,$ & $\,\,$\color{red} 0.5377\color{black} $\,\,$ & $\,\,$ 1  $\,\,$ \\
\end{pmatrix},
\end{equation*}

\begin{equation*}
\mathbf{w}^{\prime} =
\begin{pmatrix}
0.537423\\
0.187638\\
0.179141\\
0.095798
\end{pmatrix} =
0.999027\cdot
\begin{pmatrix}
0.537946\\
0.187821\\
\color{gr} 0.179315\color{black} \\
0.095892
\end{pmatrix},
\end{equation*}
\begin{equation*}
\left[ \frac{{w}^{\prime}_i}{{w}^{\prime}_j} \right] =
\begin{pmatrix}
$\,\,$ 1 $\,\,$ & $\,\,$2.8641$\,\,$ & $\,\,$\color{gr} \color{blue} 3\color{black} $\,\,$ & $\,\,$5.6099$\,\,$ \\
$\,\,$0.3491$\,\,$ & $\,\,$ 1 $\,\,$ & $\,\,$\color{gr} 1.0474\color{black} $\,\,$ & $\,\,$1.9587  $\,\,$ \\
$\,\,$\color{gr} \color{blue}  1/3\color{black} $\,\,$ & $\,\,$\color{gr} 0.9547\color{black} $\,\,$ & $\,\,$ 1 $\,\,$ & $\,\,$\color{gr} 1.8700\color{black}  $\,\,$ \\
$\,\,$0.1783$\,\,$ & $\,\,$0.5105$\,\,$ & $\,\,$\color{gr} 0.5348\color{black} $\,\,$ & $\,\,$ 1  $\,\,$ \\
\end{pmatrix},
\end{equation*}
\end{example}
\newpage
\begin{example}
\begin{equation*}
\mathbf{A} =
\begin{pmatrix}
$\,\,$ 1 $\,\,$ & $\,\,$5$\,\,$ & $\,\,$3$\,\,$ & $\,\,$4 $\,\,$ \\
$\,\,$ 1/5$\,\,$ & $\,\,$ 1 $\,\,$ & $\,\,$1$\,\,$ & $\,\,$4 $\,\,$ \\
$\,\,$ 1/3$\,\,$ & $\,\,$ 1 $\,\,$ & $\,\,$ 1 $\,\,$ & $\,\,$2 $\,\,$ \\
$\,\,$ 1/4$\,\,$ & $\,\,$ 1/4$\,\,$ & $\,\,$ 1/2$\,\,$ & $\,\,$ 1  $\,\,$ \\
\end{pmatrix},
\qquad
\lambda_{\max} =
4.2277,
\qquad
CR = 0.0859
\end{equation*}

\begin{equation*}
\mathbf{w}^{cos} =
\begin{pmatrix}
0.529500\\
0.205787\\
\color{red} 0.173628\color{black} \\
0.091084
\end{pmatrix}\end{equation*}
\begin{equation*}
\left[ \frac{{w}^{cos}_i}{{w}^{cos}_j} \right] =
\begin{pmatrix}
$\,\,$ 1 $\,\,$ & $\,\,$2.5730$\,\,$ & $\,\,$\color{red} 3.0496\color{black} $\,\,$ & $\,\,$5.8133$\,\,$ \\
$\,\,$0.3886$\,\,$ & $\,\,$ 1 $\,\,$ & $\,\,$\color{red} 1.1852\color{black} $\,\,$ & $\,\,$2.2593  $\,\,$ \\
$\,\,$\color{red} 0.3279\color{black} $\,\,$ & $\,\,$\color{red} 0.8437\color{black} $\,\,$ & $\,\,$ 1 $\,\,$ & $\,\,$\color{red} 1.9062\color{black}  $\,\,$ \\
$\,\,$0.1720$\,\,$ & $\,\,$0.4426$\,\,$ & $\,\,$\color{red} 0.5246\color{black} $\,\,$ & $\,\,$ 1  $\,\,$ \\
\end{pmatrix},
\end{equation*}

\begin{equation*}
\mathbf{w}^{\prime} =
\begin{pmatrix}
0.527984\\
0.205198\\
0.175995\\
0.090824
\end{pmatrix} =
0.997137\cdot
\begin{pmatrix}
0.529500\\
0.205787\\
\color{gr} 0.176500\color{black} \\
0.091084
\end{pmatrix},
\end{equation*}
\begin{equation*}
\left[ \frac{{w}^{\prime}_i}{{w}^{\prime}_j} \right] =
\begin{pmatrix}
$\,\,$ 1 $\,\,$ & $\,\,$2.5730$\,\,$ & $\,\,$\color{gr} \color{blue} 3\color{black} $\,\,$ & $\,\,$5.8133$\,\,$ \\
$\,\,$0.3886$\,\,$ & $\,\,$ 1 $\,\,$ & $\,\,$\color{gr} 1.1659\color{black} $\,\,$ & $\,\,$2.2593  $\,\,$ \\
$\,\,$\color{gr} \color{blue}  1/3\color{black} $\,\,$ & $\,\,$\color{gr} 0.8577\color{black} $\,\,$ & $\,\,$ 1 $\,\,$ & $\,\,$\color{gr} 1.9378\color{black}  $\,\,$ \\
$\,\,$0.1720$\,\,$ & $\,\,$0.4426$\,\,$ & $\,\,$\color{gr} 0.5161\color{black} $\,\,$ & $\,\,$ 1  $\,\,$ \\
\end{pmatrix},
\end{equation*}
\end{example}
\newpage
\begin{example}
\begin{equation*}
\mathbf{A} =
\begin{pmatrix}
$\,\,$ 1 $\,\,$ & $\,\,$5$\,\,$ & $\,\,$3$\,\,$ & $\,\,$5 $\,\,$ \\
$\,\,$ 1/5$\,\,$ & $\,\,$ 1 $\,\,$ & $\,\,$3$\,\,$ & $\,\,$2 $\,\,$ \\
$\,\,$ 1/3$\,\,$ & $\,\,$ 1/3$\,\,$ & $\,\,$ 1 $\,\,$ & $\,\,$1 $\,\,$ \\
$\,\,$ 1/5$\,\,$ & $\,\,$ 1/2$\,\,$ & $\,\,$ 1 $\,\,$ & $\,\,$ 1  $\,\,$ \\
\end{pmatrix},
\qquad
\lambda_{\max} =
4.2277,
\qquad
CR = 0.0859
\end{equation*}

\begin{equation*}
\mathbf{w}^{cos} =
\begin{pmatrix}
0.546921\\
0.222638\\
0.122313\\
\color{red} 0.108129\color{black}
\end{pmatrix}\end{equation*}
\begin{equation*}
\left[ \frac{{w}^{cos}_i}{{w}^{cos}_j} \right] =
\begin{pmatrix}
$\,\,$ 1 $\,\,$ & $\,\,$2.4566$\,\,$ & $\,\,$4.4715$\,\,$ & $\,\,$\color{red} 5.0581\color{black} $\,\,$ \\
$\,\,$0.4071$\,\,$ & $\,\,$ 1 $\,\,$ & $\,\,$1.8202$\,\,$ & $\,\,$\color{red} 2.0590\color{black}   $\,\,$ \\
$\,\,$0.2236$\,\,$ & $\,\,$0.5494$\,\,$ & $\,\,$ 1 $\,\,$ & $\,\,$\color{red} 1.1312\color{black}  $\,\,$ \\
$\,\,$\color{red} 0.1977\color{black} $\,\,$ & $\,\,$\color{red} 0.4857\color{black} $\,\,$ & $\,\,$\color{red} 0.8840\color{black} $\,\,$ & $\,\,$ 1  $\,\,$ \\
\end{pmatrix},
\end{equation*}

\begin{equation*}
\mathbf{w}^{\prime} =
\begin{pmatrix}
0.546235\\
0.222358\\
0.122160\\
0.109247
\end{pmatrix} =
0.998746\cdot
\begin{pmatrix}
0.546921\\
0.222638\\
0.122313\\
\color{gr} 0.109384\color{black}
\end{pmatrix},
\end{equation*}
\begin{equation*}
\left[ \frac{{w}^{\prime}_i}{{w}^{\prime}_j} \right] =
\begin{pmatrix}
$\,\,$ 1 $\,\,$ & $\,\,$2.4566$\,\,$ & $\,\,$4.4715$\,\,$ & $\,\,$\color{gr} \color{blue} 5\color{black} $\,\,$ \\
$\,\,$0.4071$\,\,$ & $\,\,$ 1 $\,\,$ & $\,\,$1.8202$\,\,$ & $\,\,$\color{gr} 2.0354\color{black}   $\,\,$ \\
$\,\,$0.2236$\,\,$ & $\,\,$0.5494$\,\,$ & $\,\,$ 1 $\,\,$ & $\,\,$\color{gr} 1.1182\color{black}  $\,\,$ \\
$\,\,$\color{gr} \color{blue}  1/5\color{black} $\,\,$ & $\,\,$\color{gr} 0.4913\color{black} $\,\,$ & $\,\,$\color{gr} 0.8943\color{black} $\,\,$ & $\,\,$ 1  $\,\,$ \\
\end{pmatrix},
\end{equation*}
\end{example}
\newpage
\begin{example}
\begin{equation*}
\mathbf{A} =
\begin{pmatrix}
$\,\,$ 1 $\,\,$ & $\,\,$5$\,\,$ & $\,\,$3$\,\,$ & $\,\,$6 $\,\,$ \\
$\,\,$ 1/5$\,\,$ & $\,\,$ 1 $\,\,$ & $\,\,$1$\,\,$ & $\,\,$4 $\,\,$ \\
$\,\,$ 1/3$\,\,$ & $\,\,$ 1 $\,\,$ & $\,\,$ 1 $\,\,$ & $\,\,$3 $\,\,$ \\
$\,\,$ 1/6$\,\,$ & $\,\,$ 1/4$\,\,$ & $\,\,$ 1/3$\,\,$ & $\,\,$ 1  $\,\,$ \\
\end{pmatrix},
\qquad
\lambda_{\max} =
4.1252,
\qquad
CR = 0.0472
\end{equation*}

\begin{equation*}
\mathbf{w}^{cos} =
\begin{pmatrix}
0.559246\\
0.186896\\
\color{red} 0.186234\color{black} \\
0.067624
\end{pmatrix}\end{equation*}
\begin{equation*}
\left[ \frac{{w}^{cos}_i}{{w}^{cos}_j} \right] =
\begin{pmatrix}
$\,\,$ 1 $\,\,$ & $\,\,$2.9923$\,\,$ & $\,\,$\color{red} 3.0029\color{black} $\,\,$ & $\,\,$8.2699$\,\,$ \\
$\,\,$0.3342$\,\,$ & $\,\,$ 1 $\,\,$ & $\,\,$\color{red} 1.0036\color{black} $\,\,$ & $\,\,$2.7638  $\,\,$ \\
$\,\,$\color{red} 0.3330\color{black} $\,\,$ & $\,\,$\color{red} 0.9965\color{black} $\,\,$ & $\,\,$ 1 $\,\,$ & $\,\,$\color{red} 2.7540\color{black}  $\,\,$ \\
$\,\,$0.1209$\,\,$ & $\,\,$0.3618$\,\,$ & $\,\,$\color{red} 0.3631\color{black} $\,\,$ & $\,\,$ 1  $\,\,$ \\
\end{pmatrix},
\end{equation*}

\begin{equation*}
\mathbf{w}^{\prime} =
\begin{pmatrix}
0.559144\\
0.186862\\
0.186381\\
0.067612
\end{pmatrix} =
0.999818\cdot
\begin{pmatrix}
0.559246\\
0.186896\\
\color{gr} 0.186415\color{black} \\
0.067624
\end{pmatrix},
\end{equation*}
\begin{equation*}
\left[ \frac{{w}^{\prime}_i}{{w}^{\prime}_j} \right] =
\begin{pmatrix}
$\,\,$ 1 $\,\,$ & $\,\,$2.9923$\,\,$ & $\,\,$\color{gr} \color{blue} 3\color{black} $\,\,$ & $\,\,$8.2699$\,\,$ \\
$\,\,$0.3342$\,\,$ & $\,\,$ 1 $\,\,$ & $\,\,$\color{gr} 1.0026\color{black} $\,\,$ & $\,\,$2.7638  $\,\,$ \\
$\,\,$\color{gr} \color{blue}  1/3\color{black} $\,\,$ & $\,\,$\color{gr} 0.9974\color{black} $\,\,$ & $\,\,$ 1 $\,\,$ & $\,\,$\color{gr} 2.7566\color{black}  $\,\,$ \\
$\,\,$0.1209$\,\,$ & $\,\,$0.3618$\,\,$ & $\,\,$\color{gr} 0.3628\color{black} $\,\,$ & $\,\,$ 1  $\,\,$ \\
\end{pmatrix},
\end{equation*}
\end{example}
\newpage
\begin{example}
\begin{equation*}
\mathbf{A} =
\begin{pmatrix}
$\,\,$ 1 $\,\,$ & $\,\,$5$\,\,$ & $\,\,$3$\,\,$ & $\,\,$6 $\,\,$ \\
$\,\,$ 1/5$\,\,$ & $\,\,$ 1 $\,\,$ & $\,\,$1$\,\,$ & $\,\,$5 $\,\,$ \\
$\,\,$ 1/3$\,\,$ & $\,\,$ 1 $\,\,$ & $\,\,$ 1 $\,\,$ & $\,\,$3 $\,\,$ \\
$\,\,$ 1/6$\,\,$ & $\,\,$ 1/5$\,\,$ & $\,\,$ 1/3$\,\,$ & $\,\,$ 1  $\,\,$ \\
\end{pmatrix},
\qquad
\lambda_{\max} =
4.1758,
\qquad
CR = 0.0663
\end{equation*}

\begin{equation*}
\mathbf{w}^{cos} =
\begin{pmatrix}
0.552028\\
0.200600\\
\color{red} 0.182504\color{black} \\
0.064867
\end{pmatrix}\end{equation*}
\begin{equation*}
\left[ \frac{{w}^{cos}_i}{{w}^{cos}_j} \right] =
\begin{pmatrix}
$\,\,$ 1 $\,\,$ & $\,\,$2.7519$\,\,$ & $\,\,$\color{red} 3.0247\color{black} $\,\,$ & $\,\,$8.5101$\,\,$ \\
$\,\,$0.3634$\,\,$ & $\,\,$ 1 $\,\,$ & $\,\,$\color{red} 1.0992\color{black} $\,\,$ & $\,\,$3.0925  $\,\,$ \\
$\,\,$\color{red} 0.3306\color{black} $\,\,$ & $\,\,$\color{red} 0.9098\color{black} $\,\,$ & $\,\,$ 1 $\,\,$ & $\,\,$\color{red} 2.8135\color{black}  $\,\,$ \\
$\,\,$0.1175$\,\,$ & $\,\,$0.3234$\,\,$ & $\,\,$\color{red} 0.3554\color{black} $\,\,$ & $\,\,$ 1  $\,\,$ \\
\end{pmatrix},
\end{equation*}

\begin{equation*}
\mathbf{w}^{\prime} =
\begin{pmatrix}
0.551199\\
0.200299\\
0.183733\\
0.064770
\end{pmatrix} =
0.998497\cdot
\begin{pmatrix}
0.552028\\
0.200600\\
\color{gr} 0.184009\color{black} \\
0.064867
\end{pmatrix},
\end{equation*}
\begin{equation*}
\left[ \frac{{w}^{\prime}_i}{{w}^{\prime}_j} \right] =
\begin{pmatrix}
$\,\,$ 1 $\,\,$ & $\,\,$2.7519$\,\,$ & $\,\,$\color{gr} \color{blue} 3\color{black} $\,\,$ & $\,\,$8.5101$\,\,$ \\
$\,\,$0.3634$\,\,$ & $\,\,$ 1 $\,\,$ & $\,\,$\color{gr} 1.0902\color{black} $\,\,$ & $\,\,$3.0925  $\,\,$ \\
$\,\,$\color{gr} \color{blue}  1/3\color{black} $\,\,$ & $\,\,$\color{gr} 0.9173\color{black} $\,\,$ & $\,\,$ 1 $\,\,$ & $\,\,$\color{gr} 2.8367\color{black}  $\,\,$ \\
$\,\,$0.1175$\,\,$ & $\,\,$0.3234$\,\,$ & $\,\,$\color{gr} 0.3525\color{black} $\,\,$ & $\,\,$ 1  $\,\,$ \\
\end{pmatrix},
\end{equation*}
\end{example}
\newpage
\begin{example}
\begin{equation*}
\mathbf{A} =
\begin{pmatrix}
$\,\,$ 1 $\,\,$ & $\,\,$5$\,\,$ & $\,\,$3$\,\,$ & $\,\,$6 $\,\,$ \\
$\,\,$ 1/5$\,\,$ & $\,\,$ 1 $\,\,$ & $\,\,$1$\,\,$ & $\,\,$6 $\,\,$ \\
$\,\,$ 1/3$\,\,$ & $\,\,$ 1 $\,\,$ & $\,\,$ 1 $\,\,$ & $\,\,$3 $\,\,$ \\
$\,\,$ 1/6$\,\,$ & $\,\,$ 1/6$\,\,$ & $\,\,$ 1/3$\,\,$ & $\,\,$ 1  $\,\,$ \\
\end{pmatrix},
\qquad
\lambda_{\max} =
4.2277,
\qquad
CR = 0.0859
\end{equation*}

\begin{equation*}
\mathbf{w}^{cos} =
\begin{pmatrix}
0.545917\\
0.212187\\
\color{red} 0.179151\color{black} \\
0.062745
\end{pmatrix}\end{equation*}
\begin{equation*}
\left[ \frac{{w}^{cos}_i}{{w}^{cos}_j} \right] =
\begin{pmatrix}
$\,\,$ 1 $\,\,$ & $\,\,$2.5728$\,\,$ & $\,\,$\color{red} 3.0472\color{black} $\,\,$ & $\,\,$8.7006$\,\,$ \\
$\,\,$0.3887$\,\,$ & $\,\,$ 1 $\,\,$ & $\,\,$\color{red} 1.1844\color{black} $\,\,$ & $\,\,$3.3817  $\,\,$ \\
$\,\,$\color{red} 0.3282\color{black} $\,\,$ & $\,\,$\color{red} 0.8443\color{black} $\,\,$ & $\,\,$ 1 $\,\,$ & $\,\,$\color{red} 2.8552\color{black}  $\,\,$ \\
$\,\,$0.1149$\,\,$ & $\,\,$0.2957$\,\,$ & $\,\,$\color{red} 0.3502\color{black} $\,\,$ & $\,\,$ 1  $\,\,$ \\
\end{pmatrix},
\end{equation*}

\begin{equation*}
\mathbf{w}^{\prime} =
\begin{pmatrix}
0.544381\\
0.211590\\
0.181460\\
0.062568
\end{pmatrix} =
0.997187\cdot
\begin{pmatrix}
0.545917\\
0.212187\\
\color{gr} 0.181972\color{black} \\
0.062745
\end{pmatrix},
\end{equation*}
\begin{equation*}
\left[ \frac{{w}^{\prime}_i}{{w}^{\prime}_j} \right] =
\begin{pmatrix}
$\,\,$ 1 $\,\,$ & $\,\,$2.5728$\,\,$ & $\,\,$\color{gr} \color{blue} 3\color{black} $\,\,$ & $\,\,$8.7006$\,\,$ \\
$\,\,$0.3887$\,\,$ & $\,\,$ 1 $\,\,$ & $\,\,$\color{gr} 1.1660\color{black} $\,\,$ & $\,\,$3.3817  $\,\,$ \\
$\,\,$\color{gr} \color{blue}  1/3\color{black} $\,\,$ & $\,\,$\color{gr} 0.8576\color{black} $\,\,$ & $\,\,$ 1 $\,\,$ & $\,\,$\color{gr} 2.9002\color{black}  $\,\,$ \\
$\,\,$0.1149$\,\,$ & $\,\,$0.2957$\,\,$ & $\,\,$\color{gr} 0.3448\color{black} $\,\,$ & $\,\,$ 1  $\,\,$ \\
\end{pmatrix},
\end{equation*}
\end{example}
\newpage
\begin{example}
\begin{equation*}
\mathbf{A} =
\begin{pmatrix}
$\,\,$ 1 $\,\,$ & $\,\,$5$\,\,$ & $\,\,$3$\,\,$ & $\,\,$7 $\,\,$ \\
$\,\,$ 1/5$\,\,$ & $\,\,$ 1 $\,\,$ & $\,\,$1$\,\,$ & $\,\,$5 $\,\,$ \\
$\,\,$ 1/3$\,\,$ & $\,\,$ 1 $\,\,$ & $\,\,$ 1 $\,\,$ & $\,\,$3 $\,\,$ \\
$\,\,$ 1/7$\,\,$ & $\,\,$ 1/5$\,\,$ & $\,\,$ 1/3$\,\,$ & $\,\,$ 1  $\,\,$ \\
\end{pmatrix},
\qquad
\lambda_{\max} =
4.1415,
\qquad
CR = 0.0533
\end{equation*}

\begin{equation*}
\mathbf{w}^{cos} =
\begin{pmatrix}
0.565250\\
0.194718\\
\color{red} 0.179583\color{black} \\
0.060449
\end{pmatrix}\end{equation*}
\begin{equation*}
\left[ \frac{{w}^{cos}_i}{{w}^{cos}_j} \right] =
\begin{pmatrix}
$\,\,$ 1 $\,\,$ & $\,\,$2.9029$\,\,$ & $\,\,$\color{red} 3.1476\color{black} $\,\,$ & $\,\,$9.3509$\,\,$ \\
$\,\,$0.3445$\,\,$ & $\,\,$ 1 $\,\,$ & $\,\,$\color{red} 1.0843\color{black} $\,\,$ & $\,\,$3.2212  $\,\,$ \\
$\,\,$\color{red} 0.3177\color{black} $\,\,$ & $\,\,$\color{red} 0.9223\color{black} $\,\,$ & $\,\,$ 1 $\,\,$ & $\,\,$\color{red} 2.9708\color{black}  $\,\,$ \\
$\,\,$0.1069$\,\,$ & $\,\,$0.3104$\,\,$ & $\,\,$\color{red} 0.3366\color{black} $\,\,$ & $\,\,$ 1  $\,\,$ \\
\end{pmatrix},
\end{equation*}

\begin{equation*}
\mathbf{w}^{\prime} =
\begin{pmatrix}
0.564256\\
0.194375\\
0.181027\\
0.060342
\end{pmatrix} =
0.998240\cdot
\begin{pmatrix}
0.565250\\
0.194718\\
\color{gr} 0.181346\color{black} \\
0.060449
\end{pmatrix},
\end{equation*}
\begin{equation*}
\left[ \frac{{w}^{\prime}_i}{{w}^{\prime}_j} \right] =
\begin{pmatrix}
$\,\,$ 1 $\,\,$ & $\,\,$2.9029$\,\,$ & $\,\,$\color{gr} 3.1170\color{black} $\,\,$ & $\,\,$9.3509$\,\,$ \\
$\,\,$0.3445$\,\,$ & $\,\,$ 1 $\,\,$ & $\,\,$\color{gr} 1.0737\color{black} $\,\,$ & $\,\,$3.2212  $\,\,$ \\
$\,\,$\color{gr} 0.3208\color{black} $\,\,$ & $\,\,$\color{gr} 0.9313\color{black} $\,\,$ & $\,\,$ 1 $\,\,$ & $\,\,$\color{gr} \color{blue} 3\color{black}  $\,\,$ \\
$\,\,$0.1069$\,\,$ & $\,\,$0.3104$\,\,$ & $\,\,$\color{gr} \color{blue}  1/3\color{black} $\,\,$ & $\,\,$ 1  $\,\,$ \\
\end{pmatrix},
\end{equation*}
\end{example}
\newpage
\begin{example}
\begin{equation*}
\mathbf{A} =
\begin{pmatrix}
$\,\,$ 1 $\,\,$ & $\,\,$5$\,\,$ & $\,\,$3$\,\,$ & $\,\,$8 $\,\,$ \\
$\,\,$ 1/5$\,\,$ & $\,\,$ 1 $\,\,$ & $\,\,$1$\,\,$ & $\,\,$6 $\,\,$ \\
$\,\,$ 1/3$\,\,$ & $\,\,$ 1 $\,\,$ & $\,\,$ 1 $\,\,$ & $\,\,$4 $\,\,$ \\
$\,\,$ 1/8$\,\,$ & $\,\,$ 1/6$\,\,$ & $\,\,$ 1/4$\,\,$ & $\,\,$ 1  $\,\,$ \\
\end{pmatrix},
\qquad
\lambda_{\max} =
4.1502,
\qquad
CR = 0.0566
\end{equation*}

\begin{equation*}
\mathbf{w}^{cos} =
\begin{pmatrix}
0.564769\\
0.197297\\
\color{red} 0.187461\color{black} \\
0.050473
\end{pmatrix}\end{equation*}
\begin{equation*}
\left[ \frac{{w}^{cos}_i}{{w}^{cos}_j} \right] =
\begin{pmatrix}
$\,\,$ 1 $\,\,$ & $\,\,$2.8625$\,\,$ & $\,\,$\color{red} 3.0127\color{black} $\,\,$ & $\,\,$11.1895$\,\,$ \\
$\,\,$0.3493$\,\,$ & $\,\,$ 1 $\,\,$ & $\,\,$\color{red} 1.0525\color{black} $\,\,$ & $\,\,$3.9089  $\,\,$ \\
$\,\,$\color{red} 0.3319\color{black} $\,\,$ & $\,\,$\color{red} 0.9501\color{black} $\,\,$ & $\,\,$ 1 $\,\,$ & $\,\,$\color{red} 3.7141\color{black}  $\,\,$ \\
$\,\,$0.0894$\,\,$ & $\,\,$0.2558$\,\,$ & $\,\,$\color{red} 0.2692\color{black} $\,\,$ & $\,\,$ 1  $\,\,$ \\
\end{pmatrix},
\end{equation*}

\begin{equation*}
\mathbf{w}^{\prime} =
\begin{pmatrix}
0.564320\\
0.197140\\
0.188107\\
0.050433
\end{pmatrix} =
0.999206\cdot
\begin{pmatrix}
0.564769\\
0.197297\\
\color{gr} 0.188256\color{black} \\
0.050473
\end{pmatrix},
\end{equation*}
\begin{equation*}
\left[ \frac{{w}^{\prime}_i}{{w}^{\prime}_j} \right] =
\begin{pmatrix}
$\,\,$ 1 $\,\,$ & $\,\,$2.8625$\,\,$ & $\,\,$\color{gr} \color{blue} 3\color{black} $\,\,$ & $\,\,$11.1895$\,\,$ \\
$\,\,$0.3493$\,\,$ & $\,\,$ 1 $\,\,$ & $\,\,$\color{gr} 1.0480\color{black} $\,\,$ & $\,\,$3.9089  $\,\,$ \\
$\,\,$\color{gr} \color{blue}  1/3\color{black} $\,\,$ & $\,\,$\color{gr} 0.9542\color{black} $\,\,$ & $\,\,$ 1 $\,\,$ & $\,\,$\color{gr} 3.7298\color{black}  $\,\,$ \\
$\,\,$0.0894$\,\,$ & $\,\,$0.2558$\,\,$ & $\,\,$\color{gr} 0.2681\color{black} $\,\,$ & $\,\,$ 1  $\,\,$ \\
\end{pmatrix},
\end{equation*}
\end{example}
\newpage
\begin{example}
\begin{equation*}
\mathbf{A} =
\begin{pmatrix}
$\,\,$ 1 $\,\,$ & $\,\,$5$\,\,$ & $\,\,$3$\,\,$ & $\,\,$8 $\,\,$ \\
$\,\,$ 1/5$\,\,$ & $\,\,$ 1 $\,\,$ & $\,\,$1$\,\,$ & $\,\,$7 $\,\,$ \\
$\,\,$ 1/3$\,\,$ & $\,\,$ 1 $\,\,$ & $\,\,$ 1 $\,\,$ & $\,\,$4 $\,\,$ \\
$\,\,$ 1/8$\,\,$ & $\,\,$ 1/7$\,\,$ & $\,\,$ 1/4$\,\,$ & $\,\,$ 1  $\,\,$ \\
\end{pmatrix},
\qquad
\lambda_{\max} =
4.1888,
\qquad
CR = 0.0712
\end{equation*}

\begin{equation*}
\mathbf{w}^{cos} =
\begin{pmatrix}
0.559327\\
0.207008\\
\color{red} 0.184630\color{black} \\
0.049035
\end{pmatrix}\end{equation*}
\begin{equation*}
\left[ \frac{{w}^{cos}_i}{{w}^{cos}_j} \right] =
\begin{pmatrix}
$\,\,$ 1 $\,\,$ & $\,\,$2.7020$\,\,$ & $\,\,$\color{red} 3.0294\color{black} $\,\,$ & $\,\,$11.4066$\,\,$ \\
$\,\,$0.3701$\,\,$ & $\,\,$ 1 $\,\,$ & $\,\,$\color{red} 1.1212\color{black} $\,\,$ & $\,\,$4.2216  $\,\,$ \\
$\,\,$\color{red} 0.3301\color{black} $\,\,$ & $\,\,$\color{red} 0.8919\color{black} $\,\,$ & $\,\,$ 1 $\,\,$ & $\,\,$\color{red} 3.7653\color{black}  $\,\,$ \\
$\,\,$0.0877$\,\,$ & $\,\,$0.2369$\,\,$ & $\,\,$\color{red} 0.2656\color{black} $\,\,$ & $\,\,$ 1  $\,\,$ \\
\end{pmatrix},
\end{equation*}

\begin{equation*}
\mathbf{w}^{\prime} =
\begin{pmatrix}
0.558315\\
0.206633\\
0.186105\\
0.048947
\end{pmatrix} =
0.998191\cdot
\begin{pmatrix}
0.559327\\
0.207008\\
\color{gr} 0.186442\color{black} \\
0.049035
\end{pmatrix},
\end{equation*}
\begin{equation*}
\left[ \frac{{w}^{\prime}_i}{{w}^{\prime}_j} \right] =
\begin{pmatrix}
$\,\,$ 1 $\,\,$ & $\,\,$2.7020$\,\,$ & $\,\,$\color{gr} \color{blue} 3\color{black} $\,\,$ & $\,\,$11.4066$\,\,$ \\
$\,\,$0.3701$\,\,$ & $\,\,$ 1 $\,\,$ & $\,\,$\color{gr} 1.1103\color{black} $\,\,$ & $\,\,$4.2216  $\,\,$ \\
$\,\,$\color{gr} \color{blue}  1/3\color{black} $\,\,$ & $\,\,$\color{gr} 0.9007\color{black} $\,\,$ & $\,\,$ 1 $\,\,$ & $\,\,$\color{gr} 3.8022\color{black}  $\,\,$ \\
$\,\,$0.0877$\,\,$ & $\,\,$0.2369$\,\,$ & $\,\,$\color{gr} 0.2630\color{black} $\,\,$ & $\,\,$ 1  $\,\,$ \\
\end{pmatrix},
\end{equation*}
\end{example}
\newpage
\begin{example}
\begin{equation*}
\mathbf{A} =
\begin{pmatrix}
$\,\,$ 1 $\,\,$ & $\,\,$5$\,\,$ & $\,\,$3$\,\,$ & $\,\,$8 $\,\,$ \\
$\,\,$ 1/5$\,\,$ & $\,\,$ 1 $\,\,$ & $\,\,$1$\,\,$ & $\,\,$8 $\,\,$ \\
$\,\,$ 1/3$\,\,$ & $\,\,$ 1 $\,\,$ & $\,\,$ 1 $\,\,$ & $\,\,$4 $\,\,$ \\
$\,\,$ 1/8$\,\,$ & $\,\,$ 1/8$\,\,$ & $\,\,$ 1/4$\,\,$ & $\,\,$ 1  $\,\,$ \\
\end{pmatrix},
\qquad
\lambda_{\max} =
4.2277,
\qquad
CR = 0.0859
\end{equation*}

\begin{equation*}
\mathbf{w}^{cos} =
\begin{pmatrix}
0.554565\\
0.215554\\
\color{red} 0.182040\color{black} \\
0.047842
\end{pmatrix}\end{equation*}
\begin{equation*}
\left[ \frac{{w}^{cos}_i}{{w}^{cos}_j} \right] =
\begin{pmatrix}
$\,\,$ 1 $\,\,$ & $\,\,$2.5727$\,\,$ & $\,\,$\color{red} 3.0464\color{black} $\,\,$ & $\,\,$11.5916$\,\,$ \\
$\,\,$0.3887$\,\,$ & $\,\,$ 1 $\,\,$ & $\,\,$\color{red} 1.1841\color{black} $\,\,$ & $\,\,$4.5055  $\,\,$ \\
$\,\,$\color{red} 0.3283\color{black} $\,\,$ & $\,\,$\color{red} 0.8445\color{black} $\,\,$ & $\,\,$ 1 $\,\,$ & $\,\,$\color{red} 3.8050\color{black}  $\,\,$ \\
$\,\,$0.0863$\,\,$ & $\,\,$0.2219$\,\,$ & $\,\,$\color{red} 0.2628\color{black} $\,\,$ & $\,\,$ 1  $\,\,$ \\
\end{pmatrix},
\end{equation*}

\begin{equation*}
\mathbf{w}^{\prime} =
\begin{pmatrix}
0.553008\\
0.214949\\
0.184336\\
0.047708
\end{pmatrix} =
0.997193\cdot
\begin{pmatrix}
0.554565\\
0.215554\\
\color{gr} 0.184855\color{black} \\
0.047842
\end{pmatrix},
\end{equation*}
\begin{equation*}
\left[ \frac{{w}^{\prime}_i}{{w}^{\prime}_j} \right] =
\begin{pmatrix}
$\,\,$ 1 $\,\,$ & $\,\,$2.5727$\,\,$ & $\,\,$\color{gr} \color{blue} 3\color{black} $\,\,$ & $\,\,$11.5916$\,\,$ \\
$\,\,$0.3887$\,\,$ & $\,\,$ 1 $\,\,$ & $\,\,$\color{gr} 1.1661\color{black} $\,\,$ & $\,\,$4.5055  $\,\,$ \\
$\,\,$\color{gr} \color{blue}  1/3\color{black} $\,\,$ & $\,\,$\color{gr} 0.8576\color{black} $\,\,$ & $\,\,$ 1 $\,\,$ & $\,\,$\color{gr} 3.8639\color{black}  $\,\,$ \\
$\,\,$0.0863$\,\,$ & $\,\,$0.2219$\,\,$ & $\,\,$\color{gr} 0.2588\color{black} $\,\,$ & $\,\,$ 1  $\,\,$ \\
\end{pmatrix},
\end{equation*}
\end{example}
\newpage
\begin{example}
\begin{equation*}
\mathbf{A} =
\begin{pmatrix}
$\,\,$ 1 $\,\,$ & $\,\,$5$\,\,$ & $\,\,$3$\,\,$ & $\,\,$9 $\,\,$ \\
$\,\,$ 1/5$\,\,$ & $\,\,$ 1 $\,\,$ & $\,\,$1$\,\,$ & $\,\,$6 $\,\,$ \\
$\,\,$ 1/3$\,\,$ & $\,\,$ 1 $\,\,$ & $\,\,$ 1 $\,\,$ & $\,\,$4 $\,\,$ \\
$\,\,$ 1/9$\,\,$ & $\,\,$ 1/6$\,\,$ & $\,\,$ 1/4$\,\,$ & $\,\,$ 1  $\,\,$ \\
\end{pmatrix},
\qquad
\lambda_{\max} =
4.1252,
\qquad
CR = 0.0472
\end{equation*}

\begin{equation*}
\mathbf{w}^{cos} =
\begin{pmatrix}
0.574675\\
0.192736\\
\color{red} 0.184819\color{black} \\
0.047771
\end{pmatrix}\end{equation*}
\begin{equation*}
\left[ \frac{{w}^{cos}_i}{{w}^{cos}_j} \right] =
\begin{pmatrix}
$\,\,$ 1 $\,\,$ & $\,\,$2.9817$\,\,$ & $\,\,$\color{red} 3.1094\color{black} $\,\,$ & $\,\,$12.0299$\,\,$ \\
$\,\,$0.3354$\,\,$ & $\,\,$ 1 $\,\,$ & $\,\,$\color{red} 1.0428\color{black} $\,\,$ & $\,\,$4.0346  $\,\,$ \\
$\,\,$\color{red} 0.3216\color{black} $\,\,$ & $\,\,$\color{red} 0.9589\color{black} $\,\,$ & $\,\,$ 1 $\,\,$ & $\,\,$\color{red} 3.8689\color{black}  $\,\,$ \\
$\,\,$0.0831$\,\,$ & $\,\,$0.2479$\,\,$ & $\,\,$\color{red} 0.2585\color{black} $\,\,$ & $\,\,$ 1  $\,\,$ \\
\end{pmatrix},
\end{equation*}

\begin{equation*}
\mathbf{w}^{\prime} =
\begin{pmatrix}
0.571097\\
0.191536\\
0.189893\\
0.047473
\end{pmatrix} =
0.993775\cdot
\begin{pmatrix}
0.574675\\
0.192736\\
\color{gr} 0.191083\color{black} \\
0.047771
\end{pmatrix},
\end{equation*}
\begin{equation*}
\left[ \frac{{w}^{\prime}_i}{{w}^{\prime}_j} \right] =
\begin{pmatrix}
$\,\,$ 1 $\,\,$ & $\,\,$2.9817$\,\,$ & $\,\,$\color{gr} 3.0075\color{black} $\,\,$ & $\,\,$12.0299$\,\,$ \\
$\,\,$0.3354$\,\,$ & $\,\,$ 1 $\,\,$ & $\,\,$\color{gr} 1.0087\color{black} $\,\,$ & $\,\,$4.0346  $\,\,$ \\
$\,\,$\color{gr} 0.3325\color{black} $\,\,$ & $\,\,$\color{gr} 0.9914\color{black} $\,\,$ & $\,\,$ 1 $\,\,$ & $\,\,$\color{gr} \color{blue} 4\color{black}  $\,\,$ \\
$\,\,$0.0831$\,\,$ & $\,\,$0.2479$\,\,$ & $\,\,$\color{gr} \color{blue}  1/4\color{black} $\,\,$ & $\,\,$ 1  $\,\,$ \\
\end{pmatrix},
\end{equation*}
\end{example}
\newpage
\begin{example}
\begin{equation*}
\mathbf{A} =
\begin{pmatrix}
$\,\,$ 1 $\,\,$ & $\,\,$5$\,\,$ & $\,\,$3$\,\,$ & $\,\,$9 $\,\,$ \\
$\,\,$ 1/5$\,\,$ & $\,\,$ 1 $\,\,$ & $\,\,$1$\,\,$ & $\,\,$7 $\,\,$ \\
$\,\,$ 1/3$\,\,$ & $\,\,$ 1 $\,\,$ & $\,\,$ 1 $\,\,$ & $\,\,$4 $\,\,$ \\
$\,\,$ 1/9$\,\,$ & $\,\,$ 1/7$\,\,$ & $\,\,$ 1/4$\,\,$ & $\,\,$ 1  $\,\,$ \\
\end{pmatrix},
\qquad
\lambda_{\max} =
4.1610,
\qquad
CR = 0.0607
\end{equation*}

\begin{equation*}
\mathbf{w}^{cos} =
\begin{pmatrix}
0.569038\\
0.202249\\
\color{red} 0.182308\color{black} \\
0.046405
\end{pmatrix}\end{equation*}
\begin{equation*}
\left[ \frac{{w}^{cos}_i}{{w}^{cos}_j} \right] =
\begin{pmatrix}
$\,\,$ 1 $\,\,$ & $\,\,$2.8135$\,\,$ & $\,\,$\color{red} 3.1213\color{black} $\,\,$ & $\,\,$12.2625$\,\,$ \\
$\,\,$0.3554$\,\,$ & $\,\,$ 1 $\,\,$ & $\,\,$\color{red} 1.1094\color{black} $\,\,$ & $\,\,$4.3584  $\,\,$ \\
$\,\,$\color{red} 0.3204\color{black} $\,\,$ & $\,\,$\color{red} 0.9014\color{black} $\,\,$ & $\,\,$ 1 $\,\,$ & $\,\,$\color{red} 3.9286\color{black}  $\,\,$ \\
$\,\,$0.0815$\,\,$ & $\,\,$0.2294$\,\,$ & $\,\,$\color{red} 0.2545\color{black} $\,\,$ & $\,\,$ 1  $\,\,$ \\
\end{pmatrix},
\end{equation*}

\begin{equation*}
\mathbf{w}^{\prime} =
\begin{pmatrix}
0.567160\\
0.201582\\
0.185007\\
0.046252
\end{pmatrix} =
0.996699\cdot
\begin{pmatrix}
0.569038\\
0.202249\\
\color{gr} 0.185619\color{black} \\
0.046405
\end{pmatrix},
\end{equation*}
\begin{equation*}
\left[ \frac{{w}^{\prime}_i}{{w}^{\prime}_j} \right] =
\begin{pmatrix}
$\,\,$ 1 $\,\,$ & $\,\,$2.8135$\,\,$ & $\,\,$\color{gr} 3.0656\color{black} $\,\,$ & $\,\,$12.2625$\,\,$ \\
$\,\,$0.3554$\,\,$ & $\,\,$ 1 $\,\,$ & $\,\,$\color{gr} 1.0896\color{black} $\,\,$ & $\,\,$4.3584  $\,\,$ \\
$\,\,$\color{gr} 0.3262\color{black} $\,\,$ & $\,\,$\color{gr} 0.9178\color{black} $\,\,$ & $\,\,$ 1 $\,\,$ & $\,\,$\color{gr} \color{blue} 4\color{black}  $\,\,$ \\
$\,\,$0.0815$\,\,$ & $\,\,$0.2294$\,\,$ & $\,\,$\color{gr} \color{blue}  1/4\color{black} $\,\,$ & $\,\,$ 1  $\,\,$ \\
\end{pmatrix},
\end{equation*}
\end{example}
\newpage
\begin{example}
\begin{equation*}
\mathbf{A} =
\begin{pmatrix}
$\,\,$ 1 $\,\,$ & $\,\,$5$\,\,$ & $\,\,$3$\,\,$ & $\,\,$9 $\,\,$ \\
$\,\,$ 1/5$\,\,$ & $\,\,$ 1 $\,\,$ & $\,\,$1$\,\,$ & $\,\,$8 $\,\,$ \\
$\,\,$ 1/3$\,\,$ & $\,\,$ 1 $\,\,$ & $\,\,$ 1 $\,\,$ & $\,\,$4 $\,\,$ \\
$\,\,$ 1/9$\,\,$ & $\,\,$ 1/8$\,\,$ & $\,\,$ 1/4$\,\,$ & $\,\,$ 1  $\,\,$ \\
\end{pmatrix},
\qquad
\lambda_{\max} =
4.1974,
\qquad
CR = 0.0744
\end{equation*}

\begin{equation*}
\mathbf{w}^{cos} =
\begin{pmatrix}
0.564008\\
0.210730\\
\color{red} 0.179990\color{black} \\
0.045272
\end{pmatrix}\end{equation*}
\begin{equation*}
\left[ \frac{{w}^{cos}_i}{{w}^{cos}_j} \right] =
\begin{pmatrix}
$\,\,$ 1 $\,\,$ & $\,\,$2.6764$\,\,$ & $\,\,$\color{red} 3.1335\color{black} $\,\,$ & $\,\,$12.4582$\,\,$ \\
$\,\,$0.3736$\,\,$ & $\,\,$ 1 $\,\,$ & $\,\,$\color{red} 1.1708\color{black} $\,\,$ & $\,\,$4.6547  $\,\,$ \\
$\,\,$\color{red} 0.3191\color{black} $\,\,$ & $\,\,$\color{red} 0.8541\color{black} $\,\,$ & $\,\,$ 1 $\,\,$ & $\,\,$\color{red} 3.9757\color{black}  $\,\,$ \\
$\,\,$0.0803$\,\,$ & $\,\,$0.2148$\,\,$ & $\,\,$\color{red} 0.2515\color{black} $\,\,$ & $\,\,$ 1  $\,\,$ \\
\end{pmatrix},
\end{equation*}

\begin{equation*}
\mathbf{w}^{\prime} =
\begin{pmatrix}
0.563389\\
0.210499\\
0.180890\\
0.045223
\end{pmatrix} =
0.998903\cdot
\begin{pmatrix}
0.564008\\
0.210730\\
\color{gr} 0.181089\color{black} \\
0.045272
\end{pmatrix},
\end{equation*}
\begin{equation*}
\left[ \frac{{w}^{\prime}_i}{{w}^{\prime}_j} \right] =
\begin{pmatrix}
$\,\,$ 1 $\,\,$ & $\,\,$2.6764$\,\,$ & $\,\,$\color{gr} 3.1145\color{black} $\,\,$ & $\,\,$12.4582$\,\,$ \\
$\,\,$0.3736$\,\,$ & $\,\,$ 1 $\,\,$ & $\,\,$\color{gr} 1.1637\color{black} $\,\,$ & $\,\,$4.6547  $\,\,$ \\
$\,\,$\color{gr} 0.3211\color{black} $\,\,$ & $\,\,$\color{gr} 0.8593\color{black} $\,\,$ & $\,\,$ 1 $\,\,$ & $\,\,$\color{gr} \color{blue} 4\color{black}  $\,\,$ \\
$\,\,$0.0803$\,\,$ & $\,\,$0.2148$\,\,$ & $\,\,$\color{gr} \color{blue}  1/4\color{black} $\,\,$ & $\,\,$ 1  $\,\,$ \\
\end{pmatrix},
\end{equation*}
\end{example}
\newpage
\begin{example}
\begin{equation*}
\mathbf{A} =
\begin{pmatrix}
$\,\,$ 1 $\,\,$ & $\,\,$5$\,\,$ & $\,\,$4$\,\,$ & $\,\,$5 $\,\,$ \\
$\,\,$ 1/5$\,\,$ & $\,\,$ 1 $\,\,$ & $\,\,$3$\,\,$ & $\,\,$2 $\,\,$ \\
$\,\,$ 1/4$\,\,$ & $\,\,$ 1/3$\,\,$ & $\,\,$ 1 $\,\,$ & $\,\,$1 $\,\,$ \\
$\,\,$ 1/5$\,\,$ & $\,\,$ 1/2$\,\,$ & $\,\,$ 1 $\,\,$ & $\,\,$ 1  $\,\,$ \\
\end{pmatrix},
\qquad
\lambda_{\max} =
4.1569,
\qquad
CR = 0.0592
\end{equation*}

\begin{equation*}
\mathbf{w}^{cos} =
\begin{pmatrix}
0.574832\\
0.211863\\
0.107803\\
\color{red} 0.105503\color{black}
\end{pmatrix}\end{equation*}
\begin{equation*}
\left[ \frac{{w}^{cos}_i}{{w}^{cos}_j} \right] =
\begin{pmatrix}
$\,\,$ 1 $\,\,$ & $\,\,$2.7132$\,\,$ & $\,\,$5.3322$\,\,$ & $\,\,$\color{red} 5.4485\color{black} $\,\,$ \\
$\,\,$0.3686$\,\,$ & $\,\,$ 1 $\,\,$ & $\,\,$1.9653$\,\,$ & $\,\,$\color{red} 2.0081\color{black}   $\,\,$ \\
$\,\,$0.1875$\,\,$ & $\,\,$0.5088$\,\,$ & $\,\,$ 1 $\,\,$ & $\,\,$\color{red} 1.0218\color{black}  $\,\,$ \\
$\,\,$\color{red} 0.1835\color{black} $\,\,$ & $\,\,$\color{red} 0.4980\color{black} $\,\,$ & $\,\,$\color{red} 0.9787\color{black} $\,\,$ & $\,\,$ 1  $\,\,$ \\
\end{pmatrix},
\end{equation*}

\begin{equation*}
\mathbf{w}^{\prime} =
\begin{pmatrix}
0.574585\\
0.211772\\
0.107757\\
0.105886
\end{pmatrix} =
0.999571\cdot
\begin{pmatrix}
0.574832\\
0.211863\\
0.107803\\
\color{gr} 0.105931\color{black}
\end{pmatrix},
\end{equation*}
\begin{equation*}
\left[ \frac{{w}^{\prime}_i}{{w}^{\prime}_j} \right] =
\begin{pmatrix}
$\,\,$ 1 $\,\,$ & $\,\,$2.7132$\,\,$ & $\,\,$5.3322$\,\,$ & $\,\,$\color{gr} 5.4265\color{black} $\,\,$ \\
$\,\,$0.3686$\,\,$ & $\,\,$ 1 $\,\,$ & $\,\,$1.9653$\,\,$ & $\,\,$\color{gr} \color{blue} 2\color{black}   $\,\,$ \\
$\,\,$0.1875$\,\,$ & $\,\,$0.5088$\,\,$ & $\,\,$ 1 $\,\,$ & $\,\,$\color{gr} 1.0177\color{black}  $\,\,$ \\
$\,\,$\color{gr} 0.1843\color{black} $\,\,$ & $\,\,$\color{gr} \color{blue}  1/2\color{black} $\,\,$ & $\,\,$\color{gr} 0.9826\color{black} $\,\,$ & $\,\,$ 1  $\,\,$ \\
\end{pmatrix},
\end{equation*}
\end{example}
\newpage
\begin{example}
\begin{equation*}
\mathbf{A} =
\begin{pmatrix}
$\,\,$ 1 $\,\,$ & $\,\,$5$\,\,$ & $\,\,$4$\,\,$ & $\,\,$6 $\,\,$ \\
$\,\,$ 1/5$\,\,$ & $\,\,$ 1 $\,\,$ & $\,\,$3$\,\,$ & $\,\,$2 $\,\,$ \\
$\,\,$ 1/4$\,\,$ & $\,\,$ 1/3$\,\,$ & $\,\,$ 1 $\,\,$ & $\,\,$1 $\,\,$ \\
$\,\,$ 1/6$\,\,$ & $\,\,$ 1/2$\,\,$ & $\,\,$ 1 $\,\,$ & $\,\,$ 1  $\,\,$ \\
\end{pmatrix},
\qquad
\lambda_{\max} =
4.1502,
\qquad
CR = 0.0566
\end{equation*}

\begin{equation*}
\mathbf{w}^{cos} =
\begin{pmatrix}
0.589240\\
0.207103\\
0.105576\\
\color{red} 0.098081\color{black}
\end{pmatrix}\end{equation*}
\begin{equation*}
\left[ \frac{{w}^{cos}_i}{{w}^{cos}_j} \right] =
\begin{pmatrix}
$\,\,$ 1 $\,\,$ & $\,\,$2.8452$\,\,$ & $\,\,$5.5812$\,\,$ & $\,\,$\color{red} 6.0077\color{black} $\,\,$ \\
$\,\,$0.3515$\,\,$ & $\,\,$ 1 $\,\,$ & $\,\,$1.9616$\,\,$ & $\,\,$\color{red} 2.1116\color{black}   $\,\,$ \\
$\,\,$0.1792$\,\,$ & $\,\,$0.5098$\,\,$ & $\,\,$ 1 $\,\,$ & $\,\,$\color{red} 1.0764\color{black}  $\,\,$ \\
$\,\,$\color{red} 0.1665\color{black} $\,\,$ & $\,\,$\color{red} 0.4736\color{black} $\,\,$ & $\,\,$\color{red} 0.9290\color{black} $\,\,$ & $\,\,$ 1  $\,\,$ \\
\end{pmatrix},
\end{equation*}

\begin{equation*}
\mathbf{w}^{\prime} =
\begin{pmatrix}
0.589166\\
0.207077\\
0.105563\\
0.098194
\end{pmatrix} =
0.999874\cdot
\begin{pmatrix}
0.589240\\
0.207103\\
0.105576\\
\color{gr} 0.098207\color{black}
\end{pmatrix},
\end{equation*}
\begin{equation*}
\left[ \frac{{w}^{\prime}_i}{{w}^{\prime}_j} \right] =
\begin{pmatrix}
$\,\,$ 1 $\,\,$ & $\,\,$2.8452$\,\,$ & $\,\,$5.5812$\,\,$ & $\,\,$\color{gr} \color{blue} 6\color{black} $\,\,$ \\
$\,\,$0.3515$\,\,$ & $\,\,$ 1 $\,\,$ & $\,\,$1.9616$\,\,$ & $\,\,$\color{gr} 2.1088\color{black}   $\,\,$ \\
$\,\,$0.1792$\,\,$ & $\,\,$0.5098$\,\,$ & $\,\,$ 1 $\,\,$ & $\,\,$\color{gr} 1.0750\color{black}  $\,\,$ \\
$\,\,$\color{gr} \color{blue}  1/6\color{black} $\,\,$ & $\,\,$\color{gr} 0.4742\color{black} $\,\,$ & $\,\,$\color{gr} 0.9302\color{black} $\,\,$ & $\,\,$ 1  $\,\,$ \\
\end{pmatrix},
\end{equation*}
\end{example}
\newpage
\begin{example}
\begin{equation*}
\mathbf{A} =
\begin{pmatrix}
$\,\,$ 1 $\,\,$ & $\,\,$5$\,\,$ & $\,\,$4$\,\,$ & $\,\,$6 $\,\,$ \\
$\,\,$ 1/5$\,\,$ & $\,\,$ 1 $\,\,$ & $\,\,$4$\,\,$ & $\,\,$2 $\,\,$ \\
$\,\,$ 1/4$\,\,$ & $\,\,$ 1/4$\,\,$ & $\,\,$ 1 $\,\,$ & $\,\,$1 $\,\,$ \\
$\,\,$ 1/6$\,\,$ & $\,\,$ 1/2$\,\,$ & $\,\,$ 1 $\,\,$ & $\,\,$ 1  $\,\,$ \\
\end{pmatrix},
\qquad
\lambda_{\max} =
4.2277,
\qquad
CR = 0.0859
\end{equation*}

\begin{equation*}
\mathbf{w}^{cos} =
\begin{pmatrix}
0.578739\\
0.226049\\
0.100038\\
\color{red} 0.095174\color{black}
\end{pmatrix}\end{equation*}
\begin{equation*}
\left[ \frac{{w}^{cos}_i}{{w}^{cos}_j} \right] =
\begin{pmatrix}
$\,\,$ 1 $\,\,$ & $\,\,$2.5602$\,\,$ & $\,\,$5.7852$\,\,$ & $\,\,$\color{red} 6.0809\color{black} $\,\,$ \\
$\,\,$0.3906$\,\,$ & $\,\,$ 1 $\,\,$ & $\,\,$2.2596$\,\,$ & $\,\,$\color{red} 2.3751\color{black}   $\,\,$ \\
$\,\,$0.1729$\,\,$ & $\,\,$0.4426$\,\,$ & $\,\,$ 1 $\,\,$ & $\,\,$\color{red} 1.0511\color{black}  $\,\,$ \\
$\,\,$\color{red} 0.1645\color{black} $\,\,$ & $\,\,$\color{red} 0.4210\color{black} $\,\,$ & $\,\,$\color{red} 0.9514\color{black} $\,\,$ & $\,\,$ 1  $\,\,$ \\
\end{pmatrix},
\end{equation*}

\begin{equation*}
\mathbf{w}^{\prime} =
\begin{pmatrix}
0.577997\\
0.225759\\
0.099910\\
0.096333
\end{pmatrix} =
0.998719\cdot
\begin{pmatrix}
0.578739\\
0.226049\\
0.100038\\
\color{gr} 0.096456\color{black}
\end{pmatrix},
\end{equation*}
\begin{equation*}
\left[ \frac{{w}^{\prime}_i}{{w}^{\prime}_j} \right] =
\begin{pmatrix}
$\,\,$ 1 $\,\,$ & $\,\,$2.5602$\,\,$ & $\,\,$5.7852$\,\,$ & $\,\,$\color{gr} \color{blue} 6\color{black} $\,\,$ \\
$\,\,$0.3906$\,\,$ & $\,\,$ 1 $\,\,$ & $\,\,$2.2596$\,\,$ & $\,\,$\color{gr} 2.3435\color{black}   $\,\,$ \\
$\,\,$0.1729$\,\,$ & $\,\,$0.4426$\,\,$ & $\,\,$ 1 $\,\,$ & $\,\,$\color{gr} 1.0371\color{black}  $\,\,$ \\
$\,\,$\color{gr} \color{blue}  1/6\color{black} $\,\,$ & $\,\,$\color{gr} 0.4267\color{black} $\,\,$ & $\,\,$\color{gr} 0.9642\color{black} $\,\,$ & $\,\,$ 1  $\,\,$ \\
\end{pmatrix},
\end{equation*}
\end{example}
\newpage
\begin{example}
\begin{equation*}
\mathbf{A} =
\begin{pmatrix}
$\,\,$ 1 $\,\,$ & $\,\,$5$\,\,$ & $\,\,$5$\,\,$ & $\,\,$6 $\,\,$ \\
$\,\,$ 1/5$\,\,$ & $\,\,$ 1 $\,\,$ & $\,\,$2$\,\,$ & $\,\,$6 $\,\,$ \\
$\,\,$ 1/5$\,\,$ & $\,\,$ 1/2$\,\,$ & $\,\,$ 1 $\,\,$ & $\,\,$2 $\,\,$ \\
$\,\,$ 1/6$\,\,$ & $\,\,$ 1/6$\,\,$ & $\,\,$ 1/2$\,\,$ & $\,\,$ 1  $\,\,$ \\
\end{pmatrix},
\qquad
\lambda_{\max} =
4.2277,
\qquad
CR = 0.0859
\end{equation*}

\begin{equation*}
\mathbf{w}^{cos} =
\begin{pmatrix}
0.582134\\
0.237016\\
\color{red} 0.115373\color{black} \\
0.065477
\end{pmatrix}\end{equation*}
\begin{equation*}
\left[ \frac{{w}^{cos}_i}{{w}^{cos}_j} \right] =
\begin{pmatrix}
$\,\,$ 1 $\,\,$ & $\,\,$2.4561$\,\,$ & $\,\,$\color{red} 5.0457\color{black} $\,\,$ & $\,\,$8.8907$\,\,$ \\
$\,\,$0.4072$\,\,$ & $\,\,$ 1 $\,\,$ & $\,\,$\color{red} 2.0544\color{black} $\,\,$ & $\,\,$3.6199  $\,\,$ \\
$\,\,$\color{red} 0.1982\color{black} $\,\,$ & $\,\,$\color{red} 0.4868\color{black} $\,\,$ & $\,\,$ 1 $\,\,$ & $\,\,$\color{red} 1.7620\color{black}  $\,\,$ \\
$\,\,$0.1125$\,\,$ & $\,\,$0.2763$\,\,$ & $\,\,$\color{red} 0.5675\color{black} $\,\,$ & $\,\,$ 1  $\,\,$ \\
\end{pmatrix},
\end{equation*}

\begin{equation*}
\mathbf{w}^{\prime} =
\begin{pmatrix}
0.581521\\
0.236767\\
0.116304\\
0.065408
\end{pmatrix} =
0.998947\cdot
\begin{pmatrix}
0.582134\\
0.237016\\
\color{gr} 0.116427\color{black} \\
0.065477
\end{pmatrix},
\end{equation*}
\begin{equation*}
\left[ \frac{{w}^{\prime}_i}{{w}^{\prime}_j} \right] =
\begin{pmatrix}
$\,\,$ 1 $\,\,$ & $\,\,$2.4561$\,\,$ & $\,\,$\color{gr} \color{blue} 5\color{black} $\,\,$ & $\,\,$8.8907$\,\,$ \\
$\,\,$0.4072$\,\,$ & $\,\,$ 1 $\,\,$ & $\,\,$\color{gr} 2.0358\color{black} $\,\,$ & $\,\,$3.6199  $\,\,$ \\
$\,\,$\color{gr} \color{blue}  1/5\color{black} $\,\,$ & $\,\,$\color{gr} 0.4912\color{black} $\,\,$ & $\,\,$ 1 $\,\,$ & $\,\,$\color{gr} 1.7781\color{black}  $\,\,$ \\
$\,\,$0.1125$\,\,$ & $\,\,$0.2763$\,\,$ & $\,\,$\color{gr} 0.5624\color{black} $\,\,$ & $\,\,$ 1  $\,\,$ \\
\end{pmatrix},
\end{equation*}
\end{example}
\newpage
\begin{example}
\begin{equation*}
\mathbf{A} =
\begin{pmatrix}
$\,\,$ 1 $\,\,$ & $\,\,$5$\,\,$ & $\,\,$5$\,\,$ & $\,\,$7 $\,\,$ \\
$\,\,$ 1/5$\,\,$ & $\,\,$ 1 $\,\,$ & $\,\,$2$\,\,$ & $\,\,$6 $\,\,$ \\
$\,\,$ 1/5$\,\,$ & $\,\,$ 1/2$\,\,$ & $\,\,$ 1 $\,\,$ & $\,\,$2 $\,\,$ \\
$\,\,$ 1/7$\,\,$ & $\,\,$ 1/6$\,\,$ & $\,\,$ 1/2$\,\,$ & $\,\,$ 1  $\,\,$ \\
\end{pmatrix},
\qquad
\lambda_{\max} =
4.1868,
\qquad
CR = 0.0704
\end{equation*}

\begin{equation*}
\mathbf{w}^{cos} =
\begin{pmatrix}
0.595610\\
0.230178\\
\color{red} 0.113410\color{black} \\
0.060802
\end{pmatrix}\end{equation*}
\begin{equation*}
\left[ \frac{{w}^{cos}_i}{{w}^{cos}_j} \right] =
\begin{pmatrix}
$\,\,$ 1 $\,\,$ & $\,\,$2.5876$\,\,$ & $\,\,$\color{red} 5.2518\color{black} $\,\,$ & $\,\,$9.7958$\,\,$ \\
$\,\,$0.3865$\,\,$ & $\,\,$ 1 $\,\,$ & $\,\,$\color{red} 2.0296\color{black} $\,\,$ & $\,\,$3.7857  $\,\,$ \\
$\,\,$\color{red} 0.1904\color{black} $\,\,$ & $\,\,$\color{red} 0.4927\color{black} $\,\,$ & $\,\,$ 1 $\,\,$ & $\,\,$\color{red} 1.8652\color{black}  $\,\,$ \\
$\,\,$0.1021$\,\,$ & $\,\,$0.2642$\,\,$ & $\,\,$\color{red} 0.5361\color{black} $\,\,$ & $\,\,$ 1  $\,\,$ \\
\end{pmatrix},
\end{equation*}

\begin{equation*}
\mathbf{w}^{\prime} =
\begin{pmatrix}
0.594612\\
0.229792\\
0.114896\\
0.060700
\end{pmatrix} =
0.998324\cdot
\begin{pmatrix}
0.595610\\
0.230178\\
\color{gr} 0.115089\color{black} \\
0.060802
\end{pmatrix},
\end{equation*}
\begin{equation*}
\left[ \frac{{w}^{\prime}_i}{{w}^{\prime}_j} \right] =
\begin{pmatrix}
$\,\,$ 1 $\,\,$ & $\,\,$2.5876$\,\,$ & $\,\,$\color{gr} 5.1752\color{black} $\,\,$ & $\,\,$9.7958$\,\,$ \\
$\,\,$0.3865$\,\,$ & $\,\,$ 1 $\,\,$ & $\,\,$\color{gr} \color{blue} 2\color{black} $\,\,$ & $\,\,$3.7857  $\,\,$ \\
$\,\,$\color{gr} 0.1932\color{black} $\,\,$ & $\,\,$\color{gr} \color{blue}  1/2\color{black} $\,\,$ & $\,\,$ 1 $\,\,$ & $\,\,$\color{gr} 1.8928\color{black}  $\,\,$ \\
$\,\,$0.1021$\,\,$ & $\,\,$0.2642$\,\,$ & $\,\,$\color{gr} 0.5283\color{black} $\,\,$ & $\,\,$ 1  $\,\,$ \\
\end{pmatrix},
\end{equation*}
\end{example}
\newpage
\begin{example}
\begin{equation*}
\mathbf{A} =
\begin{pmatrix}
$\,\,$ 1 $\,\,$ & $\,\,$5$\,\,$ & $\,\,$5$\,\,$ & $\,\,$7 $\,\,$ \\
$\,\,$ 1/5$\,\,$ & $\,\,$ 1 $\,\,$ & $\,\,$2$\,\,$ & $\,\,$7 $\,\,$ \\
$\,\,$ 1/5$\,\,$ & $\,\,$ 1/2$\,\,$ & $\,\,$ 1 $\,\,$ & $\,\,$2 $\,\,$ \\
$\,\,$ 1/7$\,\,$ & $\,\,$ 1/7$\,\,$ & $\,\,$ 1/2$\,\,$ & $\,\,$ 1  $\,\,$ \\
\end{pmatrix},
\qquad
\lambda_{\max} =
4.2287,
\qquad
CR = 0.0862
\end{equation*}

\begin{equation*}
\mathbf{w}^{cos} =
\begin{pmatrix}
0.589180\\
0.240362\\
\color{red} 0.111417\color{black} \\
0.059041
\end{pmatrix}\end{equation*}
\begin{equation*}
\left[ \frac{{w}^{cos}_i}{{w}^{cos}_j} \right] =
\begin{pmatrix}
$\,\,$ 1 $\,\,$ & $\,\,$2.4512$\,\,$ & $\,\,$\color{red} 5.2881\color{black} $\,\,$ & $\,\,$9.9792$\,\,$ \\
$\,\,$0.4080$\,\,$ & $\,\,$ 1 $\,\,$ & $\,\,$\color{red} 2.1573\color{black} $\,\,$ & $\,\,$4.0711  $\,\,$ \\
$\,\,$\color{red} 0.1891\color{black} $\,\,$ & $\,\,$\color{red} 0.4635\color{black} $\,\,$ & $\,\,$ 1 $\,\,$ & $\,\,$\color{red} 1.8871\color{black}  $\,\,$ \\
$\,\,$0.1002$\,\,$ & $\,\,$0.2456$\,\,$ & $\,\,$\color{red} 0.5299\color{black} $\,\,$ & $\,\,$ 1  $\,\,$ \\
\end{pmatrix},
\end{equation*}

\begin{equation*}
\mathbf{w}^{\prime} =
\begin{pmatrix}
0.585423\\
0.238829\\
0.117085\\
0.058664
\end{pmatrix} =
0.993622\cdot
\begin{pmatrix}
0.589180\\
0.240362\\
\color{gr} 0.117836\color{black} \\
0.059041
\end{pmatrix},
\end{equation*}
\begin{equation*}
\left[ \frac{{w}^{\prime}_i}{{w}^{\prime}_j} \right] =
\begin{pmatrix}
$\,\,$ 1 $\,\,$ & $\,\,$2.4512$\,\,$ & $\,\,$\color{gr} \color{blue} 5\color{black} $\,\,$ & $\,\,$9.9792$\,\,$ \\
$\,\,$0.4080$\,\,$ & $\,\,$ 1 $\,\,$ & $\,\,$\color{gr} 2.0398\color{black} $\,\,$ & $\,\,$4.0711  $\,\,$ \\
$\,\,$\color{gr} \color{blue}  1/5\color{black} $\,\,$ & $\,\,$\color{gr} 0.4902\color{black} $\,\,$ & $\,\,$ 1 $\,\,$ & $\,\,$\color{gr} 1.9958\color{black}  $\,\,$ \\
$\,\,$0.1002$\,\,$ & $\,\,$0.2456$\,\,$ & $\,\,$\color{gr} 0.5010\color{black} $\,\,$ & $\,\,$ 1  $\,\,$ \\
\end{pmatrix},
\end{equation*}
\end{example}
\newpage
\begin{example}
\begin{equation*}
\mathbf{A} =
\begin{pmatrix}
$\,\,$ 1 $\,\,$ & $\,\,$5$\,\,$ & $\,\,$5$\,\,$ & $\,\,$7 $\,\,$ \\
$\,\,$ 1/5$\,\,$ & $\,\,$ 1 $\,\,$ & $\,\,$5$\,\,$ & $\,\,$3 $\,\,$ \\
$\,\,$ 1/5$\,\,$ & $\,\,$ 1/5$\,\,$ & $\,\,$ 1 $\,\,$ & $\,\,$1 $\,\,$ \\
$\,\,$ 1/7$\,\,$ & $\,\,$ 1/3$\,\,$ & $\,\,$ 1 $\,\,$ & $\,\,$ 1  $\,\,$ \\
\end{pmatrix},
\qquad
\lambda_{\max} =
4.2309,
\qquad
CR = 0.0871
\end{equation*}

\begin{equation*}
\mathbf{w}^{cos} =
\begin{pmatrix}
0.591798\\
0.246052\\
0.083457\\
\color{red} 0.078693\color{black}
\end{pmatrix}\end{equation*}
\begin{equation*}
\left[ \frac{{w}^{cos}_i}{{w}^{cos}_j} \right] =
\begin{pmatrix}
$\,\,$ 1 $\,\,$ & $\,\,$2.4052$\,\,$ & $\,\,$7.0911$\,\,$ & $\,\,$\color{red} 7.5203\color{black} $\,\,$ \\
$\,\,$0.4158$\,\,$ & $\,\,$ 1 $\,\,$ & $\,\,$2.9483$\,\,$ & $\,\,$\color{red} 3.1267\color{black}   $\,\,$ \\
$\,\,$0.1410$\,\,$ & $\,\,$0.3392$\,\,$ & $\,\,$ 1 $\,\,$ & $\,\,$\color{red} 1.0605\color{black}  $\,\,$ \\
$\,\,$\color{red} 0.1330\color{black} $\,\,$ & $\,\,$\color{red} 0.3198\color{black} $\,\,$ & $\,\,$\color{red} 0.9429\color{black} $\,\,$ & $\,\,$ 1  $\,\,$ \\
\end{pmatrix},
\end{equation*}

\begin{equation*}
\mathbf{w}^{\prime} =
\begin{pmatrix}
0.589837\\
0.245237\\
0.083180\\
0.081746
\end{pmatrix} =
0.996687\cdot
\begin{pmatrix}
0.591798\\
0.246052\\
0.083457\\
\color{gr} 0.082017\color{black}
\end{pmatrix},
\end{equation*}
\begin{equation*}
\left[ \frac{{w}^{\prime}_i}{{w}^{\prime}_j} \right] =
\begin{pmatrix}
$\,\,$ 1 $\,\,$ & $\,\,$2.4052$\,\,$ & $\,\,$7.0911$\,\,$ & $\,\,$\color{gr} 7.2155\color{black} $\,\,$ \\
$\,\,$0.4158$\,\,$ & $\,\,$ 1 $\,\,$ & $\,\,$2.9483$\,\,$ & $\,\,$\color{gr} \color{blue} 3\color{black}   $\,\,$ \\
$\,\,$0.1410$\,\,$ & $\,\,$0.3392$\,\,$ & $\,\,$ 1 $\,\,$ & $\,\,$\color{gr} 1.0176\color{black}  $\,\,$ \\
$\,\,$\color{gr} 0.1386\color{black} $\,\,$ & $\,\,$\color{gr} \color{blue}  1/3\color{black} $\,\,$ & $\,\,$\color{gr} 0.9828\color{black} $\,\,$ & $\,\,$ 1  $\,\,$ \\
\end{pmatrix},
\end{equation*}
\end{example}
\newpage
\begin{example}
\begin{equation*}
\mathbf{A} =
\begin{pmatrix}
$\,\,$ 1 $\,\,$ & $\,\,$5$\,\,$ & $\,\,$5$\,\,$ & $\,\,$8 $\,\,$ \\
$\,\,$ 1/5$\,\,$ & $\,\,$ 1 $\,\,$ & $\,\,$2$\,\,$ & $\,\,$6 $\,\,$ \\
$\,\,$ 1/5$\,\,$ & $\,\,$ 1/2$\,\,$ & $\,\,$ 1 $\,\,$ & $\,\,$2 $\,\,$ \\
$\,\,$ 1/8$\,\,$ & $\,\,$ 1/6$\,\,$ & $\,\,$ 1/2$\,\,$ & $\,\,$ 1  $\,\,$ \\
\end{pmatrix},
\qquad
\lambda_{\max} =
4.1569,
\qquad
CR = 0.0592
\end{equation*}

\begin{equation*}
\mathbf{w}^{cos} =
\begin{pmatrix}
0.607280\\
0.223976\\
\color{red} 0.111643\color{black} \\
0.057101
\end{pmatrix}\end{equation*}
\begin{equation*}
\left[ \frac{{w}^{cos}_i}{{w}^{cos}_j} \right] =
\begin{pmatrix}
$\,\,$ 1 $\,\,$ & $\,\,$2.7114$\,\,$ & $\,\,$\color{red} 5.4395\color{black} $\,\,$ & $\,\,$10.6352$\,\,$ \\
$\,\,$0.3688$\,\,$ & $\,\,$ 1 $\,\,$ & $\,\,$\color{red} 2.0062\color{black} $\,\,$ & $\,\,$3.9224  $\,\,$ \\
$\,\,$\color{red} 0.1838\color{black} $\,\,$ & $\,\,$\color{red} 0.4985\color{black} $\,\,$ & $\,\,$ 1 $\,\,$ & $\,\,$\color{red} 1.9552\color{black}  $\,\,$ \\
$\,\,$0.0940$\,\,$ & $\,\,$0.2549$\,\,$ & $\,\,$\color{red} 0.5115\color{black} $\,\,$ & $\,\,$ 1  $\,\,$ \\
\end{pmatrix},
\end{equation*}

\begin{equation*}
\mathbf{w}^{\prime} =
\begin{pmatrix}
0.607071\\
0.223898\\
0.111949\\
0.057081
\end{pmatrix} =
0.999656\cdot
\begin{pmatrix}
0.607280\\
0.223976\\
\color{gr} 0.111988\color{black} \\
0.057101
\end{pmatrix},
\end{equation*}
\begin{equation*}
\left[ \frac{{w}^{\prime}_i}{{w}^{\prime}_j} \right] =
\begin{pmatrix}
$\,\,$ 1 $\,\,$ & $\,\,$2.7114$\,\,$ & $\,\,$\color{gr} 5.4227\color{black} $\,\,$ & $\,\,$10.6352$\,\,$ \\
$\,\,$0.3688$\,\,$ & $\,\,$ 1 $\,\,$ & $\,\,$\color{gr} \color{blue} 2\color{black} $\,\,$ & $\,\,$3.9224  $\,\,$ \\
$\,\,$\color{gr} 0.1844\color{black} $\,\,$ & $\,\,$\color{gr} \color{blue}  1/2\color{black} $\,\,$ & $\,\,$ 1 $\,\,$ & $\,\,$\color{gr} 1.9612\color{black}  $\,\,$ \\
$\,\,$0.0940$\,\,$ & $\,\,$0.2549$\,\,$ & $\,\,$\color{gr} 0.5099\color{black} $\,\,$ & $\,\,$ 1  $\,\,$ \\
\end{pmatrix},
\end{equation*}
\end{example}
\newpage
\begin{example}
\begin{equation*}
\mathbf{A} =
\begin{pmatrix}
$\,\,$ 1 $\,\,$ & $\,\,$5$\,\,$ & $\,\,$5$\,\,$ & $\,\,$8 $\,\,$ \\
$\,\,$ 1/5$\,\,$ & $\,\,$ 1 $\,\,$ & $\,\,$2$\,\,$ & $\,\,$7 $\,\,$ \\
$\,\,$ 1/5$\,\,$ & $\,\,$ 1/2$\,\,$ & $\,\,$ 1 $\,\,$ & $\,\,$2 $\,\,$ \\
$\,\,$ 1/8$\,\,$ & $\,\,$ 1/7$\,\,$ & $\,\,$ 1/2$\,\,$ & $\,\,$ 1  $\,\,$ \\
\end{pmatrix},
\qquad
\lambda_{\max} =
4.1955,
\qquad
CR = 0.0737
\end{equation*}

\begin{equation*}
\mathbf{w}^{cos} =
\begin{pmatrix}
0.600530\\
0.234107\\
\color{red} 0.109909\color{black} \\
0.055455
\end{pmatrix}\end{equation*}
\begin{equation*}
\left[ \frac{{w}^{cos}_i}{{w}^{cos}_j} \right] =
\begin{pmatrix}
$\,\,$ 1 $\,\,$ & $\,\,$2.5652$\,\,$ & $\,\,$\color{red} 5.4639\color{black} $\,\,$ & $\,\,$10.8292$\,\,$ \\
$\,\,$0.3898$\,\,$ & $\,\,$ 1 $\,\,$ & $\,\,$\color{red} 2.1300\color{black} $\,\,$ & $\,\,$4.2216  $\,\,$ \\
$\,\,$\color{red} 0.1830\color{black} $\,\,$ & $\,\,$\color{red} 0.4695\color{black} $\,\,$ & $\,\,$ 1 $\,\,$ & $\,\,$\color{red} 1.9820\color{black}  $\,\,$ \\
$\,\,$0.0923$\,\,$ & $\,\,$0.2369$\,\,$ & $\,\,$\color{red} 0.5046\color{black} $\,\,$ & $\,\,$ 1  $\,\,$ \\
\end{pmatrix},
\end{equation*}

\begin{equation*}
\mathbf{w}^{\prime} =
\begin{pmatrix}
0.599929\\
0.233873\\
0.110799\\
0.055399
\end{pmatrix} =
0.999000\cdot
\begin{pmatrix}
0.600530\\
0.234107\\
\color{gr} 0.110910\color{black} \\
0.055455
\end{pmatrix},
\end{equation*}
\begin{equation*}
\left[ \frac{{w}^{\prime}_i}{{w}^{\prime}_j} \right] =
\begin{pmatrix}
$\,\,$ 1 $\,\,$ & $\,\,$2.5652$\,\,$ & $\,\,$\color{gr} 5.4146\color{black} $\,\,$ & $\,\,$10.8292$\,\,$ \\
$\,\,$0.3898$\,\,$ & $\,\,$ 1 $\,\,$ & $\,\,$\color{gr} 2.1108\color{black} $\,\,$ & $\,\,$4.2216  $\,\,$ \\
$\,\,$\color{gr} 0.1847\color{black} $\,\,$ & $\,\,$\color{gr} 0.4738\color{black} $\,\,$ & $\,\,$ 1 $\,\,$ & $\,\,$\color{gr} \color{blue} 2\color{black}  $\,\,$ \\
$\,\,$0.0923$\,\,$ & $\,\,$0.2369$\,\,$ & $\,\,$\color{gr} \color{blue}  1/2\color{black} $\,\,$ & $\,\,$ 1  $\,\,$ \\
\end{pmatrix},
\end{equation*}
\end{example}
\newpage
\begin{example}
\begin{equation*}
\mathbf{A} =
\begin{pmatrix}
$\,\,$ 1 $\,\,$ & $\,\,$5$\,\,$ & $\,\,$5$\,\,$ & $\,\,$8 $\,\,$ \\
$\,\,$ 1/5$\,\,$ & $\,\,$ 1 $\,\,$ & $\,\,$5$\,\,$ & $\,\,$3 $\,\,$ \\
$\,\,$ 1/5$\,\,$ & $\,\,$ 1/5$\,\,$ & $\,\,$ 1 $\,\,$ & $\,\,$1 $\,\,$ \\
$\,\,$ 1/8$\,\,$ & $\,\,$ 1/3$\,\,$ & $\,\,$ 1 $\,\,$ & $\,\,$ 1  $\,\,$ \\
\end{pmatrix},
\qquad
\lambda_{\max} =
4.2259,
\qquad
CR = 0.0852
\end{equation*}

\begin{equation*}
\mathbf{w}^{cos} =
\begin{pmatrix}
0.601764\\
0.241775\\
0.082093\\
\color{red} 0.074368\color{black}
\end{pmatrix}\end{equation*}
\begin{equation*}
\left[ \frac{{w}^{cos}_i}{{w}^{cos}_j} \right] =
\begin{pmatrix}
$\,\,$ 1 $\,\,$ & $\,\,$2.4889$\,\,$ & $\,\,$7.3303$\,\,$ & $\,\,$\color{red} 8.0917\color{black} $\,\,$ \\
$\,\,$0.4018$\,\,$ & $\,\,$ 1 $\,\,$ & $\,\,$2.9451$\,\,$ & $\,\,$\color{red} 3.2511\color{black}   $\,\,$ \\
$\,\,$0.1364$\,\,$ & $\,\,$0.3395$\,\,$ & $\,\,$ 1 $\,\,$ & $\,\,$\color{red} 1.1039\color{black}  $\,\,$ \\
$\,\,$\color{red} 0.1236\color{black} $\,\,$ & $\,\,$\color{red} 0.3076\color{black} $\,\,$ & $\,\,$\color{red} 0.9059\color{black} $\,\,$ & $\,\,$ 1  $\,\,$ \\
\end{pmatrix},
\end{equation*}

\begin{equation*}
\mathbf{w}^{\prime} =
\begin{pmatrix}
0.601251\\
0.241569\\
0.082023\\
0.075156
\end{pmatrix} =
0.999148\cdot
\begin{pmatrix}
0.601764\\
0.241775\\
0.082093\\
\color{gr} 0.075221\color{black}
\end{pmatrix},
\end{equation*}
\begin{equation*}
\left[ \frac{{w}^{\prime}_i}{{w}^{\prime}_j} \right] =
\begin{pmatrix}
$\,\,$ 1 $\,\,$ & $\,\,$2.4889$\,\,$ & $\,\,$7.3303$\,\,$ & $\,\,$\color{gr} \color{blue} 8\color{black} $\,\,$ \\
$\,\,$0.4018$\,\,$ & $\,\,$ 1 $\,\,$ & $\,\,$2.9451$\,\,$ & $\,\,$\color{gr} 3.2142\color{black}   $\,\,$ \\
$\,\,$0.1364$\,\,$ & $\,\,$0.3395$\,\,$ & $\,\,$ 1 $\,\,$ & $\,\,$\color{gr} 1.0914\color{black}  $\,\,$ \\
$\,\,$\color{gr} \color{blue}  1/8\color{black} $\,\,$ & $\,\,$\color{gr} 0.3111\color{black} $\,\,$ & $\,\,$\color{gr} 0.9163\color{black} $\,\,$ & $\,\,$ 1  $\,\,$ \\
\end{pmatrix},
\end{equation*}
\end{example}
\newpage
\begin{example}
\begin{equation*}
\mathbf{A} =
\begin{pmatrix}
$\,\,$ 1 $\,\,$ & $\,\,$5$\,\,$ & $\,\,$5$\,\,$ & $\,\,$9 $\,\,$ \\
$\,\,$ 1/5$\,\,$ & $\,\,$ 1 $\,\,$ & $\,\,$2$\,\,$ & $\,\,$9 $\,\,$ \\
$\,\,$ 1/5$\,\,$ & $\,\,$ 1/2$\,\,$ & $\,\,$ 1 $\,\,$ & $\,\,$3 $\,\,$ \\
$\,\,$ 1/9$\,\,$ & $\,\,$ 1/9$\,\,$ & $\,\,$ 1/3$\,\,$ & $\,\,$ 1  $\,\,$ \\
\end{pmatrix},
\qquad
\lambda_{\max} =
4.2277,
\qquad
CR = 0.0859
\end{equation*}

\begin{equation*}
\mathbf{w}^{cos} =
\begin{pmatrix}
0.595059\\
0.242281\\
\color{red} 0.117989\color{black} \\
0.044671
\end{pmatrix}\end{equation*}
\begin{equation*}
\left[ \frac{{w}^{cos}_i}{{w}^{cos}_j} \right] =
\begin{pmatrix}
$\,\,$ 1 $\,\,$ & $\,\,$2.4561$\,\,$ & $\,\,$\color{red} 5.0433\color{black} $\,\,$ & $\,\,$13.3210$\,\,$ \\
$\,\,$0.4072$\,\,$ & $\,\,$ 1 $\,\,$ & $\,\,$\color{red} 2.0534\color{black} $\,\,$ & $\,\,$5.4237  $\,\,$ \\
$\,\,$\color{red} 0.1983\color{black} $\,\,$ & $\,\,$\color{red} 0.4870\color{black} $\,\,$ & $\,\,$ 1 $\,\,$ & $\,\,$\color{red} 2.6413\color{black}  $\,\,$ \\
$\,\,$0.0751$\,\,$ & $\,\,$0.1844$\,\,$ & $\,\,$\color{red} 0.3786\color{black} $\,\,$ & $\,\,$ 1  $\,\,$ \\
\end{pmatrix},
\end{equation*}

\begin{equation*}
\mathbf{w}^{\prime} =
\begin{pmatrix}
0.594451\\
0.242033\\
0.118890\\
0.044625
\end{pmatrix} =
0.998978\cdot
\begin{pmatrix}
0.595059\\
0.242281\\
\color{gr} 0.119012\color{black} \\
0.044671
\end{pmatrix},
\end{equation*}
\begin{equation*}
\left[ \frac{{w}^{\prime}_i}{{w}^{\prime}_j} \right] =
\begin{pmatrix}
$\,\,$ 1 $\,\,$ & $\,\,$2.4561$\,\,$ & $\,\,$\color{gr} \color{blue} 5\color{black} $\,\,$ & $\,\,$13.3210$\,\,$ \\
$\,\,$0.4072$\,\,$ & $\,\,$ 1 $\,\,$ & $\,\,$\color{gr} 2.0358\color{black} $\,\,$ & $\,\,$5.4237  $\,\,$ \\
$\,\,$\color{gr} \color{blue}  1/5\color{black} $\,\,$ & $\,\,$\color{gr} 0.4912\color{black} $\,\,$ & $\,\,$ 1 $\,\,$ & $\,\,$\color{gr} 2.6642\color{black}  $\,\,$ \\
$\,\,$0.0751$\,\,$ & $\,\,$0.1844$\,\,$ & $\,\,$\color{gr} 0.3753\color{black} $\,\,$ & $\,\,$ 1  $\,\,$ \\
\end{pmatrix},
\end{equation*}
\end{example}
\newpage
\begin{example}
\begin{equation*}
\mathbf{A} =
\begin{pmatrix}
$\,\,$ 1 $\,\,$ & $\,\,$5$\,\,$ & $\,\,$6$\,\,$ & $\,\,$8 $\,\,$ \\
$\,\,$ 1/5$\,\,$ & $\,\,$ 1 $\,\,$ & $\,\,$2$\,\,$ & $\,\,$7 $\,\,$ \\
$\,\,$ 1/6$\,\,$ & $\,\,$ 1/2$\,\,$ & $\,\,$ 1 $\,\,$ & $\,\,$2 $\,\,$ \\
$\,\,$ 1/8$\,\,$ & $\,\,$ 1/7$\,\,$ & $\,\,$ 1/2$\,\,$ & $\,\,$ 1  $\,\,$ \\
\end{pmatrix},
\qquad
\lambda_{\max} =
4.1888,
\qquad
CR = 0.0712
\end{equation*}

\begin{equation*}
\mathbf{w}^{cos} =
\begin{pmatrix}
0.614955\\
0.228982\\
\color{red} 0.101868\color{black} \\
0.054195
\end{pmatrix}\end{equation*}
\begin{equation*}
\left[ \frac{{w}^{cos}_i}{{w}^{cos}_j} \right] =
\begin{pmatrix}
$\,\,$ 1 $\,\,$ & $\,\,$2.6856$\,\,$ & $\,\,$\color{red} 6.0368\color{black} $\,\,$ & $\,\,$11.3471$\,\,$ \\
$\,\,$0.3724$\,\,$ & $\,\,$ 1 $\,\,$ & $\,\,$\color{red} 2.2478\color{black} $\,\,$ & $\,\,$4.2251  $\,\,$ \\
$\,\,$\color{red} 0.1657\color{black} $\,\,$ & $\,\,$\color{red} 0.4449\color{black} $\,\,$ & $\,\,$ 1 $\,\,$ & $\,\,$\color{red} 1.8797\color{black}  $\,\,$ \\
$\,\,$0.0881$\,\,$ & $\,\,$0.2367$\,\,$ & $\,\,$\color{red} 0.5320\color{black} $\,\,$ & $\,\,$ 1  $\,\,$ \\
\end{pmatrix},
\end{equation*}

\begin{equation*}
\mathbf{w}^{\prime} =
\begin{pmatrix}
0.614572\\
0.228839\\
0.102429\\
0.054161
\end{pmatrix} =
0.999376\cdot
\begin{pmatrix}
0.614955\\
0.228982\\
\color{gr} 0.102493\color{black} \\
0.054195
\end{pmatrix},
\end{equation*}
\begin{equation*}
\left[ \frac{{w}^{\prime}_i}{{w}^{\prime}_j} \right] =
\begin{pmatrix}
$\,\,$ 1 $\,\,$ & $\,\,$2.6856$\,\,$ & $\,\,$\color{gr} \color{blue} 6\color{black} $\,\,$ & $\,\,$11.3471$\,\,$ \\
$\,\,$0.3724$\,\,$ & $\,\,$ 1 $\,\,$ & $\,\,$\color{gr} 2.2341\color{black} $\,\,$ & $\,\,$4.2251  $\,\,$ \\
$\,\,$\color{gr} \color{blue}  1/6\color{black} $\,\,$ & $\,\,$\color{gr} 0.4476\color{black} $\,\,$ & $\,\,$ 1 $\,\,$ & $\,\,$\color{gr} 1.8912\color{black}  $\,\,$ \\
$\,\,$0.0881$\,\,$ & $\,\,$0.2367$\,\,$ & $\,\,$\color{gr} 0.5288\color{black} $\,\,$ & $\,\,$ 1  $\,\,$ \\
\end{pmatrix},
\end{equation*}
\end{example}
\newpage
\begin{example}
\begin{equation*}
\mathbf{A} =
\begin{pmatrix}
$\,\,$ 1 $\,\,$ & $\,\,$5$\,\,$ & $\,\,$6$\,\,$ & $\,\,$8 $\,\,$ \\
$\,\,$ 1/5$\,\,$ & $\,\,$ 1 $\,\,$ & $\,\,$2$\,\,$ & $\,\,$8 $\,\,$ \\
$\,\,$ 1/6$\,\,$ & $\,\,$ 1/2$\,\,$ & $\,\,$ 1 $\,\,$ & $\,\,$2 $\,\,$ \\
$\,\,$ 1/8$\,\,$ & $\,\,$ 1/8$\,\,$ & $\,\,$ 1/2$\,\,$ & $\,\,$ 1  $\,\,$ \\
\end{pmatrix},
\qquad
\lambda_{\max} =
4.2277,
\qquad
CR = 0.0859
\end{equation*}

\begin{equation*}
\mathbf{w}^{cos} =
\begin{pmatrix}
0.609017\\
0.237910\\
\color{red} 0.100267\color{black} \\
0.052806
\end{pmatrix}\end{equation*}
\begin{equation*}
\left[ \frac{{w}^{cos}_i}{{w}^{cos}_j} \right] =
\begin{pmatrix}
$\,\,$ 1 $\,\,$ & $\,\,$2.5599$\,\,$ & $\,\,$\color{red} 6.0740\color{black} $\,\,$ & $\,\,$11.5331$\,\,$ \\
$\,\,$0.3906$\,\,$ & $\,\,$ 1 $\,\,$ & $\,\,$\color{red} 2.3728\color{black} $\,\,$ & $\,\,$4.5054  $\,\,$ \\
$\,\,$\color{red} 0.1646\color{black} $\,\,$ & $\,\,$\color{red} 0.4214\color{black} $\,\,$ & $\,\,$ 1 $\,\,$ & $\,\,$\color{red} 1.8988\color{black}  $\,\,$ \\
$\,\,$0.0867$\,\,$ & $\,\,$0.2220$\,\,$ & $\,\,$\color{red} 0.5267\color{black} $\,\,$ & $\,\,$ 1  $\,\,$ \\
\end{pmatrix},
\end{equation*}

\begin{equation*}
\mathbf{w}^{\prime} =
\begin{pmatrix}
0.608265\\
0.237617\\
0.101377\\
0.052741
\end{pmatrix} =
0.998766\cdot
\begin{pmatrix}
0.609017\\
0.237910\\
\color{gr} 0.101503\color{black} \\
0.052806
\end{pmatrix},
\end{equation*}
\begin{equation*}
\left[ \frac{{w}^{\prime}_i}{{w}^{\prime}_j} \right] =
\begin{pmatrix}
$\,\,$ 1 $\,\,$ & $\,\,$2.5599$\,\,$ & $\,\,$\color{gr} \color{blue} 6\color{black} $\,\,$ & $\,\,$11.5331$\,\,$ \\
$\,\,$0.3906$\,\,$ & $\,\,$ 1 $\,\,$ & $\,\,$\color{gr} 2.3439\color{black} $\,\,$ & $\,\,$4.5054  $\,\,$ \\
$\,\,$\color{gr} \color{blue}  1/6\color{black} $\,\,$ & $\,\,$\color{gr} 0.4266\color{black} $\,\,$ & $\,\,$ 1 $\,\,$ & $\,\,$\color{gr} 1.9222\color{black}  $\,\,$ \\
$\,\,$0.0867$\,\,$ & $\,\,$0.2220$\,\,$ & $\,\,$\color{gr} 0.5202\color{black} $\,\,$ & $\,\,$ 1  $\,\,$ \\
\end{pmatrix},
\end{equation*}
\end{example}
\newpage
\begin{example}
\begin{equation*}
\mathbf{A} =
\begin{pmatrix}
$\,\,$ 1 $\,\,$ & $\,\,$5$\,\,$ & $\,\,$6$\,\,$ & $\,\,$8 $\,\,$ \\
$\,\,$ 1/5$\,\,$ & $\,\,$ 1 $\,\,$ & $\,\,$5$\,\,$ & $\,\,$3 $\,\,$ \\
$\,\,$ 1/6$\,\,$ & $\,\,$ 1/5$\,\,$ & $\,\,$ 1 $\,\,$ & $\,\,$1 $\,\,$ \\
$\,\,$ 1/8$\,\,$ & $\,\,$ 1/3$\,\,$ & $\,\,$ 1 $\,\,$ & $\,\,$ 1  $\,\,$ \\
\end{pmatrix},
\qquad
\lambda_{\max} =
4.1792,
\qquad
CR = 0.0676
\end{equation*}

\begin{equation*}
\mathbf{w}^{cos} =
\begin{pmatrix}
0.618529\\
0.233185\\
0.075336\\
\color{red} 0.072950\color{black}
\end{pmatrix}\end{equation*}
\begin{equation*}
\left[ \frac{{w}^{cos}_i}{{w}^{cos}_j} \right] =
\begin{pmatrix}
$\,\,$ 1 $\,\,$ & $\,\,$2.6525$\,\,$ & $\,\,$8.2102$\,\,$ & $\,\,$\color{red} 8.4788\color{black} $\,\,$ \\
$\,\,$0.3770$\,\,$ & $\,\,$ 1 $\,\,$ & $\,\,$3.0952$\,\,$ & $\,\,$\color{red} 3.1965\color{black}   $\,\,$ \\
$\,\,$0.1218$\,\,$ & $\,\,$0.3231$\,\,$ & $\,\,$ 1 $\,\,$ & $\,\,$\color{red} 1.0327\color{black}  $\,\,$ \\
$\,\,$\color{red} 0.1179\color{black} $\,\,$ & $\,\,$\color{red} 0.3128\color{black} $\,\,$ & $\,\,$\color{red} 0.9683\color{black} $\,\,$ & $\,\,$ 1  $\,\,$ \\
\end{pmatrix},
\end{equation*}

\begin{equation*}
\mathbf{w}^{\prime} =
\begin{pmatrix}
0.617056\\
0.232630\\
0.075157\\
0.075157
\end{pmatrix} =
0.997620\cdot
\begin{pmatrix}
0.618529\\
0.233185\\
0.075336\\
\color{gr} 0.075336\color{black}
\end{pmatrix},
\end{equation*}
\begin{equation*}
\left[ \frac{{w}^{\prime}_i}{{w}^{\prime}_j} \right] =
\begin{pmatrix}
$\,\,$ 1 $\,\,$ & $\,\,$2.6525$\,\,$ & $\,\,$8.2102$\,\,$ & $\,\,$\color{gr} 8.2102\color{black} $\,\,$ \\
$\,\,$0.3770$\,\,$ & $\,\,$ 1 $\,\,$ & $\,\,$3.0952$\,\,$ & $\,\,$\color{gr} 3.0952\color{black}   $\,\,$ \\
$\,\,$0.1218$\,\,$ & $\,\,$0.3231$\,\,$ & $\,\,$ 1 $\,\,$ & $\,\,$\color{gr} \color{blue} 1\color{black}  $\,\,$ \\
$\,\,$\color{gr} 0.1218\color{black} $\,\,$ & $\,\,$\color{gr} 0.3231\color{black} $\,\,$ & $\,\,$\color{gr} \color{blue} 1\color{black} $\,\,$ & $\,\,$ 1  $\,\,$ \\
\end{pmatrix},
\end{equation*}
\end{example}
\newpage
\begin{example}
\begin{equation*}
\mathbf{A} =
\begin{pmatrix}
$\,\,$ 1 $\,\,$ & $\,\,$5$\,\,$ & $\,\,$6$\,\,$ & $\,\,$8 $\,\,$ \\
$\,\,$ 1/5$\,\,$ & $\,\,$ 1 $\,\,$ & $\,\,$6$\,\,$ & $\,\,$3 $\,\,$ \\
$\,\,$ 1/6$\,\,$ & $\,\,$ 1/6$\,\,$ & $\,\,$ 1 $\,\,$ & $\,\,$1 $\,\,$ \\
$\,\,$ 1/8$\,\,$ & $\,\,$ 1/3$\,\,$ & $\,\,$ 1 $\,\,$ & $\,\,$ 1  $\,\,$ \\
\end{pmatrix},
\qquad
\lambda_{\max} =
4.2311,
\qquad
CR = 0.0871
\end{equation*}

\begin{equation*}
\mathbf{w}^{cos} =
\begin{pmatrix}
0.610309\\
0.245436\\
0.072765\\
\color{red} 0.071491\color{black}
\end{pmatrix}\end{equation*}
\begin{equation*}
\left[ \frac{{w}^{cos}_i}{{w}^{cos}_j} \right] =
\begin{pmatrix}
$\,\,$ 1 $\,\,$ & $\,\,$2.4866$\,\,$ & $\,\,$8.3874$\,\,$ & $\,\,$\color{red} 8.5369\color{black} $\,\,$ \\
$\,\,$0.4022$\,\,$ & $\,\,$ 1 $\,\,$ & $\,\,$3.3730$\,\,$ & $\,\,$\color{red} 3.4331\color{black}   $\,\,$ \\
$\,\,$0.1192$\,\,$ & $\,\,$0.2965$\,\,$ & $\,\,$ 1 $\,\,$ & $\,\,$\color{red} 1.0178\color{black}  $\,\,$ \\
$\,\,$\color{red} 0.1171\color{black} $\,\,$ & $\,\,$\color{red} 0.2913\color{black} $\,\,$ & $\,\,$\color{red} 0.9825\color{black} $\,\,$ & $\,\,$ 1  $\,\,$ \\
\end{pmatrix},
\end{equation*}

\begin{equation*}
\mathbf{w}^{\prime} =
\begin{pmatrix}
0.609532\\
0.245124\\
0.072672\\
0.072672
\end{pmatrix} =
0.998728\cdot
\begin{pmatrix}
0.610309\\
0.245436\\
0.072765\\
\color{gr} 0.072765\color{black}
\end{pmatrix},
\end{equation*}
\begin{equation*}
\left[ \frac{{w}^{\prime}_i}{{w}^{\prime}_j} \right] =
\begin{pmatrix}
$\,\,$ 1 $\,\,$ & $\,\,$2.4866$\,\,$ & $\,\,$8.3874$\,\,$ & $\,\,$\color{gr} 8.3874\color{black} $\,\,$ \\
$\,\,$0.4022$\,\,$ & $\,\,$ 1 $\,\,$ & $\,\,$3.3730$\,\,$ & $\,\,$\color{gr} 3.3730\color{black}   $\,\,$ \\
$\,\,$0.1192$\,\,$ & $\,\,$0.2965$\,\,$ & $\,\,$ 1 $\,\,$ & $\,\,$\color{gr} \color{blue} 1\color{black}  $\,\,$ \\
$\,\,$\color{gr} 0.1192\color{black} $\,\,$ & $\,\,$\color{gr} 0.2965\color{black} $\,\,$ & $\,\,$\color{gr} \color{blue} 1\color{black} $\,\,$ & $\,\,$ 1  $\,\,$ \\
\end{pmatrix},
\end{equation*}
\end{example}
\newpage
\begin{example}
\begin{equation*}
\mathbf{A} =
\begin{pmatrix}
$\,\,$ 1 $\,\,$ & $\,\,$5$\,\,$ & $\,\,$6$\,\,$ & $\,\,$9 $\,\,$ \\
$\,\,$ 1/5$\,\,$ & $\,\,$ 1 $\,\,$ & $\,\,$2$\,\,$ & $\,\,$6 $\,\,$ \\
$\,\,$ 1/6$\,\,$ & $\,\,$ 1/2$\,\,$ & $\,\,$ 1 $\,\,$ & $\,\,$2 $\,\,$ \\
$\,\,$ 1/9$\,\,$ & $\,\,$ 1/6$\,\,$ & $\,\,$ 1/2$\,\,$ & $\,\,$ 1  $\,\,$ \\
\end{pmatrix},
\qquad
\lambda_{\max} =
4.1252,
\qquad
CR = 0.0472
\end{equation*}

\begin{equation*}
\mathbf{w}^{cos} =
\begin{pmatrix}
0.632218\\
0.213014\\
\color{red} 0.101981\color{black} \\
0.052786
\end{pmatrix}\end{equation*}
\begin{equation*}
\left[ \frac{{w}^{cos}_i}{{w}^{cos}_j} \right] =
\begin{pmatrix}
$\,\,$ 1 $\,\,$ & $\,\,$2.9680$\,\,$ & $\,\,$\color{red} 6.1993\color{black} $\,\,$ & $\,\,$11.9769$\,\,$ \\
$\,\,$0.3369$\,\,$ & $\,\,$ 1 $\,\,$ & $\,\,$\color{red} 2.0888\color{black} $\,\,$ & $\,\,$4.0354  $\,\,$ \\
$\,\,$\color{red} 0.1613\color{black} $\,\,$ & $\,\,$\color{red} 0.4788\color{black} $\,\,$ & $\,\,$ 1 $\,\,$ & $\,\,$\color{red} 1.9320\color{black}  $\,\,$ \\
$\,\,$0.0835$\,\,$ & $\,\,$0.2478$\,\,$ & $\,\,$\color{red} 0.5176\color{black} $\,\,$ & $\,\,$ 1  $\,\,$ \\
\end{pmatrix},
\end{equation*}

\begin{equation*}
\mathbf{w}^{\prime} =
\begin{pmatrix}
0.630083\\
0.212295\\
0.105014\\
0.052608
\end{pmatrix} =
0.996623\cdot
\begin{pmatrix}
0.632218\\
0.213014\\
\color{gr} 0.105370\color{black} \\
0.052786
\end{pmatrix},
\end{equation*}
\begin{equation*}
\left[ \frac{{w}^{\prime}_i}{{w}^{\prime}_j} \right] =
\begin{pmatrix}
$\,\,$ 1 $\,\,$ & $\,\,$2.9680$\,\,$ & $\,\,$\color{gr} \color{blue} 6\color{black} $\,\,$ & $\,\,$11.9769$\,\,$ \\
$\,\,$0.3369$\,\,$ & $\,\,$ 1 $\,\,$ & $\,\,$\color{gr} 2.0216\color{black} $\,\,$ & $\,\,$4.0354  $\,\,$ \\
$\,\,$\color{gr} \color{blue}  1/6\color{black} $\,\,$ & $\,\,$\color{gr} 0.4947\color{black} $\,\,$ & $\,\,$ 1 $\,\,$ & $\,\,$\color{gr} 1.9962\color{black}  $\,\,$ \\
$\,\,$0.0835$\,\,$ & $\,\,$0.2478$\,\,$ & $\,\,$\color{gr} 0.5010\color{black} $\,\,$ & $\,\,$ 1  $\,\,$ \\
\end{pmatrix},
\end{equation*}
\end{example}
\newpage
\begin{example}
\begin{equation*}
\mathbf{A} =
\begin{pmatrix}
$\,\,$ 1 $\,\,$ & $\,\,$5$\,\,$ & $\,\,$6$\,\,$ & $\,\,$9 $\,\,$ \\
$\,\,$ 1/5$\,\,$ & $\,\,$ 1 $\,\,$ & $\,\,$2$\,\,$ & $\,\,$7 $\,\,$ \\
$\,\,$ 1/6$\,\,$ & $\,\,$ 1/2$\,\,$ & $\,\,$ 1 $\,\,$ & $\,\,$2 $\,\,$ \\
$\,\,$ 1/9$\,\,$ & $\,\,$ 1/7$\,\,$ & $\,\,$ 1/2$\,\,$ & $\,\,$ 1  $\,\,$ \\
\end{pmatrix},
\qquad
\lambda_{\max} =
4.1610,
\qquad
CR = 0.0607
\end{equation*}

\begin{equation*}
\mathbf{w}^{cos} =
\begin{pmatrix}
0.625220\\
0.223135\\
\color{red} 0.100431\color{black} \\
0.051214
\end{pmatrix}\end{equation*}
\begin{equation*}
\left[ \frac{{w}^{cos}_i}{{w}^{cos}_j} \right] =
\begin{pmatrix}
$\,\,$ 1 $\,\,$ & $\,\,$2.8020$\,\,$ & $\,\,$\color{red} 6.2254\color{black} $\,\,$ & $\,\,$12.2080$\,\,$ \\
$\,\,$0.3569$\,\,$ & $\,\,$ 1 $\,\,$ & $\,\,$\color{red} 2.2218\color{black} $\,\,$ & $\,\,$4.3569  $\,\,$ \\
$\,\,$\color{red} 0.1606\color{black} $\,\,$ & $\,\,$\color{red} 0.4501\color{black} $\,\,$ & $\,\,$ 1 $\,\,$ & $\,\,$\color{red} 1.9610\color{black}  $\,\,$ \\
$\,\,$0.0819$\,\,$ & $\,\,$0.2295$\,\,$ & $\,\,$\color{red} 0.5099\color{black} $\,\,$ & $\,\,$ 1  $\,\,$ \\
\end{pmatrix},
\end{equation*}

\begin{equation*}
\mathbf{w}^{\prime} =
\begin{pmatrix}
0.623974\\
0.222691\\
0.102224\\
0.051112
\end{pmatrix} =
0.998007\cdot
\begin{pmatrix}
0.625220\\
0.223135\\
\color{gr} 0.102428\color{black} \\
0.051214
\end{pmatrix},
\end{equation*}
\begin{equation*}
\left[ \frac{{w}^{\prime}_i}{{w}^{\prime}_j} \right] =
\begin{pmatrix}
$\,\,$ 1 $\,\,$ & $\,\,$2.8020$\,\,$ & $\,\,$\color{gr} 6.1040\color{black} $\,\,$ & $\,\,$12.2080$\,\,$ \\
$\,\,$0.3569$\,\,$ & $\,\,$ 1 $\,\,$ & $\,\,$\color{gr} 2.1785\color{black} $\,\,$ & $\,\,$4.3569  $\,\,$ \\
$\,\,$\color{gr} 0.1638\color{black} $\,\,$ & $\,\,$\color{gr} 0.4590\color{black} $\,\,$ & $\,\,$ 1 $\,\,$ & $\,\,$\color{gr} \color{blue} 2\color{black}  $\,\,$ \\
$\,\,$0.0819$\,\,$ & $\,\,$0.2295$\,\,$ & $\,\,$\color{gr} \color{blue}  1/2\color{black} $\,\,$ & $\,\,$ 1  $\,\,$ \\
\end{pmatrix},
\end{equation*}
\end{example}
\newpage
\begin{example}
\begin{equation*}
\mathbf{A} =
\begin{pmatrix}
$\,\,$ 1 $\,\,$ & $\,\,$5$\,\,$ & $\,\,$6$\,\,$ & $\,\,$9 $\,\,$ \\
$\,\,$ 1/5$\,\,$ & $\,\,$ 1 $\,\,$ & $\,\,$2$\,\,$ & $\,\,$8 $\,\,$ \\
$\,\,$ 1/6$\,\,$ & $\,\,$ 1/2$\,\,$ & $\,\,$ 1 $\,\,$ & $\,\,$2 $\,\,$ \\
$\,\,$ 1/9$\,\,$ & $\,\,$ 1/8$\,\,$ & $\,\,$ 1/2$\,\,$ & $\,\,$ 1  $\,\,$ \\
\end{pmatrix},
\qquad
\lambda_{\max} =
4.1974,
\qquad
CR = 0.0744
\end{equation*}

\begin{equation*}
\mathbf{w}^{cos} =
\begin{pmatrix}
0.619017\\
0.232061\\
\color{red} 0.099014\color{black} \\
0.049908
\end{pmatrix}\end{equation*}
\begin{equation*}
\left[ \frac{{w}^{cos}_i}{{w}^{cos}_j} \right] =
\begin{pmatrix}
$\,\,$ 1 $\,\,$ & $\,\,$2.6675$\,\,$ & $\,\,$\color{red} 6.2518\color{black} $\,\,$ & $\,\,$12.4032$\,\,$ \\
$\,\,$0.3749$\,\,$ & $\,\,$ 1 $\,\,$ & $\,\,$\color{red} 2.3437\color{black} $\,\,$ & $\,\,$4.6498  $\,\,$ \\
$\,\,$\color{red} 0.1600\color{black} $\,\,$ & $\,\,$\color{red} 0.4267\color{black} $\,\,$ & $\,\,$ 1 $\,\,$ & $\,\,$\color{red} 1.9839\color{black}  $\,\,$ \\
$\,\,$0.0806$\,\,$ & $\,\,$0.2151$\,\,$ & $\,\,$\color{red} 0.5041\color{black} $\,\,$ & $\,\,$ 1  $\,\,$ \\
\end{pmatrix},
\end{equation*}

\begin{equation*}
\mathbf{w}^{\prime} =
\begin{pmatrix}
0.618521\\
0.231875\\
0.099736\\
0.049868
\end{pmatrix} =
0.999198\cdot
\begin{pmatrix}
0.619017\\
0.232061\\
\color{gr} 0.099816\color{black} \\
0.049908
\end{pmatrix},
\end{equation*}
\begin{equation*}
\left[ \frac{{w}^{\prime}_i}{{w}^{\prime}_j} \right] =
\begin{pmatrix}
$\,\,$ 1 $\,\,$ & $\,\,$2.6675$\,\,$ & $\,\,$\color{gr} 6.2016\color{black} $\,\,$ & $\,\,$12.4032$\,\,$ \\
$\,\,$0.3749$\,\,$ & $\,\,$ 1 $\,\,$ & $\,\,$\color{gr} 2.3249\color{black} $\,\,$ & $\,\,$4.6498  $\,\,$ \\
$\,\,$\color{gr} 0.1612\color{black} $\,\,$ & $\,\,$\color{gr} 0.4301\color{black} $\,\,$ & $\,\,$ 1 $\,\,$ & $\,\,$\color{gr} \color{blue} 2\color{black}  $\,\,$ \\
$\,\,$0.0806$\,\,$ & $\,\,$0.2151$\,\,$ & $\,\,$\color{gr} \color{blue}  1/2\color{black} $\,\,$ & $\,\,$ 1  $\,\,$ \\
\end{pmatrix},
\end{equation*}
\end{example}
\newpage
\begin{example}
\begin{equation*}
\mathbf{A} =
\begin{pmatrix}
$\,\,$ 1 $\,\,$ & $\,\,$5$\,\,$ & $\,\,$6$\,\,$ & $\,\,$9 $\,\,$ \\
$\,\,$ 1/5$\,\,$ & $\,\,$ 1 $\,\,$ & $\,\,$5$\,\,$ & $\,\,$3 $\,\,$ \\
$\,\,$ 1/6$\,\,$ & $\,\,$ 1/5$\,\,$ & $\,\,$ 1 $\,\,$ & $\,\,$1 $\,\,$ \\
$\,\,$ 1/9$\,\,$ & $\,\,$ 1/3$\,\,$ & $\,\,$ 1 $\,\,$ & $\,\,$ 1  $\,\,$ \\
\end{pmatrix},
\qquad
\lambda_{\max} =
4.1758,
\qquad
CR = 0.0663
\end{equation*}

\begin{equation*}
\mathbf{w}^{cos} =
\begin{pmatrix}
0.626914\\
0.229552\\
0.074129\\
\color{red} 0.069405\color{black}
\end{pmatrix}\end{equation*}
\begin{equation*}
\left[ \frac{{w}^{cos}_i}{{w}^{cos}_j} \right] =
\begin{pmatrix}
$\,\,$ 1 $\,\,$ & $\,\,$2.7310$\,\,$ & $\,\,$8.4571$\,\,$ & $\,\,$\color{red} 9.0326\color{black} $\,\,$ \\
$\,\,$0.3662$\,\,$ & $\,\,$ 1 $\,\,$ & $\,\,$3.0967$\,\,$ & $\,\,$\color{red} 3.3074\color{black}   $\,\,$ \\
$\,\,$0.1182$\,\,$ & $\,\,$0.3229$\,\,$ & $\,\,$ 1 $\,\,$ & $\,\,$\color{red} 1.0681\color{black}  $\,\,$ \\
$\,\,$\color{red} 0.1107\color{black} $\,\,$ & $\,\,$\color{red} 0.3024\color{black} $\,\,$ & $\,\,$\color{red} 0.9363\color{black} $\,\,$ & $\,\,$ 1  $\,\,$ \\
\end{pmatrix},
\end{equation*}

\begin{equation*}
\mathbf{w}^{\prime} =
\begin{pmatrix}
0.626756\\
0.229494\\
0.074110\\
0.069640
\end{pmatrix} =
0.999748\cdot
\begin{pmatrix}
0.626914\\
0.229552\\
0.074129\\
\color{gr} 0.069657\color{black}
\end{pmatrix},
\end{equation*}
\begin{equation*}
\left[ \frac{{w}^{\prime}_i}{{w}^{\prime}_j} \right] =
\begin{pmatrix}
$\,\,$ 1 $\,\,$ & $\,\,$2.7310$\,\,$ & $\,\,$8.4571$\,\,$ & $\,\,$\color{gr} \color{blue} 9\color{black} $\,\,$ \\
$\,\,$0.3662$\,\,$ & $\,\,$ 1 $\,\,$ & $\,\,$3.0967$\,\,$ & $\,\,$\color{gr} 3.2955\color{black}   $\,\,$ \\
$\,\,$0.1182$\,\,$ & $\,\,$0.3229$\,\,$ & $\,\,$ 1 $\,\,$ & $\,\,$\color{gr} 1.0642\color{black}  $\,\,$ \\
$\,\,$\color{gr} \color{blue}  1/9\color{black} $\,\,$ & $\,\,$\color{gr} 0.3034\color{black} $\,\,$ & $\,\,$\color{gr} 0.9397\color{black} $\,\,$ & $\,\,$ 1  $\,\,$ \\
\end{pmatrix},
\end{equation*}
\end{example}
\newpage
\begin{example}
\begin{equation*}
\mathbf{A} =
\begin{pmatrix}
$\,\,$ 1 $\,\,$ & $\,\,$5$\,\,$ & $\,\,$6$\,\,$ & $\,\,$9 $\,\,$ \\
$\,\,$ 1/5$\,\,$ & $\,\,$ 1 $\,\,$ & $\,\,$6$\,\,$ & $\,\,$3 $\,\,$ \\
$\,\,$ 1/6$\,\,$ & $\,\,$ 1/6$\,\,$ & $\,\,$ 1 $\,\,$ & $\,\,$1 $\,\,$ \\
$\,\,$ 1/9$\,\,$ & $\,\,$ 1/3$\,\,$ & $\,\,$ 1 $\,\,$ & $\,\,$ 1  $\,\,$ \\
\end{pmatrix},
\qquad
\lambda_{\max} =
4.2277,
\qquad
CR = 0.0859
\end{equation*}

\begin{equation*}
\mathbf{w}^{cos} =
\begin{pmatrix}
0.618643\\
0.241905\\
0.071530\\
\color{red} 0.067922\color{black}
\end{pmatrix}\end{equation*}
\begin{equation*}
\left[ \frac{{w}^{cos}_i}{{w}^{cos}_j} \right] =
\begin{pmatrix}
$\,\,$ 1 $\,\,$ & $\,\,$2.5574$\,\,$ & $\,\,$8.6487$\,\,$ & $\,\,$\color{red} 9.1081\color{black} $\,\,$ \\
$\,\,$0.3910$\,\,$ & $\,\,$ 1 $\,\,$ & $\,\,$3.3818$\,\,$ & $\,\,$\color{red} 3.5615\color{black}   $\,\,$ \\
$\,\,$0.1156$\,\,$ & $\,\,$0.2957$\,\,$ & $\,\,$ 1 $\,\,$ & $\,\,$\color{red} 1.0531\color{black}  $\,\,$ \\
$\,\,$\color{red} 0.1098\color{black} $\,\,$ & $\,\,$\color{red} 0.2808\color{black} $\,\,$ & $\,\,$\color{red} 0.9496\color{black} $\,\,$ & $\,\,$ 1  $\,\,$ \\
\end{pmatrix},
\end{equation*}

\begin{equation*}
\mathbf{w}^{\prime} =
\begin{pmatrix}
0.618139\\
0.241707\\
0.071472\\
0.068682
\end{pmatrix} =
0.999185\cdot
\begin{pmatrix}
0.618643\\
0.241905\\
0.071530\\
\color{gr} 0.068738\color{black}
\end{pmatrix},
\end{equation*}
\begin{equation*}
\left[ \frac{{w}^{\prime}_i}{{w}^{\prime}_j} \right] =
\begin{pmatrix}
$\,\,$ 1 $\,\,$ & $\,\,$2.5574$\,\,$ & $\,\,$8.6487$\,\,$ & $\,\,$\color{gr} \color{blue} 9\color{black} $\,\,$ \\
$\,\,$0.3910$\,\,$ & $\,\,$ 1 $\,\,$ & $\,\,$3.3818$\,\,$ & $\,\,$\color{gr} 3.5192\color{black}   $\,\,$ \\
$\,\,$0.1156$\,\,$ & $\,\,$0.2957$\,\,$ & $\,\,$ 1 $\,\,$ & $\,\,$\color{gr} 1.0406\color{black}  $\,\,$ \\
$\,\,$\color{gr} \color{blue}  1/9\color{black} $\,\,$ & $\,\,$\color{gr} 0.2842\color{black} $\,\,$ & $\,\,$\color{gr} 0.9610\color{black} $\,\,$ & $\,\,$ 1  $\,\,$ \\
\end{pmatrix},
\end{equation*}
\end{example}
\newpage
\begin{example}
\begin{equation*}
\mathbf{A} =
\begin{pmatrix}
$\,\,$ 1 $\,\,$ & $\,\,$5$\,\,$ & $\,\,$7$\,\,$ & $\,\,$9 $\,\,$ \\
$\,\,$ 1/5$\,\,$ & $\,\,$ 1 $\,\,$ & $\,\,$5$\,\,$ & $\,\,$3 $\,\,$ \\
$\,\,$ 1/7$\,\,$ & $\,\,$ 1/5$\,\,$ & $\,\,$ 1 $\,\,$ & $\,\,$1 $\,\,$ \\
$\,\,$ 1/9$\,\,$ & $\,\,$ 1/3$\,\,$ & $\,\,$ 1 $\,\,$ & $\,\,$ 1  $\,\,$ \\
\end{pmatrix},
\qquad
\lambda_{\max} =
4.1415,
\qquad
CR = 0.0533
\end{equation*}

\begin{equation*}
\mathbf{w}^{cos} =
\begin{pmatrix}
0.641130\\
0.221852\\
0.068907\\
\color{red} 0.068111\color{black}
\end{pmatrix}\end{equation*}
\begin{equation*}
\left[ \frac{{w}^{cos}_i}{{w}^{cos}_j} \right] =
\begin{pmatrix}
$\,\,$ 1 $\,\,$ & $\,\,$2.8899$\,\,$ & $\,\,$9.3043$\,\,$ & $\,\,$\color{red} 9.4131\color{black} $\,\,$ \\
$\,\,$0.3460$\,\,$ & $\,\,$ 1 $\,\,$ & $\,\,$3.2196$\,\,$ & $\,\,$\color{red} 3.2572\color{black}   $\,\,$ \\
$\,\,$0.1075$\,\,$ & $\,\,$0.3106$\,\,$ & $\,\,$ 1 $\,\,$ & $\,\,$\color{red} 1.0117\color{black}  $\,\,$ \\
$\,\,$\color{red} 0.1062\color{black} $\,\,$ & $\,\,$\color{red} 0.3070\color{black} $\,\,$ & $\,\,$\color{red} 0.9884\color{black} $\,\,$ & $\,\,$ 1  $\,\,$ \\
\end{pmatrix},
\end{equation*}

\begin{equation*}
\mathbf{w}^{\prime} =
\begin{pmatrix}
0.640620\\
0.221675\\
0.068852\\
0.068852
\end{pmatrix} =
0.999204\cdot
\begin{pmatrix}
0.641130\\
0.221852\\
0.068907\\
\color{gr} 0.068907\color{black}
\end{pmatrix},
\end{equation*}
\begin{equation*}
\left[ \frac{{w}^{\prime}_i}{{w}^{\prime}_j} \right] =
\begin{pmatrix}
$\,\,$ 1 $\,\,$ & $\,\,$2.8899$\,\,$ & $\,\,$9.3043$\,\,$ & $\,\,$\color{gr} 9.3043\color{black} $\,\,$ \\
$\,\,$0.3460$\,\,$ & $\,\,$ 1 $\,\,$ & $\,\,$3.2196$\,\,$ & $\,\,$\color{gr} 3.2196\color{black}   $\,\,$ \\
$\,\,$0.1075$\,\,$ & $\,\,$0.3106$\,\,$ & $\,\,$ 1 $\,\,$ & $\,\,$\color{gr} \color{blue} 1\color{black}  $\,\,$ \\
$\,\,$\color{gr} 0.1075\color{black} $\,\,$ & $\,\,$\color{gr} 0.3106\color{black} $\,\,$ & $\,\,$\color{gr} \color{blue} 1\color{black} $\,\,$ & $\,\,$ 1  $\,\,$ \\
\end{pmatrix},
\end{equation*}
\end{example}
\newpage
\begin{example}
\begin{equation*}
\mathbf{A} =
\begin{pmatrix}
$\,\,$ 1 $\,\,$ & $\,\,$5$\,\,$ & $\,\,$7$\,\,$ & $\,\,$9 $\,\,$ \\
$\,\,$ 1/5$\,\,$ & $\,\,$ 1 $\,\,$ & $\,\,$7$\,\,$ & $\,\,$4 $\,\,$ \\
$\,\,$ 1/7$\,\,$ & $\,\,$ 1/7$\,\,$ & $\,\,$ 1 $\,\,$ & $\,\,$1 $\,\,$ \\
$\,\,$ 1/9$\,\,$ & $\,\,$ 1/4$\,\,$ & $\,\,$ 1 $\,\,$ & $\,\,$ 1  $\,\,$ \\
\end{pmatrix},
\qquad
\lambda_{\max} =
4.2346,
\qquad
CR = 0.0885
\end{equation*}

\begin{equation*}
\mathbf{w}^{cos} =
\begin{pmatrix}
0.615823\\
0.258701\\
0.063577\\
\color{red} 0.061899\color{black}
\end{pmatrix}\end{equation*}
\begin{equation*}
\left[ \frac{{w}^{cos}_i}{{w}^{cos}_j} \right] =
\begin{pmatrix}
$\,\,$ 1 $\,\,$ & $\,\,$2.3804$\,\,$ & $\,\,$9.6862$\,\,$ & $\,\,$\color{red} 9.9488\color{black} $\,\,$ \\
$\,\,$0.4201$\,\,$ & $\,\,$ 1 $\,\,$ & $\,\,$4.0691$\,\,$ & $\,\,$\color{red} 4.1794\color{black}   $\,\,$ \\
$\,\,$0.1032$\,\,$ & $\,\,$0.2458$\,\,$ & $\,\,$ 1 $\,\,$ & $\,\,$\color{red} 1.0271\color{black}  $\,\,$ \\
$\,\,$\color{red} 0.1005\color{black} $\,\,$ & $\,\,$\color{red} 0.2393\color{black} $\,\,$ & $\,\,$\color{red} 0.9736\color{black} $\,\,$ & $\,\,$ 1  $\,\,$ \\
\end{pmatrix},
\end{equation*}

\begin{equation*}
\mathbf{w}^{\prime} =
\begin{pmatrix}
0.614791\\
0.258267\\
0.063471\\
0.063471
\end{pmatrix} =
0.998325\cdot
\begin{pmatrix}
0.615823\\
0.258701\\
0.063577\\
\color{gr} 0.063577\color{black}
\end{pmatrix},
\end{equation*}
\begin{equation*}
\left[ \frac{{w}^{\prime}_i}{{w}^{\prime}_j} \right] =
\begin{pmatrix}
$\,\,$ 1 $\,\,$ & $\,\,$2.3804$\,\,$ & $\,\,$9.6862$\,\,$ & $\,\,$\color{gr} 9.6862\color{black} $\,\,$ \\
$\,\,$0.4201$\,\,$ & $\,\,$ 1 $\,\,$ & $\,\,$4.0691$\,\,$ & $\,\,$\color{gr} 4.0691\color{black}   $\,\,$ \\
$\,\,$0.1032$\,\,$ & $\,\,$0.2458$\,\,$ & $\,\,$ 1 $\,\,$ & $\,\,$\color{gr} \color{blue} 1\color{black}  $\,\,$ \\
$\,\,$\color{gr} 0.1032\color{black} $\,\,$ & $\,\,$\color{gr} 0.2458\color{black} $\,\,$ & $\,\,$\color{gr} \color{blue} 1\color{black} $\,\,$ & $\,\,$ 1  $\,\,$ \\
\end{pmatrix},
\end{equation*}
\end{example}
\newpage
\begin{example}
\begin{equation*}
\mathbf{A} =
\begin{pmatrix}
$\,\,$ 1 $\,\,$ & $\,\,$6$\,\,$ & $\,\,$3$\,\,$ & $\,\,$4 $\,\,$ \\
$\,\,$ 1/6$\,\,$ & $\,\,$ 1 $\,\,$ & $\,\,$1$\,\,$ & $\,\,$3 $\,\,$ \\
$\,\,$ 1/3$\,\,$ & $\,\,$ 1 $\,\,$ & $\,\,$ 1 $\,\,$ & $\,\,$2 $\,\,$ \\
$\,\,$ 1/4$\,\,$ & $\,\,$ 1/3$\,\,$ & $\,\,$ 1/2$\,\,$ & $\,\,$ 1  $\,\,$ \\
\end{pmatrix},
\qquad
\lambda_{\max} =
4.1990,
\qquad
CR = 0.0750
\end{equation*}

\begin{equation*}
\mathbf{w}^{cos} =
\begin{pmatrix}
0.547251\\
0.180896\\
\color{red} 0.176137\color{black} \\
0.095715
\end{pmatrix}\end{equation*}
\begin{equation*}
\left[ \frac{{w}^{cos}_i}{{w}^{cos}_j} \right] =
\begin{pmatrix}
$\,\,$ 1 $\,\,$ & $\,\,$3.0252$\,\,$ & $\,\,$\color{red} 3.1070\color{black} $\,\,$ & $\,\,$5.7175$\,\,$ \\
$\,\,$0.3306$\,\,$ & $\,\,$ 1 $\,\,$ & $\,\,$\color{red} 1.0270\color{black} $\,\,$ & $\,\,$1.8899  $\,\,$ \\
$\,\,$\color{red} 0.3219\color{black} $\,\,$ & $\,\,$\color{red} 0.9737\color{black} $\,\,$ & $\,\,$ 1 $\,\,$ & $\,\,$\color{red} 1.8402\color{black}  $\,\,$ \\
$\,\,$0.1749$\,\,$ & $\,\,$0.5291$\,\,$ & $\,\,$\color{red} 0.5434\color{black} $\,\,$ & $\,\,$ 1  $\,\,$ \\
\end{pmatrix},
\end{equation*}

\begin{equation*}
\mathbf{w}^{\prime} =
\begin{pmatrix}
0.544659\\
0.180039\\
0.180039\\
0.095262
\end{pmatrix} =
0.995263\cdot
\begin{pmatrix}
0.547251\\
0.180896\\
\color{gr} 0.180896\color{black} \\
0.095715
\end{pmatrix},
\end{equation*}
\begin{equation*}
\left[ \frac{{w}^{\prime}_i}{{w}^{\prime}_j} \right] =
\begin{pmatrix}
$\,\,$ 1 $\,\,$ & $\,\,$3.0252$\,\,$ & $\,\,$\color{gr} 3.0252\color{black} $\,\,$ & $\,\,$5.7175$\,\,$ \\
$\,\,$0.3306$\,\,$ & $\,\,$ 1 $\,\,$ & $\,\,$\color{gr} \color{blue} 1\color{black} $\,\,$ & $\,\,$1.8899  $\,\,$ \\
$\,\,$\color{gr} 0.3306\color{black} $\,\,$ & $\,\,$\color{gr} \color{blue} 1\color{black} $\,\,$ & $\,\,$ 1 $\,\,$ & $\,\,$\color{gr} 1.8899\color{black}  $\,\,$ \\
$\,\,$0.1749$\,\,$ & $\,\,$0.5291$\,\,$ & $\,\,$\color{gr} 0.5291\color{black} $\,\,$ & $\,\,$ 1  $\,\,$ \\
\end{pmatrix},
\end{equation*}
\end{example}
\newpage
\begin{example}
\begin{equation*}
\mathbf{A} =
\begin{pmatrix}
$\,\,$ 1 $\,\,$ & $\,\,$6$\,\,$ & $\,\,$3$\,\,$ & $\,\,$5 $\,\,$ \\
$\,\,$ 1/6$\,\,$ & $\,\,$ 1 $\,\,$ & $\,\,$1$\,\,$ & $\,\,$3 $\,\,$ \\
$\,\,$ 1/3$\,\,$ & $\,\,$ 1 $\,\,$ & $\,\,$ 1 $\,\,$ & $\,\,$2 $\,\,$ \\
$\,\,$ 1/5$\,\,$ & $\,\,$ 1/3$\,\,$ & $\,\,$ 1/2$\,\,$ & $\,\,$ 1  $\,\,$ \\
\end{pmatrix},
\qquad
\lambda_{\max} =
4.1488,
\qquad
CR = 0.0561
\end{equation*}

\begin{equation*}
\mathbf{w}^{cos} =
\begin{pmatrix}
0.568009\\
0.173192\\
\color{red} 0.172142\color{black} \\
0.086657
\end{pmatrix}\end{equation*}
\begin{equation*}
\left[ \frac{{w}^{cos}_i}{{w}^{cos}_j} \right] =
\begin{pmatrix}
$\,\,$ 1 $\,\,$ & $\,\,$3.2797$\,\,$ & $\,\,$\color{red} 3.2996\color{black} $\,\,$ & $\,\,$6.5546$\,\,$ \\
$\,\,$0.3049$\,\,$ & $\,\,$ 1 $\,\,$ & $\,\,$\color{red} 1.0061\color{black} $\,\,$ & $\,\,$1.9986  $\,\,$ \\
$\,\,$\color{red} 0.3031\color{black} $\,\,$ & $\,\,$\color{red} 0.9939\color{black} $\,\,$ & $\,\,$ 1 $\,\,$ & $\,\,$\color{red} 1.9865\color{black}  $\,\,$ \\
$\,\,$0.1526$\,\,$ & $\,\,$0.5004$\,\,$ & $\,\,$\color{red} 0.5034\color{black} $\,\,$ & $\,\,$ 1  $\,\,$ \\
\end{pmatrix},
\end{equation*}

\begin{equation*}
\mathbf{w}^{\prime} =
\begin{pmatrix}
0.567413\\
0.173010\\
0.173010\\
0.086567
\end{pmatrix} =
0.998952\cdot
\begin{pmatrix}
0.568009\\
0.173192\\
\color{gr} 0.173192\color{black} \\
0.086657
\end{pmatrix},
\end{equation*}
\begin{equation*}
\left[ \frac{{w}^{\prime}_i}{{w}^{\prime}_j} \right] =
\begin{pmatrix}
$\,\,$ 1 $\,\,$ & $\,\,$3.2797$\,\,$ & $\,\,$\color{gr} 3.2797\color{black} $\,\,$ & $\,\,$6.5546$\,\,$ \\
$\,\,$0.3049$\,\,$ & $\,\,$ 1 $\,\,$ & $\,\,$\color{gr} \color{blue} 1\color{black} $\,\,$ & $\,\,$1.9986  $\,\,$ \\
$\,\,$\color{gr} 0.3049\color{black} $\,\,$ & $\,\,$\color{gr} \color{blue} 1\color{black} $\,\,$ & $\,\,$ 1 $\,\,$ & $\,\,$\color{gr} 1.9986\color{black}  $\,\,$ \\
$\,\,$0.1526$\,\,$ & $\,\,$0.5004$\,\,$ & $\,\,$\color{gr} 0.5004\color{black} $\,\,$ & $\,\,$ 1  $\,\,$ \\
\end{pmatrix},
\end{equation*}
\end{example}
\newpage
\begin{example}
\begin{equation*}
\mathbf{A} =
\begin{pmatrix}
$\,\,$ 1 $\,\,$ & $\,\,$6$\,\,$ & $\,\,$3$\,\,$ & $\,\,$6 $\,\,$ \\
$\,\,$ 1/6$\,\,$ & $\,\,$ 1 $\,\,$ & $\,\,$1$\,\,$ & $\,\,$5 $\,\,$ \\
$\,\,$ 1/3$\,\,$ & $\,\,$ 1 $\,\,$ & $\,\,$ 1 $\,\,$ & $\,\,$3 $\,\,$ \\
$\,\,$ 1/6$\,\,$ & $\,\,$ 1/5$\,\,$ & $\,\,$ 1/3$\,\,$ & $\,\,$ 1  $\,\,$ \\
\end{pmatrix},
\qquad
\lambda_{\max} =
4.2277,
\qquad
CR = 0.0859
\end{equation*}

\begin{equation*}
\mathbf{w}^{cos} =
\begin{pmatrix}
0.561468\\
0.193489\\
\color{red} 0.180206\color{black} \\
0.064837
\end{pmatrix}\end{equation*}
\begin{equation*}
\left[ \frac{{w}^{cos}_i}{{w}^{cos}_j} \right] =
\begin{pmatrix}
$\,\,$ 1 $\,\,$ & $\,\,$2.9018$\,\,$ & $\,\,$\color{red} 3.1157\color{black} $\,\,$ & $\,\,$8.6597$\,\,$ \\
$\,\,$0.3446$\,\,$ & $\,\,$ 1 $\,\,$ & $\,\,$\color{red} 1.0737\color{black} $\,\,$ & $\,\,$2.9842  $\,\,$ \\
$\,\,$\color{red} 0.3210\color{black} $\,\,$ & $\,\,$\color{red} 0.9314\color{black} $\,\,$ & $\,\,$ 1 $\,\,$ & $\,\,$\color{red} 2.7794\color{black}  $\,\,$ \\
$\,\,$0.1155$\,\,$ & $\,\,$0.3351$\,\,$ & $\,\,$\color{red} 0.3598\color{black} $\,\,$ & $\,\,$ 1  $\,\,$ \\
\end{pmatrix},
\end{equation*}

\begin{equation*}
\mathbf{w}^{\prime} =
\begin{pmatrix}
0.557593\\
0.192153\\
0.185864\\
0.064390
\end{pmatrix} =
0.993098\cdot
\begin{pmatrix}
0.561468\\
0.193489\\
\color{gr} 0.187156\color{black} \\
0.064837
\end{pmatrix},
\end{equation*}
\begin{equation*}
\left[ \frac{{w}^{\prime}_i}{{w}^{\prime}_j} \right] =
\begin{pmatrix}
$\,\,$ 1 $\,\,$ & $\,\,$2.9018$\,\,$ & $\,\,$\color{gr} \color{blue} 3\color{black} $\,\,$ & $\,\,$8.6597$\,\,$ \\
$\,\,$0.3446$\,\,$ & $\,\,$ 1 $\,\,$ & $\,\,$\color{gr} 1.0338\color{black} $\,\,$ & $\,\,$2.9842  $\,\,$ \\
$\,\,$\color{gr} \color{blue}  1/3\color{black} $\,\,$ & $\,\,$\color{gr} 0.9673\color{black} $\,\,$ & $\,\,$ 1 $\,\,$ & $\,\,$\color{gr} 2.8866\color{black}  $\,\,$ \\
$\,\,$0.1155$\,\,$ & $\,\,$0.3351$\,\,$ & $\,\,$\color{gr} 0.3464\color{black} $\,\,$ & $\,\,$ 1  $\,\,$ \\
\end{pmatrix},
\end{equation*}
\end{example}
\newpage
\begin{example}
\begin{equation*}
\mathbf{A} =
\begin{pmatrix}
$\,\,$ 1 $\,\,$ & $\,\,$6$\,\,$ & $\,\,$3$\,\,$ & $\,\,$7 $\,\,$ \\
$\,\,$ 1/6$\,\,$ & $\,\,$ 1 $\,\,$ & $\,\,$1$\,\,$ & $\,\,$5 $\,\,$ \\
$\,\,$ 1/3$\,\,$ & $\,\,$ 1 $\,\,$ & $\,\,$ 1 $\,\,$ & $\,\,$3 $\,\,$ \\
$\,\,$ 1/7$\,\,$ & $\,\,$ 1/5$\,\,$ & $\,\,$ 1/3$\,\,$ & $\,\,$ 1  $\,\,$ \\
\end{pmatrix},
\qquad
\lambda_{\max} =
4.1889,
\qquad
CR = 0.0712
\end{equation*}

\begin{equation*}
\mathbf{w}^{cos} =
\begin{pmatrix}
0.574955\\
0.187463\\
\color{red} 0.177235\color{black} \\
0.060347
\end{pmatrix}\end{equation*}
\begin{equation*}
\left[ \frac{{w}^{cos}_i}{{w}^{cos}_j} \right] =
\begin{pmatrix}
$\,\,$ 1 $\,\,$ & $\,\,$3.0670$\,\,$ & $\,\,$\color{red} 3.2440\color{black} $\,\,$ & $\,\,$9.5275$\,\,$ \\
$\,\,$0.3260$\,\,$ & $\,\,$ 1 $\,\,$ & $\,\,$\color{red} 1.0577\color{black} $\,\,$ & $\,\,$3.1064  $\,\,$ \\
$\,\,$\color{red} 0.3083\color{black} $\,\,$ & $\,\,$\color{red} 0.9454\color{black} $\,\,$ & $\,\,$ 1 $\,\,$ & $\,\,$\color{red} 2.9369\color{black}  $\,\,$ \\
$\,\,$0.1050$\,\,$ & $\,\,$0.3219$\,\,$ & $\,\,$\color{red} 0.3405\color{black} $\,\,$ & $\,\,$ 1  $\,\,$ \\
\end{pmatrix},
\end{equation*}

\begin{equation*}
\mathbf{w}^{\prime} =
\begin{pmatrix}
0.572775\\
0.186753\\
0.180354\\
0.060118
\end{pmatrix} =
0.996209\cdot
\begin{pmatrix}
0.574955\\
0.187463\\
\color{gr} 0.181041\color{black} \\
0.060347
\end{pmatrix},
\end{equation*}
\begin{equation*}
\left[ \frac{{w}^{\prime}_i}{{w}^{\prime}_j} \right] =
\begin{pmatrix}
$\,\,$ 1 $\,\,$ & $\,\,$3.0670$\,\,$ & $\,\,$\color{gr} 3.1758\color{black} $\,\,$ & $\,\,$9.5275$\,\,$ \\
$\,\,$0.3260$\,\,$ & $\,\,$ 1 $\,\,$ & $\,\,$\color{gr} 1.0355\color{black} $\,\,$ & $\,\,$3.1064  $\,\,$ \\
$\,\,$\color{gr} 0.3149\color{black} $\,\,$ & $\,\,$\color{gr} 0.9657\color{black} $\,\,$ & $\,\,$ 1 $\,\,$ & $\,\,$\color{gr} \color{blue} 3\color{black}  $\,\,$ \\
$\,\,$0.1050$\,\,$ & $\,\,$0.3219$\,\,$ & $\,\,$\color{gr} \color{blue}  1/3\color{black} $\,\,$ & $\,\,$ 1  $\,\,$ \\
\end{pmatrix},
\end{equation*}
\end{example}
\newpage
\begin{example}
\begin{equation*}
\mathbf{A} =
\begin{pmatrix}
$\,\,$ 1 $\,\,$ & $\,\,$6$\,\,$ & $\,\,$3$\,\,$ & $\,\,$7 $\,\,$ \\
$\,\,$ 1/6$\,\,$ & $\,\,$ 1 $\,\,$ & $\,\,$1$\,\,$ & $\,\,$6 $\,\,$ \\
$\,\,$ 1/3$\,\,$ & $\,\,$ 1 $\,\,$ & $\,\,$ 1 $\,\,$ & $\,\,$3 $\,\,$ \\
$\,\,$ 1/7$\,\,$ & $\,\,$ 1/6$\,\,$ & $\,\,$ 1/3$\,\,$ & $\,\,$ 1  $\,\,$ \\
\end{pmatrix},
\qquad
\lambda_{\max} =
4.2416,
\qquad
CR = 0.0911
\end{equation*}

\begin{equation*}
\mathbf{w}^{cos} =
\begin{pmatrix}
0.568250\\
0.198980\\
\color{red} 0.174280\color{black} \\
0.058490
\end{pmatrix}\end{equation*}
\begin{equation*}
\left[ \frac{{w}^{cos}_i}{{w}^{cos}_j} \right] =
\begin{pmatrix}
$\,\,$ 1 $\,\,$ & $\,\,$2.8558$\,\,$ & $\,\,$\color{red} 3.2606\color{black} $\,\,$ & $\,\,$9.7153$\,\,$ \\
$\,\,$0.3502$\,\,$ & $\,\,$ 1 $\,\,$ & $\,\,$\color{red} 1.1417\color{black} $\,\,$ & $\,\,$3.4020  $\,\,$ \\
$\,\,$\color{red} 0.3067\color{black} $\,\,$ & $\,\,$\color{red} 0.8759\color{black} $\,\,$ & $\,\,$ 1 $\,\,$ & $\,\,$\color{red} 2.9797\color{black}  $\,\,$ \\
$\,\,$0.1029$\,\,$ & $\,\,$0.2939$\,\,$ & $\,\,$\color{red} 0.3356\color{black} $\,\,$ & $\,\,$ 1  $\,\,$ \\
\end{pmatrix},
\end{equation*}

\begin{equation*}
\mathbf{w}^{\prime} =
\begin{pmatrix}
0.567575\\
0.198743\\
0.175261\\
0.058420
\end{pmatrix} =
0.998811\cdot
\begin{pmatrix}
0.568250\\
0.198980\\
\color{gr} 0.175470\color{black} \\
0.058490
\end{pmatrix},
\end{equation*}
\begin{equation*}
\left[ \frac{{w}^{\prime}_i}{{w}^{\prime}_j} \right] =
\begin{pmatrix}
$\,\,$ 1 $\,\,$ & $\,\,$2.8558$\,\,$ & $\,\,$\color{gr} 3.2384\color{black} $\,\,$ & $\,\,$9.7153$\,\,$ \\
$\,\,$0.3502$\,\,$ & $\,\,$ 1 $\,\,$ & $\,\,$\color{gr} 1.1340\color{black} $\,\,$ & $\,\,$3.4020  $\,\,$ \\
$\,\,$\color{gr} 0.3088\color{black} $\,\,$ & $\,\,$\color{gr} 0.8818\color{black} $\,\,$ & $\,\,$ 1 $\,\,$ & $\,\,$\color{gr} \color{blue} 3\color{black}  $\,\,$ \\
$\,\,$0.1029$\,\,$ & $\,\,$0.2939$\,\,$ & $\,\,$\color{gr} \color{blue}  1/3\color{black} $\,\,$ & $\,\,$ 1  $\,\,$ \\
\end{pmatrix},
\end{equation*}
\end{example}
\newpage
\begin{example}
\begin{equation*}
\mathbf{A} =
\begin{pmatrix}
$\,\,$ 1 $\,\,$ & $\,\,$6$\,\,$ & $\,\,$3$\,\,$ & $\,\,$8 $\,\,$ \\
$\,\,$ 1/6$\,\,$ & $\,\,$ 1 $\,\,$ & $\,\,$1$\,\,$ & $\,\,$6 $\,\,$ \\
$\,\,$ 1/3$\,\,$ & $\,\,$ 1 $\,\,$ & $\,\,$ 1 $\,\,$ & $\,\,$4 $\,\,$ \\
$\,\,$ 1/8$\,\,$ & $\,\,$ 1/6$\,\,$ & $\,\,$ 1/4$\,\,$ & $\,\,$ 1  $\,\,$ \\
\end{pmatrix},
\qquad
\lambda_{\max} =
4.1990,
\qquad
CR = 0.0750
\end{equation*}

\begin{equation*}
\mathbf{w}^{cos} =
\begin{pmatrix}
0.574434\\
0.190002\\
\color{red} 0.185176\color{black} \\
0.050388
\end{pmatrix}\end{equation*}
\begin{equation*}
\left[ \frac{{w}^{cos}_i}{{w}^{cos}_j} \right] =
\begin{pmatrix}
$\,\,$ 1 $\,\,$ & $\,\,$3.0233$\,\,$ & $\,\,$\color{red} 3.1021\color{black} $\,\,$ & $\,\,$11.4003$\,\,$ \\
$\,\,$0.3308$\,\,$ & $\,\,$ 1 $\,\,$ & $\,\,$\color{red} 1.0261\color{black} $\,\,$ & $\,\,$3.7708  $\,\,$ \\
$\,\,$\color{red} 0.3224\color{black} $\,\,$ & $\,\,$\color{red} 0.9746\color{black} $\,\,$ & $\,\,$ 1 $\,\,$ & $\,\,$\color{red} 3.6750\color{black}  $\,\,$ \\
$\,\,$0.0877$\,\,$ & $\,\,$0.2652$\,\,$ & $\,\,$\color{red} 0.2721\color{black} $\,\,$ & $\,\,$ 1  $\,\,$ \\
\end{pmatrix},
\end{equation*}

\begin{equation*}
\mathbf{w}^{\prime} =
\begin{pmatrix}
0.571675\\
0.189090\\
0.189090\\
0.050146
\end{pmatrix} =
0.995197\cdot
\begin{pmatrix}
0.574434\\
0.190002\\
\color{gr} 0.190002\color{black} \\
0.050388
\end{pmatrix},
\end{equation*}
\begin{equation*}
\left[ \frac{{w}^{\prime}_i}{{w}^{\prime}_j} \right] =
\begin{pmatrix}
$\,\,$ 1 $\,\,$ & $\,\,$3.0233$\,\,$ & $\,\,$\color{gr} 3.0233\color{black} $\,\,$ & $\,\,$11.4003$\,\,$ \\
$\,\,$0.3308$\,\,$ & $\,\,$ 1 $\,\,$ & $\,\,$\color{gr} \color{blue} 1\color{black} $\,\,$ & $\,\,$3.7708  $\,\,$ \\
$\,\,$\color{gr} 0.3308\color{black} $\,\,$ & $\,\,$\color{gr} \color{blue} 1\color{black} $\,\,$ & $\,\,$ 1 $\,\,$ & $\,\,$\color{gr} 3.7708\color{black}  $\,\,$ \\
$\,\,$0.0877$\,\,$ & $\,\,$0.2652$\,\,$ & $\,\,$\color{gr} 0.2652\color{black} $\,\,$ & $\,\,$ 1  $\,\,$ \\
\end{pmatrix},
\end{equation*}
\end{example}
\newpage
\begin{example}
\begin{equation*}
\mathbf{A} =
\begin{pmatrix}
$\,\,$ 1 $\,\,$ & $\,\,$6$\,\,$ & $\,\,$3$\,\,$ & $\,\,$8 $\,\,$ \\
$\,\,$ 1/6$\,\,$ & $\,\,$ 1 $\,\,$ & $\,\,$1$\,\,$ & $\,\,$7 $\,\,$ \\
$\,\,$ 1/3$\,\,$ & $\,\,$ 1 $\,\,$ & $\,\,$ 1 $\,\,$ & $\,\,$4 $\,\,$ \\
$\,\,$ 1/8$\,\,$ & $\,\,$ 1/7$\,\,$ & $\,\,$ 1/4$\,\,$ & $\,\,$ 1  $\,\,$ \\
\end{pmatrix},
\qquad
\lambda_{\max} =
4.2421,
\qquad
CR = 0.0913
\end{equation*}

\begin{equation*}
\mathbf{w}^{cos} =
\begin{pmatrix}
0.568868\\
0.199803\\
\color{red} 0.182286\color{black} \\
0.049043
\end{pmatrix}\end{equation*}
\begin{equation*}
\left[ \frac{{w}^{cos}_i}{{w}^{cos}_j} \right] =
\begin{pmatrix}
$\,\,$ 1 $\,\,$ & $\,\,$2.8471$\,\,$ & $\,\,$\color{red} 3.1208\color{black} $\,\,$ & $\,\,$11.5995$\,\,$ \\
$\,\,$0.3512$\,\,$ & $\,\,$ 1 $\,\,$ & $\,\,$\color{red} 1.0961\color{black} $\,\,$ & $\,\,$4.0741  $\,\,$ \\
$\,\,$\color{red} 0.3204\color{black} $\,\,$ & $\,\,$\color{red} 0.9123\color{black} $\,\,$ & $\,\,$ 1 $\,\,$ & $\,\,$\color{red} 3.7169\color{black}  $\,\,$ \\
$\,\,$0.0862$\,\,$ & $\,\,$0.2455$\,\,$ & $\,\,$\color{red} 0.2690\color{black} $\,\,$ & $\,\,$ 1  $\,\,$ \\
\end{pmatrix},
\end{equation*}

\begin{equation*}
\mathbf{w}^{\prime} =
\begin{pmatrix}
0.564725\\
0.198348\\
0.188242\\
0.048685
\end{pmatrix} =
0.992716\cdot
\begin{pmatrix}
0.568868\\
0.199803\\
\color{gr} 0.189623\color{black} \\
0.049043
\end{pmatrix},
\end{equation*}
\begin{equation*}
\left[ \frac{{w}^{\prime}_i}{{w}^{\prime}_j} \right] =
\begin{pmatrix}
$\,\,$ 1 $\,\,$ & $\,\,$2.8471$\,\,$ & $\,\,$\color{gr} \color{blue} 3\color{black} $\,\,$ & $\,\,$11.5995$\,\,$ \\
$\,\,$0.3512$\,\,$ & $\,\,$ 1 $\,\,$ & $\,\,$\color{gr} 1.0537\color{black} $\,\,$ & $\,\,$4.0741  $\,\,$ \\
$\,\,$\color{gr} \color{blue}  1/3\color{black} $\,\,$ & $\,\,$\color{gr} 0.9490\color{black} $\,\,$ & $\,\,$ 1 $\,\,$ & $\,\,$\color{gr} 3.8665\color{black}  $\,\,$ \\
$\,\,$0.0862$\,\,$ & $\,\,$0.2455$\,\,$ & $\,\,$\color{gr} 0.2586\color{black} $\,\,$ & $\,\,$ 1  $\,\,$ \\
\end{pmatrix},
\end{equation*}
\end{example}
\newpage
\begin{example}
\begin{equation*}
\mathbf{A} =
\begin{pmatrix}
$\,\,$ 1 $\,\,$ & $\,\,$6$\,\,$ & $\,\,$3$\,\,$ & $\,\,$9 $\,\,$ \\
$\,\,$ 1/6$\,\,$ & $\,\,$ 1 $\,\,$ & $\,\,$1$\,\,$ & $\,\,$6 $\,\,$ \\
$\,\,$ 1/3$\,\,$ & $\,\,$ 1 $\,\,$ & $\,\,$ 1 $\,\,$ & $\,\,$4 $\,\,$ \\
$\,\,$ 1/9$\,\,$ & $\,\,$ 1/6$\,\,$ & $\,\,$ 1/4$\,\,$ & $\,\,$ 1  $\,\,$ \\
\end{pmatrix},
\qquad
\lambda_{\max} =
4.1707,
\qquad
CR = 0.0644
\end{equation*}

\begin{equation*}
\mathbf{w}^{cos} =
\begin{pmatrix}
0.584538\\
0.185333\\
\color{red} 0.182489\color{black} \\
0.047640
\end{pmatrix}\end{equation*}
\begin{equation*}
\left[ \frac{{w}^{cos}_i}{{w}^{cos}_j} \right] =
\begin{pmatrix}
$\,\,$ 1 $\,\,$ & $\,\,$3.1540$\,\,$ & $\,\,$\color{red} 3.2031\color{black} $\,\,$ & $\,\,$12.2698$\,\,$ \\
$\,\,$0.3171$\,\,$ & $\,\,$ 1 $\,\,$ & $\,\,$\color{red} 1.0156\color{black} $\,\,$ & $\,\,$3.8902  $\,\,$ \\
$\,\,$\color{red} 0.3122\color{black} $\,\,$ & $\,\,$\color{red} 0.9847\color{black} $\,\,$ & $\,\,$ 1 $\,\,$ & $\,\,$\color{red} 3.8306\color{black}  $\,\,$ \\
$\,\,$0.0815$\,\,$ & $\,\,$0.2571$\,\,$ & $\,\,$\color{red} 0.2611\color{black} $\,\,$ & $\,\,$ 1  $\,\,$ \\
\end{pmatrix},
\end{equation*}

\begin{equation*}
\mathbf{w}^{\prime} =
\begin{pmatrix}
0.582880\\
0.184807\\
0.184807\\
0.047505
\end{pmatrix} =
0.997164\cdot
\begin{pmatrix}
0.584538\\
0.185333\\
\color{gr} 0.185333\color{black} \\
0.047640
\end{pmatrix},
\end{equation*}
\begin{equation*}
\left[ \frac{{w}^{\prime}_i}{{w}^{\prime}_j} \right] =
\begin{pmatrix}
$\,\,$ 1 $\,\,$ & $\,\,$3.1540$\,\,$ & $\,\,$\color{gr} 3.1540\color{black} $\,\,$ & $\,\,$12.2698$\,\,$ \\
$\,\,$0.3171$\,\,$ & $\,\,$ 1 $\,\,$ & $\,\,$\color{gr} \color{blue} 1\color{black} $\,\,$ & $\,\,$3.8902  $\,\,$ \\
$\,\,$\color{gr} 0.3171\color{black} $\,\,$ & $\,\,$\color{gr} \color{blue} 1\color{black} $\,\,$ & $\,\,$ 1 $\,\,$ & $\,\,$\color{gr} 3.8902\color{black}  $\,\,$ \\
$\,\,$0.0815$\,\,$ & $\,\,$0.2571$\,\,$ & $\,\,$\color{gr} 0.2571\color{black} $\,\,$ & $\,\,$ 1  $\,\,$ \\
\end{pmatrix},
\end{equation*}
\end{example}
\newpage
\begin{example}
\begin{equation*}
\mathbf{A} =
\begin{pmatrix}
$\,\,$ 1 $\,\,$ & $\,\,$6$\,\,$ & $\,\,$3$\,\,$ & $\,\,$9 $\,\,$ \\
$\,\,$ 1/6$\,\,$ & $\,\,$ 1 $\,\,$ & $\,\,$1$\,\,$ & $\,\,$7 $\,\,$ \\
$\,\,$ 1/3$\,\,$ & $\,\,$ 1 $\,\,$ & $\,\,$ 1 $\,\,$ & $\,\,$4 $\,\,$ \\
$\,\,$ 1/9$\,\,$ & $\,\,$ 1/7$\,\,$ & $\,\,$ 1/4$\,\,$ & $\,\,$ 1  $\,\,$ \\
\end{pmatrix},
\qquad
\lambda_{\max} =
4.2109,
\qquad
CR = 0.0795
\end{equation*}

\begin{equation*}
\mathbf{w}^{cos} =
\begin{pmatrix}
0.578762\\
0.194943\\
\color{red} 0.179925\color{black} \\
0.046369
\end{pmatrix}\end{equation*}
\begin{equation*}
\left[ \frac{{w}^{cos}_i}{{w}^{cos}_j} \right] =
\begin{pmatrix}
$\,\,$ 1 $\,\,$ & $\,\,$2.9689$\,\,$ & $\,\,$\color{red} 3.2167\color{black} $\,\,$ & $\,\,$12.4816$\,\,$ \\
$\,\,$0.3368$\,\,$ & $\,\,$ 1 $\,\,$ & $\,\,$\color{red} 1.0835\color{black} $\,\,$ & $\,\,$4.2041  $\,\,$ \\
$\,\,$\color{red} 0.3109\color{black} $\,\,$ & $\,\,$\color{red} 0.9230\color{black} $\,\,$ & $\,\,$ 1 $\,\,$ & $\,\,$\color{red} 3.8803\color{black}  $\,\,$ \\
$\,\,$0.0801$\,\,$ & $\,\,$0.2379$\,\,$ & $\,\,$\color{red} 0.2577\color{black} $\,\,$ & $\,\,$ 1  $\,\,$ \\
\end{pmatrix},
\end{equation*}

\begin{equation*}
\mathbf{w}^{\prime} =
\begin{pmatrix}
0.575567\\
0.193867\\
0.184453\\
0.046113
\end{pmatrix} =
0.994478\cdot
\begin{pmatrix}
0.578762\\
0.194943\\
\color{gr} 0.185477\color{black} \\
0.046369
\end{pmatrix},
\end{equation*}
\begin{equation*}
\left[ \frac{{w}^{\prime}_i}{{w}^{\prime}_j} \right] =
\begin{pmatrix}
$\,\,$ 1 $\,\,$ & $\,\,$2.9689$\,\,$ & $\,\,$\color{gr} 3.1204\color{black} $\,\,$ & $\,\,$12.4816$\,\,$ \\
$\,\,$0.3368$\,\,$ & $\,\,$ 1 $\,\,$ & $\,\,$\color{gr} 1.0510\color{black} $\,\,$ & $\,\,$4.2041  $\,\,$ \\
$\,\,$\color{gr} 0.3205\color{black} $\,\,$ & $\,\,$\color{gr} 0.9514\color{black} $\,\,$ & $\,\,$ 1 $\,\,$ & $\,\,$\color{gr} \color{blue} 4\color{black}  $\,\,$ \\
$\,\,$0.0801$\,\,$ & $\,\,$0.2379$\,\,$ & $\,\,$\color{gr} \color{blue}  1/4\color{black} $\,\,$ & $\,\,$ 1  $\,\,$ \\
\end{pmatrix},
\end{equation*}
\end{example}
\newpage
\begin{example}
\begin{equation*}
\mathbf{A} =
\begin{pmatrix}
$\,\,$ 1 $\,\,$ & $\,\,$6$\,\,$ & $\,\,$3$\,\,$ & $\,\,$9 $\,\,$ \\
$\,\,$ 1/6$\,\,$ & $\,\,$ 1 $\,\,$ & $\,\,$1$\,\,$ & $\,\,$7 $\,\,$ \\
$\,\,$ 1/3$\,\,$ & $\,\,$ 1 $\,\,$ & $\,\,$ 1 $\,\,$ & $\,\,$5 $\,\,$ \\
$\,\,$ 1/9$\,\,$ & $\,\,$ 1/7$\,\,$ & $\,\,$ 1/5$\,\,$ & $\,\,$ 1  $\,\,$ \\
\end{pmatrix},
\qquad
\lambda_{\max} =
4.2095,
\qquad
CR = 0.0790
\end{equation*}

\begin{equation*}
\mathbf{w}^{cos} =
\begin{pmatrix}
0.573879\\
0.191739\\
\color{red} 0.190960\color{black} \\
0.043422
\end{pmatrix}\end{equation*}
\begin{equation*}
\left[ \frac{{w}^{cos}_i}{{w}^{cos}_j} \right] =
\begin{pmatrix}
$\,\,$ 1 $\,\,$ & $\,\,$2.9930$\,\,$ & $\,\,$\color{red} 3.0052\color{black} $\,\,$ & $\,\,$13.2164$\,\,$ \\
$\,\,$0.3341$\,\,$ & $\,\,$ 1 $\,\,$ & $\,\,$\color{red} 1.0041\color{black} $\,\,$ & $\,\,$4.4158  $\,\,$ \\
$\,\,$\color{red} 0.3328\color{black} $\,\,$ & $\,\,$\color{red} 0.9959\color{black} $\,\,$ & $\,\,$ 1 $\,\,$ & $\,\,$\color{red} 4.3978\color{black}  $\,\,$ \\
$\,\,$0.0757$\,\,$ & $\,\,$0.2265$\,\,$ & $\,\,$\color{red} 0.2274\color{black} $\,\,$ & $\,\,$ 1  $\,\,$ \\
\end{pmatrix},
\end{equation*}

\begin{equation*}
\mathbf{w}^{\prime} =
\begin{pmatrix}
0.573688\\
0.191676\\
0.191229\\
0.043407
\end{pmatrix} =
0.999667\cdot
\begin{pmatrix}
0.573879\\
0.191739\\
\color{gr} 0.191293\color{black} \\
0.043422
\end{pmatrix},
\end{equation*}
\begin{equation*}
\left[ \frac{{w}^{\prime}_i}{{w}^{\prime}_j} \right] =
\begin{pmatrix}
$\,\,$ 1 $\,\,$ & $\,\,$2.9930$\,\,$ & $\,\,$\color{gr} \color{blue} 3\color{black} $\,\,$ & $\,\,$13.2164$\,\,$ \\
$\,\,$0.3341$\,\,$ & $\,\,$ 1 $\,\,$ & $\,\,$\color{gr} 1.0023\color{black} $\,\,$ & $\,\,$4.4158  $\,\,$ \\
$\,\,$\color{gr} \color{blue}  1/3\color{black} $\,\,$ & $\,\,$\color{gr} 0.9977\color{black} $\,\,$ & $\,\,$ 1 $\,\,$ & $\,\,$\color{gr} 4.4055\color{black}  $\,\,$ \\
$\,\,$0.0757$\,\,$ & $\,\,$0.2265$\,\,$ & $\,\,$\color{gr} 0.2270\color{black} $\,\,$ & $\,\,$ 1  $\,\,$ \\
\end{pmatrix},
\end{equation*}
\end{example}
\newpage
\begin{example}
\begin{equation*}
\mathbf{A} =
\begin{pmatrix}
$\,\,$ 1 $\,\,$ & $\,\,$6$\,\,$ & $\,\,$3$\,\,$ & $\,\,$9 $\,\,$ \\
$\,\,$ 1/6$\,\,$ & $\,\,$ 1 $\,\,$ & $\,\,$1$\,\,$ & $\,\,$8 $\,\,$ \\
$\,\,$ 1/3$\,\,$ & $\,\,$ 1 $\,\,$ & $\,\,$ 1 $\,\,$ & $\,\,$4 $\,\,$ \\
$\,\,$ 1/9$\,\,$ & $\,\,$ 1/8$\,\,$ & $\,\,$ 1/4$\,\,$ & $\,\,$ 1  $\,\,$ \\
\end{pmatrix},
\qquad
\lambda_{\max} =
4.2512,
\qquad
CR = 0.0947
\end{equation*}

\begin{equation*}
\mathbf{w}^{cos} =
\begin{pmatrix}
0.573629\\
0.203504\\
\color{red} 0.177560\color{black} \\
0.045307
\end{pmatrix}\end{equation*}
\begin{equation*}
\left[ \frac{{w}^{cos}_i}{{w}^{cos}_j} \right] =
\begin{pmatrix}
$\,\,$ 1 $\,\,$ & $\,\,$2.8188$\,\,$ & $\,\,$\color{red} 3.2306\color{black} $\,\,$ & $\,\,$12.6610$\,\,$ \\
$\,\,$0.3548$\,\,$ & $\,\,$ 1 $\,\,$ & $\,\,$\color{red} 1.1461\color{black} $\,\,$ & $\,\,$4.4917  $\,\,$ \\
$\,\,$\color{red} 0.3095\color{black} $\,\,$ & $\,\,$\color{red} 0.8725\color{black} $\,\,$ & $\,\,$ 1 $\,\,$ & $\,\,$\color{red} 3.9191\color{black}  $\,\,$ \\
$\,\,$0.0790$\,\,$ & $\,\,$0.2226$\,\,$ & $\,\,$\color{red} 0.2552\color{black} $\,\,$ & $\,\,$ 1  $\,\,$ \\
\end{pmatrix},
\end{equation*}

\begin{equation*}
\mathbf{w}^{\prime} =
\begin{pmatrix}
0.571534\\
0.202760\\
0.180565\\
0.045141
\end{pmatrix} =
0.996347\cdot
\begin{pmatrix}
0.573629\\
0.203504\\
\color{gr} 0.181227\color{black} \\
0.045307
\end{pmatrix},
\end{equation*}
\begin{equation*}
\left[ \frac{{w}^{\prime}_i}{{w}^{\prime}_j} \right] =
\begin{pmatrix}
$\,\,$ 1 $\,\,$ & $\,\,$2.8188$\,\,$ & $\,\,$\color{gr} 3.1653\color{black} $\,\,$ & $\,\,$12.6610$\,\,$ \\
$\,\,$0.3548$\,\,$ & $\,\,$ 1 $\,\,$ & $\,\,$\color{gr} 1.1229\color{black} $\,\,$ & $\,\,$4.4917  $\,\,$ \\
$\,\,$\color{gr} 0.3159\color{black} $\,\,$ & $\,\,$\color{gr} 0.8905\color{black} $\,\,$ & $\,\,$ 1 $\,\,$ & $\,\,$\color{gr} \color{blue} 4\color{black}  $\,\,$ \\
$\,\,$0.0790$\,\,$ & $\,\,$0.2226$\,\,$ & $\,\,$\color{gr} \color{blue}  1/4\color{black} $\,\,$ & $\,\,$ 1  $\,\,$ \\
\end{pmatrix},
\end{equation*}
\end{example}
\newpage
\begin{example}
\begin{equation*}
\mathbf{A} =
\begin{pmatrix}
$\,\,$ 1 $\,\,$ & $\,\,$6$\,\,$ & $\,\,$3$\,\,$ & $\,\,$9 $\,\,$ \\
$\,\,$ 1/6$\,\,$ & $\,\,$ 1 $\,\,$ & $\,\,$1$\,\,$ & $\,\,$8 $\,\,$ \\
$\,\,$ 1/3$\,\,$ & $\,\,$ 1 $\,\,$ & $\,\,$ 1 $\,\,$ & $\,\,$5 $\,\,$ \\
$\,\,$ 1/9$\,\,$ & $\,\,$ 1/8$\,\,$ & $\,\,$ 1/5$\,\,$ & $\,\,$ 1  $\,\,$ \\
\end{pmatrix},
\qquad
\lambda_{\max} =
4.2460,
\qquad
CR = 0.0928
\end{equation*}

\begin{equation*}
\mathbf{w}^{cos} =
\begin{pmatrix}
0.569128\\
0.200264\\
\color{red} 0.188205\color{black} \\
0.042404
\end{pmatrix}\end{equation*}
\begin{equation*}
\left[ \frac{{w}^{cos}_i}{{w}^{cos}_j} \right] =
\begin{pmatrix}
$\,\,$ 1 $\,\,$ & $\,\,$2.8419$\,\,$ & $\,\,$\color{red} 3.0240\color{black} $\,\,$ & $\,\,$13.4217$\,\,$ \\
$\,\,$0.3519$\,\,$ & $\,\,$ 1 $\,\,$ & $\,\,$\color{red} 1.0641\color{black} $\,\,$ & $\,\,$4.7228  $\,\,$ \\
$\,\,$\color{red} 0.3307\color{black} $\,\,$ & $\,\,$\color{red} 0.9398\color{black} $\,\,$ & $\,\,$ 1 $\,\,$ & $\,\,$\color{red} 4.4384\color{black}  $\,\,$ \\
$\,\,$0.0745$\,\,$ & $\,\,$0.2117$\,\,$ & $\,\,$\color{red} 0.2253\color{black} $\,\,$ & $\,\,$ 1  $\,\,$ \\
\end{pmatrix},
\end{equation*}

\begin{equation*}
\mathbf{w}^{\prime} =
\begin{pmatrix}
0.568273\\
0.199963\\
0.189424\\
0.042340
\end{pmatrix} =
0.998497\cdot
\begin{pmatrix}
0.569128\\
0.200264\\
\color{gr} 0.189709\color{black} \\
0.042404
\end{pmatrix},
\end{equation*}
\begin{equation*}
\left[ \frac{{w}^{\prime}_i}{{w}^{\prime}_j} \right] =
\begin{pmatrix}
$\,\,$ 1 $\,\,$ & $\,\,$2.8419$\,\,$ & $\,\,$\color{gr} \color{blue} 3\color{black} $\,\,$ & $\,\,$13.4217$\,\,$ \\
$\,\,$0.3519$\,\,$ & $\,\,$ 1 $\,\,$ & $\,\,$\color{gr} 1.0556\color{black} $\,\,$ & $\,\,$4.7228  $\,\,$ \\
$\,\,$\color{gr} \color{blue}  1/3\color{black} $\,\,$ & $\,\,$\color{gr} 0.9473\color{black} $\,\,$ & $\,\,$ 1 $\,\,$ & $\,\,$\color{gr} 4.4739\color{black}  $\,\,$ \\
$\,\,$0.0745$\,\,$ & $\,\,$0.2117$\,\,$ & $\,\,$\color{gr} 0.2235\color{black} $\,\,$ & $\,\,$ 1  $\,\,$ \\
\end{pmatrix},
\end{equation*}
\end{example}
\newpage
\begin{example}
\begin{equation*}
\mathbf{A} =
\begin{pmatrix}
$\,\,$ 1 $\,\,$ & $\,\,$6$\,\,$ & $\,\,$4$\,\,$ & $\,\,$6 $\,\,$ \\
$\,\,$ 1/6$\,\,$ & $\,\,$ 1 $\,\,$ & $\,\,$1$\,\,$ & $\,\,$3 $\,\,$ \\
$\,\,$ 1/4$\,\,$ & $\,\,$ 1 $\,\,$ & $\,\,$ 1 $\,\,$ & $\,\,$2 $\,\,$ \\
$\,\,$ 1/6$\,\,$ & $\,\,$ 1/3$\,\,$ & $\,\,$ 1/2$\,\,$ & $\,\,$ 1  $\,\,$ \\
\end{pmatrix},
\qquad
\lambda_{\max} =
4.1031,
\qquad
CR = 0.0389
\end{equation*}

\begin{equation*}
\mathbf{w}^{cos} =
\begin{pmatrix}
0.610382\\
0.161222\\
\color{red} 0.150900\color{black} \\
0.077496
\end{pmatrix}\end{equation*}
\begin{equation*}
\left[ \frac{{w}^{cos}_i}{{w}^{cos}_j} \right] =
\begin{pmatrix}
$\,\,$ 1 $\,\,$ & $\,\,$3.7860$\,\,$ & $\,\,$\color{red} 4.0449\color{black} $\,\,$ & $\,\,$7.8763$\,\,$ \\
$\,\,$0.2641$\,\,$ & $\,\,$ 1 $\,\,$ & $\,\,$\color{red} 1.0684\color{black} $\,\,$ & $\,\,$2.0804  $\,\,$ \\
$\,\,$\color{red} 0.2472\color{black} $\,\,$ & $\,\,$\color{red} 0.9360\color{black} $\,\,$ & $\,\,$ 1 $\,\,$ & $\,\,$\color{red} 1.9472\color{black}  $\,\,$ \\
$\,\,$0.1270$\,\,$ & $\,\,$0.4807$\,\,$ & $\,\,$\color{red} 0.5136\color{black} $\,\,$ & $\,\,$ 1  $\,\,$ \\
\end{pmatrix},
\end{equation*}

\begin{equation*}
\mathbf{w}^{\prime} =
\begin{pmatrix}
0.609349\\
0.160949\\
0.152337\\
0.077365
\end{pmatrix} =
0.998308\cdot
\begin{pmatrix}
0.610382\\
0.161222\\
\color{gr} 0.152595\color{black} \\
0.077496
\end{pmatrix},
\end{equation*}
\begin{equation*}
\left[ \frac{{w}^{\prime}_i}{{w}^{\prime}_j} \right] =
\begin{pmatrix}
$\,\,$ 1 $\,\,$ & $\,\,$3.7860$\,\,$ & $\,\,$\color{gr} \color{blue} 4\color{black} $\,\,$ & $\,\,$7.8763$\,\,$ \\
$\,\,$0.2641$\,\,$ & $\,\,$ 1 $\,\,$ & $\,\,$\color{gr} 1.0565\color{black} $\,\,$ & $\,\,$2.0804  $\,\,$ \\
$\,\,$\color{gr} \color{blue}  1/4\color{black} $\,\,$ & $\,\,$\color{gr} 0.9465\color{black} $\,\,$ & $\,\,$ 1 $\,\,$ & $\,\,$\color{gr} 1.9691\color{black}  $\,\,$ \\
$\,\,$0.1270$\,\,$ & $\,\,$0.4807$\,\,$ & $\,\,$\color{gr} 0.5079\color{black} $\,\,$ & $\,\,$ 1  $\,\,$ \\
\end{pmatrix},
\end{equation*}
\end{example}
\newpage
\begin{example}
\begin{equation*}
\mathbf{A} =
\begin{pmatrix}
$\,\,$ 1 $\,\,$ & $\,\,$6$\,\,$ & $\,\,$4$\,\,$ & $\,\,$6 $\,\,$ \\
$\,\,$ 1/6$\,\,$ & $\,\,$ 1 $\,\,$ & $\,\,$3$\,\,$ & $\,\,$2 $\,\,$ \\
$\,\,$ 1/4$\,\,$ & $\,\,$ 1/3$\,\,$ & $\,\,$ 1 $\,\,$ & $\,\,$1 $\,\,$ \\
$\,\,$ 1/6$\,\,$ & $\,\,$ 1/2$\,\,$ & $\,\,$ 1 $\,\,$ & $\,\,$ 1  $\,\,$ \\
\end{pmatrix},
\qquad
\lambda_{\max} =
4.1990,
\qquad
CR = 0.0750
\end{equation*}

\begin{equation*}
\mathbf{w}^{cos} =
\begin{pmatrix}
0.598437\\
0.199401\\
0.105338\\
\color{red} 0.096824\color{black}
\end{pmatrix}\end{equation*}
\begin{equation*}
\left[ \frac{{w}^{cos}_i}{{w}^{cos}_j} \right] =
\begin{pmatrix}
$\,\,$ 1 $\,\,$ & $\,\,$3.0012$\,\,$ & $\,\,$5.6811$\,\,$ & $\,\,$\color{red} 6.1807\color{black} $\,\,$ \\
$\,\,$0.3332$\,\,$ & $\,\,$ 1 $\,\,$ & $\,\,$1.8930$\,\,$ & $\,\,$\color{red} 2.0594\color{black}   $\,\,$ \\
$\,\,$0.1760$\,\,$ & $\,\,$0.5283$\,\,$ & $\,\,$ 1 $\,\,$ & $\,\,$\color{red} 1.0879\color{black}  $\,\,$ \\
$\,\,$\color{red} 0.1618\color{black} $\,\,$ & $\,\,$\color{red} 0.4856\color{black} $\,\,$ & $\,\,$\color{red} 0.9192\color{black} $\,\,$ & $\,\,$ 1  $\,\,$ \\
\end{pmatrix},
\end{equation*}

\begin{equation*}
\mathbf{w}^{\prime} =
\begin{pmatrix}
0.596721\\
0.198829\\
0.105036\\
0.099414
\end{pmatrix} =
0.997132\cdot
\begin{pmatrix}
0.598437\\
0.199401\\
0.105338\\
\color{gr} 0.099700\color{black}
\end{pmatrix},
\end{equation*}
\begin{equation*}
\left[ \frac{{w}^{\prime}_i}{{w}^{\prime}_j} \right] =
\begin{pmatrix}
$\,\,$ 1 $\,\,$ & $\,\,$3.0012$\,\,$ & $\,\,$5.6811$\,\,$ & $\,\,$\color{gr} 6.0024\color{black} $\,\,$ \\
$\,\,$0.3332$\,\,$ & $\,\,$ 1 $\,\,$ & $\,\,$1.8930$\,\,$ & $\,\,$\color{gr} \color{blue} 2\color{black}   $\,\,$ \\
$\,\,$0.1760$\,\,$ & $\,\,$0.5283$\,\,$ & $\,\,$ 1 $\,\,$ & $\,\,$\color{gr} 1.0565\color{black}  $\,\,$ \\
$\,\,$\color{gr} 0.1666\color{black} $\,\,$ & $\,\,$\color{gr} \color{blue}  1/2\color{black} $\,\,$ & $\,\,$\color{gr} 0.9465\color{black} $\,\,$ & $\,\,$ 1  $\,\,$ \\
\end{pmatrix},
\end{equation*}
\end{example}
\newpage
\begin{example}
\begin{equation*}
\mathbf{A} =
\begin{pmatrix}
$\,\,$ 1 $\,\,$ & $\,\,$6$\,\,$ & $\,\,$4$\,\,$ & $\,\,$9 $\,\,$ \\
$\,\,$ 1/6$\,\,$ & $\,\,$ 1 $\,\,$ & $\,\,$1$\,\,$ & $\,\,$4 $\,\,$ \\
$\,\,$ 1/4$\,\,$ & $\,\,$ 1 $\,\,$ & $\,\,$ 1 $\,\,$ & $\,\,$3 $\,\,$ \\
$\,\,$ 1/9$\,\,$ & $\,\,$ 1/4$\,\,$ & $\,\,$ 1/3$\,\,$ & $\,\,$ 1  $\,\,$ \\
\end{pmatrix},
\qquad
\lambda_{\max} =
4.0820,
\qquad
CR = 0.0309
\end{equation*}

\begin{equation*}
\mathbf{w}^{cos} =
\begin{pmatrix}
0.630477\\
0.159100\\
\color{red} 0.156178\color{black} \\
0.054245
\end{pmatrix}\end{equation*}
\begin{equation*}
\left[ \frac{{w}^{cos}_i}{{w}^{cos}_j} \right] =
\begin{pmatrix}
$\,\,$ 1 $\,\,$ & $\,\,$3.9628$\,\,$ & $\,\,$\color{red} 4.0369\color{black} $\,\,$ & $\,\,$11.6228$\,\,$ \\
$\,\,$0.2523$\,\,$ & $\,\,$ 1 $\,\,$ & $\,\,$\color{red} 1.0187\color{black} $\,\,$ & $\,\,$2.9330  $\,\,$ \\
$\,\,$\color{red} 0.2477\color{black} $\,\,$ & $\,\,$\color{red} 0.9816\color{black} $\,\,$ & $\,\,$ 1 $\,\,$ & $\,\,$\color{red} 2.8791\color{black}  $\,\,$ \\
$\,\,$0.0860$\,\,$ & $\,\,$0.3409$\,\,$ & $\,\,$\color{red} 0.3473\color{black} $\,\,$ & $\,\,$ 1  $\,\,$ \\
\end{pmatrix},
\end{equation*}

\begin{equation*}
\mathbf{w}^{\prime} =
\begin{pmatrix}
0.629570\\
0.158871\\
0.157392\\
0.054167
\end{pmatrix} =
0.998561\cdot
\begin{pmatrix}
0.630477\\
0.159100\\
\color{gr} 0.157619\color{black} \\
0.054245
\end{pmatrix},
\end{equation*}
\begin{equation*}
\left[ \frac{{w}^{\prime}_i}{{w}^{\prime}_j} \right] =
\begin{pmatrix}
$\,\,$ 1 $\,\,$ & $\,\,$3.9628$\,\,$ & $\,\,$\color{gr} \color{blue} 4\color{black} $\,\,$ & $\,\,$11.6228$\,\,$ \\
$\,\,$0.2523$\,\,$ & $\,\,$ 1 $\,\,$ & $\,\,$\color{gr} 1.0094\color{black} $\,\,$ & $\,\,$2.9330  $\,\,$ \\
$\,\,$\color{gr} \color{blue}  1/4\color{black} $\,\,$ & $\,\,$\color{gr} 0.9907\color{black} $\,\,$ & $\,\,$ 1 $\,\,$ & $\,\,$\color{gr} 2.9057\color{black}  $\,\,$ \\
$\,\,$0.0860$\,\,$ & $\,\,$0.3409$\,\,$ & $\,\,$\color{gr} 0.3442\color{black} $\,\,$ & $\,\,$ 1  $\,\,$ \\
\end{pmatrix},
\end{equation*}
\end{example}
\newpage
\begin{example}
\begin{equation*}
\mathbf{A} =
\begin{pmatrix}
$\,\,$ 1 $\,\,$ & $\,\,$6$\,\,$ & $\,\,$4$\,\,$ & $\,\,$9 $\,\,$ \\
$\,\,$ 1/6$\,\,$ & $\,\,$ 1 $\,\,$ & $\,\,$1$\,\,$ & $\,\,$5 $\,\,$ \\
$\,\,$ 1/4$\,\,$ & $\,\,$ 1 $\,\,$ & $\,\,$ 1 $\,\,$ & $\,\,$3 $\,\,$ \\
$\,\,$ 1/9$\,\,$ & $\,\,$ 1/5$\,\,$ & $\,\,$ 1/3$\,\,$ & $\,\,$ 1  $\,\,$ \\
\end{pmatrix},
\qquad
\lambda_{\max} =
4.1252,
\qquad
CR = 0.0472
\end{equation*}

\begin{equation*}
\mathbf{w}^{cos} =
\begin{pmatrix}
0.622601\\
0.171617\\
\color{red} 0.153727\color{black} \\
0.052056
\end{pmatrix}\end{equation*}
\begin{equation*}
\left[ \frac{{w}^{cos}_i}{{w}^{cos}_j} \right] =
\begin{pmatrix}
$\,\,$ 1 $\,\,$ & $\,\,$3.6279$\,\,$ & $\,\,$\color{red} 4.0500\color{black} $\,\,$ & $\,\,$11.9603$\,\,$ \\
$\,\,$0.2756$\,\,$ & $\,\,$ 1 $\,\,$ & $\,\,$\color{red} 1.1164\color{black} $\,\,$ & $\,\,$3.2968  $\,\,$ \\
$\,\,$\color{red} 0.2469\color{black} $\,\,$ & $\,\,$\color{red} 0.8958\color{black} $\,\,$ & $\,\,$ 1 $\,\,$ & $\,\,$\color{red} 2.9531\color{black}  $\,\,$ \\
$\,\,$0.0836$\,\,$ & $\,\,$0.3033$\,\,$ & $\,\,$\color{red} 0.3386\color{black} $\,\,$ & $\,\,$ 1  $\,\,$ \\
\end{pmatrix},
\end{equation*}

\begin{equation*}
\mathbf{w}^{\prime} =
\begin{pmatrix}
0.621406\\
0.171287\\
0.155351\\
0.051956
\end{pmatrix} =
0.998080\cdot
\begin{pmatrix}
0.622601\\
0.171617\\
\color{gr} 0.155650\color{black} \\
0.052056
\end{pmatrix},
\end{equation*}
\begin{equation*}
\left[ \frac{{w}^{\prime}_i}{{w}^{\prime}_j} \right] =
\begin{pmatrix}
$\,\,$ 1 $\,\,$ & $\,\,$3.6279$\,\,$ & $\,\,$\color{gr} \color{blue} 4\color{black} $\,\,$ & $\,\,$11.9603$\,\,$ \\
$\,\,$0.2756$\,\,$ & $\,\,$ 1 $\,\,$ & $\,\,$\color{gr} 1.1026\color{black} $\,\,$ & $\,\,$3.2968  $\,\,$ \\
$\,\,$\color{gr} \color{blue}  1/4\color{black} $\,\,$ & $\,\,$\color{gr} 0.9070\color{black} $\,\,$ & $\,\,$ 1 $\,\,$ & $\,\,$\color{gr} 2.9901\color{black}  $\,\,$ \\
$\,\,$0.0836$\,\,$ & $\,\,$0.3033$\,\,$ & $\,\,$\color{gr} 0.3344\color{black} $\,\,$ & $\,\,$ 1  $\,\,$ \\
\end{pmatrix},
\end{equation*}
\end{example}
\newpage
\begin{example}
\begin{equation*}
\mathbf{A} =
\begin{pmatrix}
$\,\,$ 1 $\,\,$ & $\,\,$6$\,\,$ & $\,\,$5$\,\,$ & $\,\,$6 $\,\,$ \\
$\,\,$ 1/6$\,\,$ & $\,\,$ 1 $\,\,$ & $\,\,$3$\,\,$ & $\,\,$2 $\,\,$ \\
$\,\,$ 1/5$\,\,$ & $\,\,$ 1/3$\,\,$ & $\,\,$ 1 $\,\,$ & $\,\,$1 $\,\,$ \\
$\,\,$ 1/6$\,\,$ & $\,\,$ 1/2$\,\,$ & $\,\,$ 1 $\,\,$ & $\,\,$ 1  $\,\,$ \\
\end{pmatrix},
\qquad
\lambda_{\max} =
4.1488,
\qquad
CR = 0.0561
\end{equation*}

\begin{equation*}
\mathbf{w}^{cos} =
\begin{pmatrix}
0.620558\\
0.189961\\
0.095125\\
\color{red} 0.094356\color{black}
\end{pmatrix}\end{equation*}
\begin{equation*}
\left[ \frac{{w}^{cos}_i}{{w}^{cos}_j} \right] =
\begin{pmatrix}
$\,\,$ 1 $\,\,$ & $\,\,$3.2668$\,\,$ & $\,\,$6.5236$\,\,$ & $\,\,$\color{red} 6.5768\color{black} $\,\,$ \\
$\,\,$0.3061$\,\,$ & $\,\,$ 1 $\,\,$ & $\,\,$1.9970$\,\,$ & $\,\,$\color{red} 2.0132\color{black}   $\,\,$ \\
$\,\,$0.1533$\,\,$ & $\,\,$0.5008$\,\,$ & $\,\,$ 1 $\,\,$ & $\,\,$\color{red} 1.0081\color{black}  $\,\,$ \\
$\,\,$\color{red} 0.1521\color{black} $\,\,$ & $\,\,$\color{red} 0.4967\color{black} $\,\,$ & $\,\,$\color{red} 0.9919\color{black} $\,\,$ & $\,\,$ 1  $\,\,$ \\
\end{pmatrix},
\end{equation*}

\begin{equation*}
\mathbf{w}^{\prime} =
\begin{pmatrix}
0.620171\\
0.189842\\
0.095066\\
0.094921
\end{pmatrix} =
0.999376\cdot
\begin{pmatrix}
0.620558\\
0.189961\\
0.095125\\
\color{gr} 0.094980\color{black}
\end{pmatrix},
\end{equation*}
\begin{equation*}
\left[ \frac{{w}^{\prime}_i}{{w}^{\prime}_j} \right] =
\begin{pmatrix}
$\,\,$ 1 $\,\,$ & $\,\,$3.2668$\,\,$ & $\,\,$6.5236$\,\,$ & $\,\,$\color{gr} 6.5335\color{black} $\,\,$ \\
$\,\,$0.3061$\,\,$ & $\,\,$ 1 $\,\,$ & $\,\,$1.9970$\,\,$ & $\,\,$\color{gr} \color{blue} 2\color{black}   $\,\,$ \\
$\,\,$0.1533$\,\,$ & $\,\,$0.5008$\,\,$ & $\,\,$ 1 $\,\,$ & $\,\,$\color{gr} 1.0015\color{black}  $\,\,$ \\
$\,\,$\color{gr} 0.1531\color{black} $\,\,$ & $\,\,$\color{gr} \color{blue}  1/2\color{black} $\,\,$ & $\,\,$\color{gr} 0.9985\color{black} $\,\,$ & $\,\,$ 1  $\,\,$ \\
\end{pmatrix},
\end{equation*}
\end{example}
\newpage
\begin{example}
\begin{equation*}
\mathbf{A} =
\begin{pmatrix}
$\,\,$ 1 $\,\,$ & $\,\,$6$\,\,$ & $\,\,$5$\,\,$ & $\,\,$7 $\,\,$ \\
$\,\,$ 1/6$\,\,$ & $\,\,$ 1 $\,\,$ & $\,\,$3$\,\,$ & $\,\,$2 $\,\,$ \\
$\,\,$ 1/5$\,\,$ & $\,\,$ 1/3$\,\,$ & $\,\,$ 1 $\,\,$ & $\,\,$1 $\,\,$ \\
$\,\,$ 1/7$\,\,$ & $\,\,$ 1/2$\,\,$ & $\,\,$ 1 $\,\,$ & $\,\,$ 1  $\,\,$ \\
\end{pmatrix},
\qquad
\lambda_{\max} =
4.1417,
\qquad
CR = 0.0534
\end{equation*}

\begin{equation*}
\mathbf{w}^{cos} =
\begin{pmatrix}
0.632378\\
0.185940\\
0.093181\\
\color{red} 0.088501\color{black}
\end{pmatrix}\end{equation*}
\begin{equation*}
\left[ \frac{{w}^{cos}_i}{{w}^{cos}_j} \right] =
\begin{pmatrix}
$\,\,$ 1 $\,\,$ & $\,\,$3.4010$\,\,$ & $\,\,$6.7866$\,\,$ & $\,\,$\color{red} 7.1455\color{black} $\,\,$ \\
$\,\,$0.2940$\,\,$ & $\,\,$ 1 $\,\,$ & $\,\,$1.9955$\,\,$ & $\,\,$\color{red} 2.1010\color{black}   $\,\,$ \\
$\,\,$0.1473$\,\,$ & $\,\,$0.5011$\,\,$ & $\,\,$ 1 $\,\,$ & $\,\,$\color{red} 1.0529\color{black}  $\,\,$ \\
$\,\,$\color{red} 0.1399\color{black} $\,\,$ & $\,\,$\color{red} 0.4760\color{black} $\,\,$ & $\,\,$\color{red} 0.9498\color{black} $\,\,$ & $\,\,$ 1  $\,\,$ \\
\end{pmatrix},
\end{equation*}

\begin{equation*}
\mathbf{w}^{\prime} =
\begin{pmatrix}
0.631218\\
0.185599\\
0.093010\\
0.090174
\end{pmatrix} =
0.998164\cdot
\begin{pmatrix}
0.632378\\
0.185940\\
0.093181\\
\color{gr} 0.090340\color{black}
\end{pmatrix},
\end{equation*}
\begin{equation*}
\left[ \frac{{w}^{\prime}_i}{{w}^{\prime}_j} \right] =
\begin{pmatrix}
$\,\,$ 1 $\,\,$ & $\,\,$3.4010$\,\,$ & $\,\,$6.7866$\,\,$ & $\,\,$\color{gr} \color{blue} 7\color{black} $\,\,$ \\
$\,\,$0.2940$\,\,$ & $\,\,$ 1 $\,\,$ & $\,\,$1.9955$\,\,$ & $\,\,$\color{gr} 2.0582\color{black}   $\,\,$ \\
$\,\,$0.1473$\,\,$ & $\,\,$0.5011$\,\,$ & $\,\,$ 1 $\,\,$ & $\,\,$\color{gr} 1.0314\color{black}  $\,\,$ \\
$\,\,$\color{gr} \color{blue}  1/7\color{black} $\,\,$ & $\,\,$\color{gr} 0.4859\color{black} $\,\,$ & $\,\,$\color{gr} 0.9695\color{black} $\,\,$ & $\,\,$ 1  $\,\,$ \\
\end{pmatrix},
\end{equation*}
\end{example}
\newpage
\begin{example}
\begin{equation*}
\mathbf{A} =
\begin{pmatrix}
$\,\,$ 1 $\,\,$ & $\,\,$6$\,\,$ & $\,\,$5$\,\,$ & $\,\,$7 $\,\,$ \\
$\,\,$ 1/6$\,\,$ & $\,\,$ 1 $\,\,$ & $\,\,$4$\,\,$ & $\,\,$2 $\,\,$ \\
$\,\,$ 1/5$\,\,$ & $\,\,$ 1/4$\,\,$ & $\,\,$ 1 $\,\,$ & $\,\,$1 $\,\,$ \\
$\,\,$ 1/7$\,\,$ & $\,\,$ 1/2$\,\,$ & $\,\,$ 1 $\,\,$ & $\,\,$ 1  $\,\,$ \\
\end{pmatrix},
\qquad
\lambda_{\max} =
4.2174,
\qquad
CR = 0.0820
\end{equation*}

\begin{equation*}
\mathbf{w}^{cos} =
\begin{pmatrix}
0.620647\\
0.204739\\
0.088486\\
\color{red} 0.086129\color{black}
\end{pmatrix}\end{equation*}
\begin{equation*}
\left[ \frac{{w}^{cos}_i}{{w}^{cos}_j} \right] =
\begin{pmatrix}
$\,\,$ 1 $\,\,$ & $\,\,$3.0314$\,\,$ & $\,\,$7.0141$\,\,$ & $\,\,$\color{red} 7.2060\color{black} $\,\,$ \\
$\,\,$0.3299$\,\,$ & $\,\,$ 1 $\,\,$ & $\,\,$2.3138$\,\,$ & $\,\,$\color{red} 2.3771\color{black}   $\,\,$ \\
$\,\,$0.1426$\,\,$ & $\,\,$0.4322$\,\,$ & $\,\,$ 1 $\,\,$ & $\,\,$\color{red} 1.0274\color{black}  $\,\,$ \\
$\,\,$\color{red} 0.1388\color{black} $\,\,$ & $\,\,$\color{red} 0.4207\color{black} $\,\,$ & $\,\,$\color{red} 0.9734\color{black} $\,\,$ & $\,\,$ 1  $\,\,$ \\
\end{pmatrix},
\end{equation*}

\begin{equation*}
\mathbf{w}^{\prime} =
\begin{pmatrix}
0.619187\\
0.204257\\
0.088278\\
0.088278
\end{pmatrix} =
0.997648\cdot
\begin{pmatrix}
0.620647\\
0.204739\\
0.088486\\
\color{gr} 0.088486\color{black}
\end{pmatrix},
\end{equation*}
\begin{equation*}
\left[ \frac{{w}^{\prime}_i}{{w}^{\prime}_j} \right] =
\begin{pmatrix}
$\,\,$ 1 $\,\,$ & $\,\,$3.0314$\,\,$ & $\,\,$7.0141$\,\,$ & $\,\,$\color{gr} 7.0141\color{black} $\,\,$ \\
$\,\,$0.3299$\,\,$ & $\,\,$ 1 $\,\,$ & $\,\,$2.3138$\,\,$ & $\,\,$\color{gr} 2.3138\color{black}   $\,\,$ \\
$\,\,$0.1426$\,\,$ & $\,\,$0.4322$\,\,$ & $\,\,$ 1 $\,\,$ & $\,\,$\color{gr} \color{blue} 1\color{black}  $\,\,$ \\
$\,\,$\color{gr} 0.1426\color{black} $\,\,$ & $\,\,$\color{gr} 0.4322\color{black} $\,\,$ & $\,\,$\color{gr} \color{blue} 1\color{black} $\,\,$ & $\,\,$ 1  $\,\,$ \\
\end{pmatrix},
\end{equation*}
\end{example}
\newpage
\begin{example}
\begin{equation*}
\mathbf{A} =
\begin{pmatrix}
$\,\,$ 1 $\,\,$ & $\,\,$6$\,\,$ & $\,\,$5$\,\,$ & $\,\,$8 $\,\,$ \\
$\,\,$ 1/6$\,\,$ & $\,\,$ 1 $\,\,$ & $\,\,$2$\,\,$ & $\,\,$7 $\,\,$ \\
$\,\,$ 1/5$\,\,$ & $\,\,$ 1/2$\,\,$ & $\,\,$ 1 $\,\,$ & $\,\,$2 $\,\,$ \\
$\,\,$ 1/8$\,\,$ & $\,\,$ 1/7$\,\,$ & $\,\,$ 1/2$\,\,$ & $\,\,$ 1  $\,\,$ \\
\end{pmatrix},
\qquad
\lambda_{\max} =
4.2531,
\qquad
CR = 0.0954
\end{equation*}

\begin{equation*}
\mathbf{w}^{cos} =
\begin{pmatrix}
0.609613\\
0.226236\\
\color{red} 0.108702\color{black} \\
0.055450
\end{pmatrix}\end{equation*}
\begin{equation*}
\left[ \frac{{w}^{cos}_i}{{w}^{cos}_j} \right] =
\begin{pmatrix}
$\,\,$ 1 $\,\,$ & $\,\,$2.6946$\,\,$ & $\,\,$\color{red} 5.6081\color{black} $\,\,$ & $\,\,$10.9940$\,\,$ \\
$\,\,$0.3711$\,\,$ & $\,\,$ 1 $\,\,$ & $\,\,$\color{red} 2.0813\color{black} $\,\,$ & $\,\,$4.0800  $\,\,$ \\
$\,\,$\color{red} 0.1783\color{black} $\,\,$ & $\,\,$\color{red} 0.4805\color{black} $\,\,$ & $\,\,$ 1 $\,\,$ & $\,\,$\color{red} 1.9604\color{black}  $\,\,$ \\
$\,\,$0.0910$\,\,$ & $\,\,$0.2451$\,\,$ & $\,\,$\color{red} 0.5101\color{black} $\,\,$ & $\,\,$ 1  $\,\,$ \\
\end{pmatrix},
\end{equation*}

\begin{equation*}
\mathbf{w}^{\prime} =
\begin{pmatrix}
0.608277\\
0.225740\\
0.110656\\
0.055328
\end{pmatrix} =
0.997808\cdot
\begin{pmatrix}
0.609613\\
0.226236\\
\color{gr} 0.110899\color{black} \\
0.055450
\end{pmatrix},
\end{equation*}
\begin{equation*}
\left[ \frac{{w}^{\prime}_i}{{w}^{\prime}_j} \right] =
\begin{pmatrix}
$\,\,$ 1 $\,\,$ & $\,\,$2.6946$\,\,$ & $\,\,$\color{gr} 5.4970\color{black} $\,\,$ & $\,\,$10.9940$\,\,$ \\
$\,\,$0.3711$\,\,$ & $\,\,$ 1 $\,\,$ & $\,\,$\color{gr} 2.0400\color{black} $\,\,$ & $\,\,$4.0800  $\,\,$ \\
$\,\,$\color{gr} 0.1819\color{black} $\,\,$ & $\,\,$\color{gr} 0.4902\color{black} $\,\,$ & $\,\,$ 1 $\,\,$ & $\,\,$\color{gr} \color{blue} 2\color{black}  $\,\,$ \\
$\,\,$0.0910$\,\,$ & $\,\,$0.2451$\,\,$ & $\,\,$\color{gr} \color{blue}  1/2\color{black} $\,\,$ & $\,\,$ 1  $\,\,$ \\
\end{pmatrix},
\end{equation*}
\end{example}
\newpage
\begin{example}
\begin{equation*}
\mathbf{A} =
\begin{pmatrix}
$\,\,$ 1 $\,\,$ & $\,\,$6$\,\,$ & $\,\,$6$\,\,$ & $\,\,$8 $\,\,$ \\
$\,\,$ 1/6$\,\,$ & $\,\,$ 1 $\,\,$ & $\,\,$2$\,\,$ & $\,\,$6 $\,\,$ \\
$\,\,$ 1/6$\,\,$ & $\,\,$ 1/2$\,\,$ & $\,\,$ 1 $\,\,$ & $\,\,$2 $\,\,$ \\
$\,\,$ 1/8$\,\,$ & $\,\,$ 1/6$\,\,$ & $\,\,$ 1/2$\,\,$ & $\,\,$ 1  $\,\,$ \\
\end{pmatrix},
\qquad
\lambda_{\max} =
4.1990,
\qquad
CR = 0.0750
\end{equation*}

\begin{equation*}
\mathbf{w}^{cos} =
\begin{pmatrix}
0.631379\\
0.210542\\
\color{red} 0.102329\color{black} \\
0.055750
\end{pmatrix}\end{equation*}
\begin{equation*}
\left[ \frac{{w}^{cos}_i}{{w}^{cos}_j} \right] =
\begin{pmatrix}
$\,\,$ 1 $\,\,$ & $\,\,$2.9988$\,\,$ & $\,\,$\color{red} 6.1701\color{black} $\,\,$ & $\,\,$11.3252$\,\,$ \\
$\,\,$0.3335$\,\,$ & $\,\,$ 1 $\,\,$ & $\,\,$\color{red} 2.0575\color{black} $\,\,$ & $\,\,$3.7765  $\,\,$ \\
$\,\,$\color{red} 0.1621\color{black} $\,\,$ & $\,\,$\color{red} 0.4860\color{black} $\,\,$ & $\,\,$ 1 $\,\,$ & $\,\,$\color{red} 1.8355\color{black}  $\,\,$ \\
$\,\,$0.0883$\,\,$ & $\,\,$0.2648$\,\,$ & $\,\,$\color{red} 0.5448\color{black} $\,\,$ & $\,\,$ 1  $\,\,$ \\
\end{pmatrix},
\end{equation*}

\begin{equation*}
\mathbf{w}^{\prime} =
\begin{pmatrix}
0.629552\\
0.209933\\
0.104925\\
0.055589
\end{pmatrix} =
0.997108\cdot
\begin{pmatrix}
0.631379\\
0.210542\\
\color{gr} 0.105230\color{black} \\
0.055750
\end{pmatrix},
\end{equation*}
\begin{equation*}
\left[ \frac{{w}^{\prime}_i}{{w}^{\prime}_j} \right] =
\begin{pmatrix}
$\,\,$ 1 $\,\,$ & $\,\,$2.9988$\,\,$ & $\,\,$\color{gr} \color{blue} 6\color{black} $\,\,$ & $\,\,$11.3252$\,\,$ \\
$\,\,$0.3335$\,\,$ & $\,\,$ 1 $\,\,$ & $\,\,$\color{gr} 2.0008\color{black} $\,\,$ & $\,\,$3.7765  $\,\,$ \\
$\,\,$\color{gr} \color{blue}  1/6\color{black} $\,\,$ & $\,\,$\color{gr} 0.4998\color{black} $\,\,$ & $\,\,$ 1 $\,\,$ & $\,\,$\color{gr} 1.8875\color{black}  $\,\,$ \\
$\,\,$0.0883$\,\,$ & $\,\,$0.2648$\,\,$ & $\,\,$\color{gr} 0.5298\color{black} $\,\,$ & $\,\,$ 1  $\,\,$ \\
\end{pmatrix},
\end{equation*}
\end{example}
\newpage
\begin{example}
\begin{equation*}
\mathbf{A} =
\begin{pmatrix}
$\,\,$ 1 $\,\,$ & $\,\,$6$\,\,$ & $\,\,$6$\,\,$ & $\,\,$8 $\,\,$ \\
$\,\,$ 1/6$\,\,$ & $\,\,$ 1 $\,\,$ & $\,\,$2$\,\,$ & $\,\,$7 $\,\,$ \\
$\,\,$ 1/6$\,\,$ & $\,\,$ 1/2$\,\,$ & $\,\,$ 1 $\,\,$ & $\,\,$2 $\,\,$ \\
$\,\,$ 1/8$\,\,$ & $\,\,$ 1/7$\,\,$ & $\,\,$ 1/2$\,\,$ & $\,\,$ 1  $\,\,$ \\
\end{pmatrix},
\qquad
\lambda_{\max} =
4.2421,
\qquad
CR = 0.0913
\end{equation*}

\begin{equation*}
\mathbf{w}^{cos} =
\begin{pmatrix}
0.624389\\
0.220908\\
\color{red} 0.100524\color{black} \\
0.054179
\end{pmatrix}\end{equation*}
\begin{equation*}
\left[ \frac{{w}^{cos}_i}{{w}^{cos}_j} \right] =
\begin{pmatrix}
$\,\,$ 1 $\,\,$ & $\,\,$2.8265$\,\,$ & $\,\,$\color{red} 6.2114\color{black} $\,\,$ & $\,\,$11.5246$\,\,$ \\
$\,\,$0.3538$\,\,$ & $\,\,$ 1 $\,\,$ & $\,\,$\color{red} 2.1976\color{black} $\,\,$ & $\,\,$4.0774  $\,\,$ \\
$\,\,$\color{red} 0.1610\color{black} $\,\,$ & $\,\,$\color{red} 0.4550\color{black} $\,\,$ & $\,\,$ 1 $\,\,$ & $\,\,$\color{red} 1.8554\color{black}  $\,\,$ \\
$\,\,$0.0868$\,\,$ & $\,\,$0.2453$\,\,$ & $\,\,$\color{red} 0.5390\color{black} $\,\,$ & $\,\,$ 1  $\,\,$ \\
\end{pmatrix},
\end{equation*}

\begin{equation*}
\mathbf{w}^{\prime} =
\begin{pmatrix}
0.622186\\
0.220129\\
0.103698\\
0.053988
\end{pmatrix} =
0.996471\cdot
\begin{pmatrix}
0.624389\\
0.220908\\
\color{gr} 0.104065\color{black} \\
0.054179
\end{pmatrix},
\end{equation*}
\begin{equation*}
\left[ \frac{{w}^{\prime}_i}{{w}^{\prime}_j} \right] =
\begin{pmatrix}
$\,\,$ 1 $\,\,$ & $\,\,$2.8265$\,\,$ & $\,\,$\color{gr} \color{blue} 6\color{black} $\,\,$ & $\,\,$11.5246$\,\,$ \\
$\,\,$0.3538$\,\,$ & $\,\,$ 1 $\,\,$ & $\,\,$\color{gr} 2.1228\color{black} $\,\,$ & $\,\,$4.0774  $\,\,$ \\
$\,\,$\color{gr} \color{blue}  1/6\color{black} $\,\,$ & $\,\,$\color{gr} 0.4711\color{black} $\,\,$ & $\,\,$ 1 $\,\,$ & $\,\,$\color{gr} 1.9208\color{black}  $\,\,$ \\
$\,\,$0.0868$\,\,$ & $\,\,$0.2453$\,\,$ & $\,\,$\color{gr} 0.5206\color{black} $\,\,$ & $\,\,$ 1  $\,\,$ \\
\end{pmatrix},
\end{equation*}
\end{example}
\newpage
\begin{example}
\begin{equation*}
\mathbf{A} =
\begin{pmatrix}
$\,\,$ 1 $\,\,$ & $\,\,$6$\,\,$ & $\,\,$6$\,\,$ & $\,\,$8 $\,\,$ \\
$\,\,$ 1/6$\,\,$ & $\,\,$ 1 $\,\,$ & $\,\,$3$\,\,$ & $\,\,$2 $\,\,$ \\
$\,\,$ 1/6$\,\,$ & $\,\,$ 1/3$\,\,$ & $\,\,$ 1 $\,\,$ & $\,\,$1 $\,\,$ \\
$\,\,$ 1/8$\,\,$ & $\,\,$ 1/2$\,\,$ & $\,\,$ 1 $\,\,$ & $\,\,$ 1  $\,\,$ \\
\end{pmatrix},
\qquad
\lambda_{\max} =
4.1031,
\qquad
CR = 0.0389
\end{equation*}

\begin{equation*}
\mathbf{w}^{cos} =
\begin{pmatrix}
0.659615\\
0.174762\\
0.083945\\
\color{red} 0.081678\color{black}
\end{pmatrix}\end{equation*}
\begin{equation*}
\left[ \frac{{w}^{cos}_i}{{w}^{cos}_j} \right] =
\begin{pmatrix}
$\,\,$ 1 $\,\,$ & $\,\,$3.7744$\,\,$ & $\,\,$7.8577$\,\,$ & $\,\,$\color{red} 8.0758\color{black} $\,\,$ \\
$\,\,$0.2649$\,\,$ & $\,\,$ 1 $\,\,$ & $\,\,$2.0818$\,\,$ & $\,\,$\color{red} 2.1396\color{black}   $\,\,$ \\
$\,\,$0.1273$\,\,$ & $\,\,$0.4803$\,\,$ & $\,\,$ 1 $\,\,$ & $\,\,$\color{red} 1.0278\color{black}  $\,\,$ \\
$\,\,$\color{red} 0.1238\color{black} $\,\,$ & $\,\,$\color{red} 0.4674\color{black} $\,\,$ & $\,\,$\color{red} 0.9730\color{black} $\,\,$ & $\,\,$ 1  $\,\,$ \\
\end{pmatrix},
\end{equation*}

\begin{equation*}
\mathbf{w}^{\prime} =
\begin{pmatrix}
0.659105\\
0.174626\\
0.083880\\
0.082388
\end{pmatrix} =
0.999227\cdot
\begin{pmatrix}
0.659615\\
0.174762\\
0.083945\\
\color{gr} 0.082452\color{black}
\end{pmatrix},
\end{equation*}
\begin{equation*}
\left[ \frac{{w}^{\prime}_i}{{w}^{\prime}_j} \right] =
\begin{pmatrix}
$\,\,$ 1 $\,\,$ & $\,\,$3.7744$\,\,$ & $\,\,$7.8577$\,\,$ & $\,\,$\color{gr} \color{blue} 8\color{black} $\,\,$ \\
$\,\,$0.2649$\,\,$ & $\,\,$ 1 $\,\,$ & $\,\,$2.0818$\,\,$ & $\,\,$\color{gr} 2.1196\color{black}   $\,\,$ \\
$\,\,$0.1273$\,\,$ & $\,\,$0.4803$\,\,$ & $\,\,$ 1 $\,\,$ & $\,\,$\color{gr} 1.0181\color{black}  $\,\,$ \\
$\,\,$\color{gr} \color{blue}  1/8\color{black} $\,\,$ & $\,\,$\color{gr} 0.4718\color{black} $\,\,$ & $\,\,$\color{gr} 0.9822\color{black} $\,\,$ & $\,\,$ 1  $\,\,$ \\
\end{pmatrix},
\end{equation*}
\end{example}
\newpage
\begin{example}
\begin{equation*}
\mathbf{A} =
\begin{pmatrix}
$\,\,$ 1 $\,\,$ & $\,\,$6$\,\,$ & $\,\,$6$\,\,$ & $\,\,$8 $\,\,$ \\
$\,\,$ 1/6$\,\,$ & $\,\,$ 1 $\,\,$ & $\,\,$5$\,\,$ & $\,\,$3 $\,\,$ \\
$\,\,$ 1/6$\,\,$ & $\,\,$ 1/5$\,\,$ & $\,\,$ 1 $\,\,$ & $\,\,$1 $\,\,$ \\
$\,\,$ 1/8$\,\,$ & $\,\,$ 1/3$\,\,$ & $\,\,$ 1 $\,\,$ & $\,\,$ 1  $\,\,$ \\
\end{pmatrix},
\qquad
\lambda_{\max} =
4.2339,
\qquad
CR = 0.0882
\end{equation*}

\begin{equation*}
\mathbf{w}^{cos} =
\begin{pmatrix}
0.627646\\
0.225010\\
0.075261\\
\color{red} 0.072082\color{black}
\end{pmatrix}\end{equation*}
\begin{equation*}
\left[ \frac{{w}^{cos}_i}{{w}^{cos}_j} \right] =
\begin{pmatrix}
$\,\,$ 1 $\,\,$ & $\,\,$2.7894$\,\,$ & $\,\,$8.3396$\,\,$ & $\,\,$\color{red} 8.7073\color{black} $\,\,$ \\
$\,\,$0.3585$\,\,$ & $\,\,$ 1 $\,\,$ & $\,\,$2.9897$\,\,$ & $\,\,$\color{red} 3.1216\color{black}   $\,\,$ \\
$\,\,$0.1199$\,\,$ & $\,\,$0.3345$\,\,$ & $\,\,$ 1 $\,\,$ & $\,\,$\color{red} 1.0441\color{black}  $\,\,$ \\
$\,\,$\color{red} 0.1148\color{black} $\,\,$ & $\,\,$\color{red} 0.3204\color{black} $\,\,$ & $\,\,$\color{red} 0.9578\color{black} $\,\,$ & $\,\,$ 1  $\,\,$ \\
\end{pmatrix},
\end{equation*}

\begin{equation*}
\mathbf{w}^{\prime} =
\begin{pmatrix}
0.625818\\
0.224355\\
0.075042\\
0.074785
\end{pmatrix} =
0.997088\cdot
\begin{pmatrix}
0.627646\\
0.225010\\
0.075261\\
\color{gr} 0.075003\color{black}
\end{pmatrix},
\end{equation*}
\begin{equation*}
\left[ \frac{{w}^{\prime}_i}{{w}^{\prime}_j} \right] =
\begin{pmatrix}
$\,\,$ 1 $\,\,$ & $\,\,$2.7894$\,\,$ & $\,\,$8.3396$\,\,$ & $\,\,$\color{gr} 8.3682\color{black} $\,\,$ \\
$\,\,$0.3585$\,\,$ & $\,\,$ 1 $\,\,$ & $\,\,$2.9897$\,\,$ & $\,\,$\color{gr} \color{blue} 3\color{black}   $\,\,$ \\
$\,\,$0.1199$\,\,$ & $\,\,$0.3345$\,\,$ & $\,\,$ 1 $\,\,$ & $\,\,$\color{gr} 1.0034\color{black}  $\,\,$ \\
$\,\,$\color{gr} 0.1195\color{black} $\,\,$ & $\,\,$\color{gr} \color{blue}  1/3\color{black} $\,\,$ & $\,\,$\color{gr} 0.9966\color{black} $\,\,$ & $\,\,$ 1  $\,\,$ \\
\end{pmatrix},
\end{equation*}
\end{example}
\newpage
\begin{example}
\begin{equation*}
\mathbf{A} =
\begin{pmatrix}
$\,\,$ 1 $\,\,$ & $\,\,$6$\,\,$ & $\,\,$6$\,\,$ & $\,\,$9 $\,\,$ \\
$\,\,$ 1/6$\,\,$ & $\,\,$ 1 $\,\,$ & $\,\,$2$\,\,$ & $\,\,$6 $\,\,$ \\
$\,\,$ 1/6$\,\,$ & $\,\,$ 1/2$\,\,$ & $\,\,$ 1 $\,\,$ & $\,\,$2 $\,\,$ \\
$\,\,$ 1/9$\,\,$ & $\,\,$ 1/6$\,\,$ & $\,\,$ 1/2$\,\,$ & $\,\,$ 1  $\,\,$ \\
\end{pmatrix},
\qquad
\lambda_{\max} =
4.1707,
\qquad
CR = 0.0644
\end{equation*}

\begin{equation*}
\mathbf{w}^{cos} =
\begin{pmatrix}
0.642010\\
0.204726\\
\color{red} 0.100646\color{black} \\
0.052619
\end{pmatrix}\end{equation*}
\begin{equation*}
\left[ \frac{{w}^{cos}_i}{{w}^{cos}_j} \right] =
\begin{pmatrix}
$\,\,$ 1 $\,\,$ & $\,\,$3.1359$\,\,$ & $\,\,$\color{red} 6.3789\color{black} $\,\,$ & $\,\,$12.2011$\,\,$ \\
$\,\,$0.3189$\,\,$ & $\,\,$ 1 $\,\,$ & $\,\,$\color{red} 2.0341\color{black} $\,\,$ & $\,\,$3.8907  $\,\,$ \\
$\,\,$\color{red} 0.1568\color{black} $\,\,$ & $\,\,$\color{red} 0.4916\color{black} $\,\,$ & $\,\,$ 1 $\,\,$ & $\,\,$\color{red} 1.9127\color{black}  $\,\,$ \\
$\,\,$0.0820$\,\,$ & $\,\,$0.2570$\,\,$ & $\,\,$\color{red} 0.5228\color{black} $\,\,$ & $\,\,$ 1  $\,\,$ \\
\end{pmatrix},
\end{equation*}

\begin{equation*}
\mathbf{w}^{\prime} =
\begin{pmatrix}
0.640909\\
0.204375\\
0.102187\\
0.052529
\end{pmatrix} =
0.998286\cdot
\begin{pmatrix}
0.642010\\
0.204726\\
\color{gr} 0.102363\color{black} \\
0.052619
\end{pmatrix},
\end{equation*}
\begin{equation*}
\left[ \frac{{w}^{\prime}_i}{{w}^{\prime}_j} \right] =
\begin{pmatrix}
$\,\,$ 1 $\,\,$ & $\,\,$3.1359$\,\,$ & $\,\,$\color{gr} 6.2719\color{black} $\,\,$ & $\,\,$12.2011$\,\,$ \\
$\,\,$0.3189$\,\,$ & $\,\,$ 1 $\,\,$ & $\,\,$\color{gr} \color{blue} 2\color{black} $\,\,$ & $\,\,$3.8907  $\,\,$ \\
$\,\,$\color{gr} 0.1594\color{black} $\,\,$ & $\,\,$\color{gr} \color{blue}  1/2\color{black} $\,\,$ & $\,\,$ 1 $\,\,$ & $\,\,$\color{gr} 1.9454\color{black}  $\,\,$ \\
$\,\,$0.0820$\,\,$ & $\,\,$0.2570$\,\,$ & $\,\,$\color{gr} 0.5140\color{black} $\,\,$ & $\,\,$ 1  $\,\,$ \\
\end{pmatrix},
\end{equation*}
\end{example}
\newpage
\begin{example}
\begin{equation*}
\mathbf{A} =
\begin{pmatrix}
$\,\,$ 1 $\,\,$ & $\,\,$6$\,\,$ & $\,\,$6$\,\,$ & $\,\,$9 $\,\,$ \\
$\,\,$ 1/6$\,\,$ & $\,\,$ 1 $\,\,$ & $\,\,$2$\,\,$ & $\,\,$7 $\,\,$ \\
$\,\,$ 1/6$\,\,$ & $\,\,$ 1/2$\,\,$ & $\,\,$ 1 $\,\,$ & $\,\,$2 $\,\,$ \\
$\,\,$ 1/9$\,\,$ & $\,\,$ 1/7$\,\,$ & $\,\,$ 1/2$\,\,$ & $\,\,$ 1  $\,\,$ \\
\end{pmatrix},
\qquad
\lambda_{\max} =
4.2109,
\qquad
CR = 0.0795
\end{equation*}

\begin{equation*}
\mathbf{w}^{cos} =
\begin{pmatrix}
0.634843\\
0.214942\\
\color{red} 0.099064\color{black} \\
0.051151
\end{pmatrix}\end{equation*}
\begin{equation*}
\left[ \frac{{w}^{cos}_i}{{w}^{cos}_j} \right] =
\begin{pmatrix}
$\,\,$ 1 $\,\,$ & $\,\,$2.9536$\,\,$ & $\,\,$\color{red} 6.4084\color{black} $\,\,$ & $\,\,$12.4112$\,\,$ \\
$\,\,$0.3386$\,\,$ & $\,\,$ 1 $\,\,$ & $\,\,$\color{red} 2.1697\color{black} $\,\,$ & $\,\,$4.2021  $\,\,$ \\
$\,\,$\color{red} 0.1560\color{black} $\,\,$ & $\,\,$\color{red} 0.4609\color{black} $\,\,$ & $\,\,$ 1 $\,\,$ & $\,\,$\color{red} 1.9367\color{black}  $\,\,$ \\
$\,\,$0.0806$\,\,$ & $\,\,$0.2380$\,\,$ & $\,\,$\color{red} 0.5163\color{black} $\,\,$ & $\,\,$ 1  $\,\,$ \\
\end{pmatrix},
\end{equation*}

\begin{equation*}
\mathbf{w}^{\prime} =
\begin{pmatrix}
0.632794\\
0.214249\\
0.101971\\
0.050986
\end{pmatrix} =
0.996773\cdot
\begin{pmatrix}
0.634843\\
0.214942\\
\color{gr} 0.102301\color{black} \\
0.051151
\end{pmatrix},
\end{equation*}
\begin{equation*}
\left[ \frac{{w}^{\prime}_i}{{w}^{\prime}_j} \right] =
\begin{pmatrix}
$\,\,$ 1 $\,\,$ & $\,\,$2.9536$\,\,$ & $\,\,$\color{gr} 6.2056\color{black} $\,\,$ & $\,\,$12.4112$\,\,$ \\
$\,\,$0.3386$\,\,$ & $\,\,$ 1 $\,\,$ & $\,\,$\color{gr} 2.1011\color{black} $\,\,$ & $\,\,$4.2021  $\,\,$ \\
$\,\,$\color{gr} 0.1611\color{black} $\,\,$ & $\,\,$\color{gr} 0.4759\color{black} $\,\,$ & $\,\,$ 1 $\,\,$ & $\,\,$\color{gr} \color{blue} 2\color{black}  $\,\,$ \\
$\,\,$0.0806$\,\,$ & $\,\,$0.2380$\,\,$ & $\,\,$\color{gr} \color{blue}  1/2\color{black} $\,\,$ & $\,\,$ 1  $\,\,$ \\
\end{pmatrix},
\end{equation*}
\end{example}
\newpage
\begin{example}
\begin{equation*}
\mathbf{A} =
\begin{pmatrix}
$\,\,$ 1 $\,\,$ & $\,\,$6$\,\,$ & $\,\,$6$\,\,$ & $\,\,$9 $\,\,$ \\
$\,\,$ 1/6$\,\,$ & $\,\,$ 1 $\,\,$ & $\,\,$2$\,\,$ & $\,\,$8 $\,\,$ \\
$\,\,$ 1/6$\,\,$ & $\,\,$ 1/2$\,\,$ & $\,\,$ 1 $\,\,$ & $\,\,$2 $\,\,$ \\
$\,\,$ 1/9$\,\,$ & $\,\,$ 1/8$\,\,$ & $\,\,$ 1/2$\,\,$ & $\,\,$ 1  $\,\,$ \\
\end{pmatrix},
\qquad
\lambda_{\max} =
4.2512,
\qquad
CR = 0.0947
\end{equation*}

\begin{equation*}
\mathbf{w}^{cos} =
\begin{pmatrix}
0.628515\\
0.223944\\
\color{red} 0.097619\color{black} \\
0.049921
\end{pmatrix}\end{equation*}
\begin{equation*}
\left[ \frac{{w}^{cos}_i}{{w}^{cos}_j} \right] =
\begin{pmatrix}
$\,\,$ 1 $\,\,$ & $\,\,$2.8066$\,\,$ & $\,\,$\color{red} 6.4384\color{black} $\,\,$ & $\,\,$12.5901$\,\,$ \\
$\,\,$0.3563$\,\,$ & $\,\,$ 1 $\,\,$ & $\,\,$\color{red} 2.2941\color{black} $\,\,$ & $\,\,$4.4859  $\,\,$ \\
$\,\,$\color{red} 0.1553\color{black} $\,\,$ & $\,\,$\color{red} 0.4359\color{black} $\,\,$ & $\,\,$ 1 $\,\,$ & $\,\,$\color{red} 1.9555\color{black}  $\,\,$ \\
$\,\,$0.0794$\,\,$ & $\,\,$0.2229$\,\,$ & $\,\,$\color{red} 0.5114\color{black} $\,\,$ & $\,\,$ 1  $\,\,$ \\
\end{pmatrix},
\end{equation*}

\begin{equation*}
\mathbf{w}^{\prime} =
\begin{pmatrix}
0.627120\\
0.223447\\
0.099621\\
0.049811
\end{pmatrix} =
0.997781\cdot
\begin{pmatrix}
0.628515\\
0.223944\\
\color{gr} 0.099843\color{black} \\
0.049921
\end{pmatrix},
\end{equation*}
\begin{equation*}
\left[ \frac{{w}^{\prime}_i}{{w}^{\prime}_j} \right] =
\begin{pmatrix}
$\,\,$ 1 $\,\,$ & $\,\,$2.8066$\,\,$ & $\,\,$\color{gr} 6.2950\color{black} $\,\,$ & $\,\,$12.5901$\,\,$ \\
$\,\,$0.3563$\,\,$ & $\,\,$ 1 $\,\,$ & $\,\,$\color{gr} 2.2430\color{black} $\,\,$ & $\,\,$4.4859  $\,\,$ \\
$\,\,$\color{gr} 0.1589\color{black} $\,\,$ & $\,\,$\color{gr} 0.4458\color{black} $\,\,$ & $\,\,$ 1 $\,\,$ & $\,\,$\color{gr} \color{blue} 2\color{black}  $\,\,$ \\
$\,\,$0.0794$\,\,$ & $\,\,$0.2229$\,\,$ & $\,\,$\color{gr} \color{blue}  1/2\color{black} $\,\,$ & $\,\,$ 1  $\,\,$ \\
\end{pmatrix},
\end{equation*}
\end{example}
\newpage
\begin{example}
\begin{equation*}
\mathbf{A} =
\begin{pmatrix}
$\,\,$ 1 $\,\,$ & $\,\,$6$\,\,$ & $\,\,$6$\,\,$ & $\,\,$9 $\,\,$ \\
$\,\,$ 1/6$\,\,$ & $\,\,$ 1 $\,\,$ & $\,\,$5$\,\,$ & $\,\,$3 $\,\,$ \\
$\,\,$ 1/6$\,\,$ & $\,\,$ 1/5$\,\,$ & $\,\,$ 1 $\,\,$ & $\,\,$1 $\,\,$ \\
$\,\,$ 1/9$\,\,$ & $\,\,$ 1/3$\,\,$ & $\,\,$ 1 $\,\,$ & $\,\,$ 1  $\,\,$ \\
\end{pmatrix},
\qquad
\lambda_{\max} =
4.2277,
\qquad
CR = 0.0859
\end{equation*}

\begin{equation*}
\mathbf{w}^{cos} =
\begin{pmatrix}
0.636226\\
0.221256\\
0.074041\\
\color{red} 0.068477\color{black}
\end{pmatrix}\end{equation*}
\begin{equation*}
\left[ \frac{{w}^{cos}_i}{{w}^{cos}_j} \right] =
\begin{pmatrix}
$\,\,$ 1 $\,\,$ & $\,\,$2.8755$\,\,$ & $\,\,$8.5929$\,\,$ & $\,\,$\color{red} 9.2911\color{black} $\,\,$ \\
$\,\,$0.3478$\,\,$ & $\,\,$ 1 $\,\,$ & $\,\,$2.9883$\,\,$ & $\,\,$\color{red} 3.2311\color{black}   $\,\,$ \\
$\,\,$0.1164$\,\,$ & $\,\,$0.3346$\,\,$ & $\,\,$ 1 $\,\,$ & $\,\,$\color{red} 1.0812\color{black}  $\,\,$ \\
$\,\,$\color{red} 0.1076\color{black} $\,\,$ & $\,\,$\color{red} 0.3095\color{black} $\,\,$ & $\,\,$\color{red} 0.9249\color{black} $\,\,$ & $\,\,$ 1  $\,\,$ \\
\end{pmatrix},
\end{equation*}

\begin{equation*}
\mathbf{w}^{\prime} =
\begin{pmatrix}
0.634820\\
0.220768\\
0.073877\\
0.070536
\end{pmatrix} =
0.997790\cdot
\begin{pmatrix}
0.636226\\
0.221256\\
0.074041\\
\color{gr} 0.070692\color{black}
\end{pmatrix},
\end{equation*}
\begin{equation*}
\left[ \frac{{w}^{\prime}_i}{{w}^{\prime}_j} \right] =
\begin{pmatrix}
$\,\,$ 1 $\,\,$ & $\,\,$2.8755$\,\,$ & $\,\,$8.5929$\,\,$ & $\,\,$\color{gr} \color{blue} 9\color{black} $\,\,$ \\
$\,\,$0.3478$\,\,$ & $\,\,$ 1 $\,\,$ & $\,\,$2.9883$\,\,$ & $\,\,$\color{gr} 3.1299\color{black}   $\,\,$ \\
$\,\,$0.1164$\,\,$ & $\,\,$0.3346$\,\,$ & $\,\,$ 1 $\,\,$ & $\,\,$\color{gr} 1.0474\color{black}  $\,\,$ \\
$\,\,$\color{gr} \color{blue}  1/9\color{black} $\,\,$ & $\,\,$\color{gr} 0.3195\color{black} $\,\,$ & $\,\,$\color{gr} 0.9548\color{black} $\,\,$ & $\,\,$ 1  $\,\,$ \\
\end{pmatrix},
\end{equation*}
\end{example}
\newpage
\begin{example}
\begin{equation*}
\mathbf{A} =
\begin{pmatrix}
$\,\,$ 1 $\,\,$ & $\,\,$6$\,\,$ & $\,\,$7$\,\,$ & $\,\,$9 $\,\,$ \\
$\,\,$ 1/6$\,\,$ & $\,\,$ 1 $\,\,$ & $\,\,$2$\,\,$ & $\,\,$8 $\,\,$ \\
$\,\,$ 1/7$\,\,$ & $\,\,$ 1/2$\,\,$ & $\,\,$ 1 $\,\,$ & $\,\,$2 $\,\,$ \\
$\,\,$ 1/9$\,\,$ & $\,\,$ 1/8$\,\,$ & $\,\,$ 1/2$\,\,$ & $\,\,$ 1  $\,\,$ \\
\end{pmatrix},
\qquad
\lambda_{\max} =
4.2463,
\qquad
CR = 0.0929
\end{equation*}

\begin{equation*}
\mathbf{w}^{cos} =
\begin{pmatrix}
0.639891\\
0.219863\\
\color{red} 0.091381\color{black} \\
0.048865
\end{pmatrix}\end{equation*}
\begin{equation*}
\left[ \frac{{w}^{cos}_i}{{w}^{cos}_j} \right] =
\begin{pmatrix}
$\,\,$ 1 $\,\,$ & $\,\,$2.9104$\,\,$ & $\,\,$\color{red} 7.0025\color{black} $\,\,$ & $\,\,$13.0950$\,\,$ \\
$\,\,$0.3436$\,\,$ & $\,\,$ 1 $\,\,$ & $\,\,$\color{red} 2.4060\color{black} $\,\,$ & $\,\,$4.4993  $\,\,$ \\
$\,\,$\color{red} 0.1428\color{black} $\,\,$ & $\,\,$\color{red} 0.4156\color{black} $\,\,$ & $\,\,$ 1 $\,\,$ & $\,\,$\color{red} 1.8700\color{black}  $\,\,$ \\
$\,\,$0.0764$\,\,$ & $\,\,$0.2223$\,\,$ & $\,\,$\color{red} 0.5347\color{black} $\,\,$ & $\,\,$ 1  $\,\,$ \\
\end{pmatrix},
\end{equation*}

\begin{equation*}
\mathbf{w}^{\prime} =
\begin{pmatrix}
0.639870\\
0.219856\\
0.091410\\
0.048864
\end{pmatrix} =
0.999968\cdot
\begin{pmatrix}
0.639891\\
0.219863\\
\color{gr} 0.091413\color{black} \\
0.048865
\end{pmatrix},
\end{equation*}
\begin{equation*}
\left[ \frac{{w}^{\prime}_i}{{w}^{\prime}_j} \right] =
\begin{pmatrix}
$\,\,$ 1 $\,\,$ & $\,\,$2.9104$\,\,$ & $\,\,$\color{gr} \color{blue} 7\color{black} $\,\,$ & $\,\,$13.0950$\,\,$ \\
$\,\,$0.3436$\,\,$ & $\,\,$ 1 $\,\,$ & $\,\,$\color{gr} 2.4052\color{black} $\,\,$ & $\,\,$4.4993  $\,\,$ \\
$\,\,$\color{gr} \color{blue}  1/7\color{black} $\,\,$ & $\,\,$\color{gr} 0.4158\color{black} $\,\,$ & $\,\,$ 1 $\,\,$ & $\,\,$\color{gr} 1.8707\color{black}  $\,\,$ \\
$\,\,$0.0764$\,\,$ & $\,\,$0.2223$\,\,$ & $\,\,$\color{gr} 0.5346\color{black} $\,\,$ & $\,\,$ 1  $\,\,$ \\
\end{pmatrix},
\end{equation*}
\end{example}
\newpage
\begin{example}
\begin{equation*}
\mathbf{A} =
\begin{pmatrix}
$\,\,$ 1 $\,\,$ & $\,\,$6$\,\,$ & $\,\,$7$\,\,$ & $\,\,$9 $\,\,$ \\
$\,\,$ 1/6$\,\,$ & $\,\,$ 1 $\,\,$ & $\,\,$5$\,\,$ & $\,\,$3 $\,\,$ \\
$\,\,$ 1/7$\,\,$ & $\,\,$ 1/5$\,\,$ & $\,\,$ 1 $\,\,$ & $\,\,$1 $\,\,$ \\
$\,\,$ 1/9$\,\,$ & $\,\,$ 1/3$\,\,$ & $\,\,$ 1 $\,\,$ & $\,\,$ 1  $\,\,$ \\
\end{pmatrix},
\qquad
\lambda_{\max} =
4.1889,
\qquad
CR = 0.0712
\end{equation*}

\begin{equation*}
\mathbf{w}^{cos} =
\begin{pmatrix}
0.650717\\
0.213382\\
0.068738\\
\color{red} 0.067162\color{black}
\end{pmatrix}\end{equation*}
\begin{equation*}
\left[ \frac{{w}^{cos}_i}{{w}^{cos}_j} \right] =
\begin{pmatrix}
$\,\,$ 1 $\,\,$ & $\,\,$3.0495$\,\,$ & $\,\,$9.4666$\,\,$ & $\,\,$\color{red} 9.6887\color{black} $\,\,$ \\
$\,\,$0.3279$\,\,$ & $\,\,$ 1 $\,\,$ & $\,\,$3.1043$\,\,$ & $\,\,$\color{red} 3.1771\color{black}   $\,\,$ \\
$\,\,$0.1056$\,\,$ & $\,\,$0.3221$\,\,$ & $\,\,$ 1 $\,\,$ & $\,\,$\color{red} 1.0235\color{black}  $\,\,$ \\
$\,\,$\color{red} 0.1032\color{black} $\,\,$ & $\,\,$\color{red} 0.3148\color{black} $\,\,$ & $\,\,$\color{red} 0.9771\color{black} $\,\,$ & $\,\,$ 1  $\,\,$ \\
\end{pmatrix},
\end{equation*}

\begin{equation*}
\mathbf{w}^{\prime} =
\begin{pmatrix}
0.649694\\
0.213047\\
0.068630\\
0.068630
\end{pmatrix} =
0.998427\cdot
\begin{pmatrix}
0.650717\\
0.213382\\
0.068738\\
\color{gr} 0.068738\color{black}
\end{pmatrix},
\end{equation*}
\begin{equation*}
\left[ \frac{{w}^{\prime}_i}{{w}^{\prime}_j} \right] =
\begin{pmatrix}
$\,\,$ 1 $\,\,$ & $\,\,$3.0495$\,\,$ & $\,\,$9.4666$\,\,$ & $\,\,$\color{gr} 9.4666\color{black} $\,\,$ \\
$\,\,$0.3279$\,\,$ & $\,\,$ 1 $\,\,$ & $\,\,$3.1043$\,\,$ & $\,\,$\color{gr} 3.1043\color{black}   $\,\,$ \\
$\,\,$0.1056$\,\,$ & $\,\,$0.3221$\,\,$ & $\,\,$ 1 $\,\,$ & $\,\,$\color{gr} \color{blue} 1\color{black}  $\,\,$ \\
$\,\,$\color{gr} 0.1056\color{black} $\,\,$ & $\,\,$\color{gr} 0.3221\color{black} $\,\,$ & $\,\,$\color{gr} \color{blue} 1\color{black} $\,\,$ & $\,\,$ 1  $\,\,$ \\
\end{pmatrix},
\end{equation*}
\end{example}
\newpage
\begin{example}
\begin{equation*}
\mathbf{A} =
\begin{pmatrix}
$\,\,$ 1 $\,\,$ & $\,\,$6$\,\,$ & $\,\,$7$\,\,$ & $\,\,$9 $\,\,$ \\
$\,\,$ 1/6$\,\,$ & $\,\,$ 1 $\,\,$ & $\,\,$6$\,\,$ & $\,\,$3 $\,\,$ \\
$\,\,$ 1/7$\,\,$ & $\,\,$ 1/6$\,\,$ & $\,\,$ 1 $\,\,$ & $\,\,$1 $\,\,$ \\
$\,\,$ 1/9$\,\,$ & $\,\,$ 1/3$\,\,$ & $\,\,$ 1 $\,\,$ & $\,\,$ 1  $\,\,$ \\
\end{pmatrix},
\qquad
\lambda_{\max} =
4.2416,
\qquad
CR = 0.0911
\end{equation*}

\begin{equation*}
\mathbf{w}^{cos} =
\begin{pmatrix}
0.641848\\
0.225781\\
0.066492\\
\color{red} 0.065879\color{black}
\end{pmatrix}\end{equation*}
\begin{equation*}
\left[ \frac{{w}^{cos}_i}{{w}^{cos}_j} \right] =
\begin{pmatrix}
$\,\,$ 1 $\,\,$ & $\,\,$2.8428$\,\,$ & $\,\,$9.6530$\,\,$ & $\,\,$\color{red} 9.7428\color{black} $\,\,$ \\
$\,\,$0.3518$\,\,$ & $\,\,$ 1 $\,\,$ & $\,\,$3.3956$\,\,$ & $\,\,$\color{red} 3.4272\color{black}   $\,\,$ \\
$\,\,$0.1036$\,\,$ & $\,\,$0.2945$\,\,$ & $\,\,$ 1 $\,\,$ & $\,\,$\color{red} 1.0093\color{black}  $\,\,$ \\
$\,\,$\color{red} 0.1026\color{black} $\,\,$ & $\,\,$\color{red} 0.2918\color{black} $\,\,$ & $\,\,$\color{red} 0.9908\color{black} $\,\,$ & $\,\,$ 1  $\,\,$ \\
\end{pmatrix},
\end{equation*}

\begin{equation*}
\mathbf{w}^{\prime} =
\begin{pmatrix}
0.641455\\
0.225642\\
0.066451\\
0.066451
\end{pmatrix} =
0.999388\cdot
\begin{pmatrix}
0.641848\\
0.225781\\
0.066492\\
\color{gr} 0.066492\color{black}
\end{pmatrix},
\end{equation*}
\begin{equation*}
\left[ \frac{{w}^{\prime}_i}{{w}^{\prime}_j} \right] =
\begin{pmatrix}
$\,\,$ 1 $\,\,$ & $\,\,$2.8428$\,\,$ & $\,\,$9.6530$\,\,$ & $\,\,$\color{gr} 9.6530\color{black} $\,\,$ \\
$\,\,$0.3518$\,\,$ & $\,\,$ 1 $\,\,$ & $\,\,$3.3956$\,\,$ & $\,\,$\color{gr} 3.3956\color{black}   $\,\,$ \\
$\,\,$0.1036$\,\,$ & $\,\,$0.2945$\,\,$ & $\,\,$ 1 $\,\,$ & $\,\,$\color{gr} \color{blue} 1\color{black}  $\,\,$ \\
$\,\,$\color{gr} 0.1036\color{black} $\,\,$ & $\,\,$\color{gr} 0.2945\color{black} $\,\,$ & $\,\,$\color{gr} \color{blue} 1\color{black} $\,\,$ & $\,\,$ 1  $\,\,$ \\
\end{pmatrix},
\end{equation*}
\end{example}
\newpage
\begin{example}
\begin{equation*}
\mathbf{A} =
\begin{pmatrix}
$\,\,$ 1 $\,\,$ & $\,\,$7$\,\,$ & $\,\,$3$\,\,$ & $\,\,$4 $\,\,$ \\
$\,\,$ 1/7$\,\,$ & $\,\,$ 1 $\,\,$ & $\,\,$1$\,\,$ & $\,\,$3 $\,\,$ \\
$\,\,$ 1/3$\,\,$ & $\,\,$ 1 $\,\,$ & $\,\,$ 1 $\,\,$ & $\,\,$2 $\,\,$ \\
$\,\,$ 1/4$\,\,$ & $\,\,$ 1/3$\,\,$ & $\,\,$ 1/2$\,\,$ & $\,\,$ 1  $\,\,$ \\
\end{pmatrix},
\qquad
\lambda_{\max} =
4.2478,
\qquad
CR = 0.0935
\end{equation*}

\begin{equation*}
\mathbf{w}^{cos} =
\begin{pmatrix}
0.554108\\
0.175802\\
\color{red} 0.174502\color{black} \\
0.095588
\end{pmatrix}\end{equation*}
\begin{equation*}
\left[ \frac{{w}^{cos}_i}{{w}^{cos}_j} \right] =
\begin{pmatrix}
$\,\,$ 1 $\,\,$ & $\,\,$3.1519$\,\,$ & $\,\,$\color{red} 3.1754\color{black} $\,\,$ & $\,\,$5.7968$\,\,$ \\
$\,\,$0.3173$\,\,$ & $\,\,$ 1 $\,\,$ & $\,\,$\color{red} 1.0075\color{black} $\,\,$ & $\,\,$1.8392  $\,\,$ \\
$\,\,$\color{red} 0.3149\color{black} $\,\,$ & $\,\,$\color{red} 0.9926\color{black} $\,\,$ & $\,\,$ 1 $\,\,$ & $\,\,$\color{red} 1.8256\color{black}  $\,\,$ \\
$\,\,$0.1725$\,\,$ & $\,\,$0.5437$\,\,$ & $\,\,$\color{red} 0.5478\color{black} $\,\,$ & $\,\,$ 1  $\,\,$ \\
\end{pmatrix},
\end{equation*}

\begin{equation*}
\mathbf{w}^{\prime} =
\begin{pmatrix}
0.553388\\
0.175574\\
0.175574\\
0.095464
\end{pmatrix} =
0.998701\cdot
\begin{pmatrix}
0.554108\\
0.175802\\
\color{gr} 0.175802\color{black} \\
0.095588
\end{pmatrix},
\end{equation*}
\begin{equation*}
\left[ \frac{{w}^{\prime}_i}{{w}^{\prime}_j} \right] =
\begin{pmatrix}
$\,\,$ 1 $\,\,$ & $\,\,$3.1519$\,\,$ & $\,\,$\color{gr} 3.1519\color{black} $\,\,$ & $\,\,$5.7968$\,\,$ \\
$\,\,$0.3173$\,\,$ & $\,\,$ 1 $\,\,$ & $\,\,$\color{gr} \color{blue} 1\color{black} $\,\,$ & $\,\,$1.8392  $\,\,$ \\
$\,\,$\color{gr} 0.3173\color{black} $\,\,$ & $\,\,$\color{gr} \color{blue} 1\color{black} $\,\,$ & $\,\,$ 1 $\,\,$ & $\,\,$\color{gr} 1.8392\color{black}  $\,\,$ \\
$\,\,$0.1725$\,\,$ & $\,\,$0.5437$\,\,$ & $\,\,$\color{gr} 0.5437\color{black} $\,\,$ & $\,\,$ 1  $\,\,$ \\
\end{pmatrix},
\end{equation*}
\end{example}
\newpage
\begin{example}
\begin{equation*}
\mathbf{A} =
\begin{pmatrix}
$\,\,$ 1 $\,\,$ & $\,\,$7$\,\,$ & $\,\,$3$\,\,$ & $\,\,$7 $\,\,$ \\
$\,\,$ 1/7$\,\,$ & $\,\,$ 1 $\,\,$ & $\,\,$1$\,\,$ & $\,\,$5 $\,\,$ \\
$\,\,$ 1/3$\,\,$ & $\,\,$ 1 $\,\,$ & $\,\,$ 1 $\,\,$ & $\,\,$3 $\,\,$ \\
$\,\,$ 1/7$\,\,$ & $\,\,$ 1/5$\,\,$ & $\,\,$ 1/3$\,\,$ & $\,\,$ 1  $\,\,$ \\
\end{pmatrix},
\qquad
\lambda_{\max} =
4.2365,
\qquad
CR = 0.0892
\end{equation*}

\begin{equation*}
\mathbf{w}^{cos} =
\begin{pmatrix}
0.582106\\
0.182127\\
\color{red} 0.175492\color{black} \\
0.060275
\end{pmatrix}\end{equation*}
\begin{equation*}
\left[ \frac{{w}^{cos}_i}{{w}^{cos}_j} \right] =
\begin{pmatrix}
$\,\,$ 1 $\,\,$ & $\,\,$3.1962$\,\,$ & $\,\,$\color{red} 3.3170\color{black} $\,\,$ & $\,\,$9.6576$\,\,$ \\
$\,\,$0.3129$\,\,$ & $\,\,$ 1 $\,\,$ & $\,\,$\color{red} 1.0378\color{black} $\,\,$ & $\,\,$3.0216  $\,\,$ \\
$\,\,$\color{red} 0.3015\color{black} $\,\,$ & $\,\,$\color{red} 0.9636\color{black} $\,\,$ & $\,\,$ 1 $\,\,$ & $\,\,$\color{red} 2.9115\color{black}  $\,\,$ \\
$\,\,$0.1035$\,\,$ & $\,\,$0.3309$\,\,$ & $\,\,$\color{red} 0.3435\color{black} $\,\,$ & $\,\,$ 1  $\,\,$ \\
\end{pmatrix},
\end{equation*}

\begin{equation*}
\mathbf{w}^{\prime} =
\begin{pmatrix}
0.579019\\
0.181161\\
0.179865\\
0.059955
\end{pmatrix} =
0.994697\cdot
\begin{pmatrix}
0.582106\\
0.182127\\
\color{gr} 0.180824\color{black} \\
0.060275
\end{pmatrix},
\end{equation*}
\begin{equation*}
\left[ \frac{{w}^{\prime}_i}{{w}^{\prime}_j} \right] =
\begin{pmatrix}
$\,\,$ 1 $\,\,$ & $\,\,$3.1962$\,\,$ & $\,\,$\color{gr} 3.2192\color{black} $\,\,$ & $\,\,$9.6576$\,\,$ \\
$\,\,$0.3129$\,\,$ & $\,\,$ 1 $\,\,$ & $\,\,$\color{gr} 1.0072\color{black} $\,\,$ & $\,\,$3.0216  $\,\,$ \\
$\,\,$\color{gr} 0.3106\color{black} $\,\,$ & $\,\,$\color{gr} 0.9928\color{black} $\,\,$ & $\,\,$ 1 $\,\,$ & $\,\,$\color{gr} \color{blue} 3\color{black}  $\,\,$ \\
$\,\,$0.1035$\,\,$ & $\,\,$0.3309$\,\,$ & $\,\,$\color{gr} \color{blue}  1/3\color{black} $\,\,$ & $\,\,$ 1  $\,\,$ \\
\end{pmatrix},
\end{equation*}
\end{example}
\newpage
\begin{example}
\begin{equation*}
\mathbf{A} =
\begin{pmatrix}
$\,\,$ 1 $\,\,$ & $\,\,$7$\,\,$ & $\,\,$3$\,\,$ & $\,\,$8 $\,\,$ \\
$\,\,$ 1/7$\,\,$ & $\,\,$ 1 $\,\,$ & $\,\,$1$\,\,$ & $\,\,$6 $\,\,$ \\
$\,\,$ 1/3$\,\,$ & $\,\,$ 1 $\,\,$ & $\,\,$ 1 $\,\,$ & $\,\,$4 $\,\,$ \\
$\,\,$ 1/8$\,\,$ & $\,\,$ 1/6$\,\,$ & $\,\,$ 1/4$\,\,$ & $\,\,$ 1  $\,\,$ \\
\end{pmatrix},
\qquad
\lambda_{\max} =
4.2478,
\qquad
CR = 0.0935
\end{equation*}

\begin{equation*}
\mathbf{w}^{cos} =
\begin{pmatrix}
0.581554\\
0.184637\\
\color{red} 0.183483\color{black} \\
0.050326
\end{pmatrix}\end{equation*}
\begin{equation*}
\left[ \frac{{w}^{cos}_i}{{w}^{cos}_j} \right] =
\begin{pmatrix}
$\,\,$ 1 $\,\,$ & $\,\,$3.1497$\,\,$ & $\,\,$\color{red} 3.1695\color{black} $\,\,$ & $\,\,$11.5557$\,\,$ \\
$\,\,$0.3175$\,\,$ & $\,\,$ 1 $\,\,$ & $\,\,$\color{red} 1.0063\color{black} $\,\,$ & $\,\,$3.6688  $\,\,$ \\
$\,\,$\color{red} 0.3155\color{black} $\,\,$ & $\,\,$\color{red} 0.9937\color{black} $\,\,$ & $\,\,$ 1 $\,\,$ & $\,\,$\color{red} 3.6459\color{black}  $\,\,$ \\
$\,\,$0.0865$\,\,$ & $\,\,$0.2726$\,\,$ & $\,\,$\color{red} 0.2743\color{black} $\,\,$ & $\,\,$ 1  $\,\,$ \\
\end{pmatrix},
\end{equation*}

\begin{equation*}
\mathbf{w}^{\prime} =
\begin{pmatrix}
0.580883\\
0.184424\\
0.184424\\
0.050268
\end{pmatrix} =
0.998847\cdot
\begin{pmatrix}
0.581554\\
0.184637\\
\color{gr} 0.184637\color{black} \\
0.050326
\end{pmatrix},
\end{equation*}
\begin{equation*}
\left[ \frac{{w}^{\prime}_i}{{w}^{\prime}_j} \right] =
\begin{pmatrix}
$\,\,$ 1 $\,\,$ & $\,\,$3.1497$\,\,$ & $\,\,$\color{gr} 3.1497\color{black} $\,\,$ & $\,\,$11.5557$\,\,$ \\
$\,\,$0.3175$\,\,$ & $\,\,$ 1 $\,\,$ & $\,\,$\color{gr} \color{blue} 1\color{black} $\,\,$ & $\,\,$3.6688  $\,\,$ \\
$\,\,$\color{gr} 0.3175\color{black} $\,\,$ & $\,\,$\color{gr} \color{blue} 1\color{black} $\,\,$ & $\,\,$ 1 $\,\,$ & $\,\,$\color{gr} 3.6688\color{black}  $\,\,$ \\
$\,\,$0.0865$\,\,$ & $\,\,$0.2726$\,\,$ & $\,\,$\color{gr} 0.2726\color{black} $\,\,$ & $\,\,$ 1  $\,\,$ \\
\end{pmatrix},
\end{equation*}
\end{example}
\newpage
\begin{example}
\begin{equation*}
\mathbf{A} =
\begin{pmatrix}
$\,\,$ 1 $\,\,$ & $\,\,$7$\,\,$ & $\,\,$3$\,\,$ & $\,\,$8 $\,\,$ \\
$\,\,$ 1/7$\,\,$ & $\,\,$ 1 $\,\,$ & $\,\,$2$\,\,$ & $\,\,$2 $\,\,$ \\
$\,\,$ 1/3$\,\,$ & $\,\,$ 1/2$\,\,$ & $\,\,$ 1 $\,\,$ & $\,\,$2 $\,\,$ \\
$\,\,$ 1/8$\,\,$ & $\,\,$ 1/2$\,\,$ & $\,\,$ 1/2$\,\,$ & $\,\,$ 1  $\,\,$ \\
\end{pmatrix},
\qquad
\lambda_{\max} =
4.2109,
\qquad
CR = 0.0795
\end{equation*}

\begin{equation*}
\mathbf{w}^{cos} =
\begin{pmatrix}
0.607798\\
0.172859\\
0.146642\\
\color{red} 0.072701\color{black}
\end{pmatrix}\end{equation*}
\begin{equation*}
\left[ \frac{{w}^{cos}_i}{{w}^{cos}_j} \right] =
\begin{pmatrix}
$\,\,$ 1 $\,\,$ & $\,\,$3.5162$\,\,$ & $\,\,$4.1448$\,\,$ & $\,\,$\color{red} 8.3602\color{black} $\,\,$ \\
$\,\,$0.2844$\,\,$ & $\,\,$ 1 $\,\,$ & $\,\,$1.1788$\,\,$ & $\,\,$\color{red} 2.3777\color{black}   $\,\,$ \\
$\,\,$0.2413$\,\,$ & $\,\,$0.8483$\,\,$ & $\,\,$ 1 $\,\,$ & $\,\,$\color{red} 2.0171\color{black}  $\,\,$ \\
$\,\,$\color{red} 0.1196\color{black} $\,\,$ & $\,\,$\color{red} 0.4206\color{black} $\,\,$ & $\,\,$\color{red} 0.4958\color{black} $\,\,$ & $\,\,$ 1  $\,\,$ \\
\end{pmatrix},
\end{equation*}

\begin{equation*}
\mathbf{w}^{\prime} =
\begin{pmatrix}
0.607421\\
0.172752\\
0.146552\\
0.073276
\end{pmatrix} =
0.999380\cdot
\begin{pmatrix}
0.607798\\
0.172859\\
0.146642\\
\color{gr} 0.073321\color{black}
\end{pmatrix},
\end{equation*}
\begin{equation*}
\left[ \frac{{w}^{\prime}_i}{{w}^{\prime}_j} \right] =
\begin{pmatrix}
$\,\,$ 1 $\,\,$ & $\,\,$3.5162$\,\,$ & $\,\,$4.1448$\,\,$ & $\,\,$\color{gr} 8.2895\color{black} $\,\,$ \\
$\,\,$0.2844$\,\,$ & $\,\,$ 1 $\,\,$ & $\,\,$1.1788$\,\,$ & $\,\,$\color{gr} 2.3576\color{black}   $\,\,$ \\
$\,\,$0.2413$\,\,$ & $\,\,$0.8483$\,\,$ & $\,\,$ 1 $\,\,$ & $\,\,$\color{gr} \color{blue} 2\color{black}  $\,\,$ \\
$\,\,$\color{gr} 0.1206\color{black} $\,\,$ & $\,\,$\color{gr} 0.4242\color{black} $\,\,$ & $\,\,$\color{gr} \color{blue}  1/2\color{black} $\,\,$ & $\,\,$ 1  $\,\,$ \\
\end{pmatrix},
\end{equation*}
\end{example}
\newpage
\begin{example}
\begin{equation*}
\mathbf{A} =
\begin{pmatrix}
$\,\,$ 1 $\,\,$ & $\,\,$7$\,\,$ & $\,\,$3$\,\,$ & $\,\,$9 $\,\,$ \\
$\,\,$ 1/7$\,\,$ & $\,\,$ 1 $\,\,$ & $\,\,$1$\,\,$ & $\,\,$7 $\,\,$ \\
$\,\,$ 1/3$\,\,$ & $\,\,$ 1 $\,\,$ & $\,\,$ 1 $\,\,$ & $\,\,$4 $\,\,$ \\
$\,\,$ 1/9$\,\,$ & $\,\,$ 1/7$\,\,$ & $\,\,$ 1/4$\,\,$ & $\,\,$ 1  $\,\,$ \\
\end{pmatrix},
\qquad
\lambda_{\max} =
4.2606,
\qquad
CR = 0.0983
\end{equation*}

\begin{equation*}
\mathbf{w}^{cos} =
\begin{pmatrix}
0.585925\\
0.189572\\
\color{red} 0.178157\color{black} \\
0.046346
\end{pmatrix}\end{equation*}
\begin{equation*}
\left[ \frac{{w}^{cos}_i}{{w}^{cos}_j} \right] =
\begin{pmatrix}
$\,\,$ 1 $\,\,$ & $\,\,$3.0908$\,\,$ & $\,\,$\color{red} 3.2888\color{black} $\,\,$ & $\,\,$12.6425$\,\,$ \\
$\,\,$0.3235$\,\,$ & $\,\,$ 1 $\,\,$ & $\,\,$\color{red} 1.0641\color{black} $\,\,$ & $\,\,$4.0904  $\,\,$ \\
$\,\,$\color{red} 0.3041\color{black} $\,\,$ & $\,\,$\color{red} 0.9398\color{black} $\,\,$ & $\,\,$ 1 $\,\,$ & $\,\,$\color{red} 3.8441\color{black}  $\,\,$ \\
$\,\,$0.0791$\,\,$ & $\,\,$0.2445$\,\,$ & $\,\,$\color{red} 0.2601\color{black} $\,\,$ & $\,\,$ 1  $\,\,$ \\
\end{pmatrix},
\end{equation*}

\begin{equation*}
\mathbf{w}^{\prime} =
\begin{pmatrix}
0.581723\\
0.188212\\
0.184052\\
0.046013
\end{pmatrix} =
0.992827\cdot
\begin{pmatrix}
0.585925\\
0.189572\\
\color{gr} 0.185382\color{black} \\
0.046346
\end{pmatrix},
\end{equation*}
\begin{equation*}
\left[ \frac{{w}^{\prime}_i}{{w}^{\prime}_j} \right] =
\begin{pmatrix}
$\,\,$ 1 $\,\,$ & $\,\,$3.0908$\,\,$ & $\,\,$\color{gr} 3.1606\color{black} $\,\,$ & $\,\,$12.6425$\,\,$ \\
$\,\,$0.3235$\,\,$ & $\,\,$ 1 $\,\,$ & $\,\,$\color{gr} 1.0226\color{black} $\,\,$ & $\,\,$4.0904  $\,\,$ \\
$\,\,$\color{gr} 0.3164\color{black} $\,\,$ & $\,\,$\color{gr} 0.9779\color{black} $\,\,$ & $\,\,$ 1 $\,\,$ & $\,\,$\color{gr} \color{blue} 4\color{black}  $\,\,$ \\
$\,\,$0.0791$\,\,$ & $\,\,$0.2445$\,\,$ & $\,\,$\color{gr} \color{blue}  1/4\color{black} $\,\,$ & $\,\,$ 1  $\,\,$ \\
\end{pmatrix},
\end{equation*}
\end{example}
\newpage
\begin{example}
\begin{equation*}
\mathbf{A} =
\begin{pmatrix}
$\,\,$ 1 $\,\,$ & $\,\,$7$\,\,$ & $\,\,$4$\,\,$ & $\,\,$6 $\,\,$ \\
$\,\,$ 1/7$\,\,$ & $\,\,$ 1 $\,\,$ & $\,\,$1$\,\,$ & $\,\,$3 $\,\,$ \\
$\,\,$ 1/4$\,\,$ & $\,\,$ 1 $\,\,$ & $\,\,$ 1 $\,\,$ & $\,\,$2 $\,\,$ \\
$\,\,$ 1/6$\,\,$ & $\,\,$ 1/3$\,\,$ & $\,\,$ 1/2$\,\,$ & $\,\,$ 1  $\,\,$ \\
\end{pmatrix},
\qquad
\lambda_{\max} =
4.1365,
\qquad
CR = 0.0515
\end{equation*}

\begin{equation*}
\mathbf{w}^{cos} =
\begin{pmatrix}
0.618409\\
0.155529\\
\color{red} 0.148907\color{black} \\
0.077155
\end{pmatrix}\end{equation*}
\begin{equation*}
\left[ \frac{{w}^{cos}_i}{{w}^{cos}_j} \right] =
\begin{pmatrix}
$\,\,$ 1 $\,\,$ & $\,\,$3.9762$\,\,$ & $\,\,$\color{red} 4.1530\color{black} $\,\,$ & $\,\,$8.0151$\,\,$ \\
$\,\,$0.2515$\,\,$ & $\,\,$ 1 $\,\,$ & $\,\,$\color{red} 1.0445\color{black} $\,\,$ & $\,\,$2.0158  $\,\,$ \\
$\,\,$\color{red} 0.2408\color{black} $\,\,$ & $\,\,$\color{red} 0.9574\color{black} $\,\,$ & $\,\,$ 1 $\,\,$ & $\,\,$\color{red} 1.9300\color{black}  $\,\,$ \\
$\,\,$0.1248$\,\,$ & $\,\,$0.4961$\,\,$ & $\,\,$\color{red} 0.5181\color{black} $\,\,$ & $\,\,$ 1  $\,\,$ \\
\end{pmatrix},
\end{equation*}

\begin{equation*}
\mathbf{w}^{\prime} =
\begin{pmatrix}
0.615085\\
0.154693\\
0.153482\\
0.076741
\end{pmatrix} =
0.994625\cdot
\begin{pmatrix}
0.618409\\
0.155529\\
\color{gr} 0.154311\color{black} \\
0.077155
\end{pmatrix},
\end{equation*}
\begin{equation*}
\left[ \frac{{w}^{\prime}_i}{{w}^{\prime}_j} \right] =
\begin{pmatrix}
$\,\,$ 1 $\,\,$ & $\,\,$3.9762$\,\,$ & $\,\,$\color{gr} 4.0076\color{black} $\,\,$ & $\,\,$8.0151$\,\,$ \\
$\,\,$0.2515$\,\,$ & $\,\,$ 1 $\,\,$ & $\,\,$\color{gr} 1.0079\color{black} $\,\,$ & $\,\,$2.0158  $\,\,$ \\
$\,\,$\color{gr} 0.2495\color{black} $\,\,$ & $\,\,$\color{gr} 0.9922\color{black} $\,\,$ & $\,\,$ 1 $\,\,$ & $\,\,$\color{gr} \color{blue} 2\color{black}  $\,\,$ \\
$\,\,$0.1248$\,\,$ & $\,\,$0.4961$\,\,$ & $\,\,$\color{gr} \color{blue}  1/2\color{black} $\,\,$ & $\,\,$ 1  $\,\,$ \\
\end{pmatrix},
\end{equation*}
\end{example}
\newpage
\begin{example}
\begin{equation*}
\mathbf{A} =
\begin{pmatrix}
$\,\,$ 1 $\,\,$ & $\,\,$7$\,\,$ & $\,\,$4$\,\,$ & $\,\,$6 $\,\,$ \\
$\,\,$ 1/7$\,\,$ & $\,\,$ 1 $\,\,$ & $\,\,$1$\,\,$ & $\,\,$4 $\,\,$ \\
$\,\,$ 1/4$\,\,$ & $\,\,$ 1 $\,\,$ & $\,\,$ 1 $\,\,$ & $\,\,$2 $\,\,$ \\
$\,\,$ 1/6$\,\,$ & $\,\,$ 1/4$\,\,$ & $\,\,$ 1/2$\,\,$ & $\,\,$ 1  $\,\,$ \\
\end{pmatrix},
\qquad
\lambda_{\max} =
4.2109,
\qquad
CR = 0.0795
\end{equation*}

\begin{equation*}
\mathbf{w}^{cos} =
\begin{pmatrix}
0.608127\\
0.172723\\
\color{red} 0.145596\color{black} \\
0.073554
\end{pmatrix}\end{equation*}
\begin{equation*}
\left[ \frac{{w}^{cos}_i}{{w}^{cos}_j} \right] =
\begin{pmatrix}
$\,\,$ 1 $\,\,$ & $\,\,$3.5208$\,\,$ & $\,\,$\color{red} 4.1768\color{black} $\,\,$ & $\,\,$8.2678$\,\,$ \\
$\,\,$0.2840$\,\,$ & $\,\,$ 1 $\,\,$ & $\,\,$\color{red} 1.1863\color{black} $\,\,$ & $\,\,$2.3482  $\,\,$ \\
$\,\,$\color{red} 0.2394\color{black} $\,\,$ & $\,\,$\color{red} 0.8429\color{black} $\,\,$ & $\,\,$ 1 $\,\,$ & $\,\,$\color{red} 1.9794\color{black}  $\,\,$ \\
$\,\,$0.1210$\,\,$ & $\,\,$0.4259$\,\,$ & $\,\,$\color{red} 0.5052\color{black} $\,\,$ & $\,\,$ 1  $\,\,$ \\
\end{pmatrix},
\end{equation*}

\begin{equation*}
\mathbf{w}^{\prime} =
\begin{pmatrix}
0.607209\\
0.172462\\
0.146886\\
0.073443
\end{pmatrix} =
0.998490\cdot
\begin{pmatrix}
0.608127\\
0.172723\\
\color{gr} 0.147108\color{black} \\
0.073554
\end{pmatrix},
\end{equation*}
\begin{equation*}
\left[ \frac{{w}^{\prime}_i}{{w}^{\prime}_j} \right] =
\begin{pmatrix}
$\,\,$ 1 $\,\,$ & $\,\,$3.5208$\,\,$ & $\,\,$\color{gr} 4.1339\color{black} $\,\,$ & $\,\,$8.2678$\,\,$ \\
$\,\,$0.2840$\,\,$ & $\,\,$ 1 $\,\,$ & $\,\,$\color{gr} 1.1741\color{black} $\,\,$ & $\,\,$2.3482  $\,\,$ \\
$\,\,$\color{gr} 0.2419\color{black} $\,\,$ & $\,\,$\color{gr} 0.8517\color{black} $\,\,$ & $\,\,$ 1 $\,\,$ & $\,\,$\color{gr} \color{blue} 2\color{black}  $\,\,$ \\
$\,\,$0.1210$\,\,$ & $\,\,$0.4259$\,\,$ & $\,\,$\color{gr} \color{blue}  1/2\color{black} $\,\,$ & $\,\,$ 1  $\,\,$ \\
\end{pmatrix},
\end{equation*}
\end{example}
\newpage
\begin{example}
\begin{equation*}
\mathbf{A} =
\begin{pmatrix}
$\,\,$ 1 $\,\,$ & $\,\,$7$\,\,$ & $\,\,$4$\,\,$ & $\,\,$6 $\,\,$ \\
$\,\,$ 1/7$\,\,$ & $\,\,$ 1 $\,\,$ & $\,\,$3$\,\,$ & $\,\,$2 $\,\,$ \\
$\,\,$ 1/4$\,\,$ & $\,\,$ 1/3$\,\,$ & $\,\,$ 1 $\,\,$ & $\,\,$1 $\,\,$ \\
$\,\,$ 1/6$\,\,$ & $\,\,$ 1/2$\,\,$ & $\,\,$ 1 $\,\,$ & $\,\,$ 1  $\,\,$ \\
\end{pmatrix},
\qquad
\lambda_{\max} =
4.2478,
\qquad
CR = 0.0935
\end{equation*}

\begin{equation*}
\mathbf{w}^{cos} =
\begin{pmatrix}
0.605187\\
0.193750\\
0.105164\\
\color{red} 0.095899\color{black}
\end{pmatrix}\end{equation*}
\begin{equation*}
\left[ \frac{{w}^{cos}_i}{{w}^{cos}_j} \right] =
\begin{pmatrix}
$\,\,$ 1 $\,\,$ & $\,\,$3.1235$\,\,$ & $\,\,$5.7547$\,\,$ & $\,\,$\color{red} 6.3107\color{black} $\,\,$ \\
$\,\,$0.3201$\,\,$ & $\,\,$ 1 $\,\,$ & $\,\,$1.8424$\,\,$ & $\,\,$\color{red} 2.0204\color{black}   $\,\,$ \\
$\,\,$0.1738$\,\,$ & $\,\,$0.5428$\,\,$ & $\,\,$ 1 $\,\,$ & $\,\,$\color{red} 1.0966\color{black}  $\,\,$ \\
$\,\,$\color{red} 0.1585\color{black} $\,\,$ & $\,\,$\color{red} 0.4950\color{black} $\,\,$ & $\,\,$\color{red} 0.9119\color{black} $\,\,$ & $\,\,$ 1  $\,\,$ \\
\end{pmatrix},
\end{equation*}

\begin{equation*}
\mathbf{w}^{\prime} =
\begin{pmatrix}
0.604597\\
0.193561\\
0.105061\\
0.096781
\end{pmatrix} =
0.999025\cdot
\begin{pmatrix}
0.605187\\
0.193750\\
0.105164\\
\color{gr} 0.096875\color{black}
\end{pmatrix},
\end{equation*}
\begin{equation*}
\left[ \frac{{w}^{\prime}_i}{{w}^{\prime}_j} \right] =
\begin{pmatrix}
$\,\,$ 1 $\,\,$ & $\,\,$3.1235$\,\,$ & $\,\,$5.7547$\,\,$ & $\,\,$\color{gr} 6.2471\color{black} $\,\,$ \\
$\,\,$0.3201$\,\,$ & $\,\,$ 1 $\,\,$ & $\,\,$1.8424$\,\,$ & $\,\,$\color{gr} \color{blue} 2\color{black}   $\,\,$ \\
$\,\,$0.1738$\,\,$ & $\,\,$0.5428$\,\,$ & $\,\,$ 1 $\,\,$ & $\,\,$\color{gr} 1.0856\color{black}  $\,\,$ \\
$\,\,$\color{gr} 0.1601\color{black} $\,\,$ & $\,\,$\color{gr} \color{blue}  1/2\color{black} $\,\,$ & $\,\,$\color{gr} 0.9212\color{black} $\,\,$ & $\,\,$ 1  $\,\,$ \\
\end{pmatrix},
\end{equation*}
\end{example}
\newpage
\begin{example}
\begin{equation*}
\mathbf{A} =
\begin{pmatrix}
$\,\,$ 1 $\,\,$ & $\,\,$7$\,\,$ & $\,\,$4$\,\,$ & $\,\,$8 $\,\,$ \\
$\,\,$ 1/7$\,\,$ & $\,\,$ 1 $\,\,$ & $\,\,$1$\,\,$ & $\,\,$5 $\,\,$ \\
$\,\,$ 1/4$\,\,$ & $\,\,$ 1 $\,\,$ & $\,\,$ 1 $\,\,$ & $\,\,$3 $\,\,$ \\
$\,\,$ 1/8$\,\,$ & $\,\,$ 1/5$\,\,$ & $\,\,$ 1/3$\,\,$ & $\,\,$ 1  $\,\,$ \\
\end{pmatrix},
\qquad
\lambda_{\max} =
4.1888,
\qquad
CR = 0.0712
\end{equation*}

\begin{equation*}
\mathbf{w}^{cos} =
\begin{pmatrix}
0.619929\\
0.170878\\
\color{red} 0.154228\color{black} \\
0.054964
\end{pmatrix}\end{equation*}
\begin{equation*}
\left[ \frac{{w}^{cos}_i}{{w}^{cos}_j} \right] =
\begin{pmatrix}
$\,\,$ 1 $\,\,$ & $\,\,$3.6279$\,\,$ & $\,\,$\color{red} 4.0196\color{black} $\,\,$ & $\,\,$11.2788$\,\,$ \\
$\,\,$0.2756$\,\,$ & $\,\,$ 1 $\,\,$ & $\,\,$\color{red} 1.1080\color{black} $\,\,$ & $\,\,$3.1089  $\,\,$ \\
$\,\,$\color{red} 0.2488\color{black} $\,\,$ & $\,\,$\color{red} 0.9026\color{black} $\,\,$ & $\,\,$ 1 $\,\,$ & $\,\,$\color{red} 2.8060\color{black}  $\,\,$ \\
$\,\,$0.0887$\,\,$ & $\,\,$0.3217$\,\,$ & $\,\,$\color{red} 0.3564\color{black} $\,\,$ & $\,\,$ 1  $\,\,$ \\
\end{pmatrix},
\end{equation*}

\begin{equation*}
\mathbf{w}^{\prime} =
\begin{pmatrix}
0.619462\\
0.170750\\
0.154866\\
0.054923
\end{pmatrix} =
0.999246\cdot
\begin{pmatrix}
0.619929\\
0.170878\\
\color{gr} 0.154982\color{black} \\
0.054964
\end{pmatrix},
\end{equation*}
\begin{equation*}
\left[ \frac{{w}^{\prime}_i}{{w}^{\prime}_j} \right] =
\begin{pmatrix}
$\,\,$ 1 $\,\,$ & $\,\,$3.6279$\,\,$ & $\,\,$\color{gr} \color{blue} 4\color{black} $\,\,$ & $\,\,$11.2788$\,\,$ \\
$\,\,$0.2756$\,\,$ & $\,\,$ 1 $\,\,$ & $\,\,$\color{gr} 1.1026\color{black} $\,\,$ & $\,\,$3.1089  $\,\,$ \\
$\,\,$\color{gr} \color{blue}  1/4\color{black} $\,\,$ & $\,\,$\color{gr} 0.9070\color{black} $\,\,$ & $\,\,$ 1 $\,\,$ & $\,\,$\color{gr} 2.8197\color{black}  $\,\,$ \\
$\,\,$0.0887$\,\,$ & $\,\,$0.3217$\,\,$ & $\,\,$\color{gr} 0.3546\color{black} $\,\,$ & $\,\,$ 1  $\,\,$ \\
\end{pmatrix},
\end{equation*}
\end{example}
\newpage
\begin{example}
\begin{equation*}
\mathbf{A} =
\begin{pmatrix}
$\,\,$ 1 $\,\,$ & $\,\,$7$\,\,$ & $\,\,$4$\,\,$ & $\,\,$8 $\,\,$ \\
$\,\,$ 1/7$\,\,$ & $\,\,$ 1 $\,\,$ & $\,\,$1$\,\,$ & $\,\,$6 $\,\,$ \\
$\,\,$ 1/4$\,\,$ & $\,\,$ 1 $\,\,$ & $\,\,$ 1 $\,\,$ & $\,\,$3 $\,\,$ \\
$\,\,$ 1/8$\,\,$ & $\,\,$ 1/6$\,\,$ & $\,\,$ 1/3$\,\,$ & $\,\,$ 1  $\,\,$ \\
\end{pmatrix},
\qquad
\lambda_{\max} =
4.2421,
\qquad
CR = 0.0913
\end{equation*}

\begin{equation*}
\mathbf{w}^{cos} =
\begin{pmatrix}
0.612668\\
0.182509\\
\color{red} 0.151542\color{black} \\
0.053281
\end{pmatrix}\end{equation*}
\begin{equation*}
\left[ \frac{{w}^{cos}_i}{{w}^{cos}_j} \right] =
\begin{pmatrix}
$\,\,$ 1 $\,\,$ & $\,\,$3.3569$\,\,$ & $\,\,$\color{red} 4.0429\color{black} $\,\,$ & $\,\,$11.4989$\,\,$ \\
$\,\,$0.2979$\,\,$ & $\,\,$ 1 $\,\,$ & $\,\,$\color{red} 1.2043\color{black} $\,\,$ & $\,\,$3.4254  $\,\,$ \\
$\,\,$\color{red} 0.2473\color{black} $\,\,$ & $\,\,$\color{red} 0.8303\color{black} $\,\,$ & $\,\,$ 1 $\,\,$ & $\,\,$\color{red} 2.8442\color{black}  $\,\,$ \\
$\,\,$0.0870$\,\,$ & $\,\,$0.2919$\,\,$ & $\,\,$\color{red} 0.3516\color{black} $\,\,$ & $\,\,$ 1  $\,\,$ \\
\end{pmatrix},
\end{equation*}

\begin{equation*}
\mathbf{w}^{\prime} =
\begin{pmatrix}
0.611674\\
0.182213\\
0.152919\\
0.053194
\end{pmatrix} =
0.998378\cdot
\begin{pmatrix}
0.612668\\
0.182509\\
\color{gr} 0.153167\color{black} \\
0.053281
\end{pmatrix},
\end{equation*}
\begin{equation*}
\left[ \frac{{w}^{\prime}_i}{{w}^{\prime}_j} \right] =
\begin{pmatrix}
$\,\,$ 1 $\,\,$ & $\,\,$3.3569$\,\,$ & $\,\,$\color{gr} \color{blue} 4\color{black} $\,\,$ & $\,\,$11.4989$\,\,$ \\
$\,\,$0.2979$\,\,$ & $\,\,$ 1 $\,\,$ & $\,\,$\color{gr} 1.1916\color{black} $\,\,$ & $\,\,$3.4254  $\,\,$ \\
$\,\,$\color{gr} \color{blue}  1/4\color{black} $\,\,$ & $\,\,$\color{gr} 0.8392\color{black} $\,\,$ & $\,\,$ 1 $\,\,$ & $\,\,$\color{gr} 2.8747\color{black}  $\,\,$ \\
$\,\,$0.0870$\,\,$ & $\,\,$0.2919$\,\,$ & $\,\,$\color{gr} 0.3479\color{black} $\,\,$ & $\,\,$ 1  $\,\,$ \\
\end{pmatrix},
\end{equation*}
\end{example}
\newpage
\begin{example}
\begin{equation*}
\mathbf{A} =
\begin{pmatrix}
$\,\,$ 1 $\,\,$ & $\,\,$7$\,\,$ & $\,\,$4$\,\,$ & $\,\,$9 $\,\,$ \\
$\,\,$ 1/7$\,\,$ & $\,\,$ 1 $\,\,$ & $\,\,$1$\,\,$ & $\,\,$5 $\,\,$ \\
$\,\,$ 1/4$\,\,$ & $\,\,$ 1 $\,\,$ & $\,\,$ 1 $\,\,$ & $\,\,$3 $\,\,$ \\
$\,\,$ 1/9$\,\,$ & $\,\,$ 1/5$\,\,$ & $\,\,$ 1/3$\,\,$ & $\,\,$ 1  $\,\,$ \\
\end{pmatrix},
\qquad
\lambda_{\max} =
4.1610,
\qquad
CR = 0.0607
\end{equation*}

\begin{equation*}
\mathbf{w}^{cos} =
\begin{pmatrix}
0.630642\\
0.165813\\
\color{red} 0.151654\color{black} \\
0.051890
\end{pmatrix}\end{equation*}
\begin{equation*}
\left[ \frac{{w}^{cos}_i}{{w}^{cos}_j} \right] =
\begin{pmatrix}
$\,\,$ 1 $\,\,$ & $\,\,$3.8033$\,\,$ & $\,\,$\color{red} 4.1584\color{black} $\,\,$ & $\,\,$12.1534$\,\,$ \\
$\,\,$0.2629$\,\,$ & $\,\,$ 1 $\,\,$ & $\,\,$\color{red} 1.0934\color{black} $\,\,$ & $\,\,$3.1955  $\,\,$ \\
$\,\,$\color{red} 0.2405\color{black} $\,\,$ & $\,\,$\color{red} 0.9146\color{black} $\,\,$ & $\,\,$ 1 $\,\,$ & $\,\,$\color{red} 2.9226\color{black}  $\,\,$ \\
$\,\,$0.0823$\,\,$ & $\,\,$0.3129$\,\,$ & $\,\,$\color{red} 0.3422\color{black} $\,\,$ & $\,\,$ 1  $\,\,$ \\
\end{pmatrix},
\end{equation*}

\begin{equation*}
\mathbf{w}^{\prime} =
\begin{pmatrix}
0.628119\\
0.165150\\
0.155048\\
0.051683
\end{pmatrix} =
0.995999\cdot
\begin{pmatrix}
0.630642\\
0.165813\\
\color{gr} 0.155671\color{black} \\
0.051890
\end{pmatrix},
\end{equation*}
\begin{equation*}
\left[ \frac{{w}^{\prime}_i}{{w}^{\prime}_j} \right] =
\begin{pmatrix}
$\,\,$ 1 $\,\,$ & $\,\,$3.8033$\,\,$ & $\,\,$\color{gr} 4.0511\color{black} $\,\,$ & $\,\,$12.1534$\,\,$ \\
$\,\,$0.2629$\,\,$ & $\,\,$ 1 $\,\,$ & $\,\,$\color{gr} 1.0652\color{black} $\,\,$ & $\,\,$3.1955  $\,\,$ \\
$\,\,$\color{gr} 0.2468\color{black} $\,\,$ & $\,\,$\color{gr} 0.9388\color{black} $\,\,$ & $\,\,$ 1 $\,\,$ & $\,\,$\color{gr} \color{blue} 3\color{black}  $\,\,$ \\
$\,\,$0.0823$\,\,$ & $\,\,$0.3129$\,\,$ & $\,\,$\color{gr} \color{blue}  1/3\color{black} $\,\,$ & $\,\,$ 1  $\,\,$ \\
\end{pmatrix},
\end{equation*}
\end{example}
\newpage
\begin{example}
\begin{equation*}
\mathbf{A} =
\begin{pmatrix}
$\,\,$ 1 $\,\,$ & $\,\,$7$\,\,$ & $\,\,$4$\,\,$ & $\,\,$9 $\,\,$ \\
$\,\,$ 1/7$\,\,$ & $\,\,$ 1 $\,\,$ & $\,\,$1$\,\,$ & $\,\,$6 $\,\,$ \\
$\,\,$ 1/4$\,\,$ & $\,\,$ 1 $\,\,$ & $\,\,$ 1 $\,\,$ & $\,\,$3 $\,\,$ \\
$\,\,$ 1/9$\,\,$ & $\,\,$ 1/6$\,\,$ & $\,\,$ 1/3$\,\,$ & $\,\,$ 1  $\,\,$ \\
\end{pmatrix},
\qquad
\lambda_{\max} =
4.2109,
\qquad
CR = 0.0795
\end{equation*}

\begin{equation*}
\mathbf{w}^{cos} =
\begin{pmatrix}
0.623292\\
0.177095\\
\color{red} 0.149301\color{black} \\
0.050312
\end{pmatrix}\end{equation*}
\begin{equation*}
\left[ \frac{{w}^{cos}_i}{{w}^{cos}_j} \right] =
\begin{pmatrix}
$\,\,$ 1 $\,\,$ & $\,\,$3.5195$\,\,$ & $\,\,$\color{red} 4.1747\color{black} $\,\,$ & $\,\,$12.3886$\,\,$ \\
$\,\,$0.2841$\,\,$ & $\,\,$ 1 $\,\,$ & $\,\,$\color{red} 1.1862\color{black} $\,\,$ & $\,\,$3.5199  $\,\,$ \\
$\,\,$\color{red} 0.2395\color{black} $\,\,$ & $\,\,$\color{red} 0.8431\color{black} $\,\,$ & $\,\,$ 1 $\,\,$ & $\,\,$\color{red} 2.9675\color{black}  $\,\,$ \\
$\,\,$0.0807$\,\,$ & $\,\,$0.2841$\,\,$ & $\,\,$\color{red} 0.3370\color{black} $\,\,$ & $\,\,$ 1  $\,\,$ \\
\end{pmatrix},
\end{equation*}

\begin{equation*}
\mathbf{w}^{\prime} =
\begin{pmatrix}
0.622275\\
0.176806\\
0.150689\\
0.050230
\end{pmatrix} =
0.998369\cdot
\begin{pmatrix}
0.623292\\
0.177095\\
\color{gr} 0.150936\color{black} \\
0.050312
\end{pmatrix},
\end{equation*}
\begin{equation*}
\left[ \frac{{w}^{\prime}_i}{{w}^{\prime}_j} \right] =
\begin{pmatrix}
$\,\,$ 1 $\,\,$ & $\,\,$3.5195$\,\,$ & $\,\,$\color{gr} 4.1295\color{black} $\,\,$ & $\,\,$12.3886$\,\,$ \\
$\,\,$0.2841$\,\,$ & $\,\,$ 1 $\,\,$ & $\,\,$\color{gr} 1.1733\color{black} $\,\,$ & $\,\,$3.5199  $\,\,$ \\
$\,\,$\color{gr} 0.2422\color{black} $\,\,$ & $\,\,$\color{gr} 0.8523\color{black} $\,\,$ & $\,\,$ 1 $\,\,$ & $\,\,$\color{gr} \color{blue} 3\color{black}  $\,\,$ \\
$\,\,$0.0807$\,\,$ & $\,\,$0.2841$\,\,$ & $\,\,$\color{gr} \color{blue}  1/3\color{black} $\,\,$ & $\,\,$ 1  $\,\,$ \\
\end{pmatrix},
\end{equation*}
\end{example}
\newpage
\begin{example}
\begin{equation*}
\mathbf{A} =
\begin{pmatrix}
$\,\,$ 1 $\,\,$ & $\,\,$7$\,\,$ & $\,\,$5$\,\,$ & $\,\,$7 $\,\,$ \\
$\,\,$ 1/7$\,\,$ & $\,\,$ 1 $\,\,$ & $\,\,$3$\,\,$ & $\,\,$2 $\,\,$ \\
$\,\,$ 1/5$\,\,$ & $\,\,$ 1/3$\,\,$ & $\,\,$ 1 $\,\,$ & $\,\,$1 $\,\,$ \\
$\,\,$ 1/7$\,\,$ & $\,\,$ 1/2$\,\,$ & $\,\,$ 1 $\,\,$ & $\,\,$ 1  $\,\,$ \\
\end{pmatrix},
\qquad
\lambda_{\max} =
4.1820,
\qquad
CR = 0.0686
\end{equation*}

\begin{equation*}
\mathbf{w}^{cos} =
\begin{pmatrix}
0.639709\\
0.179954\\
0.092871\\
\color{red} 0.087466\color{black}
\end{pmatrix}\end{equation*}
\begin{equation*}
\left[ \frac{{w}^{cos}_i}{{w}^{cos}_j} \right] =
\begin{pmatrix}
$\,\,$ 1 $\,\,$ & $\,\,$3.5548$\,\,$ & $\,\,$6.8882$\,\,$ & $\,\,$\color{red} 7.3138\color{black} $\,\,$ \\
$\,\,$0.2813$\,\,$ & $\,\,$ 1 $\,\,$ & $\,\,$1.9377$\,\,$ & $\,\,$\color{red} 2.0574\color{black}   $\,\,$ \\
$\,\,$0.1452$\,\,$ & $\,\,$0.5161$\,\,$ & $\,\,$ 1 $\,\,$ & $\,\,$\color{red} 1.0618\color{black}  $\,\,$ \\
$\,\,$\color{red} 0.1367\color{black} $\,\,$ & $\,\,$\color{red} 0.4860\color{black} $\,\,$ & $\,\,$\color{red} 0.9418\color{black} $\,\,$ & $\,\,$ 1  $\,\,$ \\
\end{pmatrix},
\end{equation*}

\begin{equation*}
\mathbf{w}^{\prime} =
\begin{pmatrix}
0.638106\\
0.179504\\
0.092638\\
0.089752
\end{pmatrix} =
0.997495\cdot
\begin{pmatrix}
0.639709\\
0.179954\\
0.092871\\
\color{gr} 0.089977\color{black}
\end{pmatrix},
\end{equation*}
\begin{equation*}
\left[ \frac{{w}^{\prime}_i}{{w}^{\prime}_j} \right] =
\begin{pmatrix}
$\,\,$ 1 $\,\,$ & $\,\,$3.5548$\,\,$ & $\,\,$6.8882$\,\,$ & $\,\,$\color{gr} 7.1097\color{black} $\,\,$ \\
$\,\,$0.2813$\,\,$ & $\,\,$ 1 $\,\,$ & $\,\,$1.9377$\,\,$ & $\,\,$\color{gr} \color{blue} 2\color{black}   $\,\,$ \\
$\,\,$0.1452$\,\,$ & $\,\,$0.5161$\,\,$ & $\,\,$ 1 $\,\,$ & $\,\,$\color{gr} 1.0322\color{black}  $\,\,$ \\
$\,\,$\color{gr} 0.1407\color{black} $\,\,$ & $\,\,$\color{gr} \color{blue}  1/2\color{black} $\,\,$ & $\,\,$\color{gr} 0.9688\color{black} $\,\,$ & $\,\,$ 1  $\,\,$ \\
\end{pmatrix},
\end{equation*}
\end{example}
\newpage
\begin{example}
\begin{equation*}
\mathbf{A} =
\begin{pmatrix}
$\,\,$ 1 $\,\,$ & $\,\,$7$\,\,$ & $\,\,$5$\,\,$ & $\,\,$7 $\,\,$ \\
$\,\,$ 1/7$\,\,$ & $\,\,$ 1 $\,\,$ & $\,\,$4$\,\,$ & $\,\,$2 $\,\,$ \\
$\,\,$ 1/5$\,\,$ & $\,\,$ 1/4$\,\,$ & $\,\,$ 1 $\,\,$ & $\,\,$1 $\,\,$ \\
$\,\,$ 1/7$\,\,$ & $\,\,$ 1/2$\,\,$ & $\,\,$ 1 $\,\,$ & $\,\,$ 1  $\,\,$ \\
\end{pmatrix},
\qquad
\lambda_{\max} =
4.2648,
\qquad
CR = 0.0998
\end{equation*}

\begin{equation*}
\mathbf{w}^{cos} =
\begin{pmatrix}
0.627662\\
0.198865\\
0.088430\\
\color{red} 0.085043\color{black}
\end{pmatrix}\end{equation*}
\begin{equation*}
\left[ \frac{{w}^{cos}_i}{{w}^{cos}_j} \right] =
\begin{pmatrix}
$\,\,$ 1 $\,\,$ & $\,\,$3.1562$\,\,$ & $\,\,$7.0979$\,\,$ & $\,\,$\color{red} 7.3805\color{black} $\,\,$ \\
$\,\,$0.3168$\,\,$ & $\,\,$ 1 $\,\,$ & $\,\,$2.2488$\,\,$ & $\,\,$\color{red} 2.3384\color{black}   $\,\,$ \\
$\,\,$0.1409$\,\,$ & $\,\,$0.4447$\,\,$ & $\,\,$ 1 $\,\,$ & $\,\,$\color{red} 1.0398\color{black}  $\,\,$ \\
$\,\,$\color{red} 0.1355\color{black} $\,\,$ & $\,\,$\color{red} 0.4276\color{black} $\,\,$ & $\,\,$\color{red} 0.9617\color{black} $\,\,$ & $\,\,$ 1  $\,\,$ \\
\end{pmatrix},
\end{equation*}

\begin{equation*}
\mathbf{w}^{\prime} =
\begin{pmatrix}
0.625544\\
0.198193\\
0.088131\\
0.088131
\end{pmatrix} =
0.996625\cdot
\begin{pmatrix}
0.627662\\
0.198865\\
0.088430\\
\color{gr} 0.088430\color{black}
\end{pmatrix},
\end{equation*}
\begin{equation*}
\left[ \frac{{w}^{\prime}_i}{{w}^{\prime}_j} \right] =
\begin{pmatrix}
$\,\,$ 1 $\,\,$ & $\,\,$3.1562$\,\,$ & $\,\,$7.0979$\,\,$ & $\,\,$\color{gr} 7.0979\color{black} $\,\,$ \\
$\,\,$0.3168$\,\,$ & $\,\,$ 1 $\,\,$ & $\,\,$2.2488$\,\,$ & $\,\,$\color{gr} 2.2488\color{black}   $\,\,$ \\
$\,\,$0.1409$\,\,$ & $\,\,$0.4447$\,\,$ & $\,\,$ 1 $\,\,$ & $\,\,$\color{gr} \color{blue} 1\color{black}  $\,\,$ \\
$\,\,$\color{gr} 0.1409\color{black} $\,\,$ & $\,\,$\color{gr} 0.4447\color{black} $\,\,$ & $\,\,$\color{gr} \color{blue} 1\color{black} $\,\,$ & $\,\,$ 1  $\,\,$ \\
\end{pmatrix},
\end{equation*}
\end{example}
\newpage
\begin{example}
\begin{equation*}
\mathbf{A} =
\begin{pmatrix}
$\,\,$ 1 $\,\,$ & $\,\,$7$\,\,$ & $\,\,$5$\,\,$ & $\,\,$8 $\,\,$ \\
$\,\,$ 1/7$\,\,$ & $\,\,$ 1 $\,\,$ & $\,\,$1$\,\,$ & $\,\,$3 $\,\,$ \\
$\,\,$ 1/5$\,\,$ & $\,\,$ 1 $\,\,$ & $\,\,$ 1 $\,\,$ & $\,\,$2 $\,\,$ \\
$\,\,$ 1/8$\,\,$ & $\,\,$ 1/3$\,\,$ & $\,\,$ 1/2$\,\,$ & $\,\,$ 1  $\,\,$ \\
\end{pmatrix},
\qquad
\lambda_{\max} =
4.0799,
\qquad
CR = 0.0301
\end{equation*}

\begin{equation*}
\mathbf{w}^{cos} =
\begin{pmatrix}
0.662919\\
0.140852\\
\color{red} 0.130728\color{black} \\
0.065501
\end{pmatrix}\end{equation*}
\begin{equation*}
\left[ \frac{{w}^{cos}_i}{{w}^{cos}_j} \right] =
\begin{pmatrix}
$\,\,$ 1 $\,\,$ & $\,\,$4.7065$\,\,$ & $\,\,$\color{red} 5.0710\color{black} $\,\,$ & $\,\,$10.1207$\,\,$ \\
$\,\,$0.2125$\,\,$ & $\,\,$ 1 $\,\,$ & $\,\,$\color{red} 1.0774\color{black} $\,\,$ & $\,\,$2.1504  $\,\,$ \\
$\,\,$\color{red} 0.1972\color{black} $\,\,$ & $\,\,$\color{red} 0.9281\color{black} $\,\,$ & $\,\,$ 1 $\,\,$ & $\,\,$\color{red} 1.9958\color{black}  $\,\,$ \\
$\,\,$0.0988$\,\,$ & $\,\,$0.4650$\,\,$ & $\,\,$\color{red} 0.5010\color{black} $\,\,$ & $\,\,$ 1  $\,\,$ \\
\end{pmatrix},
\end{equation*}

\begin{equation*}
\mathbf{w}^{\prime} =
\begin{pmatrix}
0.662737\\
0.140814\\
0.130966\\
0.065483
\end{pmatrix} =
0.999726\cdot
\begin{pmatrix}
0.662919\\
0.140852\\
\color{gr} 0.131002\color{black} \\
0.065501
\end{pmatrix},
\end{equation*}
\begin{equation*}
\left[ \frac{{w}^{\prime}_i}{{w}^{\prime}_j} \right] =
\begin{pmatrix}
$\,\,$ 1 $\,\,$ & $\,\,$4.7065$\,\,$ & $\,\,$\color{gr} 5.0604\color{black} $\,\,$ & $\,\,$10.1207$\,\,$ \\
$\,\,$0.2125$\,\,$ & $\,\,$ 1 $\,\,$ & $\,\,$\color{gr} 1.0752\color{black} $\,\,$ & $\,\,$2.1504  $\,\,$ \\
$\,\,$\color{gr} 0.1976\color{black} $\,\,$ & $\,\,$\color{gr} 0.9301\color{black} $\,\,$ & $\,\,$ 1 $\,\,$ & $\,\,$\color{gr} \color{blue} 2\color{black}  $\,\,$ \\
$\,\,$0.0988$\,\,$ & $\,\,$0.4650$\,\,$ & $\,\,$\color{gr} \color{blue}  1/2\color{black} $\,\,$ & $\,\,$ 1  $\,\,$ \\
\end{pmatrix},
\end{equation*}
\end{example}
\newpage
\begin{example}
\begin{equation*}
\mathbf{A} =
\begin{pmatrix}
$\,\,$ 1 $\,\,$ & $\,\,$7$\,\,$ & $\,\,$6$\,\,$ & $\,\,$8 $\,\,$ \\
$\,\,$ 1/7$\,\,$ & $\,\,$ 1 $\,\,$ & $\,\,$2$\,\,$ & $\,\,$6 $\,\,$ \\
$\,\,$ 1/6$\,\,$ & $\,\,$ 1/2$\,\,$ & $\,\,$ 1 $\,\,$ & $\,\,$2 $\,\,$ \\
$\,\,$ 1/8$\,\,$ & $\,\,$ 1/6$\,\,$ & $\,\,$ 1/2$\,\,$ & $\,\,$ 1  $\,\,$ \\
\end{pmatrix},
\qquad
\lambda_{\max} =
4.2478,
\qquad
CR = 0.0935
\end{equation*}

\begin{equation*}
\mathbf{w}^{cos} =
\begin{pmatrix}
0.638411\\
0.204559\\
\color{red} 0.101366\color{black} \\
0.055664
\end{pmatrix}\end{equation*}
\begin{equation*}
\left[ \frac{{w}^{cos}_i}{{w}^{cos}_j} \right] =
\begin{pmatrix}
$\,\,$ 1 $\,\,$ & $\,\,$3.1209$\,\,$ & $\,\,$\color{red} 6.2981\color{black} $\,\,$ & $\,\,$11.4690$\,\,$ \\
$\,\,$0.3204$\,\,$ & $\,\,$ 1 $\,\,$ & $\,\,$\color{red} 2.0180\color{black} $\,\,$ & $\,\,$3.6749  $\,\,$ \\
$\,\,$\color{red} 0.1588\color{black} $\,\,$ & $\,\,$\color{red} 0.4955\color{black} $\,\,$ & $\,\,$ 1 $\,\,$ & $\,\,$\color{red} 1.8210\color{black}  $\,\,$ \\
$\,\,$0.0872$\,\,$ & $\,\,$0.2721$\,\,$ & $\,\,$\color{red} 0.5491\color{black} $\,\,$ & $\,\,$ 1  $\,\,$ \\
\end{pmatrix},
\end{equation*}

\begin{equation*}
\mathbf{w}^{\prime} =
\begin{pmatrix}
0.637828\\
0.204373\\
0.102186\\
0.055613
\end{pmatrix} =
0.999088\cdot
\begin{pmatrix}
0.638411\\
0.204559\\
\color{gr} 0.102280\color{black} \\
0.055664
\end{pmatrix},
\end{equation*}
\begin{equation*}
\left[ \frac{{w}^{\prime}_i}{{w}^{\prime}_j} \right] =
\begin{pmatrix}
$\,\,$ 1 $\,\,$ & $\,\,$3.1209$\,\,$ & $\,\,$\color{gr} 6.2418\color{black} $\,\,$ & $\,\,$11.4690$\,\,$ \\
$\,\,$0.3204$\,\,$ & $\,\,$ 1 $\,\,$ & $\,\,$\color{gr} \color{blue} 2\color{black} $\,\,$ & $\,\,$3.6749  $\,\,$ \\
$\,\,$\color{gr} 0.1602\color{black} $\,\,$ & $\,\,$\color{gr} \color{blue}  1/2\color{black} $\,\,$ & $\,\,$ 1 $\,\,$ & $\,\,$\color{gr} 1.8375\color{black}  $\,\,$ \\
$\,\,$0.0872$\,\,$ & $\,\,$0.2721$\,\,$ & $\,\,$\color{gr} 0.5442\color{black} $\,\,$ & $\,\,$ 1  $\,\,$ \\
\end{pmatrix},
\end{equation*}
\end{example}
\newpage
\begin{example}
\begin{equation*}
\mathbf{A} =
\begin{pmatrix}
$\,\,$ 1 $\,\,$ & $\,\,$7$\,\,$ & $\,\,$6$\,\,$ & $\,\,$8 $\,\,$ \\
$\,\,$ 1/7$\,\,$ & $\,\,$ 1 $\,\,$ & $\,\,$3$\,\,$ & $\,\,$2 $\,\,$ \\
$\,\,$ 1/6$\,\,$ & $\,\,$ 1/3$\,\,$ & $\,\,$ 1 $\,\,$ & $\,\,$1 $\,\,$ \\
$\,\,$ 1/8$\,\,$ & $\,\,$ 1/2$\,\,$ & $\,\,$ 1 $\,\,$ & $\,\,$ 1  $\,\,$ \\
\end{pmatrix},
\qquad
\lambda_{\max} =
4.1365,
\qquad
CR = 0.0515
\end{equation*}

\begin{equation*}
\mathbf{w}^{cos} =
\begin{pmatrix}
0.667443\\
0.168488\\
0.083525\\
\color{red} 0.080544\color{black}
\end{pmatrix}\end{equation*}
\begin{equation*}
\left[ \frac{{w}^{cos}_i}{{w}^{cos}_j} \right] =
\begin{pmatrix}
$\,\,$ 1 $\,\,$ & $\,\,$3.9614$\,\,$ & $\,\,$7.9910$\,\,$ & $\,\,$\color{red} 8.2867\color{black} $\,\,$ \\
$\,\,$0.2524$\,\,$ & $\,\,$ 1 $\,\,$ & $\,\,$2.0172$\,\,$ & $\,\,$\color{red} 2.0919\color{black}   $\,\,$ \\
$\,\,$0.1251$\,\,$ & $\,\,$0.4957$\,\,$ & $\,\,$ 1 $\,\,$ & $\,\,$\color{red} 1.0370\color{black}  $\,\,$ \\
$\,\,$\color{red} 0.1207\color{black} $\,\,$ & $\,\,$\color{red} 0.4780\color{black} $\,\,$ & $\,\,$\color{red} 0.9643\color{black} $\,\,$ & $\,\,$ 1  $\,\,$ \\
\end{pmatrix},
\end{equation*}

\begin{equation*}
\mathbf{w}^{\prime} =
\begin{pmatrix}
0.665522\\
0.168003\\
0.083284\\
0.083190
\end{pmatrix} =
0.997122\cdot
\begin{pmatrix}
0.667443\\
0.168488\\
0.083525\\
\color{gr} 0.083430\color{black}
\end{pmatrix},
\end{equation*}
\begin{equation*}
\left[ \frac{{w}^{\prime}_i}{{w}^{\prime}_j} \right] =
\begin{pmatrix}
$\,\,$ 1 $\,\,$ & $\,\,$3.9614$\,\,$ & $\,\,$7.9910$\,\,$ & $\,\,$\color{gr} \color{blue} 8\color{black} $\,\,$ \\
$\,\,$0.2524$\,\,$ & $\,\,$ 1 $\,\,$ & $\,\,$2.0172$\,\,$ & $\,\,$\color{gr} 2.0195\color{black}   $\,\,$ \\
$\,\,$0.1251$\,\,$ & $\,\,$0.4957$\,\,$ & $\,\,$ 1 $\,\,$ & $\,\,$\color{gr} 1.0011\color{black}  $\,\,$ \\
$\,\,$\color{gr} \color{blue}  1/8\color{black} $\,\,$ & $\,\,$\color{gr} 0.4952\color{black} $\,\,$ & $\,\,$\color{gr} 0.9989\color{black} $\,\,$ & $\,\,$ 1  $\,\,$ \\
\end{pmatrix},
\end{equation*}
\end{example}
\newpage
\begin{example}
\begin{equation*}
\mathbf{A} =
\begin{pmatrix}
$\,\,$ 1 $\,\,$ & $\,\,$7$\,\,$ & $\,\,$6$\,\,$ & $\,\,$8 $\,\,$ \\
$\,\,$ 1/7$\,\,$ & $\,\,$ 1 $\,\,$ & $\,\,$4$\,\,$ & $\,\,$2 $\,\,$ \\
$\,\,$ 1/6$\,\,$ & $\,\,$ 1/4$\,\,$ & $\,\,$ 1 $\,\,$ & $\,\,$1 $\,\,$ \\
$\,\,$ 1/8$\,\,$ & $\,\,$ 1/2$\,\,$ & $\,\,$ 1 $\,\,$ & $\,\,$ 1  $\,\,$ \\
\end{pmatrix},
\qquad
\lambda_{\max} =
4.2109,
\qquad
CR = 0.0795
\end{equation*}

\begin{equation*}
\mathbf{w}^{cos} =
\begin{pmatrix}
0.655182\\
0.186740\\
0.079492\\
\color{red} 0.078586\color{black}
\end{pmatrix}\end{equation*}
\begin{equation*}
\left[ \frac{{w}^{cos}_i}{{w}^{cos}_j} \right] =
\begin{pmatrix}
$\,\,$ 1 $\,\,$ & $\,\,$3.5085$\,\,$ & $\,\,$8.2422$\,\,$ & $\,\,$\color{red} 8.3371\color{black} $\,\,$ \\
$\,\,$0.2850$\,\,$ & $\,\,$ 1 $\,\,$ & $\,\,$2.3492$\,\,$ & $\,\,$\color{red} 2.3762\color{black}   $\,\,$ \\
$\,\,$0.1213$\,\,$ & $\,\,$0.4257$\,\,$ & $\,\,$ 1 $\,\,$ & $\,\,$\color{red} 1.0115\color{black}  $\,\,$ \\
$\,\,$\color{red} 0.1199\color{black} $\,\,$ & $\,\,$\color{red} 0.4208\color{black} $\,\,$ & $\,\,$\color{red} 0.9886\color{black} $\,\,$ & $\,\,$ 1  $\,\,$ \\
\end{pmatrix},
\end{equation*}

\begin{equation*}
\mathbf{w}^{\prime} =
\begin{pmatrix}
0.654590\\
0.186571\\
0.079420\\
0.079420
\end{pmatrix} =
0.999096\cdot
\begin{pmatrix}
0.655182\\
0.186740\\
0.079492\\
\color{gr} 0.079492\color{black}
\end{pmatrix},
\end{equation*}
\begin{equation*}
\left[ \frac{{w}^{\prime}_i}{{w}^{\prime}_j} \right] =
\begin{pmatrix}
$\,\,$ 1 $\,\,$ & $\,\,$3.5085$\,\,$ & $\,\,$8.2422$\,\,$ & $\,\,$\color{gr} 8.2422\color{black} $\,\,$ \\
$\,\,$0.2850$\,\,$ & $\,\,$ 1 $\,\,$ & $\,\,$2.3492$\,\,$ & $\,\,$\color{gr} 2.3492\color{black}   $\,\,$ \\
$\,\,$0.1213$\,\,$ & $\,\,$0.4257$\,\,$ & $\,\,$ 1 $\,\,$ & $\,\,$\color{gr} \color{blue} 1\color{black}  $\,\,$ \\
$\,\,$\color{gr} 0.1213\color{black} $\,\,$ & $\,\,$\color{gr} 0.4257\color{black} $\,\,$ & $\,\,$\color{gr} \color{blue} 1\color{black} $\,\,$ & $\,\,$ 1  $\,\,$ \\
\end{pmatrix},
\end{equation*}
\end{example}
\newpage
\begin{example}
\begin{equation*}
\mathbf{A} =
\begin{pmatrix}
$\,\,$ 1 $\,\,$ & $\,\,$7$\,\,$ & $\,\,$6$\,\,$ & $\,\,$9 $\,\,$ \\
$\,\,$ 1/7$\,\,$ & $\,\,$ 1 $\,\,$ & $\,\,$2$\,\,$ & $\,\,$7 $\,\,$ \\
$\,\,$ 1/6$\,\,$ & $\,\,$ 1/2$\,\,$ & $\,\,$ 1 $\,\,$ & $\,\,$2 $\,\,$ \\
$\,\,$ 1/9$\,\,$ & $\,\,$ 1/7$\,\,$ & $\,\,$ 1/2$\,\,$ & $\,\,$ 1  $\,\,$ \\
\end{pmatrix},
\qquad
\lambda_{\max} =
4.2606,
\qquad
CR = 0.0983
\end{equation*}

\begin{equation*}
\mathbf{w}^{cos} =
\begin{pmatrix}
0.641902\\
0.208935\\
\color{red} 0.098058\color{black} \\
0.051105
\end{pmatrix}\end{equation*}
\begin{equation*}
\left[ \frac{{w}^{cos}_i}{{w}^{cos}_j} \right] =
\begin{pmatrix}
$\,\,$ 1 $\,\,$ & $\,\,$3.0723$\,\,$ & $\,\,$\color{red} 6.5462\color{black} $\,\,$ & $\,\,$12.5604$\,\,$ \\
$\,\,$0.3255$\,\,$ & $\,\,$ 1 $\,\,$ & $\,\,$\color{red} 2.1307\color{black} $\,\,$ & $\,\,$4.0883  $\,\,$ \\
$\,\,$\color{red} 0.1528\color{black} $\,\,$ & $\,\,$\color{red} 0.4693\color{black} $\,\,$ & $\,\,$ 1 $\,\,$ & $\,\,$\color{red} 1.9187\color{black}  $\,\,$ \\
$\,\,$0.0796$\,\,$ & $\,\,$0.2446$\,\,$ & $\,\,$\color{red} 0.5212\color{black} $\,\,$ & $\,\,$ 1  $\,\,$ \\
\end{pmatrix},
\end{equation*}

\begin{equation*}
\mathbf{w}^{\prime} =
\begin{pmatrix}
0.639248\\
0.208071\\
0.101788\\
0.050894
\end{pmatrix} =
0.995864\cdot
\begin{pmatrix}
0.641902\\
0.208935\\
\color{gr} 0.102210\color{black} \\
0.051105
\end{pmatrix},
\end{equation*}
\begin{equation*}
\left[ \frac{{w}^{\prime}_i}{{w}^{\prime}_j} \right] =
\begin{pmatrix}
$\,\,$ 1 $\,\,$ & $\,\,$3.0723$\,\,$ & $\,\,$\color{gr} 6.2802\color{black} $\,\,$ & $\,\,$12.5604$\,\,$ \\
$\,\,$0.3255$\,\,$ & $\,\,$ 1 $\,\,$ & $\,\,$\color{gr} 2.0442\color{black} $\,\,$ & $\,\,$4.0883  $\,\,$ \\
$\,\,$\color{gr} 0.1592\color{black} $\,\,$ & $\,\,$\color{gr} 0.4892\color{black} $\,\,$ & $\,\,$ 1 $\,\,$ & $\,\,$\color{gr} \color{blue} 2\color{black}  $\,\,$ \\
$\,\,$0.0796$\,\,$ & $\,\,$0.2446$\,\,$ & $\,\,$\color{gr} \color{blue}  1/2\color{black} $\,\,$ & $\,\,$ 1  $\,\,$ \\
\end{pmatrix},
\end{equation*}
\end{example}
\newpage
\begin{example}
\begin{equation*}
\mathbf{A} =
\begin{pmatrix}
$\,\,$ 1 $\,\,$ & $\,\,$7$\,\,$ & $\,\,$7$\,\,$ & $\,\,$9 $\,\,$ \\
$\,\,$ 1/7$\,\,$ & $\,\,$ 1 $\,\,$ & $\,\,$2$\,\,$ & $\,\,$6 $\,\,$ \\
$\,\,$ 1/7$\,\,$ & $\,\,$ 1/2$\,\,$ & $\,\,$ 1 $\,\,$ & $\,\,$2 $\,\,$ \\
$\,\,$ 1/9$\,\,$ & $\,\,$ 1/6$\,\,$ & $\,\,$ 1/2$\,\,$ & $\,\,$ 1  $\,\,$ \\
\end{pmatrix},
\qquad
\lambda_{\max} =
4.2086,
\qquad
CR = 0.0786
\end{equation*}

\begin{equation*}
\mathbf{w}^{cos} =
\begin{pmatrix}
0.660954\\
0.194195\\
\color{red} 0.093385\color{black} \\
0.051466
\end{pmatrix}\end{equation*}
\begin{equation*}
\left[ \frac{{w}^{cos}_i}{{w}^{cos}_j} \right] =
\begin{pmatrix}
$\,\,$ 1 $\,\,$ & $\,\,$3.4036$\,\,$ & $\,\,$\color{red} 7.0777\color{black} $\,\,$ & $\,\,$12.8425$\,\,$ \\
$\,\,$0.2938$\,\,$ & $\,\,$ 1 $\,\,$ & $\,\,$\color{red} 2.0795\color{black} $\,\,$ & $\,\,$3.7733  $\,\,$ \\
$\,\,$\color{red} 0.1413\color{black} $\,\,$ & $\,\,$\color{red} 0.4809\color{black} $\,\,$ & $\,\,$ 1 $\,\,$ & $\,\,$\color{red} 1.8145\color{black}  $\,\,$ \\
$\,\,$0.0779$\,\,$ & $\,\,$0.2650$\,\,$ & $\,\,$\color{red} 0.5511\color{black} $\,\,$ & $\,\,$ 1  $\,\,$ \\
\end{pmatrix},
\end{equation*}

\begin{equation*}
\mathbf{w}^{\prime} =
\begin{pmatrix}
0.660269\\
0.193994\\
0.094324\\
0.051413
\end{pmatrix} =
0.998964\cdot
\begin{pmatrix}
0.660954\\
0.194195\\
\color{gr} 0.094422\color{black} \\
0.051466
\end{pmatrix},
\end{equation*}
\begin{equation*}
\left[ \frac{{w}^{\prime}_i}{{w}^{\prime}_j} \right] =
\begin{pmatrix}
$\,\,$ 1 $\,\,$ & $\,\,$3.4036$\,\,$ & $\,\,$\color{gr} \color{blue} 7\color{black} $\,\,$ & $\,\,$12.8425$\,\,$ \\
$\,\,$0.2938$\,\,$ & $\,\,$ 1 $\,\,$ & $\,\,$\color{gr} 2.0567\color{black} $\,\,$ & $\,\,$3.7733  $\,\,$ \\
$\,\,$\color{gr} \color{blue}  1/7\color{black} $\,\,$ & $\,\,$\color{gr} 0.4862\color{black} $\,\,$ & $\,\,$ 1 $\,\,$ & $\,\,$\color{gr} 1.8346\color{black}  $\,\,$ \\
$\,\,$0.0779$\,\,$ & $\,\,$0.2650$\,\,$ & $\,\,$\color{gr} 0.5451\color{black} $\,\,$ & $\,\,$ 1  $\,\,$ \\
\end{pmatrix},
\end{equation*}
\end{example}
\newpage
\begin{example}
\begin{equation*}
\mathbf{A} =
\begin{pmatrix}
$\,\,$ 1 $\,\,$ & $\,\,$7$\,\,$ & $\,\,$7$\,\,$ & $\,\,$9 $\,\,$ \\
$\,\,$ 1/7$\,\,$ & $\,\,$ 1 $\,\,$ & $\,\,$2$\,\,$ & $\,\,$7 $\,\,$ \\
$\,\,$ 1/7$\,\,$ & $\,\,$ 1/2$\,\,$ & $\,\,$ 1 $\,\,$ & $\,\,$2 $\,\,$ \\
$\,\,$ 1/9$\,\,$ & $\,\,$ 1/7$\,\,$ & $\,\,$ 1/2$\,\,$ & $\,\,$ 1  $\,\,$ \\
\end{pmatrix},
\qquad
\lambda_{\max} =
4.2526,
\qquad
CR = 0.0952
\end{equation*}

\begin{equation*}
\mathbf{w}^{cos} =
\begin{pmatrix}
0.653561\\
0.204620\\
\color{red} 0.091762\color{black} \\
0.050057
\end{pmatrix}\end{equation*}
\begin{equation*}
\left[ \frac{{w}^{cos}_i}{{w}^{cos}_j} \right] =
\begin{pmatrix}
$\,\,$ 1 $\,\,$ & $\,\,$3.1940$\,\,$ & $\,\,$\color{red} 7.1223\color{black} $\,\,$ & $\,\,$13.0563$\,\,$ \\
$\,\,$0.3131$\,\,$ & $\,\,$ 1 $\,\,$ & $\,\,$\color{red} 2.2299\color{black} $\,\,$ & $\,\,$4.0877  $\,\,$ \\
$\,\,$\color{red} 0.1404\color{black} $\,\,$ & $\,\,$\color{red} 0.4485\color{black} $\,\,$ & $\,\,$ 1 $\,\,$ & $\,\,$\color{red} 1.8332\color{black}  $\,\,$ \\
$\,\,$0.0766$\,\,$ & $\,\,$0.2446$\,\,$ & $\,\,$\color{red} 0.5455\color{black} $\,\,$ & $\,\,$ 1  $\,\,$ \\
\end{pmatrix},
\end{equation*}

\begin{equation*}
\mathbf{w}^{\prime} =
\begin{pmatrix}
0.652514\\
0.204292\\
0.093216\\
0.049977
\end{pmatrix} =
0.998399\cdot
\begin{pmatrix}
0.653561\\
0.204620\\
\color{gr} 0.093366\color{black} \\
0.050057
\end{pmatrix},
\end{equation*}
\begin{equation*}
\left[ \frac{{w}^{\prime}_i}{{w}^{\prime}_j} \right] =
\begin{pmatrix}
$\,\,$ 1 $\,\,$ & $\,\,$3.1940$\,\,$ & $\,\,$\color{gr} \color{blue} 7\color{black} $\,\,$ & $\,\,$13.0563$\,\,$ \\
$\,\,$0.3131$\,\,$ & $\,\,$ 1 $\,\,$ & $\,\,$\color{gr} 2.1916\color{black} $\,\,$ & $\,\,$4.0877  $\,\,$ \\
$\,\,$\color{gr} \color{blue}  1/7\color{black} $\,\,$ & $\,\,$\color{gr} 0.4563\color{black} $\,\,$ & $\,\,$ 1 $\,\,$ & $\,\,$\color{gr} 1.8652\color{black}  $\,\,$ \\
$\,\,$0.0766$\,\,$ & $\,\,$0.2446$\,\,$ & $\,\,$\color{gr} 0.5361\color{black} $\,\,$ & $\,\,$ 1  $\,\,$ \\
\end{pmatrix},
\end{equation*}
\end{example}
\newpage
\begin{example}
\begin{equation*}
\mathbf{A} =
\begin{pmatrix}
$\,\,$ 1 $\,\,$ & $\,\,$7$\,\,$ & $\,\,$7$\,\,$ & $\,\,$9 $\,\,$ \\
$\,\,$ 1/7$\,\,$ & $\,\,$ 1 $\,\,$ & $\,\,$3$\,\,$ & $\,\,$2 $\,\,$ \\
$\,\,$ 1/7$\,\,$ & $\,\,$ 1/3$\,\,$ & $\,\,$ 1 $\,\,$ & $\,\,$1 $\,\,$ \\
$\,\,$ 1/9$\,\,$ & $\,\,$ 1/2$\,\,$ & $\,\,$ 1 $\,\,$ & $\,\,$ 1  $\,\,$ \\
\end{pmatrix},
\qquad
\lambda_{\max} =
4.1039,
\qquad
CR = 0.0392
\end{equation*}

\begin{equation*}
\mathbf{w}^{cos} =
\begin{pmatrix}
0.690137\\
0.158925\\
0.076140\\
\color{red} 0.074798\color{black}
\end{pmatrix}\end{equation*}
\begin{equation*}
\left[ \frac{{w}^{cos}_i}{{w}^{cos}_j} \right] =
\begin{pmatrix}
$\,\,$ 1 $\,\,$ & $\,\,$4.3425$\,\,$ & $\,\,$9.0640$\,\,$ & $\,\,$\color{red} 9.2267\color{black} $\,\,$ \\
$\,\,$0.2303$\,\,$ & $\,\,$ 1 $\,\,$ & $\,\,$2.0873$\,\,$ & $\,\,$\color{red} 2.1247\color{black}   $\,\,$ \\
$\,\,$0.1103$\,\,$ & $\,\,$0.4791$\,\,$ & $\,\,$ 1 $\,\,$ & $\,\,$\color{red} 1.0179\color{black}  $\,\,$ \\
$\,\,$\color{red} 0.1084\color{black} $\,\,$ & $\,\,$\color{red} 0.4707\color{black} $\,\,$ & $\,\,$\color{red} 0.9824\color{black} $\,\,$ & $\,\,$ 1  $\,\,$ \\
\end{pmatrix},
\end{equation*}

\begin{equation*}
\mathbf{w}^{\prime} =
\begin{pmatrix}
0.689212\\
0.158712\\
0.076038\\
0.076038
\end{pmatrix} =
0.998659\cdot
\begin{pmatrix}
0.690137\\
0.158925\\
0.076140\\
\color{gr} 0.076140\color{black}
\end{pmatrix},
\end{equation*}
\begin{equation*}
\left[ \frac{{w}^{\prime}_i}{{w}^{\prime}_j} \right] =
\begin{pmatrix}
$\,\,$ 1 $\,\,$ & $\,\,$4.3425$\,\,$ & $\,\,$9.0640$\,\,$ & $\,\,$\color{gr} 9.0640\color{black} $\,\,$ \\
$\,\,$0.2303$\,\,$ & $\,\,$ 1 $\,\,$ & $\,\,$2.0873$\,\,$ & $\,\,$\color{gr} 2.0873\color{black}   $\,\,$ \\
$\,\,$0.1103$\,\,$ & $\,\,$0.4791$\,\,$ & $\,\,$ 1 $\,\,$ & $\,\,$\color{gr} \color{blue} 1\color{black}  $\,\,$ \\
$\,\,$\color{gr} 0.1103\color{black} $\,\,$ & $\,\,$\color{gr} 0.4791\color{black} $\,\,$ & $\,\,$\color{gr} \color{blue} 1\color{black} $\,\,$ & $\,\,$ 1  $\,\,$ \\
\end{pmatrix},
\end{equation*}
\end{example}
\newpage
\begin{example}
\begin{equation*}
\mathbf{A} =
\begin{pmatrix}
$\,\,$ 1 $\,\,$ & $\,\,$7$\,\,$ & $\,\,$7$\,\,$ & $\,\,$9 $\,\,$ \\
$\,\,$ 1/7$\,\,$ & $\,\,$ 1 $\,\,$ & $\,\,$5$\,\,$ & $\,\,$3 $\,\,$ \\
$\,\,$ 1/7$\,\,$ & $\,\,$ 1/5$\,\,$ & $\,\,$ 1 $\,\,$ & $\,\,$1 $\,\,$ \\
$\,\,$ 1/9$\,\,$ & $\,\,$ 1/3$\,\,$ & $\,\,$ 1 $\,\,$ & $\,\,$ 1  $\,\,$ \\
\end{pmatrix},
\qquad
\lambda_{\max} =
4.2365,
\qquad
CR = 0.0892
\end{equation*}

\begin{equation*}
\mathbf{w}^{cos} =
\begin{pmatrix}
0.657746\\
0.207174\\
0.068615\\
\color{red} 0.066465\color{black}
\end{pmatrix}\end{equation*}
\begin{equation*}
\left[ \frac{{w}^{cos}_i}{{w}^{cos}_j} \right] =
\begin{pmatrix}
$\,\,$ 1 $\,\,$ & $\,\,$3.1748$\,\,$ & $\,\,$9.5861$\,\,$ & $\,\,$\color{red} 9.8961\color{black} $\,\,$ \\
$\,\,$0.3150$\,\,$ & $\,\,$ 1 $\,\,$ & $\,\,$3.0194$\,\,$ & $\,\,$\color{red} 3.1170\color{black}   $\,\,$ \\
$\,\,$0.1043$\,\,$ & $\,\,$0.3312$\,\,$ & $\,\,$ 1 $\,\,$ & $\,\,$\color{red} 1.0323\color{black}  $\,\,$ \\
$\,\,$\color{red} 0.1010\color{black} $\,\,$ & $\,\,$\color{red} 0.3208\color{black} $\,\,$ & $\,\,$\color{red} 0.9687\color{black} $\,\,$ & $\,\,$ 1  $\,\,$ \\
\end{pmatrix},
\end{equation*}

\begin{equation*}
\mathbf{w}^{\prime} =
\begin{pmatrix}
0.656335\\
0.206730\\
0.068468\\
0.068468
\end{pmatrix} =
0.997855\cdot
\begin{pmatrix}
0.657746\\
0.207174\\
0.068615\\
\color{gr} 0.068615\color{black}
\end{pmatrix},
\end{equation*}
\begin{equation*}
\left[ \frac{{w}^{\prime}_i}{{w}^{\prime}_j} \right] =
\begin{pmatrix}
$\,\,$ 1 $\,\,$ & $\,\,$3.1748$\,\,$ & $\,\,$9.5861$\,\,$ & $\,\,$\color{gr} 9.5861\color{black} $\,\,$ \\
$\,\,$0.3150$\,\,$ & $\,\,$ 1 $\,\,$ & $\,\,$3.0194$\,\,$ & $\,\,$\color{gr} 3.0194\color{black}   $\,\,$ \\
$\,\,$0.1043$\,\,$ & $\,\,$0.3312$\,\,$ & $\,\,$ 1 $\,\,$ & $\,\,$\color{gr} \color{blue} 1\color{black}  $\,\,$ \\
$\,\,$\color{gr} 0.1043\color{black} $\,\,$ & $\,\,$\color{gr} 0.3312\color{black} $\,\,$ & $\,\,$\color{gr} \color{blue} 1\color{black} $\,\,$ & $\,\,$ 1  $\,\,$ \\
\end{pmatrix},
\end{equation*}
\end{example}
\newpage
\begin{example}
\begin{equation*}
\mathbf{A} =
\begin{pmatrix}
$\,\,$ 1 $\,\,$ & $\,\,$8$\,\,$ & $\,\,$3$\,\,$ & $\,\,$8 $\,\,$ \\
$\,\,$ 1/8$\,\,$ & $\,\,$ 1 $\,\,$ & $\,\,$2$\,\,$ & $\,\,$2 $\,\,$ \\
$\,\,$ 1/3$\,\,$ & $\,\,$ 1/2$\,\,$ & $\,\,$ 1 $\,\,$ & $\,\,$2 $\,\,$ \\
$\,\,$ 1/8$\,\,$ & $\,\,$ 1/2$\,\,$ & $\,\,$ 1/2$\,\,$ & $\,\,$ 1  $\,\,$ \\
\end{pmatrix},
\qquad
\lambda_{\max} =
4.2512,
\qquad
CR = 0.0947
\end{equation*}

\begin{equation*}
\mathbf{w}^{cos} =
\begin{pmatrix}
0.613236\\
0.168513\\
0.146406\\
\color{red} 0.071845\color{black}
\end{pmatrix}\end{equation*}
\begin{equation*}
\left[ \frac{{w}^{cos}_i}{{w}^{cos}_j} \right] =
\begin{pmatrix}
$\,\,$ 1 $\,\,$ & $\,\,$3.6391$\,\,$ & $\,\,$4.1886$\,\,$ & $\,\,$\color{red} 8.5355\color{black} $\,\,$ \\
$\,\,$0.2748$\,\,$ & $\,\,$ 1 $\,\,$ & $\,\,$1.1510$\,\,$ & $\,\,$\color{red} 2.3455\color{black}   $\,\,$ \\
$\,\,$0.2387$\,\,$ & $\,\,$0.8688$\,\,$ & $\,\,$ 1 $\,\,$ & $\,\,$\color{red} 2.0378\color{black}  $\,\,$ \\
$\,\,$\color{red} 0.1172\color{black} $\,\,$ & $\,\,$\color{red} 0.4263\color{black} $\,\,$ & $\,\,$\color{red} 0.4907\color{black} $\,\,$ & $\,\,$ 1  $\,\,$ \\
\end{pmatrix},
\end{equation*}

\begin{equation*}
\mathbf{w}^{\prime} =
\begin{pmatrix}
0.612405\\
0.168284\\
0.146207\\
0.073104
\end{pmatrix} =
0.998644\cdot
\begin{pmatrix}
0.613236\\
0.168513\\
0.146406\\
\color{gr} 0.073203\color{black}
\end{pmatrix},
\end{equation*}
\begin{equation*}
\left[ \frac{{w}^{\prime}_i}{{w}^{\prime}_j} \right] =
\begin{pmatrix}
$\,\,$ 1 $\,\,$ & $\,\,$3.6391$\,\,$ & $\,\,$4.1886$\,\,$ & $\,\,$\color{gr} 8.3772\color{black} $\,\,$ \\
$\,\,$0.2748$\,\,$ & $\,\,$ 1 $\,\,$ & $\,\,$1.1510$\,\,$ & $\,\,$\color{gr} 2.3020\color{black}   $\,\,$ \\
$\,\,$0.2387$\,\,$ & $\,\,$0.8688$\,\,$ & $\,\,$ 1 $\,\,$ & $\,\,$\color{gr} \color{blue} 2\color{black}  $\,\,$ \\
$\,\,$\color{gr} 0.1194\color{black} $\,\,$ & $\,\,$\color{gr} 0.4344\color{black} $\,\,$ & $\,\,$\color{gr} \color{blue}  1/2\color{black} $\,\,$ & $\,\,$ 1  $\,\,$ \\
\end{pmatrix},
\end{equation*}
\end{example}
\newpage
\begin{example}
\begin{equation*}
\mathbf{A} =
\begin{pmatrix}
$\,\,$ 1 $\,\,$ & $\,\,$8$\,\,$ & $\,\,$3$\,\,$ & $\,\,$9 $\,\,$ \\
$\,\,$ 1/8$\,\,$ & $\,\,$ 1 $\,\,$ & $\,\,$2$\,\,$ & $\,\,$2 $\,\,$ \\
$\,\,$ 1/3$\,\,$ & $\,\,$ 1/2$\,\,$ & $\,\,$ 1 $\,\,$ & $\,\,$2 $\,\,$ \\
$\,\,$ 1/9$\,\,$ & $\,\,$ 1/2$\,\,$ & $\,\,$ 1/2$\,\,$ & $\,\,$ 1  $\,\,$ \\
\end{pmatrix},
\qquad
\lambda_{\max} =
4.2469,
\qquad
CR = 0.0931
\end{equation*}

\begin{equation*}
\mathbf{w}^{cos} =
\begin{pmatrix}
0.621445\\
0.166219\\
0.143947\\
\color{red} 0.068389\color{black}
\end{pmatrix}\end{equation*}
\begin{equation*}
\left[ \frac{{w}^{cos}_i}{{w}^{cos}_j} \right] =
\begin{pmatrix}
$\,\,$ 1 $\,\,$ & $\,\,$3.7387$\,\,$ & $\,\,$4.3172$\,\,$ & $\,\,$\color{red} 9.0869\color{black} $\,\,$ \\
$\,\,$0.2675$\,\,$ & $\,\,$ 1 $\,\,$ & $\,\,$1.1547$\,\,$ & $\,\,$\color{red} 2.4305\color{black}   $\,\,$ \\
$\,\,$0.2316$\,\,$ & $\,\,$0.8660$\,\,$ & $\,\,$ 1 $\,\,$ & $\,\,$\color{red} 2.1048\color{black}  $\,\,$ \\
$\,\,$\color{red} 0.1100\color{black} $\,\,$ & $\,\,$\color{red} 0.4114\color{black} $\,\,$ & $\,\,$\color{red} 0.4751\color{black} $\,\,$ & $\,\,$ 1  $\,\,$ \\
\end{pmatrix},
\end{equation*}

\begin{equation*}
\mathbf{w}^{\prime} =
\begin{pmatrix}
0.621035\\
0.166109\\
0.143852\\
0.069004
\end{pmatrix} =
0.999340\cdot
\begin{pmatrix}
0.621445\\
0.166219\\
0.143947\\
\color{gr} 0.069049\color{black}
\end{pmatrix},
\end{equation*}
\begin{equation*}
\left[ \frac{{w}^{\prime}_i}{{w}^{\prime}_j} \right] =
\begin{pmatrix}
$\,\,$ 1 $\,\,$ & $\,\,$3.7387$\,\,$ & $\,\,$4.3172$\,\,$ & $\,\,$\color{gr} \color{blue} 9\color{black} $\,\,$ \\
$\,\,$0.2675$\,\,$ & $\,\,$ 1 $\,\,$ & $\,\,$1.1547$\,\,$ & $\,\,$\color{gr} 2.4072\color{black}   $\,\,$ \\
$\,\,$0.2316$\,\,$ & $\,\,$0.8660$\,\,$ & $\,\,$ 1 $\,\,$ & $\,\,$\color{gr} 2.0847\color{black}  $\,\,$ \\
$\,\,$\color{gr} \color{blue}  1/9\color{black} $\,\,$ & $\,\,$\color{gr} 0.4154\color{black} $\,\,$ & $\,\,$\color{gr} 0.4797\color{black} $\,\,$ & $\,\,$ 1  $\,\,$ \\
\end{pmatrix},
\end{equation*}
\end{example}
\newpage
\begin{example}
\begin{equation*}
\mathbf{A} =
\begin{pmatrix}
$\,\,$ 1 $\,\,$ & $\,\,$8$\,\,$ & $\,\,$4$\,\,$ & $\,\,$5 $\,\,$ \\
$\,\,$ 1/8$\,\,$ & $\,\,$ 1 $\,\,$ & $\,\,$1$\,\,$ & $\,\,$3 $\,\,$ \\
$\,\,$ 1/4$\,\,$ & $\,\,$ 1 $\,\,$ & $\,\,$ 1 $\,\,$ & $\,\,$2 $\,\,$ \\
$\,\,$ 1/5$\,\,$ & $\,\,$ 1/3$\,\,$ & $\,\,$ 1/2$\,\,$ & $\,\,$ 1  $\,\,$ \\
\end{pmatrix},
\qquad
\lambda_{\max} =
4.2162,
\qquad
CR = 0.0815
\end{equation*}

\begin{equation*}
\mathbf{w}^{cos} =
\begin{pmatrix}
0.606717\\
0.158137\\
\color{red} 0.151201\color{black} \\
0.083945
\end{pmatrix}\end{equation*}
\begin{equation*}
\left[ \frac{{w}^{cos}_i}{{w}^{cos}_j} \right] =
\begin{pmatrix}
$\,\,$ 1 $\,\,$ & $\,\,$3.8367$\,\,$ & $\,\,$\color{red} 4.0127\color{black} $\,\,$ & $\,\,$7.2275$\,\,$ \\
$\,\,$0.2606$\,\,$ & $\,\,$ 1 $\,\,$ & $\,\,$\color{red} 1.0459\color{black} $\,\,$ & $\,\,$1.8838  $\,\,$ \\
$\,\,$\color{red} 0.2492\color{black} $\,\,$ & $\,\,$\color{red} 0.9561\color{black} $\,\,$ & $\,\,$ 1 $\,\,$ & $\,\,$\color{red} 1.8012\color{black}  $\,\,$ \\
$\,\,$0.1384$\,\,$ & $\,\,$0.5308$\,\,$ & $\,\,$\color{red} 0.5552\color{black} $\,\,$ & $\,\,$ 1  $\,\,$ \\
\end{pmatrix},
\end{equation*}

\begin{equation*}
\mathbf{w}^{\prime} =
\begin{pmatrix}
0.606427\\
0.158061\\
0.151607\\
0.083905
\end{pmatrix} =
0.999522\cdot
\begin{pmatrix}
0.606717\\
0.158137\\
\color{gr} 0.151679\color{black} \\
0.083945
\end{pmatrix},
\end{equation*}
\begin{equation*}
\left[ \frac{{w}^{\prime}_i}{{w}^{\prime}_j} \right] =
\begin{pmatrix}
$\,\,$ 1 $\,\,$ & $\,\,$3.8367$\,\,$ & $\,\,$\color{gr} \color{blue} 4\color{black} $\,\,$ & $\,\,$7.2275$\,\,$ \\
$\,\,$0.2606$\,\,$ & $\,\,$ 1 $\,\,$ & $\,\,$\color{gr} 1.0426\color{black} $\,\,$ & $\,\,$1.8838  $\,\,$ \\
$\,\,$\color{gr} \color{blue}  1/4\color{black} $\,\,$ & $\,\,$\color{gr} 0.9592\color{black} $\,\,$ & $\,\,$ 1 $\,\,$ & $\,\,$\color{gr} 1.8069\color{black}  $\,\,$ \\
$\,\,$0.1384$\,\,$ & $\,\,$0.5308$\,\,$ & $\,\,$\color{gr} 0.5534\color{black} $\,\,$ & $\,\,$ 1  $\,\,$ \\
\end{pmatrix},
\end{equation*}
\end{example}
\newpage
\begin{example}
\begin{equation*}
\mathbf{A} =
\begin{pmatrix}
$\,\,$ 1 $\,\,$ & $\,\,$8$\,\,$ & $\,\,$4$\,\,$ & $\,\,$6 $\,\,$ \\
$\,\,$ 1/8$\,\,$ & $\,\,$ 1 $\,\,$ & $\,\,$1$\,\,$ & $\,\,$3 $\,\,$ \\
$\,\,$ 1/4$\,\,$ & $\,\,$ 1 $\,\,$ & $\,\,$ 1 $\,\,$ & $\,\,$2 $\,\,$ \\
$\,\,$ 1/6$\,\,$ & $\,\,$ 1/3$\,\,$ & $\,\,$ 1/2$\,\,$ & $\,\,$ 1  $\,\,$ \\
\end{pmatrix},
\qquad
\lambda_{\max} =
4.1707,
\qquad
CR = 0.0644
\end{equation*}

\begin{equation*}
\mathbf{w}^{cos} =
\begin{pmatrix}
0.624582\\
0.151155\\
\color{red} 0.147368\color{black} \\
0.076895
\end{pmatrix}\end{equation*}
\begin{equation*}
\left[ \frac{{w}^{cos}_i}{{w}^{cos}_j} \right] =
\begin{pmatrix}
$\,\,$ 1 $\,\,$ & $\,\,$4.1321$\,\,$ & $\,\,$\color{red} 4.2383\color{black} $\,\,$ & $\,\,$8.1226$\,\,$ \\
$\,\,$0.2420$\,\,$ & $\,\,$ 1 $\,\,$ & $\,\,$\color{red} 1.0257\color{black} $\,\,$ & $\,\,$1.9657  $\,\,$ \\
$\,\,$\color{red} 0.2359\color{black} $\,\,$ & $\,\,$\color{red} 0.9749\color{black} $\,\,$ & $\,\,$ 1 $\,\,$ & $\,\,$\color{red} 1.9165\color{black}  $\,\,$ \\
$\,\,$0.1231$\,\,$ & $\,\,$0.5087$\,\,$ & $\,\,$\color{red} 0.5218\color{black} $\,\,$ & $\,\,$ 1  $\,\,$ \\
\end{pmatrix},
\end{equation*}

\begin{equation*}
\mathbf{w}^{\prime} =
\begin{pmatrix}
0.622226\\
0.150585\\
0.150585\\
0.076605
\end{pmatrix} =
0.996227\cdot
\begin{pmatrix}
0.624582\\
0.151155\\
\color{gr} 0.151155\color{black} \\
0.076895
\end{pmatrix},
\end{equation*}
\begin{equation*}
\left[ \frac{{w}^{\prime}_i}{{w}^{\prime}_j} \right] =
\begin{pmatrix}
$\,\,$ 1 $\,\,$ & $\,\,$4.1321$\,\,$ & $\,\,$\color{gr} 4.1321\color{black} $\,\,$ & $\,\,$8.1226$\,\,$ \\
$\,\,$0.2420$\,\,$ & $\,\,$ 1 $\,\,$ & $\,\,$\color{gr} \color{blue} 1\color{black} $\,\,$ & $\,\,$1.9657  $\,\,$ \\
$\,\,$\color{gr} 0.2420\color{black} $\,\,$ & $\,\,$\color{gr} \color{blue} 1\color{black} $\,\,$ & $\,\,$ 1 $\,\,$ & $\,\,$\color{gr} 1.9657\color{black}  $\,\,$ \\
$\,\,$0.1231$\,\,$ & $\,\,$0.5087$\,\,$ & $\,\,$\color{gr} 0.5087\color{black} $\,\,$ & $\,\,$ 1  $\,\,$ \\
\end{pmatrix},
\end{equation*}
\end{example}
\newpage
\begin{example}
\begin{equation*}
\mathbf{A} =
\begin{pmatrix}
$\,\,$ 1 $\,\,$ & $\,\,$8$\,\,$ & $\,\,$4$\,\,$ & $\,\,$6 $\,\,$ \\
$\,\,$ 1/8$\,\,$ & $\,\,$ 1 $\,\,$ & $\,\,$1$\,\,$ & $\,\,$4 $\,\,$ \\
$\,\,$ 1/4$\,\,$ & $\,\,$ 1 $\,\,$ & $\,\,$ 1 $\,\,$ & $\,\,$2 $\,\,$ \\
$\,\,$ 1/6$\,\,$ & $\,\,$ 1/4$\,\,$ & $\,\,$ 1/2$\,\,$ & $\,\,$ 1  $\,\,$ \\
\end{pmatrix},
\qquad
\lambda_{\max} =
4.2512,
\qquad
CR = 0.0947
\end{equation*}

\begin{equation*}
\mathbf{w}^{cos} =
\begin{pmatrix}
0.614056\\
0.168460\\
\color{red} 0.144000\color{black} \\
0.073484
\end{pmatrix}\end{equation*}
\begin{equation*}
\left[ \frac{{w}^{cos}_i}{{w}^{cos}_j} \right] =
\begin{pmatrix}
$\,\,$ 1 $\,\,$ & $\,\,$3.6451$\,\,$ & $\,\,$\color{red} 4.2643\color{black} $\,\,$ & $\,\,$8.3563$\,\,$ \\
$\,\,$0.2743$\,\,$ & $\,\,$ 1 $\,\,$ & $\,\,$\color{red} 1.1699\color{black} $\,\,$ & $\,\,$2.2925  $\,\,$ \\
$\,\,$\color{red} 0.2345\color{black} $\,\,$ & $\,\,$\color{red} 0.8548\color{black} $\,\,$ & $\,\,$ 1 $\,\,$ & $\,\,$\color{red} 1.9596\color{black}  $\,\,$ \\
$\,\,$0.1197$\,\,$ & $\,\,$0.4362$\,\,$ & $\,\,$\color{red} 0.5103\color{black} $\,\,$ & $\,\,$ 1  $\,\,$ \\
\end{pmatrix},
\end{equation*}

\begin{equation*}
\mathbf{w}^{\prime} =
\begin{pmatrix}
0.612238\\
0.167961\\
0.146533\\
0.073267
\end{pmatrix} =
0.997040\cdot
\begin{pmatrix}
0.614056\\
0.168460\\
\color{gr} 0.146968\color{black} \\
0.073484
\end{pmatrix},
\end{equation*}
\begin{equation*}
\left[ \frac{{w}^{\prime}_i}{{w}^{\prime}_j} \right] =
\begin{pmatrix}
$\,\,$ 1 $\,\,$ & $\,\,$3.6451$\,\,$ & $\,\,$\color{gr} 4.1781\color{black} $\,\,$ & $\,\,$8.3563$\,\,$ \\
$\,\,$0.2743$\,\,$ & $\,\,$ 1 $\,\,$ & $\,\,$\color{gr} 1.1462\color{black} $\,\,$ & $\,\,$2.2925  $\,\,$ \\
$\,\,$\color{gr} 0.2393\color{black} $\,\,$ & $\,\,$\color{gr} 0.8724\color{black} $\,\,$ & $\,\,$ 1 $\,\,$ & $\,\,$\color{gr} \color{blue} 2\color{black}  $\,\,$ \\
$\,\,$0.1197$\,\,$ & $\,\,$0.4362$\,\,$ & $\,\,$\color{gr} \color{blue}  1/2\color{black} $\,\,$ & $\,\,$ 1  $\,\,$ \\
\end{pmatrix},
\end{equation*}
\end{example}
\newpage
\begin{example}
\begin{equation*}
\mathbf{A} =
\begin{pmatrix}
$\,\,$ 1 $\,\,$ & $\,\,$8$\,\,$ & $\,\,$4$\,\,$ & $\,\,$8 $\,\,$ \\
$\,\,$ 1/8$\,\,$ & $\,\,$ 1 $\,\,$ & $\,\,$1$\,\,$ & $\,\,$5 $\,\,$ \\
$\,\,$ 1/4$\,\,$ & $\,\,$ 1 $\,\,$ & $\,\,$ 1 $\,\,$ & $\,\,$3 $\,\,$ \\
$\,\,$ 1/8$\,\,$ & $\,\,$ 1/5$\,\,$ & $\,\,$ 1/3$\,\,$ & $\,\,$ 1  $\,\,$ \\
\end{pmatrix},
\qquad
\lambda_{\max} =
4.2277,
\qquad
CR = 0.0859
\end{equation*}

\begin{equation*}
\mathbf{w}^{cos} =
\begin{pmatrix}
0.625967\\
0.166504\\
\color{red} 0.152661\color{black} \\
0.054868
\end{pmatrix}\end{equation*}
\begin{equation*}
\left[ \frac{{w}^{cos}_i}{{w}^{cos}_j} \right] =
\begin{pmatrix}
$\,\,$ 1 $\,\,$ & $\,\,$3.7595$\,\,$ & $\,\,$\color{red} 4.1004\color{black} $\,\,$ & $\,\,$11.4086$\,\,$ \\
$\,\,$0.2660$\,\,$ & $\,\,$ 1 $\,\,$ & $\,\,$\color{red} 1.0907\color{black} $\,\,$ & $\,\,$3.0346  $\,\,$ \\
$\,\,$\color{red} 0.2439\color{black} $\,\,$ & $\,\,$\color{red} 0.9169\color{black} $\,\,$ & $\,\,$ 1 $\,\,$ & $\,\,$\color{red} 2.7823\color{black}  $\,\,$ \\
$\,\,$0.0877$\,\,$ & $\,\,$0.3295$\,\,$ & $\,\,$\color{red} 0.3594\color{black} $\,\,$ & $\,\,$ 1  $\,\,$ \\
\end{pmatrix},
\end{equation*}

\begin{equation*}
\mathbf{w}^{\prime} =
\begin{pmatrix}
0.623578\\
0.165869\\
0.155895\\
0.054659
\end{pmatrix} =
0.996184\cdot
\begin{pmatrix}
0.625967\\
0.166504\\
\color{gr} 0.156492\color{black} \\
0.054868
\end{pmatrix},
\end{equation*}
\begin{equation*}
\left[ \frac{{w}^{\prime}_i}{{w}^{\prime}_j} \right] =
\begin{pmatrix}
$\,\,$ 1 $\,\,$ & $\,\,$3.7595$\,\,$ & $\,\,$\color{gr} \color{blue} 4\color{black} $\,\,$ & $\,\,$11.4086$\,\,$ \\
$\,\,$0.2660$\,\,$ & $\,\,$ 1 $\,\,$ & $\,\,$\color{gr} 1.0640\color{black} $\,\,$ & $\,\,$3.0346  $\,\,$ \\
$\,\,$\color{gr} \color{blue}  1/4\color{black} $\,\,$ & $\,\,$\color{gr} 0.9399\color{black} $\,\,$ & $\,\,$ 1 $\,\,$ & $\,\,$\color{gr} 2.8521\color{black}  $\,\,$ \\
$\,\,$0.0877$\,\,$ & $\,\,$0.3295$\,\,$ & $\,\,$\color{gr} 0.3506\color{black} $\,\,$ & $\,\,$ 1  $\,\,$ \\
\end{pmatrix},
\end{equation*}
\end{example}
\newpage
\begin{example}
\begin{equation*}
\mathbf{A} =
\begin{pmatrix}
$\,\,$ 1 $\,\,$ & $\,\,$8$\,\,$ & $\,\,$4$\,\,$ & $\,\,$9 $\,\,$ \\
$\,\,$ 1/8$\,\,$ & $\,\,$ 1 $\,\,$ & $\,\,$1$\,\,$ & $\,\,$5 $\,\,$ \\
$\,\,$ 1/4$\,\,$ & $\,\,$ 1 $\,\,$ & $\,\,$ 1 $\,\,$ & $\,\,$3 $\,\,$ \\
$\,\,$ 1/9$\,\,$ & $\,\,$ 1/5$\,\,$ & $\,\,$ 1/3$\,\,$ & $\,\,$ 1  $\,\,$ \\
\end{pmatrix},
\qquad
\lambda_{\max} =
4.1974,
\qquad
CR = 0.0744
\end{equation*}

\begin{equation*}
\mathbf{w}^{cos} =
\begin{pmatrix}
0.636821\\
0.161359\\
\color{red} 0.150055\color{black} \\
0.051764
\end{pmatrix}\end{equation*}
\begin{equation*}
\left[ \frac{{w}^{cos}_i}{{w}^{cos}_j} \right] =
\begin{pmatrix}
$\,\,$ 1 $\,\,$ & $\,\,$3.9466$\,\,$ & $\,\,$\color{red} 4.2439\color{black} $\,\,$ & $\,\,$12.3023$\,\,$ \\
$\,\,$0.2534$\,\,$ & $\,\,$ 1 $\,\,$ & $\,\,$\color{red} 1.0753\color{black} $\,\,$ & $\,\,$3.1172  $\,\,$ \\
$\,\,$\color{red} 0.2356\color{black} $\,\,$ & $\,\,$\color{red} 0.9299\color{black} $\,\,$ & $\,\,$ 1 $\,\,$ & $\,\,$\color{red} 2.8988\color{black}  $\,\,$ \\
$\,\,$0.0813$\,\,$ & $\,\,$0.3208$\,\,$ & $\,\,$\color{red} 0.3450\color{black} $\,\,$ & $\,\,$ 1  $\,\,$ \\
\end{pmatrix},
\end{equation*}

\begin{equation*}
\mathbf{w}^{\prime} =
\begin{pmatrix}
0.633503\\
0.160519\\
0.154484\\
0.051495
\end{pmatrix} =
0.994789\cdot
\begin{pmatrix}
0.636821\\
0.161359\\
\color{gr} 0.155293\color{black} \\
0.051764
\end{pmatrix},
\end{equation*}
\begin{equation*}
\left[ \frac{{w}^{\prime}_i}{{w}^{\prime}_j} \right] =
\begin{pmatrix}
$\,\,$ 1 $\,\,$ & $\,\,$3.9466$\,\,$ & $\,\,$\color{gr} 4.1008\color{black} $\,\,$ & $\,\,$12.3023$\,\,$ \\
$\,\,$0.2534$\,\,$ & $\,\,$ 1 $\,\,$ & $\,\,$\color{gr} 1.0391\color{black} $\,\,$ & $\,\,$3.1172  $\,\,$ \\
$\,\,$\color{gr} 0.2439\color{black} $\,\,$ & $\,\,$\color{gr} 0.9624\color{black} $\,\,$ & $\,\,$ 1 $\,\,$ & $\,\,$\color{gr} \color{blue} 3\color{black}  $\,\,$ \\
$\,\,$0.0813$\,\,$ & $\,\,$0.3208$\,\,$ & $\,\,$\color{gr} \color{blue}  1/3\color{black} $\,\,$ & $\,\,$ 1  $\,\,$ \\
\end{pmatrix},
\end{equation*}
\end{example}
\newpage
\begin{example}
\begin{equation*}
\mathbf{A} =
\begin{pmatrix}
$\,\,$ 1 $\,\,$ & $\,\,$8$\,\,$ & $\,\,$4$\,\,$ & $\,\,$9 $\,\,$ \\
$\,\,$ 1/8$\,\,$ & $\,\,$ 1 $\,\,$ & $\,\,$1$\,\,$ & $\,\,$6 $\,\,$ \\
$\,\,$ 1/4$\,\,$ & $\,\,$ 1 $\,\,$ & $\,\,$ 1 $\,\,$ & $\,\,$3 $\,\,$ \\
$\,\,$ 1/9$\,\,$ & $\,\,$ 1/6$\,\,$ & $\,\,$ 1/3$\,\,$ & $\,\,$ 1  $\,\,$ \\
\end{pmatrix},
\qquad
\lambda_{\max} =
4.2512,
\qquad
CR = 0.0947
\end{equation*}

\begin{equation*}
\mathbf{w}^{cos} =
\begin{pmatrix}
0.629340\\
0.172722\\
\color{red} 0.147672\color{black} \\
0.050265
\end{pmatrix}\end{equation*}
\begin{equation*}
\left[ \frac{{w}^{cos}_i}{{w}^{cos}_j} \right] =
\begin{pmatrix}
$\,\,$ 1 $\,\,$ & $\,\,$3.6437$\,\,$ & $\,\,$\color{red} 4.2617\color{black} $\,\,$ & $\,\,$12.5204$\,\,$ \\
$\,\,$0.2744$\,\,$ & $\,\,$ 1 $\,\,$ & $\,\,$\color{red} 1.1696\color{black} $\,\,$ & $\,\,$3.4362  $\,\,$ \\
$\,\,$\color{red} 0.2346\color{black} $\,\,$ & $\,\,$\color{red} 0.8550\color{black} $\,\,$ & $\,\,$ 1 $\,\,$ & $\,\,$\color{red} 2.9379\color{black}  $\,\,$ \\
$\,\,$0.0799$\,\,$ & $\,\,$0.2910$\,\,$ & $\,\,$\color{red} 0.3404\color{black} $\,\,$ & $\,\,$ 1  $\,\,$ \\
\end{pmatrix},
\end{equation*}

\begin{equation*}
\mathbf{w}^{\prime} =
\begin{pmatrix}
0.627381\\
0.172184\\
0.150326\\
0.050109
\end{pmatrix} =
0.996886\cdot
\begin{pmatrix}
0.629340\\
0.172722\\
\color{gr} 0.150796\color{black} \\
0.050265
\end{pmatrix},
\end{equation*}
\begin{equation*}
\left[ \frac{{w}^{\prime}_i}{{w}^{\prime}_j} \right] =
\begin{pmatrix}
$\,\,$ 1 $\,\,$ & $\,\,$3.6437$\,\,$ & $\,\,$\color{gr} 4.1735\color{black} $\,\,$ & $\,\,$12.5204$\,\,$ \\
$\,\,$0.2744$\,\,$ & $\,\,$ 1 $\,\,$ & $\,\,$\color{gr} 1.1454\color{black} $\,\,$ & $\,\,$3.4362  $\,\,$ \\
$\,\,$\color{gr} 0.2396\color{black} $\,\,$ & $\,\,$\color{gr} 0.8731\color{black} $\,\,$ & $\,\,$ 1 $\,\,$ & $\,\,$\color{gr} \color{blue} 3\color{black}  $\,\,$ \\
$\,\,$0.0799$\,\,$ & $\,\,$0.2910$\,\,$ & $\,\,$\color{gr} \color{blue}  1/3\color{black} $\,\,$ & $\,\,$ 1  $\,\,$ \\
\end{pmatrix},
\end{equation*}
\end{example}
\newpage
\begin{example}
\begin{equation*}
\mathbf{A} =
\begin{pmatrix}
$\,\,$ 1 $\,\,$ & $\,\,$8$\,\,$ & $\,\,$5$\,\,$ & $\,\,$7 $\,\,$ \\
$\,\,$ 1/8$\,\,$ & $\,\,$ 1 $\,\,$ & $\,\,$1$\,\,$ & $\,\,$4 $\,\,$ \\
$\,\,$ 1/5$\,\,$ & $\,\,$ 1 $\,\,$ & $\,\,$ 1 $\,\,$ & $\,\,$2 $\,\,$ \\
$\,\,$ 1/7$\,\,$ & $\,\,$ 1/4$\,\,$ & $\,\,$ 1/2$\,\,$ & $\,\,$ 1  $\,\,$ \\
\end{pmatrix},
\qquad
\lambda_{\max} =
4.2035,
\qquad
CR = 0.0767
\end{equation*}

\begin{equation*}
\mathbf{w}^{cos} =
\begin{pmatrix}
0.646845\\
0.157823\\
\color{red} 0.129005\color{black} \\
0.066327
\end{pmatrix}\end{equation*}
\begin{equation*}
\left[ \frac{{w}^{cos}_i}{{w}^{cos}_j} \right] =
\begin{pmatrix}
$\,\,$ 1 $\,\,$ & $\,\,$4.0986$\,\,$ & $\,\,$\color{red} 5.0141\color{black} $\,\,$ & $\,\,$9.7524$\,\,$ \\
$\,\,$0.2440$\,\,$ & $\,\,$ 1 $\,\,$ & $\,\,$\color{red} 1.2234\color{black} $\,\,$ & $\,\,$2.3795  $\,\,$ \\
$\,\,$\color{red} 0.1994\color{black} $\,\,$ & $\,\,$\color{red} 0.8174\color{black} $\,\,$ & $\,\,$ 1 $\,\,$ & $\,\,$\color{red} 1.9450\color{black}  $\,\,$ \\
$\,\,$0.1025$\,\,$ & $\,\,$0.4203$\,\,$ & $\,\,$\color{red} 0.5141\color{black} $\,\,$ & $\,\,$ 1  $\,\,$ \\
\end{pmatrix},
\end{equation*}

\begin{equation*}
\mathbf{w}^{\prime} =
\begin{pmatrix}
0.646610\\
0.157765\\
0.129322\\
0.066303
\end{pmatrix} =
0.999636\cdot
\begin{pmatrix}
0.646845\\
0.157823\\
\color{gr} 0.129369\color{black} \\
0.066327
\end{pmatrix},
\end{equation*}
\begin{equation*}
\left[ \frac{{w}^{\prime}_i}{{w}^{\prime}_j} \right] =
\begin{pmatrix}
$\,\,$ 1 $\,\,$ & $\,\,$4.0986$\,\,$ & $\,\,$\color{gr} \color{blue} 5\color{black} $\,\,$ & $\,\,$9.7524$\,\,$ \\
$\,\,$0.2440$\,\,$ & $\,\,$ 1 $\,\,$ & $\,\,$\color{gr} 1.2199\color{black} $\,\,$ & $\,\,$2.3795  $\,\,$ \\
$\,\,$\color{gr} \color{blue}  1/5\color{black} $\,\,$ & $\,\,$\color{gr} 0.8197\color{black} $\,\,$ & $\,\,$ 1 $\,\,$ & $\,\,$\color{gr} 1.9505\color{black}  $\,\,$ \\
$\,\,$0.1025$\,\,$ & $\,\,$0.4203$\,\,$ & $\,\,$\color{gr} 0.5127\color{black} $\,\,$ & $\,\,$ 1  $\,\,$ \\
\end{pmatrix},
\end{equation*}
\end{example}
\newpage
\begin{example}
\begin{equation*}
\mathbf{A} =
\begin{pmatrix}
$\,\,$ 1 $\,\,$ & $\,\,$8$\,\,$ & $\,\,$5$\,\,$ & $\,\,$7 $\,\,$ \\
$\,\,$ 1/8$\,\,$ & $\,\,$ 1 $\,\,$ & $\,\,$3$\,\,$ & $\,\,$2 $\,\,$ \\
$\,\,$ 1/5$\,\,$ & $\,\,$ 1/3$\,\,$ & $\,\,$ 1 $\,\,$ & $\,\,$1 $\,\,$ \\
$\,\,$ 1/7$\,\,$ & $\,\,$ 1/2$\,\,$ & $\,\,$ 1 $\,\,$ & $\,\,$ 1  $\,\,$ \\
\end{pmatrix},
\qquad
\lambda_{\max} =
4.2224,
\qquad
CR = 0.0838
\end{equation*}

\begin{equation*}
\mathbf{w}^{cos} =
\begin{pmatrix}
0.645322\\
0.175373\\
0.092634\\
\color{red} 0.086671\color{black}
\end{pmatrix}\end{equation*}
\begin{equation*}
\left[ \frac{{w}^{cos}_i}{{w}^{cos}_j} \right] =
\begin{pmatrix}
$\,\,$ 1 $\,\,$ & $\,\,$3.6797$\,\,$ & $\,\,$6.9663$\,\,$ & $\,\,$\color{red} 7.4456\color{black} $\,\,$ \\
$\,\,$0.2718$\,\,$ & $\,\,$ 1 $\,\,$ & $\,\,$1.8932$\,\,$ & $\,\,$\color{red} 2.0234\color{black}   $\,\,$ \\
$\,\,$0.1435$\,\,$ & $\,\,$0.5282$\,\,$ & $\,\,$ 1 $\,\,$ & $\,\,$\color{red} 1.0688\color{black}  $\,\,$ \\
$\,\,$\color{red} 0.1343\color{black} $\,\,$ & $\,\,$\color{red} 0.4942\color{black} $\,\,$ & $\,\,$\color{red} 0.9356\color{black} $\,\,$ & $\,\,$ 1  $\,\,$ \\
\end{pmatrix},
\end{equation*}

\begin{equation*}
\mathbf{w}^{\prime} =
\begin{pmatrix}
0.644667\\
0.175195\\
0.092540\\
0.087597
\end{pmatrix} =
0.998986\cdot
\begin{pmatrix}
0.645322\\
0.175373\\
0.092634\\
\color{gr} 0.087686\color{black}
\end{pmatrix},
\end{equation*}
\begin{equation*}
\left[ \frac{{w}^{\prime}_i}{{w}^{\prime}_j} \right] =
\begin{pmatrix}
$\,\,$ 1 $\,\,$ & $\,\,$3.6797$\,\,$ & $\,\,$6.9663$\,\,$ & $\,\,$\color{gr} 7.3594\color{black} $\,\,$ \\
$\,\,$0.2718$\,\,$ & $\,\,$ 1 $\,\,$ & $\,\,$1.8932$\,\,$ & $\,\,$\color{gr} \color{blue} 2\color{black}   $\,\,$ \\
$\,\,$0.1435$\,\,$ & $\,\,$0.5282$\,\,$ & $\,\,$ 1 $\,\,$ & $\,\,$\color{gr} 1.0564\color{black}  $\,\,$ \\
$\,\,$\color{gr} 0.1359\color{black} $\,\,$ & $\,\,$\color{gr} \color{blue}  1/2\color{black} $\,\,$ & $\,\,$\color{gr} 0.9466\color{black} $\,\,$ & $\,\,$ 1  $\,\,$ \\
\end{pmatrix},
\end{equation*}
\end{example}
\newpage
\begin{example}
\begin{equation*}
\mathbf{A} =
\begin{pmatrix}
$\,\,$ 1 $\,\,$ & $\,\,$8$\,\,$ & $\,\,$5$\,\,$ & $\,\,$8 $\,\,$ \\
$\,\,$ 1/8$\,\,$ & $\,\,$ 1 $\,\,$ & $\,\,$1$\,\,$ & $\,\,$3 $\,\,$ \\
$\,\,$ 1/5$\,\,$ & $\,\,$ 1 $\,\,$ & $\,\,$ 1 $\,\,$ & $\,\,$2 $\,\,$ \\
$\,\,$ 1/8$\,\,$ & $\,\,$ 1/3$\,\,$ & $\,\,$ 1/2$\,\,$ & $\,\,$ 1  $\,\,$ \\
\end{pmatrix},
\qquad
\lambda_{\max} =
4.1046,
\qquad
CR = 0.0395
\end{equation*}

\begin{equation*}
\mathbf{w}^{cos} =
\begin{pmatrix}
0.669808\\
0.136115\\
\color{red} 0.128956\color{black} \\
0.065120
\end{pmatrix}\end{equation*}
\begin{equation*}
\left[ \frac{{w}^{cos}_i}{{w}^{cos}_j} \right] =
\begin{pmatrix}
$\,\,$ 1 $\,\,$ & $\,\,$4.9209$\,\,$ & $\,\,$\color{red} 5.1941\color{black} $\,\,$ & $\,\,$10.2857$\,\,$ \\
$\,\,$0.2032$\,\,$ & $\,\,$ 1 $\,\,$ & $\,\,$\color{red} 1.0555\color{black} $\,\,$ & $\,\,$2.0902  $\,\,$ \\
$\,\,$\color{red} 0.1925\color{black} $\,\,$ & $\,\,$\color{red} 0.9474\color{black} $\,\,$ & $\,\,$ 1 $\,\,$ & $\,\,$\color{red} 1.9803\color{black}  $\,\,$ \\
$\,\,$0.0972$\,\,$ & $\,\,$0.4784$\,\,$ & $\,\,$\color{red} 0.5050\color{black} $\,\,$ & $\,\,$ 1  $\,\,$ \\
\end{pmatrix},
\end{equation*}

\begin{equation*}
\mathbf{w}^{\prime} =
\begin{pmatrix}
0.668949\\
0.135941\\
0.130073\\
0.065037
\end{pmatrix} =
0.998717\cdot
\begin{pmatrix}
0.669808\\
0.136115\\
\color{gr} 0.130240\color{black} \\
0.065120
\end{pmatrix},
\end{equation*}
\begin{equation*}
\left[ \frac{{w}^{\prime}_i}{{w}^{\prime}_j} \right] =
\begin{pmatrix}
$\,\,$ 1 $\,\,$ & $\,\,$4.9209$\,\,$ & $\,\,$\color{gr} 5.1429\color{black} $\,\,$ & $\,\,$10.2857$\,\,$ \\
$\,\,$0.2032$\,\,$ & $\,\,$ 1 $\,\,$ & $\,\,$\color{gr} 1.0451\color{black} $\,\,$ & $\,\,$2.0902  $\,\,$ \\
$\,\,$\color{gr} 0.1944\color{black} $\,\,$ & $\,\,$\color{gr} 0.9568\color{black} $\,\,$ & $\,\,$ 1 $\,\,$ & $\,\,$\color{gr} \color{blue} 2\color{black}  $\,\,$ \\
$\,\,$0.0972$\,\,$ & $\,\,$0.4784$\,\,$ & $\,\,$\color{gr} \color{blue}  1/2\color{black} $\,\,$ & $\,\,$ 1  $\,\,$ \\
\end{pmatrix},
\end{equation*}
\end{example}
\newpage
\begin{example}
\begin{equation*}
\mathbf{A} =
\begin{pmatrix}
$\,\,$ 1 $\,\,$ & $\,\,$8$\,\,$ & $\,\,$6$\,\,$ & $\,\,$8 $\,\,$ \\
$\,\,$ 1/8$\,\,$ & $\,\,$ 1 $\,\,$ & $\,\,$3$\,\,$ & $\,\,$2 $\,\,$ \\
$\,\,$ 1/6$\,\,$ & $\,\,$ 1/3$\,\,$ & $\,\,$ 1 $\,\,$ & $\,\,$1 $\,\,$ \\
$\,\,$ 1/8$\,\,$ & $\,\,$ 1/2$\,\,$ & $\,\,$ 1 $\,\,$ & $\,\,$ 1  $\,\,$ \\
\end{pmatrix},
\qquad
\lambda_{\max} =
4.1707,
\qquad
CR = 0.0644
\end{equation*}

\begin{equation*}
\mathbf{w}^{cos} =
\begin{pmatrix}
0.673443\\
0.163680\\
0.083203\\
\color{red} 0.079674\color{black}
\end{pmatrix}\end{equation*}
\begin{equation*}
\left[ \frac{{w}^{cos}_i}{{w}^{cos}_j} \right] =
\begin{pmatrix}
$\,\,$ 1 $\,\,$ & $\,\,$4.1144$\,\,$ & $\,\,$8.0940$\,\,$ & $\,\,$\color{red} 8.4525\color{black} $\,\,$ \\
$\,\,$0.2430$\,\,$ & $\,\,$ 1 $\,\,$ & $\,\,$1.9672$\,\,$ & $\,\,$\color{red} 2.0544\color{black}   $\,\,$ \\
$\,\,$0.1235$\,\,$ & $\,\,$0.5083$\,\,$ & $\,\,$ 1 $\,\,$ & $\,\,$\color{red} 1.0443\color{black}  $\,\,$ \\
$\,\,$\color{red} 0.1183\color{black} $\,\,$ & $\,\,$\color{red} 0.4868\color{black} $\,\,$ & $\,\,$\color{red} 0.9576\color{black} $\,\,$ & $\,\,$ 1  $\,\,$ \\
\end{pmatrix},
\end{equation*}

\begin{equation*}
\mathbf{w}^{\prime} =
\begin{pmatrix}
0.671988\\
0.163326\\
0.083023\\
0.081663
\end{pmatrix} =
0.997839\cdot
\begin{pmatrix}
0.673443\\
0.163680\\
0.083203\\
\color{gr} 0.081840\color{black}
\end{pmatrix},
\end{equation*}
\begin{equation*}
\left[ \frac{{w}^{\prime}_i}{{w}^{\prime}_j} \right] =
\begin{pmatrix}
$\,\,$ 1 $\,\,$ & $\,\,$4.1144$\,\,$ & $\,\,$8.0940$\,\,$ & $\,\,$\color{gr} 8.2288\color{black} $\,\,$ \\
$\,\,$0.2430$\,\,$ & $\,\,$ 1 $\,\,$ & $\,\,$1.9672$\,\,$ & $\,\,$\color{gr} \color{blue} 2\color{black}   $\,\,$ \\
$\,\,$0.1235$\,\,$ & $\,\,$0.5083$\,\,$ & $\,\,$ 1 $\,\,$ & $\,\,$\color{gr} 1.0167\color{black}  $\,\,$ \\
$\,\,$\color{gr} 0.1215\color{black} $\,\,$ & $\,\,$\color{gr} \color{blue}  1/2\color{black} $\,\,$ & $\,\,$\color{gr} 0.9836\color{black} $\,\,$ & $\,\,$ 1  $\,\,$ \\
\end{pmatrix},
\end{equation*}
\end{example}
\newpage
\begin{example}
\begin{equation*}
\mathbf{A} =
\begin{pmatrix}
$\,\,$ 1 $\,\,$ & $\,\,$8$\,\,$ & $\,\,$6$\,\,$ & $\,\,$8 $\,\,$ \\
$\,\,$ 1/8$\,\,$ & $\,\,$ 1 $\,\,$ & $\,\,$4$\,\,$ & $\,\,$2 $\,\,$ \\
$\,\,$ 1/6$\,\,$ & $\,\,$ 1/4$\,\,$ & $\,\,$ 1 $\,\,$ & $\,\,$1 $\,\,$ \\
$\,\,$ 1/8$\,\,$ & $\,\,$ 1/2$\,\,$ & $\,\,$ 1 $\,\,$ & $\,\,$ 1  $\,\,$ \\
\end{pmatrix},
\qquad
\lambda_{\max} =
4.2512,
\qquad
CR = 0.0947
\end{equation*}

\begin{equation*}
\mathbf{w}^{cos} =
\begin{pmatrix}
0.660902\\
0.182039\\
0.079375\\
\color{red} 0.077684\color{black}
\end{pmatrix}\end{equation*}
\begin{equation*}
\left[ \frac{{w}^{cos}_i}{{w}^{cos}_j} \right] =
\begin{pmatrix}
$\,\,$ 1 $\,\,$ & $\,\,$3.6305$\,\,$ & $\,\,$8.3263$\,\,$ & $\,\,$\color{red} 8.5076\color{black} $\,\,$ \\
$\,\,$0.2754$\,\,$ & $\,\,$ 1 $\,\,$ & $\,\,$2.2934$\,\,$ & $\,\,$\color{red} 2.3433\color{black}   $\,\,$ \\
$\,\,$0.1201$\,\,$ & $\,\,$0.4360$\,\,$ & $\,\,$ 1 $\,\,$ & $\,\,$\color{red} 1.0218\color{black}  $\,\,$ \\
$\,\,$\color{red} 0.1175\color{black} $\,\,$ & $\,\,$\color{red} 0.4267\color{black} $\,\,$ & $\,\,$\color{red} 0.9787\color{black} $\,\,$ & $\,\,$ 1  $\,\,$ \\
\end{pmatrix},
\end{equation*}

\begin{equation*}
\mathbf{w}^{\prime} =
\begin{pmatrix}
0.659786\\
0.181732\\
0.079241\\
0.079241
\end{pmatrix} =
0.998312\cdot
\begin{pmatrix}
0.660902\\
0.182039\\
0.079375\\
\color{gr} 0.079375\color{black}
\end{pmatrix},
\end{equation*}
\begin{equation*}
\left[ \frac{{w}^{\prime}_i}{{w}^{\prime}_j} \right] =
\begin{pmatrix}
$\,\,$ 1 $\,\,$ & $\,\,$3.6305$\,\,$ & $\,\,$8.3263$\,\,$ & $\,\,$\color{gr} 8.3263\color{black} $\,\,$ \\
$\,\,$0.2754$\,\,$ & $\,\,$ 1 $\,\,$ & $\,\,$2.2934$\,\,$ & $\,\,$\color{gr} 2.2934\color{black}   $\,\,$ \\
$\,\,$0.1201$\,\,$ & $\,\,$0.4360$\,\,$ & $\,\,$ 1 $\,\,$ & $\,\,$\color{gr} \color{blue} 1\color{black}  $\,\,$ \\
$\,\,$\color{gr} 0.1201\color{black} $\,\,$ & $\,\,$\color{gr} 0.4360\color{black} $\,\,$ & $\,\,$\color{gr} \color{blue} 1\color{black} $\,\,$ & $\,\,$ 1  $\,\,$ \\
\end{pmatrix},
\end{equation*}
\end{example}
\newpage
\begin{example}
\begin{equation*}
\mathbf{A} =
\begin{pmatrix}
$\,\,$ 1 $\,\,$ & $\,\,$8$\,\,$ & $\,\,$6$\,\,$ & $\,\,$9 $\,\,$ \\
$\,\,$ 1/8$\,\,$ & $\,\,$ 1 $\,\,$ & $\,\,$4$\,\,$ & $\,\,$2 $\,\,$ \\
$\,\,$ 1/6$\,\,$ & $\,\,$ 1/4$\,\,$ & $\,\,$ 1 $\,\,$ & $\,\,$1 $\,\,$ \\
$\,\,$ 1/9$\,\,$ & $\,\,$ 1/2$\,\,$ & $\,\,$ 1 $\,\,$ & $\,\,$ 1  $\,\,$ \\
\end{pmatrix},
\qquad
\lambda_{\max} =
4.2469,
\qquad
CR = 0.0931
\end{equation*}

\begin{equation*}
\mathbf{w}^{cos} =
\begin{pmatrix}
0.668856\\
0.179362\\
0.077957\\
\color{red} 0.073824\color{black}
\end{pmatrix}\end{equation*}
\begin{equation*}
\left[ \frac{{w}^{cos}_i}{{w}^{cos}_j} \right] =
\begin{pmatrix}
$\,\,$ 1 $\,\,$ & $\,\,$3.7291$\,\,$ & $\,\,$8.5798$\,\,$ & $\,\,$\color{red} 9.0601\color{black} $\,\,$ \\
$\,\,$0.2682$\,\,$ & $\,\,$ 1 $\,\,$ & $\,\,$2.3008$\,\,$ & $\,\,$\color{red} 2.4296\color{black}   $\,\,$ \\
$\,\,$0.1166$\,\,$ & $\,\,$0.4346$\,\,$ & $\,\,$ 1 $\,\,$ & $\,\,$\color{red} 1.0560\color{black}  $\,\,$ \\
$\,\,$\color{red} 0.1104\color{black} $\,\,$ & $\,\,$\color{red} 0.4116\color{black} $\,\,$ & $\,\,$\color{red} 0.9470\color{black} $\,\,$ & $\,\,$ 1  $\,\,$ \\
\end{pmatrix},
\end{equation*}

\begin{equation*}
\mathbf{w}^{\prime} =
\begin{pmatrix}
0.668527\\
0.179274\\
0.077918\\
0.074281
\end{pmatrix} =
0.999507\cdot
\begin{pmatrix}
0.668856\\
0.179362\\
0.077957\\
\color{gr} 0.074317\color{black}
\end{pmatrix},
\end{equation*}
\begin{equation*}
\left[ \frac{{w}^{\prime}_i}{{w}^{\prime}_j} \right] =
\begin{pmatrix}
$\,\,$ 1 $\,\,$ & $\,\,$3.7291$\,\,$ & $\,\,$8.5798$\,\,$ & $\,\,$\color{gr} \color{blue} 9\color{black} $\,\,$ \\
$\,\,$0.2682$\,\,$ & $\,\,$ 1 $\,\,$ & $\,\,$2.3008$\,\,$ & $\,\,$\color{gr} 2.4135\color{black}   $\,\,$ \\
$\,\,$0.1166$\,\,$ & $\,\,$0.4346$\,\,$ & $\,\,$ 1 $\,\,$ & $\,\,$\color{gr} 1.0490\color{black}  $\,\,$ \\
$\,\,$\color{gr} \color{blue}  1/9\color{black} $\,\,$ & $\,\,$\color{gr} 0.4143\color{black} $\,\,$ & $\,\,$\color{gr} 0.9533\color{black} $\,\,$ & $\,\,$ 1  $\,\,$ \\
\end{pmatrix},
\end{equation*}
\end{example}
\newpage
\begin{example}
\begin{equation*}
\mathbf{A} =
\begin{pmatrix}
$\,\,$ 1 $\,\,$ & $\,\,$8$\,\,$ & $\,\,$7$\,\,$ & $\,\,$9 $\,\,$ \\
$\,\,$ 1/8$\,\,$ & $\,\,$ 1 $\,\,$ & $\,\,$2$\,\,$ & $\,\,$6 $\,\,$ \\
$\,\,$ 1/7$\,\,$ & $\,\,$ 1/2$\,\,$ & $\,\,$ 1 $\,\,$ & $\,\,$2 $\,\,$ \\
$\,\,$ 1/9$\,\,$ & $\,\,$ 1/6$\,\,$ & $\,\,$ 1/2$\,\,$ & $\,\,$ 1  $\,\,$ \\
\end{pmatrix},
\qquad
\lambda_{\max} =
4.2512,
\qquad
CR = 0.0947
\end{equation*}

\begin{equation*}
\mathbf{w}^{cos} =
\begin{pmatrix}
0.666618\\
0.189446\\
\color{red} 0.092570\color{black} \\
0.051365
\end{pmatrix}\end{equation*}
\begin{equation*}
\left[ \frac{{w}^{cos}_i}{{w}^{cos}_j} \right] =
\begin{pmatrix}
$\,\,$ 1 $\,\,$ & $\,\,$3.5188$\,\,$ & $\,\,$\color{red} 7.2012\color{black} $\,\,$ & $\,\,$12.9780$\,\,$ \\
$\,\,$0.2842$\,\,$ & $\,\,$ 1 $\,\,$ & $\,\,$\color{red} 2.0465\color{black} $\,\,$ & $\,\,$3.6882  $\,\,$ \\
$\,\,$\color{red} 0.1389\color{black} $\,\,$ & $\,\,$\color{red} 0.4886\color{black} $\,\,$ & $\,\,$ 1 $\,\,$ & $\,\,$\color{red} 1.8022\color{black}  $\,\,$ \\
$\,\,$0.0771$\,\,$ & $\,\,$0.2711$\,\,$ & $\,\,$\color{red} 0.5549\color{black} $\,\,$ & $\,\,$ 1  $\,\,$ \\
\end{pmatrix},
\end{equation*}

\begin{equation*}
\mathbf{w}^{\prime} =
\begin{pmatrix}
0.665186\\
0.189039\\
0.094520\\
0.051255
\end{pmatrix} =
0.997851\cdot
\begin{pmatrix}
0.666618\\
0.189446\\
\color{gr} 0.094723\color{black} \\
0.051365
\end{pmatrix},
\end{equation*}
\begin{equation*}
\left[ \frac{{w}^{\prime}_i}{{w}^{\prime}_j} \right] =
\begin{pmatrix}
$\,\,$ 1 $\,\,$ & $\,\,$3.5188$\,\,$ & $\,\,$\color{gr} 7.0375\color{black} $\,\,$ & $\,\,$12.9780$\,\,$ \\
$\,\,$0.2842$\,\,$ & $\,\,$ 1 $\,\,$ & $\,\,$\color{gr} \color{blue} 2\color{black} $\,\,$ & $\,\,$3.6882  $\,\,$ \\
$\,\,$\color{gr} 0.1421\color{black} $\,\,$ & $\,\,$\color{gr} \color{blue}  1/2\color{black} $\,\,$ & $\,\,$ 1 $\,\,$ & $\,\,$\color{gr} 1.8441\color{black}  $\,\,$ \\
$\,\,$0.0771$\,\,$ & $\,\,$0.2711$\,\,$ & $\,\,$\color{gr} 0.5423\color{black} $\,\,$ & $\,\,$ 1  $\,\,$ \\
\end{pmatrix},
\end{equation*}
\end{example}
\newpage
\begin{example}
\begin{equation*}
\mathbf{A} =
\begin{pmatrix}
$\,\,$ 1 $\,\,$ & $\,\,$8$\,\,$ & $\,\,$7$\,\,$ & $\,\,$9 $\,\,$ \\
$\,\,$ 1/8$\,\,$ & $\,\,$ 1 $\,\,$ & $\,\,$3$\,\,$ & $\,\,$2 $\,\,$ \\
$\,\,$ 1/7$\,\,$ & $\,\,$ 1/3$\,\,$ & $\,\,$ 1 $\,\,$ & $\,\,$1 $\,\,$ \\
$\,\,$ 1/9$\,\,$ & $\,\,$ 1/2$\,\,$ & $\,\,$ 1 $\,\,$ & $\,\,$ 1  $\,\,$ \\
\end{pmatrix},
\qquad
\lambda_{\max} =
4.1330,
\qquad
CR = 0.0501
\end{equation*}

\begin{equation*}
\mathbf{w}^{cos} =
\begin{pmatrix}
0.696472\\
0.153922\\
0.075747\\
\color{red} 0.073860\color{black}
\end{pmatrix}\end{equation*}
\begin{equation*}
\left[ \frac{{w}^{cos}_i}{{w}^{cos}_j} \right] =
\begin{pmatrix}
$\,\,$ 1 $\,\,$ & $\,\,$4.5249$\,\,$ & $\,\,$9.1947$\,\,$ & $\,\,$\color{red} 9.4296\color{black} $\,\,$ \\
$\,\,$0.2210$\,\,$ & $\,\,$ 1 $\,\,$ & $\,\,$2.0321$\,\,$ & $\,\,$\color{red} 2.0840\color{black}   $\,\,$ \\
$\,\,$0.1088$\,\,$ & $\,\,$0.4921$\,\,$ & $\,\,$ 1 $\,\,$ & $\,\,$\color{red} 1.0255\color{black}  $\,\,$ \\
$\,\,$\color{red} 0.1060\color{black} $\,\,$ & $\,\,$\color{red} 0.4799\color{black} $\,\,$ & $\,\,$\color{red} 0.9751\color{black} $\,\,$ & $\,\,$ 1  $\,\,$ \\
\end{pmatrix},
\end{equation*}

\begin{equation*}
\mathbf{w}^{\prime} =
\begin{pmatrix}
0.695160\\
0.153632\\
0.075604\\
0.075604
\end{pmatrix} =
0.998117\cdot
\begin{pmatrix}
0.696472\\
0.153922\\
0.075747\\
\color{gr} 0.075747\color{black}
\end{pmatrix},
\end{equation*}
\begin{equation*}
\left[ \frac{{w}^{\prime}_i}{{w}^{\prime}_j} \right] =
\begin{pmatrix}
$\,\,$ 1 $\,\,$ & $\,\,$4.5249$\,\,$ & $\,\,$9.1947$\,\,$ & $\,\,$\color{gr} 9.1947\color{black} $\,\,$ \\
$\,\,$0.2210$\,\,$ & $\,\,$ 1 $\,\,$ & $\,\,$2.0321$\,\,$ & $\,\,$\color{gr} 2.0321\color{black}   $\,\,$ \\
$\,\,$0.1088$\,\,$ & $\,\,$0.4921$\,\,$ & $\,\,$ 1 $\,\,$ & $\,\,$\color{gr} \color{blue} 1\color{black}  $\,\,$ \\
$\,\,$\color{gr} 0.1088\color{black} $\,\,$ & $\,\,$\color{gr} 0.4921\color{black} $\,\,$ & $\,\,$\color{gr} \color{blue} 1\color{black} $\,\,$ & $\,\,$ 1  $\,\,$ \\
\end{pmatrix},
\end{equation*}
\end{example}
\newpage
\begin{example}
\begin{equation*}
\mathbf{A} =
\begin{pmatrix}
$\,\,$ 1 $\,\,$ & $\,\,$8$\,\,$ & $\,\,$7$\,\,$ & $\,\,$9 $\,\,$ \\
$\,\,$ 1/8$\,\,$ & $\,\,$ 1 $\,\,$ & $\,\,$4$\,\,$ & $\,\,$2 $\,\,$ \\
$\,\,$ 1/7$\,\,$ & $\,\,$ 1/4$\,\,$ & $\,\,$ 1 $\,\,$ & $\,\,$1 $\,\,$ \\
$\,\,$ 1/9$\,\,$ & $\,\,$ 1/2$\,\,$ & $\,\,$ 1 $\,\,$ & $\,\,$ 1  $\,\,$ \\
\end{pmatrix},
\qquad
\lambda_{\max} =
4.2064,
\qquad
CR = 0.0778
\end{equation*}

\begin{equation*}
\mathbf{w}^{cos} =
\begin{pmatrix}
0.684106\\
0.171435\\
0.072236\\
\color{red} 0.072223\color{black}
\end{pmatrix}\end{equation*}
\begin{equation*}
\left[ \frac{{w}^{cos}_i}{{w}^{cos}_j} \right] =
\begin{pmatrix}
$\,\,$ 1 $\,\,$ & $\,\,$3.9905$\,\,$ & $\,\,$9.4705$\,\,$ & $\,\,$\color{red} 9.4721\color{black} $\,\,$ \\
$\,\,$0.2506$\,\,$ & $\,\,$ 1 $\,\,$ & $\,\,$2.3733$\,\,$ & $\,\,$\color{red} 2.3737\color{black}   $\,\,$ \\
$\,\,$0.1056$\,\,$ & $\,\,$0.4214$\,\,$ & $\,\,$ 1 $\,\,$ & $\,\,$\color{red} 1.0002\color{black}  $\,\,$ \\
$\,\,$\color{red} 0.1056\color{black} $\,\,$ & $\,\,$\color{red} 0.4213\color{black} $\,\,$ & $\,\,$\color{red} 0.9998\color{black} $\,\,$ & $\,\,$ 1  $\,\,$ \\
\end{pmatrix},
\end{equation*}

\begin{equation*}
\mathbf{w}^{\prime} =
\begin{pmatrix}
0.684098\\
0.171433\\
0.072235\\
0.072235
\end{pmatrix} =
0.999987\cdot
\begin{pmatrix}
0.684106\\
0.171435\\
0.072236\\
\color{gr} 0.072236\color{black}
\end{pmatrix},
\end{equation*}
\begin{equation*}
\left[ \frac{{w}^{\prime}_i}{{w}^{\prime}_j} \right] =
\begin{pmatrix}
$\,\,$ 1 $\,\,$ & $\,\,$3.9905$\,\,$ & $\,\,$9.4705$\,\,$ & $\,\,$\color{gr} 9.4705\color{black} $\,\,$ \\
$\,\,$0.2506$\,\,$ & $\,\,$ 1 $\,\,$ & $\,\,$2.3733$\,\,$ & $\,\,$\color{gr} 2.3733\color{black}   $\,\,$ \\
$\,\,$0.1056$\,\,$ & $\,\,$0.4214$\,\,$ & $\,\,$ 1 $\,\,$ & $\,\,$\color{gr} \color{blue} 1\color{black}  $\,\,$ \\
$\,\,$\color{gr} 0.1056\color{black} $\,\,$ & $\,\,$\color{gr} 0.4214\color{black} $\,\,$ & $\,\,$\color{gr} \color{blue} 1\color{black} $\,\,$ & $\,\,$ 1  $\,\,$ \\
\end{pmatrix},
\end{equation*}
\end{example}
\newpage
\begin{example}
\begin{equation*}
\mathbf{A} =
\begin{pmatrix}
$\,\,$ 1 $\,\,$ & $\,\,$9$\,\,$ & $\,\,$4$\,\,$ & $\,\,$5 $\,\,$ \\
$\,\,$ 1/9$\,\,$ & $\,\,$ 1 $\,\,$ & $\,\,$1$\,\,$ & $\,\,$3 $\,\,$ \\
$\,\,$ 1/4$\,\,$ & $\,\,$ 1 $\,\,$ & $\,\,$ 1 $\,\,$ & $\,\,$2 $\,\,$ \\
$\,\,$ 1/5$\,\,$ & $\,\,$ 1/3$\,\,$ & $\,\,$ 1/2$\,\,$ & $\,\,$ 1  $\,\,$ \\
\end{pmatrix},
\qquad
\lambda_{\max} =
4.2541,
\qquad
CR = 0.0958
\end{equation*}

\begin{equation*}
\mathbf{w}^{cos} =
\begin{pmatrix}
0.611409\\
0.154779\\
\color{red} 0.150018\color{black} \\
0.083794
\end{pmatrix}\end{equation*}
\begin{equation*}
\left[ \frac{{w}^{cos}_i}{{w}^{cos}_j} \right] =
\begin{pmatrix}
$\,\,$ 1 $\,\,$ & $\,\,$3.9502$\,\,$ & $\,\,$\color{red} 4.0756\color{black} $\,\,$ & $\,\,$7.2966$\,\,$ \\
$\,\,$0.2532$\,\,$ & $\,\,$ 1 $\,\,$ & $\,\,$\color{red} 1.0317\color{black} $\,\,$ & $\,\,$1.8471  $\,\,$ \\
$\,\,$\color{red} 0.2454\color{black} $\,\,$ & $\,\,$\color{red} 0.9692\color{black} $\,\,$ & $\,\,$ 1 $\,\,$ & $\,\,$\color{red} 1.7903\color{black}  $\,\,$ \\
$\,\,$0.1371$\,\,$ & $\,\,$0.5414$\,\,$ & $\,\,$\color{red} 0.5586\color{black} $\,\,$ & $\,\,$ 1  $\,\,$ \\
\end{pmatrix},
\end{equation*}

\begin{equation*}
\mathbf{w}^{\prime} =
\begin{pmatrix}
0.609681\\
0.154342\\
0.152420\\
0.083557
\end{pmatrix} =
0.997174\cdot
\begin{pmatrix}
0.611409\\
0.154779\\
\color{gr} 0.152852\color{black} \\
0.083794
\end{pmatrix},
\end{equation*}
\begin{equation*}
\left[ \frac{{w}^{\prime}_i}{{w}^{\prime}_j} \right] =
\begin{pmatrix}
$\,\,$ 1 $\,\,$ & $\,\,$3.9502$\,\,$ & $\,\,$\color{gr} \color{blue} 4\color{black} $\,\,$ & $\,\,$7.2966$\,\,$ \\
$\,\,$0.2532$\,\,$ & $\,\,$ 1 $\,\,$ & $\,\,$\color{gr} 1.0126\color{black} $\,\,$ & $\,\,$1.8471  $\,\,$ \\
$\,\,$\color{gr} \color{blue}  1/4\color{black} $\,\,$ & $\,\,$\color{gr} 0.9876\color{black} $\,\,$ & $\,\,$ 1 $\,\,$ & $\,\,$\color{gr} 1.8241\color{black}  $\,\,$ \\
$\,\,$0.1371$\,\,$ & $\,\,$0.5414$\,\,$ & $\,\,$\color{gr} 0.5482\color{black} $\,\,$ & $\,\,$ 1  $\,\,$ \\
\end{pmatrix},
\end{equation*}
\end{example}
\newpage
\begin{example}
\begin{equation*}
\mathbf{A} =
\begin{pmatrix}
$\,\,$ 1 $\,\,$ & $\,\,$9$\,\,$ & $\,\,$4$\,\,$ & $\,\,$6 $\,\,$ \\
$\,\,$ 1/9$\,\,$ & $\,\,$ 1 $\,\,$ & $\,\,$1$\,\,$ & $\,\,$3 $\,\,$ \\
$\,\,$ 1/4$\,\,$ & $\,\,$ 1 $\,\,$ & $\,\,$ 1 $\,\,$ & $\,\,$2 $\,\,$ \\
$\,\,$ 1/6$\,\,$ & $\,\,$ 1/3$\,\,$ & $\,\,$ 1/2$\,\,$ & $\,\,$ 1  $\,\,$ \\
\end{pmatrix},
\qquad
\lambda_{\max} =
4.2052,
\qquad
CR = 0.0774
\end{equation*}

\begin{equation*}
\mathbf{w}^{cos} =
\begin{pmatrix}
0.629473\\
0.147692\\
\color{red} 0.146145\color{black} \\
0.076689
\end{pmatrix}\end{equation*}
\begin{equation*}
\left[ \frac{{w}^{cos}_i}{{w}^{cos}_j} \right] =
\begin{pmatrix}
$\,\,$ 1 $\,\,$ & $\,\,$4.2621$\,\,$ & $\,\,$\color{red} 4.3072\color{black} $\,\,$ & $\,\,$8.2081$\,\,$ \\
$\,\,$0.2346$\,\,$ & $\,\,$ 1 $\,\,$ & $\,\,$\color{red} 1.0106\color{black} $\,\,$ & $\,\,$1.9259  $\,\,$ \\
$\,\,$\color{red} 0.2322\color{black} $\,\,$ & $\,\,$\color{red} 0.9895\color{black} $\,\,$ & $\,\,$ 1 $\,\,$ & $\,\,$\color{red} 1.9057\color{black}  $\,\,$ \\
$\,\,$0.1218$\,\,$ & $\,\,$0.5193$\,\,$ & $\,\,$\color{red} 0.5247\color{black} $\,\,$ & $\,\,$ 1  $\,\,$ \\
\end{pmatrix},
\end{equation*}

\begin{equation*}
\mathbf{w}^{\prime} =
\begin{pmatrix}
0.628501\\
0.147464\\
0.147464\\
0.076571
\end{pmatrix} =
0.998455\cdot
\begin{pmatrix}
0.629473\\
0.147692\\
\color{gr} 0.147692\color{black} \\
0.076689
\end{pmatrix},
\end{equation*}
\begin{equation*}
\left[ \frac{{w}^{\prime}_i}{{w}^{\prime}_j} \right] =
\begin{pmatrix}
$\,\,$ 1 $\,\,$ & $\,\,$4.2621$\,\,$ & $\,\,$\color{gr} 4.2621\color{black} $\,\,$ & $\,\,$8.2081$\,\,$ \\
$\,\,$0.2346$\,\,$ & $\,\,$ 1 $\,\,$ & $\,\,$\color{gr} \color{blue} 1\color{black} $\,\,$ & $\,\,$1.9259  $\,\,$ \\
$\,\,$\color{gr} 0.2346\color{black} $\,\,$ & $\,\,$\color{gr} \color{blue} 1\color{black} $\,\,$ & $\,\,$ 1 $\,\,$ & $\,\,$\color{gr} 1.9259\color{black}  $\,\,$ \\
$\,\,$0.1218$\,\,$ & $\,\,$0.5193$\,\,$ & $\,\,$\color{gr} 0.5193\color{black} $\,\,$ & $\,\,$ 1  $\,\,$ \\
\end{pmatrix},
\end{equation*}
\end{example}
\newpage
\begin{example}
\begin{equation*}
\mathbf{A} =
\begin{pmatrix}
$\,\,$ 1 $\,\,$ & $\,\,$9$\,\,$ & $\,\,$4$\,\,$ & $\,\,$9 $\,\,$ \\
$\,\,$ 1/9$\,\,$ & $\,\,$ 1 $\,\,$ & $\,\,$1$\,\,$ & $\,\,$5 $\,\,$ \\
$\,\,$ 1/4$\,\,$ & $\,\,$ 1 $\,\,$ & $\,\,$ 1 $\,\,$ & $\,\,$3 $\,\,$ \\
$\,\,$ 1/9$\,\,$ & $\,\,$ 1/5$\,\,$ & $\,\,$ 1/3$\,\,$ & $\,\,$ 1  $\,\,$ \\
\end{pmatrix},
\qquad
\lambda_{\max} =
4.2339,
\qquad
CR = 0.0882
\end{equation*}

\begin{equation*}
\mathbf{w}^{cos} =
\begin{pmatrix}
0.641714\\
0.157835\\
\color{red} 0.148785\color{black} \\
0.051665
\end{pmatrix}\end{equation*}
\begin{equation*}
\left[ \frac{{w}^{cos}_i}{{w}^{cos}_j} \right] =
\begin{pmatrix}
$\,\,$ 1 $\,\,$ & $\,\,$4.0657$\,\,$ & $\,\,$\color{red} 4.3130\color{black} $\,\,$ & $\,\,$12.4206$\,\,$ \\
$\,\,$0.2460$\,\,$ & $\,\,$ 1 $\,\,$ & $\,\,$\color{red} 1.0608\color{black} $\,\,$ & $\,\,$3.0549  $\,\,$ \\
$\,\,$\color{red} 0.2319\color{black} $\,\,$ & $\,\,$\color{red} 0.9427\color{black} $\,\,$ & $\,\,$ 1 $\,\,$ & $\,\,$\color{red} 2.8798\color{black}  $\,\,$ \\
$\,\,$0.0805$\,\,$ & $\,\,$0.3273$\,\,$ & $\,\,$\color{red} 0.3472\color{black} $\,\,$ & $\,\,$ 1  $\,\,$ \\
\end{pmatrix},
\end{equation*}

\begin{equation*}
\mathbf{w}^{\prime} =
\begin{pmatrix}
0.637753\\
0.156861\\
0.154040\\
0.051347
\end{pmatrix} =
0.993827\cdot
\begin{pmatrix}
0.641714\\
0.157835\\
\color{gr} 0.154996\color{black} \\
0.051665
\end{pmatrix},
\end{equation*}
\begin{equation*}
\left[ \frac{{w}^{\prime}_i}{{w}^{\prime}_j} \right] =
\begin{pmatrix}
$\,\,$ 1 $\,\,$ & $\,\,$4.0657$\,\,$ & $\,\,$\color{gr} 4.1402\color{black} $\,\,$ & $\,\,$12.4206$\,\,$ \\
$\,\,$0.2460$\,\,$ & $\,\,$ 1 $\,\,$ & $\,\,$\color{gr} 1.0183\color{black} $\,\,$ & $\,\,$3.0549  $\,\,$ \\
$\,\,$\color{gr} 0.2415\color{black} $\,\,$ & $\,\,$\color{gr} 0.9820\color{black} $\,\,$ & $\,\,$ 1 $\,\,$ & $\,\,$\color{gr} \color{blue} 3\color{black}  $\,\,$ \\
$\,\,$0.0805$\,\,$ & $\,\,$0.3273$\,\,$ & $\,\,$\color{gr} \color{blue}  1/3\color{black} $\,\,$ & $\,\,$ 1  $\,\,$ \\
\end{pmatrix},
\end{equation*}
\end{example}
\newpage
\begin{example}
\begin{equation*}
\mathbf{A} =
\begin{pmatrix}
$\,\,$ 1 $\,\,$ & $\,\,$9$\,\,$ & $\,\,$5$\,\,$ & $\,\,$7 $\,\,$ \\
$\,\,$ 1/9$\,\,$ & $\,\,$ 1 $\,\,$ & $\,\,$1$\,\,$ & $\,\,$3 $\,\,$ \\
$\,\,$ 1/5$\,\,$ & $\,\,$ 1 $\,\,$ & $\,\,$ 1 $\,\,$ & $\,\,$2 $\,\,$ \\
$\,\,$ 1/7$\,\,$ & $\,\,$ 1/3$\,\,$ & $\,\,$ 1/2$\,\,$ & $\,\,$ 1  $\,\,$ \\
\end{pmatrix},
\qquad
\lambda_{\max} =
4.1583,
\qquad
CR = 0.0597
\end{equation*}

\begin{equation*}
\mathbf{w}^{cos} =
\begin{pmatrix}
0.662837\\
0.137462\\
\color{red} 0.130442\color{black} \\
0.069258
\end{pmatrix}\end{equation*}
\begin{equation*}
\left[ \frac{{w}^{cos}_i}{{w}^{cos}_j} \right] =
\begin{pmatrix}
$\,\,$ 1 $\,\,$ & $\,\,$4.8220$\,\,$ & $\,\,$\color{red} 5.0815\color{black} $\,\,$ & $\,\,$9.5706$\,\,$ \\
$\,\,$0.2074$\,\,$ & $\,\,$ 1 $\,\,$ & $\,\,$\color{red} 1.0538\color{black} $\,\,$ & $\,\,$1.9848  $\,\,$ \\
$\,\,$\color{red} 0.1968\color{black} $\,\,$ & $\,\,$\color{red} 0.9489\color{black} $\,\,$ & $\,\,$ 1 $\,\,$ & $\,\,$\color{red} 1.8834\color{black}  $\,\,$ \\
$\,\,$0.1045$\,\,$ & $\,\,$0.5038$\,\,$ & $\,\,$\color{red} 0.5309\color{black} $\,\,$ & $\,\,$ 1  $\,\,$ \\
\end{pmatrix},
\end{equation*}

\begin{equation*}
\mathbf{w}^{\prime} =
\begin{pmatrix}
0.661432\\
0.137171\\
0.132286\\
0.069111
\end{pmatrix} =
0.997879\cdot
\begin{pmatrix}
0.662837\\
0.137462\\
\color{gr} 0.132567\color{black} \\
0.069258
\end{pmatrix},
\end{equation*}
\begin{equation*}
\left[ \frac{{w}^{\prime}_i}{{w}^{\prime}_j} \right] =
\begin{pmatrix}
$\,\,$ 1 $\,\,$ & $\,\,$4.8220$\,\,$ & $\,\,$\color{gr} \color{blue} 5\color{black} $\,\,$ & $\,\,$9.5706$\,\,$ \\
$\,\,$0.2074$\,\,$ & $\,\,$ 1 $\,\,$ & $\,\,$\color{gr} 1.0369\color{black} $\,\,$ & $\,\,$1.9848  $\,\,$ \\
$\,\,$\color{gr} \color{blue}  1/5\color{black} $\,\,$ & $\,\,$\color{gr} 0.9644\color{black} $\,\,$ & $\,\,$ 1 $\,\,$ & $\,\,$\color{gr} 1.9141\color{black}  $\,\,$ \\
$\,\,$0.1045$\,\,$ & $\,\,$0.5038$\,\,$ & $\,\,$\color{gr} 0.5224\color{black} $\,\,$ & $\,\,$ 1  $\,\,$ \\
\end{pmatrix},
\end{equation*}
\end{example}
\newpage
\begin{example}
\begin{equation*}
\mathbf{A} =
\begin{pmatrix}
$\,\,$ 1 $\,\,$ & $\,\,$9$\,\,$ & $\,\,$5$\,\,$ & $\,\,$7 $\,\,$ \\
$\,\,$ 1/9$\,\,$ & $\,\,$ 1 $\,\,$ & $\,\,$1$\,\,$ & $\,\,$4 $\,\,$ \\
$\,\,$ 1/5$\,\,$ & $\,\,$ 1 $\,\,$ & $\,\,$ 1 $\,\,$ & $\,\,$2 $\,\,$ \\
$\,\,$ 1/7$\,\,$ & $\,\,$ 1/4$\,\,$ & $\,\,$ 1/2$\,\,$ & $\,\,$ 1  $\,\,$ \\
\end{pmatrix},
\qquad
\lambda_{\max} =
4.2371,
\qquad
CR = 0.0894
\end{equation*}

\begin{equation*}
\mathbf{w}^{cos} =
\begin{pmatrix}
0.651936\\
0.154248\\
\color{red} 0.127593\color{black} \\
0.066223
\end{pmatrix}\end{equation*}
\begin{equation*}
\left[ \frac{{w}^{cos}_i}{{w}^{cos}_j} \right] =
\begin{pmatrix}
$\,\,$ 1 $\,\,$ & $\,\,$4.2265$\,\,$ & $\,\,$\color{red} 5.1095\color{black} $\,\,$ & $\,\,$9.8446$\,\,$ \\
$\,\,$0.2366$\,\,$ & $\,\,$ 1 $\,\,$ & $\,\,$\color{red} 1.2089\color{black} $\,\,$ & $\,\,$2.3292  $\,\,$ \\
$\,\,$\color{red} 0.1957\color{black} $\,\,$ & $\,\,$\color{red} 0.8272\color{black} $\,\,$ & $\,\,$ 1 $\,\,$ & $\,\,$\color{red} 1.9267\color{black}  $\,\,$ \\
$\,\,$0.1016$\,\,$ & $\,\,$0.4293$\,\,$ & $\,\,$\color{red} 0.5190\color{black} $\,\,$ & $\,\,$ 1  $\,\,$ \\
\end{pmatrix},
\end{equation*}

\begin{equation*}
\mathbf{w}^{\prime} =
\begin{pmatrix}
0.650119\\
0.153818\\
0.130024\\
0.066038
\end{pmatrix} =
0.997213\cdot
\begin{pmatrix}
0.651936\\
0.154248\\
\color{gr} 0.130387\color{black} \\
0.066223
\end{pmatrix},
\end{equation*}
\begin{equation*}
\left[ \frac{{w}^{\prime}_i}{{w}^{\prime}_j} \right] =
\begin{pmatrix}
$\,\,$ 1 $\,\,$ & $\,\,$4.2265$\,\,$ & $\,\,$\color{gr} \color{blue} 5\color{black} $\,\,$ & $\,\,$9.8446$\,\,$ \\
$\,\,$0.2366$\,\,$ & $\,\,$ 1 $\,\,$ & $\,\,$\color{gr} 1.1830\color{black} $\,\,$ & $\,\,$2.3292  $\,\,$ \\
$\,\,$\color{gr} \color{blue}  1/5\color{black} $\,\,$ & $\,\,$\color{gr} 0.8453\color{black} $\,\,$ & $\,\,$ 1 $\,\,$ & $\,\,$\color{gr} 1.9689\color{black}  $\,\,$ \\
$\,\,$0.1016$\,\,$ & $\,\,$0.4293$\,\,$ & $\,\,$\color{gr} 0.5079\color{black} $\,\,$ & $\,\,$ 1  $\,\,$ \\
\end{pmatrix},
\end{equation*}
\end{example}
\newpage
\begin{example}
\begin{equation*}
\mathbf{A} =
\begin{pmatrix}
$\,\,$ 1 $\,\,$ & $\,\,$9$\,\,$ & $\,\,$5$\,\,$ & $\,\,$8 $\,\,$ \\
$\,\,$ 1/9$\,\,$ & $\,\,$ 1 $\,\,$ & $\,\,$1$\,\,$ & $\,\,$3 $\,\,$ \\
$\,\,$ 1/5$\,\,$ & $\,\,$ 1 $\,\,$ & $\,\,$ 1 $\,\,$ & $\,\,$2 $\,\,$ \\
$\,\,$ 1/8$\,\,$ & $\,\,$ 1/3$\,\,$ & $\,\,$ 1/2$\,\,$ & $\,\,$ 1  $\,\,$ \\
\end{pmatrix},
\qquad
\lambda_{\max} =
4.1305,
\qquad
CR = 0.0492
\end{equation*}

\begin{equation*}
\mathbf{w}^{cos} =
\begin{pmatrix}
0.675277\\
0.132359\\
\color{red} 0.127546\color{black} \\
0.064818
\end{pmatrix}\end{equation*}
\begin{equation*}
\left[ \frac{{w}^{cos}_i}{{w}^{cos}_j} \right] =
\begin{pmatrix}
$\,\,$ 1 $\,\,$ & $\,\,$5.1019$\,\,$ & $\,\,$\color{red} 5.2944\color{black} $\,\,$ & $\,\,$10.4180$\,\,$ \\
$\,\,$0.1960$\,\,$ & $\,\,$ 1 $\,\,$ & $\,\,$\color{red} 1.0377\color{black} $\,\,$ & $\,\,$2.0420  $\,\,$ \\
$\,\,$\color{red} 0.1889\color{black} $\,\,$ & $\,\,$\color{red} 0.9636\color{black} $\,\,$ & $\,\,$ 1 $\,\,$ & $\,\,$\color{red} 1.9678\color{black}  $\,\,$ \\
$\,\,$0.0960$\,\,$ & $\,\,$0.4897$\,\,$ & $\,\,$\color{red} 0.5082\color{black} $\,\,$ & $\,\,$ 1  $\,\,$ \\
\end{pmatrix},
\end{equation*}

\begin{equation*}
\mathbf{w}^{\prime} =
\begin{pmatrix}
0.673868\\
0.132083\\
0.129366\\
0.064683
\end{pmatrix} =
0.997914\cdot
\begin{pmatrix}
0.675277\\
0.132359\\
\color{gr} 0.129636\color{black} \\
0.064818
\end{pmatrix},
\end{equation*}
\begin{equation*}
\left[ \frac{{w}^{\prime}_i}{{w}^{\prime}_j} \right] =
\begin{pmatrix}
$\,\,$ 1 $\,\,$ & $\,\,$5.1019$\,\,$ & $\,\,$\color{gr} 5.2090\color{black} $\,\,$ & $\,\,$10.4180$\,\,$ \\
$\,\,$0.1960$\,\,$ & $\,\,$ 1 $\,\,$ & $\,\,$\color{gr} 1.0210\color{black} $\,\,$ & $\,\,$2.0420  $\,\,$ \\
$\,\,$\color{gr} 0.1920\color{black} $\,\,$ & $\,\,$\color{gr} 0.9794\color{black} $\,\,$ & $\,\,$ 1 $\,\,$ & $\,\,$\color{gr} \color{blue} 2\color{black}  $\,\,$ \\
$\,\,$0.0960$\,\,$ & $\,\,$0.4897$\,\,$ & $\,\,$\color{gr} \color{blue}  1/2\color{black} $\,\,$ & $\,\,$ 1  $\,\,$ \\
\end{pmatrix},
\end{equation*}
\end{example}
\newpage
\begin{example}
\begin{equation*}
\mathbf{A} =
\begin{pmatrix}
$\,\,$ 1 $\,\,$ & $\,\,$9$\,\,$ & $\,\,$5$\,\,$ & $\,\,$8 $\,\,$ \\
$\,\,$ 1/9$\,\,$ & $\,\,$ 1 $\,\,$ & $\,\,$3$\,\,$ & $\,\,$2 $\,\,$ \\
$\,\,$ 1/5$\,\,$ & $\,\,$ 1/3$\,\,$ & $\,\,$ 1 $\,\,$ & $\,\,$1 $\,\,$ \\
$\,\,$ 1/8$\,\,$ & $\,\,$ 1/2$\,\,$ & $\,\,$ 1 $\,\,$ & $\,\,$ 1  $\,\,$ \\
\end{pmatrix},
\qquad
\lambda_{\max} =
4.2541,
\qquad
CR = 0.0958
\end{equation*}

\begin{equation*}
\mathbf{w}^{cos} =
\begin{pmatrix}
0.659512\\
0.168314\\
0.090855\\
\color{red} 0.081319\color{black}
\end{pmatrix}\end{equation*}
\begin{equation*}
\left[ \frac{{w}^{cos}_i}{{w}^{cos}_j} \right] =
\begin{pmatrix}
$\,\,$ 1 $\,\,$ & $\,\,$3.9183$\,\,$ & $\,\,$7.2589$\,\,$ & $\,\,$\color{red} 8.1102\color{black} $\,\,$ \\
$\,\,$0.2552$\,\,$ & $\,\,$ 1 $\,\,$ & $\,\,$1.8526$\,\,$ & $\,\,$\color{red} 2.0698\color{black}   $\,\,$ \\
$\,\,$0.1378$\,\,$ & $\,\,$0.5398$\,\,$ & $\,\,$ 1 $\,\,$ & $\,\,$\color{red} 1.1173\color{black}  $\,\,$ \\
$\,\,$\color{red} 0.1233\color{black} $\,\,$ & $\,\,$\color{red} 0.4831\color{black} $\,\,$ & $\,\,$\color{red} 0.8950\color{black} $\,\,$ & $\,\,$ 1  $\,\,$ \\
\end{pmatrix},
\end{equation*}

\begin{equation*}
\mathbf{w}^{\prime} =
\begin{pmatrix}
0.658774\\
0.168126\\
0.090754\\
0.082347
\end{pmatrix} =
0.998881\cdot
\begin{pmatrix}
0.659512\\
0.168314\\
0.090855\\
\color{gr} 0.082439\color{black}
\end{pmatrix},
\end{equation*}
\begin{equation*}
\left[ \frac{{w}^{\prime}_i}{{w}^{\prime}_j} \right] =
\begin{pmatrix}
$\,\,$ 1 $\,\,$ & $\,\,$3.9183$\,\,$ & $\,\,$7.2589$\,\,$ & $\,\,$\color{gr} \color{blue} 8\color{black} $\,\,$ \\
$\,\,$0.2552$\,\,$ & $\,\,$ 1 $\,\,$ & $\,\,$1.8526$\,\,$ & $\,\,$\color{gr} 2.0417\color{black}   $\,\,$ \\
$\,\,$0.1378$\,\,$ & $\,\,$0.5398$\,\,$ & $\,\,$ 1 $\,\,$ & $\,\,$\color{gr} 1.1021\color{black}  $\,\,$ \\
$\,\,$\color{gr} \color{blue}  1/8\color{black} $\,\,$ & $\,\,$\color{gr} 0.4898\color{black} $\,\,$ & $\,\,$\color{gr} 0.9074\color{black} $\,\,$ & $\,\,$ 1  $\,\,$ \\
\end{pmatrix},
\end{equation*}
\end{example}
\newpage
\begin{example}
\begin{equation*}
\mathbf{A} =
\begin{pmatrix}
$\,\,$ 1 $\,\,$ & $\,\,$9$\,\,$ & $\,\,$6$\,\,$ & $\,\,$8 $\,\,$ \\
$\,\,$ 1/9$\,\,$ & $\,\,$ 1 $\,\,$ & $\,\,$3$\,\,$ & $\,\,$2 $\,\,$ \\
$\,\,$ 1/6$\,\,$ & $\,\,$ 1/3$\,\,$ & $\,\,$ 1 $\,\,$ & $\,\,$1 $\,\,$ \\
$\,\,$ 1/8$\,\,$ & $\,\,$ 1/2$\,\,$ & $\,\,$ 1 $\,\,$ & $\,\,$ 1  $\,\,$ \\
\end{pmatrix},
\qquad
\lambda_{\max} =
4.2052,
\qquad
CR = 0.0774
\end{equation*}

\begin{equation*}
\mathbf{w}^{cos} =
\begin{pmatrix}
0.678187\\
0.159880\\
0.082948\\
\color{red} 0.078985\color{black}
\end{pmatrix}\end{equation*}
\begin{equation*}
\left[ \frac{{w}^{cos}_i}{{w}^{cos}_j} \right] =
\begin{pmatrix}
$\,\,$ 1 $\,\,$ & $\,\,$4.2418$\,\,$ & $\,\,$8.1760$\,\,$ & $\,\,$\color{red} 8.5863\color{black} $\,\,$ \\
$\,\,$0.2357$\,\,$ & $\,\,$ 1 $\,\,$ & $\,\,$1.9275$\,\,$ & $\,\,$\color{red} 2.0242\color{black}   $\,\,$ \\
$\,\,$0.1223$\,\,$ & $\,\,$0.5188$\,\,$ & $\,\,$ 1 $\,\,$ & $\,\,$\color{red} 1.0502\color{black}  $\,\,$ \\
$\,\,$\color{red} 0.1165\color{black} $\,\,$ & $\,\,$\color{red} 0.4940\color{black} $\,\,$ & $\,\,$\color{red} 0.9522\color{black} $\,\,$ & $\,\,$ 1  $\,\,$ \\
\end{pmatrix},
\end{equation*}

\begin{equation*}
\mathbf{w}^{\prime} =
\begin{pmatrix}
0.677539\\
0.159727\\
0.082869\\
0.079864
\end{pmatrix} =
0.999046\cdot
\begin{pmatrix}
0.678187\\
0.159880\\
0.082948\\
\color{gr} 0.079940\color{black}
\end{pmatrix},
\end{equation*}
\begin{equation*}
\left[ \frac{{w}^{\prime}_i}{{w}^{\prime}_j} \right] =
\begin{pmatrix}
$\,\,$ 1 $\,\,$ & $\,\,$4.2418$\,\,$ & $\,\,$8.1760$\,\,$ & $\,\,$\color{gr} 8.4837\color{black} $\,\,$ \\
$\,\,$0.2357$\,\,$ & $\,\,$ 1 $\,\,$ & $\,\,$1.9275$\,\,$ & $\,\,$\color{gr} \color{blue} 2\color{black}   $\,\,$ \\
$\,\,$0.1223$\,\,$ & $\,\,$0.5188$\,\,$ & $\,\,$ 1 $\,\,$ & $\,\,$\color{gr} 1.0376\color{black}  $\,\,$ \\
$\,\,$\color{gr} 0.1179\color{black} $\,\,$ & $\,\,$\color{gr} \color{blue}  1/2\color{black} $\,\,$ & $\,\,$\color{gr} 0.9637\color{black} $\,\,$ & $\,\,$ 1  $\,\,$ \\
\end{pmatrix},
\end{equation*}
\end{example}
\newpage
\begin{example}
\begin{equation*}
\mathbf{A} =
\begin{pmatrix}
$\,\,$ 1 $\,\,$ & $\,\,$9$\,\,$ & $\,\,$6$\,\,$ & $\,\,$9 $\,\,$ \\
$\,\,$ 1/9$\,\,$ & $\,\,$ 1 $\,\,$ & $\,\,$1$\,\,$ & $\,\,$3 $\,\,$ \\
$\,\,$ 1/6$\,\,$ & $\,\,$ 1 $\,\,$ & $\,\,$ 1 $\,\,$ & $\,\,$2 $\,\,$ \\
$\,\,$ 1/9$\,\,$ & $\,\,$ 1/3$\,\,$ & $\,\,$ 1/2$\,\,$ & $\,\,$ 1  $\,\,$ \\
\end{pmatrix},
\qquad
\lambda_{\max} =
4.1031,
\qquad
CR = 0.0389
\end{equation*}

\begin{equation*}
\mathbf{w}^{cos} =
\begin{pmatrix}
0.699686\\
0.124762\\
\color{red} 0.115971\color{black} \\
0.059581
\end{pmatrix}\end{equation*}
\begin{equation*}
\left[ \frac{{w}^{cos}_i}{{w}^{cos}_j} \right] =
\begin{pmatrix}
$\,\,$ 1 $\,\,$ & $\,\,$5.6081$\,\,$ & $\,\,$\color{red} 6.0333\color{black} $\,\,$ & $\,\,$11.7434$\,\,$ \\
$\,\,$0.1783$\,\,$ & $\,\,$ 1 $\,\,$ & $\,\,$\color{red} 1.0758\color{black} $\,\,$ & $\,\,$2.0940  $\,\,$ \\
$\,\,$\color{red} 0.1657\color{black} $\,\,$ & $\,\,$\color{red} 0.9295\color{black} $\,\,$ & $\,\,$ 1 $\,\,$ & $\,\,$\color{red} 1.9464\color{black}  $\,\,$ \\
$\,\,$0.0852$\,\,$ & $\,\,$0.4776$\,\,$ & $\,\,$\color{red} 0.5138\color{black} $\,\,$ & $\,\,$ 1  $\,\,$ \\
\end{pmatrix},
\end{equation*}

\begin{equation*}
\mathbf{w}^{\prime} =
\begin{pmatrix}
0.699236\\
0.124682\\
0.116539\\
0.059543
\end{pmatrix} =
0.999357\cdot
\begin{pmatrix}
0.699686\\
0.124762\\
\color{gr} 0.116614\color{black} \\
0.059581
\end{pmatrix},
\end{equation*}
\begin{equation*}
\left[ \frac{{w}^{\prime}_i}{{w}^{\prime}_j} \right] =
\begin{pmatrix}
$\,\,$ 1 $\,\,$ & $\,\,$5.6081$\,\,$ & $\,\,$\color{gr} \color{blue} 6\color{black} $\,\,$ & $\,\,$11.7434$\,\,$ \\
$\,\,$0.1783$\,\,$ & $\,\,$ 1 $\,\,$ & $\,\,$\color{gr} 1.0699\color{black} $\,\,$ & $\,\,$2.0940  $\,\,$ \\
$\,\,$\color{gr} \color{blue}  1/6\color{black} $\,\,$ & $\,\,$\color{gr} 0.9347\color{black} $\,\,$ & $\,\,$ 1 $\,\,$ & $\,\,$\color{gr} 1.9572\color{black}  $\,\,$ \\
$\,\,$0.0852$\,\,$ & $\,\,$0.4776$\,\,$ & $\,\,$\color{gr} 0.5109\color{black} $\,\,$ & $\,\,$ 1  $\,\,$ \\
\end{pmatrix},
\end{equation*}
\end{example}
\newpage
\begin{example}
\begin{equation*}
\mathbf{A} =
\begin{pmatrix}
$\,\,$ 1 $\,\,$ & $\,\,$9$\,\,$ & $\,\,$6$\,\,$ & $\,\,$9 $\,\,$ \\
$\,\,$ 1/9$\,\,$ & $\,\,$ 1 $\,\,$ & $\,\,$3$\,\,$ & $\,\,$2 $\,\,$ \\
$\,\,$ 1/6$\,\,$ & $\,\,$ 1/3$\,\,$ & $\,\,$ 1 $\,\,$ & $\,\,$1 $\,\,$ \\
$\,\,$ 1/9$\,\,$ & $\,\,$ 1/2$\,\,$ & $\,\,$ 1 $\,\,$ & $\,\,$ 1  $\,\,$ \\
\end{pmatrix},
\qquad
\lambda_{\max} =
4.1990,
\qquad
CR = 0.0750
\end{equation*}

\begin{equation*}
\mathbf{w}^{cos} =
\begin{pmatrix}
0.686384\\
0.156968\\
0.081551\\
\color{red} 0.075098\color{black}
\end{pmatrix}\end{equation*}
\begin{equation*}
\left[ \frac{{w}^{cos}_i}{{w}^{cos}_j} \right] =
\begin{pmatrix}
$\,\,$ 1 $\,\,$ & $\,\,$4.3728$\,\,$ & $\,\,$8.4166$\,\,$ & $\,\,$\color{red} 9.1399\color{black} $\,\,$ \\
$\,\,$0.2287$\,\,$ & $\,\,$ 1 $\,\,$ & $\,\,$1.9248$\,\,$ & $\,\,$\color{red} 2.0902\color{black}   $\,\,$ \\
$\,\,$0.1188$\,\,$ & $\,\,$0.5195$\,\,$ & $\,\,$ 1 $\,\,$ & $\,\,$\color{red} 1.0859\color{black}  $\,\,$ \\
$\,\,$\color{red} 0.1094\color{black} $\,\,$ & $\,\,$\color{red} 0.4784\color{black} $\,\,$ & $\,\,$\color{red} 0.9209\color{black} $\,\,$ & $\,\,$ 1  $\,\,$ \\
\end{pmatrix},
\end{equation*}

\begin{equation*}
\mathbf{w}^{\prime} =
\begin{pmatrix}
0.685583\\
0.156785\\
0.081456\\
0.076176
\end{pmatrix} =
0.998834\cdot
\begin{pmatrix}
0.686384\\
0.156968\\
0.081551\\
\color{gr} 0.076265\color{black}
\end{pmatrix},
\end{equation*}
\begin{equation*}
\left[ \frac{{w}^{\prime}_i}{{w}^{\prime}_j} \right] =
\begin{pmatrix}
$\,\,$ 1 $\,\,$ & $\,\,$4.3728$\,\,$ & $\,\,$8.4166$\,\,$ & $\,\,$\color{gr} \color{blue} 9\color{black} $\,\,$ \\
$\,\,$0.2287$\,\,$ & $\,\,$ 1 $\,\,$ & $\,\,$1.9248$\,\,$ & $\,\,$\color{gr} 2.0582\color{black}   $\,\,$ \\
$\,\,$0.1188$\,\,$ & $\,\,$0.5195$\,\,$ & $\,\,$ 1 $\,\,$ & $\,\,$\color{gr} 1.0693\color{black}  $\,\,$ \\
$\,\,$\color{gr} \color{blue}  1/9\color{black} $\,\,$ & $\,\,$\color{gr} 0.4859\color{black} $\,\,$ & $\,\,$\color{gr} 0.9352\color{black} $\,\,$ & $\,\,$ 1  $\,\,$ \\
\end{pmatrix},
\end{equation*}
\end{example}
\newpage
\begin{example}
\begin{equation*}
\mathbf{A} =
\begin{pmatrix}
$\,\,$ 1 $\,\,$ & $\,\,$9$\,\,$ & $\,\,$7$\,\,$ & $\,\,$9 $\,\,$ \\
$\,\,$ 1/9$\,\,$ & $\,\,$ 1 $\,\,$ & $\,\,$3$\,\,$ & $\,\,$2 $\,\,$ \\
$\,\,$ 1/7$\,\,$ & $\,\,$ 1/3$\,\,$ & $\,\,$ 1 $\,\,$ & $\,\,$1 $\,\,$ \\
$\,\,$ 1/9$\,\,$ & $\,\,$ 1/2$\,\,$ & $\,\,$ 1 $\,\,$ & $\,\,$ 1  $\,\,$ \\
\end{pmatrix},
\qquad
\lambda_{\max} =
4.1628,
\qquad
CR = 0.0614
\end{equation*}

\begin{equation*}
\mathbf{w}^{cos} =
\begin{pmatrix}
0.701483\\
0.149964\\
0.075436\\
\color{red} 0.073117\color{black}
\end{pmatrix}\end{equation*}
\begin{equation*}
\left[ \frac{{w}^{cos}_i}{{w}^{cos}_j} \right] =
\begin{pmatrix}
$\,\,$ 1 $\,\,$ & $\,\,$4.6777$\,\,$ & $\,\,$9.2991$\,\,$ & $\,\,$\color{red} 9.5940\color{black} $\,\,$ \\
$\,\,$0.2138$\,\,$ & $\,\,$ 1 $\,\,$ & $\,\,$1.9880$\,\,$ & $\,\,$\color{red} 2.0510\color{black}   $\,\,$ \\
$\,\,$0.1075$\,\,$ & $\,\,$0.5030$\,\,$ & $\,\,$ 1 $\,\,$ & $\,\,$\color{red} 1.0317\color{black}  $\,\,$ \\
$\,\,$\color{red} 0.1042\color{black} $\,\,$ & $\,\,$\color{red} 0.4876\color{black} $\,\,$ & $\,\,$\color{red} 0.9693\color{black} $\,\,$ & $\,\,$ 1  $\,\,$ \\
\end{pmatrix},
\end{equation*}

\begin{equation*}
\mathbf{w}^{\prime} =
\begin{pmatrix}
0.700177\\
0.149685\\
0.075295\\
0.074842
\end{pmatrix} =
0.998138\cdot
\begin{pmatrix}
0.701483\\
0.149964\\
0.075436\\
\color{gr} 0.074982\color{black}
\end{pmatrix},
\end{equation*}
\begin{equation*}
\left[ \frac{{w}^{\prime}_i}{{w}^{\prime}_j} \right] =
\begin{pmatrix}
$\,\,$ 1 $\,\,$ & $\,\,$4.6777$\,\,$ & $\,\,$9.2991$\,\,$ & $\,\,$\color{gr} 9.3553\color{black} $\,\,$ \\
$\,\,$0.2138$\,\,$ & $\,\,$ 1 $\,\,$ & $\,\,$1.9880$\,\,$ & $\,\,$\color{gr} \color{blue} 2\color{black}   $\,\,$ \\
$\,\,$0.1075$\,\,$ & $\,\,$0.5030$\,\,$ & $\,\,$ 1 $\,\,$ & $\,\,$\color{gr} 1.0061\color{black}  $\,\,$ \\
$\,\,$\color{gr} 0.1069\color{black} $\,\,$ & $\,\,$\color{gr} \color{blue}  1/2\color{black} $\,\,$ & $\,\,$\color{gr} 0.9940\color{black} $\,\,$ & $\,\,$ 1  $\,\,$ \\
\end{pmatrix},
\end{equation*}
\end{example}
\newpage
\begin{example}
\begin{equation*}
\mathbf{A} =
\begin{pmatrix}
$\,\,$ 1 $\,\,$ & $\,\,$9$\,\,$ & $\,\,$7$\,\,$ & $\,\,$9 $\,\,$ \\
$\,\,$ 1/9$\,\,$ & $\,\,$ 1 $\,\,$ & $\,\,$4$\,\,$ & $\,\,$2 $\,\,$ \\
$\,\,$ 1/7$\,\,$ & $\,\,$ 1/4$\,\,$ & $\,\,$ 1 $\,\,$ & $\,\,$1 $\,\,$ \\
$\,\,$ 1/9$\,\,$ & $\,\,$ 1/2$\,\,$ & $\,\,$ 1 $\,\,$ & $\,\,$ 1  $\,\,$ \\
\end{pmatrix},
\qquad
\lambda_{\max} =
4.2416,
\qquad
CR = 0.0911
\end{equation*}

\begin{equation*}
\mathbf{w}^{cos} =
\begin{pmatrix}
0.688872\\
0.167573\\
0.072095\\
\color{red} 0.071459\color{black}
\end{pmatrix}\end{equation*}
\begin{equation*}
\left[ \frac{{w}^{cos}_i}{{w}^{cos}_j} \right] =
\begin{pmatrix}
$\,\,$ 1 $\,\,$ & $\,\,$4.1109$\,\,$ & $\,\,$9.5551$\,\,$ & $\,\,$\color{red} 9.6401\color{black} $\,\,$ \\
$\,\,$0.2433$\,\,$ & $\,\,$ 1 $\,\,$ & $\,\,$2.3243$\,\,$ & $\,\,$\color{red} 2.3450\color{black}   $\,\,$ \\
$\,\,$0.1047$\,\,$ & $\,\,$0.4302$\,\,$ & $\,\,$ 1 $\,\,$ & $\,\,$\color{red} 1.0089\color{black}  $\,\,$ \\
$\,\,$\color{red} 0.1037\color{black} $\,\,$ & $\,\,$\color{red} 0.4264\color{black} $\,\,$ & $\,\,$\color{red} 0.9912\color{black} $\,\,$ & $\,\,$ 1  $\,\,$ \\
\end{pmatrix},
\end{equation*}

\begin{equation*}
\mathbf{w}^{\prime} =
\begin{pmatrix}
0.688435\\
0.167467\\
0.072049\\
0.072049
\end{pmatrix} =
0.999365\cdot
\begin{pmatrix}
0.688872\\
0.167573\\
0.072095\\
\color{gr} 0.072095\color{black}
\end{pmatrix},
\end{equation*}
\begin{equation*}
\left[ \frac{{w}^{\prime}_i}{{w}^{\prime}_j} \right] =
\begin{pmatrix}
$\,\,$ 1 $\,\,$ & $\,\,$4.1109$\,\,$ & $\,\,$9.5551$\,\,$ & $\,\,$\color{gr} 9.5551\color{black} $\,\,$ \\
$\,\,$0.2433$\,\,$ & $\,\,$ 1 $\,\,$ & $\,\,$2.3243$\,\,$ & $\,\,$\color{gr} 2.3243\color{black}   $\,\,$ \\
$\,\,$0.1047$\,\,$ & $\,\,$0.4302$\,\,$ & $\,\,$ 1 $\,\,$ & $\,\,$\color{gr} \color{blue} 1\color{black}  $\,\,$ \\
$\,\,$\color{gr} 0.1047\color{black} $\,\,$ & $\,\,$\color{gr} 0.4302\color{black} $\,\,$ & $\,\,$\color{gr} \color{blue} 1\color{black} $\,\,$ & $\,\,$ 1  $\,\,$ \\
\end{pmatrix},
\end{equation*}
\end{example}
\newpage
\begin{example}
\begin{equation*}
\mathbf{A} =
\begin{pmatrix}
$\,\,$ 1 $\,\,$ & $\,\,$1$\,\,$ & $\,\,$2$\,\,$ & $\,\,$3 $\,\,$ \\
$\,\,$ 1 $\,\,$ & $\,\,$ 1 $\,\,$ & $\,\,$3$\,\,$ & $\,\,$2 $\,\,$ \\
$\,\,$ 1/2$\,\,$ & $\,\,$ 1/3$\,\,$ & $\,\,$ 1 $\,\,$ & $\,\,$ 1/2 $\,\,$ \\
$\,\,$ 1/3$\,\,$ & $\,\,$ 1/2$\,\,$ & $\,\,$2$\,\,$ & $\,\,$ 1  $\,\,$ \\
\end{pmatrix},
\qquad
\lambda_{\max} =
4.1031,
\qquad
CR = 0.0389
\end{equation*}

\begin{equation*}
\mathbf{w}^{cos} =
\begin{pmatrix}
0.351966\\
\color{red} 0.347963\color{black} \\
0.124803\\
0.175268
\end{pmatrix}\end{equation*}
\begin{equation*}
\left[ \frac{{w}^{cos}_i}{{w}^{cos}_j} \right] =
\begin{pmatrix}
$\,\,$ 1 $\,\,$ & $\,\,$\color{red} 1.0115\color{black} $\,\,$ & $\,\,$2.8202$\,\,$ & $\,\,$2.0082$\,\,$ \\
$\,\,$\color{red} 0.9886\color{black} $\,\,$ & $\,\,$ 1 $\,\,$ & $\,\,$\color{red} 2.7881\color{black} $\,\,$ & $\,\,$\color{red} 1.9853\color{black}   $\,\,$ \\
$\,\,$0.3546$\,\,$ & $\,\,$\color{red} 0.3587\color{black} $\,\,$ & $\,\,$ 1 $\,\,$ & $\,\,$0.7121 $\,\,$ \\
$\,\,$0.4980$\,\,$ & $\,\,$\color{red} 0.5037\color{black} $\,\,$ & $\,\,$1.4044$\,\,$ & $\,\,$ 1  $\,\,$ \\
\end{pmatrix},
\end{equation*}

\begin{equation*}
\mathbf{w}^{\prime} =
\begin{pmatrix}
0.351063\\
0.349636\\
0.124483\\
0.174818
\end{pmatrix} =
0.997433\cdot
\begin{pmatrix}
0.351966\\
\color{gr} 0.350536\color{black} \\
0.124803\\
0.175268
\end{pmatrix},
\end{equation*}
\begin{equation*}
\left[ \frac{{w}^{\prime}_i}{{w}^{\prime}_j} \right] =
\begin{pmatrix}
$\,\,$ 1 $\,\,$ & $\,\,$\color{gr} 1.0041\color{black} $\,\,$ & $\,\,$2.8202$\,\,$ & $\,\,$2.0082$\,\,$ \\
$\,\,$\color{gr} 0.9959\color{black} $\,\,$ & $\,\,$ 1 $\,\,$ & $\,\,$\color{gr} 2.8087\color{black} $\,\,$ & $\,\,$\color{gr} \color{blue} 2\color{black}   $\,\,$ \\
$\,\,$0.3546$\,\,$ & $\,\,$\color{gr} 0.3560\color{black} $\,\,$ & $\,\,$ 1 $\,\,$ & $\,\,$0.7121 $\,\,$ \\
$\,\,$0.4980$\,\,$ & $\,\,$\color{gr} \color{blue}  1/2\color{black} $\,\,$ & $\,\,$1.4044$\,\,$ & $\,\,$ 1  $\,\,$ \\
\end{pmatrix},
\end{equation*}
\end{example}
\newpage
\begin{example}
\begin{equation*}
\mathbf{A} =
\begin{pmatrix}
$\,\,$ 1 $\,\,$ & $\,\,$1$\,\,$ & $\,\,$2$\,\,$ & $\,\,$3 $\,\,$ \\
$\,\,$ 1 $\,\,$ & $\,\,$ 1 $\,\,$ & $\,\,$3$\,\,$ & $\,\,$2 $\,\,$ \\
$\,\,$ 1/2$\,\,$ & $\,\,$ 1/3$\,\,$ & $\,\,$ 1 $\,\,$ & $\,\,$ 1/3 $\,\,$ \\
$\,\,$ 1/3$\,\,$ & $\,\,$ 1/2$\,\,$ & $\,\,$3$\,\,$ & $\,\,$ 1  $\,\,$ \\
\end{pmatrix},
\qquad
\lambda_{\max} =
4.1990,
\qquad
CR = 0.0750
\end{equation*}

\begin{equation*}
\mathbf{w}^{cos} =
\begin{pmatrix}
0.347184\\
\color{red} 0.339227\color{black} \\
0.115797\\
0.197792
\end{pmatrix}\end{equation*}
\begin{equation*}
\left[ \frac{{w}^{cos}_i}{{w}^{cos}_j} \right] =
\begin{pmatrix}
$\,\,$ 1 $\,\,$ & $\,\,$\color{red} 1.0235\color{black} $\,\,$ & $\,\,$2.9982$\,\,$ & $\,\,$1.7553$\,\,$ \\
$\,\,$\color{red} 0.9771\color{black} $\,\,$ & $\,\,$ 1 $\,\,$ & $\,\,$\color{red} 2.9295\color{black} $\,\,$ & $\,\,$\color{red} 1.7151\color{black}   $\,\,$ \\
$\,\,$0.3335$\,\,$ & $\,\,$\color{red} 0.3414\color{black} $\,\,$ & $\,\,$ 1 $\,\,$ & $\,\,$0.5854 $\,\,$ \\
$\,\,$0.5697$\,\,$ & $\,\,$\color{red} 0.5831\color{black} $\,\,$ & $\,\,$1.7081$\,\,$ & $\,\,$ 1  $\,\,$ \\
\end{pmatrix},
\end{equation*}

\begin{equation*}
\mathbf{w}^{\prime} =
\begin{pmatrix}
0.344443\\
0.344443\\
0.114883\\
0.196231
\end{pmatrix} =
0.992106\cdot
\begin{pmatrix}
0.347184\\
\color{gr} 0.347184\color{black} \\
0.115797\\
0.197792
\end{pmatrix},
\end{equation*}
\begin{equation*}
\left[ \frac{{w}^{\prime}_i}{{w}^{\prime}_j} \right] =
\begin{pmatrix}
$\,\,$ 1 $\,\,$ & $\,\,$\color{gr} \color{blue} 1\color{black} $\,\,$ & $\,\,$2.9982$\,\,$ & $\,\,$1.7553$\,\,$ \\
$\,\,$\color{gr} \color{blue} 1\color{black} $\,\,$ & $\,\,$ 1 $\,\,$ & $\,\,$\color{gr} 2.9982\color{black} $\,\,$ & $\,\,$\color{gr} 1.7553\color{black}   $\,\,$ \\
$\,\,$0.3335$\,\,$ & $\,\,$\color{gr} 0.3335\color{black} $\,\,$ & $\,\,$ 1 $\,\,$ & $\,\,$0.5854 $\,\,$ \\
$\,\,$0.5697$\,\,$ & $\,\,$\color{gr} 0.5697\color{black} $\,\,$ & $\,\,$1.7081$\,\,$ & $\,\,$ 1  $\,\,$ \\
\end{pmatrix},
\end{equation*}
\end{example}
\newpage
\begin{example}
\begin{equation*}
\mathbf{A} =
\begin{pmatrix}
$\,\,$ 1 $\,\,$ & $\,\,$1$\,\,$ & $\,\,$3$\,\,$ & $\,\,$3 $\,\,$ \\
$\,\,$ 1 $\,\,$ & $\,\,$ 1 $\,\,$ & $\,\,$4$\,\,$ & $\,\,$2 $\,\,$ \\
$\,\,$ 1/3$\,\,$ & $\,\,$ 1/4$\,\,$ & $\,\,$ 1 $\,\,$ & $\,\,$ 1/3 $\,\,$ \\
$\,\,$ 1/3$\,\,$ & $\,\,$ 1/2$\,\,$ & $\,\,$3$\,\,$ & $\,\,$ 1  $\,\,$ \\
\end{pmatrix},
\qquad
\lambda_{\max} =
4.1031,
\qquad
CR = 0.0389
\end{equation*}

\begin{equation*}
\mathbf{w}^{cos} =
\begin{pmatrix}
0.368794\\
\color{red} 0.355188\color{black} \\
0.090412\\
0.185606
\end{pmatrix}\end{equation*}
\begin{equation*}
\left[ \frac{{w}^{cos}_i}{{w}^{cos}_j} \right] =
\begin{pmatrix}
$\,\,$ 1 $\,\,$ & $\,\,$\color{red} 1.0383\color{black} $\,\,$ & $\,\,$4.0790$\,\,$ & $\,\,$1.9870$\,\,$ \\
$\,\,$\color{red} 0.9631\color{black} $\,\,$ & $\,\,$ 1 $\,\,$ & $\,\,$\color{red} 3.9286\color{black} $\,\,$ & $\,\,$\color{red} 1.9137\color{black}   $\,\,$ \\
$\,\,$0.2452$\,\,$ & $\,\,$\color{red} 0.2545\color{black} $\,\,$ & $\,\,$ 1 $\,\,$ & $\,\,$0.4871 $\,\,$ \\
$\,\,$0.5033$\,\,$ & $\,\,$\color{red} 0.5226\color{black} $\,\,$ & $\,\,$2.0529$\,\,$ & $\,\,$ 1  $\,\,$ \\
\end{pmatrix},
\end{equation*}

\begin{equation*}
\mathbf{w}^{\prime} =
\begin{pmatrix}
0.366427\\
0.359327\\
0.089832\\
0.184415
\end{pmatrix} =
0.993582\cdot
\begin{pmatrix}
0.368794\\
\color{gr} 0.361648\color{black} \\
0.090412\\
0.185606
\end{pmatrix},
\end{equation*}
\begin{equation*}
\left[ \frac{{w}^{\prime}_i}{{w}^{\prime}_j} \right] =
\begin{pmatrix}
$\,\,$ 1 $\,\,$ & $\,\,$\color{gr} 1.0198\color{black} $\,\,$ & $\,\,$4.0790$\,\,$ & $\,\,$1.9870$\,\,$ \\
$\,\,$\color{gr} 0.9806\color{black} $\,\,$ & $\,\,$ 1 $\,\,$ & $\,\,$\color{gr} \color{blue} 4\color{black} $\,\,$ & $\,\,$\color{gr} 1.9485\color{black}   $\,\,$ \\
$\,\,$0.2452$\,\,$ & $\,\,$\color{gr} \color{blue}  1/4\color{black} $\,\,$ & $\,\,$ 1 $\,\,$ & $\,\,$0.4871 $\,\,$ \\
$\,\,$0.5033$\,\,$ & $\,\,$\color{gr} 0.5132\color{black} $\,\,$ & $\,\,$2.0529$\,\,$ & $\,\,$ 1  $\,\,$ \\
\end{pmatrix},
\end{equation*}
\end{example}
\newpage
\begin{example}
\begin{equation*}
\mathbf{A} =
\begin{pmatrix}
$\,\,$ 1 $\,\,$ & $\,\,$1$\,\,$ & $\,\,$3$\,\,$ & $\,\,$3 $\,\,$ \\
$\,\,$ 1 $\,\,$ & $\,\,$ 1 $\,\,$ & $\,\,$5$\,\,$ & $\,\,$2 $\,\,$ \\
$\,\,$ 1/3$\,\,$ & $\,\,$ 1/5$\,\,$ & $\,\,$ 1 $\,\,$ & $\,\,$ 1/5 $\,\,$ \\
$\,\,$ 1/3$\,\,$ & $\,\,$ 1/2$\,\,$ & $\,\,$5$\,\,$ & $\,\,$ 1  $\,\,$ \\
\end{pmatrix},
\qquad
\lambda_{\max} =
4.2277,
\qquad
CR = 0.0859
\end{equation*}

\begin{equation*}
\mathbf{w}^{cos} =
\begin{pmatrix}
0.358143\\
\color{red} 0.356909\color{black} \\
0.076508\\
0.208439
\end{pmatrix}\end{equation*}
\begin{equation*}
\left[ \frac{{w}^{cos}_i}{{w}^{cos}_j} \right] =
\begin{pmatrix}
$\,\,$ 1 $\,\,$ & $\,\,$\color{red} 1.0035\color{black} $\,\,$ & $\,\,$4.6811$\,\,$ & $\,\,$1.7182$\,\,$ \\
$\,\,$\color{red} 0.9966\color{black} $\,\,$ & $\,\,$ 1 $\,\,$ & $\,\,$\color{red} 4.6650\color{black} $\,\,$ & $\,\,$\color{red} 1.7123\color{black}   $\,\,$ \\
$\,\,$0.2136$\,\,$ & $\,\,$\color{red} 0.2144\color{black} $\,\,$ & $\,\,$ 1 $\,\,$ & $\,\,$0.3671 $\,\,$ \\
$\,\,$0.5820$\,\,$ & $\,\,$\color{red} 0.5840\color{black} $\,\,$ & $\,\,$2.7244$\,\,$ & $\,\,$ 1  $\,\,$ \\
\end{pmatrix},
\end{equation*}

\begin{equation*}
\mathbf{w}^{\prime} =
\begin{pmatrix}
0.357702\\
0.357702\\
0.076414\\
0.208182
\end{pmatrix} =
0.998768\cdot
\begin{pmatrix}
0.358143\\
\color{gr} 0.358143\color{black} \\
0.076508\\
0.208439
\end{pmatrix},
\end{equation*}
\begin{equation*}
\left[ \frac{{w}^{\prime}_i}{{w}^{\prime}_j} \right] =
\begin{pmatrix}
$\,\,$ 1 $\,\,$ & $\,\,$\color{gr} \color{blue} 1\color{black} $\,\,$ & $\,\,$4.6811$\,\,$ & $\,\,$1.7182$\,\,$ \\
$\,\,$\color{gr} \color{blue} 1\color{black} $\,\,$ & $\,\,$ 1 $\,\,$ & $\,\,$\color{gr} 4.6811\color{black} $\,\,$ & $\,\,$\color{gr} 1.7182\color{black}   $\,\,$ \\
$\,\,$0.2136$\,\,$ & $\,\,$\color{gr} 0.2136\color{black} $\,\,$ & $\,\,$ 1 $\,\,$ & $\,\,$0.3671 $\,\,$ \\
$\,\,$0.5820$\,\,$ & $\,\,$\color{gr} 0.5820\color{black} $\,\,$ & $\,\,$2.7244$\,\,$ & $\,\,$ 1  $\,\,$ \\
\end{pmatrix},
\end{equation*}
\end{example}
\newpage
\begin{example}
\begin{equation*}
\mathbf{A} =
\begin{pmatrix}
$\,\,$ 1 $\,\,$ & $\,\,$1$\,\,$ & $\,\,$3$\,\,$ & $\,\,$4 $\,\,$ \\
$\,\,$ 1 $\,\,$ & $\,\,$ 1 $\,\,$ & $\,\,$4$\,\,$ & $\,\,$2 $\,\,$ \\
$\,\,$ 1/3$\,\,$ & $\,\,$ 1/4$\,\,$ & $\,\,$ 1 $\,\,$ & $\,\,$ 1/4 $\,\,$ \\
$\,\,$ 1/4$\,\,$ & $\,\,$ 1/2$\,\,$ & $\,\,$4$\,\,$ & $\,\,$ 1  $\,\,$ \\
\end{pmatrix},
\qquad
\lambda_{\max} =
4.2512,
\qquad
CR = 0.0947
\end{equation*}

\begin{equation*}
\mathbf{w}^{cos} =
\begin{pmatrix}
0.381918\\
\color{red} 0.341659\color{black} \\
0.085708\\
0.190715
\end{pmatrix}\end{equation*}
\begin{equation*}
\left[ \frac{{w}^{cos}_i}{{w}^{cos}_j} \right] =
\begin{pmatrix}
$\,\,$ 1 $\,\,$ & $\,\,$\color{red} 1.1178\color{black} $\,\,$ & $\,\,$4.4560$\,\,$ & $\,\,$2.0026$\,\,$ \\
$\,\,$\color{red} 0.8946\color{black} $\,\,$ & $\,\,$ 1 $\,\,$ & $\,\,$\color{red} 3.9863\color{black} $\,\,$ & $\,\,$\color{red} 1.7915\color{black}   $\,\,$ \\
$\,\,$0.2244$\,\,$ & $\,\,$\color{red} 0.2509\color{black} $\,\,$ & $\,\,$ 1 $\,\,$ & $\,\,$0.4494 $\,\,$ \\
$\,\,$0.4994$\,\,$ & $\,\,$\color{red} 0.5582\color{black} $\,\,$ & $\,\,$2.2252$\,\,$ & $\,\,$ 1  $\,\,$ \\
\end{pmatrix},
\end{equation*}

\begin{equation*}
\mathbf{w}^{\prime} =
\begin{pmatrix}
0.381470\\
0.342431\\
0.085608\\
0.190491
\end{pmatrix} =
0.998829\cdot
\begin{pmatrix}
0.381918\\
\color{gr} 0.342832\color{black} \\
0.085708\\
0.190715
\end{pmatrix},
\end{equation*}
\begin{equation*}
\left[ \frac{{w}^{\prime}_i}{{w}^{\prime}_j} \right] =
\begin{pmatrix}
$\,\,$ 1 $\,\,$ & $\,\,$\color{gr} 1.1140\color{black} $\,\,$ & $\,\,$4.4560$\,\,$ & $\,\,$2.0026$\,\,$ \\
$\,\,$\color{gr} 0.8977\color{black} $\,\,$ & $\,\,$ 1 $\,\,$ & $\,\,$\color{gr} \color{blue} 4\color{black} $\,\,$ & $\,\,$\color{gr} 1.7976\color{black}   $\,\,$ \\
$\,\,$0.2244$\,\,$ & $\,\,$\color{gr} \color{blue}  1/4\color{black} $\,\,$ & $\,\,$ 1 $\,\,$ & $\,\,$0.4494 $\,\,$ \\
$\,\,$0.4994$\,\,$ & $\,\,$\color{gr} 0.5563\color{black} $\,\,$ & $\,\,$2.2252$\,\,$ & $\,\,$ 1  $\,\,$ \\
\end{pmatrix},
\end{equation*}
\end{example}
\newpage
\begin{example}
\begin{equation*}
\mathbf{A} =
\begin{pmatrix}
$\,\,$ 1 $\,\,$ & $\,\,$1$\,\,$ & $\,\,$3$\,\,$ & $\,\,$4 $\,\,$ \\
$\,\,$ 1 $\,\,$ & $\,\,$ 1 $\,\,$ & $\,\,$4$\,\,$ & $\,\,$3 $\,\,$ \\
$\,\,$ 1/3$\,\,$ & $\,\,$ 1/4$\,\,$ & $\,\,$ 1 $\,\,$ & $\,\,$ 1/2 $\,\,$ \\
$\,\,$ 1/4$\,\,$ & $\,\,$ 1/3$\,\,$ & $\,\,$2$\,\,$ & $\,\,$ 1  $\,\,$ \\
\end{pmatrix},
\qquad
\lambda_{\max} =
4.0820,
\qquad
CR = 0.0309
\end{equation*}

\begin{equation*}
\mathbf{w}^{cos} =
\begin{pmatrix}
0.384163\\
\color{red} 0.382360\color{black} \\
0.096696\\
0.136781
\end{pmatrix}\end{equation*}
\begin{equation*}
\left[ \frac{{w}^{cos}_i}{{w}^{cos}_j} \right] =
\begin{pmatrix}
$\,\,$ 1 $\,\,$ & $\,\,$\color{red} 1.0047\color{black} $\,\,$ & $\,\,$3.9729$\,\,$ & $\,\,$2.8086$\,\,$ \\
$\,\,$\color{red} 0.9953\color{black} $\,\,$ & $\,\,$ 1 $\,\,$ & $\,\,$\color{red} 3.9543\color{black} $\,\,$ & $\,\,$\color{red} 2.7954\color{black}   $\,\,$ \\
$\,\,$0.2517$\,\,$ & $\,\,$\color{red} 0.2529\color{black} $\,\,$ & $\,\,$ 1 $\,\,$ & $\,\,$0.7069 $\,\,$ \\
$\,\,$0.3560$\,\,$ & $\,\,$\color{red} 0.3577\color{black} $\,\,$ & $\,\,$1.4145$\,\,$ & $\,\,$ 1  $\,\,$ \\
\end{pmatrix},
\end{equation*}

\begin{equation*}
\mathbf{w}^{\prime} =
\begin{pmatrix}
0.383472\\
0.383472\\
0.096522\\
0.136535
\end{pmatrix} =
0.998200\cdot
\begin{pmatrix}
0.384163\\
\color{gr} 0.384163\color{black} \\
0.096696\\
0.136781
\end{pmatrix},
\end{equation*}
\begin{equation*}
\left[ \frac{{w}^{\prime}_i}{{w}^{\prime}_j} \right] =
\begin{pmatrix}
$\,\,$ 1 $\,\,$ & $\,\,$\color{gr} \color{blue} 1\color{black} $\,\,$ & $\,\,$3.9729$\,\,$ & $\,\,$2.8086$\,\,$ \\
$\,\,$\color{gr} \color{blue} 1\color{black} $\,\,$ & $\,\,$ 1 $\,\,$ & $\,\,$\color{gr} 3.9729\color{black} $\,\,$ & $\,\,$\color{gr} 2.8086\color{black}   $\,\,$ \\
$\,\,$0.2517$\,\,$ & $\,\,$\color{gr} 0.2517\color{black} $\,\,$ & $\,\,$ 1 $\,\,$ & $\,\,$0.7069 $\,\,$ \\
$\,\,$0.3560$\,\,$ & $\,\,$\color{gr} 0.3560\color{black} $\,\,$ & $\,\,$1.4145$\,\,$ & $\,\,$ 1  $\,\,$ \\
\end{pmatrix},
\end{equation*}
\end{example}
\newpage
\begin{example}
\begin{equation*}
\mathbf{A} =
\begin{pmatrix}
$\,\,$ 1 $\,\,$ & $\,\,$1$\,\,$ & $\,\,$3$\,\,$ & $\,\,$4 $\,\,$ \\
$\,\,$ 1 $\,\,$ & $\,\,$ 1 $\,\,$ & $\,\,$5$\,\,$ & $\,\,$2 $\,\,$ \\
$\,\,$ 1/3$\,\,$ & $\,\,$ 1/5$\,\,$ & $\,\,$ 1 $\,\,$ & $\,\,$ 1/4 $\,\,$ \\
$\,\,$ 1/4$\,\,$ & $\,\,$ 1/2$\,\,$ & $\,\,$4$\,\,$ & $\,\,$ 1  $\,\,$ \\
\end{pmatrix},
\qquad
\lambda_{\max} =
4.2460,
\qquad
CR = 0.0928
\end{equation*}

\begin{equation*}
\mathbf{w}^{cos} =
\begin{pmatrix}
0.379262\\
\color{red} 0.356811\color{black} \\
0.079688\\
0.184240
\end{pmatrix}\end{equation*}
\begin{equation*}
\left[ \frac{{w}^{cos}_i}{{w}^{cos}_j} \right] =
\begin{pmatrix}
$\,\,$ 1 $\,\,$ & $\,\,$\color{red} 1.0629\color{black} $\,\,$ & $\,\,$4.7593$\,\,$ & $\,\,$2.0585$\,\,$ \\
$\,\,$\color{red} 0.9408\color{black} $\,\,$ & $\,\,$ 1 $\,\,$ & $\,\,$\color{red} 4.4776\color{black} $\,\,$ & $\,\,$\color{red} 1.9367\color{black}   $\,\,$ \\
$\,\,$0.2101$\,\,$ & $\,\,$\color{red} 0.2233\color{black} $\,\,$ & $\,\,$ 1 $\,\,$ & $\,\,$0.4325 $\,\,$ \\
$\,\,$0.4858$\,\,$ & $\,\,$\color{red} 0.5164\color{black} $\,\,$ & $\,\,$2.3120$\,\,$ & $\,\,$ 1  $\,\,$ \\
\end{pmatrix},
\end{equation*}

\begin{equation*}
\mathbf{w}^{\prime} =
\begin{pmatrix}
0.374887\\
0.364229\\
0.078769\\
0.182115
\end{pmatrix} =
0.988465\cdot
\begin{pmatrix}
0.379262\\
\color{gr} 0.368480\color{black} \\
0.079688\\
0.184240
\end{pmatrix},
\end{equation*}
\begin{equation*}
\left[ \frac{{w}^{\prime}_i}{{w}^{\prime}_j} \right] =
\begin{pmatrix}
$\,\,$ 1 $\,\,$ & $\,\,$\color{gr} 1.0293\color{black} $\,\,$ & $\,\,$4.7593$\,\,$ & $\,\,$2.0585$\,\,$ \\
$\,\,$\color{gr} 0.9716\color{black} $\,\,$ & $\,\,$ 1 $\,\,$ & $\,\,$\color{gr} 4.6240\color{black} $\,\,$ & $\,\,$\color{gr} \color{blue} 2\color{black}   $\,\,$ \\
$\,\,$0.2101$\,\,$ & $\,\,$\color{gr} 0.2163\color{black} $\,\,$ & $\,\,$ 1 $\,\,$ & $\,\,$0.4325 $\,\,$ \\
$\,\,$0.4858$\,\,$ & $\,\,$\color{gr} \color{blue}  1/2\color{black} $\,\,$ & $\,\,$2.3120$\,\,$ & $\,\,$ 1  $\,\,$ \\
\end{pmatrix},
\end{equation*}
\end{example}
\newpage
\begin{example}
\begin{equation*}
\mathbf{A} =
\begin{pmatrix}
$\,\,$ 1 $\,\,$ & $\,\,$1$\,\,$ & $\,\,$3$\,\,$ & $\,\,$5 $\,\,$ \\
$\,\,$ 1 $\,\,$ & $\,\,$ 1 $\,\,$ & $\,\,$4$\,\,$ & $\,\,$3 $\,\,$ \\
$\,\,$ 1/3$\,\,$ & $\,\,$ 1/4$\,\,$ & $\,\,$ 1 $\,\,$ & $\,\,$ 1/2 $\,\,$ \\
$\,\,$ 1/5$\,\,$ & $\,\,$ 1/3$\,\,$ & $\,\,$2$\,\,$ & $\,\,$ 1  $\,\,$ \\
\end{pmatrix},
\qquad
\lambda_{\max} =
4.1252,
\qquad
CR = 0.0472
\end{equation*}

\begin{equation*}
\mathbf{w}^{cos} =
\begin{pmatrix}
0.398239\\
\color{red} 0.375896\color{black} \\
0.096104\\
0.129760
\end{pmatrix}\end{equation*}
\begin{equation*}
\left[ \frac{{w}^{cos}_i}{{w}^{cos}_j} \right] =
\begin{pmatrix}
$\,\,$ 1 $\,\,$ & $\,\,$\color{red} 1.0594\color{black} $\,\,$ & $\,\,$4.1438$\,\,$ & $\,\,$3.0690$\,\,$ \\
$\,\,$\color{red} 0.9439\color{black} $\,\,$ & $\,\,$ 1 $\,\,$ & $\,\,$\color{red} 3.9113\color{black} $\,\,$ & $\,\,$\color{red} 2.8969\color{black}   $\,\,$ \\
$\,\,$0.2413$\,\,$ & $\,\,$\color{red} 0.2557\color{black} $\,\,$ & $\,\,$ 1 $\,\,$ & $\,\,$0.7406 $\,\,$ \\
$\,\,$0.3258$\,\,$ & $\,\,$\color{red} 0.3452\color{black} $\,\,$ & $\,\,$1.3502$\,\,$ & $\,\,$ 1  $\,\,$ \\
\end{pmatrix},
\end{equation*}

\begin{equation*}
\mathbf{w}^{\prime} =
\begin{pmatrix}
0.394874\\
0.381169\\
0.095292\\
0.128664
\end{pmatrix} =
0.991551\cdot
\begin{pmatrix}
0.398239\\
\color{gr} 0.384418\color{black} \\
0.096104\\
0.129760
\end{pmatrix},
\end{equation*}
\begin{equation*}
\left[ \frac{{w}^{\prime}_i}{{w}^{\prime}_j} \right] =
\begin{pmatrix}
$\,\,$ 1 $\,\,$ & $\,\,$\color{gr} 1.0360\color{black} $\,\,$ & $\,\,$4.1438$\,\,$ & $\,\,$3.0690$\,\,$ \\
$\,\,$\color{gr} 0.9653\color{black} $\,\,$ & $\,\,$ 1 $\,\,$ & $\,\,$\color{gr} \color{blue} 4\color{black} $\,\,$ & $\,\,$\color{gr} 2.9625\color{black}   $\,\,$ \\
$\,\,$0.2413$\,\,$ & $\,\,$\color{gr} \color{blue}  1/4\color{black} $\,\,$ & $\,\,$ 1 $\,\,$ & $\,\,$0.7406 $\,\,$ \\
$\,\,$0.3258$\,\,$ & $\,\,$\color{gr} 0.3375\color{black} $\,\,$ & $\,\,$1.3502$\,\,$ & $\,\,$ 1  $\,\,$ \\
\end{pmatrix},
\end{equation*}
\end{example}
\newpage
\begin{example}
\begin{equation*}
\mathbf{A} =
\begin{pmatrix}
$\,\,$ 1 $\,\,$ & $\,\,$1$\,\,$ & $\,\,$3$\,\,$ & $\,\,$5 $\,\,$ \\
$\,\,$ 1 $\,\,$ & $\,\,$ 1 $\,\,$ & $\,\,$5$\,\,$ & $\,\,$3 $\,\,$ \\
$\,\,$ 1/3$\,\,$ & $\,\,$ 1/5$\,\,$ & $\,\,$ 1 $\,\,$ & $\,\,$ 1/3 $\,\,$ \\
$\,\,$ 1/5$\,\,$ & $\,\,$ 1/3$\,\,$ & $\,\,$3$\,\,$ & $\,\,$ 1  $\,\,$ \\
\end{pmatrix},
\qquad
\lambda_{\max} =
4.2253,
\qquad
CR = 0.0849
\end{equation*}

\begin{equation*}
\mathbf{w}^{cos} =
\begin{pmatrix}
0.388920\\
\color{red} 0.383653\color{black} \\
0.083443\\
0.143984
\end{pmatrix}\end{equation*}
\begin{equation*}
\left[ \frac{{w}^{cos}_i}{{w}^{cos}_j} \right] =
\begin{pmatrix}
$\,\,$ 1 $\,\,$ & $\,\,$\color{red} 1.0137\color{black} $\,\,$ & $\,\,$4.6609$\,\,$ & $\,\,$2.7011$\,\,$ \\
$\,\,$\color{red} 0.9865\color{black} $\,\,$ & $\,\,$ 1 $\,\,$ & $\,\,$\color{red} 4.5978\color{black} $\,\,$ & $\,\,$\color{red} 2.6646\color{black}   $\,\,$ \\
$\,\,$0.2146$\,\,$ & $\,\,$\color{red} 0.2175\color{black} $\,\,$ & $\,\,$ 1 $\,\,$ & $\,\,$0.5795 $\,\,$ \\
$\,\,$0.3702$\,\,$ & $\,\,$\color{red} 0.3753\color{black} $\,\,$ & $\,\,$1.7255$\,\,$ & $\,\,$ 1  $\,\,$ \\
\end{pmatrix},
\end{equation*}

\begin{equation*}
\mathbf{w}^{\prime} =
\begin{pmatrix}
0.386882\\
0.386882\\
0.083006\\
0.143229
\end{pmatrix} =
0.994761\cdot
\begin{pmatrix}
0.388920\\
\color{gr} 0.388920\color{black} \\
0.083443\\
0.143984
\end{pmatrix},
\end{equation*}
\begin{equation*}
\left[ \frac{{w}^{\prime}_i}{{w}^{\prime}_j} \right] =
\begin{pmatrix}
$\,\,$ 1 $\,\,$ & $\,\,$\color{gr} \color{blue} 1\color{black} $\,\,$ & $\,\,$4.6609$\,\,$ & $\,\,$2.7011$\,\,$ \\
$\,\,$\color{gr} \color{blue} 1\color{black} $\,\,$ & $\,\,$ 1 $\,\,$ & $\,\,$\color{gr} 4.6609\color{black} $\,\,$ & $\,\,$\color{gr} 2.7011\color{black}   $\,\,$ \\
$\,\,$0.2146$\,\,$ & $\,\,$\color{gr} 0.2146\color{black} $\,\,$ & $\,\,$ 1 $\,\,$ & $\,\,$0.5795 $\,\,$ \\
$\,\,$0.3702$\,\,$ & $\,\,$\color{gr} 0.3702\color{black} $\,\,$ & $\,\,$1.7255$\,\,$ & $\,\,$ 1  $\,\,$ \\
\end{pmatrix},
\end{equation*}
\end{example}
\newpage
\begin{example}
\begin{equation*}
\mathbf{A} =
\begin{pmatrix}
$\,\,$ 1 $\,\,$ & $\,\,$1$\,\,$ & $\,\,$3$\,\,$ & $\,\,$6 $\,\,$ \\
$\,\,$ 1 $\,\,$ & $\,\,$ 1 $\,\,$ & $\,\,$4$\,\,$ & $\,\,$3 $\,\,$ \\
$\,\,$ 1/3$\,\,$ & $\,\,$ 1/4$\,\,$ & $\,\,$ 1 $\,\,$ & $\,\,$ 1/2 $\,\,$ \\
$\,\,$ 1/6$\,\,$ & $\,\,$ 1/3$\,\,$ & $\,\,$2$\,\,$ & $\,\,$ 1  $\,\,$ \\
\end{pmatrix},
\qquad
\lambda_{\max} =
4.1707,
\qquad
CR = 0.0644
\end{equation*}

\begin{equation*}
\mathbf{w}^{cos} =
\begin{pmatrix}
0.408709\\
\color{red} 0.370861\color{black} \\
0.095675\\
0.124755
\end{pmatrix}\end{equation*}
\begin{equation*}
\left[ \frac{{w}^{cos}_i}{{w}^{cos}_j} \right] =
\begin{pmatrix}
$\,\,$ 1 $\,\,$ & $\,\,$\color{red} 1.1021\color{black} $\,\,$ & $\,\,$4.2718$\,\,$ & $\,\,$3.2761$\,\,$ \\
$\,\,$\color{red} 0.9074\color{black} $\,\,$ & $\,\,$ 1 $\,\,$ & $\,\,$\color{red} 3.8763\color{black} $\,\,$ & $\,\,$\color{red} 2.9727\color{black}   $\,\,$ \\
$\,\,$0.2341$\,\,$ & $\,\,$\color{red} 0.2580\color{black} $\,\,$ & $\,\,$ 1 $\,\,$ & $\,\,$0.7669 $\,\,$ \\
$\,\,$0.3052$\,\,$ & $\,\,$\color{red} 0.3364\color{black} $\,\,$ & $\,\,$1.3040$\,\,$ & $\,\,$ 1  $\,\,$ \\
\end{pmatrix},
\end{equation*}

\begin{equation*}
\mathbf{w}^{\prime} =
\begin{pmatrix}
0.407322\\
0.372996\\
0.095350\\
0.124332
\end{pmatrix} =
0.996606\cdot
\begin{pmatrix}
0.408709\\
\color{gr} 0.374266\color{black} \\
0.095675\\
0.124755
\end{pmatrix},
\end{equation*}
\begin{equation*}
\left[ \frac{{w}^{\prime}_i}{{w}^{\prime}_j} \right] =
\begin{pmatrix}
$\,\,$ 1 $\,\,$ & $\,\,$\color{gr} 1.0920\color{black} $\,\,$ & $\,\,$4.2718$\,\,$ & $\,\,$3.2761$\,\,$ \\
$\,\,$\color{gr} 0.9157\color{black} $\,\,$ & $\,\,$ 1 $\,\,$ & $\,\,$\color{gr} 3.9119\color{black} $\,\,$ & $\,\,$\color{gr} \color{blue} 3\color{black}   $\,\,$ \\
$\,\,$0.2341$\,\,$ & $\,\,$\color{gr} 0.2556\color{black} $\,\,$ & $\,\,$ 1 $\,\,$ & $\,\,$0.7669 $\,\,$ \\
$\,\,$0.3052$\,\,$ & $\,\,$\color{gr} \color{blue}  1/3\color{black} $\,\,$ & $\,\,$1.3040$\,\,$ & $\,\,$ 1  $\,\,$ \\
\end{pmatrix},
\end{equation*}
\end{example}
\newpage
\begin{example}
\begin{equation*}
\mathbf{A} =
\begin{pmatrix}
$\,\,$ 1 $\,\,$ & $\,\,$1$\,\,$ & $\,\,$4$\,\,$ & $\,\,$3 $\,\,$ \\
$\,\,$ 1 $\,\,$ & $\,\,$ 1 $\,\,$ & $\,\,$6$\,\,$ & $\,\,$2 $\,\,$ \\
$\,\,$ 1/4$\,\,$ & $\,\,$ 1/6$\,\,$ & $\,\,$ 1 $\,\,$ & $\,\,$ 1/4 $\,\,$ \\
$\,\,$ 1/3$\,\,$ & $\,\,$ 1/2$\,\,$ & $\,\,$4$\,\,$ & $\,\,$ 1  $\,\,$ \\
\end{pmatrix},
\qquad
\lambda_{\max} =
4.1031,
\qquad
CR = 0.0389
\end{equation*}

\begin{equation*}
\mathbf{w}^{cos} =
\begin{pmatrix}
0.375081\\
\color{red} 0.371378\color{black} \\
0.066802\\
0.186739
\end{pmatrix}\end{equation*}
\begin{equation*}
\left[ \frac{{w}^{cos}_i}{{w}^{cos}_j} \right] =
\begin{pmatrix}
$\,\,$ 1 $\,\,$ & $\,\,$\color{red} 1.0100\color{black} $\,\,$ & $\,\,$5.6148$\,\,$ & $\,\,$2.0086$\,\,$ \\
$\,\,$\color{red} 0.9901\color{black} $\,\,$ & $\,\,$ 1 $\,\,$ & $\,\,$\color{red} 5.5594\color{black} $\,\,$ & $\,\,$\color{red} 1.9888\color{black}   $\,\,$ \\
$\,\,$0.1781$\,\,$ & $\,\,$\color{red} 0.1799\color{black} $\,\,$ & $\,\,$ 1 $\,\,$ & $\,\,$0.3577 $\,\,$ \\
$\,\,$0.4979$\,\,$ & $\,\,$\color{red} 0.5028\color{black} $\,\,$ & $\,\,$2.7954$\,\,$ & $\,\,$ 1  $\,\,$ \\
\end{pmatrix},
\end{equation*}

\begin{equation*}
\mathbf{w}^{\prime} =
\begin{pmatrix}
0.374296\\
0.372695\\
0.066662\\
0.186347
\end{pmatrix} =
0.997905\cdot
\begin{pmatrix}
0.375081\\
\color{gr} 0.373477\color{black} \\
0.066802\\
0.186739
\end{pmatrix},
\end{equation*}
\begin{equation*}
\left[ \frac{{w}^{\prime}_i}{{w}^{\prime}_j} \right] =
\begin{pmatrix}
$\,\,$ 1 $\,\,$ & $\,\,$\color{gr} 1.0043\color{black} $\,\,$ & $\,\,$5.6148$\,\,$ & $\,\,$2.0086$\,\,$ \\
$\,\,$\color{gr} 0.9957\color{black} $\,\,$ & $\,\,$ 1 $\,\,$ & $\,\,$\color{gr} 5.5908\color{black} $\,\,$ & $\,\,$\color{gr} \color{blue} 2\color{black}   $\,\,$ \\
$\,\,$0.1781$\,\,$ & $\,\,$\color{gr} 0.1789\color{black} $\,\,$ & $\,\,$ 1 $\,\,$ & $\,\,$0.3577 $\,\,$ \\
$\,\,$0.4979$\,\,$ & $\,\,$\color{gr} \color{blue}  1/2\color{black} $\,\,$ & $\,\,$2.7954$\,\,$ & $\,\,$ 1  $\,\,$ \\
\end{pmatrix},
\end{equation*}
\end{example}
\newpage
\begin{example}
\begin{equation*}
\mathbf{A} =
\begin{pmatrix}
$\,\,$ 1 $\,\,$ & $\,\,$1$\,\,$ & $\,\,$4$\,\,$ & $\,\,$3 $\,\,$ \\
$\,\,$ 1 $\,\,$ & $\,\,$ 1 $\,\,$ & $\,\,$6$\,\,$ & $\,\,$2 $\,\,$ \\
$\,\,$ 1/4$\,\,$ & $\,\,$ 1/6$\,\,$ & $\,\,$ 1 $\,\,$ & $\,\,$ 1/5 $\,\,$ \\
$\,\,$ 1/3$\,\,$ & $\,\,$ 1/2$\,\,$ & $\,\,$5$\,\,$ & $\,\,$ 1  $\,\,$ \\
\end{pmatrix},
\qquad
\lambda_{\max} =
4.1502,
\qquad
CR = 0.0566
\end{equation*}

\begin{equation*}
\mathbf{w}^{cos} =
\begin{pmatrix}
0.371405\\
\color{red} 0.365618\color{black} \\
0.063934\\
0.199044
\end{pmatrix}\end{equation*}
\begin{equation*}
\left[ \frac{{w}^{cos}_i}{{w}^{cos}_j} \right] =
\begin{pmatrix}
$\,\,$ 1 $\,\,$ & $\,\,$\color{red} 1.0158\color{black} $\,\,$ & $\,\,$5.8092$\,\,$ & $\,\,$1.8659$\,\,$ \\
$\,\,$\color{red} 0.9844\color{black} $\,\,$ & $\,\,$ 1 $\,\,$ & $\,\,$\color{red} 5.7187\color{black} $\,\,$ & $\,\,$\color{red} 1.8369\color{black}   $\,\,$ \\
$\,\,$0.1721$\,\,$ & $\,\,$\color{red} 0.1749\color{black} $\,\,$ & $\,\,$ 1 $\,\,$ & $\,\,$0.3212 $\,\,$ \\
$\,\,$0.5359$\,\,$ & $\,\,$\color{red} 0.5444\color{black} $\,\,$ & $\,\,$3.1133$\,\,$ & $\,\,$ 1  $\,\,$ \\
\end{pmatrix},
\end{equation*}

\begin{equation*}
\mathbf{w}^{\prime} =
\begin{pmatrix}
0.369268\\
0.369268\\
0.063566\\
0.197899
\end{pmatrix} =
0.994246\cdot
\begin{pmatrix}
0.371405\\
\color{gr} 0.371405\color{black} \\
0.063934\\
0.199044
\end{pmatrix},
\end{equation*}
\begin{equation*}
\left[ \frac{{w}^{\prime}_i}{{w}^{\prime}_j} \right] =
\begin{pmatrix}
$\,\,$ 1 $\,\,$ & $\,\,$\color{gr} \color{blue} 1\color{black} $\,\,$ & $\,\,$5.8092$\,\,$ & $\,\,$1.8659$\,\,$ \\
$\,\,$\color{gr} \color{blue} 1\color{black} $\,\,$ & $\,\,$ 1 $\,\,$ & $\,\,$\color{gr} 5.8092\color{black} $\,\,$ & $\,\,$\color{gr} 1.8659\color{black}   $\,\,$ \\
$\,\,$0.1721$\,\,$ & $\,\,$\color{gr} 0.1721\color{black} $\,\,$ & $\,\,$ 1 $\,\,$ & $\,\,$0.3212 $\,\,$ \\
$\,\,$0.5359$\,\,$ & $\,\,$\color{gr} 0.5359\color{black} $\,\,$ & $\,\,$3.1133$\,\,$ & $\,\,$ 1  $\,\,$ \\
\end{pmatrix},
\end{equation*}
\end{example}
\newpage
\begin{example}
\begin{equation*}
\mathbf{A} =
\begin{pmatrix}
$\,\,$ 1 $\,\,$ & $\,\,$1$\,\,$ & $\,\,$4$\,\,$ & $\,\,$3 $\,\,$ \\
$\,\,$ 1 $\,\,$ & $\,\,$ 1 $\,\,$ & $\,\,$6$\,\,$ & $\,\,$2 $\,\,$ \\
$\,\,$ 1/4$\,\,$ & $\,\,$ 1/6$\,\,$ & $\,\,$ 1 $\,\,$ & $\,\,$ 1/6 $\,\,$ \\
$\,\,$ 1/3$\,\,$ & $\,\,$ 1/2$\,\,$ & $\,\,$6$\,\,$ & $\,\,$ 1  $\,\,$ \\
\end{pmatrix},
\qquad
\lambda_{\max} =
4.1990,
\qquad
CR = 0.0750
\end{equation*}

\begin{equation*}
\mathbf{w}^{cos} =
\begin{pmatrix}
0.368255\\
\color{red} 0.360362\color{black} \\
0.061786\\
0.209596
\end{pmatrix}\end{equation*}
\begin{equation*}
\left[ \frac{{w}^{cos}_i}{{w}^{cos}_j} \right] =
\begin{pmatrix}
$\,\,$ 1 $\,\,$ & $\,\,$\color{red} 1.0219\color{black} $\,\,$ & $\,\,$5.9602$\,\,$ & $\,\,$1.7570$\,\,$ \\
$\,\,$\color{red} 0.9786\color{black} $\,\,$ & $\,\,$ 1 $\,\,$ & $\,\,$\color{red} 5.8324\color{black} $\,\,$ & $\,\,$\color{red} 1.7193\color{black}   $\,\,$ \\
$\,\,$0.1678$\,\,$ & $\,\,$\color{red} 0.1715\color{black} $\,\,$ & $\,\,$ 1 $\,\,$ & $\,\,$0.2948 $\,\,$ \\
$\,\,$0.5692$\,\,$ & $\,\,$\color{red} 0.5816\color{black} $\,\,$ & $\,\,$3.3923$\,\,$ & $\,\,$ 1  $\,\,$ \\
\end{pmatrix},
\end{equation*}

\begin{equation*}
\mathbf{w}^{\prime} =
\begin{pmatrix}
0.365372\\
0.365372\\
0.061302\\
0.207955
\end{pmatrix} =
0.992169\cdot
\begin{pmatrix}
0.368255\\
\color{gr} 0.368255\color{black} \\
0.061786\\
0.209596
\end{pmatrix},
\end{equation*}
\begin{equation*}
\left[ \frac{{w}^{\prime}_i}{{w}^{\prime}_j} \right] =
\begin{pmatrix}
$\,\,$ 1 $\,\,$ & $\,\,$\color{gr} \color{blue} 1\color{black} $\,\,$ & $\,\,$5.9602$\,\,$ & $\,\,$1.7570$\,\,$ \\
$\,\,$\color{gr} \color{blue} 1\color{black} $\,\,$ & $\,\,$ 1 $\,\,$ & $\,\,$\color{gr} 5.9602\color{black} $\,\,$ & $\,\,$\color{gr} 1.7570\color{black}   $\,\,$ \\
$\,\,$0.1678$\,\,$ & $\,\,$\color{gr} 0.1678\color{black} $\,\,$ & $\,\,$ 1 $\,\,$ & $\,\,$0.2948 $\,\,$ \\
$\,\,$0.5692$\,\,$ & $\,\,$\color{gr} 0.5692\color{black} $\,\,$ & $\,\,$3.3923$\,\,$ & $\,\,$ 1  $\,\,$ \\
\end{pmatrix},
\end{equation*}
\end{example}
\newpage
\begin{example}
\begin{equation*}
\mathbf{A} =
\begin{pmatrix}
$\,\,$ 1 $\,\,$ & $\,\,$1$\,\,$ & $\,\,$4$\,\,$ & $\,\,$3 $\,\,$ \\
$\,\,$ 1 $\,\,$ & $\,\,$ 1 $\,\,$ & $\,\,$6$\,\,$ & $\,\,$2 $\,\,$ \\
$\,\,$ 1/4$\,\,$ & $\,\,$ 1/6$\,\,$ & $\,\,$ 1 $\,\,$ & $\,\,$ 1/7 $\,\,$ \\
$\,\,$ 1/3$\,\,$ & $\,\,$ 1/2$\,\,$ & $\,\,$7$\,\,$ & $\,\,$ 1  $\,\,$ \\
\end{pmatrix},
\qquad
\lambda_{\max} =
4.2478,
\qquad
CR = 0.0935
\end{equation*}

\begin{equation*}
\mathbf{w}^{cos} =
\begin{pmatrix}
0.365622\\
\color{red} 0.355677\color{black} \\
0.060094\\
0.218607
\end{pmatrix}\end{equation*}
\begin{equation*}
\left[ \frac{{w}^{cos}_i}{{w}^{cos}_j} \right] =
\begin{pmatrix}
$\,\,$ 1 $\,\,$ & $\,\,$\color{red} 1.0280\color{black} $\,\,$ & $\,\,$6.0842$\,\,$ & $\,\,$1.6725$\,\,$ \\
$\,\,$\color{red} 0.9728\color{black} $\,\,$ & $\,\,$ 1 $\,\,$ & $\,\,$\color{red} 5.9187\color{black} $\,\,$ & $\,\,$\color{red} 1.6270\color{black}   $\,\,$ \\
$\,\,$0.1644$\,\,$ & $\,\,$\color{red} 0.1690\color{black} $\,\,$ & $\,\,$ 1 $\,\,$ & $\,\,$0.2749 $\,\,$ \\
$\,\,$0.5979$\,\,$ & $\,\,$\color{red} 0.6146\color{black} $\,\,$ & $\,\,$3.6378$\,\,$ & $\,\,$ 1  $\,\,$ \\
\end{pmatrix},
\end{equation*}

\begin{equation*}
\mathbf{w}^{\prime} =
\begin{pmatrix}
0.363844\\
0.358810\\
0.059802\\
0.217544
\end{pmatrix} =
0.995137\cdot
\begin{pmatrix}
0.365622\\
\color{gr} 0.360563\color{black} \\
0.060094\\
0.218607
\end{pmatrix},
\end{equation*}
\begin{equation*}
\left[ \frac{{w}^{\prime}_i}{{w}^{\prime}_j} \right] =
\begin{pmatrix}
$\,\,$ 1 $\,\,$ & $\,\,$\color{gr} 1.0140\color{black} $\,\,$ & $\,\,$6.0842$\,\,$ & $\,\,$1.6725$\,\,$ \\
$\,\,$\color{gr} 0.9862\color{black} $\,\,$ & $\,\,$ 1 $\,\,$ & $\,\,$\color{gr} \color{blue} 6\color{black} $\,\,$ & $\,\,$\color{gr} 1.6494\color{black}   $\,\,$ \\
$\,\,$0.1644$\,\,$ & $\,\,$\color{gr} \color{blue}  1/6\color{black} $\,\,$ & $\,\,$ 1 $\,\,$ & $\,\,$0.2749 $\,\,$ \\
$\,\,$0.5979$\,\,$ & $\,\,$\color{gr} 0.6063\color{black} $\,\,$ & $\,\,$3.6378$\,\,$ & $\,\,$ 1  $\,\,$ \\
\end{pmatrix},
\end{equation*}
\end{example}
\newpage
\begin{example}
\begin{equation*}
\mathbf{A} =
\begin{pmatrix}
$\,\,$ 1 $\,\,$ & $\,\,$1$\,\,$ & $\,\,$4$\,\,$ & $\,\,$4 $\,\,$ \\
$\,\,$ 1 $\,\,$ & $\,\,$ 1 $\,\,$ & $\,\,$5$\,\,$ & $\,\,$2 $\,\,$ \\
$\,\,$ 1/4$\,\,$ & $\,\,$ 1/5$\,\,$ & $\,\,$ 1 $\,\,$ & $\,\,$ 1/4 $\,\,$ \\
$\,\,$ 1/4$\,\,$ & $\,\,$ 1/2$\,\,$ & $\,\,$4$\,\,$ & $\,\,$ 1  $\,\,$ \\
\end{pmatrix},
\qquad
\lambda_{\max} =
4.1722,
\qquad
CR = 0.0649
\end{equation*}

\begin{equation*}
\mathbf{w}^{cos} =
\begin{pmatrix}
0.396636\\
\color{red} 0.352618\color{black} \\
0.070757\\
0.179990
\end{pmatrix}\end{equation*}
\begin{equation*}
\left[ \frac{{w}^{cos}_i}{{w}^{cos}_j} \right] =
\begin{pmatrix}
$\,\,$ 1 $\,\,$ & $\,\,$\color{red} 1.1248\color{black} $\,\,$ & $\,\,$5.6056$\,\,$ & $\,\,$2.2037$\,\,$ \\
$\,\,$\color{red} 0.8890\color{black} $\,\,$ & $\,\,$ 1 $\,\,$ & $\,\,$\color{red} 4.9835\color{black} $\,\,$ & $\,\,$\color{red} 1.9591\color{black}   $\,\,$ \\
$\,\,$0.1784$\,\,$ & $\,\,$\color{red} 0.2007\color{black} $\,\,$ & $\,\,$ 1 $\,\,$ & $\,\,$0.3931 $\,\,$ \\
$\,\,$0.4538$\,\,$ & $\,\,$\color{red} 0.5104\color{black} $\,\,$ & $\,\,$2.5438$\,\,$ & $\,\,$ 1  $\,\,$ \\
\end{pmatrix},
\end{equation*}

\begin{equation*}
\mathbf{w}^{\prime} =
\begin{pmatrix}
0.396173\\
0.353373\\
0.070675\\
0.179780
\end{pmatrix} =
0.998834\cdot
\begin{pmatrix}
0.396636\\
\color{gr} 0.353786\color{black} \\
0.070757\\
0.179990
\end{pmatrix},
\end{equation*}
\begin{equation*}
\left[ \frac{{w}^{\prime}_i}{{w}^{\prime}_j} \right] =
\begin{pmatrix}
$\,\,$ 1 $\,\,$ & $\,\,$\color{gr} 1.1211\color{black} $\,\,$ & $\,\,$5.6056$\,\,$ & $\,\,$2.2037$\,\,$ \\
$\,\,$\color{gr} 0.8920\color{black} $\,\,$ & $\,\,$ 1 $\,\,$ & $\,\,$\color{gr} \color{blue} 5\color{black} $\,\,$ & $\,\,$\color{gr} 1.9656\color{black}   $\,\,$ \\
$\,\,$0.1784$\,\,$ & $\,\,$\color{gr} \color{blue}  1/5\color{black} $\,\,$ & $\,\,$ 1 $\,\,$ & $\,\,$0.3931 $\,\,$ \\
$\,\,$0.4538$\,\,$ & $\,\,$\color{gr} 0.5088\color{black} $\,\,$ & $\,\,$2.5438$\,\,$ & $\,\,$ 1  $\,\,$ \\
\end{pmatrix},
\end{equation*}
\end{example}
\newpage
\begin{example}
\begin{equation*}
\mathbf{A} =
\begin{pmatrix}
$\,\,$ 1 $\,\,$ & $\,\,$1$\,\,$ & $\,\,$4$\,\,$ & $\,\,$4 $\,\,$ \\
$\,\,$ 1 $\,\,$ & $\,\,$ 1 $\,\,$ & $\,\,$6$\,\,$ & $\,\,$2 $\,\,$ \\
$\,\,$ 1/4$\,\,$ & $\,\,$ 1/6$\,\,$ & $\,\,$ 1 $\,\,$ & $\,\,$ 1/5 $\,\,$ \\
$\,\,$ 1/4$\,\,$ & $\,\,$ 1/2$\,\,$ & $\,\,$5$\,\,$ & $\,\,$ 1  $\,\,$ \\
\end{pmatrix},
\qquad
\lambda_{\max} =
4.2277,
\qquad
CR = 0.0859
\end{equation*}

\begin{equation*}
\mathbf{w}^{cos} =
\begin{pmatrix}
0.389546\\
\color{red} 0.359011\color{black} \\
0.063929\\
0.187513
\end{pmatrix}\end{equation*}
\begin{equation*}
\left[ \frac{{w}^{cos}_i}{{w}^{cos}_j} \right] =
\begin{pmatrix}
$\,\,$ 1 $\,\,$ & $\,\,$\color{red} 1.0851\color{black} $\,\,$ & $\,\,$6.0934$\,\,$ & $\,\,$2.0774$\,\,$ \\
$\,\,$\color{red} 0.9216\color{black} $\,\,$ & $\,\,$ 1 $\,\,$ & $\,\,$\color{red} 5.6158\color{black} $\,\,$ & $\,\,$\color{red} 1.9146\color{black}   $\,\,$ \\
$\,\,$0.1641$\,\,$ & $\,\,$\color{red} 0.1781\color{black} $\,\,$ & $\,\,$ 1 $\,\,$ & $\,\,$0.3409 $\,\,$ \\
$\,\,$0.4814$\,\,$ & $\,\,$\color{red} 0.5223\color{black} $\,\,$ & $\,\,$2.9331$\,\,$ & $\,\,$ 1  $\,\,$ \\
\end{pmatrix},
\end{equation*}

\begin{equation*}
\mathbf{w}^{\prime} =
\begin{pmatrix}
0.383406\\
0.369115\\
0.062922\\
0.184558
\end{pmatrix} =
0.984238\cdot
\begin{pmatrix}
0.389546\\
\color{gr} 0.375026\color{black} \\
0.063929\\
0.187513
\end{pmatrix},
\end{equation*}
\begin{equation*}
\left[ \frac{{w}^{\prime}_i}{{w}^{\prime}_j} \right] =
\begin{pmatrix}
$\,\,$ 1 $\,\,$ & $\,\,$\color{gr} 1.0387\color{black} $\,\,$ & $\,\,$6.0934$\,\,$ & $\,\,$2.0774$\,\,$ \\
$\,\,$\color{gr} 0.9627\color{black} $\,\,$ & $\,\,$ 1 $\,\,$ & $\,\,$\color{gr} 5.8663\color{black} $\,\,$ & $\,\,$\color{gr} \color{blue} 2\color{black}   $\,\,$ \\
$\,\,$0.1641$\,\,$ & $\,\,$\color{gr} 0.1705\color{black} $\,\,$ & $\,\,$ 1 $\,\,$ & $\,\,$0.3409 $\,\,$ \\
$\,\,$0.4814$\,\,$ & $\,\,$\color{gr} \color{blue}  1/2\color{black} $\,\,$ & $\,\,$2.9331$\,\,$ & $\,\,$ 1  $\,\,$ \\
\end{pmatrix},
\end{equation*}
\end{example}
\newpage
\begin{example}
\begin{equation*}
\mathbf{A} =
\begin{pmatrix}
$\,\,$ 1 $\,\,$ & $\,\,$1$\,\,$ & $\,\,$4$\,\,$ & $\,\,$5 $\,\,$ \\
$\,\,$ 1 $\,\,$ & $\,\,$ 1 $\,\,$ & $\,\,$6$\,\,$ & $\,\,$3 $\,\,$ \\
$\,\,$ 1/4$\,\,$ & $\,\,$ 1/6$\,\,$ & $\,\,$ 1 $\,\,$ & $\,\,$ 1/3 $\,\,$ \\
$\,\,$ 1/5$\,\,$ & $\,\,$ 1/3$\,\,$ & $\,\,$3$\,\,$ & $\,\,$ 1  $\,\,$ \\
\end{pmatrix},
\qquad
\lambda_{\max} =
4.1502,
\qquad
CR = 0.0566
\end{equation*}

\begin{equation*}
\mathbf{w}^{cos} =
\begin{pmatrix}
0.403608\\
\color{red} 0.391237\color{black} \\
0.069552\\
0.135603
\end{pmatrix}\end{equation*}
\begin{equation*}
\left[ \frac{{w}^{cos}_i}{{w}^{cos}_j} \right] =
\begin{pmatrix}
$\,\,$ 1 $\,\,$ & $\,\,$\color{red} 1.0316\color{black} $\,\,$ & $\,\,$5.8030$\,\,$ & $\,\,$2.9764$\,\,$ \\
$\,\,$\color{red} 0.9694\color{black} $\,\,$ & $\,\,$ 1 $\,\,$ & $\,\,$\color{red} 5.6251\color{black} $\,\,$ & $\,\,$\color{red} 2.8852\color{black}   $\,\,$ \\
$\,\,$0.1723$\,\,$ & $\,\,$\color{red} 0.1778\color{black} $\,\,$ & $\,\,$ 1 $\,\,$ & $\,\,$0.5129 $\,\,$ \\
$\,\,$0.3360$\,\,$ & $\,\,$\color{red} 0.3466\color{black} $\,\,$ & $\,\,$1.9497$\,\,$ & $\,\,$ 1  $\,\,$ \\
\end{pmatrix},
\end{equation*}

\begin{equation*}
\mathbf{w}^{\prime} =
\begin{pmatrix}
0.398676\\
0.398676\\
0.068702\\
0.133946
\end{pmatrix} =
0.987781\cdot
\begin{pmatrix}
0.403608\\
\color{gr} 0.403608\color{black} \\
0.069552\\
0.135603
\end{pmatrix},
\end{equation*}
\begin{equation*}
\left[ \frac{{w}^{\prime}_i}{{w}^{\prime}_j} \right] =
\begin{pmatrix}
$\,\,$ 1 $\,\,$ & $\,\,$\color{gr} \color{blue} 1\color{black} $\,\,$ & $\,\,$5.8030$\,\,$ & $\,\,$2.9764$\,\,$ \\
$\,\,$\color{gr} \color{blue} 1\color{black} $\,\,$ & $\,\,$ 1 $\,\,$ & $\,\,$\color{gr} 5.8030\color{black} $\,\,$ & $\,\,$\color{gr} 2.9764\color{black}   $\,\,$ \\
$\,\,$0.1723$\,\,$ & $\,\,$\color{gr} 0.1723\color{black} $\,\,$ & $\,\,$ 1 $\,\,$ & $\,\,$0.5129 $\,\,$ \\
$\,\,$0.3360$\,\,$ & $\,\,$\color{gr} 0.3360\color{black} $\,\,$ & $\,\,$1.9497$\,\,$ & $\,\,$ 1  $\,\,$ \\
\end{pmatrix},
\end{equation*}
\end{example}
\newpage
\begin{example}
\begin{equation*}
\mathbf{A} =
\begin{pmatrix}
$\,\,$ 1 $\,\,$ & $\,\,$1$\,\,$ & $\,\,$4$\,\,$ & $\,\,$5 $\,\,$ \\
$\,\,$ 1 $\,\,$ & $\,\,$ 1 $\,\,$ & $\,\,$6$\,\,$ & $\,\,$3 $\,\,$ \\
$\,\,$ 1/4$\,\,$ & $\,\,$ 1/6$\,\,$ & $\,\,$ 1 $\,\,$ & $\,\,$ 1/4 $\,\,$ \\
$\,\,$ 1/5$\,\,$ & $\,\,$ 1/3$\,\,$ & $\,\,$4$\,\,$ & $\,\,$ 1  $\,\,$ \\
\end{pmatrix},
\qquad
\lambda_{\max} =
4.2277,
\qquad
CR = 0.0859
\end{equation*}

\begin{equation*}
\mathbf{w}^{cos} =
\begin{pmatrix}
0.398901\\
\color{red} 0.384569\color{black} \\
0.066349\\
0.150181
\end{pmatrix}\end{equation*}
\begin{equation*}
\left[ \frac{{w}^{cos}_i}{{w}^{cos}_j} \right] =
\begin{pmatrix}
$\,\,$ 1 $\,\,$ & $\,\,$\color{red} 1.0373\color{black} $\,\,$ & $\,\,$6.0122$\,\,$ & $\,\,$2.6561$\,\,$ \\
$\,\,$\color{red} 0.9641\color{black} $\,\,$ & $\,\,$ 1 $\,\,$ & $\,\,$\color{red} 5.7962\color{black} $\,\,$ & $\,\,$\color{red} 2.5607\color{black}   $\,\,$ \\
$\,\,$0.1663$\,\,$ & $\,\,$\color{red} 0.1725\color{black} $\,\,$ & $\,\,$ 1 $\,\,$ & $\,\,$0.4418 $\,\,$ \\
$\,\,$0.3765$\,\,$ & $\,\,$\color{red} 0.3905\color{black} $\,\,$ & $\,\,$2.2635$\,\,$ & $\,\,$ 1  $\,\,$ \\
\end{pmatrix},
\end{equation*}

\begin{equation*}
\mathbf{w}^{\prime} =
\begin{pmatrix}
0.393579\\
0.392780\\
0.065463\\
0.148178
\end{pmatrix} =
0.986658\cdot
\begin{pmatrix}
0.398901\\
\color{gr} 0.398091\color{black} \\
0.066349\\
0.150181
\end{pmatrix},
\end{equation*}
\begin{equation*}
\left[ \frac{{w}^{\prime}_i}{{w}^{\prime}_j} \right] =
\begin{pmatrix}
$\,\,$ 1 $\,\,$ & $\,\,$\color{gr} 1.0020\color{black} $\,\,$ & $\,\,$6.0122$\,\,$ & $\,\,$2.6561$\,\,$ \\
$\,\,$\color{gr} 0.9980\color{black} $\,\,$ & $\,\,$ 1 $\,\,$ & $\,\,$\color{gr} \color{blue} 6\color{black} $\,\,$ & $\,\,$\color{gr} 2.6507\color{black}   $\,\,$ \\
$\,\,$0.1663$\,\,$ & $\,\,$\color{gr} \color{blue}  1/6\color{black} $\,\,$ & $\,\,$ 1 $\,\,$ & $\,\,$0.4418 $\,\,$ \\
$\,\,$0.3765$\,\,$ & $\,\,$\color{gr} 0.3773\color{black} $\,\,$ & $\,\,$2.2635$\,\,$ & $\,\,$ 1  $\,\,$ \\
\end{pmatrix},
\end{equation*}
\end{example}
\newpage
\begin{example}
\begin{equation*}
\mathbf{A} =
\begin{pmatrix}
$\,\,$ 1 $\,\,$ & $\,\,$1$\,\,$ & $\,\,$4$\,\,$ & $\,\,$5 $\,\,$ \\
$\,\,$ 1 $\,\,$ & $\,\,$ 1 $\,\,$ & $\,\,$7$\,\,$ & $\,\,$3 $\,\,$ \\
$\,\,$ 1/4$\,\,$ & $\,\,$ 1/7$\,\,$ & $\,\,$ 1 $\,\,$ & $\,\,$ 1/4 $\,\,$ \\
$\,\,$ 1/5$\,\,$ & $\,\,$ 1/3$\,\,$ & $\,\,$4$\,\,$ & $\,\,$ 1  $\,\,$ \\
\end{pmatrix},
\qquad
\lambda_{\max} =
4.2251,
\qquad
CR = 0.0849
\end{equation*}

\begin{equation*}
\mathbf{w}^{cos} =
\begin{pmatrix}
0.396251\\
\color{red} 0.395288\color{black} \\
0.062983\\
0.145478
\end{pmatrix}\end{equation*}
\begin{equation*}
\left[ \frac{{w}^{cos}_i}{{w}^{cos}_j} \right] =
\begin{pmatrix}
$\,\,$ 1 $\,\,$ & $\,\,$\color{red} 1.0024\color{black} $\,\,$ & $\,\,$6.2914$\,\,$ & $\,\,$2.7238$\,\,$ \\
$\,\,$\color{red} 0.9976\color{black} $\,\,$ & $\,\,$ 1 $\,\,$ & $\,\,$\color{red} 6.2761\color{black} $\,\,$ & $\,\,$\color{red} 2.7172\color{black}   $\,\,$ \\
$\,\,$0.1589$\,\,$ & $\,\,$\color{red} 0.1593\color{black} $\,\,$ & $\,\,$ 1 $\,\,$ & $\,\,$0.4329 $\,\,$ \\
$\,\,$0.3671$\,\,$ & $\,\,$\color{red} 0.3680\color{black} $\,\,$ & $\,\,$2.3098$\,\,$ & $\,\,$ 1  $\,\,$ \\
\end{pmatrix},
\end{equation*}

\begin{equation*}
\mathbf{w}^{\prime} =
\begin{pmatrix}
0.395870\\
0.395870\\
0.062922\\
0.145339
\end{pmatrix} =
0.999038\cdot
\begin{pmatrix}
0.396251\\
\color{gr} 0.396251\color{black} \\
0.062983\\
0.145478
\end{pmatrix},
\end{equation*}
\begin{equation*}
\left[ \frac{{w}^{\prime}_i}{{w}^{\prime}_j} \right] =
\begin{pmatrix}
$\,\,$ 1 $\,\,$ & $\,\,$\color{gr} \color{blue} 1\color{black} $\,\,$ & $\,\,$6.2914$\,\,$ & $\,\,$2.7238$\,\,$ \\
$\,\,$\color{gr} \color{blue} 1\color{black} $\,\,$ & $\,\,$ 1 $\,\,$ & $\,\,$\color{gr} 6.2914\color{black} $\,\,$ & $\,\,$\color{gr} 2.7238\color{black}   $\,\,$ \\
$\,\,$0.1589$\,\,$ & $\,\,$\color{gr} 0.1589\color{black} $\,\,$ & $\,\,$ 1 $\,\,$ & $\,\,$0.4329 $\,\,$ \\
$\,\,$0.3671$\,\,$ & $\,\,$\color{gr} 0.3671\color{black} $\,\,$ & $\,\,$2.3098$\,\,$ & $\,\,$ 1  $\,\,$ \\
\end{pmatrix},
\end{equation*}
\end{example}
\newpage
\begin{example}
\begin{equation*}
\mathbf{A} =
\begin{pmatrix}
$\,\,$ 1 $\,\,$ & $\,\,$1$\,\,$ & $\,\,$4$\,\,$ & $\,\,$6 $\,\,$ \\
$\,\,$ 1 $\,\,$ & $\,\,$ 1 $\,\,$ & $\,\,$6$\,\,$ & $\,\,$3 $\,\,$ \\
$\,\,$ 1/4$\,\,$ & $\,\,$ 1/6$\,\,$ & $\,\,$ 1 $\,\,$ & $\,\,$ 1/3 $\,\,$ \\
$\,\,$ 1/6$\,\,$ & $\,\,$ 1/3$\,\,$ & $\,\,$3$\,\,$ & $\,\,$ 1  $\,\,$ \\
\end{pmatrix},
\qquad
\lambda_{\max} =
4.1990,
\qquad
CR = 0.0750
\end{equation*}

\begin{equation*}
\mathbf{w}^{cos} =
\begin{pmatrix}
0.414182\\
\color{red} 0.386073\color{black} \\
0.069299\\
0.130446
\end{pmatrix}\end{equation*}
\begin{equation*}
\left[ \frac{{w}^{cos}_i}{{w}^{cos}_j} \right] =
\begin{pmatrix}
$\,\,$ 1 $\,\,$ & $\,\,$\color{red} 1.0728\color{black} $\,\,$ & $\,\,$5.9767$\,\,$ & $\,\,$3.1751$\,\,$ \\
$\,\,$\color{red} 0.9321\color{black} $\,\,$ & $\,\,$ 1 $\,\,$ & $\,\,$\color{red} 5.5711\color{black} $\,\,$ & $\,\,$\color{red} 2.9596\color{black}   $\,\,$ \\
$\,\,$0.1673$\,\,$ & $\,\,$\color{red} 0.1795\color{black} $\,\,$ & $\,\,$ 1 $\,\,$ & $\,\,$0.5313 $\,\,$ \\
$\,\,$0.3149$\,\,$ & $\,\,$\color{red} 0.3379\color{black} $\,\,$ & $\,\,$1.8823$\,\,$ & $\,\,$ 1  $\,\,$ \\
\end{pmatrix},
\end{equation*}

\begin{equation*}
\mathbf{w}^{\prime} =
\begin{pmatrix}
0.412014\\
0.389287\\
0.068936\\
0.129762
\end{pmatrix} =
0.994764\cdot
\begin{pmatrix}
0.414182\\
\color{gr} 0.391337\color{black} \\
0.069299\\
0.130446
\end{pmatrix},
\end{equation*}
\begin{equation*}
\left[ \frac{{w}^{\prime}_i}{{w}^{\prime}_j} \right] =
\begin{pmatrix}
$\,\,$ 1 $\,\,$ & $\,\,$\color{gr} 1.0584\color{black} $\,\,$ & $\,\,$5.9767$\,\,$ & $\,\,$3.1751$\,\,$ \\
$\,\,$\color{gr} 0.9448\color{black} $\,\,$ & $\,\,$ 1 $\,\,$ & $\,\,$\color{gr} 5.6470\color{black} $\,\,$ & $\,\,$\color{gr} \color{blue} 3\color{black}   $\,\,$ \\
$\,\,$0.1673$\,\,$ & $\,\,$\color{gr} 0.1771\color{black} $\,\,$ & $\,\,$ 1 $\,\,$ & $\,\,$0.5313 $\,\,$ \\
$\,\,$0.3149$\,\,$ & $\,\,$\color{gr} \color{blue}  1/3\color{black} $\,\,$ & $\,\,$1.8823$\,\,$ & $\,\,$ 1  $\,\,$ \\
\end{pmatrix},
\end{equation*}
\end{example}
\newpage
\begin{example}
\begin{equation*}
\mathbf{A} =
\begin{pmatrix}
$\,\,$ 1 $\,\,$ & $\,\,$1$\,\,$ & $\,\,$4$\,\,$ & $\,\,$6 $\,\,$ \\
$\,\,$ 1 $\,\,$ & $\,\,$ 1 $\,\,$ & $\,\,$6$\,\,$ & $\,\,$4 $\,\,$ \\
$\,\,$ 1/4$\,\,$ & $\,\,$ 1/6$\,\,$ & $\,\,$ 1 $\,\,$ & $\,\,$ 1/2 $\,\,$ \\
$\,\,$ 1/6$\,\,$ & $\,\,$ 1/4$\,\,$ & $\,\,$2$\,\,$ & $\,\,$ 1  $\,\,$ \\
\end{pmatrix},
\qquad
\lambda_{\max} =
4.1031,
\qquad
CR = 0.0389
\end{equation*}

\begin{equation*}
\mathbf{w}^{cos} =
\begin{pmatrix}
0.411899\\
\color{red} 0.410518\color{black} \\
0.073660\\
0.103923
\end{pmatrix}\end{equation*}
\begin{equation*}
\left[ \frac{{w}^{cos}_i}{{w}^{cos}_j} \right] =
\begin{pmatrix}
$\,\,$ 1 $\,\,$ & $\,\,$\color{red} 1.0034\color{black} $\,\,$ & $\,\,$5.5919$\,\,$ & $\,\,$3.9635$\,\,$ \\
$\,\,$\color{red} 0.9966\color{black} $\,\,$ & $\,\,$ 1 $\,\,$ & $\,\,$\color{red} 5.5732\color{black} $\,\,$ & $\,\,$\color{red} 3.9502\color{black}   $\,\,$ \\
$\,\,$0.1788$\,\,$ & $\,\,$\color{red} 0.1794\color{black} $\,\,$ & $\,\,$ 1 $\,\,$ & $\,\,$0.7088 $\,\,$ \\
$\,\,$0.2523$\,\,$ & $\,\,$\color{red} 0.2532\color{black} $\,\,$ & $\,\,$1.4108$\,\,$ & $\,\,$ 1  $\,\,$ \\
\end{pmatrix},
\end{equation*}

\begin{equation*}
\mathbf{w}^{\prime} =
\begin{pmatrix}
0.411331\\
0.411331\\
0.073558\\
0.103779
\end{pmatrix} =
0.998621\cdot
\begin{pmatrix}
0.411899\\
\color{gr} 0.411899\color{black} \\
0.073660\\
0.103923
\end{pmatrix},
\end{equation*}
\begin{equation*}
\left[ \frac{{w}^{\prime}_i}{{w}^{\prime}_j} \right] =
\begin{pmatrix}
$\,\,$ 1 $\,\,$ & $\,\,$\color{gr} \color{blue} 1\color{black} $\,\,$ & $\,\,$5.5919$\,\,$ & $\,\,$3.9635$\,\,$ \\
$\,\,$\color{gr} \color{blue} 1\color{black} $\,\,$ & $\,\,$ 1 $\,\,$ & $\,\,$\color{gr} 5.5919\color{black} $\,\,$ & $\,\,$\color{gr} 3.9635\color{black}   $\,\,$ \\
$\,\,$0.1788$\,\,$ & $\,\,$\color{gr} 0.1788\color{black} $\,\,$ & $\,\,$ 1 $\,\,$ & $\,\,$0.7088 $\,\,$ \\
$\,\,$0.2523$\,\,$ & $\,\,$\color{gr} 0.2523\color{black} $\,\,$ & $\,\,$1.4108$\,\,$ & $\,\,$ 1  $\,\,$ \\
\end{pmatrix},
\end{equation*}
\end{example}
\newpage
\begin{example}
\begin{equation*}
\mathbf{A} =
\begin{pmatrix}
$\,\,$ 1 $\,\,$ & $\,\,$1$\,\,$ & $\,\,$4$\,\,$ & $\,\,$6 $\,\,$ \\
$\,\,$ 1 $\,\,$ & $\,\,$ 1 $\,\,$ & $\,\,$6$\,\,$ & $\,\,$4 $\,\,$ \\
$\,\,$ 1/4$\,\,$ & $\,\,$ 1/6$\,\,$ & $\,\,$ 1 $\,\,$ & $\,\,$ 1/3 $\,\,$ \\
$\,\,$ 1/6$\,\,$ & $\,\,$ 1/4$\,\,$ & $\,\,$3$\,\,$ & $\,\,$ 1  $\,\,$ \\
\end{pmatrix},
\qquad
\lambda_{\max} =
4.1990,
\qquad
CR = 0.0750
\end{equation*}

\begin{equation*}
\mathbf{w}^{cos} =
\begin{pmatrix}
0.406615\\
\color{red} 0.403554\color{black} \\
0.069042\\
0.120789
\end{pmatrix}\end{equation*}
\begin{equation*}
\left[ \frac{{w}^{cos}_i}{{w}^{cos}_j} \right] =
\begin{pmatrix}
$\,\,$ 1 $\,\,$ & $\,\,$\color{red} 1.0076\color{black} $\,\,$ & $\,\,$5.8894$\,\,$ & $\,\,$3.3663$\,\,$ \\
$\,\,$\color{red} 0.9925\color{black} $\,\,$ & $\,\,$ 1 $\,\,$ & $\,\,$\color{red} 5.8451\color{black} $\,\,$ & $\,\,$\color{red} 3.3410\color{black}   $\,\,$ \\
$\,\,$0.1698$\,\,$ & $\,\,$\color{red} 0.1711\color{black} $\,\,$ & $\,\,$ 1 $\,\,$ & $\,\,$0.5716 $\,\,$ \\
$\,\,$0.2971$\,\,$ & $\,\,$\color{red} 0.2993\color{black} $\,\,$ & $\,\,$1.7495$\,\,$ & $\,\,$ 1  $\,\,$ \\
\end{pmatrix},
\end{equation*}

\begin{equation*}
\mathbf{w}^{\prime} =
\begin{pmatrix}
0.405374\\
0.405374\\
0.068831\\
0.120421
\end{pmatrix} =
0.996949\cdot
\begin{pmatrix}
0.406615\\
\color{gr} 0.406615\color{black} \\
0.069042\\
0.120789
\end{pmatrix},
\end{equation*}
\begin{equation*}
\left[ \frac{{w}^{\prime}_i}{{w}^{\prime}_j} \right] =
\begin{pmatrix}
$\,\,$ 1 $\,\,$ & $\,\,$\color{gr} \color{blue} 1\color{black} $\,\,$ & $\,\,$5.8894$\,\,$ & $\,\,$3.3663$\,\,$ \\
$\,\,$\color{gr} \color{blue} 1\color{black} $\,\,$ & $\,\,$ 1 $\,\,$ & $\,\,$\color{gr} 5.8894\color{black} $\,\,$ & $\,\,$\color{gr} 3.3663\color{black}   $\,\,$ \\
$\,\,$0.1698$\,\,$ & $\,\,$\color{gr} 0.1698\color{black} $\,\,$ & $\,\,$ 1 $\,\,$ & $\,\,$0.5716 $\,\,$ \\
$\,\,$0.2971$\,\,$ & $\,\,$\color{gr} 0.2971\color{black} $\,\,$ & $\,\,$1.7495$\,\,$ & $\,\,$ 1  $\,\,$ \\
\end{pmatrix},
\end{equation*}
\end{example}
\newpage
\begin{example}
\begin{equation*}
\mathbf{A} =
\begin{pmatrix}
$\,\,$ 1 $\,\,$ & $\,\,$1$\,\,$ & $\,\,$4$\,\,$ & $\,\,$7 $\,\,$ \\
$\,\,$ 1 $\,\,$ & $\,\,$ 1 $\,\,$ & $\,\,$6$\,\,$ & $\,\,$4 $\,\,$ \\
$\,\,$ 1/4$\,\,$ & $\,\,$ 1/6$\,\,$ & $\,\,$ 1 $\,\,$ & $\,\,$ 1/3 $\,\,$ \\
$\,\,$ 1/7$\,\,$ & $\,\,$ 1/4$\,\,$ & $\,\,$3$\,\,$ & $\,\,$ 1  $\,\,$ \\
\end{pmatrix},
\qquad
\lambda_{\max} =
4.2421,
\qquad
CR = 0.0913
\end{equation*}

\begin{equation*}
\mathbf{w}^{cos} =
\begin{pmatrix}
0.415658\\
\color{red} 0.398304\color{black} \\
0.068912\\
0.117127
\end{pmatrix}\end{equation*}
\begin{equation*}
\left[ \frac{{w}^{cos}_i}{{w}^{cos}_j} \right] =
\begin{pmatrix}
$\,\,$ 1 $\,\,$ & $\,\,$\color{red} 1.0436\color{black} $\,\,$ & $\,\,$6.0318$\,\,$ & $\,\,$3.5488$\,\,$ \\
$\,\,$\color{red} 0.9582\color{black} $\,\,$ & $\,\,$ 1 $\,\,$ & $\,\,$\color{red} 5.7799\color{black} $\,\,$ & $\,\,$\color{red} 3.4006\color{black}   $\,\,$ \\
$\,\,$0.1658$\,\,$ & $\,\,$\color{red} 0.1730\color{black} $\,\,$ & $\,\,$ 1 $\,\,$ & $\,\,$0.5884 $\,\,$ \\
$\,\,$0.2818$\,\,$ & $\,\,$\color{red} 0.2941\color{black} $\,\,$ & $\,\,$1.6997$\,\,$ & $\,\,$ 1  $\,\,$ \\
\end{pmatrix},
\end{equation*}

\begin{equation*}
\mathbf{w}^{\prime} =
\begin{pmatrix}
0.409448\\
0.407293\\
0.067882\\
0.115377
\end{pmatrix} =
0.985060\cdot
\begin{pmatrix}
0.415658\\
\color{gr} 0.413470\color{black} \\
0.068912\\
0.117127
\end{pmatrix},
\end{equation*}
\begin{equation*}
\left[ \frac{{w}^{\prime}_i}{{w}^{\prime}_j} \right] =
\begin{pmatrix}
$\,\,$ 1 $\,\,$ & $\,\,$\color{gr} 1.0053\color{black} $\,\,$ & $\,\,$6.0318$\,\,$ & $\,\,$3.5488$\,\,$ \\
$\,\,$\color{gr} 0.9947\color{black} $\,\,$ & $\,\,$ 1 $\,\,$ & $\,\,$\color{gr} \color{blue} 6\color{black} $\,\,$ & $\,\,$\color{gr} 3.5301\color{black}   $\,\,$ \\
$\,\,$0.1658$\,\,$ & $\,\,$\color{gr} \color{blue}  1/6\color{black} $\,\,$ & $\,\,$ 1 $\,\,$ & $\,\,$0.5884 $\,\,$ \\
$\,\,$0.2818$\,\,$ & $\,\,$\color{gr} 0.2833\color{black} $\,\,$ & $\,\,$1.6997$\,\,$ & $\,\,$ 1  $\,\,$ \\
\end{pmatrix},
\end{equation*}
\end{example}
\newpage
\begin{example}
\begin{equation*}
\mathbf{A} =
\begin{pmatrix}
$\,\,$ 1 $\,\,$ & $\,\,$1$\,\,$ & $\,\,$4$\,\,$ & $\,\,$7 $\,\,$ \\
$\,\,$ 1 $\,\,$ & $\,\,$ 1 $\,\,$ & $\,\,$7$\,\,$ & $\,\,$4 $\,\,$ \\
$\,\,$ 1/4$\,\,$ & $\,\,$ 1/7$\,\,$ & $\,\,$ 1 $\,\,$ & $\,\,$ 1/3 $\,\,$ \\
$\,\,$ 1/7$\,\,$ & $\,\,$ 1/4$\,\,$ & $\,\,$3$\,\,$ & $\,\,$ 1  $\,\,$ \\
\end{pmatrix},
\qquad
\lambda_{\max} =
4.2395,
\qquad
CR = 0.0903
\end{equation*}

\begin{equation*}
\mathbf{w}^{cos} =
\begin{pmatrix}
0.412586\\
\color{red} 0.409150\color{black} \\
0.065298\\
0.112966
\end{pmatrix}\end{equation*}
\begin{equation*}
\left[ \frac{{w}^{cos}_i}{{w}^{cos}_j} \right] =
\begin{pmatrix}
$\,\,$ 1 $\,\,$ & $\,\,$\color{red} 1.0084\color{black} $\,\,$ & $\,\,$6.3185$\,\,$ & $\,\,$3.6523$\,\,$ \\
$\,\,$\color{red} 0.9917\color{black} $\,\,$ & $\,\,$ 1 $\,\,$ & $\,\,$\color{red} 6.2659\color{black} $\,\,$ & $\,\,$\color{red} 3.6219\color{black}   $\,\,$ \\
$\,\,$0.1583$\,\,$ & $\,\,$\color{red} 0.1596\color{black} $\,\,$ & $\,\,$ 1 $\,\,$ & $\,\,$0.5780 $\,\,$ \\
$\,\,$0.2738$\,\,$ & $\,\,$\color{red} 0.2761\color{black} $\,\,$ & $\,\,$1.7300$\,\,$ & $\,\,$ 1  $\,\,$ \\
\end{pmatrix},
\end{equation*}

\begin{equation*}
\mathbf{w}^{\prime} =
\begin{pmatrix}
0.411173\\
0.411173\\
0.065074\\
0.112579
\end{pmatrix} =
0.996576\cdot
\begin{pmatrix}
0.412586\\
\color{gr} 0.412586\color{black} \\
0.065298\\
0.112966
\end{pmatrix},
\end{equation*}
\begin{equation*}
\left[ \frac{{w}^{\prime}_i}{{w}^{\prime}_j} \right] =
\begin{pmatrix}
$\,\,$ 1 $\,\,$ & $\,\,$\color{gr} \color{blue} 1\color{black} $\,\,$ & $\,\,$6.3185$\,\,$ & $\,\,$3.6523$\,\,$ \\
$\,\,$\color{gr} \color{blue} 1\color{black} $\,\,$ & $\,\,$ 1 $\,\,$ & $\,\,$\color{gr} 6.3185\color{black} $\,\,$ & $\,\,$\color{gr} 3.6523\color{black}   $\,\,$ \\
$\,\,$0.1583$\,\,$ & $\,\,$\color{gr} 0.1583\color{black} $\,\,$ & $\,\,$ 1 $\,\,$ & $\,\,$0.5780 $\,\,$ \\
$\,\,$0.2738$\,\,$ & $\,\,$\color{gr} 0.2738\color{black} $\,\,$ & $\,\,$1.7300$\,\,$ & $\,\,$ 1  $\,\,$ \\
\end{pmatrix},
\end{equation*}
\end{example}
\newpage
\begin{example}
\begin{equation*}
\mathbf{A} =
\begin{pmatrix}
$\,\,$ 1 $\,\,$ & $\,\,$1$\,\,$ & $\,\,$4$\,\,$ & $\,\,$8 $\,\,$ \\
$\,\,$ 1 $\,\,$ & $\,\,$ 1 $\,\,$ & $\,\,$5$\,\,$ & $\,\,$4 $\,\,$ \\
$\,\,$ 1/4$\,\,$ & $\,\,$ 1/5$\,\,$ & $\,\,$ 1 $\,\,$ & $\,\,$ 1/2 $\,\,$ \\
$\,\,$ 1/8$\,\,$ & $\,\,$ 1/4$\,\,$ & $\,\,$2$\,\,$ & $\,\,$ 1  $\,\,$ \\
\end{pmatrix},
\qquad
\lambda_{\max} =
4.1722,
\qquad
CR = 0.0649
\end{equation*}

\begin{equation*}
\mathbf{w}^{cos} =
\begin{pmatrix}
0.433156\\
\color{red} 0.388288\color{black} \\
0.077870\\
0.100686
\end{pmatrix}\end{equation*}
\begin{equation*}
\left[ \frac{{w}^{cos}_i}{{w}^{cos}_j} \right] =
\begin{pmatrix}
$\,\,$ 1 $\,\,$ & $\,\,$\color{red} 1.1156\color{black} $\,\,$ & $\,\,$5.5626$\,\,$ & $\,\,$4.3020$\,\,$ \\
$\,\,$\color{red} 0.8964\color{black} $\,\,$ & $\,\,$ 1 $\,\,$ & $\,\,$\color{red} 4.9864\color{black} $\,\,$ & $\,\,$\color{red} 3.8564\color{black}   $\,\,$ \\
$\,\,$0.1798$\,\,$ & $\,\,$\color{red} 0.2005\color{black} $\,\,$ & $\,\,$ 1 $\,\,$ & $\,\,$0.7734 $\,\,$ \\
$\,\,$0.2324$\,\,$ & $\,\,$\color{red} 0.2593\color{black} $\,\,$ & $\,\,$1.2930$\,\,$ & $\,\,$ 1  $\,\,$ \\
\end{pmatrix},
\end{equation*}

\begin{equation*}
\mathbf{w}^{\prime} =
\begin{pmatrix}
0.432697\\
0.388936\\
0.077787\\
0.100580
\end{pmatrix} =
0.998939\cdot
\begin{pmatrix}
0.433156\\
\color{gr} 0.389349\color{black} \\
0.077870\\
0.100686
\end{pmatrix},
\end{equation*}
\begin{equation*}
\left[ \frac{{w}^{\prime}_i}{{w}^{\prime}_j} \right] =
\begin{pmatrix}
$\,\,$ 1 $\,\,$ & $\,\,$\color{gr} 1.1125\color{black} $\,\,$ & $\,\,$5.5626$\,\,$ & $\,\,$4.3020$\,\,$ \\
$\,\,$\color{gr} 0.8989\color{black} $\,\,$ & $\,\,$ 1 $\,\,$ & $\,\,$\color{gr} \color{blue} 5\color{black} $\,\,$ & $\,\,$\color{gr} 3.8670\color{black}   $\,\,$ \\
$\,\,$0.1798$\,\,$ & $\,\,$\color{gr} \color{blue}  1/5\color{black} $\,\,$ & $\,\,$ 1 $\,\,$ & $\,\,$0.7734 $\,\,$ \\
$\,\,$0.2324$\,\,$ & $\,\,$\color{gr} 0.2586\color{black} $\,\,$ & $\,\,$1.2930$\,\,$ & $\,\,$ 1  $\,\,$ \\
\end{pmatrix},
\end{equation*}
\end{example}
\newpage
\begin{example}
\begin{equation*}
\mathbf{A} =
\begin{pmatrix}
$\,\,$ 1 $\,\,$ & $\,\,$1$\,\,$ & $\,\,$4$\,\,$ & $\,\,$9 $\,\,$ \\
$\,\,$ 1 $\,\,$ & $\,\,$ 1 $\,\,$ & $\,\,$5$\,\,$ & $\,\,$4 $\,\,$ \\
$\,\,$ 1/4$\,\,$ & $\,\,$ 1/5$\,\,$ & $\,\,$ 1 $\,\,$ & $\,\,$ 1/2 $\,\,$ \\
$\,\,$ 1/9$\,\,$ & $\,\,$ 1/4$\,\,$ & $\,\,$2$\,\,$ & $\,\,$ 1  $\,\,$ \\
\end{pmatrix},
\qquad
\lambda_{\max} =
4.2067,
\qquad
CR = 0.0779
\end{equation*}

\begin{equation*}
\mathbf{w}^{cos} =
\begin{pmatrix}
0.439208\\
\color{red} 0.384797\color{black} \\
0.077637\\
0.098357
\end{pmatrix}\end{equation*}
\begin{equation*}
\left[ \frac{{w}^{cos}_i}{{w}^{cos}_j} \right] =
\begin{pmatrix}
$\,\,$ 1 $\,\,$ & $\,\,$\color{red} 1.1414\color{black} $\,\,$ & $\,\,$5.6572$\,\,$ & $\,\,$4.4654$\,\,$ \\
$\,\,$\color{red} 0.8761\color{black} $\,\,$ & $\,\,$ 1 $\,\,$ & $\,\,$\color{red} 4.9564\color{black} $\,\,$ & $\,\,$\color{red} 3.9122\color{black}   $\,\,$ \\
$\,\,$0.1768$\,\,$ & $\,\,$\color{red} 0.2018\color{black} $\,\,$ & $\,\,$ 1 $\,\,$ & $\,\,$0.7893 $\,\,$ \\
$\,\,$0.2239$\,\,$ & $\,\,$\color{red} 0.2556\color{black} $\,\,$ & $\,\,$1.2669$\,\,$ & $\,\,$ 1  $\,\,$ \\
\end{pmatrix},
\end{equation*}

\begin{equation*}
\mathbf{w}^{\prime} =
\begin{pmatrix}
0.437725\\
0.386874\\
0.077375\\
0.098025
\end{pmatrix} =
0.996623\cdot
\begin{pmatrix}
0.439208\\
\color{gr} 0.388185\color{black} \\
0.077637\\
0.098357
\end{pmatrix},
\end{equation*}
\begin{equation*}
\left[ \frac{{w}^{\prime}_i}{{w}^{\prime}_j} \right] =
\begin{pmatrix}
$\,\,$ 1 $\,\,$ & $\,\,$\color{gr} 1.1314\color{black} $\,\,$ & $\,\,$5.6572$\,\,$ & $\,\,$4.4654$\,\,$ \\
$\,\,$\color{gr} 0.8838\color{black} $\,\,$ & $\,\,$ 1 $\,\,$ & $\,\,$\color{gr} \color{blue} 5\color{black} $\,\,$ & $\,\,$\color{gr} 3.9467\color{black}   $\,\,$ \\
$\,\,$0.1768$\,\,$ & $\,\,$\color{gr} \color{blue}  1/5\color{black} $\,\,$ & $\,\,$ 1 $\,\,$ & $\,\,$0.7893 $\,\,$ \\
$\,\,$0.2239$\,\,$ & $\,\,$\color{gr} 0.2534\color{black} $\,\,$ & $\,\,$1.2669$\,\,$ & $\,\,$ 1  $\,\,$ \\
\end{pmatrix},
\end{equation*}
\end{example}
\newpage
\begin{example}
\begin{equation*}
\mathbf{A} =
\begin{pmatrix}
$\,\,$ 1 $\,\,$ & $\,\,$1$\,\,$ & $\,\,$5$\,\,$ & $\,\,$3 $\,\,$ \\
$\,\,$ 1 $\,\,$ & $\,\,$ 1 $\,\,$ & $\,\,$7$\,\,$ & $\,\,$2 $\,\,$ \\
$\,\,$ 1/5$\,\,$ & $\,\,$ 1/7$\,\,$ & $\,\,$ 1 $\,\,$ & $\,\,$ 1/5 $\,\,$ \\
$\,\,$ 1/3$\,\,$ & $\,\,$ 1/2$\,\,$ & $\,\,$5$\,\,$ & $\,\,$ 1  $\,\,$ \\
\end{pmatrix},
\qquad
\lambda_{\max} =
4.1027,
\qquad
CR = 0.0387
\end{equation*}

\begin{equation*}
\mathbf{w}^{cos} =
\begin{pmatrix}
0.381511\\
\color{red} 0.371869\color{black} \\
0.055443\\
0.191177
\end{pmatrix}\end{equation*}
\begin{equation*}
\left[ \frac{{w}^{cos}_i}{{w}^{cos}_j} \right] =
\begin{pmatrix}
$\,\,$ 1 $\,\,$ & $\,\,$\color{red} 1.0259\color{black} $\,\,$ & $\,\,$6.8811$\,\,$ & $\,\,$1.9956$\,\,$ \\
$\,\,$\color{red} 0.9747\color{black} $\,\,$ & $\,\,$ 1 $\,\,$ & $\,\,$\color{red} 6.7072\color{black} $\,\,$ & $\,\,$\color{red} 1.9452\color{black}   $\,\,$ \\
$\,\,$0.1453$\,\,$ & $\,\,$\color{red} 0.1491\color{black} $\,\,$ & $\,\,$ 1 $\,\,$ & $\,\,$0.2900 $\,\,$ \\
$\,\,$0.5011$\,\,$ & $\,\,$\color{red} 0.5141\color{black} $\,\,$ & $\,\,$3.4482$\,\,$ & $\,\,$ 1  $\,\,$ \\
\end{pmatrix},
\end{equation*}

\begin{equation*}
\mathbf{w}^{\prime} =
\begin{pmatrix}
0.377868\\
0.377868\\
0.054914\\
0.189351
\end{pmatrix} =
0.990449\cdot
\begin{pmatrix}
0.381511\\
\color{gr} 0.381511\color{black} \\
0.055443\\
0.191177
\end{pmatrix},
\end{equation*}
\begin{equation*}
\left[ \frac{{w}^{\prime}_i}{{w}^{\prime}_j} \right] =
\begin{pmatrix}
$\,\,$ 1 $\,\,$ & $\,\,$\color{gr} \color{blue} 1\color{black} $\,\,$ & $\,\,$6.8811$\,\,$ & $\,\,$1.9956$\,\,$ \\
$\,\,$\color{gr} \color{blue} 1\color{black} $\,\,$ & $\,\,$ 1 $\,\,$ & $\,\,$\color{gr} 6.8811\color{black} $\,\,$ & $\,\,$\color{gr} 1.9956\color{black}   $\,\,$ \\
$\,\,$0.1453$\,\,$ & $\,\,$\color{gr} 0.1453\color{black} $\,\,$ & $\,\,$ 1 $\,\,$ & $\,\,$0.2900 $\,\,$ \\
$\,\,$0.5011$\,\,$ & $\,\,$\color{gr} 0.5011\color{black} $\,\,$ & $\,\,$3.4482$\,\,$ & $\,\,$ 1  $\,\,$ \\
\end{pmatrix},
\end{equation*}
\end{example}
\newpage
\begin{example}
\begin{equation*}
\mathbf{A} =
\begin{pmatrix}
$\,\,$ 1 $\,\,$ & $\,\,$1$\,\,$ & $\,\,$5$\,\,$ & $\,\,$3 $\,\,$ \\
$\,\,$ 1 $\,\,$ & $\,\,$ 1 $\,\,$ & $\,\,$7$\,\,$ & $\,\,$2 $\,\,$ \\
$\,\,$ 1/5$\,\,$ & $\,\,$ 1/7$\,\,$ & $\,\,$ 1 $\,\,$ & $\,\,$ 1/6 $\,\,$ \\
$\,\,$ 1/3$\,\,$ & $\,\,$ 1/2$\,\,$ & $\,\,$6$\,\,$ & $\,\,$ 1  $\,\,$ \\
\end{pmatrix},
\qquad
\lambda_{\max} =
4.1417,
\qquad
CR = 0.0534
\end{equation*}

\begin{equation*}
\mathbf{w}^{cos} =
\begin{pmatrix}
0.378157\\
\color{red} 0.367024\color{black} \\
0.053467\\
0.201352
\end{pmatrix}\end{equation*}
\begin{equation*}
\left[ \frac{{w}^{cos}_i}{{w}^{cos}_j} \right] =
\begin{pmatrix}
$\,\,$ 1 $\,\,$ & $\,\,$\color{red} 1.0303\color{black} $\,\,$ & $\,\,$7.0727$\,\,$ & $\,\,$1.8781$\,\,$ \\
$\,\,$\color{red} 0.9706\color{black} $\,\,$ & $\,\,$ 1 $\,\,$ & $\,\,$\color{red} 6.8645\color{black} $\,\,$ & $\,\,$\color{red} 1.8228\color{black}   $\,\,$ \\
$\,\,$0.1414$\,\,$ & $\,\,$\color{red} 0.1457\color{black} $\,\,$ & $\,\,$ 1 $\,\,$ & $\,\,$0.2655 $\,\,$ \\
$\,\,$0.5325$\,\,$ & $\,\,$\color{red} 0.5486\color{black} $\,\,$ & $\,\,$3.7659$\,\,$ & $\,\,$ 1  $\,\,$ \\
\end{pmatrix},
\end{equation*}

\begin{equation*}
\mathbf{w}^{\prime} =
\begin{pmatrix}
0.375436\\
0.371578\\
0.053083\\
0.199903
\end{pmatrix} =
0.992805\cdot
\begin{pmatrix}
0.378157\\
\color{gr} 0.374271\color{black} \\
0.053467\\
0.201352
\end{pmatrix},
\end{equation*}
\begin{equation*}
\left[ \frac{{w}^{\prime}_i}{{w}^{\prime}_j} \right] =
\begin{pmatrix}
$\,\,$ 1 $\,\,$ & $\,\,$\color{gr} 1.0104\color{black} $\,\,$ & $\,\,$7.0727$\,\,$ & $\,\,$1.8781$\,\,$ \\
$\,\,$\color{gr} 0.9897\color{black} $\,\,$ & $\,\,$ 1 $\,\,$ & $\,\,$\color{gr} \color{blue} 7\color{black} $\,\,$ & $\,\,$\color{gr} 1.8588\color{black}   $\,\,$ \\
$\,\,$0.1414$\,\,$ & $\,\,$\color{gr} \color{blue}  1/7\color{black} $\,\,$ & $\,\,$ 1 $\,\,$ & $\,\,$0.2655 $\,\,$ \\
$\,\,$0.5325$\,\,$ & $\,\,$\color{gr} 0.5380\color{black} $\,\,$ & $\,\,$3.7659$\,\,$ & $\,\,$ 1  $\,\,$ \\
\end{pmatrix},
\end{equation*}
\end{example}
\newpage
\begin{example}
\begin{equation*}
\mathbf{A} =
\begin{pmatrix}
$\,\,$ 1 $\,\,$ & $\,\,$1$\,\,$ & $\,\,$5$\,\,$ & $\,\,$3 $\,\,$ \\
$\,\,$ 1 $\,\,$ & $\,\,$ 1 $\,\,$ & $\,\,$7$\,\,$ & $\,\,$2 $\,\,$ \\
$\,\,$ 1/5$\,\,$ & $\,\,$ 1/7$\,\,$ & $\,\,$ 1 $\,\,$ & $\,\,$ 1/7 $\,\,$ \\
$\,\,$ 1/3$\,\,$ & $\,\,$ 1/2$\,\,$ & $\,\,$7$\,\,$ & $\,\,$ 1  $\,\,$ \\
\end{pmatrix},
\qquad
\lambda_{\max} =
4.1820,
\qquad
CR = 0.0686
\end{equation*}

\begin{equation*}
\mathbf{w}^{cos} =
\begin{pmatrix}
0.375217\\
\color{red} 0.362574\color{black} \\
0.051913\\
0.210295
\end{pmatrix}\end{equation*}
\begin{equation*}
\left[ \frac{{w}^{cos}_i}{{w}^{cos}_j} \right] =
\begin{pmatrix}
$\,\,$ 1 $\,\,$ & $\,\,$\color{red} 1.0349\color{black} $\,\,$ & $\,\,$7.2278$\,\,$ & $\,\,$1.7842$\,\,$ \\
$\,\,$\color{red} 0.9663\color{black} $\,\,$ & $\,\,$ 1 $\,\,$ & $\,\,$\color{red} 6.9842\color{black} $\,\,$ & $\,\,$\color{red} 1.7241\color{black}   $\,\,$ \\
$\,\,$0.1384$\,\,$ & $\,\,$\color{red} 0.1432\color{black} $\,\,$ & $\,\,$ 1 $\,\,$ & $\,\,$0.2469 $\,\,$ \\
$\,\,$0.5605$\,\,$ & $\,\,$\color{red} 0.5800\color{black} $\,\,$ & $\,\,$4.0509$\,\,$ & $\,\,$ 1  $\,\,$ \\
\end{pmatrix},
\end{equation*}

\begin{equation*}
\mathbf{w}^{\prime} =
\begin{pmatrix}
0.374910\\
0.363096\\
0.051871\\
0.210123
\end{pmatrix} =
0.999182\cdot
\begin{pmatrix}
0.375217\\
\color{gr} 0.363393\color{black} \\
0.051913\\
0.210295
\end{pmatrix},
\end{equation*}
\begin{equation*}
\left[ \frac{{w}^{\prime}_i}{{w}^{\prime}_j} \right] =
\begin{pmatrix}
$\,\,$ 1 $\,\,$ & $\,\,$\color{gr} 1.0325\color{black} $\,\,$ & $\,\,$7.2278$\,\,$ & $\,\,$1.7842$\,\,$ \\
$\,\,$\color{gr} 0.9685\color{black} $\,\,$ & $\,\,$ 1 $\,\,$ & $\,\,$\color{gr} \color{blue} 7\color{black} $\,\,$ & $\,\,$\color{gr} 1.7280\color{black}   $\,\,$ \\
$\,\,$0.1384$\,\,$ & $\,\,$\color{gr} \color{blue}  1/7\color{black} $\,\,$ & $\,\,$ 1 $\,\,$ & $\,\,$0.2469 $\,\,$ \\
$\,\,$0.5605$\,\,$ & $\,\,$\color{gr} 0.5787\color{black} $\,\,$ & $\,\,$4.0509$\,\,$ & $\,\,$ 1  $\,\,$ \\
\end{pmatrix},
\end{equation*}
\end{example}
\newpage
\begin{example}
\begin{equation*}
\mathbf{A} =
\begin{pmatrix}
$\,\,$ 1 $\,\,$ & $\,\,$1$\,\,$ & $\,\,$5$\,\,$ & $\,\,$3 $\,\,$ \\
$\,\,$ 1 $\,\,$ & $\,\,$ 1 $\,\,$ & $\,\,$8$\,\,$ & $\,\,$2 $\,\,$ \\
$\,\,$ 1/5$\,\,$ & $\,\,$ 1/8$\,\,$ & $\,\,$ 1 $\,\,$ & $\,\,$ 1/7 $\,\,$ \\
$\,\,$ 1/3$\,\,$ & $\,\,$ 1/2$\,\,$ & $\,\,$7$\,\,$ & $\,\,$ 1  $\,\,$ \\
\end{pmatrix},
\qquad
\lambda_{\max} =
4.1782,
\qquad
CR = 0.0672
\end{equation*}

\begin{equation*}
\mathbf{w}^{cos} =
\begin{pmatrix}
0.373057\\
\color{red} 0.371228\color{black} \\
0.049671\\
0.206044
\end{pmatrix}\end{equation*}
\begin{equation*}
\left[ \frac{{w}^{cos}_i}{{w}^{cos}_j} \right] =
\begin{pmatrix}
$\,\,$ 1 $\,\,$ & $\,\,$\color{red} 1.0049\color{black} $\,\,$ & $\,\,$7.5106$\,\,$ & $\,\,$1.8106$\,\,$ \\
$\,\,$\color{red} 0.9951\color{black} $\,\,$ & $\,\,$ 1 $\,\,$ & $\,\,$\color{red} 7.4738\color{black} $\,\,$ & $\,\,$\color{red} 1.8017\color{black}   $\,\,$ \\
$\,\,$0.1331$\,\,$ & $\,\,$\color{red} 0.1338\color{black} $\,\,$ & $\,\,$ 1 $\,\,$ & $\,\,$0.2411 $\,\,$ \\
$\,\,$0.5523$\,\,$ & $\,\,$\color{red} 0.5550\color{black} $\,\,$ & $\,\,$4.1482$\,\,$ & $\,\,$ 1  $\,\,$ \\
\end{pmatrix},
\end{equation*}

\begin{equation*}
\mathbf{w}^{\prime} =
\begin{pmatrix}
0.372376\\
0.372376\\
0.049580\\
0.205667
\end{pmatrix} =
0.998174\cdot
\begin{pmatrix}
0.373057\\
\color{gr} 0.373057\color{black} \\
0.049671\\
0.206044
\end{pmatrix},
\end{equation*}
\begin{equation*}
\left[ \frac{{w}^{\prime}_i}{{w}^{\prime}_j} \right] =
\begin{pmatrix}
$\,\,$ 1 $\,\,$ & $\,\,$\color{gr} \color{blue} 1\color{black} $\,\,$ & $\,\,$7.5106$\,\,$ & $\,\,$1.8106$\,\,$ \\
$\,\,$\color{gr} \color{blue} 1\color{black} $\,\,$ & $\,\,$ 1 $\,\,$ & $\,\,$\color{gr} 7.5106\color{black} $\,\,$ & $\,\,$\color{gr} 1.8106\color{black}   $\,\,$ \\
$\,\,$0.1331$\,\,$ & $\,\,$\color{gr} 0.1331\color{black} $\,\,$ & $\,\,$ 1 $\,\,$ & $\,\,$0.2411 $\,\,$ \\
$\,\,$0.5523$\,\,$ & $\,\,$\color{gr} 0.5523\color{black} $\,\,$ & $\,\,$4.1482$\,\,$ & $\,\,$ 1  $\,\,$ \\
\end{pmatrix},
\end{equation*}
\end{example}
\newpage
\begin{example}
\begin{equation*}
\mathbf{A} =
\begin{pmatrix}
$\,\,$ 1 $\,\,$ & $\,\,$1$\,\,$ & $\,\,$5$\,\,$ & $\,\,$3 $\,\,$ \\
$\,\,$ 1 $\,\,$ & $\,\,$ 1 $\,\,$ & $\,\,$8$\,\,$ & $\,\,$2 $\,\,$ \\
$\,\,$ 1/5$\,\,$ & $\,\,$ 1/8$\,\,$ & $\,\,$ 1 $\,\,$ & $\,\,$ 1/8 $\,\,$ \\
$\,\,$ 1/3$\,\,$ & $\,\,$ 1/2$\,\,$ & $\,\,$8$\,\,$ & $\,\,$ 1  $\,\,$ \\
\end{pmatrix},
\qquad
\lambda_{\max} =
4.2162,
\qquad
CR = 0.0815
\end{equation*}

\begin{equation*}
\mathbf{w}^{cos} =
\begin{pmatrix}
0.370697\\
\color{red} 0.367010\color{black} \\
0.048456\\
0.213838
\end{pmatrix}\end{equation*}
\begin{equation*}
\left[ \frac{{w}^{cos}_i}{{w}^{cos}_j} \right] =
\begin{pmatrix}
$\,\,$ 1 $\,\,$ & $\,\,$\color{red} 1.0100\color{black} $\,\,$ & $\,\,$7.6502$\,\,$ & $\,\,$1.7335$\,\,$ \\
$\,\,$\color{red} 0.9901\color{black} $\,\,$ & $\,\,$ 1 $\,\,$ & $\,\,$\color{red} 7.5741\color{black} $\,\,$ & $\,\,$\color{red} 1.7163\color{black}   $\,\,$ \\
$\,\,$0.1307$\,\,$ & $\,\,$\color{red} 0.1320\color{black} $\,\,$ & $\,\,$ 1 $\,\,$ & $\,\,$0.2266 $\,\,$ \\
$\,\,$0.5769$\,\,$ & $\,\,$\color{red} 0.5826\color{black} $\,\,$ & $\,\,$4.4131$\,\,$ & $\,\,$ 1  $\,\,$ \\
\end{pmatrix},
\end{equation*}

\begin{equation*}
\mathbf{w}^{\prime} =
\begin{pmatrix}
0.369335\\
0.369335\\
0.048278\\
0.213052
\end{pmatrix} =
0.996327\cdot
\begin{pmatrix}
0.370697\\
\color{gr} 0.370697\color{black} \\
0.048456\\
0.213838
\end{pmatrix},
\end{equation*}
\begin{equation*}
\left[ \frac{{w}^{\prime}_i}{{w}^{\prime}_j} \right] =
\begin{pmatrix}
$\,\,$ 1 $\,\,$ & $\,\,$\color{gr} \color{blue} 1\color{black} $\,\,$ & $\,\,$7.6502$\,\,$ & $\,\,$1.7335$\,\,$ \\
$\,\,$\color{gr} \color{blue} 1\color{black} $\,\,$ & $\,\,$ 1 $\,\,$ & $\,\,$\color{gr} 7.6502\color{black} $\,\,$ & $\,\,$\color{gr} 1.7335\color{black}   $\,\,$ \\
$\,\,$0.1307$\,\,$ & $\,\,$\color{gr} 0.1307\color{black} $\,\,$ & $\,\,$ 1 $\,\,$ & $\,\,$0.2266 $\,\,$ \\
$\,\,$0.5769$\,\,$ & $\,\,$\color{gr} 0.5769\color{black} $\,\,$ & $\,\,$4.4131$\,\,$ & $\,\,$ 1  $\,\,$ \\
\end{pmatrix},
\end{equation*}
\end{example}
\newpage
\begin{example}
\begin{equation*}
\mathbf{A} =
\begin{pmatrix}
$\,\,$ 1 $\,\,$ & $\,\,$1$\,\,$ & $\,\,$5$\,\,$ & $\,\,$3 $\,\,$ \\
$\,\,$ 1 $\,\,$ & $\,\,$ 1 $\,\,$ & $\,\,$8$\,\,$ & $\,\,$2 $\,\,$ \\
$\,\,$ 1/5$\,\,$ & $\,\,$ 1/8$\,\,$ & $\,\,$ 1 $\,\,$ & $\,\,$ 1/9 $\,\,$ \\
$\,\,$ 1/3$\,\,$ & $\,\,$ 1/2$\,\,$ & $\,\,$9$\,\,$ & $\,\,$ 1  $\,\,$ \\
\end{pmatrix},
\qquad
\lambda_{\max} =
4.2541,
\qquad
CR = 0.0958
\end{equation*}

\begin{equation*}
\mathbf{w}^{cos} =
\begin{pmatrix}
0.368660\\
\color{red} 0.363162\color{black} \\
0.047440\\
0.220738
\end{pmatrix}\end{equation*}
\begin{equation*}
\left[ \frac{{w}^{cos}_i}{{w}^{cos}_j} \right] =
\begin{pmatrix}
$\,\,$ 1 $\,\,$ & $\,\,$\color{red} 1.0151\color{black} $\,\,$ & $\,\,$7.7711$\,\,$ & $\,\,$1.6701$\,\,$ \\
$\,\,$\color{red} 0.9851\color{black} $\,\,$ & $\,\,$ 1 $\,\,$ & $\,\,$\color{red} 7.6552\color{black} $\,\,$ & $\,\,$\color{red} 1.6452\color{black}   $\,\,$ \\
$\,\,$0.1287$\,\,$ & $\,\,$\color{red} 0.1306\color{black} $\,\,$ & $\,\,$ 1 $\,\,$ & $\,\,$0.2149 $\,\,$ \\
$\,\,$0.5988$\,\,$ & $\,\,$\color{red} 0.6078\color{black} $\,\,$ & $\,\,$4.6530$\,\,$ & $\,\,$ 1  $\,\,$ \\
\end{pmatrix},
\end{equation*}

\begin{equation*}
\mathbf{w}^{\prime} =
\begin{pmatrix}
0.366644\\
0.366644\\
0.047181\\
0.219532
\end{pmatrix} =
0.994532\cdot
\begin{pmatrix}
0.368660\\
\color{gr} 0.368660\color{black} \\
0.047440\\
0.220738
\end{pmatrix},
\end{equation*}
\begin{equation*}
\left[ \frac{{w}^{\prime}_i}{{w}^{\prime}_j} \right] =
\begin{pmatrix}
$\,\,$ 1 $\,\,$ & $\,\,$\color{gr} \color{blue} 1\color{black} $\,\,$ & $\,\,$7.7711$\,\,$ & $\,\,$1.6701$\,\,$ \\
$\,\,$\color{gr} \color{blue} 1\color{black} $\,\,$ & $\,\,$ 1 $\,\,$ & $\,\,$\color{gr} 7.7711\color{black} $\,\,$ & $\,\,$\color{gr} 1.6701\color{black}   $\,\,$ \\
$\,\,$0.1287$\,\,$ & $\,\,$\color{gr} 0.1287\color{black} $\,\,$ & $\,\,$ 1 $\,\,$ & $\,\,$0.2149 $\,\,$ \\
$\,\,$0.5988$\,\,$ & $\,\,$\color{gr} 0.5988\color{black} $\,\,$ & $\,\,$4.6530$\,\,$ & $\,\,$ 1  $\,\,$ \\
\end{pmatrix},
\end{equation*}
\end{example}
\newpage
\begin{example}
\begin{equation*}
\mathbf{A} =
\begin{pmatrix}
$\,\,$ 1 $\,\,$ & $\,\,$1$\,\,$ & $\,\,$5$\,\,$ & $\,\,$4 $\,\,$ \\
$\,\,$ 1 $\,\,$ & $\,\,$ 1 $\,\,$ & $\,\,$7$\,\,$ & $\,\,$2 $\,\,$ \\
$\,\,$ 1/5$\,\,$ & $\,\,$ 1/7$\,\,$ & $\,\,$ 1 $\,\,$ & $\,\,$ 1/6 $\,\,$ \\
$\,\,$ 1/4$\,\,$ & $\,\,$ 1/2$\,\,$ & $\,\,$6$\,\,$ & $\,\,$ 1  $\,\,$ \\
\end{pmatrix},
\qquad
\lambda_{\max} =
4.2174,
\qquad
CR = 0.0820
\end{equation*}

\begin{equation*}
\mathbf{w}^{cos} =
\begin{pmatrix}
0.396511\\
\color{red} 0.360340\color{black} \\
0.053452\\
0.189697
\end{pmatrix}\end{equation*}
\begin{equation*}
\left[ \frac{{w}^{cos}_i}{{w}^{cos}_j} \right] =
\begin{pmatrix}
$\,\,$ 1 $\,\,$ & $\,\,$\color{red} 1.1004\color{black} $\,\,$ & $\,\,$7.4181$\,\,$ & $\,\,$2.0902$\,\,$ \\
$\,\,$\color{red} 0.9088\color{black} $\,\,$ & $\,\,$ 1 $\,\,$ & $\,\,$\color{red} 6.7414\color{black} $\,\,$ & $\,\,$\color{red} 1.8996\color{black}   $\,\,$ \\
$\,\,$0.1348$\,\,$ & $\,\,$\color{red} 0.1483\color{black} $\,\,$ & $\,\,$ 1 $\,\,$ & $\,\,$0.2818 $\,\,$ \\
$\,\,$0.4784$\,\,$ & $\,\,$\color{red} 0.5264\color{black} $\,\,$ & $\,\,$3.5489$\,\,$ & $\,\,$ 1  $\,\,$ \\
\end{pmatrix},
\end{equation*}

\begin{equation*}
\mathbf{w}^{\prime} =
\begin{pmatrix}
0.391104\\
0.369062\\
0.052723\\
0.187110
\end{pmatrix} =
0.986365\cdot
\begin{pmatrix}
0.396511\\
\color{gr} 0.374164\color{black} \\
0.053452\\
0.189697
\end{pmatrix},
\end{equation*}
\begin{equation*}
\left[ \frac{{w}^{\prime}_i}{{w}^{\prime}_j} \right] =
\begin{pmatrix}
$\,\,$ 1 $\,\,$ & $\,\,$\color{gr} 1.0597\color{black} $\,\,$ & $\,\,$7.4181$\,\,$ & $\,\,$2.0902$\,\,$ \\
$\,\,$\color{gr} 0.9436\color{black} $\,\,$ & $\,\,$ 1 $\,\,$ & $\,\,$\color{gr} \color{blue} 7\color{black} $\,\,$ & $\,\,$\color{gr} 1.9724\color{black}   $\,\,$ \\
$\,\,$0.1348$\,\,$ & $\,\,$\color{gr} \color{blue}  1/7\color{black} $\,\,$ & $\,\,$ 1 $\,\,$ & $\,\,$0.2818 $\,\,$ \\
$\,\,$0.4784$\,\,$ & $\,\,$\color{gr} 0.5070\color{black} $\,\,$ & $\,\,$3.5489$\,\,$ & $\,\,$ 1  $\,\,$ \\
\end{pmatrix},
\end{equation*}
\end{example}
\newpage
\begin{example}
\begin{equation*}
\mathbf{A} =
\begin{pmatrix}
$\,\,$ 1 $\,\,$ & $\,\,$1$\,\,$ & $\,\,$5$\,\,$ & $\,\,$4 $\,\,$ \\
$\,\,$ 1 $\,\,$ & $\,\,$ 1 $\,\,$ & $\,\,$7$\,\,$ & $\,\,$2 $\,\,$ \\
$\,\,$ 1/5$\,\,$ & $\,\,$ 1/7$\,\,$ & $\,\,$ 1 $\,\,$ & $\,\,$ 1/7 $\,\,$ \\
$\,\,$ 1/4$\,\,$ & $\,\,$ 1/2$\,\,$ & $\,\,$7$\,\,$ & $\,\,$ 1  $\,\,$ \\
\end{pmatrix},
\qquad
\lambda_{\max} =
4.2648,
\qquad
CR = 0.0998
\end{equation*}

\begin{equation*}
\mathbf{w}^{cos} =
\begin{pmatrix}
0.393488\\
\color{red} 0.355737\color{black} \\
0.052033\\
0.198743
\end{pmatrix}\end{equation*}
\begin{equation*}
\left[ \frac{{w}^{cos}_i}{{w}^{cos}_j} \right] =
\begin{pmatrix}
$\,\,$ 1 $\,\,$ & $\,\,$\color{red} 1.1061\color{black} $\,\,$ & $\,\,$7.5622$\,\,$ & $\,\,$1.9799$\,\,$ \\
$\,\,$\color{red} 0.9041\color{black} $\,\,$ & $\,\,$ 1 $\,\,$ & $\,\,$\color{red} 6.8367\color{black} $\,\,$ & $\,\,$\color{red} 1.7899\color{black}   $\,\,$ \\
$\,\,$0.1322$\,\,$ & $\,\,$\color{red} 0.1463\color{black} $\,\,$ & $\,\,$ 1 $\,\,$ & $\,\,$0.2618 $\,\,$ \\
$\,\,$0.5051$\,\,$ & $\,\,$\color{red} 0.5587\color{black} $\,\,$ & $\,\,$3.8195$\,\,$ & $\,\,$ 1  $\,\,$ \\
\end{pmatrix},
\end{equation*}

\begin{equation*}
\mathbf{w}^{\prime} =
\begin{pmatrix}
0.390173\\
0.361164\\
0.051595\\
0.197068
\end{pmatrix} =
0.991576\cdot
\begin{pmatrix}
0.393488\\
\color{gr} 0.364232\color{black} \\
0.052033\\
0.198743
\end{pmatrix},
\end{equation*}
\begin{equation*}
\left[ \frac{{w}^{\prime}_i}{{w}^{\prime}_j} \right] =
\begin{pmatrix}
$\,\,$ 1 $\,\,$ & $\,\,$\color{gr} 1.0803\color{black} $\,\,$ & $\,\,$7.5622$\,\,$ & $\,\,$1.9799$\,\,$ \\
$\,\,$\color{gr} 0.9257\color{black} $\,\,$ & $\,\,$ 1 $\,\,$ & $\,\,$\color{gr} \color{blue} 7\color{black} $\,\,$ & $\,\,$\color{gr} 1.8327\color{black}   $\,\,$ \\
$\,\,$0.1322$\,\,$ & $\,\,$\color{gr} \color{blue}  1/7\color{black} $\,\,$ & $\,\,$ 1 $\,\,$ & $\,\,$0.2618 $\,\,$ \\
$\,\,$0.5051$\,\,$ & $\,\,$\color{gr} 0.5456\color{black} $\,\,$ & $\,\,$3.8195$\,\,$ & $\,\,$ 1  $\,\,$ \\
\end{pmatrix},
\end{equation*}
\end{example}
\newpage
\begin{example}
\begin{equation*}
\mathbf{A} =
\begin{pmatrix}
$\,\,$ 1 $\,\,$ & $\,\,$1$\,\,$ & $\,\,$5$\,\,$ & $\,\,$4 $\,\,$ \\
$\,\,$ 1 $\,\,$ & $\,\,$ 1 $\,\,$ & $\,\,$8$\,\,$ & $\,\,$2 $\,\,$ \\
$\,\,$ 1/5$\,\,$ & $\,\,$ 1/8$\,\,$ & $\,\,$ 1 $\,\,$ & $\,\,$ 1/6 $\,\,$ \\
$\,\,$ 1/4$\,\,$ & $\,\,$ 1/2$\,\,$ & $\,\,$6$\,\,$ & $\,\,$ 1  $\,\,$ \\
\end{pmatrix},
\qquad
\lambda_{\max} =
4.2162,
\qquad
CR = 0.0815
\end{equation*}

\begin{equation*}
\mathbf{w}^{cos} =
\begin{pmatrix}
0.394167\\
\color{red} 0.369251\color{black} \\
0.051106\\
0.185475
\end{pmatrix}\end{equation*}
\begin{equation*}
\left[ \frac{{w}^{cos}_i}{{w}^{cos}_j} \right] =
\begin{pmatrix}
$\,\,$ 1 $\,\,$ & $\,\,$\color{red} 1.0675\color{black} $\,\,$ & $\,\,$7.7127$\,\,$ & $\,\,$2.1252$\,\,$ \\
$\,\,$\color{red} 0.9368\color{black} $\,\,$ & $\,\,$ 1 $\,\,$ & $\,\,$\color{red} 7.2252\color{black} $\,\,$ & $\,\,$\color{red} 1.9908\color{black}   $\,\,$ \\
$\,\,$0.1297$\,\,$ & $\,\,$\color{red} 0.1384\color{black} $\,\,$ & $\,\,$ 1 $\,\,$ & $\,\,$0.2755 $\,\,$ \\
$\,\,$0.4706$\,\,$ & $\,\,$\color{red} 0.5023\color{black} $\,\,$ & $\,\,$3.6292$\,\,$ & $\,\,$ 1  $\,\,$ \\
\end{pmatrix},
\end{equation*}

\begin{equation*}
\mathbf{w}^{\prime} =
\begin{pmatrix}
0.393498\\
0.370321\\
0.051020\\
0.185161
\end{pmatrix} =
0.998303\cdot
\begin{pmatrix}
0.394167\\
\color{gr} 0.370951\color{black} \\
0.051106\\
0.185475
\end{pmatrix},
\end{equation*}
\begin{equation*}
\left[ \frac{{w}^{\prime}_i}{{w}^{\prime}_j} \right] =
\begin{pmatrix}
$\,\,$ 1 $\,\,$ & $\,\,$\color{gr} 1.0626\color{black} $\,\,$ & $\,\,$7.7127$\,\,$ & $\,\,$2.1252$\,\,$ \\
$\,\,$\color{gr} 0.9411\color{black} $\,\,$ & $\,\,$ 1 $\,\,$ & $\,\,$\color{gr} 7.2584\color{black} $\,\,$ & $\,\,$\color{gr} \color{blue} 2\color{black}   $\,\,$ \\
$\,\,$0.1297$\,\,$ & $\,\,$\color{gr} 0.1378\color{black} $\,\,$ & $\,\,$ 1 $\,\,$ & $\,\,$0.2755 $\,\,$ \\
$\,\,$0.4706$\,\,$ & $\,\,$\color{gr} \color{blue}  1/2\color{black} $\,\,$ & $\,\,$3.6292$\,\,$ & $\,\,$ 1  $\,\,$ \\
\end{pmatrix},
\end{equation*}
\end{example}
\newpage
\begin{example}
\begin{equation*}
\mathbf{A} =
\begin{pmatrix}
$\,\,$ 1 $\,\,$ & $\,\,$1$\,\,$ & $\,\,$5$\,\,$ & $\,\,$4 $\,\,$ \\
$\,\,$ 1 $\,\,$ & $\,\,$ 1 $\,\,$ & $\,\,$8$\,\,$ & $\,\,$2 $\,\,$ \\
$\,\,$ 1/5$\,\,$ & $\,\,$ 1/8$\,\,$ & $\,\,$ 1 $\,\,$ & $\,\,$ 1/7 $\,\,$ \\
$\,\,$ 1/4$\,\,$ & $\,\,$ 1/2$\,\,$ & $\,\,$7$\,\,$ & $\,\,$ 1  $\,\,$ \\
\end{pmatrix},
\qquad
\lambda_{\max} =
4.2610,
\qquad
CR = 0.0984
\end{equation*}

\begin{equation*}
\mathbf{w}^{cos} =
\begin{pmatrix}
0.391357\\
\color{red} 0.364519\color{black} \\
0.049751\\
0.194373
\end{pmatrix}\end{equation*}
\begin{equation*}
\left[ \frac{{w}^{cos}_i}{{w}^{cos}_j} \right] =
\begin{pmatrix}
$\,\,$ 1 $\,\,$ & $\,\,$\color{red} 1.0736\color{black} $\,\,$ & $\,\,$7.8662$\,\,$ & $\,\,$2.0134$\,\,$ \\
$\,\,$\color{red} 0.9314\color{black} $\,\,$ & $\,\,$ 1 $\,\,$ & $\,\,$\color{red} 7.3268\color{black} $\,\,$ & $\,\,$\color{red} 1.8754\color{black}   $\,\,$ \\
$\,\,$0.1271$\,\,$ & $\,\,$\color{red} 0.1365\color{black} $\,\,$ & $\,\,$ 1 $\,\,$ & $\,\,$0.2560 $\,\,$ \\
$\,\,$0.4967$\,\,$ & $\,\,$\color{red} 0.5332\color{black} $\,\,$ & $\,\,$3.9069$\,\,$ & $\,\,$ 1  $\,\,$ \\
\end{pmatrix},
\end{equation*}

\begin{equation*}
\mathbf{w}^{\prime} =
\begin{pmatrix}
0.382099\\
0.379551\\
0.048575\\
0.189775
\end{pmatrix} =
0.976346\cdot
\begin{pmatrix}
0.391357\\
\color{gr} 0.388746\color{black} \\
0.049751\\
0.194373
\end{pmatrix},
\end{equation*}
\begin{equation*}
\left[ \frac{{w}^{\prime}_i}{{w}^{\prime}_j} \right] =
\begin{pmatrix}
$\,\,$ 1 $\,\,$ & $\,\,$\color{gr} 1.0067\color{black} $\,\,$ & $\,\,$7.8662$\,\,$ & $\,\,$2.0134$\,\,$ \\
$\,\,$\color{gr} 0.9933\color{black} $\,\,$ & $\,\,$ 1 $\,\,$ & $\,\,$\color{gr} 7.8138\color{black} $\,\,$ & $\,\,$\color{gr} \color{blue} 2\color{black}   $\,\,$ \\
$\,\,$0.1271$\,\,$ & $\,\,$\color{gr} 0.1280\color{black} $\,\,$ & $\,\,$ 1 $\,\,$ & $\,\,$0.2560 $\,\,$ \\
$\,\,$0.4967$\,\,$ & $\,\,$\color{gr} \color{blue}  1/2\color{black} $\,\,$ & $\,\,$3.9069$\,\,$ & $\,\,$ 1  $\,\,$ \\
\end{pmatrix},
\end{equation*}
\end{example}
\newpage
\begin{example}
\begin{equation*}
\mathbf{A} =
\begin{pmatrix}
$\,\,$ 1 $\,\,$ & $\,\,$1$\,\,$ & $\,\,$5$\,\,$ & $\,\,$4 $\,\,$ \\
$\,\,$ 1 $\,\,$ & $\,\,$ 1 $\,\,$ & $\,\,$9$\,\,$ & $\,\,$2 $\,\,$ \\
$\,\,$ 1/5$\,\,$ & $\,\,$ 1/9$\,\,$ & $\,\,$ 1 $\,\,$ & $\,\,$ 1/7 $\,\,$ \\
$\,\,$ 1/4$\,\,$ & $\,\,$ 1/2$\,\,$ & $\,\,$7$\,\,$ & $\,\,$ 1  $\,\,$ \\
\end{pmatrix},
\qquad
\lambda_{\max} =
4.2614,
\qquad
CR = 0.0986
\end{equation*}

\begin{equation*}
\mathbf{w}^{cos} =
\begin{pmatrix}
0.389514\\
\color{red} 0.372321\color{black} \\
0.047891\\
0.190274
\end{pmatrix}\end{equation*}
\begin{equation*}
\left[ \frac{{w}^{cos}_i}{{w}^{cos}_j} \right] =
\begin{pmatrix}
$\,\,$ 1 $\,\,$ & $\,\,$\color{red} 1.0462\color{black} $\,\,$ & $\,\,$8.1333$\,\,$ & $\,\,$2.0471$\,\,$ \\
$\,\,$\color{red} 0.9559\color{black} $\,\,$ & $\,\,$ 1 $\,\,$ & $\,\,$\color{red} 7.7743\color{black} $\,\,$ & $\,\,$\color{red} 1.9568\color{black}   $\,\,$ \\
$\,\,$0.1230$\,\,$ & $\,\,$\color{red} 0.1286\color{black} $\,\,$ & $\,\,$ 1 $\,\,$ & $\,\,$0.2517 $\,\,$ \\
$\,\,$0.4885$\,\,$ & $\,\,$\color{red} 0.5110\color{black} $\,\,$ & $\,\,$3.9730$\,\,$ & $\,\,$ 1  $\,\,$ \\
\end{pmatrix},
\end{equation*}

\begin{equation*}
\mathbf{w}^{\prime} =
\begin{pmatrix}
0.386335\\
0.377443\\
0.047500\\
0.188721
\end{pmatrix} =
0.991840\cdot
\begin{pmatrix}
0.389514\\
\color{gr} 0.380548\color{black} \\
0.047891\\
0.190274
\end{pmatrix},
\end{equation*}
\begin{equation*}
\left[ \frac{{w}^{\prime}_i}{{w}^{\prime}_j} \right] =
\begin{pmatrix}
$\,\,$ 1 $\,\,$ & $\,\,$\color{gr} 1.0236\color{black} $\,\,$ & $\,\,$8.1333$\,\,$ & $\,\,$2.0471$\,\,$ \\
$\,\,$\color{gr} 0.9770\color{black} $\,\,$ & $\,\,$ 1 $\,\,$ & $\,\,$\color{gr} 7.9461\color{black} $\,\,$ & $\,\,$\color{gr} \color{blue} 2\color{black}   $\,\,$ \\
$\,\,$0.1230$\,\,$ & $\,\,$\color{gr} 0.1258\color{black} $\,\,$ & $\,\,$ 1 $\,\,$ & $\,\,$0.2517 $\,\,$ \\
$\,\,$0.4885$\,\,$ & $\,\,$\color{gr} \color{blue}  1/2\color{black} $\,\,$ & $\,\,$3.9730$\,\,$ & $\,\,$ 1  $\,\,$ \\
\end{pmatrix},
\end{equation*}
\end{example}
\newpage
\begin{example}
\begin{equation*}
\mathbf{A} =
\begin{pmatrix}
$\,\,$ 1 $\,\,$ & $\,\,$1$\,\,$ & $\,\,$5$\,\,$ & $\,\,$5 $\,\,$ \\
$\,\,$ 1 $\,\,$ & $\,\,$ 1 $\,\,$ & $\,\,$7$\,\,$ & $\,\,$3 $\,\,$ \\
$\,\,$ 1/5$\,\,$ & $\,\,$ 1/7$\,\,$ & $\,\,$ 1 $\,\,$ & $\,\,$ 1/4 $\,\,$ \\
$\,\,$ 1/5$\,\,$ & $\,\,$ 1/3$\,\,$ & $\,\,$4$\,\,$ & $\,\,$ 1  $\,\,$ \\
\end{pmatrix},
\qquad
\lambda_{\max} =
4.1667,
\qquad
CR = 0.0629
\end{equation*}

\begin{equation*}
\mathbf{w}^{cos} =
\begin{pmatrix}
0.409712\\
\color{red} 0.390604\color{black} \\
0.057212\\
0.142472
\end{pmatrix}\end{equation*}
\begin{equation*}
\left[ \frac{{w}^{cos}_i}{{w}^{cos}_j} \right] =
\begin{pmatrix}
$\,\,$ 1 $\,\,$ & $\,\,$\color{red} 1.0489\color{black} $\,\,$ & $\,\,$7.1613$\,\,$ & $\,\,$2.8757$\,\,$ \\
$\,\,$\color{red} 0.9534\color{black} $\,\,$ & $\,\,$ 1 $\,\,$ & $\,\,$\color{red} 6.8273\color{black} $\,\,$ & $\,\,$\color{red} 2.7416\color{black}   $\,\,$ \\
$\,\,$0.1396$\,\,$ & $\,\,$\color{red} 0.1465\color{black} $\,\,$ & $\,\,$ 1 $\,\,$ & $\,\,$0.4016 $\,\,$ \\
$\,\,$0.3477$\,\,$ & $\,\,$\color{red} 0.3647\color{black} $\,\,$ & $\,\,$2.4902$\,\,$ & $\,\,$ 1  $\,\,$ \\
\end{pmatrix},
\end{equation*}

\begin{equation*}
\mathbf{w}^{\prime} =
\begin{pmatrix}
0.405703\\
0.396567\\
0.056652\\
0.141078
\end{pmatrix} =
0.990215\cdot
\begin{pmatrix}
0.409712\\
\color{gr} 0.400486\color{black} \\
0.057212\\
0.142472
\end{pmatrix},
\end{equation*}
\begin{equation*}
\left[ \frac{{w}^{\prime}_i}{{w}^{\prime}_j} \right] =
\begin{pmatrix}
$\,\,$ 1 $\,\,$ & $\,\,$\color{gr} 1.0230\color{black} $\,\,$ & $\,\,$7.1613$\,\,$ & $\,\,$2.8757$\,\,$ \\
$\,\,$\color{gr} 0.9775\color{black} $\,\,$ & $\,\,$ 1 $\,\,$ & $\,\,$\color{gr} \color{blue} 7\color{black} $\,\,$ & $\,\,$\color{gr} 2.8110\color{black}   $\,\,$ \\
$\,\,$0.1396$\,\,$ & $\,\,$\color{gr} \color{blue}  1/7\color{black} $\,\,$ & $\,\,$ 1 $\,\,$ & $\,\,$0.4016 $\,\,$ \\
$\,\,$0.3477$\,\,$ & $\,\,$\color{gr} 0.3557\color{black} $\,\,$ & $\,\,$2.4902$\,\,$ & $\,\,$ 1  $\,\,$ \\
\end{pmatrix},
\end{equation*}
\end{example}
\newpage
\begin{example}
\begin{equation*}
\mathbf{A} =
\begin{pmatrix}
$\,\,$ 1 $\,\,$ & $\,\,$1$\,\,$ & $\,\,$5$\,\,$ & $\,\,$5 $\,\,$ \\
$\,\,$ 1 $\,\,$ & $\,\,$ 1 $\,\,$ & $\,\,$7$\,\,$ & $\,\,$3 $\,\,$ \\
$\,\,$ 1/5$\,\,$ & $\,\,$ 1/7$\,\,$ & $\,\,$ 1 $\,\,$ & $\,\,$ 1/5 $\,\,$ \\
$\,\,$ 1/5$\,\,$ & $\,\,$ 1/3$\,\,$ & $\,\,$5$\,\,$ & $\,\,$ 1  $\,\,$ \\
\end{pmatrix},
\qquad
\lambda_{\max} =
4.2309,
\qquad
CR = 0.0871
\end{equation*}

\begin{equation*}
\mathbf{w}^{cos} =
\begin{pmatrix}
0.405548\\
\color{red} 0.385015\color{black} \\
0.055152\\
0.154285
\end{pmatrix}\end{equation*}
\begin{equation*}
\left[ \frac{{w}^{cos}_i}{{w}^{cos}_j} \right] =
\begin{pmatrix}
$\,\,$ 1 $\,\,$ & $\,\,$\color{red} 1.0533\color{black} $\,\,$ & $\,\,$7.3533$\,\,$ & $\,\,$2.6286$\,\,$ \\
$\,\,$\color{red} 0.9494\color{black} $\,\,$ & $\,\,$ 1 $\,\,$ & $\,\,$\color{red} 6.9809\color{black} $\,\,$ & $\,\,$\color{red} 2.4955\color{black}   $\,\,$ \\
$\,\,$0.1360$\,\,$ & $\,\,$\color{red} 0.1432\color{black} $\,\,$ & $\,\,$ 1 $\,\,$ & $\,\,$0.3575 $\,\,$ \\
$\,\,$0.3804$\,\,$ & $\,\,$\color{red} 0.4007\color{black} $\,\,$ & $\,\,$2.7974$\,\,$ & $\,\,$ 1  $\,\,$ \\
\end{pmatrix},
\end{equation*}

\begin{equation*}
\mathbf{w}^{\prime} =
\begin{pmatrix}
0.405122\\
0.385660\\
0.055094\\
0.154123
\end{pmatrix} =
0.998950\cdot
\begin{pmatrix}
0.405548\\
\color{gr} 0.386065\color{black} \\
0.055152\\
0.154285
\end{pmatrix},
\end{equation*}
\begin{equation*}
\left[ \frac{{w}^{\prime}_i}{{w}^{\prime}_j} \right] =
\begin{pmatrix}
$\,\,$ 1 $\,\,$ & $\,\,$\color{gr} 1.0505\color{black} $\,\,$ & $\,\,$7.3533$\,\,$ & $\,\,$2.6286$\,\,$ \\
$\,\,$\color{gr} 0.9520\color{black} $\,\,$ & $\,\,$ 1 $\,\,$ & $\,\,$\color{gr} \color{blue} 7\color{black} $\,\,$ & $\,\,$\color{gr} 2.5023\color{black}   $\,\,$ \\
$\,\,$0.1360$\,\,$ & $\,\,$\color{gr} \color{blue}  1/7\color{black} $\,\,$ & $\,\,$ 1 $\,\,$ & $\,\,$0.3575 $\,\,$ \\
$\,\,$0.3804$\,\,$ & $\,\,$\color{gr} 0.3996\color{black} $\,\,$ & $\,\,$2.7974$\,\,$ & $\,\,$ 1  $\,\,$ \\
\end{pmatrix},
\end{equation*}
\end{example}
\newpage
\begin{example}
\begin{equation*}
\mathbf{A} =
\begin{pmatrix}
$\,\,$ 1 $\,\,$ & $\,\,$1$\,\,$ & $\,\,$5$\,\,$ & $\,\,$5 $\,\,$ \\
$\,\,$ 1 $\,\,$ & $\,\,$ 1 $\,\,$ & $\,\,$8$\,\,$ & $\,\,$3 $\,\,$ \\
$\,\,$ 1/5$\,\,$ & $\,\,$ 1/8$\,\,$ & $\,\,$ 1 $\,\,$ & $\,\,$ 1/4 $\,\,$ \\
$\,\,$ 1/5$\,\,$ & $\,\,$ 1/3$\,\,$ & $\,\,$4$\,\,$ & $\,\,$ 1  $\,\,$ \\
\end{pmatrix},
\qquad
\lambda_{\max} =
4.1655,
\qquad
CR = 0.0624
\end{equation*}

\begin{equation*}
\mathbf{w}^{cos} =
\begin{pmatrix}
0.406760\\
\color{red} 0.399797\color{black} \\
0.054613\\
0.138829
\end{pmatrix}\end{equation*}
\begin{equation*}
\left[ \frac{{w}^{cos}_i}{{w}^{cos}_j} \right] =
\begin{pmatrix}
$\,\,$ 1 $\,\,$ & $\,\,$\color{red} 1.0174\color{black} $\,\,$ & $\,\,$7.4480$\,\,$ & $\,\,$2.9299$\,\,$ \\
$\,\,$\color{red} 0.9829\color{black} $\,\,$ & $\,\,$ 1 $\,\,$ & $\,\,$\color{red} 7.3205\color{black} $\,\,$ & $\,\,$\color{red} 2.8798\color{black}   $\,\,$ \\
$\,\,$0.1343$\,\,$ & $\,\,$\color{red} 0.1366\color{black} $\,\,$ & $\,\,$ 1 $\,\,$ & $\,\,$0.3934 $\,\,$ \\
$\,\,$0.3413$\,\,$ & $\,\,$\color{red} 0.3472\color{black} $\,\,$ & $\,\,$2.5421$\,\,$ & $\,\,$ 1  $\,\,$ \\
\end{pmatrix},
\end{equation*}

\begin{equation*}
\mathbf{w}^{\prime} =
\begin{pmatrix}
0.403948\\
0.403948\\
0.054235\\
0.137869
\end{pmatrix} =
0.993085\cdot
\begin{pmatrix}
0.406760\\
\color{gr} 0.406760\color{black} \\
0.054613\\
0.138829
\end{pmatrix},
\end{equation*}
\begin{equation*}
\left[ \frac{{w}^{\prime}_i}{{w}^{\prime}_j} \right] =
\begin{pmatrix}
$\,\,$ 1 $\,\,$ & $\,\,$\color{gr} \color{blue} 1\color{black} $\,\,$ & $\,\,$7.4480$\,\,$ & $\,\,$2.9299$\,\,$ \\
$\,\,$\color{gr} \color{blue} 1\color{black} $\,\,$ & $\,\,$ 1 $\,\,$ & $\,\,$\color{gr} 7.4480\color{black} $\,\,$ & $\,\,$\color{gr} 2.9299\color{black}   $\,\,$ \\
$\,\,$0.1343$\,\,$ & $\,\,$\color{gr} 0.1343\color{black} $\,\,$ & $\,\,$ 1 $\,\,$ & $\,\,$0.3934 $\,\,$ \\
$\,\,$0.3413$\,\,$ & $\,\,$\color{gr} 0.3413\color{black} $\,\,$ & $\,\,$2.5421$\,\,$ & $\,\,$ 1  $\,\,$ \\
\end{pmatrix},
\end{equation*}
\end{example}
\newpage
\begin{example}
\begin{equation*}
\mathbf{A} =
\begin{pmatrix}
$\,\,$ 1 $\,\,$ & $\,\,$1$\,\,$ & $\,\,$5$\,\,$ & $\,\,$5 $\,\,$ \\
$\,\,$ 1 $\,\,$ & $\,\,$ 1 $\,\,$ & $\,\,$8$\,\,$ & $\,\,$3 $\,\,$ \\
$\,\,$ 1/5$\,\,$ & $\,\,$ 1/8$\,\,$ & $\,\,$ 1 $\,\,$ & $\,\,$ 1/5 $\,\,$ \\
$\,\,$ 1/5$\,\,$ & $\,\,$ 1/3$\,\,$ & $\,\,$5$\,\,$ & $\,\,$ 1  $\,\,$ \\
\end{pmatrix},
\qquad
\lambda_{\max} =
4.2259,
\qquad
CR = 0.0852
\end{equation*}

\begin{equation*}
\mathbf{w}^{cos} =
\begin{pmatrix}
0.402952\\
\color{red} 0.394246\color{black} \\
0.052642\\
0.150160
\end{pmatrix}\end{equation*}
\begin{equation*}
\left[ \frac{{w}^{cos}_i}{{w}^{cos}_j} \right] =
\begin{pmatrix}
$\,\,$ 1 $\,\,$ & $\,\,$\color{red} 1.0221\color{black} $\,\,$ & $\,\,$7.6546$\,\,$ & $\,\,$2.6835$\,\,$ \\
$\,\,$\color{red} 0.9784\color{black} $\,\,$ & $\,\,$ 1 $\,\,$ & $\,\,$\color{red} 7.4892\color{black} $\,\,$ & $\,\,$\color{red} 2.6255\color{black}   $\,\,$ \\
$\,\,$0.1306$\,\,$ & $\,\,$\color{red} 0.1335\color{black} $\,\,$ & $\,\,$ 1 $\,\,$ & $\,\,$0.3506 $\,\,$ \\
$\,\,$0.3726$\,\,$ & $\,\,$\color{red} 0.3809\color{black} $\,\,$ & $\,\,$2.8525$\,\,$ & $\,\,$ 1  $\,\,$ \\
\end{pmatrix},
\end{equation*}

\begin{equation*}
\mathbf{w}^{\prime} =
\begin{pmatrix}
0.399474\\
0.399474\\
0.052188\\
0.148864
\end{pmatrix} =
0.991369\cdot
\begin{pmatrix}
0.402952\\
\color{gr} 0.402952\color{black} \\
0.052642\\
0.150160
\end{pmatrix},
\end{equation*}
\begin{equation*}
\left[ \frac{{w}^{\prime}_i}{{w}^{\prime}_j} \right] =
\begin{pmatrix}
$\,\,$ 1 $\,\,$ & $\,\,$\color{gr} \color{blue} 1\color{black} $\,\,$ & $\,\,$7.6546$\,\,$ & $\,\,$2.6835$\,\,$ \\
$\,\,$\color{gr} \color{blue} 1\color{black} $\,\,$ & $\,\,$ 1 $\,\,$ & $\,\,$\color{gr} 7.6546\color{black} $\,\,$ & $\,\,$\color{gr} 2.6835\color{black}   $\,\,$ \\
$\,\,$0.1306$\,\,$ & $\,\,$\color{gr} 0.1306\color{black} $\,\,$ & $\,\,$ 1 $\,\,$ & $\,\,$0.3506 $\,\,$ \\
$\,\,$0.3726$\,\,$ & $\,\,$\color{gr} 0.3726\color{black} $\,\,$ & $\,\,$2.8525$\,\,$ & $\,\,$ 1  $\,\,$ \\
\end{pmatrix},
\end{equation*}
\end{example}
\newpage
\begin{example}
\begin{equation*}
\mathbf{A} =
\begin{pmatrix}
$\,\,$ 1 $\,\,$ & $\,\,$1$\,\,$ & $\,\,$5$\,\,$ & $\,\,$6 $\,\,$ \\
$\,\,$ 1 $\,\,$ & $\,\,$ 1 $\,\,$ & $\,\,$6$\,\,$ & $\,\,$4 $\,\,$ \\
$\,\,$ 1/5$\,\,$ & $\,\,$ 1/6$\,\,$ & $\,\,$ 1 $\,\,$ & $\,\,$ 1/2 $\,\,$ \\
$\,\,$ 1/6$\,\,$ & $\,\,$ 1/4$\,\,$ & $\,\,$2$\,\,$ & $\,\,$ 1  $\,\,$ \\
\end{pmatrix},
\qquad
\lambda_{\max} =
4.0662,
\qquad
CR = 0.0250
\end{equation*}

\begin{equation*}
\mathbf{w}^{cos} =
\begin{pmatrix}
0.426924\\
\color{red} 0.404105\color{black} \\
0.067402\\
0.101569
\end{pmatrix}\end{equation*}
\begin{equation*}
\left[ \frac{{w}^{cos}_i}{{w}^{cos}_j} \right] =
\begin{pmatrix}
$\,\,$ 1 $\,\,$ & $\,\,$\color{red} 1.0565\color{black} $\,\,$ & $\,\,$6.3340$\,\,$ & $\,\,$4.2033$\,\,$ \\
$\,\,$\color{red} 0.9466\color{black} $\,\,$ & $\,\,$ 1 $\,\,$ & $\,\,$\color{red} 5.9954\color{black} $\,\,$ & $\,\,$\color{red} 3.9786\color{black}   $\,\,$ \\
$\,\,$0.1579$\,\,$ & $\,\,$\color{red} 0.1668\color{black} $\,\,$ & $\,\,$ 1 $\,\,$ & $\,\,$0.6636 $\,\,$ \\
$\,\,$0.2379$\,\,$ & $\,\,$\color{red} 0.2513\color{black} $\,\,$ & $\,\,$1.5069$\,\,$ & $\,\,$ 1  $\,\,$ \\
\end{pmatrix},
\end{equation*}

\begin{equation*}
\mathbf{w}^{\prime} =
\begin{pmatrix}
0.426792\\
0.404289\\
0.067381\\
0.101538
\end{pmatrix} =
0.999691\cdot
\begin{pmatrix}
0.426924\\
\color{gr} 0.404414\color{black} \\
0.067402\\
0.101569
\end{pmatrix},
\end{equation*}
\begin{equation*}
\left[ \frac{{w}^{\prime}_i}{{w}^{\prime}_j} \right] =
\begin{pmatrix}
$\,\,$ 1 $\,\,$ & $\,\,$\color{gr} 1.0557\color{black} $\,\,$ & $\,\,$6.3340$\,\,$ & $\,\,$4.2033$\,\,$ \\
$\,\,$\color{gr} 0.9473\color{black} $\,\,$ & $\,\,$ 1 $\,\,$ & $\,\,$\color{gr} \color{blue} 6\color{black} $\,\,$ & $\,\,$\color{gr} 3.9817\color{black}   $\,\,$ \\
$\,\,$0.1579$\,\,$ & $\,\,$\color{gr} \color{blue}  1/6\color{black} $\,\,$ & $\,\,$ 1 $\,\,$ & $\,\,$0.6636 $\,\,$ \\
$\,\,$0.2379$\,\,$ & $\,\,$\color{gr} 0.2512\color{black} $\,\,$ & $\,\,$1.5069$\,\,$ & $\,\,$ 1  $\,\,$ \\
\end{pmatrix},
\end{equation*}
\end{example}
\newpage
\begin{example}
\begin{equation*}
\mathbf{A} =
\begin{pmatrix}
$\,\,$ 1 $\,\,$ & $\,\,$1$\,\,$ & $\,\,$5$\,\,$ & $\,\,$6 $\,\,$ \\
$\,\,$ 1 $\,\,$ & $\,\,$ 1 $\,\,$ & $\,\,$7$\,\,$ & $\,\,$3 $\,\,$ \\
$\,\,$ 1/5$\,\,$ & $\,\,$ 1/7$\,\,$ & $\,\,$ 1 $\,\,$ & $\,\,$ 1/4 $\,\,$ \\
$\,\,$ 1/6$\,\,$ & $\,\,$ 1/3$\,\,$ & $\,\,$4$\,\,$ & $\,\,$ 1  $\,\,$ \\
\end{pmatrix},
\qquad
\lambda_{\max} =
4.2174,
\qquad
CR = 0.0820
\end{equation*}

\begin{equation*}
\mathbf{w}^{cos} =
\begin{pmatrix}
0.420330\\
\color{red} 0.385304\color{black} \\
0.057067\\
0.137299
\end{pmatrix}\end{equation*}
\begin{equation*}
\left[ \frac{{w}^{cos}_i}{{w}^{cos}_j} \right] =
\begin{pmatrix}
$\,\,$ 1 $\,\,$ & $\,\,$\color{red} 1.0909\color{black} $\,\,$ & $\,\,$7.3655$\,\,$ & $\,\,$3.0614$\,\,$ \\
$\,\,$\color{red} 0.9167\color{black} $\,\,$ & $\,\,$ 1 $\,\,$ & $\,\,$\color{red} 6.7517\color{black} $\,\,$ & $\,\,$\color{red} 2.8063\color{black}   $\,\,$ \\
$\,\,$0.1358$\,\,$ & $\,\,$\color{red} 0.1481\color{black} $\,\,$ & $\,\,$ 1 $\,\,$ & $\,\,$0.4156 $\,\,$ \\
$\,\,$0.3266$\,\,$ & $\,\,$\color{red} 0.3563\color{black} $\,\,$ & $\,\,$2.4059$\,\,$ & $\,\,$ 1  $\,\,$ \\
\end{pmatrix},
\end{equation*}

\begin{equation*}
\mathbf{w}^{\prime} =
\begin{pmatrix}
0.414458\\
0.393891\\
0.056270\\
0.135381
\end{pmatrix} =
0.986030\cdot
\begin{pmatrix}
0.420330\\
\color{gr} 0.399471\color{black} \\
0.057067\\
0.137299
\end{pmatrix},
\end{equation*}
\begin{equation*}
\left[ \frac{{w}^{\prime}_i}{{w}^{\prime}_j} \right] =
\begin{pmatrix}
$\,\,$ 1 $\,\,$ & $\,\,$\color{gr} 1.0522\color{black} $\,\,$ & $\,\,$7.3655$\,\,$ & $\,\,$3.0614$\,\,$ \\
$\,\,$\color{gr} 0.9504\color{black} $\,\,$ & $\,\,$ 1 $\,\,$ & $\,\,$\color{gr} \color{blue} 7\color{black} $\,\,$ & $\,\,$\color{gr} 2.9095\color{black}   $\,\,$ \\
$\,\,$0.1358$\,\,$ & $\,\,$\color{gr} \color{blue}  1/7\color{black} $\,\,$ & $\,\,$ 1 $\,\,$ & $\,\,$0.4156 $\,\,$ \\
$\,\,$0.3266$\,\,$ & $\,\,$\color{gr} 0.3437\color{black} $\,\,$ & $\,\,$2.4059$\,\,$ & $\,\,$ 1  $\,\,$ \\
\end{pmatrix},
\end{equation*}
\end{example}
\newpage
\begin{example}
\begin{equation*}
\mathbf{A} =
\begin{pmatrix}
$\,\,$ 1 $\,\,$ & $\,\,$1$\,\,$ & $\,\,$5$\,\,$ & $\,\,$6 $\,\,$ \\
$\,\,$ 1 $\,\,$ & $\,\,$ 1 $\,\,$ & $\,\,$7$\,\,$ & $\,\,$4 $\,\,$ \\
$\,\,$ 1/5$\,\,$ & $\,\,$ 1/7$\,\,$ & $\,\,$ 1 $\,\,$ & $\,\,$ 1/3 $\,\,$ \\
$\,\,$ 1/6$\,\,$ & $\,\,$ 1/4$\,\,$ & $\,\,$3$\,\,$ & $\,\,$ 1  $\,\,$ \\
\end{pmatrix},
\qquad
\lambda_{\max} =
4.1417,
\qquad
CR = 0.0534
\end{equation*}

\begin{equation*}
\mathbf{w}^{cos} =
\begin{pmatrix}
0.417455\\
\color{red} 0.408943\color{black} \\
0.059491\\
0.114111
\end{pmatrix}\end{equation*}
\begin{equation*}
\left[ \frac{{w}^{cos}_i}{{w}^{cos}_j} \right] =
\begin{pmatrix}
$\,\,$ 1 $\,\,$ & $\,\,$\color{red} 1.0208\color{black} $\,\,$ & $\,\,$7.0172$\,\,$ & $\,\,$3.6583$\,\,$ \\
$\,\,$\color{red} 0.9796\color{black} $\,\,$ & $\,\,$ 1 $\,\,$ & $\,\,$\color{red} 6.8741\color{black} $\,\,$ & $\,\,$\color{red} 3.5837\color{black}   $\,\,$ \\
$\,\,$0.1425$\,\,$ & $\,\,$\color{red} 0.1455\color{black} $\,\,$ & $\,\,$ 1 $\,\,$ & $\,\,$0.5213 $\,\,$ \\
$\,\,$0.2734$\,\,$ & $\,\,$\color{red} 0.2790\color{black} $\,\,$ & $\,\,$1.9181$\,\,$ & $\,\,$ 1  $\,\,$ \\
\end{pmatrix},
\end{equation*}

\begin{equation*}
\mathbf{w}^{\prime} =
\begin{pmatrix}
0.414351\\
0.413338\\
0.059048\\
0.113263
\end{pmatrix} =
0.992565\cdot
\begin{pmatrix}
0.417455\\
\color{gr} 0.416434\color{black} \\
0.059491\\
0.114111
\end{pmatrix},
\end{equation*}
\begin{equation*}
\left[ \frac{{w}^{\prime}_i}{{w}^{\prime}_j} \right] =
\begin{pmatrix}
$\,\,$ 1 $\,\,$ & $\,\,$\color{gr} 1.0025\color{black} $\,\,$ & $\,\,$7.0172$\,\,$ & $\,\,$3.6583$\,\,$ \\
$\,\,$\color{gr} 0.9976\color{black} $\,\,$ & $\,\,$ 1 $\,\,$ & $\,\,$\color{gr} \color{blue} 7\color{black} $\,\,$ & $\,\,$\color{gr} 3.6494\color{black}   $\,\,$ \\
$\,\,$0.1425$\,\,$ & $\,\,$\color{gr} \color{blue}  1/7\color{black} $\,\,$ & $\,\,$ 1 $\,\,$ & $\,\,$0.5213 $\,\,$ \\
$\,\,$0.2734$\,\,$ & $\,\,$\color{gr} 0.2740\color{black} $\,\,$ & $\,\,$1.9181$\,\,$ & $\,\,$ 1  $\,\,$ \\
\end{pmatrix},
\end{equation*}
\end{example}
\newpage
\begin{example}
\begin{equation*}
\mathbf{A} =
\begin{pmatrix}
$\,\,$ 1 $\,\,$ & $\,\,$1$\,\,$ & $\,\,$5$\,\,$ & $\,\,$6 $\,\,$ \\
$\,\,$ 1 $\,\,$ & $\,\,$ 1 $\,\,$ & $\,\,$8$\,\,$ & $\,\,$3 $\,\,$ \\
$\,\,$ 1/5$\,\,$ & $\,\,$ 1/8$\,\,$ & $\,\,$ 1 $\,\,$ & $\,\,$ 1/4 $\,\,$ \\
$\,\,$ 1/6$\,\,$ & $\,\,$ 1/3$\,\,$ & $\,\,$4$\,\,$ & $\,\,$ 1  $\,\,$ \\
\end{pmatrix},
\qquad
\lambda_{\max} =
4.2162,
\qquad
CR = 0.0815
\end{equation*}

\begin{equation*}
\mathbf{w}^{cos} =
\begin{pmatrix}
0.417418\\
\color{red} 0.394559\color{black} \\
0.054440\\
0.133584
\end{pmatrix}\end{equation*}
\begin{equation*}
\left[ \frac{{w}^{cos}_i}{{w}^{cos}_j} \right] =
\begin{pmatrix}
$\,\,$ 1 $\,\,$ & $\,\,$\color{red} 1.0579\color{black} $\,\,$ & $\,\,$7.6675$\,\,$ & $\,\,$3.1248$\,\,$ \\
$\,\,$\color{red} 0.9452\color{black} $\,\,$ & $\,\,$ 1 $\,\,$ & $\,\,$\color{red} 7.2477\color{black} $\,\,$ & $\,\,$\color{red} 2.9536\color{black}   $\,\,$ \\
$\,\,$0.1304$\,\,$ & $\,\,$\color{red} 0.1380\color{black} $\,\,$ & $\,\,$ 1 $\,\,$ & $\,\,$0.4075 $\,\,$ \\
$\,\,$0.3200$\,\,$ & $\,\,$\color{red} 0.3386\color{black} $\,\,$ & $\,\,$2.4538$\,\,$ & $\,\,$ 1  $\,\,$ \\
\end{pmatrix},
\end{equation*}

\begin{equation*}
\mathbf{w}^{\prime} =
\begin{pmatrix}
0.414849\\
0.398285\\
0.054105\\
0.132762
\end{pmatrix} =
0.993846\cdot
\begin{pmatrix}
0.417418\\
\color{gr} 0.400751\color{black} \\
0.054440\\
0.133584
\end{pmatrix},
\end{equation*}
\begin{equation*}
\left[ \frac{{w}^{\prime}_i}{{w}^{\prime}_j} \right] =
\begin{pmatrix}
$\,\,$ 1 $\,\,$ & $\,\,$\color{gr} 1.0416\color{black} $\,\,$ & $\,\,$7.6675$\,\,$ & $\,\,$3.1248$\,\,$ \\
$\,\,$\color{gr} 0.9601\color{black} $\,\,$ & $\,\,$ 1 $\,\,$ & $\,\,$\color{gr} 7.3614\color{black} $\,\,$ & $\,\,$\color{gr} \color{blue} 3\color{black}   $\,\,$ \\
$\,\,$0.1304$\,\,$ & $\,\,$\color{gr} 0.1358\color{black} $\,\,$ & $\,\,$ 1 $\,\,$ & $\,\,$0.4075 $\,\,$ \\
$\,\,$0.3200$\,\,$ & $\,\,$\color{gr} \color{blue}  1/3\color{black} $\,\,$ & $\,\,$2.4538$\,\,$ & $\,\,$ 1  $\,\,$ \\
\end{pmatrix},
\end{equation*}
\end{example}
\newpage
\begin{example}
\begin{equation*}
\mathbf{A} =
\begin{pmatrix}
$\,\,$ 1 $\,\,$ & $\,\,$1$\,\,$ & $\,\,$5$\,\,$ & $\,\,$7 $\,\,$ \\
$\,\,$ 1 $\,\,$ & $\,\,$ 1 $\,\,$ & $\,\,$7$\,\,$ & $\,\,$4 $\,\,$ \\
$\,\,$ 1/5$\,\,$ & $\,\,$ 1/7$\,\,$ & $\,\,$ 1 $\,\,$ & $\,\,$ 1/3 $\,\,$ \\
$\,\,$ 1/7$\,\,$ & $\,\,$ 1/4$\,\,$ & $\,\,$3$\,\,$ & $\,\,$ 1  $\,\,$ \\
\end{pmatrix},
\qquad
\lambda_{\max} =
4.1793,
\qquad
CR = 0.0676
\end{equation*}

\begin{equation*}
\mathbf{w}^{cos} =
\begin{pmatrix}
0.426692\\
\color{red} 0.403687\color{black} \\
0.059276\\
0.110344
\end{pmatrix}\end{equation*}
\begin{equation*}
\left[ \frac{{w}^{cos}_i}{{w}^{cos}_j} \right] =
\begin{pmatrix}
$\,\,$ 1 $\,\,$ & $\,\,$\color{red} 1.0570\color{black} $\,\,$ & $\,\,$7.1984$\,\,$ & $\,\,$3.8669$\,\,$ \\
$\,\,$\color{red} 0.9461\color{black} $\,\,$ & $\,\,$ 1 $\,\,$ & $\,\,$\color{red} 6.8103\color{black} $\,\,$ & $\,\,$\color{red} 3.6584\color{black}   $\,\,$ \\
$\,\,$0.1389$\,\,$ & $\,\,$\color{red} 0.1468\color{black} $\,\,$ & $\,\,$ 1 $\,\,$ & $\,\,$0.5372 $\,\,$ \\
$\,\,$0.2586$\,\,$ & $\,\,$\color{red} 0.2733\color{black} $\,\,$ & $\,\,$1.8615$\,\,$ & $\,\,$ 1  $\,\,$ \\
\end{pmatrix},
\end{equation*}

\begin{equation*}
\mathbf{w}^{\prime} =
\begin{pmatrix}
0.421947\\
0.410318\\
0.058617\\
0.109117
\end{pmatrix} =
0.988880\cdot
\begin{pmatrix}
0.426692\\
\color{gr} 0.414932\color{black} \\
0.059276\\
0.110344
\end{pmatrix},
\end{equation*}
\begin{equation*}
\left[ \frac{{w}^{\prime}_i}{{w}^{\prime}_j} \right] =
\begin{pmatrix}
$\,\,$ 1 $\,\,$ & $\,\,$\color{gr} 1.0283\color{black} $\,\,$ & $\,\,$7.1984$\,\,$ & $\,\,$3.8669$\,\,$ \\
$\,\,$\color{gr} 0.9724\color{black} $\,\,$ & $\,\,$ 1 $\,\,$ & $\,\,$\color{gr} \color{blue} 7\color{black} $\,\,$ & $\,\,$\color{gr} 3.7603\color{black}   $\,\,$ \\
$\,\,$0.1389$\,\,$ & $\,\,$\color{gr} \color{blue}  1/7\color{black} $\,\,$ & $\,\,$ 1 $\,\,$ & $\,\,$0.5372 $\,\,$ \\
$\,\,$0.2586$\,\,$ & $\,\,$\color{gr} 0.2659\color{black} $\,\,$ & $\,\,$1.8615$\,\,$ & $\,\,$ 1  $\,\,$ \\
\end{pmatrix},
\end{equation*}
\end{example}
\newpage
\begin{example}
\begin{equation*}
\mathbf{A} =
\begin{pmatrix}
$\,\,$ 1 $\,\,$ & $\,\,$1$\,\,$ & $\,\,$5$\,\,$ & $\,\,$7 $\,\,$ \\
$\,\,$ 1 $\,\,$ & $\,\,$ 1 $\,\,$ & $\,\,$7$\,\,$ & $\,\,$5 $\,\,$ \\
$\,\,$ 1/5$\,\,$ & $\,\,$ 1/7$\,\,$ & $\,\,$ 1 $\,\,$ & $\,\,$ 1/2 $\,\,$ \\
$\,\,$ 1/7$\,\,$ & $\,\,$ 1/5$\,\,$ & $\,\,$2$\,\,$ & $\,\,$ 1  $\,\,$ \\
\end{pmatrix},
\qquad
\lambda_{\max} =
4.0899,
\qquad
CR = 0.0339
\end{equation*}

\begin{equation*}
\mathbf{w}^{cos} =
\begin{pmatrix}
0.424510\\
\color{red} 0.423659\color{black} \\
0.062844\\
0.088987
\end{pmatrix}\end{equation*}
\begin{equation*}
\left[ \frac{{w}^{cos}_i}{{w}^{cos}_j} \right] =
\begin{pmatrix}
$\,\,$ 1 $\,\,$ & $\,\,$\color{red} 1.0020\color{black} $\,\,$ & $\,\,$6.7550$\,\,$ & $\,\,$4.7705$\,\,$ \\
$\,\,$\color{red} 0.9980\color{black} $\,\,$ & $\,\,$ 1 $\,\,$ & $\,\,$\color{red} 6.7414\color{black} $\,\,$ & $\,\,$\color{red} 4.7609\color{black}   $\,\,$ \\
$\,\,$0.1480$\,\,$ & $\,\,$\color{red} 0.1483\color{black} $\,\,$ & $\,\,$ 1 $\,\,$ & $\,\,$0.7062 $\,\,$ \\
$\,\,$0.2096$\,\,$ & $\,\,$\color{red} 0.2100\color{black} $\,\,$ & $\,\,$1.4160$\,\,$ & $\,\,$ 1  $\,\,$ \\
\end{pmatrix},
\end{equation*}

\begin{equation*}
\mathbf{w}^{\prime} =
\begin{pmatrix}
0.424149\\
0.424149\\
0.062790\\
0.088912
\end{pmatrix} =
0.999150\cdot
\begin{pmatrix}
0.424510\\
\color{gr} 0.424510\color{black} \\
0.062844\\
0.088987
\end{pmatrix},
\end{equation*}
\begin{equation*}
\left[ \frac{{w}^{\prime}_i}{{w}^{\prime}_j} \right] =
\begin{pmatrix}
$\,\,$ 1 $\,\,$ & $\,\,$\color{gr} \color{blue} 1\color{black} $\,\,$ & $\,\,$6.7550$\,\,$ & $\,\,$4.7705$\,\,$ \\
$\,\,$\color{gr} \color{blue} 1\color{black} $\,\,$ & $\,\,$ 1 $\,\,$ & $\,\,$\color{gr} 6.7550\color{black} $\,\,$ & $\,\,$\color{gr} 4.7705\color{black}   $\,\,$ \\
$\,\,$0.1480$\,\,$ & $\,\,$\color{gr} 0.1480\color{black} $\,\,$ & $\,\,$ 1 $\,\,$ & $\,\,$0.7062 $\,\,$ \\
$\,\,$0.2096$\,\,$ & $\,\,$\color{gr} 0.2096\color{black} $\,\,$ & $\,\,$1.4160$\,\,$ & $\,\,$ 1  $\,\,$ \\
\end{pmatrix},
\end{equation*}
\end{example}
\newpage
\begin{example}
\begin{equation*}
\mathbf{A} =
\begin{pmatrix}
$\,\,$ 1 $\,\,$ & $\,\,$1$\,\,$ & $\,\,$5$\,\,$ & $\,\,$7 $\,\,$ \\
$\,\,$ 1 $\,\,$ & $\,\,$ 1 $\,\,$ & $\,\,$8$\,\,$ & $\,\,$4 $\,\,$ \\
$\,\,$ 1/5$\,\,$ & $\,\,$ 1/8$\,\,$ & $\,\,$ 1 $\,\,$ & $\,\,$ 1/3 $\,\,$ \\
$\,\,$ 1/7$\,\,$ & $\,\,$ 1/4$\,\,$ & $\,\,$3$\,\,$ & $\,\,$ 1  $\,\,$ \\
\end{pmatrix},
\qquad
\lambda_{\max} =
4.1782,
\qquad
CR = 0.0672
\end{equation*}

\begin{equation*}
\mathbf{w}^{cos} =
\begin{pmatrix}
0.423327\\
\color{red} 0.412958\color{black} \\
0.056501\\
0.107215
\end{pmatrix}\end{equation*}
\begin{equation*}
\left[ \frac{{w}^{cos}_i}{{w}^{cos}_j} \right] =
\begin{pmatrix}
$\,\,$ 1 $\,\,$ & $\,\,$\color{red} 1.0251\color{black} $\,\,$ & $\,\,$7.4924$\,\,$ & $\,\,$3.9484$\,\,$ \\
$\,\,$\color{red} 0.9755\color{black} $\,\,$ & $\,\,$ 1 $\,\,$ & $\,\,$\color{red} 7.3089\color{black} $\,\,$ & $\,\,$\color{red} 3.8517\color{black}   $\,\,$ \\
$\,\,$0.1335$\,\,$ & $\,\,$\color{red} 0.1368\color{black} $\,\,$ & $\,\,$ 1 $\,\,$ & $\,\,$0.5270 $\,\,$ \\
$\,\,$0.2533$\,\,$ & $\,\,$\color{red} 0.2596\color{black} $\,\,$ & $\,\,$1.8976$\,\,$ & $\,\,$ 1  $\,\,$ \\
\end{pmatrix},
\end{equation*}

\begin{equation*}
\mathbf{w}^{\prime} =
\begin{pmatrix}
0.418982\\
0.418982\\
0.055921\\
0.106115
\end{pmatrix} =
0.989737\cdot
\begin{pmatrix}
0.423327\\
\color{gr} 0.423327\color{black} \\
0.056501\\
0.107215
\end{pmatrix},
\end{equation*}
\begin{equation*}
\left[ \frac{{w}^{\prime}_i}{{w}^{\prime}_j} \right] =
\begin{pmatrix}
$\,\,$ 1 $\,\,$ & $\,\,$\color{gr} \color{blue} 1\color{black} $\,\,$ & $\,\,$7.4924$\,\,$ & $\,\,$3.9484$\,\,$ \\
$\,\,$\color{gr} \color{blue} 1\color{black} $\,\,$ & $\,\,$ 1 $\,\,$ & $\,\,$\color{gr} 7.4924\color{black} $\,\,$ & $\,\,$\color{gr} 3.9484\color{black}   $\,\,$ \\
$\,\,$0.1335$\,\,$ & $\,\,$\color{gr} 0.1335\color{black} $\,\,$ & $\,\,$ 1 $\,\,$ & $\,\,$0.5270 $\,\,$ \\
$\,\,$0.2533$\,\,$ & $\,\,$\color{gr} 0.2533\color{black} $\,\,$ & $\,\,$1.8976$\,\,$ & $\,\,$ 1  $\,\,$ \\
\end{pmatrix},
\end{equation*}
\end{example}
\newpage
\begin{example}
\begin{equation*}
\mathbf{A} =
\begin{pmatrix}
$\,\,$ 1 $\,\,$ & $\,\,$1$\,\,$ & $\,\,$5$\,\,$ & $\,\,$7 $\,\,$ \\
$\,\,$ 1 $\,\,$ & $\,\,$ 1 $\,\,$ & $\,\,$8$\,\,$ & $\,\,$4 $\,\,$ \\
$\,\,$ 1/5$\,\,$ & $\,\,$ 1/8$\,\,$ & $\,\,$ 1 $\,\,$ & $\,\,$ 1/4 $\,\,$ \\
$\,\,$ 1/7$\,\,$ & $\,\,$ 1/4$\,\,$ & $\,\,$4$\,\,$ & $\,\,$ 1  $\,\,$ \\
\end{pmatrix},
\qquad
\lambda_{\max} =
4.2610,
\qquad
CR = 0.0984
\end{equation*}

\begin{equation*}
\mathbf{w}^{cos} =
\begin{pmatrix}
0.418846\\
\color{red} 0.406954\color{black} \\
0.054158\\
0.120042
\end{pmatrix}\end{equation*}
\begin{equation*}
\left[ \frac{{w}^{cos}_i}{{w}^{cos}_j} \right] =
\begin{pmatrix}
$\,\,$ 1 $\,\,$ & $\,\,$\color{red} 1.0292\color{black} $\,\,$ & $\,\,$7.7338$\,\,$ & $\,\,$3.4892$\,\,$ \\
$\,\,$\color{red} 0.9716\color{black} $\,\,$ & $\,\,$ 1 $\,\,$ & $\,\,$\color{red} 7.5142\color{black} $\,\,$ & $\,\,$\color{red} 3.3901\color{black}   $\,\,$ \\
$\,\,$0.1293$\,\,$ & $\,\,$\color{red} 0.1331\color{black} $\,\,$ & $\,\,$ 1 $\,\,$ & $\,\,$0.4512 $\,\,$ \\
$\,\,$0.2866$\,\,$ & $\,\,$\color{red} 0.2950\color{black} $\,\,$ & $\,\,$2.2165$\,\,$ & $\,\,$ 1  $\,\,$ \\
\end{pmatrix},
\end{equation*}

\begin{equation*}
\mathbf{w}^{\prime} =
\begin{pmatrix}
0.413924\\
0.413924\\
0.053521\\
0.118631
\end{pmatrix} =
0.988248\cdot
\begin{pmatrix}
0.418846\\
\color{gr} 0.418846\color{black} \\
0.054158\\
0.120042
\end{pmatrix},
\end{equation*}
\begin{equation*}
\left[ \frac{{w}^{\prime}_i}{{w}^{\prime}_j} \right] =
\begin{pmatrix}
$\,\,$ 1 $\,\,$ & $\,\,$\color{gr} \color{blue} 1\color{black} $\,\,$ & $\,\,$7.7338$\,\,$ & $\,\,$3.4892$\,\,$ \\
$\,\,$\color{gr} \color{blue} 1\color{black} $\,\,$ & $\,\,$ 1 $\,\,$ & $\,\,$\color{gr} 7.7338\color{black} $\,\,$ & $\,\,$\color{gr} 3.4892\color{black}   $\,\,$ \\
$\,\,$0.1293$\,\,$ & $\,\,$\color{gr} 0.1293\color{black} $\,\,$ & $\,\,$ 1 $\,\,$ & $\,\,$0.4512 $\,\,$ \\
$\,\,$0.2866$\,\,$ & $\,\,$\color{gr} 0.2866\color{black} $\,\,$ & $\,\,$2.2165$\,\,$ & $\,\,$ 1  $\,\,$ \\
\end{pmatrix},
\end{equation*}
\end{example}
\newpage
\begin{example}
\begin{equation*}
\mathbf{A} =
\begin{pmatrix}
$\,\,$ 1 $\,\,$ & $\,\,$1$\,\,$ & $\,\,$5$\,\,$ & $\,\,$7 $\,\,$ \\
$\,\,$ 1 $\,\,$ & $\,\,$ 1 $\,\,$ & $\,\,$9$\,\,$ & $\,\,$4 $\,\,$ \\
$\,\,$ 1/5$\,\,$ & $\,\,$ 1/9$\,\,$ & $\,\,$ 1 $\,\,$ & $\,\,$ 1/4 $\,\,$ \\
$\,\,$ 1/7$\,\,$ & $\,\,$ 1/4$\,\,$ & $\,\,$4$\,\,$ & $\,\,$ 1  $\,\,$ \\
\end{pmatrix},
\qquad
\lambda_{\max} =
4.2594,
\qquad
CR = 0.0978
\end{equation*}

\begin{equation*}
\mathbf{w}^{cos} =
\begin{pmatrix}
0.416374\\
\color{red} 0.415075\color{black} \\
0.051973\\
0.116577
\end{pmatrix}\end{equation*}
\begin{equation*}
\left[ \frac{{w}^{cos}_i}{{w}^{cos}_j} \right] =
\begin{pmatrix}
$\,\,$ 1 $\,\,$ & $\,\,$\color{red} 1.0031\color{black} $\,\,$ & $\,\,$8.0113$\,\,$ & $\,\,$3.5717$\,\,$ \\
$\,\,$\color{red} 0.9969\color{black} $\,\,$ & $\,\,$ 1 $\,\,$ & $\,\,$\color{red} 7.9863\color{black} $\,\,$ & $\,\,$\color{red} 3.5605\color{black}   $\,\,$ \\
$\,\,$0.1248$\,\,$ & $\,\,$\color{red} 0.1252\color{black} $\,\,$ & $\,\,$ 1 $\,\,$ & $\,\,$0.4458 $\,\,$ \\
$\,\,$0.2800$\,\,$ & $\,\,$\color{red} 0.2809\color{black} $\,\,$ & $\,\,$2.2430$\,\,$ & $\,\,$ 1  $\,\,$ \\
\end{pmatrix},
\end{equation*}

\begin{equation*}
\mathbf{w}^{\prime} =
\begin{pmatrix}
0.415834\\
0.415834\\
0.051906\\
0.116426
\end{pmatrix} =
0.998703\cdot
\begin{pmatrix}
0.416374\\
\color{gr} 0.416374\color{black} \\
0.051973\\
0.116577
\end{pmatrix},
\end{equation*}
\begin{equation*}
\left[ \frac{{w}^{\prime}_i}{{w}^{\prime}_j} \right] =
\begin{pmatrix}
$\,\,$ 1 $\,\,$ & $\,\,$\color{gr} \color{blue} 1\color{black} $\,\,$ & $\,\,$8.0113$\,\,$ & $\,\,$3.5717$\,\,$ \\
$\,\,$\color{gr} \color{blue} 1\color{black} $\,\,$ & $\,\,$ 1 $\,\,$ & $\,\,$\color{gr} 8.0113\color{black} $\,\,$ & $\,\,$\color{gr} 3.5717\color{black}   $\,\,$ \\
$\,\,$0.1248$\,\,$ & $\,\,$\color{gr} 0.1248\color{black} $\,\,$ & $\,\,$ 1 $\,\,$ & $\,\,$0.4458 $\,\,$ \\
$\,\,$0.2800$\,\,$ & $\,\,$\color{gr} 0.2800\color{black} $\,\,$ & $\,\,$2.2430$\,\,$ & $\,\,$ 1  $\,\,$ \\
\end{pmatrix},
\end{equation*}
\end{example}
\newpage
\begin{example}
\begin{equation*}
\mathbf{A} =
\begin{pmatrix}
$\,\,$ 1 $\,\,$ & $\,\,$1$\,\,$ & $\,\,$5$\,\,$ & $\,\,$8 $\,\,$ \\
$\,\,$ 1 $\,\,$ & $\,\,$ 1 $\,\,$ & $\,\,$7$\,\,$ & $\,\,$4 $\,\,$ \\
$\,\,$ 1/5$\,\,$ & $\,\,$ 1/7$\,\,$ & $\,\,$ 1 $\,\,$ & $\,\,$ 1/3 $\,\,$ \\
$\,\,$ 1/8$\,\,$ & $\,\,$ 1/4$\,\,$ & $\,\,$3$\,\,$ & $\,\,$ 1  $\,\,$ \\
\end{pmatrix},
\qquad
\lambda_{\max} =
4.2174,
\qquad
CR = 0.0820
\end{equation*}

\begin{equation*}
\mathbf{w}^{cos} =
\begin{pmatrix}
0.434102\\
\color{red} 0.399376\color{black} \\
0.059116\\
0.107407
\end{pmatrix}\end{equation*}
\begin{equation*}
\left[ \frac{{w}^{cos}_i}{{w}^{cos}_j} \right] =
\begin{pmatrix}
$\,\,$ 1 $\,\,$ & $\,\,$\color{red} 1.0870\color{black} $\,\,$ & $\,\,$7.3432$\,\,$ & $\,\,$4.0417$\,\,$ \\
$\,\,$\color{red} 0.9200\color{black} $\,\,$ & $\,\,$ 1 $\,\,$ & $\,\,$\color{red} 6.7558\color{black} $\,\,$ & $\,\,$\color{red} 3.7184\color{black}   $\,\,$ \\
$\,\,$0.1362$\,\,$ & $\,\,$\color{red} 0.1480\color{black} $\,\,$ & $\,\,$ 1 $\,\,$ & $\,\,$0.5504 $\,\,$ \\
$\,\,$0.2474$\,\,$ & $\,\,$\color{red} 0.2689\color{black} $\,\,$ & $\,\,$1.8169$\,\,$ & $\,\,$ 1  $\,\,$ \\
\end{pmatrix},
\end{equation*}

\begin{equation*}
\mathbf{w}^{\prime} =
\begin{pmatrix}
0.427923\\
0.407924\\
0.058275\\
0.105878
\end{pmatrix} =
0.985767\cdot
\begin{pmatrix}
0.434102\\
\color{gr} 0.413814\color{black} \\
0.059116\\
0.107407
\end{pmatrix},
\end{equation*}
\begin{equation*}
\left[ \frac{{w}^{\prime}_i}{{w}^{\prime}_j} \right] =
\begin{pmatrix}
$\,\,$ 1 $\,\,$ & $\,\,$\color{gr} 1.0490\color{black} $\,\,$ & $\,\,$7.3432$\,\,$ & $\,\,$4.0417$\,\,$ \\
$\,\,$\color{gr} 0.9533\color{black} $\,\,$ & $\,\,$ 1 $\,\,$ & $\,\,$\color{gr} \color{blue} 7\color{black} $\,\,$ & $\,\,$\color{gr} 3.8528\color{black}   $\,\,$ \\
$\,\,$0.1362$\,\,$ & $\,\,$\color{gr} \color{blue}  1/7\color{black} $\,\,$ & $\,\,$ 1 $\,\,$ & $\,\,$0.5504 $\,\,$ \\
$\,\,$0.2474$\,\,$ & $\,\,$\color{gr} 0.2596\color{black} $\,\,$ & $\,\,$1.8169$\,\,$ & $\,\,$ 1  $\,\,$ \\
\end{pmatrix},
\end{equation*}
\end{example}
\newpage
\begin{example}
\begin{equation*}
\mathbf{A} =
\begin{pmatrix}
$\,\,$ 1 $\,\,$ & $\,\,$1$\,\,$ & $\,\,$5$\,\,$ & $\,\,$8 $\,\,$ \\
$\,\,$ 1 $\,\,$ & $\,\,$ 1 $\,\,$ & $\,\,$7$\,\,$ & $\,\,$5 $\,\,$ \\
$\,\,$ 1/5$\,\,$ & $\,\,$ 1/7$\,\,$ & $\,\,$ 1 $\,\,$ & $\,\,$ 1/2 $\,\,$ \\
$\,\,$ 1/8$\,\,$ & $\,\,$ 1/5$\,\,$ & $\,\,$2$\,\,$ & $\,\,$ 1  $\,\,$ \\
\end{pmatrix},
\qquad
\lambda_{\max} =
4.1159,
\qquad
CR = 0.0437
\end{equation*}

\begin{equation*}
\mathbf{w}^{cos} =
\begin{pmatrix}
0.432729\\
\color{red} 0.418695\color{black} \\
0.062488\\
0.086088
\end{pmatrix}\end{equation*}
\begin{equation*}
\left[ \frac{{w}^{cos}_i}{{w}^{cos}_j} \right] =
\begin{pmatrix}
$\,\,$ 1 $\,\,$ & $\,\,$\color{red} 1.0335\color{black} $\,\,$ & $\,\,$6.9250$\,\,$ & $\,\,$5.0266$\,\,$ \\
$\,\,$\color{red} 0.9676\color{black} $\,\,$ & $\,\,$ 1 $\,\,$ & $\,\,$\color{red} 6.7004\color{black} $\,\,$ & $\,\,$\color{red} 4.8636\color{black}   $\,\,$ \\
$\,\,$0.1444$\,\,$ & $\,\,$\color{red} 0.1492\color{black} $\,\,$ & $\,\,$ 1 $\,\,$ & $\,\,$0.7259 $\,\,$ \\
$\,\,$0.1989$\,\,$ & $\,\,$\color{red} 0.2056\color{black} $\,\,$ & $\,\,$1.3777$\,\,$ & $\,\,$ 1  $\,\,$ \\
\end{pmatrix},
\end{equation*}

\begin{equation*}
\mathbf{w}^{\prime} =
\begin{pmatrix}
0.427705\\
0.425444\\
0.061763\\
0.085089
\end{pmatrix} =
0.988390\cdot
\begin{pmatrix}
0.432729\\
\color{gr} 0.430441\color{black} \\
0.062488\\
0.086088
\end{pmatrix},
\end{equation*}
\begin{equation*}
\left[ \frac{{w}^{\prime}_i}{{w}^{\prime}_j} \right] =
\begin{pmatrix}
$\,\,$ 1 $\,\,$ & $\,\,$\color{gr} 1.0053\color{black} $\,\,$ & $\,\,$6.9250$\,\,$ & $\,\,$5.0266$\,\,$ \\
$\,\,$\color{gr} 0.9947\color{black} $\,\,$ & $\,\,$ 1 $\,\,$ & $\,\,$\color{gr} 6.8884\color{black} $\,\,$ & $\,\,$\color{gr} \color{blue} 5\color{black}   $\,\,$ \\
$\,\,$0.1444$\,\,$ & $\,\,$\color{gr} 0.1452\color{black} $\,\,$ & $\,\,$ 1 $\,\,$ & $\,\,$0.7259 $\,\,$ \\
$\,\,$0.1989$\,\,$ & $\,\,$\color{gr} \color{blue}  1/5\color{black} $\,\,$ & $\,\,$1.3777$\,\,$ & $\,\,$ 1  $\,\,$ \\
\end{pmatrix},
\end{equation*}
\end{example}
\newpage
\begin{example}
\begin{equation*}
\mathbf{A} =
\begin{pmatrix}
$\,\,$ 1 $\,\,$ & $\,\,$1$\,\,$ & $\,\,$5$\,\,$ & $\,\,$8 $\,\,$ \\
$\,\,$ 1 $\,\,$ & $\,\,$ 1 $\,\,$ & $\,\,$8$\,\,$ & $\,\,$4 $\,\,$ \\
$\,\,$ 1/5$\,\,$ & $\,\,$ 1/8$\,\,$ & $\,\,$ 1 $\,\,$ & $\,\,$ 1/3 $\,\,$ \\
$\,\,$ 1/8$\,\,$ & $\,\,$ 1/4$\,\,$ & $\,\,$3$\,\,$ & $\,\,$ 1  $\,\,$ \\
\end{pmatrix},
\qquad
\lambda_{\max} =
4.2162,
\qquad
CR = 0.0815
\end{equation*}

\begin{equation*}
\mathbf{w}^{cos} =
\begin{pmatrix}
0.430766\\
\color{red} 0.408681\color{black} \\
0.056320\\
0.104233
\end{pmatrix}\end{equation*}
\begin{equation*}
\left[ \frac{{w}^{cos}_i}{{w}^{cos}_j} \right] =
\begin{pmatrix}
$\,\,$ 1 $\,\,$ & $\,\,$\color{red} 1.0540\color{black} $\,\,$ & $\,\,$7.6486$\,\,$ & $\,\,$4.1327$\,\,$ \\
$\,\,$\color{red} 0.9487\color{black} $\,\,$ & $\,\,$ 1 $\,\,$ & $\,\,$\color{red} 7.2565\color{black} $\,\,$ & $\,\,$\color{red} 3.9209\color{black}   $\,\,$ \\
$\,\,$0.1307$\,\,$ & $\,\,$\color{red} 0.1378\color{black} $\,\,$ & $\,\,$ 1 $\,\,$ & $\,\,$0.5403 $\,\,$ \\
$\,\,$0.2420$\,\,$ & $\,\,$\color{red} 0.2550\color{black} $\,\,$ & $\,\,$1.8507$\,\,$ & $\,\,$ 1  $\,\,$ \\
\end{pmatrix},
\end{equation*}

\begin{equation*}
\mathbf{w}^{\prime} =
\begin{pmatrix}
0.427242\\
0.413520\\
0.055859\\
0.103380
\end{pmatrix} =
0.991818\cdot
\begin{pmatrix}
0.430766\\
\color{gr} 0.416931\color{black} \\
0.056320\\
0.104233
\end{pmatrix},
\end{equation*}
\begin{equation*}
\left[ \frac{{w}^{\prime}_i}{{w}^{\prime}_j} \right] =
\begin{pmatrix}
$\,\,$ 1 $\,\,$ & $\,\,$\color{gr} 1.0332\color{black} $\,\,$ & $\,\,$7.6486$\,\,$ & $\,\,$4.1327$\,\,$ \\
$\,\,$\color{gr} 0.9679\color{black} $\,\,$ & $\,\,$ 1 $\,\,$ & $\,\,$\color{gr} 7.4030\color{black} $\,\,$ & $\,\,$\color{gr} \color{blue} 4\color{black}   $\,\,$ \\
$\,\,$0.1307$\,\,$ & $\,\,$\color{gr} 0.1351\color{black} $\,\,$ & $\,\,$ 1 $\,\,$ & $\,\,$0.5403 $\,\,$ \\
$\,\,$0.2420$\,\,$ & $\,\,$\color{gr} \color{blue}  1/4\color{black} $\,\,$ & $\,\,$1.8507$\,\,$ & $\,\,$ 1  $\,\,$ \\
\end{pmatrix},
\end{equation*}
\end{example}
\newpage
\begin{example}
\begin{equation*}
\mathbf{A} =
\begin{pmatrix}
$\,\,$ 1 $\,\,$ & $\,\,$1$\,\,$ & $\,\,$5$\,\,$ & $\,\,$8 $\,\,$ \\
$\,\,$ 1 $\,\,$ & $\,\,$ 1 $\,\,$ & $\,\,$8$\,\,$ & $\,\,$5 $\,\,$ \\
$\,\,$ 1/5$\,\,$ & $\,\,$ 1/8$\,\,$ & $\,\,$ 1 $\,\,$ & $\,\,$ 1/3 $\,\,$ \\
$\,\,$ 1/8$\,\,$ & $\,\,$ 1/5$\,\,$ & $\,\,$3$\,\,$ & $\,\,$ 1  $\,\,$ \\
\end{pmatrix},
\qquad
\lambda_{\max} =
4.2144,
\qquad
CR = 0.0808
\end{equation*}

\begin{equation*}
\mathbf{w}^{cos} =
\begin{pmatrix}
0.424068\\
\color{red} 0.421820\color{black} \\
0.056068\\
0.098044
\end{pmatrix}\end{equation*}
\begin{equation*}
\left[ \frac{{w}^{cos}_i}{{w}^{cos}_j} \right] =
\begin{pmatrix}
$\,\,$ 1 $\,\,$ & $\,\,$\color{red} 1.0053\color{black} $\,\,$ & $\,\,$7.5635$\,\,$ & $\,\,$4.3253$\,\,$ \\
$\,\,$\color{red} 0.9947\color{black} $\,\,$ & $\,\,$ 1 $\,\,$ & $\,\,$\color{red} 7.5234\color{black} $\,\,$ & $\,\,$\color{red} 4.3024\color{black}   $\,\,$ \\
$\,\,$0.1322$\,\,$ & $\,\,$\color{red} 0.1329\color{black} $\,\,$ & $\,\,$ 1 $\,\,$ & $\,\,$0.5719 $\,\,$ \\
$\,\,$0.2312$\,\,$ & $\,\,$\color{red} 0.2324\color{black} $\,\,$ & $\,\,$1.7487$\,\,$ & $\,\,$ 1  $\,\,$ \\
\end{pmatrix},
\end{equation*}

\begin{equation*}
\mathbf{w}^{\prime} =
\begin{pmatrix}
0.423117\\
0.423117\\
0.055942\\
0.097824
\end{pmatrix} =
0.997758\cdot
\begin{pmatrix}
0.424068\\
\color{gr} 0.424068\color{black} \\
0.056068\\
0.098044
\end{pmatrix},
\end{equation*}
\begin{equation*}
\left[ \frac{{w}^{\prime}_i}{{w}^{\prime}_j} \right] =
\begin{pmatrix}
$\,\,$ 1 $\,\,$ & $\,\,$\color{gr} \color{blue} 1\color{black} $\,\,$ & $\,\,$7.5635$\,\,$ & $\,\,$4.3253$\,\,$ \\
$\,\,$\color{gr} \color{blue} 1\color{black} $\,\,$ & $\,\,$ 1 $\,\,$ & $\,\,$\color{gr} 7.5635\color{black} $\,\,$ & $\,\,$\color{gr} 4.3253\color{black}   $\,\,$ \\
$\,\,$0.1322$\,\,$ & $\,\,$\color{gr} 0.1322\color{black} $\,\,$ & $\,\,$ 1 $\,\,$ & $\,\,$0.5719 $\,\,$ \\
$\,\,$0.2312$\,\,$ & $\,\,$\color{gr} 0.2312\color{black} $\,\,$ & $\,\,$1.7487$\,\,$ & $\,\,$ 1  $\,\,$ \\
\end{pmatrix},
\end{equation*}
\end{example}
\newpage
\begin{example}
\begin{equation*}
\mathbf{A} =
\begin{pmatrix}
$\,\,$ 1 $\,\,$ & $\,\,$1$\,\,$ & $\,\,$5$\,\,$ & $\,\,$9 $\,\,$ \\
$\,\,$ 1 $\,\,$ & $\,\,$ 1 $\,\,$ & $\,\,$7$\,\,$ & $\,\,$4 $\,\,$ \\
$\,\,$ 1/5$\,\,$ & $\,\,$ 1/7$\,\,$ & $\,\,$ 1 $\,\,$ & $\,\,$ 1/3 $\,\,$ \\
$\,\,$ 1/9$\,\,$ & $\,\,$ 1/4$\,\,$ & $\,\,$3$\,\,$ & $\,\,$ 1  $\,\,$ \\
\end{pmatrix},
\qquad
\lambda_{\max} =
4.2553,
\qquad
CR = 0.0963
\end{equation*}

\begin{equation*}
\mathbf{w}^{cos} =
\begin{pmatrix}
0.440142\\
\color{red} 0.395809\color{black} \\
0.058996\\
0.105053
\end{pmatrix}\end{equation*}
\begin{equation*}
\left[ \frac{{w}^{cos}_i}{{w}^{cos}_j} \right] =
\begin{pmatrix}
$\,\,$ 1 $\,\,$ & $\,\,$\color{red} 1.1120\color{black} $\,\,$ & $\,\,$7.4605$\,\,$ & $\,\,$4.1897$\,\,$ \\
$\,\,$\color{red} 0.8993\color{black} $\,\,$ & $\,\,$ 1 $\,\,$ & $\,\,$\color{red} 6.7091\color{black} $\,\,$ & $\,\,$\color{red} 3.7677\color{black}   $\,\,$ \\
$\,\,$0.1340$\,\,$ & $\,\,$\color{red} 0.1491\color{black} $\,\,$ & $\,\,$ 1 $\,\,$ & $\,\,$0.5616 $\,\,$ \\
$\,\,$0.2387$\,\,$ & $\,\,$\color{red} 0.2654\color{black} $\,\,$ & $\,\,$1.7807$\,\,$ & $\,\,$ 1  $\,\,$ \\
\end{pmatrix},
\end{equation*}

\begin{equation*}
\mathbf{w}^{\prime} =
\begin{pmatrix}
0.432715\\
0.406004\\
0.058001\\
0.103281
\end{pmatrix} =
0.983126\cdot
\begin{pmatrix}
0.440142\\
\color{gr} 0.412973\color{black} \\
0.058996\\
0.105053
\end{pmatrix},
\end{equation*}
\begin{equation*}
\left[ \frac{{w}^{\prime}_i}{{w}^{\prime}_j} \right] =
\begin{pmatrix}
$\,\,$ 1 $\,\,$ & $\,\,$\color{gr} 1.0658\color{black} $\,\,$ & $\,\,$7.4605$\,\,$ & $\,\,$4.1897$\,\,$ \\
$\,\,$\color{gr} 0.9383\color{black} $\,\,$ & $\,\,$ 1 $\,\,$ & $\,\,$\color{gr} \color{blue} 7\color{black} $\,\,$ & $\,\,$\color{gr} 3.9311\color{black}   $\,\,$ \\
$\,\,$0.1340$\,\,$ & $\,\,$\color{gr} \color{blue}  1/7\color{black} $\,\,$ & $\,\,$ 1 $\,\,$ & $\,\,$0.5616 $\,\,$ \\
$\,\,$0.2387$\,\,$ & $\,\,$\color{gr} 0.2544\color{black} $\,\,$ & $\,\,$1.7807$\,\,$ & $\,\,$ 1  $\,\,$ \\
\end{pmatrix},
\end{equation*}
\end{example}
\newpage
\begin{example}
\begin{equation*}
\mathbf{A} =
\begin{pmatrix}
$\,\,$ 1 $\,\,$ & $\,\,$1$\,\,$ & $\,\,$5$\,\,$ & $\,\,$9 $\,\,$ \\
$\,\,$ 1 $\,\,$ & $\,\,$ 1 $\,\,$ & $\,\,$7$\,\,$ & $\,\,$5 $\,\,$ \\
$\,\,$ 1/5$\,\,$ & $\,\,$ 1/7$\,\,$ & $\,\,$ 1 $\,\,$ & $\,\,$ 1/2 $\,\,$ \\
$\,\,$ 1/9$\,\,$ & $\,\,$ 1/5$\,\,$ & $\,\,$2$\,\,$ & $\,\,$ 1  $\,\,$ \\
\end{pmatrix},
\qquad
\lambda_{\max} =
4.1429,
\qquad
CR = 0.0539
\end{equation*}

\begin{equation*}
\mathbf{w}^{cos} =
\begin{pmatrix}
0.439565\\
\color{red} 0.414490\color{black} \\
0.062198\\
0.083747
\end{pmatrix}\end{equation*}
\begin{equation*}
\left[ \frac{{w}^{cos}_i}{{w}^{cos}_j} \right] =
\begin{pmatrix}
$\,\,$ 1 $\,\,$ & $\,\,$\color{red} 1.0605\color{black} $\,\,$ & $\,\,$7.0672$\,\,$ & $\,\,$5.2487$\,\,$ \\
$\,\,$\color{red} 0.9430\color{black} $\,\,$ & $\,\,$ 1 $\,\,$ & $\,\,$\color{red} 6.6640\color{black} $\,\,$ & $\,\,$\color{red} 4.9493\color{black}   $\,\,$ \\
$\,\,$0.1415$\,\,$ & $\,\,$\color{red} 0.1501\color{black} $\,\,$ & $\,\,$ 1 $\,\,$ & $\,\,$0.7427 $\,\,$ \\
$\,\,$0.1905$\,\,$ & $\,\,$\color{red} 0.2020\color{black} $\,\,$ & $\,\,$1.3464$\,\,$ & $\,\,$ 1  $\,\,$ \\
\end{pmatrix},
\end{equation*}

\begin{equation*}
\mathbf{w}^{\prime} =
\begin{pmatrix}
0.437707\\
0.416965\\
0.061935\\
0.083393
\end{pmatrix} =
0.995774\cdot
\begin{pmatrix}
0.439565\\
\color{gr} 0.418734\color{black} \\
0.062198\\
0.083747
\end{pmatrix},
\end{equation*}
\begin{equation*}
\left[ \frac{{w}^{\prime}_i}{{w}^{\prime}_j} \right] =
\begin{pmatrix}
$\,\,$ 1 $\,\,$ & $\,\,$\color{gr} 1.0497\color{black} $\,\,$ & $\,\,$7.0672$\,\,$ & $\,\,$5.2487$\,\,$ \\
$\,\,$\color{gr} 0.9526\color{black} $\,\,$ & $\,\,$ 1 $\,\,$ & $\,\,$\color{gr} 6.7322\color{black} $\,\,$ & $\,\,$\color{gr} \color{blue} 5\color{black}   $\,\,$ \\
$\,\,$0.1415$\,\,$ & $\,\,$\color{gr} 0.1485\color{black} $\,\,$ & $\,\,$ 1 $\,\,$ & $\,\,$0.7427 $\,\,$ \\
$\,\,$0.1905$\,\,$ & $\,\,$\color{gr} \color{blue}  1/5\color{black} $\,\,$ & $\,\,$1.3464$\,\,$ & $\,\,$ 1  $\,\,$ \\
\end{pmatrix},
\end{equation*}
\end{example}
\newpage
\begin{example}
\begin{equation*}
\mathbf{A} =
\begin{pmatrix}
$\,\,$ 1 $\,\,$ & $\,\,$1$\,\,$ & $\,\,$5$\,\,$ & $\,\,$9 $\,\,$ \\
$\,\,$ 1 $\,\,$ & $\,\,$ 1 $\,\,$ & $\,\,$7$\,\,$ & $\,\,$5 $\,\,$ \\
$\,\,$ 1/5$\,\,$ & $\,\,$ 1/7$\,\,$ & $\,\,$ 1 $\,\,$ & $\,\,$ 1/3 $\,\,$ \\
$\,\,$ 1/9$\,\,$ & $\,\,$ 1/5$\,\,$ & $\,\,$3$\,\,$ & $\,\,$ 1  $\,\,$ \\
\end{pmatrix},
\qquad
\lambda_{\max} =
4.2539,
\qquad
CR = 0.0957
\end{equation*}

\begin{equation*}
\mathbf{w}^{cos} =
\begin{pmatrix}
0.434107\\
\color{red} 0.408109\color{black} \\
0.058777\\
0.099007
\end{pmatrix}\end{equation*}
\begin{equation*}
\left[ \frac{{w}^{cos}_i}{{w}^{cos}_j} \right] =
\begin{pmatrix}
$\,\,$ 1 $\,\,$ & $\,\,$\color{red} 1.0637\color{black} $\,\,$ & $\,\,$7.3857$\,\,$ & $\,\,$4.3846$\,\,$ \\
$\,\,$\color{red} 0.9401\color{black} $\,\,$ & $\,\,$ 1 $\,\,$ & $\,\,$\color{red} 6.9434\color{black} $\,\,$ & $\,\,$\color{red} 4.1220\color{black}   $\,\,$ \\
$\,\,$0.1354$\,\,$ & $\,\,$\color{red} 0.1440\color{black} $\,\,$ & $\,\,$ 1 $\,\,$ & $\,\,$0.5937 $\,\,$ \\
$\,\,$0.2281$\,\,$ & $\,\,$\color{red} 0.2426\color{black} $\,\,$ & $\,\,$1.6844$\,\,$ & $\,\,$ 1  $\,\,$ \\
\end{pmatrix},
\end{equation*}

\begin{equation*}
\mathbf{w}^{\prime} =
\begin{pmatrix}
0.432667\\
0.410073\\
0.058582\\
0.098678
\end{pmatrix} =
0.996682\cdot
\begin{pmatrix}
0.434107\\
\color{gr} 0.411439\color{black} \\
0.058777\\
0.099007
\end{pmatrix},
\end{equation*}
\begin{equation*}
\left[ \frac{{w}^{\prime}_i}{{w}^{\prime}_j} \right] =
\begin{pmatrix}
$\,\,$ 1 $\,\,$ & $\,\,$\color{gr} 1.0551\color{black} $\,\,$ & $\,\,$7.3857$\,\,$ & $\,\,$4.3846$\,\,$ \\
$\,\,$\color{gr} 0.9478\color{black} $\,\,$ & $\,\,$ 1 $\,\,$ & $\,\,$\color{gr} \color{blue} 7\color{black} $\,\,$ & $\,\,$\color{gr} 4.1557\color{black}   $\,\,$ \\
$\,\,$0.1354$\,\,$ & $\,\,$\color{gr} \color{blue}  1/7\color{black} $\,\,$ & $\,\,$ 1 $\,\,$ & $\,\,$0.5937 $\,\,$ \\
$\,\,$0.2281$\,\,$ & $\,\,$\color{gr} 0.2406\color{black} $\,\,$ & $\,\,$1.6844$\,\,$ & $\,\,$ 1  $\,\,$ \\
\end{pmatrix},
\end{equation*}
\end{example}
\newpage
\begin{example}
\begin{equation*}
\mathbf{A} =
\begin{pmatrix}
$\,\,$ 1 $\,\,$ & $\,\,$1$\,\,$ & $\,\,$5$\,\,$ & $\,\,$9 $\,\,$ \\
$\,\,$ 1 $\,\,$ & $\,\,$ 1 $\,\,$ & $\,\,$8$\,\,$ & $\,\,$4 $\,\,$ \\
$\,\,$ 1/5$\,\,$ & $\,\,$ 1/8$\,\,$ & $\,\,$ 1 $\,\,$ & $\,\,$ 1/3 $\,\,$ \\
$\,\,$ 1/9$\,\,$ & $\,\,$ 1/4$\,\,$ & $\,\,$3$\,\,$ & $\,\,$ 1  $\,\,$ \\
\end{pmatrix},
\qquad
\lambda_{\max} =
4.2541,
\qquad
CR = 0.0958
\end{equation*}

\begin{equation*}
\mathbf{w}^{cos} =
\begin{pmatrix}
0.436830\\
\color{red} 0.405149\color{black} \\
0.056180\\
0.101841
\end{pmatrix}\end{equation*}
\begin{equation*}
\left[ \frac{{w}^{cos}_i}{{w}^{cos}_j} \right] =
\begin{pmatrix}
$\,\,$ 1 $\,\,$ & $\,\,$\color{red} 1.0782\color{black} $\,\,$ & $\,\,$7.7755$\,\,$ & $\,\,$4.2893$\,\,$ \\
$\,\,$\color{red} 0.9275\color{black} $\,\,$ & $\,\,$ 1 $\,\,$ & $\,\,$\color{red} 7.2116\color{black} $\,\,$ & $\,\,$\color{red} 3.9782\color{black}   $\,\,$ \\
$\,\,$0.1286$\,\,$ & $\,\,$\color{red} 0.1387\color{black} $\,\,$ & $\,\,$ 1 $\,\,$ & $\,\,$0.5516 $\,\,$ \\
$\,\,$0.2331$\,\,$ & $\,\,$\color{red} 0.2514\color{black} $\,\,$ & $\,\,$1.8128$\,\,$ & $\,\,$ 1  $\,\,$ \\
\end{pmatrix},
\end{equation*}

\begin{equation*}
\mathbf{w}^{\prime} =
\begin{pmatrix}
0.435864\\
0.406464\\
0.056056\\
0.101616
\end{pmatrix} =
0.997789\cdot
\begin{pmatrix}
0.436830\\
\color{gr} 0.407365\color{black} \\
0.056180\\
0.101841
\end{pmatrix},
\end{equation*}
\begin{equation*}
\left[ \frac{{w}^{\prime}_i}{{w}^{\prime}_j} \right] =
\begin{pmatrix}
$\,\,$ 1 $\,\,$ & $\,\,$\color{gr} 1.0723\color{black} $\,\,$ & $\,\,$7.7755$\,\,$ & $\,\,$4.2893$\,\,$ \\
$\,\,$\color{gr} 0.9325\color{black} $\,\,$ & $\,\,$ 1 $\,\,$ & $\,\,$\color{gr} 7.2510\color{black} $\,\,$ & $\,\,$\color{gr} \color{blue} 4\color{black}   $\,\,$ \\
$\,\,$0.1286$\,\,$ & $\,\,$\color{gr} 0.1379\color{black} $\,\,$ & $\,\,$ 1 $\,\,$ & $\,\,$0.5516 $\,\,$ \\
$\,\,$0.2331$\,\,$ & $\,\,$\color{gr} \color{blue}  1/4\color{black} $\,\,$ & $\,\,$1.8128$\,\,$ & $\,\,$ 1  $\,\,$ \\
\end{pmatrix},
\end{equation*}
\end{example}
\newpage
\begin{example}
\begin{equation*}
\mathbf{A} =
\begin{pmatrix}
$\,\,$ 1 $\,\,$ & $\,\,$1$\,\,$ & $\,\,$5$\,\,$ & $\,\,$9 $\,\,$ \\
$\,\,$ 1 $\,\,$ & $\,\,$ 1 $\,\,$ & $\,\,$8$\,\,$ & $\,\,$5 $\,\,$ \\
$\,\,$ 1/5$\,\,$ & $\,\,$ 1/8$\,\,$ & $\,\,$ 1 $\,\,$ & $\,\,$ 1/3 $\,\,$ \\
$\,\,$ 1/9$\,\,$ & $\,\,$ 1/5$\,\,$ & $\,\,$3$\,\,$ & $\,\,$ 1  $\,\,$ \\
\end{pmatrix},
\qquad
\lambda_{\max} =
4.2489,
\qquad
CR = 0.0939
\end{equation*}

\begin{equation*}
\mathbf{w}^{cos} =
\begin{pmatrix}
0.430762\\
\color{red} 0.417535\color{black} \\
0.055954\\
0.095750
\end{pmatrix}\end{equation*}
\begin{equation*}
\left[ \frac{{w}^{cos}_i}{{w}^{cos}_j} \right] =
\begin{pmatrix}
$\,\,$ 1 $\,\,$ & $\,\,$\color{red} 1.0317\color{black} $\,\,$ & $\,\,$7.6985$\,\,$ & $\,\,$4.4988$\,\,$ \\
$\,\,$\color{red} 0.9693\color{black} $\,\,$ & $\,\,$ 1 $\,\,$ & $\,\,$\color{red} 7.4621\color{black} $\,\,$ & $\,\,$\color{red} 4.3607\color{black}   $\,\,$ \\
$\,\,$0.1299$\,\,$ & $\,\,$\color{red} 0.1340\color{black} $\,\,$ & $\,\,$ 1 $\,\,$ & $\,\,$0.5844 $\,\,$ \\
$\,\,$0.2223$\,\,$ & $\,\,$\color{red} 0.2293\color{black} $\,\,$ & $\,\,$1.7112$\,\,$ & $\,\,$ 1  $\,\,$ \\
\end{pmatrix},
\end{equation*}

\begin{equation*}
\mathbf{w}^{\prime} =
\begin{pmatrix}
0.425139\\
0.425139\\
0.055223\\
0.094500
\end{pmatrix} =
0.986946\cdot
\begin{pmatrix}
0.430762\\
\color{gr} 0.430762\color{black} \\
0.055954\\
0.095750
\end{pmatrix},
\end{equation*}
\begin{equation*}
\left[ \frac{{w}^{\prime}_i}{{w}^{\prime}_j} \right] =
\begin{pmatrix}
$\,\,$ 1 $\,\,$ & $\,\,$\color{gr} \color{blue} 1\color{black} $\,\,$ & $\,\,$7.6985$\,\,$ & $\,\,$4.4988$\,\,$ \\
$\,\,$\color{gr} \color{blue} 1\color{black} $\,\,$ & $\,\,$ 1 $\,\,$ & $\,\,$\color{gr} 7.6985\color{black} $\,\,$ & $\,\,$\color{gr} 4.4988\color{black}   $\,\,$ \\
$\,\,$0.1299$\,\,$ & $\,\,$\color{gr} 0.1299\color{black} $\,\,$ & $\,\,$ 1 $\,\,$ & $\,\,$0.5844 $\,\,$ \\
$\,\,$0.2223$\,\,$ & $\,\,$\color{gr} 0.2223\color{black} $\,\,$ & $\,\,$1.7112$\,\,$ & $\,\,$ 1  $\,\,$ \\
\end{pmatrix},
\end{equation*}
\end{example}
\newpage
\begin{example}
\begin{equation*}
\mathbf{A} =
\begin{pmatrix}
$\,\,$ 1 $\,\,$ & $\,\,$1$\,\,$ & $\,\,$5$\,\,$ & $\,\,$9 $\,\,$ \\
$\,\,$ 1 $\,\,$ & $\,\,$ 1 $\,\,$ & $\,\,$9$\,\,$ & $\,\,$5 $\,\,$ \\
$\,\,$ 1/5$\,\,$ & $\,\,$ 1/9$\,\,$ & $\,\,$ 1 $\,\,$ & $\,\,$ 1/3 $\,\,$ \\
$\,\,$ 1/9$\,\,$ & $\,\,$ 1/5$\,\,$ & $\,\,$3$\,\,$ & $\,\,$ 1  $\,\,$ \\
\end{pmatrix},
\qquad
\lambda_{\max} =
4.2483,
\qquad
CR = 0.0936
\end{equation*}

\begin{equation*}
\mathbf{w}^{cos} =
\begin{pmatrix}
0.427935\\
\color{red} 0.425567\color{black} \\
0.053646\\
0.092853
\end{pmatrix}\end{equation*}
\begin{equation*}
\left[ \frac{{w}^{cos}_i}{{w}^{cos}_j} \right] =
\begin{pmatrix}
$\,\,$ 1 $\,\,$ & $\,\,$\color{red} 1.0056\color{black} $\,\,$ & $\,\,$7.9771$\,\,$ & $\,\,$4.6088$\,\,$ \\
$\,\,$\color{red} 0.9945\color{black} $\,\,$ & $\,\,$ 1 $\,\,$ & $\,\,$\color{red} 7.9329\color{black} $\,\,$ & $\,\,$\color{red} 4.5833\color{black}   $\,\,$ \\
$\,\,$0.1254$\,\,$ & $\,\,$\color{red} 0.1261\color{black} $\,\,$ & $\,\,$ 1 $\,\,$ & $\,\,$0.5778 $\,\,$ \\
$\,\,$0.2170$\,\,$ & $\,\,$\color{red} 0.2182\color{black} $\,\,$ & $\,\,$1.7308$\,\,$ & $\,\,$ 1  $\,\,$ \\
\end{pmatrix},
\end{equation*}

\begin{equation*}
\mathbf{w}^{\prime} =
\begin{pmatrix}
0.426924\\
0.426924\\
0.053519\\
0.092633
\end{pmatrix} =
0.997638\cdot
\begin{pmatrix}
0.427935\\
\color{gr} 0.427935\color{black} \\
0.053646\\
0.092853
\end{pmatrix},
\end{equation*}
\begin{equation*}
\left[ \frac{{w}^{\prime}_i}{{w}^{\prime}_j} \right] =
\begin{pmatrix}
$\,\,$ 1 $\,\,$ & $\,\,$\color{gr} \color{blue} 1\color{black} $\,\,$ & $\,\,$7.9771$\,\,$ & $\,\,$4.6088$\,\,$ \\
$\,\,$\color{gr} \color{blue} 1\color{black} $\,\,$ & $\,\,$ 1 $\,\,$ & $\,\,$\color{gr} 7.9771\color{black} $\,\,$ & $\,\,$\color{gr} 4.6088\color{black}   $\,\,$ \\
$\,\,$0.1254$\,\,$ & $\,\,$\color{gr} 0.1254\color{black} $\,\,$ & $\,\,$ 1 $\,\,$ & $\,\,$0.5778 $\,\,$ \\
$\,\,$0.2170$\,\,$ & $\,\,$\color{gr} 0.2170\color{black} $\,\,$ & $\,\,$1.7308$\,\,$ & $\,\,$ 1  $\,\,$ \\
\end{pmatrix},
\end{equation*}
\end{example}
\newpage
\begin{example}
\begin{equation*}
\mathbf{A} =
\begin{pmatrix}
$\,\,$ 1 $\,\,$ & $\,\,$1$\,\,$ & $\,\,$6$\,\,$ & $\,\,$3 $\,\,$ \\
$\,\,$ 1 $\,\,$ & $\,\,$ 1 $\,\,$ & $\,\,$8$\,\,$ & $\,\,$2 $\,\,$ \\
$\,\,$ 1/6$\,\,$ & $\,\,$ 1/8$\,\,$ & $\,\,$ 1 $\,\,$ & $\,\,$ 1/6 $\,\,$ \\
$\,\,$ 1/3$\,\,$ & $\,\,$ 1/2$\,\,$ & $\,\,$6$\,\,$ & $\,\,$ 1  $\,\,$ \\
\end{pmatrix},
\qquad
\lambda_{\max} =
4.1031,
\qquad
CR = 0.0389
\end{equation*}

\begin{equation*}
\mathbf{w}^{cos} =
\begin{pmatrix}
0.386094\\
\color{red} 0.372151\color{black} \\
0.047424\\
0.194331
\end{pmatrix}\end{equation*}
\begin{equation*}
\left[ \frac{{w}^{cos}_i}{{w}^{cos}_j} \right] =
\begin{pmatrix}
$\,\,$ 1 $\,\,$ & $\,\,$\color{red} 1.0375\color{black} $\,\,$ & $\,\,$8.1413$\,\,$ & $\,\,$1.9868$\,\,$ \\
$\,\,$\color{red} 0.9639\color{black} $\,\,$ & $\,\,$ 1 $\,\,$ & $\,\,$\color{red} 7.8473\color{black} $\,\,$ & $\,\,$\color{red} 1.9150\color{black}   $\,\,$ \\
$\,\,$0.1228$\,\,$ & $\,\,$\color{red} 0.1274\color{black} $\,\,$ & $\,\,$ 1 $\,\,$ & $\,\,$0.2440 $\,\,$ \\
$\,\,$0.5033$\,\,$ & $\,\,$\color{red} 0.5222\color{black} $\,\,$ & $\,\,$4.0977$\,\,$ & $\,\,$ 1  $\,\,$ \\
\end{pmatrix},
\end{equation*}

\begin{equation*}
\mathbf{w}^{\prime} =
\begin{pmatrix}
0.383318\\
0.376665\\
0.047083\\
0.192934
\end{pmatrix} =
0.992810\cdot
\begin{pmatrix}
0.386094\\
\color{gr} 0.379393\color{black} \\
0.047424\\
0.194331
\end{pmatrix},
\end{equation*}
\begin{equation*}
\left[ \frac{{w}^{\prime}_i}{{w}^{\prime}_j} \right] =
\begin{pmatrix}
$\,\,$ 1 $\,\,$ & $\,\,$\color{gr} 1.0177\color{black} $\,\,$ & $\,\,$8.1413$\,\,$ & $\,\,$1.9868$\,\,$ \\
$\,\,$\color{gr} 0.9826\color{black} $\,\,$ & $\,\,$ 1 $\,\,$ & $\,\,$\color{gr} \color{blue} 8\color{black} $\,\,$ & $\,\,$\color{gr} 1.9523\color{black}   $\,\,$ \\
$\,\,$0.1228$\,\,$ & $\,\,$\color{gr} \color{blue}  1/8\color{black} $\,\,$ & $\,\,$ 1 $\,\,$ & $\,\,$0.2440 $\,\,$ \\
$\,\,$0.5033$\,\,$ & $\,\,$\color{gr} 0.5122\color{black} $\,\,$ & $\,\,$4.0977$\,\,$ & $\,\,$ 1  $\,\,$ \\
\end{pmatrix},
\end{equation*}
\end{example}
\newpage
\begin{example}
\begin{equation*}
\mathbf{A} =
\begin{pmatrix}
$\,\,$ 1 $\,\,$ & $\,\,$1$\,\,$ & $\,\,$6$\,\,$ & $\,\,$3 $\,\,$ \\
$\,\,$ 1 $\,\,$ & $\,\,$ 1 $\,\,$ & $\,\,$9$\,\,$ & $\,\,$2 $\,\,$ \\
$\,\,$ 1/6$\,\,$ & $\,\,$ 1/9$\,\,$ & $\,\,$ 1 $\,\,$ & $\,\,$ 1/6 $\,\,$ \\
$\,\,$ 1/3$\,\,$ & $\,\,$ 1/2$\,\,$ & $\,\,$6$\,\,$ & $\,\,$ 1  $\,\,$ \\
\end{pmatrix},
\qquad
\lambda_{\max} =
4.1031,
\qquad
CR = 0.0389
\end{equation*}

\begin{equation*}
\mathbf{w}^{cos} =
\begin{pmatrix}
0.383571\\
\color{red} 0.379893\color{black} \\
0.045583\\
0.190953
\end{pmatrix}\end{equation*}
\begin{equation*}
\left[ \frac{{w}^{cos}_i}{{w}^{cos}_j} \right] =
\begin{pmatrix}
$\,\,$ 1 $\,\,$ & $\,\,$\color{red} 1.0097\color{black} $\,\,$ & $\,\,$8.4147$\,\,$ & $\,\,$2.0087$\,\,$ \\
$\,\,$\color{red} 0.9904\color{black} $\,\,$ & $\,\,$ 1 $\,\,$ & $\,\,$\color{red} 8.3341\color{black} $\,\,$ & $\,\,$\color{red} 1.9895\color{black}   $\,\,$ \\
$\,\,$0.1188$\,\,$ & $\,\,$\color{red} 0.1200\color{black} $\,\,$ & $\,\,$ 1 $\,\,$ & $\,\,$0.2387 $\,\,$ \\
$\,\,$0.4978$\,\,$ & $\,\,$\color{red} 0.5027\color{black} $\,\,$ & $\,\,$4.1891$\,\,$ & $\,\,$ 1  $\,\,$ \\
\end{pmatrix},
\end{equation*}

\begin{equation*}
\mathbf{w}^{\prime} =
\begin{pmatrix}
0.382800\\
0.381139\\
0.045492\\
0.190570
\end{pmatrix} =
0.997990\cdot
\begin{pmatrix}
0.383571\\
\color{gr} 0.381907\color{black} \\
0.045583\\
0.190953
\end{pmatrix},
\end{equation*}
\begin{equation*}
\left[ \frac{{w}^{\prime}_i}{{w}^{\prime}_j} \right] =
\begin{pmatrix}
$\,\,$ 1 $\,\,$ & $\,\,$\color{gr} 1.0044\color{black} $\,\,$ & $\,\,$8.4147$\,\,$ & $\,\,$2.0087$\,\,$ \\
$\,\,$\color{gr} 0.9957\color{black} $\,\,$ & $\,\,$ 1 $\,\,$ & $\,\,$\color{gr} 8.3782\color{black} $\,\,$ & $\,\,$\color{gr} \color{blue} 2\color{black}   $\,\,$ \\
$\,\,$0.1188$\,\,$ & $\,\,$\color{gr} 0.1194\color{black} $\,\,$ & $\,\,$ 1 $\,\,$ & $\,\,$0.2387 $\,\,$ \\
$\,\,$0.4978$\,\,$ & $\,\,$\color{gr} \color{blue}  1/2\color{black} $\,\,$ & $\,\,$4.1891$\,\,$ & $\,\,$ 1  $\,\,$ \\
\end{pmatrix},
\end{equation*}
\end{example}
\newpage
\begin{example}
\begin{equation*}
\mathbf{A} =
\begin{pmatrix}
$\,\,$ 1 $\,\,$ & $\,\,$1$\,\,$ & $\,\,$6$\,\,$ & $\,\,$3 $\,\,$ \\
$\,\,$ 1 $\,\,$ & $\,\,$ 1 $\,\,$ & $\,\,$9$\,\,$ & $\,\,$2 $\,\,$ \\
$\,\,$ 1/6$\,\,$ & $\,\,$ 1/9$\,\,$ & $\,\,$ 1 $\,\,$ & $\,\,$ 1/7 $\,\,$ \\
$\,\,$ 1/3$\,\,$ & $\,\,$ 1/2$\,\,$ & $\,\,$7$\,\,$ & $\,\,$ 1  $\,\,$ \\
\end{pmatrix},
\qquad
\lambda_{\max} =
4.1342,
\qquad
CR = 0.0506
\end{equation*}

\begin{equation*}
\mathbf{w}^{cos} =
\begin{pmatrix}
0.380746\\
\color{red} 0.375660\color{black} \\
0.044186\\
0.199408
\end{pmatrix}\end{equation*}
\begin{equation*}
\left[ \frac{{w}^{cos}_i}{{w}^{cos}_j} \right] =
\begin{pmatrix}
$\,\,$ 1 $\,\,$ & $\,\,$\color{red} 1.0135\color{black} $\,\,$ & $\,\,$8.6170$\,\,$ & $\,\,$1.9094$\,\,$ \\
$\,\,$\color{red} 0.9866\color{black} $\,\,$ & $\,\,$ 1 $\,\,$ & $\,\,$\color{red} 8.5019\color{black} $\,\,$ & $\,\,$\color{red} 1.8839\color{black}   $\,\,$ \\
$\,\,$0.1161$\,\,$ & $\,\,$\color{red} 0.1176\color{black} $\,\,$ & $\,\,$ 1 $\,\,$ & $\,\,$0.2216 $\,\,$ \\
$\,\,$0.5237$\,\,$ & $\,\,$\color{red} 0.5308\color{black} $\,\,$ & $\,\,$4.5130$\,\,$ & $\,\,$ 1  $\,\,$ \\
\end{pmatrix},
\end{equation*}

\begin{equation*}
\mathbf{w}^{\prime} =
\begin{pmatrix}
0.378819\\
0.378819\\
0.043962\\
0.198399
\end{pmatrix} =
0.994940\cdot
\begin{pmatrix}
0.380746\\
\color{gr} 0.380746\color{black} \\
0.044186\\
0.199408
\end{pmatrix},
\end{equation*}
\begin{equation*}
\left[ \frac{{w}^{\prime}_i}{{w}^{\prime}_j} \right] =
\begin{pmatrix}
$\,\,$ 1 $\,\,$ & $\,\,$\color{gr} \color{blue} 1\color{black} $\,\,$ & $\,\,$8.6170$\,\,$ & $\,\,$1.9094$\,\,$ \\
$\,\,$\color{gr} \color{blue} 1\color{black} $\,\,$ & $\,\,$ 1 $\,\,$ & $\,\,$\color{gr} 8.6170\color{black} $\,\,$ & $\,\,$\color{gr} 1.9094\color{black}   $\,\,$ \\
$\,\,$0.1161$\,\,$ & $\,\,$\color{gr} 0.1161\color{black} $\,\,$ & $\,\,$ 1 $\,\,$ & $\,\,$0.2216 $\,\,$ \\
$\,\,$0.5237$\,\,$ & $\,\,$\color{gr} 0.5237\color{black} $\,\,$ & $\,\,$4.5130$\,\,$ & $\,\,$ 1  $\,\,$ \\
\end{pmatrix},
\end{equation*}
\end{example}
\newpage
\begin{example}
\begin{equation*}
\mathbf{A} =
\begin{pmatrix}
$\,\,$ 1 $\,\,$ & $\,\,$1$\,\,$ & $\,\,$6$\,\,$ & $\,\,$3 $\,\,$ \\
$\,\,$ 1 $\,\,$ & $\,\,$ 1 $\,\,$ & $\,\,$9$\,\,$ & $\,\,$2 $\,\,$ \\
$\,\,$ 1/6$\,\,$ & $\,\,$ 1/9$\,\,$ & $\,\,$ 1 $\,\,$ & $\,\,$ 1/8 $\,\,$ \\
$\,\,$ 1/3$\,\,$ & $\,\,$ 1/2$\,\,$ & $\,\,$8$\,\,$ & $\,\,$ 1  $\,\,$ \\
\end{pmatrix},
\qquad
\lambda_{\max} =
4.1664,
\qquad
CR = 0.0627
\end{equation*}

\begin{equation*}
\mathbf{w}^{cos} =
\begin{pmatrix}
0.378210\\
\color{red} 0.371689\color{black} \\
0.043051\\
0.207050
\end{pmatrix}\end{equation*}
\begin{equation*}
\left[ \frac{{w}^{cos}_i}{{w}^{cos}_j} \right] =
\begin{pmatrix}
$\,\,$ 1 $\,\,$ & $\,\,$\color{red} 1.0175\color{black} $\,\,$ & $\,\,$8.7853$\,\,$ & $\,\,$1.8267$\,\,$ \\
$\,\,$\color{red} 0.9828\color{black} $\,\,$ & $\,\,$ 1 $\,\,$ & $\,\,$\color{red} 8.6338\color{black} $\,\,$ & $\,\,$\color{red} 1.7952\color{black}   $\,\,$ \\
$\,\,$0.1138$\,\,$ & $\,\,$\color{red} 0.1158\color{black} $\,\,$ & $\,\,$ 1 $\,\,$ & $\,\,$0.2079 $\,\,$ \\
$\,\,$0.5474$\,\,$ & $\,\,$\color{red} 0.5571\color{black} $\,\,$ & $\,\,$4.8095$\,\,$ & $\,\,$ 1  $\,\,$ \\
\end{pmatrix},
\end{equation*}

\begin{equation*}
\mathbf{w}^{\prime} =
\begin{pmatrix}
0.375760\\
0.375760\\
0.042772\\
0.205708
\end{pmatrix} =
0.993521\cdot
\begin{pmatrix}
0.378210\\
\color{gr} 0.378210\color{black} \\
0.043051\\
0.207050
\end{pmatrix},
\end{equation*}
\begin{equation*}
\left[ \frac{{w}^{\prime}_i}{{w}^{\prime}_j} \right] =
\begin{pmatrix}
$\,\,$ 1 $\,\,$ & $\,\,$\color{gr} \color{blue} 1\color{black} $\,\,$ & $\,\,$8.7853$\,\,$ & $\,\,$1.8267$\,\,$ \\
$\,\,$\color{gr} \color{blue} 1\color{black} $\,\,$ & $\,\,$ 1 $\,\,$ & $\,\,$\color{gr} 8.7853\color{black} $\,\,$ & $\,\,$\color{gr} 1.8267\color{black}   $\,\,$ \\
$\,\,$0.1138$\,\,$ & $\,\,$\color{gr} 0.1138\color{black} $\,\,$ & $\,\,$ 1 $\,\,$ & $\,\,$0.2079 $\,\,$ \\
$\,\,$0.5474$\,\,$ & $\,\,$\color{gr} 0.5474\color{black} $\,\,$ & $\,\,$4.8095$\,\,$ & $\,\,$ 1  $\,\,$ \\
\end{pmatrix},
\end{equation*}
\end{example}
\newpage
\begin{example}
\begin{equation*}
\mathbf{A} =
\begin{pmatrix}
$\,\,$ 1 $\,\,$ & $\,\,$1$\,\,$ & $\,\,$6$\,\,$ & $\,\,$3 $\,\,$ \\
$\,\,$ 1 $\,\,$ & $\,\,$ 1 $\,\,$ & $\,\,$9$\,\,$ & $\,\,$2 $\,\,$ \\
$\,\,$ 1/6$\,\,$ & $\,\,$ 1/9$\,\,$ & $\,\,$ 1 $\,\,$ & $\,\,$ 1/9 $\,\,$ \\
$\,\,$ 1/3$\,\,$ & $\,\,$ 1/2$\,\,$ & $\,\,$9$\,\,$ & $\,\,$ 1  $\,\,$ \\
\end{pmatrix},
\qquad
\lambda_{\max} =
4.1990,
\qquad
CR = 0.0750
\end{equation*}

\begin{equation*}
\mathbf{w}^{cos} =
\begin{pmatrix}
0.375958\\
\color{red} 0.368005\color{black} \\
0.042102\\
0.213934
\end{pmatrix}\end{equation*}
\begin{equation*}
\left[ \frac{{w}^{cos}_i}{{w}^{cos}_j} \right] =
\begin{pmatrix}
$\,\,$ 1 $\,\,$ & $\,\,$\color{red} 1.0216\color{black} $\,\,$ & $\,\,$8.9298$\,\,$ & $\,\,$1.7574$\,\,$ \\
$\,\,$\color{red} 0.9788\color{black} $\,\,$ & $\,\,$ 1 $\,\,$ & $\,\,$\color{red} 8.7409\color{black} $\,\,$ & $\,\,$\color{red} 1.7202\color{black}   $\,\,$ \\
$\,\,$0.1120$\,\,$ & $\,\,$\color{red} 0.1144\color{black} $\,\,$ & $\,\,$ 1 $\,\,$ & $\,\,$0.1968 $\,\,$ \\
$\,\,$0.5690$\,\,$ & $\,\,$\color{red} 0.5813\color{black} $\,\,$ & $\,\,$5.0814$\,\,$ & $\,\,$ 1  $\,\,$ \\
\end{pmatrix},
\end{equation*}

\begin{equation*}
\mathbf{w}^{\prime} =
\begin{pmatrix}
0.372992\\
0.372992\\
0.041770\\
0.212246
\end{pmatrix} =
0.992110\cdot
\begin{pmatrix}
0.375958\\
\color{gr} 0.375958\color{black} \\
0.042102\\
0.213934
\end{pmatrix},
\end{equation*}
\begin{equation*}
\left[ \frac{{w}^{\prime}_i}{{w}^{\prime}_j} \right] =
\begin{pmatrix}
$\,\,$ 1 $\,\,$ & $\,\,$\color{gr} \color{blue} 1\color{black} $\,\,$ & $\,\,$8.9298$\,\,$ & $\,\,$1.7574$\,\,$ \\
$\,\,$\color{gr} \color{blue} 1\color{black} $\,\,$ & $\,\,$ 1 $\,\,$ & $\,\,$\color{gr} 8.9298\color{black} $\,\,$ & $\,\,$\color{gr} 1.7574\color{black}   $\,\,$ \\
$\,\,$0.1120$\,\,$ & $\,\,$\color{gr} 0.1120\color{black} $\,\,$ & $\,\,$ 1 $\,\,$ & $\,\,$0.1968 $\,\,$ \\
$\,\,$0.5690$\,\,$ & $\,\,$\color{gr} 0.5690\color{black} $\,\,$ & $\,\,$5.0814$\,\,$ & $\,\,$ 1  $\,\,$ \\
\end{pmatrix},
\end{equation*}
\end{example}
\newpage
\begin{example}
\begin{equation*}
\mathbf{A} =
\begin{pmatrix}
$\,\,$ 1 $\,\,$ & $\,\,$1$\,\,$ & $\,\,$6$\,\,$ & $\,\,$4 $\,\,$ \\
$\,\,$ 1 $\,\,$ & $\,\,$ 1 $\,\,$ & $\,\,$5$\,\,$ & $\,\,$2 $\,\,$ \\
$\,\,$ 1/6$\,\,$ & $\,\,$ 1/5$\,\,$ & $\,\,$ 1 $\,\,$ & $\,\,$ 1/2 $\,\,$ \\
$\,\,$ 1/4$\,\,$ & $\,\,$ 1/2$\,\,$ & $\,\,$2$\,\,$ & $\,\,$ 1  $\,\,$ \\
\end{pmatrix},
\qquad
\lambda_{\max} =
4.0407,
\qquad
CR = 0.0153
\end{equation*}

\begin{equation*}
\mathbf{w}^{cos} =
\begin{pmatrix}
0.434728\\
0.352895\\
\color{red} 0.070379\color{black} \\
0.141998
\end{pmatrix}\end{equation*}
\begin{equation*}
\left[ \frac{{w}^{cos}_i}{{w}^{cos}_j} \right] =
\begin{pmatrix}
$\,\,$ 1 $\,\,$ & $\,\,$1.2319$\,\,$ & $\,\,$\color{red} 6.1770\color{black} $\,\,$ & $\,\,$3.0615$\,\,$ \\
$\,\,$0.8118$\,\,$ & $\,\,$ 1 $\,\,$ & $\,\,$\color{red} 5.0142\color{black} $\,\,$ & $\,\,$2.4852  $\,\,$ \\
$\,\,$\color{red} 0.1619\color{black} $\,\,$ & $\,\,$\color{red} 0.1994\color{black} $\,\,$ & $\,\,$ 1 $\,\,$ & $\,\,$\color{red} 0.4956\color{black}  $\,\,$ \\
$\,\,$0.3266$\,\,$ & $\,\,$0.4024$\,\,$ & $\,\,$\color{red} 2.0176\color{black} $\,\,$ & $\,\,$ 1  $\,\,$ \\
\end{pmatrix},
\end{equation*}

\begin{equation*}
\mathbf{w}^{\prime} =
\begin{pmatrix}
0.434641\\
0.352825\\
0.070565\\
0.141970
\end{pmatrix} =
0.999800\cdot
\begin{pmatrix}
0.434728\\
0.352895\\
\color{gr} 0.070579\color{black} \\
0.141998
\end{pmatrix},
\end{equation*}
\begin{equation*}
\left[ \frac{{w}^{\prime}_i}{{w}^{\prime}_j} \right] =
\begin{pmatrix}
$\,\,$ 1 $\,\,$ & $\,\,$1.2319$\,\,$ & $\,\,$\color{gr} 6.1594\color{black} $\,\,$ & $\,\,$3.0615$\,\,$ \\
$\,\,$0.8118$\,\,$ & $\,\,$ 1 $\,\,$ & $\,\,$\color{gr} \color{blue} 5\color{black} $\,\,$ & $\,\,$2.4852  $\,\,$ \\
$\,\,$\color{gr} 0.1624\color{black} $\,\,$ & $\,\,$\color{gr} \color{blue}  1/5\color{black} $\,\,$ & $\,\,$ 1 $\,\,$ & $\,\,$\color{gr} 0.4970\color{black}  $\,\,$ \\
$\,\,$0.3266$\,\,$ & $\,\,$0.4024$\,\,$ & $\,\,$\color{gr} 2.0119\color{black} $\,\,$ & $\,\,$ 1  $\,\,$ \\
\end{pmatrix},
\end{equation*}
\end{example}
\newpage
\begin{example}
\begin{equation*}
\mathbf{A} =
\begin{pmatrix}
$\,\,$ 1 $\,\,$ & $\,\,$1$\,\,$ & $\,\,$6$\,\,$ & $\,\,$4 $\,\,$ \\
$\,\,$ 1 $\,\,$ & $\,\,$ 1 $\,\,$ & $\,\,$8$\,\,$ & $\,\,$2 $\,\,$ \\
$\,\,$ 1/6$\,\,$ & $\,\,$ 1/8$\,\,$ & $\,\,$ 1 $\,\,$ & $\,\,$ 1/7 $\,\,$ \\
$\,\,$ 1/4$\,\,$ & $\,\,$ 1/2$\,\,$ & $\,\,$7$\,\,$ & $\,\,$ 1  $\,\,$ \\
\end{pmatrix},
\qquad
\lambda_{\max} =
4.2109,
\qquad
CR = 0.0795
\end{equation*}

\begin{equation*}
\mathbf{w}^{cos} =
\begin{pmatrix}
0.401549\\
\color{red} 0.361234\color{black} \\
0.045959\\
0.191257
\end{pmatrix}\end{equation*}
\begin{equation*}
\left[ \frac{{w}^{cos}_i}{{w}^{cos}_j} \right] =
\begin{pmatrix}
$\,\,$ 1 $\,\,$ & $\,\,$\color{red} 1.1116\color{black} $\,\,$ & $\,\,$8.7371$\,\,$ & $\,\,$2.0995$\,\,$ \\
$\,\,$\color{red} 0.8996\color{black} $\,\,$ & $\,\,$ 1 $\,\,$ & $\,\,$\color{red} 7.8599\color{black} $\,\,$ & $\,\,$\color{red} 1.8887\color{black}   $\,\,$ \\
$\,\,$0.1145$\,\,$ & $\,\,$\color{red} 0.1272\color{black} $\,\,$ & $\,\,$ 1 $\,\,$ & $\,\,$0.2403 $\,\,$ \\
$\,\,$0.4763$\,\,$ & $\,\,$\color{red} 0.5295\color{black} $\,\,$ & $\,\,$4.1614$\,\,$ & $\,\,$ 1  $\,\,$ \\
\end{pmatrix},
\end{equation*}

\begin{equation*}
\mathbf{w}^{\prime} =
\begin{pmatrix}
0.398980\\
0.365322\\
0.045665\\
0.190033
\end{pmatrix} =
0.993600\cdot
\begin{pmatrix}
0.401549\\
\color{gr} 0.367675\color{black} \\
0.045959\\
0.191257
\end{pmatrix},
\end{equation*}
\begin{equation*}
\left[ \frac{{w}^{\prime}_i}{{w}^{\prime}_j} \right] =
\begin{pmatrix}
$\,\,$ 1 $\,\,$ & $\,\,$\color{gr} 1.0921\color{black} $\,\,$ & $\,\,$8.7371$\,\,$ & $\,\,$2.0995$\,\,$ \\
$\,\,$\color{gr} 0.9156\color{black} $\,\,$ & $\,\,$ 1 $\,\,$ & $\,\,$\color{gr} \color{blue} 8\color{black} $\,\,$ & $\,\,$\color{gr} 1.9224\color{black}   $\,\,$ \\
$\,\,$0.1145$\,\,$ & $\,\,$\color{gr} \color{blue}  1/8\color{black} $\,\,$ & $\,\,$ 1 $\,\,$ & $\,\,$0.2403 $\,\,$ \\
$\,\,$0.4763$\,\,$ & $\,\,$\color{gr} 0.5202\color{black} $\,\,$ & $\,\,$4.1614$\,\,$ & $\,\,$ 1  $\,\,$ \\
\end{pmatrix},
\end{equation*}
\end{example}
\newpage
\begin{example}
\begin{equation*}
\mathbf{A} =
\begin{pmatrix}
$\,\,$ 1 $\,\,$ & $\,\,$1$\,\,$ & $\,\,$6$\,\,$ & $\,\,$4 $\,\,$ \\
$\,\,$ 1 $\,\,$ & $\,\,$ 1 $\,\,$ & $\,\,$8$\,\,$ & $\,\,$2 $\,\,$ \\
$\,\,$ 1/6$\,\,$ & $\,\,$ 1/8$\,\,$ & $\,\,$ 1 $\,\,$ & $\,\,$ 1/8 $\,\,$ \\
$\,\,$ 1/4$\,\,$ & $\,\,$ 1/2$\,\,$ & $\,\,$8$\,\,$ & $\,\,$ 1  $\,\,$ \\
\end{pmatrix},
\qquad
\lambda_{\max} =
4.2512,
\qquad
CR = 0.0947
\end{equation*}

\begin{equation*}
\mathbf{w}^{cos} =
\begin{pmatrix}
0.398754\\
\color{red} 0.357250\color{black} \\
0.044886\\
0.199110
\end{pmatrix}\end{equation*}
\begin{equation*}
\left[ \frac{{w}^{cos}_i}{{w}^{cos}_j} \right] =
\begin{pmatrix}
$\,\,$ 1 $\,\,$ & $\,\,$\color{red} 1.1162\color{black} $\,\,$ & $\,\,$8.8837$\,\,$ & $\,\,$2.0027$\,\,$ \\
$\,\,$\color{red} 0.8959\color{black} $\,\,$ & $\,\,$ 1 $\,\,$ & $\,\,$\color{red} 7.9591\color{black} $\,\,$ & $\,\,$\color{red} 1.7942\color{black}   $\,\,$ \\
$\,\,$0.1126$\,\,$ & $\,\,$\color{red} 0.1256\color{black} $\,\,$ & $\,\,$ 1 $\,\,$ & $\,\,$0.2254 $\,\,$ \\
$\,\,$0.4993$\,\,$ & $\,\,$\color{red} 0.5573\color{black} $\,\,$ & $\,\,$4.4359$\,\,$ & $\,\,$ 1  $\,\,$ \\
\end{pmatrix},
\end{equation*}

\begin{equation*}
\mathbf{w}^{\prime} =
\begin{pmatrix}
0.398023\\
0.358429\\
0.044804\\
0.198745
\end{pmatrix} =
0.998166\cdot
\begin{pmatrix}
0.398754\\
\color{gr} 0.359087\color{black} \\
0.044886\\
0.199110
\end{pmatrix},
\end{equation*}
\begin{equation*}
\left[ \frac{{w}^{\prime}_i}{{w}^{\prime}_j} \right] =
\begin{pmatrix}
$\,\,$ 1 $\,\,$ & $\,\,$\color{gr} 1.1105\color{black} $\,\,$ & $\,\,$8.8837$\,\,$ & $\,\,$2.0027$\,\,$ \\
$\,\,$\color{gr} 0.9005\color{black} $\,\,$ & $\,\,$ 1 $\,\,$ & $\,\,$\color{gr} \color{blue} 8\color{black} $\,\,$ & $\,\,$\color{gr} 1.8035\color{black}   $\,\,$ \\
$\,\,$0.1126$\,\,$ & $\,\,$\color{gr} \color{blue}  1/8\color{black} $\,\,$ & $\,\,$ 1 $\,\,$ & $\,\,$0.2254 $\,\,$ \\
$\,\,$0.4993$\,\,$ & $\,\,$\color{gr} 0.5545\color{black} $\,\,$ & $\,\,$4.4359$\,\,$ & $\,\,$ 1  $\,\,$ \\
\end{pmatrix},
\end{equation*}
\end{example}
\newpage
\begin{example}
\begin{equation*}
\mathbf{A} =
\begin{pmatrix}
$\,\,$ 1 $\,\,$ & $\,\,$1$\,\,$ & $\,\,$6$\,\,$ & $\,\,$4 $\,\,$ \\
$\,\,$ 1 $\,\,$ & $\,\,$ 1 $\,\,$ & $\,\,$8$\,\,$ & $\,\,$3 $\,\,$ \\
$\,\,$ 1/6$\,\,$ & $\,\,$ 1/8$\,\,$ & $\,\,$ 1 $\,\,$ & $\,\,$ 1/4 $\,\,$ \\
$\,\,$ 1/4$\,\,$ & $\,\,$ 1/3$\,\,$ & $\,\,$4$\,\,$ & $\,\,$ 1  $\,\,$ \\
\end{pmatrix},
\qquad
\lambda_{\max} =
4.0820,
\qquad
CR = 0.0309
\end{equation*}

\begin{equation*}
\mathbf{w}^{cos} =
\begin{pmatrix}
0.403519\\
\color{red} 0.401887\color{black} \\
0.050880\\
0.143714
\end{pmatrix}\end{equation*}
\begin{equation*}
\left[ \frac{{w}^{cos}_i}{{w}^{cos}_j} \right] =
\begin{pmatrix}
$\,\,$ 1 $\,\,$ & $\,\,$\color{red} 1.0041\color{black} $\,\,$ & $\,\,$7.9308$\,\,$ & $\,\,$2.8078$\,\,$ \\
$\,\,$\color{red} 0.9960\color{black} $\,\,$ & $\,\,$ 1 $\,\,$ & $\,\,$\color{red} 7.8987\color{black} $\,\,$ & $\,\,$\color{red} 2.7964\color{black}   $\,\,$ \\
$\,\,$0.1261$\,\,$ & $\,\,$\color{red} 0.1266\color{black} $\,\,$ & $\,\,$ 1 $\,\,$ & $\,\,$0.3540 $\,\,$ \\
$\,\,$0.3562$\,\,$ & $\,\,$\color{red} 0.3576\color{black} $\,\,$ & $\,\,$2.8246$\,\,$ & $\,\,$ 1  $\,\,$ \\
\end{pmatrix},
\end{equation*}

\begin{equation*}
\mathbf{w}^{\prime} =
\begin{pmatrix}
0.402861\\
0.402861\\
0.050797\\
0.143480
\end{pmatrix} =
0.998371\cdot
\begin{pmatrix}
0.403519\\
\color{gr} 0.403519\color{black} \\
0.050880\\
0.143714
\end{pmatrix},
\end{equation*}
\begin{equation*}
\left[ \frac{{w}^{\prime}_i}{{w}^{\prime}_j} \right] =
\begin{pmatrix}
$\,\,$ 1 $\,\,$ & $\,\,$\color{gr} \color{blue} 1\color{black} $\,\,$ & $\,\,$7.9308$\,\,$ & $\,\,$2.8078$\,\,$ \\
$\,\,$\color{gr} \color{blue} 1\color{black} $\,\,$ & $\,\,$ 1 $\,\,$ & $\,\,$\color{gr} 7.9308\color{black} $\,\,$ & $\,\,$\color{gr} 2.8078\color{black}   $\,\,$ \\
$\,\,$0.1261$\,\,$ & $\,\,$\color{gr} 0.1261\color{black} $\,\,$ & $\,\,$ 1 $\,\,$ & $\,\,$0.3540 $\,\,$ \\
$\,\,$0.3562$\,\,$ & $\,\,$\color{gr} 0.3562\color{black} $\,\,$ & $\,\,$2.8246$\,\,$ & $\,\,$ 1  $\,\,$ \\
\end{pmatrix},
\end{equation*}
\end{example}
\newpage
\begin{example}
\begin{equation*}
\mathbf{A} =
\begin{pmatrix}
$\,\,$ 1 $\,\,$ & $\,\,$1$\,\,$ & $\,\,$6$\,\,$ & $\,\,$4 $\,\,$ \\
$\,\,$ 1 $\,\,$ & $\,\,$ 1 $\,\,$ & $\,\,$9$\,\,$ & $\,\,$2 $\,\,$ \\
$\,\,$ 1/6$\,\,$ & $\,\,$ 1/9$\,\,$ & $\,\,$ 1 $\,\,$ & $\,\,$ 1/7 $\,\,$ \\
$\,\,$ 1/4$\,\,$ & $\,\,$ 1/2$\,\,$ & $\,\,$7$\,\,$ & $\,\,$ 1  $\,\,$ \\
\end{pmatrix},
\qquad
\lambda_{\max} =
4.2086,
\qquad
CR = 0.0786
\end{equation*}

\begin{equation*}
\mathbf{w}^{cos} =
\begin{pmatrix}
0.399282\\
\color{red} 0.369027\color{black} \\
0.044133\\
0.187558
\end{pmatrix}\end{equation*}
\begin{equation*}
\left[ \frac{{w}^{cos}_i}{{w}^{cos}_j} \right] =
\begin{pmatrix}
$\,\,$ 1 $\,\,$ & $\,\,$\color{red} 1.0820\color{black} $\,\,$ & $\,\,$9.0472$\,\,$ & $\,\,$2.1288$\,\,$ \\
$\,\,$\color{red} 0.9242\color{black} $\,\,$ & $\,\,$ 1 $\,\,$ & $\,\,$\color{red} 8.3616\color{black} $\,\,$ & $\,\,$\color{red} 1.9675\color{black}   $\,\,$ \\
$\,\,$0.1105$\,\,$ & $\,\,$\color{red} 0.1196\color{black} $\,\,$ & $\,\,$ 1 $\,\,$ & $\,\,$0.2353 $\,\,$ \\
$\,\,$0.4697$\,\,$ & $\,\,$\color{red} 0.5083\color{black} $\,\,$ & $\,\,$4.2498$\,\,$ & $\,\,$ 1  $\,\,$ \\
\end{pmatrix},
\end{equation*}

\begin{equation*}
\mathbf{w}^{\prime} =
\begin{pmatrix}
0.396865\\
0.372846\\
0.043866\\
0.186423
\end{pmatrix} =
0.993947\cdot
\begin{pmatrix}
0.399282\\
\color{gr} 0.375117\color{black} \\
0.044133\\
0.187558
\end{pmatrix},
\end{equation*}
\begin{equation*}
\left[ \frac{{w}^{\prime}_i}{{w}^{\prime}_j} \right] =
\begin{pmatrix}
$\,\,$ 1 $\,\,$ & $\,\,$\color{gr} 1.0644\color{black} $\,\,$ & $\,\,$9.0472$\,\,$ & $\,\,$2.1288$\,\,$ \\
$\,\,$\color{gr} 0.9395\color{black} $\,\,$ & $\,\,$ 1 $\,\,$ & $\,\,$\color{gr} 8.4996\color{black} $\,\,$ & $\,\,$\color{gr} \color{blue} 2\color{black}   $\,\,$ \\
$\,\,$0.1105$\,\,$ & $\,\,$\color{gr} 0.1177\color{black} $\,\,$ & $\,\,$ 1 $\,\,$ & $\,\,$0.2353 $\,\,$ \\
$\,\,$0.4697$\,\,$ & $\,\,$\color{gr} \color{blue}  1/2\color{black} $\,\,$ & $\,\,$4.2498$\,\,$ & $\,\,$ 1  $\,\,$ \\
\end{pmatrix},
\end{equation*}
\end{example}
\newpage
\begin{example}
\begin{equation*}
\mathbf{A} =
\begin{pmatrix}
$\,\,$ 1 $\,\,$ & $\,\,$1$\,\,$ & $\,\,$6$\,\,$ & $\,\,$4 $\,\,$ \\
$\,\,$ 1 $\,\,$ & $\,\,$ 1 $\,\,$ & $\,\,$9$\,\,$ & $\,\,$2 $\,\,$ \\
$\,\,$ 1/6$\,\,$ & $\,\,$ 1/9$\,\,$ & $\,\,$ 1 $\,\,$ & $\,\,$ 1/8 $\,\,$ \\
$\,\,$ 1/4$\,\,$ & $\,\,$ 1/2$\,\,$ & $\,\,$8$\,\,$ & $\,\,$ 1  $\,\,$ \\
\end{pmatrix},
\qquad
\lambda_{\max} =
4.2469,
\qquad
CR = 0.0931
\end{equation*}

\begin{equation*}
\mathbf{w}^{cos} =
\begin{pmatrix}
0.396664\\
\color{red} 0.364932\color{black} \\
0.043102\\
0.195301
\end{pmatrix}\end{equation*}
\begin{equation*}
\left[ \frac{{w}^{cos}_i}{{w}^{cos}_j} \right] =
\begin{pmatrix}
$\,\,$ 1 $\,\,$ & $\,\,$\color{red} 1.0870\color{black} $\,\,$ & $\,\,$9.2028$\,\,$ & $\,\,$2.0310$\,\,$ \\
$\,\,$\color{red} 0.9200\color{black} $\,\,$ & $\,\,$ 1 $\,\,$ & $\,\,$\color{red} 8.4666\color{black} $\,\,$ & $\,\,$\color{red} 1.8686\color{black}   $\,\,$ \\
$\,\,$0.1087$\,\,$ & $\,\,$\color{red} 0.1181\color{black} $\,\,$ & $\,\,$ 1 $\,\,$ & $\,\,$0.2207 $\,\,$ \\
$\,\,$0.4924$\,\,$ & $\,\,$\color{red} 0.5352\color{black} $\,\,$ & $\,\,$4.5311$\,\,$ & $\,\,$ 1  $\,\,$ \\
\end{pmatrix},
\end{equation*}

\begin{equation*}
\mathbf{w}^{\prime} =
\begin{pmatrix}
0.387750\\
0.379204\\
0.042134\\
0.190912
\end{pmatrix} =
0.977527\cdot
\begin{pmatrix}
0.396664\\
\color{gr} 0.387921\color{black} \\
0.043102\\
0.195301
\end{pmatrix},
\end{equation*}
\begin{equation*}
\left[ \frac{{w}^{\prime}_i}{{w}^{\prime}_j} \right] =
\begin{pmatrix}
$\,\,$ 1 $\,\,$ & $\,\,$\color{gr} 1.0225\color{black} $\,\,$ & $\,\,$9.2028$\,\,$ & $\,\,$2.0310$\,\,$ \\
$\,\,$\color{gr} 0.9780\color{black} $\,\,$ & $\,\,$ 1 $\,\,$ & $\,\,$\color{gr} \color{blue} 9\color{black} $\,\,$ & $\,\,$\color{gr} 1.9863\color{black}   $\,\,$ \\
$\,\,$0.1087$\,\,$ & $\,\,$\color{gr} \color{blue}  1/9\color{black} $\,\,$ & $\,\,$ 1 $\,\,$ & $\,\,$0.2207 $\,\,$ \\
$\,\,$0.4924$\,\,$ & $\,\,$\color{gr} 0.5035\color{black} $\,\,$ & $\,\,$4.5311$\,\,$ & $\,\,$ 1  $\,\,$ \\
\end{pmatrix},
\end{equation*}
\end{example}
\newpage
\begin{example}
\begin{equation*}
\mathbf{A} =
\begin{pmatrix}
$\,\,$ 1 $\,\,$ & $\,\,$1$\,\,$ & $\,\,$6$\,\,$ & $\,\,$5 $\,\,$ \\
$\,\,$ 1 $\,\,$ & $\,\,$ 1 $\,\,$ & $\,\,$8$\,\,$ & $\,\,$3 $\,\,$ \\
$\,\,$ 1/6$\,\,$ & $\,\,$ 1/8$\,\,$ & $\,\,$ 1 $\,\,$ & $\,\,$ 1/4 $\,\,$ \\
$\,\,$ 1/5$\,\,$ & $\,\,$ 1/3$\,\,$ & $\,\,$4$\,\,$ & $\,\,$ 1  $\,\,$ \\
\end{pmatrix},
\qquad
\lambda_{\max} =
4.1252,
\qquad
CR = 0.0472
\end{equation*}

\begin{equation*}
\mathbf{w}^{cos} =
\begin{pmatrix}
0.418066\\
\color{red} 0.395080\color{black} \\
0.050568\\
0.136286
\end{pmatrix}\end{equation*}
\begin{equation*}
\left[ \frac{{w}^{cos}_i}{{w}^{cos}_j} \right] =
\begin{pmatrix}
$\,\,$ 1 $\,\,$ & $\,\,$\color{red} 1.0582\color{black} $\,\,$ & $\,\,$8.2674$\,\,$ & $\,\,$3.0676$\,\,$ \\
$\,\,$\color{red} 0.9450\color{black} $\,\,$ & $\,\,$ 1 $\,\,$ & $\,\,$\color{red} 7.8129\color{black} $\,\,$ & $\,\,$\color{red} 2.8989\color{black}   $\,\,$ \\
$\,\,$0.1210$\,\,$ & $\,\,$\color{red} 0.1280\color{black} $\,\,$ & $\,\,$ 1 $\,\,$ & $\,\,$0.3710 $\,\,$ \\
$\,\,$0.3260$\,\,$ & $\,\,$\color{red} 0.3450\color{black} $\,\,$ & $\,\,$2.6951$\,\,$ & $\,\,$ 1  $\,\,$ \\
\end{pmatrix},
\end{equation*}

\begin{equation*}
\mathbf{w}^{\prime} =
\begin{pmatrix}
0.414147\\
0.400750\\
0.050094\\
0.135009
\end{pmatrix} =
0.990626\cdot
\begin{pmatrix}
0.418066\\
\color{gr} 0.404542\color{black} \\
0.050568\\
0.136286
\end{pmatrix},
\end{equation*}
\begin{equation*}
\left[ \frac{{w}^{\prime}_i}{{w}^{\prime}_j} \right] =
\begin{pmatrix}
$\,\,$ 1 $\,\,$ & $\,\,$\color{gr} 1.0334\color{black} $\,\,$ & $\,\,$8.2674$\,\,$ & $\,\,$3.0676$\,\,$ \\
$\,\,$\color{gr} 0.9677\color{black} $\,\,$ & $\,\,$ 1 $\,\,$ & $\,\,$\color{gr} \color{blue} 8\color{black} $\,\,$ & $\,\,$\color{gr} 2.9683\color{black}   $\,\,$ \\
$\,\,$0.1210$\,\,$ & $\,\,$\color{gr} \color{blue}  1/8\color{black} $\,\,$ & $\,\,$ 1 $\,\,$ & $\,\,$0.3710 $\,\,$ \\
$\,\,$0.3260$\,\,$ & $\,\,$\color{gr} 0.3369\color{black} $\,\,$ & $\,\,$2.6951$\,\,$ & $\,\,$ 1  $\,\,$ \\
\end{pmatrix},
\end{equation*}
\end{example}
\newpage
\begin{example}
\begin{equation*}
\mathbf{A} =
\begin{pmatrix}
$\,\,$ 1 $\,\,$ & $\,\,$1$\,\,$ & $\,\,$6$\,\,$ & $\,\,$5 $\,\,$ \\
$\,\,$ 1 $\,\,$ & $\,\,$ 1 $\,\,$ & $\,\,$9$\,\,$ & $\,\,$3 $\,\,$ \\
$\,\,$ 1/6$\,\,$ & $\,\,$ 1/9$\,\,$ & $\,\,$ 1 $\,\,$ & $\,\,$ 1/5 $\,\,$ \\
$\,\,$ 1/5$\,\,$ & $\,\,$ 1/3$\,\,$ & $\,\,$5$\,\,$ & $\,\,$ 1  $\,\,$ \\
\end{pmatrix},
\qquad
\lambda_{\max} =
4.1758,
\qquad
CR = 0.0663
\end{equation*}

\begin{equation*}
\mathbf{w}^{cos} =
\begin{pmatrix}
0.411251\\
\color{red} 0.398136\color{black} \\
0.046666\\
0.143947
\end{pmatrix}\end{equation*}
\begin{equation*}
\left[ \frac{{w}^{cos}_i}{{w}^{cos}_j} \right] =
\begin{pmatrix}
$\,\,$ 1 $\,\,$ & $\,\,$\color{red} 1.0329\color{black} $\,\,$ & $\,\,$8.8127$\,\,$ & $\,\,$2.8570$\,\,$ \\
$\,\,$\color{red} 0.9681\color{black} $\,\,$ & $\,\,$ 1 $\,\,$ & $\,\,$\color{red} 8.5316\color{black} $\,\,$ & $\,\,$\color{red} 2.7659\color{black}   $\,\,$ \\
$\,\,$0.1135$\,\,$ & $\,\,$\color{red} 0.1172\color{black} $\,\,$ & $\,\,$ 1 $\,\,$ & $\,\,$0.3242 $\,\,$ \\
$\,\,$0.3500$\,\,$ & $\,\,$\color{red} 0.3616\color{black} $\,\,$ & $\,\,$3.0846$\,\,$ & $\,\,$ 1  $\,\,$ \\
\end{pmatrix},
\end{equation*}

\begin{equation*}
\mathbf{w}^{\prime} =
\begin{pmatrix}
0.405927\\
0.405927\\
0.046062\\
0.142083
\end{pmatrix} =
0.987055\cdot
\begin{pmatrix}
0.411251\\
\color{gr} 0.411251\color{black} \\
0.046666\\
0.143947
\end{pmatrix},
\end{equation*}
\begin{equation*}
\left[ \frac{{w}^{\prime}_i}{{w}^{\prime}_j} \right] =
\begin{pmatrix}
$\,\,$ 1 $\,\,$ & $\,\,$\color{gr} \color{blue} 1\color{black} $\,\,$ & $\,\,$8.8127$\,\,$ & $\,\,$2.8570$\,\,$ \\
$\,\,$\color{gr} \color{blue} 1\color{black} $\,\,$ & $\,\,$ 1 $\,\,$ & $\,\,$\color{gr} 8.8127\color{black} $\,\,$ & $\,\,$\color{gr} 2.8570\color{black}   $\,\,$ \\
$\,\,$0.1135$\,\,$ & $\,\,$\color{gr} 0.1135\color{black} $\,\,$ & $\,\,$ 1 $\,\,$ & $\,\,$0.3242 $\,\,$ \\
$\,\,$0.3500$\,\,$ & $\,\,$\color{gr} 0.3500\color{black} $\,\,$ & $\,\,$3.0846$\,\,$ & $\,\,$ 1  $\,\,$ \\
\end{pmatrix},
\end{equation*}
\end{example}
\newpage
\begin{example}
\begin{equation*}
\mathbf{A} =
\begin{pmatrix}
$\,\,$ 1 $\,\,$ & $\,\,$1$\,\,$ & $\,\,$6$\,\,$ & $\,\,$5 $\,\,$ \\
$\,\,$ 1 $\,\,$ & $\,\,$ 1 $\,\,$ & $\,\,$9$\,\,$ & $\,\,$3 $\,\,$ \\
$\,\,$ 1/6$\,\,$ & $\,\,$ 1/9$\,\,$ & $\,\,$ 1 $\,\,$ & $\,\,$ 1/6 $\,\,$ \\
$\,\,$ 1/5$\,\,$ & $\,\,$ 1/3$\,\,$ & $\,\,$6$\,\,$ & $\,\,$ 1  $\,\,$ \\
\end{pmatrix},
\qquad
\lambda_{\max} =
4.2277,
\qquad
CR = 0.0859
\end{equation*}

\begin{equation*}
\mathbf{w}^{cos} =
\begin{pmatrix}
0.407825\\
\color{red} 0.393350\color{black} \\
0.045282\\
0.153543
\end{pmatrix}\end{equation*}
\begin{equation*}
\left[ \frac{{w}^{cos}_i}{{w}^{cos}_j} \right] =
\begin{pmatrix}
$\,\,$ 1 $\,\,$ & $\,\,$\color{red} 1.0368\color{black} $\,\,$ & $\,\,$9.0064$\,\,$ & $\,\,$2.6561$\,\,$ \\
$\,\,$\color{red} 0.9645\color{black} $\,\,$ & $\,\,$ 1 $\,\,$ & $\,\,$\color{red} 8.6867\color{black} $\,\,$ & $\,\,$\color{red} 2.5618\color{black}   $\,\,$ \\
$\,\,$0.1110$\,\,$ & $\,\,$\color{red} 0.1151\color{black} $\,\,$ & $\,\,$ 1 $\,\,$ & $\,\,$0.2949 $\,\,$ \\
$\,\,$0.3765$\,\,$ & $\,\,$\color{red} 0.3903\color{black} $\,\,$ & $\,\,$3.3908$\,\,$ & $\,\,$ 1  $\,\,$ \\
\end{pmatrix},
\end{equation*}

\begin{equation*}
\mathbf{w}^{\prime} =
\begin{pmatrix}
0.402120\\
0.401836\\
0.044648\\
0.151396
\end{pmatrix} =
0.986012\cdot
\begin{pmatrix}
0.407825\\
\color{gr} 0.407536\color{black} \\
0.045282\\
0.153543
\end{pmatrix},
\end{equation*}
\begin{equation*}
\left[ \frac{{w}^{\prime}_i}{{w}^{\prime}_j} \right] =
\begin{pmatrix}
$\,\,$ 1 $\,\,$ & $\,\,$\color{gr} 1.0007\color{black} $\,\,$ & $\,\,$9.0064$\,\,$ & $\,\,$2.6561$\,\,$ \\
$\,\,$\color{gr} 0.9993\color{black} $\,\,$ & $\,\,$ 1 $\,\,$ & $\,\,$\color{gr} \color{blue} 9\color{black} $\,\,$ & $\,\,$\color{gr} 2.6542\color{black}   $\,\,$ \\
$\,\,$0.1110$\,\,$ & $\,\,$\color{gr} \color{blue}  1/9\color{black} $\,\,$ & $\,\,$ 1 $\,\,$ & $\,\,$0.2949 $\,\,$ \\
$\,\,$0.3765$\,\,$ & $\,\,$\color{gr} 0.3768\color{black} $\,\,$ & $\,\,$3.3908$\,\,$ & $\,\,$ 1  $\,\,$ \\
\end{pmatrix},
\end{equation*}
\end{example}
\newpage
\begin{example}
\begin{equation*}
\mathbf{A} =
\begin{pmatrix}
$\,\,$ 1 $\,\,$ & $\,\,$1$\,\,$ & $\,\,$6$\,\,$ & $\,\,$6 $\,\,$ \\
$\,\,$ 1 $\,\,$ & $\,\,$ 1 $\,\,$ & $\,\,$8$\,\,$ & $\,\,$3 $\,\,$ \\
$\,\,$ 1/6$\,\,$ & $\,\,$ 1/8$\,\,$ & $\,\,$ 1 $\,\,$ & $\,\,$ 1/4 $\,\,$ \\
$\,\,$ 1/6$\,\,$ & $\,\,$ 1/3$\,\,$ & $\,\,$4$\,\,$ & $\,\,$ 1  $\,\,$ \\
\end{pmatrix},
\qquad
\lambda_{\max} =
4.1707,
\qquad
CR = 0.0644
\end{equation*}

\begin{equation*}
\mathbf{w}^{cos} =
\begin{pmatrix}
0.428869\\
\color{red} 0.389792\color{black} \\
0.050342\\
0.130997
\end{pmatrix}\end{equation*}
\begin{equation*}
\left[ \frac{{w}^{cos}_i}{{w}^{cos}_j} \right] =
\begin{pmatrix}
$\,\,$ 1 $\,\,$ & $\,\,$\color{red} 1.1003\color{black} $\,\,$ & $\,\,$8.5192$\,\,$ & $\,\,$3.2739$\,\,$ \\
$\,\,$\color{red} 0.9089\color{black} $\,\,$ & $\,\,$ 1 $\,\,$ & $\,\,$\color{red} 7.7429\color{black} $\,\,$ & $\,\,$\color{red} 2.9756\color{black}   $\,\,$ \\
$\,\,$0.1174$\,\,$ & $\,\,$\color{red} 0.1292\color{black} $\,\,$ & $\,\,$ 1 $\,\,$ & $\,\,$0.3843 $\,\,$ \\
$\,\,$0.3054$\,\,$ & $\,\,$\color{red} 0.3361\color{black} $\,\,$ & $\,\,$2.6022$\,\,$ & $\,\,$ 1  $\,\,$ \\
\end{pmatrix},
\end{equation*}

\begin{equation*}
\mathbf{w}^{\prime} =
\begin{pmatrix}
0.427501\\
0.391739\\
0.050181\\
0.130580
\end{pmatrix} =
0.996810\cdot
\begin{pmatrix}
0.428869\\
\color{gr} 0.392992\color{black} \\
0.050342\\
0.130997
\end{pmatrix},
\end{equation*}
\begin{equation*}
\left[ \frac{{w}^{\prime}_i}{{w}^{\prime}_j} \right] =
\begin{pmatrix}
$\,\,$ 1 $\,\,$ & $\,\,$\color{gr} 1.0913\color{black} $\,\,$ & $\,\,$8.5192$\,\,$ & $\,\,$3.2739$\,\,$ \\
$\,\,$\color{gr} 0.9163\color{black} $\,\,$ & $\,\,$ 1 $\,\,$ & $\,\,$\color{gr} 7.8065\color{black} $\,\,$ & $\,\,$\color{gr} \color{blue} 3\color{black}   $\,\,$ \\
$\,\,$0.1174$\,\,$ & $\,\,$\color{gr} 0.1281\color{black} $\,\,$ & $\,\,$ 1 $\,\,$ & $\,\,$0.3843 $\,\,$ \\
$\,\,$0.3054$\,\,$ & $\,\,$\color{gr} \color{blue}  1/3\color{black} $\,\,$ & $\,\,$2.6022$\,\,$ & $\,\,$ 1  $\,\,$ \\
\end{pmatrix},
\end{equation*}
\end{example}
\newpage
\begin{example}
\begin{equation*}
\mathbf{A} =
\begin{pmatrix}
$\,\,$ 1 $\,\,$ & $\,\,$1$\,\,$ & $\,\,$6$\,\,$ & $\,\,$6 $\,\,$ \\
$\,\,$ 1 $\,\,$ & $\,\,$ 1 $\,\,$ & $\,\,$8$\,\,$ & $\,\,$3 $\,\,$ \\
$\,\,$ 1/6$\,\,$ & $\,\,$ 1/8$\,\,$ & $\,\,$ 1 $\,\,$ & $\,\,$ 1/5 $\,\,$ \\
$\,\,$ 1/6$\,\,$ & $\,\,$ 1/3$\,\,$ & $\,\,$5$\,\,$ & $\,\,$ 1  $\,\,$ \\
\end{pmatrix},
\qquad
\lambda_{\max} =
4.2311,
\qquad
CR = 0.0871
\end{equation*}

\begin{equation*}
\mathbf{w}^{cos} =
\begin{pmatrix}
0.424661\\
\color{red} 0.384656\color{black} \\
0.048578\\
0.142105
\end{pmatrix}\end{equation*}
\begin{equation*}
\left[ \frac{{w}^{cos}_i}{{w}^{cos}_j} \right] =
\begin{pmatrix}
$\,\,$ 1 $\,\,$ & $\,\,$\color{red} 1.1040\color{black} $\,\,$ & $\,\,$8.7419$\,\,$ & $\,\,$2.9884$\,\,$ \\
$\,\,$\color{red} 0.9058\color{black} $\,\,$ & $\,\,$ 1 $\,\,$ & $\,\,$\color{red} 7.9184\color{black} $\,\,$ & $\,\,$\color{red} 2.7069\color{black}   $\,\,$ \\
$\,\,$0.1144$\,\,$ & $\,\,$\color{red} 0.1263\color{black} $\,\,$ & $\,\,$ 1 $\,\,$ & $\,\,$0.3418 $\,\,$ \\
$\,\,$0.3346$\,\,$ & $\,\,$\color{red} 0.3694\color{black} $\,\,$ & $\,\,$2.9253$\,\,$ & $\,\,$ 1  $\,\,$ \\
\end{pmatrix},
\end{equation*}

\begin{equation*}
\mathbf{w}^{\prime} =
\begin{pmatrix}
0.422984\\
0.387087\\
0.048386\\
0.141543
\end{pmatrix} =
0.996050\cdot
\begin{pmatrix}
0.424661\\
\color{gr} 0.388622\color{black} \\
0.048578\\
0.142105
\end{pmatrix},
\end{equation*}
\begin{equation*}
\left[ \frac{{w}^{\prime}_i}{{w}^{\prime}_j} \right] =
\begin{pmatrix}
$\,\,$ 1 $\,\,$ & $\,\,$\color{gr} 1.0927\color{black} $\,\,$ & $\,\,$8.7419$\,\,$ & $\,\,$2.9884$\,\,$ \\
$\,\,$\color{gr} 0.9151\color{black} $\,\,$ & $\,\,$ 1 $\,\,$ & $\,\,$\color{gr} \color{blue} 8\color{black} $\,\,$ & $\,\,$\color{gr} 2.7348\color{black}   $\,\,$ \\
$\,\,$0.1144$\,\,$ & $\,\,$\color{gr} \color{blue}  1/8\color{black} $\,\,$ & $\,\,$ 1 $\,\,$ & $\,\,$0.3418 $\,\,$ \\
$\,\,$0.3346$\,\,$ & $\,\,$\color{gr} 0.3657\color{black} $\,\,$ & $\,\,$2.9253$\,\,$ & $\,\,$ 1  $\,\,$ \\
\end{pmatrix},
\end{equation*}
\end{example}
\newpage
\begin{example}
\begin{equation*}
\mathbf{A} =
\begin{pmatrix}
$\,\,$ 1 $\,\,$ & $\,\,$1$\,\,$ & $\,\,$6$\,\,$ & $\,\,$6 $\,\,$ \\
$\,\,$ 1 $\,\,$ & $\,\,$ 1 $\,\,$ & $\,\,$8$\,\,$ & $\,\,$4 $\,\,$ \\
$\,\,$ 1/6$\,\,$ & $\,\,$ 1/8$\,\,$ & $\,\,$ 1 $\,\,$ & $\,\,$ 1/3 $\,\,$ \\
$\,\,$ 1/6$\,\,$ & $\,\,$ 1/4$\,\,$ & $\,\,$3$\,\,$ & $\,\,$ 1  $\,\,$ \\
\end{pmatrix},
\qquad
\lambda_{\max} =
4.1031,
\qquad
CR = 0.0389
\end{equation*}

\begin{equation*}
\mathbf{w}^{cos} =
\begin{pmatrix}
0.425666\\
\color{red} 0.412874\color{black} \\
0.052555\\
0.108905
\end{pmatrix}\end{equation*}
\begin{equation*}
\left[ \frac{{w}^{cos}_i}{{w}^{cos}_j} \right] =
\begin{pmatrix}
$\,\,$ 1 $\,\,$ & $\,\,$\color{red} 1.0310\color{black} $\,\,$ & $\,\,$8.0994$\,\,$ & $\,\,$3.9086$\,\,$ \\
$\,\,$\color{red} 0.9699\color{black} $\,\,$ & $\,\,$ 1 $\,\,$ & $\,\,$\color{red} 7.8560\color{black} $\,\,$ & $\,\,$\color{red} 3.7912\color{black}   $\,\,$ \\
$\,\,$0.1235$\,\,$ & $\,\,$\color{red} 0.1273\color{black} $\,\,$ & $\,\,$ 1 $\,\,$ & $\,\,$0.4826 $\,\,$ \\
$\,\,$0.2558$\,\,$ & $\,\,$\color{red} 0.2638\color{black} $\,\,$ & $\,\,$2.0722$\,\,$ & $\,\,$ 1  $\,\,$ \\
\end{pmatrix},
\end{equation*}

\begin{equation*}
\mathbf{w}^{\prime} =
\begin{pmatrix}
0.422469\\
0.417284\\
0.052160\\
0.108087
\end{pmatrix} =
0.992489\cdot
\begin{pmatrix}
0.425666\\
\color{gr} 0.420442\color{black} \\
0.052555\\
0.108905
\end{pmatrix},
\end{equation*}
\begin{equation*}
\left[ \frac{{w}^{\prime}_i}{{w}^{\prime}_j} \right] =
\begin{pmatrix}
$\,\,$ 1 $\,\,$ & $\,\,$\color{gr} 1.0124\color{black} $\,\,$ & $\,\,$8.0994$\,\,$ & $\,\,$3.9086$\,\,$ \\
$\,\,$\color{gr} 0.9877\color{black} $\,\,$ & $\,\,$ 1 $\,\,$ & $\,\,$\color{gr} \color{blue} 8\color{black} $\,\,$ & $\,\,$\color{gr} 3.8606\color{black}   $\,\,$ \\
$\,\,$0.1235$\,\,$ & $\,\,$\color{gr} \color{blue}  1/8\color{black} $\,\,$ & $\,\,$ 1 $\,\,$ & $\,\,$0.4826 $\,\,$ \\
$\,\,$0.2558$\,\,$ & $\,\,$\color{gr} 0.2590\color{black} $\,\,$ & $\,\,$2.0722$\,\,$ & $\,\,$ 1  $\,\,$ \\
\end{pmatrix},
\end{equation*}
\end{example}
\newpage
\begin{example}
\begin{equation*}
\mathbf{A} =
\begin{pmatrix}
$\,\,$ 1 $\,\,$ & $\,\,$1$\,\,$ & $\,\,$6$\,\,$ & $\,\,$6 $\,\,$ \\
$\,\,$ 1 $\,\,$ & $\,\,$ 1 $\,\,$ & $\,\,$9$\,\,$ & $\,\,$3 $\,\,$ \\
$\,\,$ 1/6$\,\,$ & $\,\,$ 1/9$\,\,$ & $\,\,$ 1 $\,\,$ & $\,\,$ 1/5 $\,\,$ \\
$\,\,$ 1/6$\,\,$ & $\,\,$ 1/3$\,\,$ & $\,\,$5$\,\,$ & $\,\,$ 1  $\,\,$ \\
\end{pmatrix},
\qquad
\lambda_{\max} =
4.2277,
\qquad
CR = 0.0859
\end{equation*}

\begin{equation*}
\mathbf{w}^{cos} =
\begin{pmatrix}
0.421939\\
\color{red} 0.392803\color{black} \\
0.046554\\
0.138704
\end{pmatrix}\end{equation*}
\begin{equation*}
\left[ \frac{{w}^{cos}_i}{{w}^{cos}_j} \right] =
\begin{pmatrix}
$\,\,$ 1 $\,\,$ & $\,\,$\color{red} 1.0742\color{black} $\,\,$ & $\,\,$9.0634$\,\,$ & $\,\,$3.0420$\,\,$ \\
$\,\,$\color{red} 0.9309\color{black} $\,\,$ & $\,\,$ 1 $\,\,$ & $\,\,$\color{red} 8.4375\color{black} $\,\,$ & $\,\,$\color{red} 2.8320\color{black}   $\,\,$ \\
$\,\,$0.1103$\,\,$ & $\,\,$\color{red} 0.1185\color{black} $\,\,$ & $\,\,$ 1 $\,\,$ & $\,\,$0.3356 $\,\,$ \\
$\,\,$0.3287$\,\,$ & $\,\,$\color{red} 0.3531\color{black} $\,\,$ & $\,\,$2.9794$\,\,$ & $\,\,$ 1  $\,\,$ \\
\end{pmatrix},
\end{equation*}

\begin{equation*}
\mathbf{w}^{\prime} =
\begin{pmatrix}
0.412329\\
0.406633\\
0.045494\\
0.135544
\end{pmatrix} =
0.977223\cdot
\begin{pmatrix}
0.421939\\
\color{gr} 0.416111\color{black} \\
0.046554\\
0.138704
\end{pmatrix},
\end{equation*}
\begin{equation*}
\left[ \frac{{w}^{\prime}_i}{{w}^{\prime}_j} \right] =
\begin{pmatrix}
$\,\,$ 1 $\,\,$ & $\,\,$\color{gr} 1.0140\color{black} $\,\,$ & $\,\,$9.0634$\,\,$ & $\,\,$3.0420$\,\,$ \\
$\,\,$\color{gr} 0.9862\color{black} $\,\,$ & $\,\,$ 1 $\,\,$ & $\,\,$\color{gr} 8.9382\color{black} $\,\,$ & $\,\,$\color{gr} \color{blue} 3\color{black}   $\,\,$ \\
$\,\,$0.1103$\,\,$ & $\,\,$\color{gr} 0.1119\color{black} $\,\,$ & $\,\,$ 1 $\,\,$ & $\,\,$0.3356 $\,\,$ \\
$\,\,$0.3287$\,\,$ & $\,\,$\color{gr} \color{blue}  1/3\color{black} $\,\,$ & $\,\,$2.9794$\,\,$ & $\,\,$ 1  $\,\,$ \\
\end{pmatrix},
\end{equation*}
\end{example}
\newpage
\begin{example}
\begin{equation*}
\mathbf{A} =
\begin{pmatrix}
$\,\,$ 1 $\,\,$ & $\,\,$1$\,\,$ & $\,\,$6$\,\,$ & $\,\,$6 $\,\,$ \\
$\,\,$ 1 $\,\,$ & $\,\,$ 1 $\,\,$ & $\,\,$9$\,\,$ & $\,\,$4 $\,\,$ \\
$\,\,$ 1/6$\,\,$ & $\,\,$ 1/9$\,\,$ & $\,\,$ 1 $\,\,$ & $\,\,$ 1/3 $\,\,$ \\
$\,\,$ 1/6$\,\,$ & $\,\,$ 1/4$\,\,$ & $\,\,$3$\,\,$ & $\,\,$ 1  $\,\,$ \\
\end{pmatrix},
\qquad
\lambda_{\max} =
4.1031,
\qquad
CR = 0.0389
\end{equation*}

\begin{equation*}
\mathbf{w}^{cos} =
\begin{pmatrix}
0.422180\\
\color{red} 0.420916\color{black} \\
0.050380\\
0.106524
\end{pmatrix}\end{equation*}
\begin{equation*}
\left[ \frac{{w}^{cos}_i}{{w}^{cos}_j} \right] =
\begin{pmatrix}
$\,\,$ 1 $\,\,$ & $\,\,$\color{red} 1.0030\color{black} $\,\,$ & $\,\,$8.3799$\,\,$ & $\,\,$3.9632$\,\,$ \\
$\,\,$\color{red} 0.9970\color{black} $\,\,$ & $\,\,$ 1 $\,\,$ & $\,\,$\color{red} 8.3548\color{black} $\,\,$ & $\,\,$\color{red} 3.9514\color{black}   $\,\,$ \\
$\,\,$0.1193$\,\,$ & $\,\,$\color{red} 0.1197\color{black} $\,\,$ & $\,\,$ 1 $\,\,$ & $\,\,$0.4729 $\,\,$ \\
$\,\,$0.2523$\,\,$ & $\,\,$\color{red} 0.2531\color{black} $\,\,$ & $\,\,$2.1144$\,\,$ & $\,\,$ 1  $\,\,$ \\
\end{pmatrix},
\end{equation*}

\begin{equation*}
\mathbf{w}^{\prime} =
\begin{pmatrix}
0.421647\\
0.421647\\
0.050316\\
0.106390
\end{pmatrix} =
0.998737\cdot
\begin{pmatrix}
0.422180\\
\color{gr} 0.422180\color{black} \\
0.050380\\
0.106524
\end{pmatrix},
\end{equation*}
\begin{equation*}
\left[ \frac{{w}^{\prime}_i}{{w}^{\prime}_j} \right] =
\begin{pmatrix}
$\,\,$ 1 $\,\,$ & $\,\,$\color{gr} \color{blue} 1\color{black} $\,\,$ & $\,\,$8.3799$\,\,$ & $\,\,$3.9632$\,\,$ \\
$\,\,$\color{gr} \color{blue} 1\color{black} $\,\,$ & $\,\,$ 1 $\,\,$ & $\,\,$\color{gr} 8.3799\color{black} $\,\,$ & $\,\,$\color{gr} 3.9632\color{black}   $\,\,$ \\
$\,\,$0.1193$\,\,$ & $\,\,$\color{gr} 0.1193\color{black} $\,\,$ & $\,\,$ 1 $\,\,$ & $\,\,$0.4729 $\,\,$ \\
$\,\,$0.2523$\,\,$ & $\,\,$\color{gr} 0.2523\color{black} $\,\,$ & $\,\,$2.1144$\,\,$ & $\,\,$ 1  $\,\,$ \\
\end{pmatrix},
\end{equation*}
\end{example}
\newpage
\begin{example}
\begin{equation*}
\mathbf{A} =
\begin{pmatrix}
$\,\,$ 1 $\,\,$ & $\,\,$1$\,\,$ & $\,\,$6$\,\,$ & $\,\,$6 $\,\,$ \\
$\,\,$ 1 $\,\,$ & $\,\,$ 1 $\,\,$ & $\,\,$9$\,\,$ & $\,\,$4 $\,\,$ \\
$\,\,$ 1/6$\,\,$ & $\,\,$ 1/9$\,\,$ & $\,\,$ 1 $\,\,$ & $\,\,$ 1/4 $\,\,$ \\
$\,\,$ 1/6$\,\,$ & $\,\,$ 1/4$\,\,$ & $\,\,$4$\,\,$ & $\,\,$ 1  $\,\,$ \\
\end{pmatrix},
\qquad
\lambda_{\max} =
4.1664,
\qquad
CR = 0.0627
\end{equation*}

\begin{equation*}
\mathbf{w}^{cos} =
\begin{pmatrix}
0.418079\\
\color{red} 0.415724\color{black} \\
0.048035\\
0.118162
\end{pmatrix}\end{equation*}
\begin{equation*}
\left[ \frac{{w}^{cos}_i}{{w}^{cos}_j} \right] =
\begin{pmatrix}
$\,\,$ 1 $\,\,$ & $\,\,$\color{red} 1.0057\color{black} $\,\,$ & $\,\,$8.7036$\,\,$ & $\,\,$3.5382$\,\,$ \\
$\,\,$\color{red} 0.9944\color{black} $\,\,$ & $\,\,$ 1 $\,\,$ & $\,\,$\color{red} 8.6545\color{black} $\,\,$ & $\,\,$\color{red} 3.5183\color{black}   $\,\,$ \\
$\,\,$0.1149$\,\,$ & $\,\,$\color{red} 0.1155\color{black} $\,\,$ & $\,\,$ 1 $\,\,$ & $\,\,$0.4065 $\,\,$ \\
$\,\,$0.2826$\,\,$ & $\,\,$\color{red} 0.2842\color{black} $\,\,$ & $\,\,$2.4599$\,\,$ & $\,\,$ 1  $\,\,$ \\
\end{pmatrix},
\end{equation*}

\begin{equation*}
\mathbf{w}^{\prime} =
\begin{pmatrix}
0.417097\\
0.417097\\
0.047922\\
0.117884
\end{pmatrix} =
0.997651\cdot
\begin{pmatrix}
0.418079\\
\color{gr} 0.418079\color{black} \\
0.048035\\
0.118162
\end{pmatrix},
\end{equation*}
\begin{equation*}
\left[ \frac{{w}^{\prime}_i}{{w}^{\prime}_j} \right] =
\begin{pmatrix}
$\,\,$ 1 $\,\,$ & $\,\,$\color{gr} \color{blue} 1\color{black} $\,\,$ & $\,\,$8.7036$\,\,$ & $\,\,$3.5382$\,\,$ \\
$\,\,$\color{gr} \color{blue} 1\color{black} $\,\,$ & $\,\,$ 1 $\,\,$ & $\,\,$\color{gr} 8.7036\color{black} $\,\,$ & $\,\,$\color{gr} 3.5382\color{black}   $\,\,$ \\
$\,\,$0.1149$\,\,$ & $\,\,$\color{gr} 0.1149\color{black} $\,\,$ & $\,\,$ 1 $\,\,$ & $\,\,$0.4065 $\,\,$ \\
$\,\,$0.2826$\,\,$ & $\,\,$\color{gr} 0.2826\color{black} $\,\,$ & $\,\,$2.4599$\,\,$ & $\,\,$ 1  $\,\,$ \\
\end{pmatrix},
\end{equation*}
\end{example}
\newpage
\begin{example}
\begin{equation*}
\mathbf{A} =
\begin{pmatrix}
$\,\,$ 1 $\,\,$ & $\,\,$1$\,\,$ & $\,\,$6$\,\,$ & $\,\,$6 $\,\,$ \\
$\,\,$ 1 $\,\,$ & $\,\,$ 1 $\,\,$ & $\,\,$9$\,\,$ & $\,\,$4 $\,\,$ \\
$\,\,$ 1/6$\,\,$ & $\,\,$ 1/9$\,\,$ & $\,\,$ 1 $\,\,$ & $\,\,$ 1/5 $\,\,$ \\
$\,\,$ 1/6$\,\,$ & $\,\,$ 1/4$\,\,$ & $\,\,$5$\,\,$ & $\,\,$ 1  $\,\,$ \\
\end{pmatrix},
\qquad
\lambda_{\max} =
4.2316,
\qquad
CR = 0.0873
\end{equation*}

\begin{equation*}
\mathbf{w}^{cos} =
\begin{pmatrix}
0.414188\\
\color{red} 0.410568\color{black} \\
0.046409\\
0.128835
\end{pmatrix}\end{equation*}
\begin{equation*}
\left[ \frac{{w}^{cos}_i}{{w}^{cos}_j} \right] =
\begin{pmatrix}
$\,\,$ 1 $\,\,$ & $\,\,$\color{red} 1.0088\color{black} $\,\,$ & $\,\,$8.9247$\,\,$ & $\,\,$3.2149$\,\,$ \\
$\,\,$\color{red} 0.9913\color{black} $\,\,$ & $\,\,$ 1 $\,\,$ & $\,\,$\color{red} 8.8467\color{black} $\,\,$ & $\,\,$\color{red} 3.1868\color{black}   $\,\,$ \\
$\,\,$0.1120$\,\,$ & $\,\,$\color{red} 0.1130\color{black} $\,\,$ & $\,\,$ 1 $\,\,$ & $\,\,$0.3602 $\,\,$ \\
$\,\,$0.3111$\,\,$ & $\,\,$\color{red} 0.3138\color{black} $\,\,$ & $\,\,$2.7761$\,\,$ & $\,\,$ 1  $\,\,$ \\
\end{pmatrix},
\end{equation*}

\begin{equation*}
\mathbf{w}^{\prime} =
\begin{pmatrix}
0.412694\\
0.412694\\
0.046242\\
0.128371
\end{pmatrix} =
0.996394\cdot
\begin{pmatrix}
0.414188\\
\color{gr} 0.414188\color{black} \\
0.046409\\
0.128835
\end{pmatrix},
\end{equation*}
\begin{equation*}
\left[ \frac{{w}^{\prime}_i}{{w}^{\prime}_j} \right] =
\begin{pmatrix}
$\,\,$ 1 $\,\,$ & $\,\,$\color{gr} \color{blue} 1\color{black} $\,\,$ & $\,\,$8.9247$\,\,$ & $\,\,$3.2149$\,\,$ \\
$\,\,$\color{gr} \color{blue} 1\color{black} $\,\,$ & $\,\,$ 1 $\,\,$ & $\,\,$\color{gr} 8.9247\color{black} $\,\,$ & $\,\,$\color{gr} 3.2149\color{black}   $\,\,$ \\
$\,\,$0.1120$\,\,$ & $\,\,$\color{gr} 0.1120\color{black} $\,\,$ & $\,\,$ 1 $\,\,$ & $\,\,$0.3602 $\,\,$ \\
$\,\,$0.3111$\,\,$ & $\,\,$\color{gr} 0.3111\color{black} $\,\,$ & $\,\,$2.7761$\,\,$ & $\,\,$ 1  $\,\,$ \\
\end{pmatrix},
\end{equation*}
\end{example}
\newpage
\begin{example}
\begin{equation*}
\mathbf{A} =
\begin{pmatrix}
$\,\,$ 1 $\,\,$ & $\,\,$1$\,\,$ & $\,\,$6$\,\,$ & $\,\,$7 $\,\,$ \\
$\,\,$ 1 $\,\,$ & $\,\,$ 1 $\,\,$ & $\,\,$8$\,\,$ & $\,\,$4 $\,\,$ \\
$\,\,$ 1/6$\,\,$ & $\,\,$ 1/8$\,\,$ & $\,\,$ 1 $\,\,$ & $\,\,$ 1/3 $\,\,$ \\
$\,\,$ 1/7$\,\,$ & $\,\,$ 1/4$\,\,$ & $\,\,$3$\,\,$ & $\,\,$ 1  $\,\,$ \\
\end{pmatrix},
\qquad
\lambda_{\max} =
4.1365,
\qquad
CR = 0.0515
\end{equation*}

\begin{equation*}
\mathbf{w}^{cos} =
\begin{pmatrix}
0.435058\\
\color{red} 0.407608\color{black} \\
0.052278\\
0.105056
\end{pmatrix}\end{equation*}
\begin{equation*}
\left[ \frac{{w}^{cos}_i}{{w}^{cos}_j} \right] =
\begin{pmatrix}
$\,\,$ 1 $\,\,$ & $\,\,$\color{red} 1.0673\color{black} $\,\,$ & $\,\,$8.3221$\,\,$ & $\,\,$4.1412$\,\,$ \\
$\,\,$\color{red} 0.9369\color{black} $\,\,$ & $\,\,$ 1 $\,\,$ & $\,\,$\color{red} 7.7970\color{black} $\,\,$ & $\,\,$\color{red} 3.8799\color{black}   $\,\,$ \\
$\,\,$0.1202$\,\,$ & $\,\,$\color{red} 0.1283\color{black} $\,\,$ & $\,\,$ 1 $\,\,$ & $\,\,$0.4976 $\,\,$ \\
$\,\,$0.2415$\,\,$ & $\,\,$\color{red} 0.2577\color{black} $\,\,$ & $\,\,$2.0096$\,\,$ & $\,\,$ 1  $\,\,$ \\
\end{pmatrix},
\end{equation*}

\begin{equation*}
\mathbf{w}^{\prime} =
\begin{pmatrix}
0.430489\\
0.413830\\
0.051729\\
0.103952
\end{pmatrix} =
0.989498\cdot
\begin{pmatrix}
0.435058\\
\color{gr} 0.418222\color{black} \\
0.052278\\
0.105056
\end{pmatrix},
\end{equation*}
\begin{equation*}
\left[ \frac{{w}^{\prime}_i}{{w}^{\prime}_j} \right] =
\begin{pmatrix}
$\,\,$ 1 $\,\,$ & $\,\,$\color{gr} 1.0403\color{black} $\,\,$ & $\,\,$8.3221$\,\,$ & $\,\,$4.1412$\,\,$ \\
$\,\,$\color{gr} 0.9613\color{black} $\,\,$ & $\,\,$ 1 $\,\,$ & $\,\,$\color{gr} \color{blue} 8\color{black} $\,\,$ & $\,\,$\color{gr} 3.9810\color{black}   $\,\,$ \\
$\,\,$0.1202$\,\,$ & $\,\,$\color{gr} \color{blue}  1/8\color{black} $\,\,$ & $\,\,$ 1 $\,\,$ & $\,\,$0.4976 $\,\,$ \\
$\,\,$0.2415$\,\,$ & $\,\,$\color{gr} 0.2512\color{black} $\,\,$ & $\,\,$2.0096$\,\,$ & $\,\,$ 1  $\,\,$ \\
\end{pmatrix},
\end{equation*}
\end{example}
\newpage
\begin{example}
\begin{equation*}
\mathbf{A} =
\begin{pmatrix}
$\,\,$ 1 $\,\,$ & $\,\,$1$\,\,$ & $\,\,$6$\,\,$ & $\,\,$7 $\,\,$ \\
$\,\,$ 1 $\,\,$ & $\,\,$ 1 $\,\,$ & $\,\,$9$\,\,$ & $\,\,$4 $\,\,$ \\
$\,\,$ 1/6$\,\,$ & $\,\,$ 1/9$\,\,$ & $\,\,$ 1 $\,\,$ & $\,\,$ 1/4 $\,\,$ \\
$\,\,$ 1/7$\,\,$ & $\,\,$ 1/4$\,\,$ & $\,\,$4$\,\,$ & $\,\,$ 1  $\,\,$ \\
\end{pmatrix},
\qquad
\lambda_{\max} =
4.2065,
\qquad
CR = 0.0779
\end{equation*}

\begin{equation*}
\mathbf{w}^{cos} =
\begin{pmatrix}
0.427344\\
\color{red} 0.410398\color{black} \\
0.047896\\
0.114362
\end{pmatrix}\end{equation*}
\begin{equation*}
\left[ \frac{{w}^{cos}_i}{{w}^{cos}_j} \right] =
\begin{pmatrix}
$\,\,$ 1 $\,\,$ & $\,\,$\color{red} 1.0413\color{black} $\,\,$ & $\,\,$8.9224$\,\,$ & $\,\,$3.7368$\,\,$ \\
$\,\,$\color{red} 0.9603\color{black} $\,\,$ & $\,\,$ 1 $\,\,$ & $\,\,$\color{red} 8.5686\color{black} $\,\,$ & $\,\,$\color{red} 3.5886\color{black}   $\,\,$ \\
$\,\,$0.1121$\,\,$ & $\,\,$\color{red} 0.1167\color{black} $\,\,$ & $\,\,$ 1 $\,\,$ & $\,\,$0.4188 $\,\,$ \\
$\,\,$0.2676$\,\,$ & $\,\,$\color{red} 0.2787\color{black} $\,\,$ & $\,\,$2.3877$\,\,$ & $\,\,$ 1  $\,\,$ \\
\end{pmatrix},
\end{equation*}

\begin{equation*}
\mathbf{w}^{\prime} =
\begin{pmatrix}
0.420223\\
0.420223\\
0.047098\\
0.112456
\end{pmatrix} =
0.983337\cdot
\begin{pmatrix}
0.427344\\
\color{gr} 0.427344\color{black} \\
0.047896\\
0.114362
\end{pmatrix},
\end{equation*}
\begin{equation*}
\left[ \frac{{w}^{\prime}_i}{{w}^{\prime}_j} \right] =
\begin{pmatrix}
$\,\,$ 1 $\,\,$ & $\,\,$\color{gr} \color{blue} 1\color{black} $\,\,$ & $\,\,$8.9224$\,\,$ & $\,\,$3.7368$\,\,$ \\
$\,\,$\color{gr} \color{blue} 1\color{black} $\,\,$ & $\,\,$ 1 $\,\,$ & $\,\,$\color{gr} 8.9224\color{black} $\,\,$ & $\,\,$\color{gr} 3.7368\color{black}   $\,\,$ \\
$\,\,$0.1121$\,\,$ & $\,\,$\color{gr} 0.1121\color{black} $\,\,$ & $\,\,$ 1 $\,\,$ & $\,\,$0.4188 $\,\,$ \\
$\,\,$0.2676$\,\,$ & $\,\,$\color{gr} 0.2676\color{black} $\,\,$ & $\,\,$2.3877$\,\,$ & $\,\,$ 1  $\,\,$ \\
\end{pmatrix},
\end{equation*}
\end{example}
\newpage
\begin{example}
\begin{equation*}
\mathbf{A} =
\begin{pmatrix}
$\,\,$ 1 $\,\,$ & $\,\,$1$\,\,$ & $\,\,$6$\,\,$ & $\,\,$8 $\,\,$ \\
$\,\,$ 1 $\,\,$ & $\,\,$ 1 $\,\,$ & $\,\,$8$\,\,$ & $\,\,$4 $\,\,$ \\
$\,\,$ 1/6$\,\,$ & $\,\,$ 1/8$\,\,$ & $\,\,$ 1 $\,\,$ & $\,\,$ 1/3 $\,\,$ \\
$\,\,$ 1/8$\,\,$ & $\,\,$ 1/4$\,\,$ & $\,\,$3$\,\,$ & $\,\,$ 1  $\,\,$ \\
\end{pmatrix},
\qquad
\lambda_{\max} =
4.1707,
\qquad
CR = 0.0644
\end{equation*}

\begin{equation*}
\mathbf{w}^{cos} =
\begin{pmatrix}
0.442600\\
\color{red} 0.403290\color{black} \\
0.052062\\
0.102048
\end{pmatrix}\end{equation*}
\begin{equation*}
\left[ \frac{{w}^{cos}_i}{{w}^{cos}_j} \right] =
\begin{pmatrix}
$\,\,$ 1 $\,\,$ & $\,\,$\color{red} 1.0975\color{black} $\,\,$ & $\,\,$8.5014$\,\,$ & $\,\,$4.3372$\,\,$ \\
$\,\,$\color{red} 0.9112\color{black} $\,\,$ & $\,\,$ 1 $\,\,$ & $\,\,$\color{red} 7.7464\color{black} $\,\,$ & $\,\,$\color{red} 3.9520\color{black}   $\,\,$ \\
$\,\,$0.1176$\,\,$ & $\,\,$\color{red} 0.1291\color{black} $\,\,$ & $\,\,$ 1 $\,\,$ & $\,\,$0.5102 $\,\,$ \\
$\,\,$0.2306$\,\,$ & $\,\,$\color{red} 0.2530\color{black} $\,\,$ & $\,\,$1.9601$\,\,$ & $\,\,$ 1  $\,\,$ \\
\end{pmatrix},
\end{equation*}

\begin{equation*}
\mathbf{w}^{\prime} =
\begin{pmatrix}
0.440441\\
0.406201\\
0.051808\\
0.101550
\end{pmatrix} =
0.995122\cdot
\begin{pmatrix}
0.442600\\
\color{gr} 0.408192\color{black} \\
0.052062\\
0.102048
\end{pmatrix},
\end{equation*}
\begin{equation*}
\left[ \frac{{w}^{\prime}_i}{{w}^{\prime}_j} \right] =
\begin{pmatrix}
$\,\,$ 1 $\,\,$ & $\,\,$\color{gr} 1.0843\color{black} $\,\,$ & $\,\,$8.5014$\,\,$ & $\,\,$4.3372$\,\,$ \\
$\,\,$\color{gr} 0.9223\color{black} $\,\,$ & $\,\,$ 1 $\,\,$ & $\,\,$\color{gr} 7.8405\color{black} $\,\,$ & $\,\,$\color{gr} \color{blue} 4\color{black}   $\,\,$ \\
$\,\,$0.1176$\,\,$ & $\,\,$\color{gr} 0.1275\color{black} $\,\,$ & $\,\,$ 1 $\,\,$ & $\,\,$0.5102 $\,\,$ \\
$\,\,$0.2306$\,\,$ & $\,\,$\color{gr} \color{blue}  1/4\color{black} $\,\,$ & $\,\,$1.9601$\,\,$ & $\,\,$ 1  $\,\,$ \\
\end{pmatrix},
\end{equation*}
\end{example}
\newpage
\begin{example}
\begin{equation*}
\mathbf{A} =
\begin{pmatrix}
$\,\,$ 1 $\,\,$ & $\,\,$1$\,\,$ & $\,\,$6$\,\,$ & $\,\,$8 $\,\,$ \\
$\,\,$ 1 $\,\,$ & $\,\,$ 1 $\,\,$ & $\,\,$8$\,\,$ & $\,\,$4 $\,\,$ \\
$\,\,$ 1/6$\,\,$ & $\,\,$ 1/8$\,\,$ & $\,\,$ 1 $\,\,$ & $\,\,$ 1/4 $\,\,$ \\
$\,\,$ 1/8$\,\,$ & $\,\,$ 1/4$\,\,$ & $\,\,$4$\,\,$ & $\,\,$ 1  $\,\,$ \\
\end{pmatrix},
\qquad
\lambda_{\max} =
4.2512,
\qquad
CR = 0.0947
\end{equation*}

\begin{equation*}
\mathbf{w}^{cos} =
\begin{pmatrix}
0.437807\\
\color{red} 0.397773\color{black} \\
0.049935\\
0.114485
\end{pmatrix}\end{equation*}
\begin{equation*}
\left[ \frac{{w}^{cos}_i}{{w}^{cos}_j} \right] =
\begin{pmatrix}
$\,\,$ 1 $\,\,$ & $\,\,$\color{red} 1.1006\color{black} $\,\,$ & $\,\,$8.7676$\,\,$ & $\,\,$3.8241$\,\,$ \\
$\,\,$\color{red} 0.9086\color{black} $\,\,$ & $\,\,$ 1 $\,\,$ & $\,\,$\color{red} 7.9658\color{black} $\,\,$ & $\,\,$\color{red} 3.4744\color{black}   $\,\,$ \\
$\,\,$0.1141$\,\,$ & $\,\,$\color{red} 0.1255\color{black} $\,\,$ & $\,\,$ 1 $\,\,$ & $\,\,$0.4362 $\,\,$ \\
$\,\,$0.2615$\,\,$ & $\,\,$\color{red} 0.2878\color{black} $\,\,$ & $\,\,$2.2927$\,\,$ & $\,\,$ 1  $\,\,$ \\
\end{pmatrix},
\end{equation*}

\begin{equation*}
\mathbf{w}^{\prime} =
\begin{pmatrix}
0.437061\\
0.398798\\
0.049850\\
0.114290
\end{pmatrix} =
0.998298\cdot
\begin{pmatrix}
0.437807\\
\color{gr} 0.399478\color{black} \\
0.049935\\
0.114485
\end{pmatrix},
\end{equation*}
\begin{equation*}
\left[ \frac{{w}^{\prime}_i}{{w}^{\prime}_j} \right] =
\begin{pmatrix}
$\,\,$ 1 $\,\,$ & $\,\,$\color{gr} 1.0959\color{black} $\,\,$ & $\,\,$8.7676$\,\,$ & $\,\,$3.8241$\,\,$ \\
$\,\,$\color{gr} 0.9125\color{black} $\,\,$ & $\,\,$ 1 $\,\,$ & $\,\,$\color{gr} \color{blue} 8\color{black} $\,\,$ & $\,\,$\color{gr} 3.4893\color{black}   $\,\,$ \\
$\,\,$0.1141$\,\,$ & $\,\,$\color{gr} \color{blue}  1/8\color{black} $\,\,$ & $\,\,$ 1 $\,\,$ & $\,\,$0.4362 $\,\,$ \\
$\,\,$0.2615$\,\,$ & $\,\,$\color{gr} 0.2866\color{black} $\,\,$ & $\,\,$2.2927$\,\,$ & $\,\,$ 1  $\,\,$ \\
\end{pmatrix},
\end{equation*}
\end{example}
\newpage
\begin{example}
\begin{equation*}
\mathbf{A} =
\begin{pmatrix}
$\,\,$ 1 $\,\,$ & $\,\,$1$\,\,$ & $\,\,$6$\,\,$ & $\,\,$8 $\,\,$ \\
$\,\,$ 1 $\,\,$ & $\,\,$ 1 $\,\,$ & $\,\,$8$\,\,$ & $\,\,$6 $\,\,$ \\
$\,\,$ 1/6$\,\,$ & $\,\,$ 1/8$\,\,$ & $\,\,$ 1 $\,\,$ & $\,\,$ 1/2 $\,\,$ \\
$\,\,$ 1/8$\,\,$ & $\,\,$ 1/6$\,\,$ & $\,\,$2$\,\,$ & $\,\,$ 1  $\,\,$ \\
\end{pmatrix},
\qquad
\lambda_{\max} =
4.0820,
\qquad
CR = 0.0309
\end{equation*}

\begin{equation*}
\mathbf{w}^{cos} =
\begin{pmatrix}
0.433939\\
\color{red} 0.433380\color{black} \\
0.054844\\
0.077836
\end{pmatrix}\end{equation*}
\begin{equation*}
\left[ \frac{{w}^{cos}_i}{{w}^{cos}_j} \right] =
\begin{pmatrix}
$\,\,$ 1 $\,\,$ & $\,\,$\color{red} 1.0013\color{black} $\,\,$ & $\,\,$7.9122$\,\,$ & $\,\,$5.5750$\,\,$ \\
$\,\,$\color{red} 0.9987\color{black} $\,\,$ & $\,\,$ 1 $\,\,$ & $\,\,$\color{red} 7.9020\color{black} $\,\,$ & $\,\,$\color{red} 5.5678\color{black}   $\,\,$ \\
$\,\,$0.1264$\,\,$ & $\,\,$\color{red} 0.1266\color{black} $\,\,$ & $\,\,$ 1 $\,\,$ & $\,\,$0.7046 $\,\,$ \\
$\,\,$0.1794$\,\,$ & $\,\,$\color{red} 0.1796\color{black} $\,\,$ & $\,\,$1.4192$\,\,$ & $\,\,$ 1  $\,\,$ \\
\end{pmatrix},
\end{equation*}

\begin{equation*}
\mathbf{w}^{\prime} =
\begin{pmatrix}
0.433697\\
0.433697\\
0.054814\\
0.077793
\end{pmatrix} =
0.999441\cdot
\begin{pmatrix}
0.433939\\
\color{gr} 0.433939\color{black} \\
0.054844\\
0.077836
\end{pmatrix},
\end{equation*}
\begin{equation*}
\left[ \frac{{w}^{\prime}_i}{{w}^{\prime}_j} \right] =
\begin{pmatrix}
$\,\,$ 1 $\,\,$ & $\,\,$\color{gr} \color{blue} 1\color{black} $\,\,$ & $\,\,$7.9122$\,\,$ & $\,\,$5.5750$\,\,$ \\
$\,\,$\color{gr} \color{blue} 1\color{black} $\,\,$ & $\,\,$ 1 $\,\,$ & $\,\,$\color{gr} 7.9122\color{black} $\,\,$ & $\,\,$\color{gr} 5.5750\color{black}   $\,\,$ \\
$\,\,$0.1264$\,\,$ & $\,\,$\color{gr} 0.1264\color{black} $\,\,$ & $\,\,$ 1 $\,\,$ & $\,\,$0.7046 $\,\,$ \\
$\,\,$0.1794$\,\,$ & $\,\,$\color{gr} 0.1794\color{black} $\,\,$ & $\,\,$1.4192$\,\,$ & $\,\,$ 1  $\,\,$ \\
\end{pmatrix},
\end{equation*}
\end{example}
\newpage
\begin{example}
\begin{equation*}
\mathbf{A} =
\begin{pmatrix}
$\,\,$ 1 $\,\,$ & $\,\,$1$\,\,$ & $\,\,$6$\,\,$ & $\,\,$8 $\,\,$ \\
$\,\,$ 1 $\,\,$ & $\,\,$ 1 $\,\,$ & $\,\,$9$\,\,$ & $\,\,$4 $\,\,$ \\
$\,\,$ 1/6$\,\,$ & $\,\,$ 1/9$\,\,$ & $\,\,$ 1 $\,\,$ & $\,\,$ 1/4 $\,\,$ \\
$\,\,$ 1/8$\,\,$ & $\,\,$ 1/4$\,\,$ & $\,\,$4$\,\,$ & $\,\,$ 1  $\,\,$ \\
\end{pmatrix},
\qquad
\lambda_{\max} =
4.2469,
\qquad
CR = 0.0931
\end{equation*}

\begin{equation*}
\mathbf{w}^{cos} =
\begin{pmatrix}
0.434770\\
\color{red} 0.406033\color{black} \\
0.047795\\
0.111402
\end{pmatrix}\end{equation*}
\begin{equation*}
\left[ \frac{{w}^{cos}_i}{{w}^{cos}_j} \right] =
\begin{pmatrix}
$\,\,$ 1 $\,\,$ & $\,\,$\color{red} 1.0708\color{black} $\,\,$ & $\,\,$9.0965$\,\,$ & $\,\,$3.9027$\,\,$ \\
$\,\,$\color{red} 0.9339\color{black} $\,\,$ & $\,\,$ 1 $\,\,$ & $\,\,$\color{red} 8.4952\color{black} $\,\,$ & $\,\,$\color{red} 3.6448\color{black}   $\,\,$ \\
$\,\,$0.1099$\,\,$ & $\,\,$\color{red} 0.1177\color{black} $\,\,$ & $\,\,$ 1 $\,\,$ & $\,\,$0.4290 $\,\,$ \\
$\,\,$0.2562$\,\,$ & $\,\,$\color{red} 0.2744\color{black} $\,\,$ & $\,\,$2.3308$\,\,$ & $\,\,$ 1  $\,\,$ \\
\end{pmatrix},
\end{equation*}

\begin{equation*}
\mathbf{w}^{\prime} =
\begin{pmatrix}
0.424528\\
0.420025\\
0.046669\\
0.108778
\end{pmatrix} =
0.976443\cdot
\begin{pmatrix}
0.434770\\
\color{gr} 0.430158\color{black} \\
0.047795\\
0.111402
\end{pmatrix},
\end{equation*}
\begin{equation*}
\left[ \frac{{w}^{\prime}_i}{{w}^{\prime}_j} \right] =
\begin{pmatrix}
$\,\,$ 1 $\,\,$ & $\,\,$\color{gr} 1.0107\color{black} $\,\,$ & $\,\,$9.0965$\,\,$ & $\,\,$3.9027$\,\,$ \\
$\,\,$\color{gr} 0.9894\color{black} $\,\,$ & $\,\,$ 1 $\,\,$ & $\,\,$\color{gr} \color{blue} 9\color{black} $\,\,$ & $\,\,$\color{gr} 3.8613\color{black}   $\,\,$ \\
$\,\,$0.1099$\,\,$ & $\,\,$\color{gr} \color{blue}  1/9\color{black} $\,\,$ & $\,\,$ 1 $\,\,$ & $\,\,$0.4290 $\,\,$ \\
$\,\,$0.2562$\,\,$ & $\,\,$\color{gr} 0.2590\color{black} $\,\,$ & $\,\,$2.3308$\,\,$ & $\,\,$ 1  $\,\,$ \\
\end{pmatrix},
\end{equation*}
\end{example}
\newpage
\begin{example}
\begin{equation*}
\mathbf{A} =
\begin{pmatrix}
$\,\,$ 1 $\,\,$ & $\,\,$1$\,\,$ & $\,\,$6$\,\,$ & $\,\,$8 $\,\,$ \\
$\,\,$ 1 $\,\,$ & $\,\,$ 1 $\,\,$ & $\,\,$9$\,\,$ & $\,\,$5 $\,\,$ \\
$\,\,$ 1/6$\,\,$ & $\,\,$ 1/9$\,\,$ & $\,\,$ 1 $\,\,$ & $\,\,$ 1/3 $\,\,$ \\
$\,\,$ 1/8$\,\,$ & $\,\,$ 1/5$\,\,$ & $\,\,$3$\,\,$ & $\,\,$ 1  $\,\,$ \\
\end{pmatrix},
\qquad
\lambda_{\max} =
4.1655,
\qquad
CR = 0.0624
\end{equation*}

\begin{equation*}
\mathbf{w}^{cos} =
\begin{pmatrix}
0.432413\\
\color{red} 0.424642\color{black} \\
0.049593\\
0.093352
\end{pmatrix}\end{equation*}
\begin{equation*}
\left[ \frac{{w}^{cos}_i}{{w}^{cos}_j} \right] =
\begin{pmatrix}
$\,\,$ 1 $\,\,$ & $\,\,$\color{red} 1.0183\color{black} $\,\,$ & $\,\,$8.7193$\,\,$ & $\,\,$4.6321$\,\,$ \\
$\,\,$\color{red} 0.9820\color{black} $\,\,$ & $\,\,$ 1 $\,\,$ & $\,\,$\color{red} 8.5626\color{black} $\,\,$ & $\,\,$\color{red} 4.5488\color{black}   $\,\,$ \\
$\,\,$0.1147$\,\,$ & $\,\,$\color{red} 0.1168\color{black} $\,\,$ & $\,\,$ 1 $\,\,$ & $\,\,$0.5312 $\,\,$ \\
$\,\,$0.2159$\,\,$ & $\,\,$\color{red} 0.2198\color{black} $\,\,$ & $\,\,$1.8824$\,\,$ & $\,\,$ 1  $\,\,$ \\
\end{pmatrix},
\end{equation*}

\begin{equation*}
\mathbf{w}^{\prime} =
\begin{pmatrix}
0.429079\\
0.429079\\
0.049210\\
0.092633
\end{pmatrix} =
0.992290\cdot
\begin{pmatrix}
0.432413\\
\color{gr} 0.432413\color{black} \\
0.049593\\
0.093352
\end{pmatrix},
\end{equation*}
\begin{equation*}
\left[ \frac{{w}^{\prime}_i}{{w}^{\prime}_j} \right] =
\begin{pmatrix}
$\,\,$ 1 $\,\,$ & $\,\,$\color{gr} \color{blue} 1\color{black} $\,\,$ & $\,\,$8.7193$\,\,$ & $\,\,$4.6321$\,\,$ \\
$\,\,$\color{gr} \color{blue} 1\color{black} $\,\,$ & $\,\,$ 1 $\,\,$ & $\,\,$\color{gr} 8.7193\color{black} $\,\,$ & $\,\,$\color{gr} 4.6321\color{black}   $\,\,$ \\
$\,\,$0.1147$\,\,$ & $\,\,$\color{gr} 0.1147\color{black} $\,\,$ & $\,\,$ 1 $\,\,$ & $\,\,$0.5312 $\,\,$ \\
$\,\,$0.2159$\,\,$ & $\,\,$\color{gr} 0.2159\color{black} $\,\,$ & $\,\,$1.8824$\,\,$ & $\,\,$ 1  $\,\,$ \\
\end{pmatrix},
\end{equation*}
\end{example}
\newpage
\begin{example}
\begin{equation*}
\mathbf{A} =
\begin{pmatrix}
$\,\,$ 1 $\,\,$ & $\,\,$1$\,\,$ & $\,\,$6$\,\,$ & $\,\,$8 $\,\,$ \\
$\,\,$ 1 $\,\,$ & $\,\,$ 1 $\,\,$ & $\,\,$9$\,\,$ & $\,\,$5 $\,\,$ \\
$\,\,$ 1/6$\,\,$ & $\,\,$ 1/9$\,\,$ & $\,\,$ 1 $\,\,$ & $\,\,$ 1/4 $\,\,$ \\
$\,\,$ 1/8$\,\,$ & $\,\,$ 1/5$\,\,$ & $\,\,$4$\,\,$ & $\,\,$ 1  $\,\,$ \\
\end{pmatrix},
\qquad
\lambda_{\max} =
4.2500,
\qquad
CR = 0.0942
\end{equation*}

\begin{equation*}
\mathbf{w}^{cos} =
\begin{pmatrix}
0.428022\\
\color{red} 0.419185\color{black} \\
0.047610\\
0.105182
\end{pmatrix}\end{equation*}
\begin{equation*}
\left[ \frac{{w}^{cos}_i}{{w}^{cos}_j} \right] =
\begin{pmatrix}
$\,\,$ 1 $\,\,$ & $\,\,$\color{red} 1.0211\color{black} $\,\,$ & $\,\,$8.9901$\,\,$ & $\,\,$4.0693$\,\,$ \\
$\,\,$\color{red} 0.9794\color{black} $\,\,$ & $\,\,$ 1 $\,\,$ & $\,\,$\color{red} 8.8045\color{black} $\,\,$ & $\,\,$\color{red} 3.9853\color{black}   $\,\,$ \\
$\,\,$0.1112$\,\,$ & $\,\,$\color{red} 0.1136\color{black} $\,\,$ & $\,\,$ 1 $\,\,$ & $\,\,$0.4526 $\,\,$ \\
$\,\,$0.2457$\,\,$ & $\,\,$\color{red} 0.2509\color{black} $\,\,$ & $\,\,$2.2092$\,\,$ & $\,\,$ 1  $\,\,$ \\
\end{pmatrix},
\end{equation*}

\begin{equation*}
\mathbf{w}^{\prime} =
\begin{pmatrix}
0.424273\\
0.424273\\
0.047193\\
0.104261
\end{pmatrix} =
0.991241\cdot
\begin{pmatrix}
0.428022\\
\color{gr} 0.428022\color{black} \\
0.047610\\
0.105182
\end{pmatrix},
\end{equation*}
\begin{equation*}
\left[ \frac{{w}^{\prime}_i}{{w}^{\prime}_j} \right] =
\begin{pmatrix}
$\,\,$ 1 $\,\,$ & $\,\,$\color{gr} \color{blue} 1\color{black} $\,\,$ & $\,\,$8.9901$\,\,$ & $\,\,$4.0693$\,\,$ \\
$\,\,$\color{gr} \color{blue} 1\color{black} $\,\,$ & $\,\,$ 1 $\,\,$ & $\,\,$\color{gr} 8.9901\color{black} $\,\,$ & $\,\,$\color{gr} 4.0693\color{black}   $\,\,$ \\
$\,\,$0.1112$\,\,$ & $\,\,$\color{gr} 0.1112\color{black} $\,\,$ & $\,\,$ 1 $\,\,$ & $\,\,$0.4526 $\,\,$ \\
$\,\,$0.2457$\,\,$ & $\,\,$\color{gr} 0.2457\color{black} $\,\,$ & $\,\,$2.2092$\,\,$ & $\,\,$ 1  $\,\,$ \\
\end{pmatrix},
\end{equation*}
\end{example}
\newpage
\begin{example}
\begin{equation*}
\mathbf{A} =
\begin{pmatrix}
$\,\,$ 1 $\,\,$ & $\,\,$1$\,\,$ & $\,\,$6$\,\,$ & $\,\,$9 $\,\,$ \\
$\,\,$ 1 $\,\,$ & $\,\,$ 1 $\,\,$ & $\,\,$8$\,\,$ & $\,\,$5 $\,\,$ \\
$\,\,$ 1/6$\,\,$ & $\,\,$ 1/8$\,\,$ & $\,\,$ 1 $\,\,$ & $\,\,$ 1/3 $\,\,$ \\
$\,\,$ 1/9$\,\,$ & $\,\,$ 1/5$\,\,$ & $\,\,$3$\,\,$ & $\,\,$ 1  $\,\,$ \\
\end{pmatrix},
\qquad
\lambda_{\max} =
4.1999,
\qquad
CR = 0.0754
\end{equation*}

\begin{equation*}
\mathbf{w}^{cos} =
\begin{pmatrix}
0.442660\\
\color{red} 0.412134\color{black} \\
0.051671\\
0.093535
\end{pmatrix}\end{equation*}
\begin{equation*}
\left[ \frac{{w}^{cos}_i}{{w}^{cos}_j} \right] =
\begin{pmatrix}
$\,\,$ 1 $\,\,$ & $\,\,$\color{red} 1.0741\color{black} $\,\,$ & $\,\,$8.5669$\,\,$ & $\,\,$4.7325$\,\,$ \\
$\,\,$\color{red} 0.9310\color{black} $\,\,$ & $\,\,$ 1 $\,\,$ & $\,\,$\color{red} 7.9761\color{black} $\,\,$ & $\,\,$\color{red} 4.4062\color{black}   $\,\,$ \\
$\,\,$0.1167$\,\,$ & $\,\,$\color{red} 0.1254\color{black} $\,\,$ & $\,\,$ 1 $\,\,$ & $\,\,$0.5524 $\,\,$ \\
$\,\,$0.2113$\,\,$ & $\,\,$\color{red} 0.2270\color{black} $\,\,$ & $\,\,$1.8102$\,\,$ & $\,\,$ 1  $\,\,$ \\
\end{pmatrix},
\end{equation*}

\begin{equation*}
\mathbf{w}^{\prime} =
\begin{pmatrix}
0.442114\\
0.412858\\
0.051607\\
0.093420
\end{pmatrix} =
0.998767\cdot
\begin{pmatrix}
0.442660\\
\color{gr} 0.413368\color{black} \\
0.051671\\
0.093535
\end{pmatrix},
\end{equation*}
\begin{equation*}
\left[ \frac{{w}^{\prime}_i}{{w}^{\prime}_j} \right] =
\begin{pmatrix}
$\,\,$ 1 $\,\,$ & $\,\,$\color{gr} 1.0709\color{black} $\,\,$ & $\,\,$8.5669$\,\,$ & $\,\,$4.7325$\,\,$ \\
$\,\,$\color{gr} 0.9338\color{black} $\,\,$ & $\,\,$ 1 $\,\,$ & $\,\,$\color{gr} \color{blue} 8\color{black} $\,\,$ & $\,\,$\color{gr} 4.4194\color{black}   $\,\,$ \\
$\,\,$0.1167$\,\,$ & $\,\,$\color{gr} \color{blue}  1/8\color{black} $\,\,$ & $\,\,$ 1 $\,\,$ & $\,\,$0.5524 $\,\,$ \\
$\,\,$0.2113$\,\,$ & $\,\,$\color{gr} 0.2263\color{black} $\,\,$ & $\,\,$1.8102$\,\,$ & $\,\,$ 1  $\,\,$ \\
\end{pmatrix},
\end{equation*}
\end{example}
\newpage
\begin{example}
\begin{equation*}
\mathbf{A} =
\begin{pmatrix}
$\,\,$ 1 $\,\,$ & $\,\,$1$\,\,$ & $\,\,$6$\,\,$ & $\,\,$9 $\,\,$ \\
$\,\,$ 1 $\,\,$ & $\,\,$ 1 $\,\,$ & $\,\,$8$\,\,$ & $\,\,$6 $\,\,$ \\
$\,\,$ 1/6$\,\,$ & $\,\,$ 1/8$\,\,$ & $\,\,$ 1 $\,\,$ & $\,\,$ 1/2 $\,\,$ \\
$\,\,$ 1/9$\,\,$ & $\,\,$ 1/6$\,\,$ & $\,\,$2$\,\,$ & $\,\,$ 1  $\,\,$ \\
\end{pmatrix},
\qquad
\lambda_{\max} =
4.1031,
\qquad
CR = 0.0389
\end{equation*}

\begin{equation*}
\mathbf{w}^{cos} =
\begin{pmatrix}
0.441285\\
\color{red} 0.428625\color{black} \\
0.054548\\
0.075542
\end{pmatrix}\end{equation*}
\begin{equation*}
\left[ \frac{{w}^{cos}_i}{{w}^{cos}_j} \right] =
\begin{pmatrix}
$\,\,$ 1 $\,\,$ & $\,\,$\color{red} 1.0295\color{black} $\,\,$ & $\,\,$8.0899$\,\,$ & $\,\,$5.8416$\,\,$ \\
$\,\,$\color{red} 0.9713\color{black} $\,\,$ & $\,\,$ 1 $\,\,$ & $\,\,$\color{red} 7.8578\color{black} $\,\,$ & $\,\,$\color{red} 5.6740\color{black}   $\,\,$ \\
$\,\,$0.1236$\,\,$ & $\,\,$\color{red} 0.1273\color{black} $\,\,$ & $\,\,$ 1 $\,\,$ & $\,\,$0.7221 $\,\,$ \\
$\,\,$0.1712$\,\,$ & $\,\,$\color{red} 0.1762\color{black} $\,\,$ & $\,\,$1.3849$\,\,$ & $\,\,$ 1  $\,\,$ \\
\end{pmatrix},
\end{equation*}

\begin{equation*}
\mathbf{w}^{\prime} =
\begin{pmatrix}
0.437888\\
0.433024\\
0.054128\\
0.074961
\end{pmatrix} =
0.992302\cdot
\begin{pmatrix}
0.441285\\
\color{gr} 0.436383\color{black} \\
0.054548\\
0.075542
\end{pmatrix},
\end{equation*}
\begin{equation*}
\left[ \frac{{w}^{\prime}_i}{{w}^{\prime}_j} \right] =
\begin{pmatrix}
$\,\,$ 1 $\,\,$ & $\,\,$\color{gr} 1.0112\color{black} $\,\,$ & $\,\,$8.0899$\,\,$ & $\,\,$5.8416$\,\,$ \\
$\,\,$\color{gr} 0.9889\color{black} $\,\,$ & $\,\,$ 1 $\,\,$ & $\,\,$\color{gr} \color{blue} 8\color{black} $\,\,$ & $\,\,$\color{gr} 5.7767\color{black}   $\,\,$ \\
$\,\,$0.1236$\,\,$ & $\,\,$\color{gr} \color{blue}  1/8\color{black} $\,\,$ & $\,\,$ 1 $\,\,$ & $\,\,$0.7221 $\,\,$ \\
$\,\,$0.1712$\,\,$ & $\,\,$\color{gr} 0.1731\color{black} $\,\,$ & $\,\,$1.3849$\,\,$ & $\,\,$ 1  $\,\,$ \\
\end{pmatrix},
\end{equation*}
\end{example}
\newpage
\begin{example}
\begin{equation*}
\mathbf{A} =
\begin{pmatrix}
$\,\,$ 1 $\,\,$ & $\,\,$1$\,\,$ & $\,\,$6$\,\,$ & $\,\,$9 $\,\,$ \\
$\,\,$ 1 $\,\,$ & $\,\,$ 1 $\,\,$ & $\,\,$9$\,\,$ & $\,\,$5 $\,\,$ \\
$\,\,$ 1/6$\,\,$ & $\,\,$ 1/9$\,\,$ & $\,\,$ 1 $\,\,$ & $\,\,$ 1/3 $\,\,$ \\
$\,\,$ 1/9$\,\,$ & $\,\,$ 1/5$\,\,$ & $\,\,$3$\,\,$ & $\,\,$ 1  $\,\,$ \\
\end{pmatrix},
\qquad
\lambda_{\max} =
4.1966,
\qquad
CR = 0.0741
\end{equation*}

\begin{equation*}
\mathbf{w}^{cos} =
\begin{pmatrix}
0.439210\\
\color{red} 0.420343\color{black} \\
0.049438\\
0.091009
\end{pmatrix}\end{equation*}
\begin{equation*}
\left[ \frac{{w}^{cos}_i}{{w}^{cos}_j} \right] =
\begin{pmatrix}
$\,\,$ 1 $\,\,$ & $\,\,$\color{red} 1.0449\color{black} $\,\,$ & $\,\,$8.8841$\,\,$ & $\,\,$4.8260$\,\,$ \\
$\,\,$\color{red} 0.9570\color{black} $\,\,$ & $\,\,$ 1 $\,\,$ & $\,\,$\color{red} 8.5025\color{black} $\,\,$ & $\,\,$\color{red} 4.6187\color{black}   $\,\,$ \\
$\,\,$0.1126$\,\,$ & $\,\,$\color{red} 0.1176\color{black} $\,\,$ & $\,\,$ 1 $\,\,$ & $\,\,$0.5432 $\,\,$ \\
$\,\,$0.2072$\,\,$ & $\,\,$\color{red} 0.2165\color{black} $\,\,$ & $\,\,$1.8409$\,\,$ & $\,\,$ 1  $\,\,$ \\
\end{pmatrix},
\end{equation*}

\begin{equation*}
\mathbf{w}^{\prime} =
\begin{pmatrix}
0.431077\\
0.431077\\
0.048522\\
0.089324
\end{pmatrix} =
0.981482\cdot
\begin{pmatrix}
0.439210\\
\color{gr} 0.439210\color{black} \\
0.049438\\
0.091009
\end{pmatrix},
\end{equation*}
\begin{equation*}
\left[ \frac{{w}^{\prime}_i}{{w}^{\prime}_j} \right] =
\begin{pmatrix}
$\,\,$ 1 $\,\,$ & $\,\,$\color{gr} \color{blue} 1\color{black} $\,\,$ & $\,\,$8.8841$\,\,$ & $\,\,$4.8260$\,\,$ \\
$\,\,$\color{gr} \color{blue} 1\color{black} $\,\,$ & $\,\,$ 1 $\,\,$ & $\,\,$\color{gr} 8.8841\color{black} $\,\,$ & $\,\,$\color{gr} 4.8260\color{black}   $\,\,$ \\
$\,\,$0.1126$\,\,$ & $\,\,$\color{gr} 0.1126\color{black} $\,\,$ & $\,\,$ 1 $\,\,$ & $\,\,$0.5432 $\,\,$ \\
$\,\,$0.2072$\,\,$ & $\,\,$\color{gr} 0.2072\color{black} $\,\,$ & $\,\,$1.8409$\,\,$ & $\,\,$ 1  $\,\,$ \\
\end{pmatrix},
\end{equation*}
\end{example}
\newpage
\begin{example}
\begin{equation*}
\mathbf{A} =
\begin{pmatrix}
$\,\,$ 1 $\,\,$ & $\,\,$1$\,\,$ & $\,\,$6$\,\,$ & $\,\,$9 $\,\,$ \\
$\,\,$ 1 $\,\,$ & $\,\,$ 1 $\,\,$ & $\,\,$9$\,\,$ & $\,\,$6 $\,\,$ \\
$\,\,$ 1/6$\,\,$ & $\,\,$ 1/9$\,\,$ & $\,\,$ 1 $\,\,$ & $\,\,$ 1/2 $\,\,$ \\
$\,\,$ 1/9$\,\,$ & $\,\,$ 1/6$\,\,$ & $\,\,$2$\,\,$ & $\,\,$ 1  $\,\,$ \\
\end{pmatrix},
\qquad
\lambda_{\max} =
4.1031,
\qquad
CR = 0.0389
\end{equation*}

\begin{equation*}
\mathbf{w}^{cos} =
\begin{pmatrix}
0.437328\\
\color{red} 0.436654\color{black} \\
0.052237\\
0.073781
\end{pmatrix}\end{equation*}
\begin{equation*}
\left[ \frac{{w}^{cos}_i}{{w}^{cos}_j} \right] =
\begin{pmatrix}
$\,\,$ 1 $\,\,$ & $\,\,$\color{red} 1.0015\color{black} $\,\,$ & $\,\,$8.3721$\,\,$ & $\,\,$5.9274$\,\,$ \\
$\,\,$\color{red} 0.9985\color{black} $\,\,$ & $\,\,$ 1 $\,\,$ & $\,\,$\color{red} 8.3591\color{black} $\,\,$ & $\,\,$\color{red} 5.9182\color{black}   $\,\,$ \\
$\,\,$0.1194$\,\,$ & $\,\,$\color{red} 0.1196\color{black} $\,\,$ & $\,\,$ 1 $\,\,$ & $\,\,$0.7080 $\,\,$ \\
$\,\,$0.1687$\,\,$ & $\,\,$\color{red} 0.1690\color{black} $\,\,$ & $\,\,$1.4124$\,\,$ & $\,\,$ 1  $\,\,$ \\
\end{pmatrix},
\end{equation*}

\begin{equation*}
\mathbf{w}^{\prime} =
\begin{pmatrix}
0.437033\\
0.437033\\
0.052201\\
0.073732
\end{pmatrix} =
0.999326\cdot
\begin{pmatrix}
0.437328\\
\color{gr} 0.437328\color{black} \\
0.052237\\
0.073781
\end{pmatrix},
\end{equation*}
\begin{equation*}
\left[ \frac{{w}^{\prime}_i}{{w}^{\prime}_j} \right] =
\begin{pmatrix}
$\,\,$ 1 $\,\,$ & $\,\,$\color{gr} \color{blue} 1\color{black} $\,\,$ & $\,\,$8.3721$\,\,$ & $\,\,$5.9274$\,\,$ \\
$\,\,$\color{gr} \color{blue} 1\color{black} $\,\,$ & $\,\,$ 1 $\,\,$ & $\,\,$\color{gr} 8.3721\color{black} $\,\,$ & $\,\,$\color{gr} 5.9274\color{black}   $\,\,$ \\
$\,\,$0.1194$\,\,$ & $\,\,$\color{gr} 0.1194\color{black} $\,\,$ & $\,\,$ 1 $\,\,$ & $\,\,$0.7080 $\,\,$ \\
$\,\,$0.1687$\,\,$ & $\,\,$\color{gr} 0.1687\color{black} $\,\,$ & $\,\,$1.4124$\,\,$ & $\,\,$ 1  $\,\,$ \\
\end{pmatrix},
\end{equation*}
\end{example}
\newpage
\begin{example}
\begin{equation*}
\mathbf{A} =
\begin{pmatrix}
$\,\,$ 1 $\,\,$ & $\,\,$1$\,\,$ & $\,\,$6$\,\,$ & $\,\,$9 $\,\,$ \\
$\,\,$ 1 $\,\,$ & $\,\,$ 1 $\,\,$ & $\,\,$9$\,\,$ & $\,\,$6 $\,\,$ \\
$\,\,$ 1/6$\,\,$ & $\,\,$ 1/9$\,\,$ & $\,\,$ 1 $\,\,$ & $\,\,$ 1/3 $\,\,$ \\
$\,\,$ 1/9$\,\,$ & $\,\,$ 1/6$\,\,$ & $\,\,$3$\,\,$ & $\,\,$ 1  $\,\,$ \\
\end{pmatrix},
\qquad
\lambda_{\max} =
4.1990,
\qquad
CR = 0.0750
\end{equation*}

\begin{equation*}
\mathbf{w}^{cos} =
\begin{pmatrix}
0.432839\\
\color{red} 0.431297\color{black} \\
0.049211\\
0.086653
\end{pmatrix}\end{equation*}
\begin{equation*}
\left[ \frac{{w}^{cos}_i}{{w}^{cos}_j} \right] =
\begin{pmatrix}
$\,\,$ 1 $\,\,$ & $\,\,$\color{red} 1.0036\color{black} $\,\,$ & $\,\,$8.7955$\,\,$ & $\,\,$4.9951$\,\,$ \\
$\,\,$\color{red} 0.9964\color{black} $\,\,$ & $\,\,$ 1 $\,\,$ & $\,\,$\color{red} 8.7642\color{black} $\,\,$ & $\,\,$\color{red} 4.9773\color{black}   $\,\,$ \\
$\,\,$0.1137$\,\,$ & $\,\,$\color{red} 0.1141\color{black} $\,\,$ & $\,\,$ 1 $\,\,$ & $\,\,$0.5679 $\,\,$ \\
$\,\,$0.2002$\,\,$ & $\,\,$\color{red} 0.2009\color{black} $\,\,$ & $\,\,$1.7608$\,\,$ & $\,\,$ 1  $\,\,$ \\
\end{pmatrix},
\end{equation*}

\begin{equation*}
\mathbf{w}^{\prime} =
\begin{pmatrix}
0.432173\\
0.432173\\
0.049135\\
0.086519
\end{pmatrix} =
0.998460\cdot
\begin{pmatrix}
0.432839\\
\color{gr} 0.432839\color{black} \\
0.049211\\
0.086653
\end{pmatrix},
\end{equation*}
\begin{equation*}
\left[ \frac{{w}^{\prime}_i}{{w}^{\prime}_j} \right] =
\begin{pmatrix}
$\,\,$ 1 $\,\,$ & $\,\,$\color{gr} \color{blue} 1\color{black} $\,\,$ & $\,\,$8.7955$\,\,$ & $\,\,$4.9951$\,\,$ \\
$\,\,$\color{gr} \color{blue} 1\color{black} $\,\,$ & $\,\,$ 1 $\,\,$ & $\,\,$\color{gr} 8.7955\color{black} $\,\,$ & $\,\,$\color{gr} 4.9951\color{black}   $\,\,$ \\
$\,\,$0.1137$\,\,$ & $\,\,$\color{gr} 0.1137\color{black} $\,\,$ & $\,\,$ 1 $\,\,$ & $\,\,$0.5679 $\,\,$ \\
$\,\,$0.2002$\,\,$ & $\,\,$\color{gr} 0.2002\color{black} $\,\,$ & $\,\,$1.7608$\,\,$ & $\,\,$ 1  $\,\,$ \\
\end{pmatrix},
\end{equation*}
\end{example}
\newpage
\begin{example}
\begin{equation*}
\mathbf{A} =
\begin{pmatrix}
$\,\,$ 1 $\,\,$ & $\,\,$1$\,\,$ & $\,\,$7$\,\,$ & $\,\,$3 $\,\,$ \\
$\,\,$ 1 $\,\,$ & $\,\,$ 1 $\,\,$ & $\,\,$9$\,\,$ & $\,\,$2 $\,\,$ \\
$\,\,$ 1/7$\,\,$ & $\,\,$ 1/9$\,\,$ & $\,\,$ 1 $\,\,$ & $\,\,$ 1/6 $\,\,$ \\
$\,\,$ 1/3$\,\,$ & $\,\,$ 1/2$\,\,$ & $\,\,$6$\,\,$ & $\,\,$ 1  $\,\,$ \\
\end{pmatrix},
\qquad
\lambda_{\max} =
4.0762,
\qquad
CR = 0.0287
\end{equation*}

\begin{equation*}
\mathbf{w}^{cos} =
\begin{pmatrix}
0.392583\\
\color{red} 0.376219\color{black} \\
0.042810\\
0.188389
\end{pmatrix}\end{equation*}
\begin{equation*}
\left[ \frac{{w}^{cos}_i}{{w}^{cos}_j} \right] =
\begin{pmatrix}
$\,\,$ 1 $\,\,$ & $\,\,$\color{red} 1.0435\color{black} $\,\,$ & $\,\,$9.1704$\,\,$ & $\,\,$2.0839$\,\,$ \\
$\,\,$\color{red} 0.9583\color{black} $\,\,$ & $\,\,$ 1 $\,\,$ & $\,\,$\color{red} 8.7881\color{black} $\,\,$ & $\,\,$\color{red} 1.9970\color{black}   $\,\,$ \\
$\,\,$0.1090$\,\,$ & $\,\,$\color{red} 0.1138\color{black} $\,\,$ & $\,\,$ 1 $\,\,$ & $\,\,$0.2272 $\,\,$ \\
$\,\,$0.4799$\,\,$ & $\,\,$\color{red} 0.5007\color{black} $\,\,$ & $\,\,$4.4006$\,\,$ & $\,\,$ 1  $\,\,$ \\
\end{pmatrix},
\end{equation*}

\begin{equation*}
\mathbf{w}^{\prime} =
\begin{pmatrix}
0.392364\\
0.376567\\
0.042786\\
0.188283
\end{pmatrix} =
0.999442\cdot
\begin{pmatrix}
0.392583\\
\color{gr} 0.376777\color{black} \\
0.042810\\
0.188389
\end{pmatrix},
\end{equation*}
\begin{equation*}
\left[ \frac{{w}^{\prime}_i}{{w}^{\prime}_j} \right] =
\begin{pmatrix}
$\,\,$ 1 $\,\,$ & $\,\,$\color{gr} 1.0419\color{black} $\,\,$ & $\,\,$9.1704$\,\,$ & $\,\,$2.0839$\,\,$ \\
$\,\,$\color{gr} 0.9597\color{black} $\,\,$ & $\,\,$ 1 $\,\,$ & $\,\,$\color{gr} 8.8012\color{black} $\,\,$ & $\,\,$\color{gr} \color{blue} 2\color{black}   $\,\,$ \\
$\,\,$0.1090$\,\,$ & $\,\,$\color{gr} 0.1136\color{black} $\,\,$ & $\,\,$ 1 $\,\,$ & $\,\,$0.2272 $\,\,$ \\
$\,\,$0.4799$\,\,$ & $\,\,$\color{gr} \color{blue}  1/2\color{black} $\,\,$ & $\,\,$4.4006$\,\,$ & $\,\,$ 1  $\,\,$ \\
\end{pmatrix},
\end{equation*}
\end{example}
\newpage
\begin{example}
\begin{equation*}
\mathbf{A} =
\begin{pmatrix}
$\,\,$ 1 $\,\,$ & $\,\,$1$\,\,$ & $\,\,$7$\,\,$ & $\,\,$3 $\,\,$ \\
$\,\,$ 1 $\,\,$ & $\,\,$ 1 $\,\,$ & $\,\,$9$\,\,$ & $\,\,$2 $\,\,$ \\
$\,\,$ 1/7$\,\,$ & $\,\,$ 1/9$\,\,$ & $\,\,$ 1 $\,\,$ & $\,\,$ 1/7 $\,\,$ \\
$\,\,$ 1/3$\,\,$ & $\,\,$ 1/2$\,\,$ & $\,\,$7$\,\,$ & $\,\,$ 1  $\,\,$ \\
\end{pmatrix},
\qquad
\lambda_{\max} =
4.1039,
\qquad
CR = 0.0392
\end{equation*}

\begin{equation*}
\mathbf{w}^{cos} =
\begin{pmatrix}
0.389530\\
\color{red} 0.372333\color{black} \\
0.041450\\
0.196686
\end{pmatrix}\end{equation*}
\begin{equation*}
\left[ \frac{{w}^{cos}_i}{{w}^{cos}_j} \right] =
\begin{pmatrix}
$\,\,$ 1 $\,\,$ & $\,\,$\color{red} 1.0462\color{black} $\,\,$ & $\,\,$9.3975$\,\,$ & $\,\,$1.9805$\,\,$ \\
$\,\,$\color{red} 0.9559\color{black} $\,\,$ & $\,\,$ 1 $\,\,$ & $\,\,$\color{red} 8.9826\color{black} $\,\,$ & $\,\,$\color{red} 1.8930\color{black}   $\,\,$ \\
$\,\,$0.1064$\,\,$ & $\,\,$\color{red} 0.1113\color{black} $\,\,$ & $\,\,$ 1 $\,\,$ & $\,\,$0.2107 $\,\,$ \\
$\,\,$0.5049$\,\,$ & $\,\,$\color{red} 0.5283\color{black} $\,\,$ & $\,\,$4.7451$\,\,$ & $\,\,$ 1  $\,\,$ \\
\end{pmatrix},
\end{equation*}

\begin{equation*}
\mathbf{w}^{\prime} =
\begin{pmatrix}
0.389250\\
0.372784\\
0.041420\\
0.196545
\end{pmatrix} =
0.999281\cdot
\begin{pmatrix}
0.389530\\
\color{gr} 0.373053\color{black} \\
0.041450\\
0.196686
\end{pmatrix},
\end{equation*}
\begin{equation*}
\left[ \frac{{w}^{\prime}_i}{{w}^{\prime}_j} \right] =
\begin{pmatrix}
$\,\,$ 1 $\,\,$ & $\,\,$\color{gr} 1.0442\color{black} $\,\,$ & $\,\,$9.3975$\,\,$ & $\,\,$1.9805$\,\,$ \\
$\,\,$\color{gr} 0.9577\color{black} $\,\,$ & $\,\,$ 1 $\,\,$ & $\,\,$\color{gr} \color{blue} 9\color{black} $\,\,$ & $\,\,$\color{gr} 1.8967\color{black}   $\,\,$ \\
$\,\,$0.1064$\,\,$ & $\,\,$\color{gr} \color{blue}  1/9\color{black} $\,\,$ & $\,\,$ 1 $\,\,$ & $\,\,$0.2107 $\,\,$ \\
$\,\,$0.5049$\,\,$ & $\,\,$\color{gr} 0.5272\color{black} $\,\,$ & $\,\,$4.7451$\,\,$ & $\,\,$ 1  $\,\,$ \\
\end{pmatrix},
\end{equation*}
\end{example}
\newpage
\begin{example}
\begin{equation*}
\mathbf{A} =
\begin{pmatrix}
$\,\,$ 1 $\,\,$ & $\,\,$1$\,\,$ & $\,\,$7$\,\,$ & $\,\,$4 $\,\,$ \\
$\,\,$ 1 $\,\,$ & $\,\,$ 1 $\,\,$ & $\,\,$9$\,\,$ & $\,\,$2 $\,\,$ \\
$\,\,$ 1/7$\,\,$ & $\,\,$ 1/9$\,\,$ & $\,\,$ 1 $\,\,$ & $\,\,$ 1/7 $\,\,$ \\
$\,\,$ 1/4$\,\,$ & $\,\,$ 1/2$\,\,$ & $\,\,$7$\,\,$ & $\,\,$ 1  $\,\,$ \\
\end{pmatrix},
\qquad
\lambda_{\max} =
4.1714,
\qquad
CR = 0.0646
\end{equation*}

\begin{equation*}
\mathbf{w}^{cos} =
\begin{pmatrix}
0.408252\\
\color{red} 0.365653\color{black} \\
0.041323\\
0.184772
\end{pmatrix}\end{equation*}
\begin{equation*}
\left[ \frac{{w}^{cos}_i}{{w}^{cos}_j} \right] =
\begin{pmatrix}
$\,\,$ 1 $\,\,$ & $\,\,$\color{red} 1.1165\color{black} $\,\,$ & $\,\,$9.8795$\,\,$ & $\,\,$2.2095$\,\,$ \\
$\,\,$\color{red} 0.8957\color{black} $\,\,$ & $\,\,$ 1 $\,\,$ & $\,\,$\color{red} 8.8486\color{black} $\,\,$ & $\,\,$\color{red} 1.9789\color{black}   $\,\,$ \\
$\,\,$0.1012$\,\,$ & $\,\,$\color{red} 0.1130\color{black} $\,\,$ & $\,\,$ 1 $\,\,$ & $\,\,$0.2236 $\,\,$ \\
$\,\,$0.4526$\,\,$ & $\,\,$\color{red} 0.5053\color{black} $\,\,$ & $\,\,$4.4714$\,\,$ & $\,\,$ 1  $\,\,$ \\
\end{pmatrix},
\end{equation*}

\begin{equation*}
\mathbf{w}^{\prime} =
\begin{pmatrix}
0.406669\\
0.368112\\
0.041163\\
0.184056
\end{pmatrix} =
0.996123\cdot
\begin{pmatrix}
0.408252\\
\color{gr} 0.369544\color{black} \\
0.041323\\
0.184772
\end{pmatrix},
\end{equation*}
\begin{equation*}
\left[ \frac{{w}^{\prime}_i}{{w}^{\prime}_j} \right] =
\begin{pmatrix}
$\,\,$ 1 $\,\,$ & $\,\,$\color{gr} 1.1047\color{black} $\,\,$ & $\,\,$9.8795$\,\,$ & $\,\,$2.2095$\,\,$ \\
$\,\,$\color{gr} 0.9052\color{black} $\,\,$ & $\,\,$ 1 $\,\,$ & $\,\,$\color{gr} 8.9428\color{black} $\,\,$ & $\,\,$\color{gr} \color{blue} 2\color{black}   $\,\,$ \\
$\,\,$0.1012$\,\,$ & $\,\,$\color{gr} 0.1118\color{black} $\,\,$ & $\,\,$ 1 $\,\,$ & $\,\,$0.2236 $\,\,$ \\
$\,\,$0.4526$\,\,$ & $\,\,$\color{gr} \color{blue}  1/2\color{black} $\,\,$ & $\,\,$4.4714$\,\,$ & $\,\,$ 1  $\,\,$ \\
\end{pmatrix},
\end{equation*}
\end{example}
\newpage
\begin{example}
\begin{equation*}
\mathbf{A} =
\begin{pmatrix}
$\,\,$ 1 $\,\,$ & $\,\,$1$\,\,$ & $\,\,$7$\,\,$ & $\,\,$4 $\,\,$ \\
$\,\,$ 1 $\,\,$ & $\,\,$ 1 $\,\,$ & $\,\,$9$\,\,$ & $\,\,$2 $\,\,$ \\
$\,\,$ 1/7$\,\,$ & $\,\,$ 1/9$\,\,$ & $\,\,$ 1 $\,\,$ & $\,\,$ 1/8 $\,\,$ \\
$\,\,$ 1/4$\,\,$ & $\,\,$ 1/2$\,\,$ & $\,\,$8$\,\,$ & $\,\,$ 1  $\,\,$ \\
\end{pmatrix},
\qquad
\lambda_{\max} =
4.2064,
\qquad
CR = 0.0778
\end{equation*}

\begin{equation*}
\mathbf{w}^{cos} =
\begin{pmatrix}
0.405368\\
\color{red} 0.361879\color{black} \\
0.040326\\
0.192428
\end{pmatrix}\end{equation*}
\begin{equation*}
\left[ \frac{{w}^{cos}_i}{{w}^{cos}_j} \right] =
\begin{pmatrix}
$\,\,$ 1 $\,\,$ & $\,\,$\color{red} 1.1202\color{black} $\,\,$ & $\,\,$10.0523$\,\,$ & $\,\,$2.1066$\,\,$ \\
$\,\,$\color{red} 0.8927\color{black} $\,\,$ & $\,\,$ 1 $\,\,$ & $\,\,$\color{red} 8.9739\color{black} $\,\,$ & $\,\,$\color{red} 1.8806\color{black}   $\,\,$ \\
$\,\,$0.0995$\,\,$ & $\,\,$\color{red} 0.1114\color{black} $\,\,$ & $\,\,$ 1 $\,\,$ & $\,\,$0.2096 $\,\,$ \\
$\,\,$0.4747$\,\,$ & $\,\,$\color{red} 0.5317\color{black} $\,\,$ & $\,\,$4.7718$\,\,$ & $\,\,$ 1  $\,\,$ \\
\end{pmatrix},
\end{equation*}

\begin{equation*}
\mathbf{w}^{\prime} =
\begin{pmatrix}
0.404941\\
0.362550\\
0.040283\\
0.192226
\end{pmatrix} =
0.998948\cdot
\begin{pmatrix}
0.405368\\
\color{gr} 0.362932\color{black} \\
0.040326\\
0.192428
\end{pmatrix},
\end{equation*}
\begin{equation*}
\left[ \frac{{w}^{\prime}_i}{{w}^{\prime}_j} \right] =
\begin{pmatrix}
$\,\,$ 1 $\,\,$ & $\,\,$\color{gr} 1.1169\color{black} $\,\,$ & $\,\,$10.0523$\,\,$ & $\,\,$2.1066$\,\,$ \\
$\,\,$\color{gr} 0.8953\color{black} $\,\,$ & $\,\,$ 1 $\,\,$ & $\,\,$\color{gr} \color{blue} 9\color{black} $\,\,$ & $\,\,$\color{gr} 1.8861\color{black}   $\,\,$ \\
$\,\,$0.0995$\,\,$ & $\,\,$\color{gr} \color{blue}  1/9\color{black} $\,\,$ & $\,\,$ 1 $\,\,$ & $\,\,$0.2096 $\,\,$ \\
$\,\,$0.4747$\,\,$ & $\,\,$\color{gr} 0.5302\color{black} $\,\,$ & $\,\,$4.7718$\,\,$ & $\,\,$ 1  $\,\,$ \\
\end{pmatrix},
\end{equation*}
\end{example}
\newpage
\begin{example}
\begin{equation*}
\mathbf{A} =
\begin{pmatrix}
$\,\,$ 1 $\,\,$ & $\,\,$1$\,\,$ & $\,\,$7$\,\,$ & $\,\,$4 $\,\,$ \\
$\,\,$ 1 $\,\,$ & $\,\,$ 1 $\,\,$ & $\,\,$9$\,\,$ & $\,\,$3 $\,\,$ \\
$\,\,$ 1/7$\,\,$ & $\,\,$ 1/9$\,\,$ & $\,\,$ 1 $\,\,$ & $\,\,$ 1/4 $\,\,$ \\
$\,\,$ 1/4$\,\,$ & $\,\,$ 1/3$\,\,$ & $\,\,$4$\,\,$ & $\,\,$ 1  $\,\,$ \\
\end{pmatrix},
\qquad
\lambda_{\max} =
4.0576,
\qquad
CR = 0.0217
\end{equation*}

\begin{equation*}
\mathbf{w}^{cos} =
\begin{pmatrix}
0.409964\\
\color{red} 0.405337\color{black} \\
0.045886\\
0.138814
\end{pmatrix}\end{equation*}
\begin{equation*}
\left[ \frac{{w}^{cos}_i}{{w}^{cos}_j} \right] =
\begin{pmatrix}
$\,\,$ 1 $\,\,$ & $\,\,$\color{red} 1.0114\color{black} $\,\,$ & $\,\,$8.9343$\,\,$ & $\,\,$2.9533$\,\,$ \\
$\,\,$\color{red} 0.9887\color{black} $\,\,$ & $\,\,$ 1 $\,\,$ & $\,\,$\color{red} 8.8335\color{black} $\,\,$ & $\,\,$\color{red} 2.9200\color{black}   $\,\,$ \\
$\,\,$0.1119$\,\,$ & $\,\,$\color{red} 0.1132\color{black} $\,\,$ & $\,\,$ 1 $\,\,$ & $\,\,$0.3306 $\,\,$ \\
$\,\,$0.3386$\,\,$ & $\,\,$\color{red} 0.3425\color{black} $\,\,$ & $\,\,$3.0252$\,\,$ & $\,\,$ 1  $\,\,$ \\
\end{pmatrix},
\end{equation*}

\begin{equation*}
\mathbf{w}^{\prime} =
\begin{pmatrix}
0.408075\\
0.408075\\
0.045675\\
0.138174
\end{pmatrix} =
0.995394\cdot
\begin{pmatrix}
0.409964\\
\color{gr} 0.409964\color{black} \\
0.045886\\
0.138814
\end{pmatrix},
\end{equation*}
\begin{equation*}
\left[ \frac{{w}^{\prime}_i}{{w}^{\prime}_j} \right] =
\begin{pmatrix}
$\,\,$ 1 $\,\,$ & $\,\,$\color{gr} \color{blue} 1\color{black} $\,\,$ & $\,\,$8.9343$\,\,$ & $\,\,$2.9533$\,\,$ \\
$\,\,$\color{gr} \color{blue} 1\color{black} $\,\,$ & $\,\,$ 1 $\,\,$ & $\,\,$\color{gr} 8.9343\color{black} $\,\,$ & $\,\,$\color{gr} 2.9533\color{black}   $\,\,$ \\
$\,\,$0.1119$\,\,$ & $\,\,$\color{gr} 0.1119\color{black} $\,\,$ & $\,\,$ 1 $\,\,$ & $\,\,$0.3306 $\,\,$ \\
$\,\,$0.3386$\,\,$ & $\,\,$\color{gr} 0.3386\color{black} $\,\,$ & $\,\,$3.0252$\,\,$ & $\,\,$ 1  $\,\,$ \\
\end{pmatrix},
\end{equation*}
\end{example}
\newpage
\begin{example}
\begin{equation*}
\mathbf{A} =
\begin{pmatrix}
$\,\,$ 1 $\,\,$ & $\,\,$1$\,\,$ & $\,\,$7$\,\,$ & $\,\,$5 $\,\,$ \\
$\,\,$ 1 $\,\,$ & $\,\,$ 1 $\,\,$ & $\,\,$5$\,\,$ & $\,\,$2 $\,\,$ \\
$\,\,$ 1/7$\,\,$ & $\,\,$ 1/5$\,\,$ & $\,\,$ 1 $\,\,$ & $\,\,$ 1/2 $\,\,$ \\
$\,\,$ 1/5$\,\,$ & $\,\,$ 1/2$\,\,$ & $\,\,$2$\,\,$ & $\,\,$ 1  $\,\,$ \\
\end{pmatrix},
\qquad
\lambda_{\max} =
4.0716,
\qquad
CR = 0.0270
\end{equation*}

\begin{equation*}
\mathbf{w}^{cos} =
\begin{pmatrix}
0.459304\\
0.343435\\
\color{red} 0.065434\color{black} \\
0.131827
\end{pmatrix}\end{equation*}
\begin{equation*}
\left[ \frac{{w}^{cos}_i}{{w}^{cos}_j} \right] =
\begin{pmatrix}
$\,\,$ 1 $\,\,$ & $\,\,$1.3374$\,\,$ & $\,\,$\color{red} 7.0193\color{black} $\,\,$ & $\,\,$3.4842$\,\,$ \\
$\,\,$0.7477$\,\,$ & $\,\,$ 1 $\,\,$ & $\,\,$\color{red} 5.2485\color{black} $\,\,$ & $\,\,$2.6052  $\,\,$ \\
$\,\,$\color{red} 0.1425\color{black} $\,\,$ & $\,\,$\color{red} 0.1905\color{black} $\,\,$ & $\,\,$ 1 $\,\,$ & $\,\,$\color{red} 0.4964\color{black}  $\,\,$ \\
$\,\,$0.2870$\,\,$ & $\,\,$0.3838$\,\,$ & $\,\,$\color{red} 2.0146\color{black} $\,\,$ & $\,\,$ 1  $\,\,$ \\
\end{pmatrix},
\end{equation*}

\begin{equation*}
\mathbf{w}^{\prime} =
\begin{pmatrix}
0.459221\\
0.343373\\
0.065603\\
0.131803
\end{pmatrix} =
0.999820\cdot
\begin{pmatrix}
0.459304\\
0.343435\\
\color{gr} 0.065615\color{black} \\
0.131827
\end{pmatrix},
\end{equation*}
\begin{equation*}
\left[ \frac{{w}^{\prime}_i}{{w}^{\prime}_j} \right] =
\begin{pmatrix}
$\,\,$ 1 $\,\,$ & $\,\,$1.3374$\,\,$ & $\,\,$\color{gr} \color{blue} 7\color{black} $\,\,$ & $\,\,$3.4842$\,\,$ \\
$\,\,$0.7477$\,\,$ & $\,\,$ 1 $\,\,$ & $\,\,$\color{gr} 5.2341\color{black} $\,\,$ & $\,\,$2.6052  $\,\,$ \\
$\,\,$\color{gr} \color{blue}  1/7\color{black} $\,\,$ & $\,\,$\color{gr} 0.1911\color{black} $\,\,$ & $\,\,$ 1 $\,\,$ & $\,\,$\color{gr} 0.4977\color{black}  $\,\,$ \\
$\,\,$0.2870$\,\,$ & $\,\,$0.3838$\,\,$ & $\,\,$\color{gr} 2.0091\color{black} $\,\,$ & $\,\,$ 1  $\,\,$ \\
\end{pmatrix},
\end{equation*}
\end{example}
\newpage
\begin{example}
\begin{equation*}
\mathbf{A} =
\begin{pmatrix}
$\,\,$ 1 $\,\,$ & $\,\,$1$\,\,$ & $\,\,$7$\,\,$ & $\,\,$6 $\,\,$ \\
$\,\,$ 1 $\,\,$ & $\,\,$ 1 $\,\,$ & $\,\,$9$\,\,$ & $\,\,$3 $\,\,$ \\
$\,\,$ 1/7$\,\,$ & $\,\,$ 1/9$\,\,$ & $\,\,$ 1 $\,\,$ & $\,\,$ 1/5 $\,\,$ \\
$\,\,$ 1/6$\,\,$ & $\,\,$ 1/3$\,\,$ & $\,\,$5$\,\,$ & $\,\,$ 1  $\,\,$ \\
\end{pmatrix},
\qquad
\lambda_{\max} =
4.1889,
\qquad
CR = 0.0712
\end{equation*}

\begin{equation*}
\mathbf{w}^{cos} =
\begin{pmatrix}
0.431626\\
\color{red} 0.388611\color{black} \\
0.043565\\
0.136198
\end{pmatrix}\end{equation*}
\begin{equation*}
\left[ \frac{{w}^{cos}_i}{{w}^{cos}_j} \right] =
\begin{pmatrix}
$\,\,$ 1 $\,\,$ & $\,\,$\color{red} 1.1107\color{black} $\,\,$ & $\,\,$9.9076$\,\,$ & $\,\,$3.1691$\,\,$ \\
$\,\,$\color{red} 0.9003\color{black} $\,\,$ & $\,\,$ 1 $\,\,$ & $\,\,$\color{red} 8.9202\color{black} $\,\,$ & $\,\,$\color{red} 2.8533\color{black}   $\,\,$ \\
$\,\,$0.1009$\,\,$ & $\,\,$\color{red} 0.1121\color{black} $\,\,$ & $\,\,$ 1 $\,\,$ & $\,\,$0.3199 $\,\,$ \\
$\,\,$0.3155$\,\,$ & $\,\,$\color{red} 0.3505\color{black} $\,\,$ & $\,\,$3.1263$\,\,$ & $\,\,$ 1  $\,\,$ \\
\end{pmatrix},
\end{equation*}

\begin{equation*}
\mathbf{w}^{\prime} =
\begin{pmatrix}
0.430131\\
0.390728\\
0.043414\\
0.135727
\end{pmatrix} =
0.996536\cdot
\begin{pmatrix}
0.431626\\
\color{gr} 0.392086\color{black} \\
0.043565\\
0.136198
\end{pmatrix},
\end{equation*}
\begin{equation*}
\left[ \frac{{w}^{\prime}_i}{{w}^{\prime}_j} \right] =
\begin{pmatrix}
$\,\,$ 1 $\,\,$ & $\,\,$\color{gr} 1.1008\color{black} $\,\,$ & $\,\,$9.9076$\,\,$ & $\,\,$3.1691$\,\,$ \\
$\,\,$\color{gr} 0.9084\color{black} $\,\,$ & $\,\,$ 1 $\,\,$ & $\,\,$\color{gr} \color{blue} 9\color{black} $\,\,$ & $\,\,$\color{gr} 2.8788\color{black}   $\,\,$ \\
$\,\,$0.1009$\,\,$ & $\,\,$\color{gr} \color{blue}  1/9\color{black} $\,\,$ & $\,\,$ 1 $\,\,$ & $\,\,$0.3199 $\,\,$ \\
$\,\,$0.3155$\,\,$ & $\,\,$\color{gr} 0.3474\color{black} $\,\,$ & $\,\,$3.1263$\,\,$ & $\,\,$ 1  $\,\,$ \\
\end{pmatrix},
\end{equation*}
\end{example}
\newpage
\begin{example}
\begin{equation*}
\mathbf{A} =
\begin{pmatrix}
$\,\,$ 1 $\,\,$ & $\,\,$1$\,\,$ & $\,\,$7$\,\,$ & $\,\,$6 $\,\,$ \\
$\,\,$ 1 $\,\,$ & $\,\,$ 1 $\,\,$ & $\,\,$9$\,\,$ & $\,\,$4 $\,\,$ \\
$\,\,$ 1/7$\,\,$ & $\,\,$ 1/9$\,\,$ & $\,\,$ 1 $\,\,$ & $\,\,$ 1/3 $\,\,$ \\
$\,\,$ 1/6$\,\,$ & $\,\,$ 1/4$\,\,$ & $\,\,$3$\,\,$ & $\,\,$ 1  $\,\,$ \\
\end{pmatrix},
\qquad
\lambda_{\max} =
4.0762,
\qquad
CR = 0.0287
\end{equation*}

\begin{equation*}
\mathbf{w}^{cos} =
\begin{pmatrix}
0.432091\\
\color{red} 0.415872\color{black} \\
0.047275\\
0.104762
\end{pmatrix}\end{equation*}
\begin{equation*}
\left[ \frac{{w}^{cos}_i}{{w}^{cos}_j} \right] =
\begin{pmatrix}
$\,\,$ 1 $\,\,$ & $\,\,$\color{red} 1.0390\color{black} $\,\,$ & $\,\,$9.1399$\,\,$ & $\,\,$4.1245$\,\,$ \\
$\,\,$\color{red} 0.9625\color{black} $\,\,$ & $\,\,$ 1 $\,\,$ & $\,\,$\color{red} 8.7968\color{black} $\,\,$ & $\,\,$\color{red} 3.9697\color{black}   $\,\,$ \\
$\,\,$0.1094$\,\,$ & $\,\,$\color{red} 0.1137\color{black} $\,\,$ & $\,\,$ 1 $\,\,$ & $\,\,$0.4513 $\,\,$ \\
$\,\,$0.2425$\,\,$ & $\,\,$\color{red} 0.2519\color{black} $\,\,$ & $\,\,$2.2160$\,\,$ & $\,\,$ 1  $\,\,$ \\
\end{pmatrix},
\end{equation*}

\begin{equation*}
\mathbf{w}^{\prime} =
\begin{pmatrix}
0.430724\\
0.417720\\
0.047126\\
0.104430
\end{pmatrix} =
0.996835\cdot
\begin{pmatrix}
0.432091\\
\color{gr} 0.419047\color{black} \\
0.047275\\
0.104762
\end{pmatrix},
\end{equation*}
\begin{equation*}
\left[ \frac{{w}^{\prime}_i}{{w}^{\prime}_j} \right] =
\begin{pmatrix}
$\,\,$ 1 $\,\,$ & $\,\,$\color{gr} 1.0311\color{black} $\,\,$ & $\,\,$9.1399$\,\,$ & $\,\,$4.1245$\,\,$ \\
$\,\,$\color{gr} 0.9698\color{black} $\,\,$ & $\,\,$ 1 $\,\,$ & $\,\,$\color{gr} 8.8640\color{black} $\,\,$ & $\,\,$\color{gr} \color{blue} 4\color{black}   $\,\,$ \\
$\,\,$0.1094$\,\,$ & $\,\,$\color{gr} 0.1128\color{black} $\,\,$ & $\,\,$ 1 $\,\,$ & $\,\,$0.4513 $\,\,$ \\
$\,\,$0.2425$\,\,$ & $\,\,$\color{gr} \color{blue}  1/4\color{black} $\,\,$ & $\,\,$2.2160$\,\,$ & $\,\,$ 1  $\,\,$ \\
\end{pmatrix},
\end{equation*}
\end{example}
\newpage
\begin{example}
\begin{equation*}
\mathbf{A} =
\begin{pmatrix}
$\,\,$ 1 $\,\,$ & $\,\,$1$\,\,$ & $\,\,$7$\,\,$ & $\,\,$7 $\,\,$ \\
$\,\,$ 1 $\,\,$ & $\,\,$ 1 $\,\,$ & $\,\,$9$\,\,$ & $\,\,$3 $\,\,$ \\
$\,\,$ 1/7$\,\,$ & $\,\,$ 1/9$\,\,$ & $\,\,$ 1 $\,\,$ & $\,\,$ 1/5 $\,\,$ \\
$\,\,$ 1/7$\,\,$ & $\,\,$ 1/3$\,\,$ & $\,\,$5$\,\,$ & $\,\,$ 1  $\,\,$ \\
\end{pmatrix},
\qquad
\lambda_{\max} =
4.2365,
\qquad
CR = 0.0892
\end{equation*}

\begin{equation*}
\mathbf{w}^{cos} =
\begin{pmatrix}
0.439883\\
\color{red} 0.384399\color{black} \\
0.043461\\
0.132258
\end{pmatrix}\end{equation*}
\begin{equation*}
\left[ \frac{{w}^{cos}_i}{{w}^{cos}_j} \right] =
\begin{pmatrix}
$\,\,$ 1 $\,\,$ & $\,\,$\color{red} 1.1443\color{black} $\,\,$ & $\,\,$10.1214$\,\,$ & $\,\,$3.3260$\,\,$ \\
$\,\,$\color{red} 0.8739\color{black} $\,\,$ & $\,\,$ 1 $\,\,$ & $\,\,$\color{red} 8.8448\color{black} $\,\,$ & $\,\,$\color{red} 2.9064\color{black}   $\,\,$ \\
$\,\,$0.0988$\,\,$ & $\,\,$\color{red} 0.1131\color{black} $\,\,$ & $\,\,$ 1 $\,\,$ & $\,\,$0.3286 $\,\,$ \\
$\,\,$0.3007$\,\,$ & $\,\,$\color{red} 0.3441\color{black} $\,\,$ & $\,\,$3.0432$\,\,$ & $\,\,$ 1  $\,\,$ \\
\end{pmatrix},
\end{equation*}

\begin{equation*}
\mathbf{w}^{\prime} =
\begin{pmatrix}
0.436935\\
0.388524\\
0.043169\\
0.131371
\end{pmatrix} =
0.993299\cdot
\begin{pmatrix}
0.439883\\
\color{gr} 0.391145\color{black} \\
0.043461\\
0.132258
\end{pmatrix},
\end{equation*}
\begin{equation*}
\left[ \frac{{w}^{\prime}_i}{{w}^{\prime}_j} \right] =
\begin{pmatrix}
$\,\,$ 1 $\,\,$ & $\,\,$\color{gr} 1.1246\color{black} $\,\,$ & $\,\,$10.1214$\,\,$ & $\,\,$3.3260$\,\,$ \\
$\,\,$\color{gr} 0.8892\color{black} $\,\,$ & $\,\,$ 1 $\,\,$ & $\,\,$\color{gr} \color{blue} 9\color{black} $\,\,$ & $\,\,$\color{gr} 2.9575\color{black}   $\,\,$ \\
$\,\,$0.0988$\,\,$ & $\,\,$\color{gr} \color{blue}  1/9\color{black} $\,\,$ & $\,\,$ 1 $\,\,$ & $\,\,$0.3286 $\,\,$ \\
$\,\,$0.3007$\,\,$ & $\,\,$\color{gr} 0.3381\color{black} $\,\,$ & $\,\,$3.0432$\,\,$ & $\,\,$ 1  $\,\,$ \\
\end{pmatrix},
\end{equation*}
\end{example}
\newpage
\begin{example}
\begin{equation*}
\mathbf{A} =
\begin{pmatrix}
$\,\,$ 1 $\,\,$ & $\,\,$1$\,\,$ & $\,\,$7$\,\,$ & $\,\,$8 $\,\,$ \\
$\,\,$ 1 $\,\,$ & $\,\,$ 1 $\,\,$ & $\,\,$9$\,\,$ & $\,\,$4 $\,\,$ \\
$\,\,$ 1/7$\,\,$ & $\,\,$ 1/9$\,\,$ & $\,\,$ 1 $\,\,$ & $\,\,$ 1/4 $\,\,$ \\
$\,\,$ 1/8$\,\,$ & $\,\,$ 1/4$\,\,$ & $\,\,$4$\,\,$ & $\,\,$ 1  $\,\,$ \\
\end{pmatrix},
\qquad
\lambda_{\max} =
4.2064,
\qquad
CR = 0.0778
\end{equation*}

\begin{equation*}
\mathbf{w}^{cos} =
\begin{pmatrix}
0.444784\\
\color{red} 0.401359\color{black} \\
0.044697\\
0.109160
\end{pmatrix}\end{equation*}
\begin{equation*}
\left[ \frac{{w}^{cos}_i}{{w}^{cos}_j} \right] =
\begin{pmatrix}
$\,\,$ 1 $\,\,$ & $\,\,$\color{red} 1.1082\color{black} $\,\,$ & $\,\,$9.9511$\,\,$ & $\,\,$4.0746$\,\,$ \\
$\,\,$\color{red} 0.9024\color{black} $\,\,$ & $\,\,$ 1 $\,\,$ & $\,\,$\color{red} 8.9796\color{black} $\,\,$ & $\,\,$\color{red} 3.6768\color{black}   $\,\,$ \\
$\,\,$0.1005$\,\,$ & $\,\,$\color{red} 0.1114\color{black} $\,\,$ & $\,\,$ 1 $\,\,$ & $\,\,$0.4095 $\,\,$ \\
$\,\,$0.2454$\,\,$ & $\,\,$\color{red} 0.2720\color{black} $\,\,$ & $\,\,$2.4422$\,\,$ & $\,\,$ 1  $\,\,$ \\
\end{pmatrix},
\end{equation*}

\begin{equation*}
\mathbf{w}^{\prime} =
\begin{pmatrix}
0.444378\\
0.401905\\
0.044656\\
0.109061
\end{pmatrix} =
0.999089\cdot
\begin{pmatrix}
0.444784\\
\color{gr} 0.402271\color{black} \\
0.044697\\
0.109160
\end{pmatrix},
\end{equation*}
\begin{equation*}
\left[ \frac{{w}^{\prime}_i}{{w}^{\prime}_j} \right] =
\begin{pmatrix}
$\,\,$ 1 $\,\,$ & $\,\,$\color{gr} 1.1057\color{black} $\,\,$ & $\,\,$9.9511$\,\,$ & $\,\,$4.0746$\,\,$ \\
$\,\,$\color{gr} 0.9044\color{black} $\,\,$ & $\,\,$ 1 $\,\,$ & $\,\,$\color{gr} \color{blue} 9\color{black} $\,\,$ & $\,\,$\color{gr} 3.6851\color{black}   $\,\,$ \\
$\,\,$0.1005$\,\,$ & $\,\,$\color{gr} \color{blue}  1/9\color{black} $\,\,$ & $\,\,$ 1 $\,\,$ & $\,\,$0.4095 $\,\,$ \\
$\,\,$0.2454$\,\,$ & $\,\,$\color{gr} 0.2714\color{black} $\,\,$ & $\,\,$2.4422$\,\,$ & $\,\,$ 1  $\,\,$ \\
\end{pmatrix},
\end{equation*}
\end{example}
\newpage
\begin{example}
\begin{equation*}
\mathbf{A} =
\begin{pmatrix}
$\,\,$ 1 $\,\,$ & $\,\,$1$\,\,$ & $\,\,$7$\,\,$ & $\,\,$8 $\,\,$ \\
$\,\,$ 1 $\,\,$ & $\,\,$ 1 $\,\,$ & $\,\,$9$\,\,$ & $\,\,$6 $\,\,$ \\
$\,\,$ 1/7$\,\,$ & $\,\,$ 1/9$\,\,$ & $\,\,$ 1 $\,\,$ & $\,\,$ 1/2 $\,\,$ \\
$\,\,$ 1/8$\,\,$ & $\,\,$ 1/6$\,\,$ & $\,\,$2$\,\,$ & $\,\,$ 1  $\,\,$ \\
\end{pmatrix},
\qquad
\lambda_{\max} =
4.0576,
\qquad
CR = 0.0217
\end{equation*}

\begin{equation*}
\mathbf{w}^{cos} =
\begin{pmatrix}
0.440026\\
\color{red} 0.435839\color{black} \\
0.049320\\
0.074815
\end{pmatrix}\end{equation*}
\begin{equation*}
\left[ \frac{{w}^{cos}_i}{{w}^{cos}_j} \right] =
\begin{pmatrix}
$\,\,$ 1 $\,\,$ & $\,\,$\color{red} 1.0096\color{black} $\,\,$ & $\,\,$8.9218$\,\,$ & $\,\,$5.8815$\,\,$ \\
$\,\,$\color{red} 0.9905\color{black} $\,\,$ & $\,\,$ 1 $\,\,$ & $\,\,$\color{red} 8.8369\color{black} $\,\,$ & $\,\,$\color{red} 5.8256\color{black}   $\,\,$ \\
$\,\,$0.1121$\,\,$ & $\,\,$\color{red} 0.1132\color{black} $\,\,$ & $\,\,$ 1 $\,\,$ & $\,\,$0.6592 $\,\,$ \\
$\,\,$0.1700$\,\,$ & $\,\,$\color{red} 0.1717\color{black} $\,\,$ & $\,\,$1.5169$\,\,$ & $\,\,$ 1  $\,\,$ \\
\end{pmatrix},
\end{equation*}

\begin{equation*}
\mathbf{w}^{\prime} =
\begin{pmatrix}
0.438191\\
0.438191\\
0.049115\\
0.074503
\end{pmatrix} =
0.995830\cdot
\begin{pmatrix}
0.440026\\
\color{gr} 0.440026\color{black} \\
0.049320\\
0.074815
\end{pmatrix},
\end{equation*}
\begin{equation*}
\left[ \frac{{w}^{\prime}_i}{{w}^{\prime}_j} \right] =
\begin{pmatrix}
$\,\,$ 1 $\,\,$ & $\,\,$\color{gr} \color{blue} 1\color{black} $\,\,$ & $\,\,$8.9218$\,\,$ & $\,\,$5.8815$\,\,$ \\
$\,\,$\color{gr} \color{blue} 1\color{black} $\,\,$ & $\,\,$ 1 $\,\,$ & $\,\,$\color{gr} 8.9218\color{black} $\,\,$ & $\,\,$\color{gr} 5.8815\color{black}   $\,\,$ \\
$\,\,$0.1121$\,\,$ & $\,\,$\color{gr} 0.1121\color{black} $\,\,$ & $\,\,$ 1 $\,\,$ & $\,\,$0.6592 $\,\,$ \\
$\,\,$0.1700$\,\,$ & $\,\,$\color{gr} 0.1700\color{black} $\,\,$ & $\,\,$1.5169$\,\,$ & $\,\,$ 1  $\,\,$ \\
\end{pmatrix},
\end{equation*}
\end{example}
\newpage
\begin{example}
\begin{equation*}
\mathbf{A} =
\begin{pmatrix}
$\,\,$ 1 $\,\,$ & $\,\,$1$\,\,$ & $\,\,$7$\,\,$ & $\,\,$9 $\,\,$ \\
$\,\,$ 1 $\,\,$ & $\,\,$ 1 $\,\,$ & $\,\,$9$\,\,$ & $\,\,$4 $\,\,$ \\
$\,\,$ 1/7$\,\,$ & $\,\,$ 1/9$\,\,$ & $\,\,$ 1 $\,\,$ & $\,\,$ 1/4 $\,\,$ \\
$\,\,$ 1/9$\,\,$ & $\,\,$ 1/4$\,\,$ & $\,\,$4$\,\,$ & $\,\,$ 1  $\,\,$ \\
\end{pmatrix},
\qquad
\lambda_{\max} =
4.2434,
\qquad
CR = 0.0918
\end{equation*}

\begin{equation*}
\mathbf{w}^{cos} =
\begin{pmatrix}
0.450909\\
\color{red} 0.397717\color{black} \\
0.044602\\
0.106771
\end{pmatrix}\end{equation*}
\begin{equation*}
\left[ \frac{{w}^{cos}_i}{{w}^{cos}_j} \right] =
\begin{pmatrix}
$\,\,$ 1 $\,\,$ & $\,\,$\color{red} 1.1337\color{black} $\,\,$ & $\,\,$10.1095$\,\,$ & $\,\,$4.2231$\,\,$ \\
$\,\,$\color{red} 0.8820\color{black} $\,\,$ & $\,\,$ 1 $\,\,$ & $\,\,$\color{red} 8.9169\color{black} $\,\,$ & $\,\,$\color{red} 3.7249\color{black}   $\,\,$ \\
$\,\,$0.0989$\,\,$ & $\,\,$\color{red} 0.1121\color{black} $\,\,$ & $\,\,$ 1 $\,\,$ & $\,\,$0.4177 $\,\,$ \\
$\,\,$0.2368$\,\,$ & $\,\,$\color{red} 0.2685\color{black} $\,\,$ & $\,\,$2.3938$\,\,$ & $\,\,$ 1  $\,\,$ \\
\end{pmatrix},
\end{equation*}

\begin{equation*}
\mathbf{w}^{\prime} =
\begin{pmatrix}
0.449245\\
0.399940\\
0.044438\\
0.106377
\end{pmatrix} =
0.996309\cdot
\begin{pmatrix}
0.450909\\
\color{gr} 0.401422\color{black} \\
0.044602\\
0.106771
\end{pmatrix},
\end{equation*}
\begin{equation*}
\left[ \frac{{w}^{\prime}_i}{{w}^{\prime}_j} \right] =
\begin{pmatrix}
$\,\,$ 1 $\,\,$ & $\,\,$\color{gr} 1.1233\color{black} $\,\,$ & $\,\,$10.1095$\,\,$ & $\,\,$4.2231$\,\,$ \\
$\,\,$\color{gr} 0.8902\color{black} $\,\,$ & $\,\,$ 1 $\,\,$ & $\,\,$\color{gr} \color{blue} 9\color{black} $\,\,$ & $\,\,$\color{gr} 3.7596\color{black}   $\,\,$ \\
$\,\,$0.0989$\,\,$ & $\,\,$\color{gr} \color{blue}  1/9\color{black} $\,\,$ & $\,\,$ 1 $\,\,$ & $\,\,$0.4177 $\,\,$ \\
$\,\,$0.2368$\,\,$ & $\,\,$\color{gr} 0.2660\color{black} $\,\,$ & $\,\,$2.3938$\,\,$ & $\,\,$ 1  $\,\,$ \\
\end{pmatrix},
\end{equation*}
\end{example}
\newpage
\begin{example}
\begin{equation*}
\mathbf{A} =
\begin{pmatrix}
$\,\,$ 1 $\,\,$ & $\,\,$1$\,\,$ & $\,\,$7$\,\,$ & $\,\,$9 $\,\,$ \\
$\,\,$ 1 $\,\,$ & $\,\,$ 1 $\,\,$ & $\,\,$9$\,\,$ & $\,\,$5 $\,\,$ \\
$\,\,$ 1/7$\,\,$ & $\,\,$ 1/9$\,\,$ & $\,\,$ 1 $\,\,$ & $\,\,$ 1/3 $\,\,$ \\
$\,\,$ 1/9$\,\,$ & $\,\,$ 1/5$\,\,$ & $\,\,$3$\,\,$ & $\,\,$ 1  $\,\,$ \\
\end{pmatrix},
\qquad
\lambda_{\max} =
4.1603,
\qquad
CR = 0.0605
\end{equation*}

\begin{equation*}
\mathbf{w}^{cos} =
\begin{pmatrix}
0.449362\\
\color{red} 0.415202\color{black} \\
0.046258\\
0.089178
\end{pmatrix}\end{equation*}
\begin{equation*}
\left[ \frac{{w}^{cos}_i}{{w}^{cos}_j} \right] =
\begin{pmatrix}
$\,\,$ 1 $\,\,$ & $\,\,$\color{red} 1.0823\color{black} $\,\,$ & $\,\,$9.7142$\,\,$ & $\,\,$5.0389$\,\,$ \\
$\,\,$\color{red} 0.9240\color{black} $\,\,$ & $\,\,$ 1 $\,\,$ & $\,\,$\color{red} 8.9757\color{black} $\,\,$ & $\,\,$\color{red} 4.6559\color{black}   $\,\,$ \\
$\,\,$0.1029$\,\,$ & $\,\,$\color{red} 0.1114\color{black} $\,\,$ & $\,\,$ 1 $\,\,$ & $\,\,$0.5187 $\,\,$ \\
$\,\,$0.1985$\,\,$ & $\,\,$\color{red} 0.2148\color{black} $\,\,$ & $\,\,$1.9278$\,\,$ & $\,\,$ 1  $\,\,$ \\
\end{pmatrix},
\end{equation*}

\begin{equation*}
\mathbf{w}^{\prime} =
\begin{pmatrix}
0.448858\\
0.415858\\
0.046206\\
0.089078
\end{pmatrix} =
0.998878\cdot
\begin{pmatrix}
0.449362\\
\color{gr} 0.416325\color{black} \\
0.046258\\
0.089178
\end{pmatrix},
\end{equation*}
\begin{equation*}
\left[ \frac{{w}^{\prime}_i}{{w}^{\prime}_j} \right] =
\begin{pmatrix}
$\,\,$ 1 $\,\,$ & $\,\,$\color{gr} 1.0794\color{black} $\,\,$ & $\,\,$9.7142$\,\,$ & $\,\,$5.0389$\,\,$ \\
$\,\,$\color{gr} 0.9265\color{black} $\,\,$ & $\,\,$ 1 $\,\,$ & $\,\,$\color{gr} \color{blue} 9\color{black} $\,\,$ & $\,\,$\color{gr} 4.6685\color{black}   $\,\,$ \\
$\,\,$0.1029$\,\,$ & $\,\,$\color{gr} \color{blue}  1/9\color{black} $\,\,$ & $\,\,$ 1 $\,\,$ & $\,\,$0.5187 $\,\,$ \\
$\,\,$0.1985$\,\,$ & $\,\,$\color{gr} 0.2142\color{black} $\,\,$ & $\,\,$1.9278$\,\,$ & $\,\,$ 1  $\,\,$ \\
\end{pmatrix},
\end{equation*}
\end{example}
\newpage
\begin{example}
\begin{equation*}
\mathbf{A} =
\begin{pmatrix}
$\,\,$ 1 $\,\,$ & $\,\,$1$\,\,$ & $\,\,$7$\,\,$ & $\,\,$9 $\,\,$ \\
$\,\,$ 1 $\,\,$ & $\,\,$ 1 $\,\,$ & $\,\,$9$\,\,$ & $\,\,$6 $\,\,$ \\
$\,\,$ 1/7$\,\,$ & $\,\,$ 1/9$\,\,$ & $\,\,$ 1 $\,\,$ & $\,\,$ 1/2 $\,\,$ \\
$\,\,$ 1/9$\,\,$ & $\,\,$ 1/6$\,\,$ & $\,\,$2$\,\,$ & $\,\,$ 1  $\,\,$ \\
\end{pmatrix},
\qquad
\lambda_{\max} =
4.0762,
\qquad
CR = 0.0287
\end{equation*}

\begin{equation*}
\mathbf{w}^{cos} =
\begin{pmatrix}
0.447458\\
\color{red} 0.431063\color{black} \\
0.048993\\
0.072487
\end{pmatrix}\end{equation*}
\begin{equation*}
\left[ \frac{{w}^{cos}_i}{{w}^{cos}_j} \right] =
\begin{pmatrix}
$\,\,$ 1 $\,\,$ & $\,\,$\color{red} 1.0380\color{black} $\,\,$ & $\,\,$9.1332$\,\,$ & $\,\,$6.1730$\,\,$ \\
$\,\,$\color{red} 0.9634\color{black} $\,\,$ & $\,\,$ 1 $\,\,$ & $\,\,$\color{red} 8.7986\color{black} $\,\,$ & $\,\,$\color{red} 5.9468\color{black}   $\,\,$ \\
$\,\,$0.1095$\,\,$ & $\,\,$\color{red} 0.1137\color{black} $\,\,$ & $\,\,$ 1 $\,\,$ & $\,\,$0.6759 $\,\,$ \\
$\,\,$0.1620$\,\,$ & $\,\,$\color{red} 0.1682\color{black} $\,\,$ & $\,\,$1.4795$\,\,$ & $\,\,$ 1  $\,\,$ \\
\end{pmatrix},
\end{equation*}

\begin{equation*}
\mathbf{w}^{\prime} =
\begin{pmatrix}
0.445739\\
0.433249\\
0.048804\\
0.072208
\end{pmatrix} =
0.996159\cdot
\begin{pmatrix}
0.447458\\
\color{gr} 0.434919\color{black} \\
0.048993\\
0.072487
\end{pmatrix},
\end{equation*}
\begin{equation*}
\left[ \frac{{w}^{\prime}_i}{{w}^{\prime}_j} \right] =
\begin{pmatrix}
$\,\,$ 1 $\,\,$ & $\,\,$\color{gr} 1.0288\color{black} $\,\,$ & $\,\,$9.1332$\,\,$ & $\,\,$6.1730$\,\,$ \\
$\,\,$\color{gr} 0.9720\color{black} $\,\,$ & $\,\,$ 1 $\,\,$ & $\,\,$\color{gr} 8.8773\color{black} $\,\,$ & $\,\,$\color{gr} \color{blue} 6\color{black}   $\,\,$ \\
$\,\,$0.1095$\,\,$ & $\,\,$\color{gr} 0.1126\color{black} $\,\,$ & $\,\,$ 1 $\,\,$ & $\,\,$0.6759 $\,\,$ \\
$\,\,$0.1620$\,\,$ & $\,\,$\color{gr} \color{blue}  1/6\color{black} $\,\,$ & $\,\,$1.4795$\,\,$ & $\,\,$ 1  $\,\,$ \\
\end{pmatrix},
\end{equation*}
\end{example}
\newpage
\begin{example}
\begin{equation*}
\mathbf{A} =
\begin{pmatrix}
$\,\,$ 1 $\,\,$ & $\,\,$1$\,\,$ & $\,\,$8$\,\,$ & $\,\,$6 $\,\,$ \\
$\,\,$ 1 $\,\,$ & $\,\,$ 1 $\,\,$ & $\,\,$6$\,\,$ & $\,\,$2 $\,\,$ \\
$\,\,$ 1/8$\,\,$ & $\,\,$ 1/6$\,\,$ & $\,\,$ 1 $\,\,$ & $\,\,$ 1/2 $\,\,$ \\
$\,\,$ 1/6$\,\,$ & $\,\,$ 1/2$\,\,$ & $\,\,$2$\,\,$ & $\,\,$ 1  $\,\,$ \\
\end{pmatrix},
\qquad
\lambda_{\max} =
4.1031,
\qquad
CR = 0.0389
\end{equation*}

\begin{equation*}
\mathbf{w}^{cos} =
\begin{pmatrix}
0.472796\\
0.347112\\
\color{red} 0.057361\color{black} \\
0.122732
\end{pmatrix}\end{equation*}
\begin{equation*}
\left[ \frac{{w}^{cos}_i}{{w}^{cos}_j} \right] =
\begin{pmatrix}
$\,\,$ 1 $\,\,$ & $\,\,$1.3621$\,\,$ & $\,\,$\color{red} 8.2425\color{black} $\,\,$ & $\,\,$3.8523$\,\,$ \\
$\,\,$0.7342$\,\,$ & $\,\,$ 1 $\,\,$ & $\,\,$\color{red} 6.0514\color{black} $\,\,$ & $\,\,$2.8282  $\,\,$ \\
$\,\,$\color{red} 0.1213\color{black} $\,\,$ & $\,\,$\color{red} 0.1653\color{black} $\,\,$ & $\,\,$ 1 $\,\,$ & $\,\,$\color{red} 0.4674\color{black}  $\,\,$ \\
$\,\,$0.2596$\,\,$ & $\,\,$0.3536$\,\,$ & $\,\,$\color{red} 2.1397\color{black} $\,\,$ & $\,\,$ 1  $\,\,$ \\
\end{pmatrix},
\end{equation*}

\begin{equation*}
\mathbf{w}^{\prime} =
\begin{pmatrix}
0.472563\\
0.346941\\
0.057824\\
0.122672
\end{pmatrix} =
0.999509\cdot
\begin{pmatrix}
0.472796\\
0.347112\\
\color{gr} 0.057852\color{black} \\
0.122732
\end{pmatrix},
\end{equation*}
\begin{equation*}
\left[ \frac{{w}^{\prime}_i}{{w}^{\prime}_j} \right] =
\begin{pmatrix}
$\,\,$ 1 $\,\,$ & $\,\,$1.3621$\,\,$ & $\,\,$\color{gr} 8.1725\color{black} $\,\,$ & $\,\,$3.8523$\,\,$ \\
$\,\,$0.7342$\,\,$ & $\,\,$ 1 $\,\,$ & $\,\,$\color{gr} \color{blue} 6\color{black} $\,\,$ & $\,\,$2.8282  $\,\,$ \\
$\,\,$\color{gr} 0.1224\color{black} $\,\,$ & $\,\,$\color{gr} \color{blue}  1/6\color{black} $\,\,$ & $\,\,$ 1 $\,\,$ & $\,\,$\color{gr} 0.4714\color{black}  $\,\,$ \\
$\,\,$0.2596$\,\,$ & $\,\,$0.3536$\,\,$ & $\,\,$\color{gr} 2.1215\color{black} $\,\,$ & $\,\,$ 1  $\,\,$ \\
\end{pmatrix},
\end{equation*}
\end{example}
\newpage
\begin{example}
\begin{equation*}
\mathbf{A} =
\begin{pmatrix}
$\,\,$ 1 $\,\,$ & $\,\,$1$\,\,$ & $\,\,$8$\,\,$ & $\,\,$7 $\,\,$ \\
$\,\,$ 1 $\,\,$ & $\,\,$ 1 $\,\,$ & $\,\,$6$\,\,$ & $\,\,$2 $\,\,$ \\
$\,\,$ 1/8$\,\,$ & $\,\,$ 1/6$\,\,$ & $\,\,$ 1 $\,\,$ & $\,\,$ 1/2 $\,\,$ \\
$\,\,$ 1/7$\,\,$ & $\,\,$ 1/2$\,\,$ & $\,\,$2$\,\,$ & $\,\,$ 1  $\,\,$ \\
\end{pmatrix},
\qquad
\lambda_{\max} =
4.1365,
\qquad
CR = 0.0515
\end{equation*}

\begin{equation*}
\mathbf{w}^{cos} =
\begin{pmatrix}
0.480449\\
0.344750\\
\color{red} 0.056438\color{black} \\
0.118363
\end{pmatrix}\end{equation*}
\begin{equation*}
\left[ \frac{{w}^{cos}_i}{{w}^{cos}_j} \right] =
\begin{pmatrix}
$\,\,$ 1 $\,\,$ & $\,\,$1.3936$\,\,$ & $\,\,$\color{red} 8.5129\color{black} $\,\,$ & $\,\,$4.0591$\,\,$ \\
$\,\,$0.7176$\,\,$ & $\,\,$ 1 $\,\,$ & $\,\,$\color{red} 6.1085\color{black} $\,\,$ & $\,\,$2.9126  $\,\,$ \\
$\,\,$\color{red} 0.1175\color{black} $\,\,$ & $\,\,$\color{red} 0.1637\color{black} $\,\,$ & $\,\,$ 1 $\,\,$ & $\,\,$\color{red} 0.4768\color{black}  $\,\,$ \\
$\,\,$0.2464$\,\,$ & $\,\,$0.3433$\,\,$ & $\,\,$\color{red} 2.0972\color{black} $\,\,$ & $\,\,$ 1  $\,\,$ \\
\end{pmatrix},
\end{equation*}

\begin{equation*}
\mathbf{w}^{\prime} =
\begin{pmatrix}
0.479960\\
0.344398\\
0.057400\\
0.118242
\end{pmatrix} =
0.998981\cdot
\begin{pmatrix}
0.480449\\
0.344750\\
\color{gr} 0.057458\color{black} \\
0.118363
\end{pmatrix},
\end{equation*}
\begin{equation*}
\left[ \frac{{w}^{\prime}_i}{{w}^{\prime}_j} \right] =
\begin{pmatrix}
$\,\,$ 1 $\,\,$ & $\,\,$1.3936$\,\,$ & $\,\,$\color{gr} 8.3617\color{black} $\,\,$ & $\,\,$4.0591$\,\,$ \\
$\,\,$0.7176$\,\,$ & $\,\,$ 1 $\,\,$ & $\,\,$\color{gr} \color{blue} 6\color{black} $\,\,$ & $\,\,$2.9126  $\,\,$ \\
$\,\,$\color{gr} 0.1196\color{black} $\,\,$ & $\,\,$\color{gr} \color{blue}  1/6\color{black} $\,\,$ & $\,\,$ 1 $\,\,$ & $\,\,$\color{gr} 0.4854\color{black}  $\,\,$ \\
$\,\,$0.2464$\,\,$ & $\,\,$0.3433$\,\,$ & $\,\,$\color{gr} 2.0600\color{black} $\,\,$ & $\,\,$ 1  $\,\,$ \\
\end{pmatrix},
\end{equation*}
\end{example}
\newpage
\begin{example}
\begin{equation*}
\mathbf{A} =
\begin{pmatrix}
$\,\,$ 1 $\,\,$ & $\,\,$1$\,\,$ & $\,\,$8$\,\,$ & $\,\,$8 $\,\,$ \\
$\,\,$ 1 $\,\,$ & $\,\,$ 1 $\,\,$ & $\,\,$6$\,\,$ & $\,\,$2 $\,\,$ \\
$\,\,$ 1/8$\,\,$ & $\,\,$ 1/6$\,\,$ & $\,\,$ 1 $\,\,$ & $\,\,$ 1/2 $\,\,$ \\
$\,\,$ 1/8$\,\,$ & $\,\,$ 1/2$\,\,$ & $\,\,$2$\,\,$ & $\,\,$ 1  $\,\,$ \\
\end{pmatrix},
\qquad
\lambda_{\max} =
4.1707,
\qquad
CR = 0.0644
\end{equation*}

\begin{equation*}
\mathbf{w}^{cos} =
\begin{pmatrix}
0.486380\\
0.342924\\
\color{red} 0.055712\color{black} \\
0.114983
\end{pmatrix}\end{equation*}
\begin{equation*}
\left[ \frac{{w}^{cos}_i}{{w}^{cos}_j} \right] =
\begin{pmatrix}
$\,\,$ 1 $\,\,$ & $\,\,$1.4183$\,\,$ & $\,\,$\color{red} 8.7302\color{black} $\,\,$ & $\,\,$4.2300$\,\,$ \\
$\,\,$0.7051$\,\,$ & $\,\,$ 1 $\,\,$ & $\,\,$\color{red} 6.1553\color{black} $\,\,$ & $\,\,$2.9824  $\,\,$ \\
$\,\,$\color{red} 0.1145\color{black} $\,\,$ & $\,\,$\color{red} 0.1625\color{black} $\,\,$ & $\,\,$ 1 $\,\,$ & $\,\,$\color{red} 0.4845\color{black}  $\,\,$ \\
$\,\,$0.2364$\,\,$ & $\,\,$0.3353$\,\,$ & $\,\,$\color{red} 2.0639\color{black} $\,\,$ & $\,\,$ 1  $\,\,$ \\
\end{pmatrix},
\end{equation*}

\begin{equation*}
\mathbf{w}^{\prime} =
\begin{pmatrix}
0.485680\\
0.342431\\
0.057072\\
0.114818
\end{pmatrix} =
0.998560\cdot
\begin{pmatrix}
0.486380\\
0.342924\\
\color{gr} 0.057154\color{black} \\
0.114983
\end{pmatrix},
\end{equation*}
\begin{equation*}
\left[ \frac{{w}^{\prime}_i}{{w}^{\prime}_j} \right] =
\begin{pmatrix}
$\,\,$ 1 $\,\,$ & $\,\,$1.4183$\,\,$ & $\,\,$\color{gr} 8.5100\color{black} $\,\,$ & $\,\,$4.2300$\,\,$ \\
$\,\,$0.7051$\,\,$ & $\,\,$ 1 $\,\,$ & $\,\,$\color{gr} \color{blue} 6\color{black} $\,\,$ & $\,\,$2.9824  $\,\,$ \\
$\,\,$\color{gr} 0.1175\color{black} $\,\,$ & $\,\,$\color{gr} \color{blue}  1/6\color{black} $\,\,$ & $\,\,$ 1 $\,\,$ & $\,\,$\color{gr} 0.4971\color{black}  $\,\,$ \\
$\,\,$0.2364$\,\,$ & $\,\,$0.3353$\,\,$ & $\,\,$\color{gr} 2.0118\color{black} $\,\,$ & $\,\,$ 1  $\,\,$ \\
\end{pmatrix},
\end{equation*}
\end{example}
\newpage
\begin{example}
\begin{equation*}
\mathbf{A} =
\begin{pmatrix}
$\,\,$ 1 $\,\,$ & $\,\,$1$\,\,$ & $\,\,$8$\,\,$ & $\,\,$9 $\,\,$ \\
$\,\,$ 1 $\,\,$ & $\,\,$ 1 $\,\,$ & $\,\,$6$\,\,$ & $\,\,$2 $\,\,$ \\
$\,\,$ 1/8$\,\,$ & $\,\,$ 1/6$\,\,$ & $\,\,$ 1 $\,\,$ & $\,\,$ 1/2 $\,\,$ \\
$\,\,$ 1/9$\,\,$ & $\,\,$ 1/2$\,\,$ & $\,\,$2$\,\,$ & $\,\,$ 1  $\,\,$ \\
\end{pmatrix},
\qquad
\lambda_{\max} =
4.2052,
\qquad
CR = 0.0774
\end{equation*}

\begin{equation*}
\mathbf{w}^{cos} =
\begin{pmatrix}
0.491099\\
0.341478\\
\color{red} 0.055129\color{black} \\
0.112293
\end{pmatrix}\end{equation*}
\begin{equation*}
\left[ \frac{{w}^{cos}_i}{{w}^{cos}_j} \right] =
\begin{pmatrix}
$\,\,$ 1 $\,\,$ & $\,\,$1.4382$\,\,$ & $\,\,$\color{red} 8.9081\color{black} $\,\,$ & $\,\,$4.3734$\,\,$ \\
$\,\,$0.6953$\,\,$ & $\,\,$ 1 $\,\,$ & $\,\,$\color{red} 6.1941\color{black} $\,\,$ & $\,\,$3.0410  $\,\,$ \\
$\,\,$\color{red} 0.1123\color{black} $\,\,$ & $\,\,$\color{red} 0.1614\color{black} $\,\,$ & $\,\,$ 1 $\,\,$ & $\,\,$\color{red} 0.4909\color{black}  $\,\,$ \\
$\,\,$0.2287$\,\,$ & $\,\,$0.3288$\,\,$ & $\,\,$\color{red} 2.0369\color{black} $\,\,$ & $\,\,$ 1  $\,\,$ \\
\end{pmatrix},
\end{equation*}

\begin{equation*}
\mathbf{w}^{\prime} =
\begin{pmatrix}
0.490600\\
0.341131\\
0.056089\\
0.112179
\end{pmatrix} =
0.998984\cdot
\begin{pmatrix}
0.491099\\
0.341478\\
\color{gr} 0.056146\color{black} \\
0.112293
\end{pmatrix},
\end{equation*}
\begin{equation*}
\left[ \frac{{w}^{\prime}_i}{{w}^{\prime}_j} \right] =
\begin{pmatrix}
$\,\,$ 1 $\,\,$ & $\,\,$1.4382$\,\,$ & $\,\,$\color{gr} 8.7468\color{black} $\,\,$ & $\,\,$4.3734$\,\,$ \\
$\,\,$0.6953$\,\,$ & $\,\,$ 1 $\,\,$ & $\,\,$\color{gr} 6.0819\color{black} $\,\,$ & $\,\,$3.0410  $\,\,$ \\
$\,\,$\color{gr} 0.1143\color{black} $\,\,$ & $\,\,$\color{gr} 0.1644\color{black} $\,\,$ & $\,\,$ 1 $\,\,$ & $\,\,$\color{gr} \color{blue}  1/2\color{black}  $\,\,$ \\
$\,\,$0.2287$\,\,$ & $\,\,$0.3288$\,\,$ & $\,\,$\color{gr} \color{blue} 2\color{black} $\,\,$ & $\,\,$ 1  $\,\,$ \\
\end{pmatrix},
\end{equation*}
\end{example}
\newpage
\begin{example}
\begin{equation*}
\mathbf{A} =
\begin{pmatrix}
$\,\,$ 1 $\,\,$ & $\,\,$1$\,\,$ & $\,\,$9$\,\,$ & $\,\,$7 $\,\,$ \\
$\,\,$ 1 $\,\,$ & $\,\,$ 1 $\,\,$ & $\,\,$6$\,\,$ & $\,\,$2 $\,\,$ \\
$\,\,$ 1/9$\,\,$ & $\,\,$ 1/6$\,\,$ & $\,\,$ 1 $\,\,$ & $\,\,$ 1/2 $\,\,$ \\
$\,\,$ 1/7$\,\,$ & $\,\,$ 1/2$\,\,$ & $\,\,$2$\,\,$ & $\,\,$ 1  $\,\,$ \\
\end{pmatrix},
\qquad
\lambda_{\max} =
4.1342,
\qquad
CR = 0.0506
\end{equation*}

\begin{equation*}
\mathbf{w}^{cos} =
\begin{pmatrix}
0.488703\\
0.340265\\
\color{red} 0.054159\color{black} \\
0.116872
\end{pmatrix}\end{equation*}
\begin{equation*}
\left[ \frac{{w}^{cos}_i}{{w}^{cos}_j} \right] =
\begin{pmatrix}
$\,\,$ 1 $\,\,$ & $\,\,$1.4362$\,\,$ & $\,\,$\color{red} 9.0234\color{black} $\,\,$ & $\,\,$4.1815$\,\,$ \\
$\,\,$0.6963$\,\,$ & $\,\,$ 1 $\,\,$ & $\,\,$\color{red} 6.2827\color{black} $\,\,$ & $\,\,$2.9114  $\,\,$ \\
$\,\,$\color{red} 0.1108\color{black} $\,\,$ & $\,\,$\color{red} 0.1592\color{black} $\,\,$ & $\,\,$ 1 $\,\,$ & $\,\,$\color{red} 0.4634\color{black}  $\,\,$ \\
$\,\,$0.2391$\,\,$ & $\,\,$0.3435$\,\,$ & $\,\,$\color{red} 2.1579\color{black} $\,\,$ & $\,\,$ 1  $\,\,$ \\
\end{pmatrix},
\end{equation*}

\begin{equation*}
\mathbf{w}^{\prime} =
\begin{pmatrix}
0.488634\\
0.340217\\
0.054293\\
0.116855
\end{pmatrix} =
0.999859\cdot
\begin{pmatrix}
0.488703\\
0.340265\\
\color{gr} 0.054300\color{black} \\
0.116872
\end{pmatrix},
\end{equation*}
\begin{equation*}
\left[ \frac{{w}^{\prime}_i}{{w}^{\prime}_j} \right] =
\begin{pmatrix}
$\,\,$ 1 $\,\,$ & $\,\,$1.4362$\,\,$ & $\,\,$\color{gr} \color{blue} 9\color{black} $\,\,$ & $\,\,$4.1815$\,\,$ \\
$\,\,$0.6963$\,\,$ & $\,\,$ 1 $\,\,$ & $\,\,$\color{gr} 6.2664\color{black} $\,\,$ & $\,\,$2.9114  $\,\,$ \\
$\,\,$\color{gr} \color{blue}  1/9\color{black} $\,\,$ & $\,\,$\color{gr} 0.1596\color{black} $\,\,$ & $\,\,$ 1 $\,\,$ & $\,\,$\color{gr} 0.4646\color{black}  $\,\,$ \\
$\,\,$0.2391$\,\,$ & $\,\,$0.3435$\,\,$ & $\,\,$\color{gr} 2.1523\color{black} $\,\,$ & $\,\,$ 1  $\,\,$ \\
\end{pmatrix},
\end{equation*}
\end{example}
\newpage
\begin{example}
\begin{equation*}
\mathbf{A} =
\begin{pmatrix}
$\,\,$ 1 $\,\,$ & $\,\,$1$\,\,$ & $\,\,$9$\,\,$ & $\,\,$8 $\,\,$ \\
$\,\,$ 1 $\,\,$ & $\,\,$ 1 $\,\,$ & $\,\,$6$\,\,$ & $\,\,$2 $\,\,$ \\
$\,\,$ 1/9$\,\,$ & $\,\,$ 1/6$\,\,$ & $\,\,$ 1 $\,\,$ & $\,\,$ 1/2 $\,\,$ \\
$\,\,$ 1/8$\,\,$ & $\,\,$ 1/2$\,\,$ & $\,\,$2$\,\,$ & $\,\,$ 1  $\,\,$ \\
\end{pmatrix},
\qquad
\lambda_{\max} =
4.1664,
\qquad
CR = 0.0627
\end{equation*}

\begin{equation*}
\mathbf{w}^{cos} =
\begin{pmatrix}
0.494742\\
0.338392\\
\color{red} 0.053409\color{black} \\
0.113458
\end{pmatrix}\end{equation*}
\begin{equation*}
\left[ \frac{{w}^{cos}_i}{{w}^{cos}_j} \right] =
\begin{pmatrix}
$\,\,$ 1 $\,\,$ & $\,\,$1.4620$\,\,$ & $\,\,$\color{red} 9.2633\color{black} $\,\,$ & $\,\,$4.3606$\,\,$ \\
$\,\,$0.6840$\,\,$ & $\,\,$ 1 $\,\,$ & $\,\,$\color{red} 6.3359\color{black} $\,\,$ & $\,\,$2.9825  $\,\,$ \\
$\,\,$\color{red} 0.1080\color{black} $\,\,$ & $\,\,$\color{red} 0.1578\color{black} $\,\,$ & $\,\,$ 1 $\,\,$ & $\,\,$\color{red} 0.4707\color{black}  $\,\,$ \\
$\,\,$0.2293$\,\,$ & $\,\,$0.3353$\,\,$ & $\,\,$\color{red} 2.1243\color{black} $\,\,$ & $\,\,$ 1  $\,\,$ \\
\end{pmatrix},
\end{equation*}

\begin{equation*}
\mathbf{w}^{\prime} =
\begin{pmatrix}
0.493970\\
0.337864\\
0.054886\\
0.113281
\end{pmatrix} =
0.998440\cdot
\begin{pmatrix}
0.494742\\
0.338392\\
\color{gr} 0.054971\color{black} \\
0.113458
\end{pmatrix},
\end{equation*}
\begin{equation*}
\left[ \frac{{w}^{\prime}_i}{{w}^{\prime}_j} \right] =
\begin{pmatrix}
$\,\,$ 1 $\,\,$ & $\,\,$1.4620$\,\,$ & $\,\,$\color{gr} \color{blue} 9\color{black} $\,\,$ & $\,\,$4.3606$\,\,$ \\
$\,\,$0.6840$\,\,$ & $\,\,$ 1 $\,\,$ & $\,\,$\color{gr} 6.1558\color{black} $\,\,$ & $\,\,$2.9825  $\,\,$ \\
$\,\,$\color{gr} \color{blue}  1/9\color{black} $\,\,$ & $\,\,$\color{gr} 0.1624\color{black} $\,\,$ & $\,\,$ 1 $\,\,$ & $\,\,$\color{gr} 0.4845\color{black}  $\,\,$ \\
$\,\,$0.2293$\,\,$ & $\,\,$0.3353$\,\,$ & $\,\,$\color{gr} 2.0639\color{black} $\,\,$ & $\,\,$ 1  $\,\,$ \\
\end{pmatrix},
\end{equation*}
\end{example}
\newpage
\begin{example}
\begin{equation*}
\mathbf{A} =
\begin{pmatrix}
$\,\,$ 1 $\,\,$ & $\,\,$1$\,\,$ & $\,\,$9$\,\,$ & $\,\,$9 $\,\,$ \\
$\,\,$ 1 $\,\,$ & $\,\,$ 1 $\,\,$ & $\,\,$6$\,\,$ & $\,\,$2 $\,\,$ \\
$\,\,$ 1/9$\,\,$ & $\,\,$ 1/6$\,\,$ & $\,\,$ 1 $\,\,$ & $\,\,$ 1/2 $\,\,$ \\
$\,\,$ 1/9$\,\,$ & $\,\,$ 1/2$\,\,$ & $\,\,$2$\,\,$ & $\,\,$ 1  $\,\,$ \\
\end{pmatrix},
\qquad
\lambda_{\max} =
4.1990,
\qquad
CR = 0.0750
\end{equation*}

\begin{equation*}
\mathbf{w}^{cos} =
\begin{pmatrix}
0.499550\\
0.336906\\
\color{red} 0.052805\color{black} \\
0.110740
\end{pmatrix}\end{equation*}
\begin{equation*}
\left[ \frac{{w}^{cos}_i}{{w}^{cos}_j} \right] =
\begin{pmatrix}
$\,\,$ 1 $\,\,$ & $\,\,$1.4828$\,\,$ & $\,\,$\color{red} 9.4604\color{black} $\,\,$ & $\,\,$4.5110$\,\,$ \\
$\,\,$0.6744$\,\,$ & $\,\,$ 1 $\,\,$ & $\,\,$\color{red} 6.3803\color{black} $\,\,$ & $\,\,$3.0423  $\,\,$ \\
$\,\,$\color{red} 0.1057\color{black} $\,\,$ & $\,\,$\color{red} 0.1567\color{black} $\,\,$ & $\,\,$ 1 $\,\,$ & $\,\,$\color{red} 0.4768\color{black}  $\,\,$ \\
$\,\,$0.2217$\,\,$ & $\,\,$0.3287$\,\,$ & $\,\,$\color{red} 2.0972\color{black} $\,\,$ & $\,\,$ 1  $\,\,$ \\
\end{pmatrix},
\end{equation*}

\begin{equation*}
\mathbf{w}^{\prime} =
\begin{pmatrix}
0.498271\\
0.336044\\
0.055228\\
0.110456
\end{pmatrix} =
0.997441\cdot
\begin{pmatrix}
0.499550\\
0.336906\\
\color{gr} 0.055370\color{black} \\
0.110740
\end{pmatrix},
\end{equation*}
\begin{equation*}
\left[ \frac{{w}^{\prime}_i}{{w}^{\prime}_j} \right] =
\begin{pmatrix}
$\,\,$ 1 $\,\,$ & $\,\,$1.4828$\,\,$ & $\,\,$\color{gr} 9.0221\color{black} $\,\,$ & $\,\,$4.5110$\,\,$ \\
$\,\,$0.6744$\,\,$ & $\,\,$ 1 $\,\,$ & $\,\,$\color{gr} 6.0847\color{black} $\,\,$ & $\,\,$3.0423  $\,\,$ \\
$\,\,$\color{gr} 0.1108\color{black} $\,\,$ & $\,\,$\color{gr} 0.1643\color{black} $\,\,$ & $\,\,$ 1 $\,\,$ & $\,\,$\color{gr} \color{blue}  1/2\color{black}  $\,\,$ \\
$\,\,$0.2217$\,\,$ & $\,\,$0.3287$\,\,$ & $\,\,$\color{gr} \color{blue} 2\color{black} $\,\,$ & $\,\,$ 1  $\,\,$ \\
\end{pmatrix},
\end{equation*}
\end{example}
\newpage
\begin{example}
\begin{equation*}
\mathbf{A} =
\begin{pmatrix}
$\,\,$ 1 $\,\,$ & $\,\,$2$\,\,$ & $\,\,$2$\,\,$ & $\,\,$1 $\,\,$ \\
$\,\,$ 1/2$\,\,$ & $\,\,$ 1 $\,\,$ & $\,\,$5$\,\,$ & $\,\,$1 $\,\,$ \\
$\,\,$ 1/2$\,\,$ & $\,\,$ 1/5$\,\,$ & $\,\,$ 1 $\,\,$ & $\,\,$ 1/3 $\,\,$ \\
$\,\,$ 1 $\,\,$ & $\,\,$ 1 $\,\,$ & $\,\,$3$\,\,$ & $\,\,$ 1  $\,\,$ \\
\end{pmatrix},
\qquad
\lambda_{\max} =
4.2277,
\qquad
CR = 0.0859
\end{equation*}

\begin{equation*}
\mathbf{w}^{cos} =
\begin{pmatrix}
0.321520\\
0.288074\\
0.102888\\
\color{red} 0.287518\color{black}
\end{pmatrix}\end{equation*}
\begin{equation*}
\left[ \frac{{w}^{cos}_i}{{w}^{cos}_j} \right] =
\begin{pmatrix}
$\,\,$ 1 $\,\,$ & $\,\,$1.1161$\,\,$ & $\,\,$3.1249$\,\,$ & $\,\,$\color{red} 1.1183\color{black} $\,\,$ \\
$\,\,$0.8960$\,\,$ & $\,\,$ 1 $\,\,$ & $\,\,$2.7999$\,\,$ & $\,\,$\color{red} 1.0019\color{black}   $\,\,$ \\
$\,\,$0.3200$\,\,$ & $\,\,$0.3572$\,\,$ & $\,\,$ 1 $\,\,$ & $\,\,$\color{red} 0.3578\color{black}  $\,\,$ \\
$\,\,$\color{red} 0.8942\color{black} $\,\,$ & $\,\,$\color{red} 0.9981\color{black} $\,\,$ & $\,\,$\color{red} 2.7945\color{black} $\,\,$ & $\,\,$ 1  $\,\,$ \\
\end{pmatrix},
\end{equation*}

\begin{equation*}
\mathbf{w}^{\prime} =
\begin{pmatrix}
0.321341\\
0.287914\\
0.102831\\
0.287914
\end{pmatrix} =
0.999445\cdot
\begin{pmatrix}
0.321520\\
0.288074\\
0.102888\\
\color{gr} 0.288074\color{black}
\end{pmatrix},
\end{equation*}
\begin{equation*}
\left[ \frac{{w}^{\prime}_i}{{w}^{\prime}_j} \right] =
\begin{pmatrix}
$\,\,$ 1 $\,\,$ & $\,\,$1.1161$\,\,$ & $\,\,$3.1249$\,\,$ & $\,\,$\color{gr} 1.1161\color{black} $\,\,$ \\
$\,\,$0.8960$\,\,$ & $\,\,$ 1 $\,\,$ & $\,\,$2.7999$\,\,$ & $\,\,$\color{gr} \color{blue} 1\color{black}   $\,\,$ \\
$\,\,$0.3200$\,\,$ & $\,\,$0.3572$\,\,$ & $\,\,$ 1 $\,\,$ & $\,\,$\color{gr} 0.3572\color{black}  $\,\,$ \\
$\,\,$\color{gr} 0.8960\color{black} $\,\,$ & $\,\,$\color{gr} \color{blue} 1\color{black} $\,\,$ & $\,\,$\color{gr} 2.7999\color{black} $\,\,$ & $\,\,$ 1  $\,\,$ \\
\end{pmatrix},
\end{equation*}
\end{example}
\newpage
\begin{example}
\begin{equation*}
\mathbf{A} =
\begin{pmatrix}
$\,\,$ 1 $\,\,$ & $\,\,$2$\,\,$ & $\,\,$3$\,\,$ & $\,\,$1 $\,\,$ \\
$\,\,$ 1/2$\,\,$ & $\,\,$ 1 $\,\,$ & $\,\,$7$\,\,$ & $\,\,$1 $\,\,$ \\
$\,\,$ 1/3$\,\,$ & $\,\,$ 1/7$\,\,$ & $\,\,$ 1 $\,\,$ & $\,\,$ 1/4 $\,\,$ \\
$\,\,$ 1 $\,\,$ & $\,\,$ 1 $\,\,$ & $\,\,$4$\,\,$ & $\,\,$ 1  $\,\,$ \\
\end{pmatrix},
\qquad
\lambda_{\max} =
4.2109,
\qquad
CR = 0.0795
\end{equation*}

\begin{equation*}
\mathbf{w}^{cos} =
\begin{pmatrix}
0.334722\\
0.296659\\
0.074911\\
\color{red} 0.293708\color{black}
\end{pmatrix}\end{equation*}
\begin{equation*}
\left[ \frac{{w}^{cos}_i}{{w}^{cos}_j} \right] =
\begin{pmatrix}
$\,\,$ 1 $\,\,$ & $\,\,$1.1283$\,\,$ & $\,\,$4.4683$\,\,$ & $\,\,$\color{red} 1.1396\color{black} $\,\,$ \\
$\,\,$0.8863$\,\,$ & $\,\,$ 1 $\,\,$ & $\,\,$3.9602$\,\,$ & $\,\,$\color{red} 1.0100\color{black}   $\,\,$ \\
$\,\,$0.2238$\,\,$ & $\,\,$0.2525$\,\,$ & $\,\,$ 1 $\,\,$ & $\,\,$\color{red} 0.2551\color{black}  $\,\,$ \\
$\,\,$\color{red} 0.8775\color{black} $\,\,$ & $\,\,$\color{red} 0.9901\color{black} $\,\,$ & $\,\,$\color{red} 3.9208\color{black} $\,\,$ & $\,\,$ 1  $\,\,$ \\
\end{pmatrix},
\end{equation*}

\begin{equation*}
\mathbf{w}^{\prime} =
\begin{pmatrix}
0.333738\\
0.295786\\
0.074690\\
0.295786
\end{pmatrix} =
0.997058\cdot
\begin{pmatrix}
0.334722\\
0.296659\\
0.074911\\
\color{gr} 0.296659\color{black}
\end{pmatrix},
\end{equation*}
\begin{equation*}
\left[ \frac{{w}^{\prime}_i}{{w}^{\prime}_j} \right] =
\begin{pmatrix}
$\,\,$ 1 $\,\,$ & $\,\,$1.1283$\,\,$ & $\,\,$4.4683$\,\,$ & $\,\,$\color{gr} 1.1283\color{black} $\,\,$ \\
$\,\,$0.8863$\,\,$ & $\,\,$ 1 $\,\,$ & $\,\,$3.9602$\,\,$ & $\,\,$\color{gr} \color{blue} 1\color{black}   $\,\,$ \\
$\,\,$0.2238$\,\,$ & $\,\,$0.2525$\,\,$ & $\,\,$ 1 $\,\,$ & $\,\,$\color{gr} 0.2525\color{black}  $\,\,$ \\
$\,\,$\color{gr} 0.8863\color{black} $\,\,$ & $\,\,$\color{gr} \color{blue} 1\color{black} $\,\,$ & $\,\,$\color{gr} 3.9602\color{black} $\,\,$ & $\,\,$ 1  $\,\,$ \\
\end{pmatrix},
\end{equation*}
\end{example}
\newpage
\begin{example}
\begin{equation*}
\mathbf{A} =
\begin{pmatrix}
$\,\,$ 1 $\,\,$ & $\,\,$2$\,\,$ & $\,\,$3$\,\,$ & $\,\,$1 $\,\,$ \\
$\,\,$ 1/2$\,\,$ & $\,\,$ 1 $\,\,$ & $\,\,$8$\,\,$ & $\,\,$1 $\,\,$ \\
$\,\,$ 1/3$\,\,$ & $\,\,$ 1/8$\,\,$ & $\,\,$ 1 $\,\,$ & $\,\,$ 1/4 $\,\,$ \\
$\,\,$ 1 $\,\,$ & $\,\,$ 1 $\,\,$ & $\,\,$4$\,\,$ & $\,\,$ 1  $\,\,$ \\
\end{pmatrix},
\qquad
\lambda_{\max} =
4.2512,
\qquad
CR = 0.0947
\end{equation*}

\begin{equation*}
\mathbf{w}^{cos} =
\begin{pmatrix}
0.332966\\
0.303817\\
0.073013\\
\color{red} 0.290203\color{black}
\end{pmatrix}\end{equation*}
\begin{equation*}
\left[ \frac{{w}^{cos}_i}{{w}^{cos}_j} \right] =
\begin{pmatrix}
$\,\,$ 1 $\,\,$ & $\,\,$1.0959$\,\,$ & $\,\,$4.5604$\,\,$ & $\,\,$\color{red} 1.1474\color{black} $\,\,$ \\
$\,\,$0.9125$\,\,$ & $\,\,$ 1 $\,\,$ & $\,\,$4.1611$\,\,$ & $\,\,$\color{red} 1.0469\color{black}   $\,\,$ \\
$\,\,$0.2193$\,\,$ & $\,\,$0.2403$\,\,$ & $\,\,$ 1 $\,\,$ & $\,\,$\color{red} 0.2516\color{black}  $\,\,$ \\
$\,\,$\color{red} 0.8716\color{black} $\,\,$ & $\,\,$\color{red} 0.9552\color{black} $\,\,$ & $\,\,$\color{red} 3.9747\color{black} $\,\,$ & $\,\,$ 1  $\,\,$ \\
\end{pmatrix},
\end{equation*}

\begin{equation*}
\mathbf{w}^{\prime} =
\begin{pmatrix}
0.332352\\
0.303256\\
0.072878\\
0.291513
\end{pmatrix} =
0.998154\cdot
\begin{pmatrix}
0.332966\\
0.303817\\
0.073013\\
\color{gr} 0.292052\color{black}
\end{pmatrix},
\end{equation*}
\begin{equation*}
\left[ \frac{{w}^{\prime}_i}{{w}^{\prime}_j} \right] =
\begin{pmatrix}
$\,\,$ 1 $\,\,$ & $\,\,$1.0959$\,\,$ & $\,\,$4.5604$\,\,$ & $\,\,$\color{gr} 1.1401\color{black} $\,\,$ \\
$\,\,$0.9125$\,\,$ & $\,\,$ 1 $\,\,$ & $\,\,$4.1611$\,\,$ & $\,\,$\color{gr} 1.0403\color{black}   $\,\,$ \\
$\,\,$0.2193$\,\,$ & $\,\,$0.2403$\,\,$ & $\,\,$ 1 $\,\,$ & $\,\,$\color{gr} \color{blue}  1/4\color{black}  $\,\,$ \\
$\,\,$\color{gr} 0.8771\color{black} $\,\,$ & $\,\,$\color{gr} 0.9613\color{black} $\,\,$ & $\,\,$\color{gr} \color{blue} 4\color{black} $\,\,$ & $\,\,$ 1  $\,\,$ \\
\end{pmatrix},
\end{equation*}
\end{example}
\newpage
\begin{example}
\begin{equation*}
\mathbf{A} =
\begin{pmatrix}
$\,\,$ 1 $\,\,$ & $\,\,$2$\,\,$ & $\,\,$3$\,\,$ & $\,\,$2 $\,\,$ \\
$\,\,$ 1/2$\,\,$ & $\,\,$ 1 $\,\,$ & $\,\,$8$\,\,$ & $\,\,$2 $\,\,$ \\
$\,\,$ 1/3$\,\,$ & $\,\,$ 1/8$\,\,$ & $\,\,$ 1 $\,\,$ & $\,\,$ 1/2 $\,\,$ \\
$\,\,$ 1/2$\,\,$ & $\,\,$ 1/2$\,\,$ & $\,\,$2$\,\,$ & $\,\,$ 1  $\,\,$ \\
\end{pmatrix},
\qquad
\lambda_{\max} =
4.2512,
\qquad
CR = 0.0947
\end{equation*}

\begin{equation*}
\mathbf{w}^{cos} =
\begin{pmatrix}
0.390025\\
0.351331\\
0.087309\\
\color{red} 0.171335\color{black}
\end{pmatrix}\end{equation*}
\begin{equation*}
\left[ \frac{{w}^{cos}_i}{{w}^{cos}_j} \right] =
\begin{pmatrix}
$\,\,$ 1 $\,\,$ & $\,\,$1.1101$\,\,$ & $\,\,$4.4672$\,\,$ & $\,\,$\color{red} 2.2764\color{black} $\,\,$ \\
$\,\,$0.9008$\,\,$ & $\,\,$ 1 $\,\,$ & $\,\,$4.0240$\,\,$ & $\,\,$\color{red} 2.0505\color{black}   $\,\,$ \\
$\,\,$0.2239$\,\,$ & $\,\,$0.2485$\,\,$ & $\,\,$ 1 $\,\,$ & $\,\,$\color{red} 0.5096\color{black}  $\,\,$ \\
$\,\,$\color{red} 0.4393\color{black} $\,\,$ & $\,\,$\color{red} 0.4877\color{black} $\,\,$ & $\,\,$\color{red} 1.9624\color{black} $\,\,$ & $\,\,$ 1  $\,\,$ \\
\end{pmatrix},
\end{equation*}

\begin{equation*}
\mathbf{w}^{\prime} =
\begin{pmatrix}
0.388749\\
0.350181\\
0.087023\\
0.174046
\end{pmatrix} =
0.996728\cdot
\begin{pmatrix}
0.390025\\
0.351331\\
0.087309\\
\color{gr} 0.174618\color{black}
\end{pmatrix},
\end{equation*}
\begin{equation*}
\left[ \frac{{w}^{\prime}_i}{{w}^{\prime}_j} \right] =
\begin{pmatrix}
$\,\,$ 1 $\,\,$ & $\,\,$1.1101$\,\,$ & $\,\,$4.4672$\,\,$ & $\,\,$\color{gr} 2.2336\color{black} $\,\,$ \\
$\,\,$0.9008$\,\,$ & $\,\,$ 1 $\,\,$ & $\,\,$4.0240$\,\,$ & $\,\,$\color{gr} 2.0120\color{black}   $\,\,$ \\
$\,\,$0.2239$\,\,$ & $\,\,$0.2485$\,\,$ & $\,\,$ 1 $\,\,$ & $\,\,$\color{gr} \color{blue}  1/2\color{black}  $\,\,$ \\
$\,\,$\color{gr} 0.4477\color{black} $\,\,$ & $\,\,$\color{gr} 0.4970\color{black} $\,\,$ & $\,\,$\color{gr} \color{blue} 2\color{black} $\,\,$ & $\,\,$ 1  $\,\,$ \\
\end{pmatrix},
\end{equation*}
\end{example}
\newpage
\begin{example}
\begin{equation*}
\mathbf{A} =
\begin{pmatrix}
$\,\,$ 1 $\,\,$ & $\,\,$2$\,\,$ & $\,\,$3$\,\,$ & $\,\,$3 $\,\,$ \\
$\,\,$ 1/2$\,\,$ & $\,\,$ 1 $\,\,$ & $\,\,$1$\,\,$ & $\,\,$2 $\,\,$ \\
$\,\,$ 1/3$\,\,$ & $\,\,$ 1 $\,\,$ & $\,\,$ 1 $\,\,$ & $\,\,$ 1/2 $\,\,$ \\
$\,\,$ 1/3$\,\,$ & $\,\,$ 1/2$\,\,$ & $\,\,$2$\,\,$ & $\,\,$ 1  $\,\,$ \\
\end{pmatrix},
\qquad
\lambda_{\max} =
4.1707,
\qquad
CR = 0.0644
\end{equation*}

\begin{equation*}
\mathbf{w}^{cos} =
\begin{pmatrix}
\color{red} 0.448839\color{black} \\
0.224891\\
0.149737\\
0.176533
\end{pmatrix}\end{equation*}
\begin{equation*}
\left[ \frac{{w}^{cos}_i}{{w}^{cos}_j} \right] =
\begin{pmatrix}
$\,\,$ 1 $\,\,$ & $\,\,$\color{red} 1.9958\color{black} $\,\,$ & $\,\,$\color{red} 2.9975\color{black} $\,\,$ & $\,\,$\color{red} 2.5425\color{black} $\,\,$ \\
$\,\,$\color{red} 0.5011\color{black} $\,\,$ & $\,\,$ 1 $\,\,$ & $\,\,$1.5019$\,\,$ & $\,\,$1.2739  $\,\,$ \\
$\,\,$\color{red} 0.3336\color{black} $\,\,$ & $\,\,$0.6658$\,\,$ & $\,\,$ 1 $\,\,$ & $\,\,$0.8482 $\,\,$ \\
$\,\,$\color{red} 0.3933\color{black} $\,\,$ & $\,\,$0.7850$\,\,$ & $\,\,$1.1790$\,\,$ & $\,\,$ 1  $\,\,$ \\
\end{pmatrix},
\end{equation*}

\begin{equation*}
\mathbf{w}^{\prime} =
\begin{pmatrix}
0.449044\\
0.224808\\
0.149681\\
0.176467
\end{pmatrix} =
0.999629\cdot
\begin{pmatrix}
\color{gr} 0.449210\color{black} \\
0.224891\\
0.149737\\
0.176533
\end{pmatrix},
\end{equation*}
\begin{equation*}
\left[ \frac{{w}^{\prime}_i}{{w}^{\prime}_j} \right] =
\begin{pmatrix}
$\,\,$ 1 $\,\,$ & $\,\,$\color{gr} 1.9975\color{black} $\,\,$ & $\,\,$\color{gr} \color{blue} 3\color{black} $\,\,$ & $\,\,$\color{gr} 2.5446\color{black} $\,\,$ \\
$\,\,$\color{gr} 0.5006\color{black} $\,\,$ & $\,\,$ 1 $\,\,$ & $\,\,$1.5019$\,\,$ & $\,\,$1.2739  $\,\,$ \\
$\,\,$\color{gr} \color{blue}  1/3\color{black} $\,\,$ & $\,\,$0.6658$\,\,$ & $\,\,$ 1 $\,\,$ & $\,\,$0.8482 $\,\,$ \\
$\,\,$\color{gr} 0.3930\color{black} $\,\,$ & $\,\,$0.7850$\,\,$ & $\,\,$1.1790$\,\,$ & $\,\,$ 1  $\,\,$ \\
\end{pmatrix},
\end{equation*}
\end{example}
\newpage
\begin{example}
\begin{equation*}
\mathbf{A} =
\begin{pmatrix}
$\,\,$ 1 $\,\,$ & $\,\,$2$\,\,$ & $\,\,$4$\,\,$ & $\,\,$3 $\,\,$ \\
$\,\,$ 1/2$\,\,$ & $\,\,$ 1 $\,\,$ & $\,\,$8$\,\,$ & $\,\,$2 $\,\,$ \\
$\,\,$ 1/4$\,\,$ & $\,\,$ 1/8$\,\,$ & $\,\,$ 1 $\,\,$ & $\,\,$ 1/2 $\,\,$ \\
$\,\,$ 1/3$\,\,$ & $\,\,$ 1/2$\,\,$ & $\,\,$2$\,\,$ & $\,\,$ 1  $\,\,$ \\
\end{pmatrix},
\qquad
\lambda_{\max} =
4.1707,
\qquad
CR = 0.0644
\end{equation*}

\begin{equation*}
\mathbf{w}^{cos} =
\begin{pmatrix}
0.440212\\
0.337452\\
0.075647\\
\color{red} 0.146689\color{black}
\end{pmatrix}\end{equation*}
\begin{equation*}
\left[ \frac{{w}^{cos}_i}{{w}^{cos}_j} \right] =
\begin{pmatrix}
$\,\,$ 1 $\,\,$ & $\,\,$1.3045$\,\,$ & $\,\,$5.8193$\,\,$ & $\,\,$\color{red} 3.0010\color{black} $\,\,$ \\
$\,\,$0.7666$\,\,$ & $\,\,$ 1 $\,\,$ & $\,\,$4.4609$\,\,$ & $\,\,$\color{red} 2.3005\color{black}   $\,\,$ \\
$\,\,$0.1718$\,\,$ & $\,\,$0.2242$\,\,$ & $\,\,$ 1 $\,\,$ & $\,\,$\color{red} 0.5157\color{black}  $\,\,$ \\
$\,\,$\color{red} 0.3332\color{black} $\,\,$ & $\,\,$\color{red} 0.4347\color{black} $\,\,$ & $\,\,$\color{red} 1.9391\color{black} $\,\,$ & $\,\,$ 1  $\,\,$ \\
\end{pmatrix},
\end{equation*}

\begin{equation*}
\mathbf{w}^{\prime} =
\begin{pmatrix}
0.440191\\
0.337435\\
0.075644\\
0.146730
\end{pmatrix} =
0.999952\cdot
\begin{pmatrix}
0.440212\\
0.337452\\
0.075647\\
\color{gr} 0.146737\color{black}
\end{pmatrix},
\end{equation*}
\begin{equation*}
\left[ \frac{{w}^{\prime}_i}{{w}^{\prime}_j} \right] =
\begin{pmatrix}
$\,\,$ 1 $\,\,$ & $\,\,$1.3045$\,\,$ & $\,\,$5.8193$\,\,$ & $\,\,$\color{gr} \color{blue} 3\color{black} $\,\,$ \\
$\,\,$0.7666$\,\,$ & $\,\,$ 1 $\,\,$ & $\,\,$4.4609$\,\,$ & $\,\,$\color{gr} 2.2997\color{black}   $\,\,$ \\
$\,\,$0.1718$\,\,$ & $\,\,$0.2242$\,\,$ & $\,\,$ 1 $\,\,$ & $\,\,$\color{gr} 0.5155\color{black}  $\,\,$ \\
$\,\,$\color{gr} \color{blue}  1/3\color{black} $\,\,$ & $\,\,$\color{gr} 0.4348\color{black} $\,\,$ & $\,\,$\color{gr} 1.9398\color{black} $\,\,$ & $\,\,$ 1  $\,\,$ \\
\end{pmatrix},
\end{equation*}
\end{example}
\newpage
\begin{example}
\begin{equation*}
\mathbf{A} =
\begin{pmatrix}
$\,\,$ 1 $\,\,$ & $\,\,$2$\,\,$ & $\,\,$4$\,\,$ & $\,\,$3 $\,\,$ \\
$\,\,$ 1/2$\,\,$ & $\,\,$ 1 $\,\,$ & $\,\,$9$\,\,$ & $\,\,$2 $\,\,$ \\
$\,\,$ 1/4$\,\,$ & $\,\,$ 1/9$\,\,$ & $\,\,$ 1 $\,\,$ & $\,\,$ 1/2 $\,\,$ \\
$\,\,$ 1/3$\,\,$ & $\,\,$ 1/2$\,\,$ & $\,\,$2$\,\,$ & $\,\,$ 1  $\,\,$ \\
\end{pmatrix},
\qquad
\lambda_{\max} =
4.2052,
\qquad
CR = 0.0774
\end{equation*}

\begin{equation*}
\mathbf{w}^{cos} =
\begin{pmatrix}
0.437757\\
0.343465\\
0.073861\\
\color{red} 0.144916\color{black}
\end{pmatrix}\end{equation*}
\begin{equation*}
\left[ \frac{{w}^{cos}_i}{{w}^{cos}_j} \right] =
\begin{pmatrix}
$\,\,$ 1 $\,\,$ & $\,\,$1.2745$\,\,$ & $\,\,$5.9267$\,\,$ & $\,\,$\color{red} 3.0208\color{black} $\,\,$ \\
$\,\,$0.7846$\,\,$ & $\,\,$ 1 $\,\,$ & $\,\,$4.6501$\,\,$ & $\,\,$\color{red} 2.3701\color{black}   $\,\,$ \\
$\,\,$0.1687$\,\,$ & $\,\,$0.2150$\,\,$ & $\,\,$ 1 $\,\,$ & $\,\,$\color{red} 0.5097\color{black}  $\,\,$ \\
$\,\,$\color{red} 0.3310\color{black} $\,\,$ & $\,\,$\color{red} 0.4219\color{black} $\,\,$ & $\,\,$\color{red} 1.9620\color{black} $\,\,$ & $\,\,$ 1  $\,\,$ \\
\end{pmatrix},
\end{equation*}

\begin{equation*}
\mathbf{w}^{\prime} =
\begin{pmatrix}
0.437319\\
0.343121\\
0.073787\\
0.145773
\end{pmatrix} =
0.998998\cdot
\begin{pmatrix}
0.437757\\
0.343465\\
0.073861\\
\color{gr} 0.145919\color{black}
\end{pmatrix},
\end{equation*}
\begin{equation*}
\left[ \frac{{w}^{\prime}_i}{{w}^{\prime}_j} \right] =
\begin{pmatrix}
$\,\,$ 1 $\,\,$ & $\,\,$1.2745$\,\,$ & $\,\,$5.9267$\,\,$ & $\,\,$\color{gr} \color{blue} 3\color{black} $\,\,$ \\
$\,\,$0.7846$\,\,$ & $\,\,$ 1 $\,\,$ & $\,\,$4.6501$\,\,$ & $\,\,$\color{gr} 2.3538\color{black}   $\,\,$ \\
$\,\,$0.1687$\,\,$ & $\,\,$0.2150$\,\,$ & $\,\,$ 1 $\,\,$ & $\,\,$\color{gr} 0.5062\color{black}  $\,\,$ \\
$\,\,$\color{gr} \color{blue}  1/3\color{black} $\,\,$ & $\,\,$\color{gr} 0.4248\color{black} $\,\,$ & $\,\,$\color{gr} 1.9756\color{black} $\,\,$ & $\,\,$ 1  $\,\,$ \\
\end{pmatrix},
\end{equation*}
\end{example}
\newpage
\begin{example}
\begin{equation*}
\mathbf{A} =
\begin{pmatrix}
$\,\,$ 1 $\,\,$ & $\,\,$2$\,\,$ & $\,\,$5$\,\,$ & $\,\,$2 $\,\,$ \\
$\,\,$ 1/2$\,\,$ & $\,\,$ 1 $\,\,$ & $\,\,$4$\,\,$ & $\,\,$3 $\,\,$ \\
$\,\,$ 1/5$\,\,$ & $\,\,$ 1/4$\,\,$ & $\,\,$ 1 $\,\,$ & $\,\,$ 1/2 $\,\,$ \\
$\,\,$ 1/2$\,\,$ & $\,\,$ 1/3$\,\,$ & $\,\,$2$\,\,$ & $\,\,$ 1  $\,\,$ \\
\end{pmatrix},
\qquad
\lambda_{\max} =
4.1046,
\qquad
CR = 0.0395
\end{equation*}

\begin{equation*}
\mathbf{w}^{cos} =
\begin{pmatrix}
0.431637\\
0.325620\\
\color{red} 0.080558\color{black} \\
0.162185
\end{pmatrix}\end{equation*}
\begin{equation*}
\left[ \frac{{w}^{cos}_i}{{w}^{cos}_j} \right] =
\begin{pmatrix}
$\,\,$ 1 $\,\,$ & $\,\,$1.3256$\,\,$ & $\,\,$\color{red} 5.3581\color{black} $\,\,$ & $\,\,$2.6614$\,\,$ \\
$\,\,$0.7544$\,\,$ & $\,\,$ 1 $\,\,$ & $\,\,$\color{red} 4.0420\color{black} $\,\,$ & $\,\,$2.0077  $\,\,$ \\
$\,\,$\color{red} 0.1866\color{black} $\,\,$ & $\,\,$\color{red} 0.2474\color{black} $\,\,$ & $\,\,$ 1 $\,\,$ & $\,\,$\color{red} 0.4967\color{black}  $\,\,$ \\
$\,\,$0.3757$\,\,$ & $\,\,$0.4981$\,\,$ & $\,\,$\color{red} 2.0133\color{black} $\,\,$ & $\,\,$ 1  $\,\,$ \\
\end{pmatrix},
\end{equation*}

\begin{equation*}
\mathbf{w}^{\prime} =
\begin{pmatrix}
0.431406\\
0.325446\\
0.081049\\
0.162099
\end{pmatrix} =
0.999466\cdot
\begin{pmatrix}
0.431637\\
0.325620\\
\color{gr} 0.081093\color{black} \\
0.162185
\end{pmatrix},
\end{equation*}
\begin{equation*}
\left[ \frac{{w}^{\prime}_i}{{w}^{\prime}_j} \right] =
\begin{pmatrix}
$\,\,$ 1 $\,\,$ & $\,\,$1.3256$\,\,$ & $\,\,$\color{gr} 5.3228\color{black} $\,\,$ & $\,\,$2.6614$\,\,$ \\
$\,\,$0.7544$\,\,$ & $\,\,$ 1 $\,\,$ & $\,\,$\color{gr} 4.0154\color{black} $\,\,$ & $\,\,$2.0077  $\,\,$ \\
$\,\,$\color{gr} 0.1879\color{black} $\,\,$ & $\,\,$\color{gr} 0.2490\color{black} $\,\,$ & $\,\,$ 1 $\,\,$ & $\,\,$\color{gr} \color{blue}  1/2\color{black}  $\,\,$ \\
$\,\,$0.3757$\,\,$ & $\,\,$0.4981$\,\,$ & $\,\,$\color{gr} \color{blue} 2\color{black} $\,\,$ & $\,\,$ 1  $\,\,$ \\
\end{pmatrix},
\end{equation*}
\end{example}
\newpage
\begin{example}
\begin{equation*}
\mathbf{A} =
\begin{pmatrix}
$\,\,$ 1 $\,\,$ & $\,\,$2$\,\,$ & $\,\,$5$\,\,$ & $\,\,$4 $\,\,$ \\
$\,\,$ 1/2$\,\,$ & $\,\,$ 1 $\,\,$ & $\,\,$9$\,\,$ & $\,\,$3 $\,\,$ \\
$\,\,$ 1/5$\,\,$ & $\,\,$ 1/9$\,\,$ & $\,\,$ 1 $\,\,$ & $\,\,$ 1/2 $\,\,$ \\
$\,\,$ 1/4$\,\,$ & $\,\,$ 1/3$\,\,$ & $\,\,$2$\,\,$ & $\,\,$ 1  $\,\,$ \\
\end{pmatrix},
\qquad
\lambda_{\max} =
4.1406,
\qquad
CR = 0.0530
\end{equation*}

\begin{equation*}
\mathbf{w}^{cos} =
\begin{pmatrix}
0.463409\\
0.357207\\
0.063910\\
\color{red} 0.115474\color{black}
\end{pmatrix}\end{equation*}
\begin{equation*}
\left[ \frac{{w}^{cos}_i}{{w}^{cos}_j} \right] =
\begin{pmatrix}
$\,\,$ 1 $\,\,$ & $\,\,$1.2973$\,\,$ & $\,\,$7.2509$\,\,$ & $\,\,$\color{red} 4.0131\color{black} $\,\,$ \\
$\,\,$0.7708$\,\,$ & $\,\,$ 1 $\,\,$ & $\,\,$5.5892$\,\,$ & $\,\,$\color{red} 3.0934\color{black}   $\,\,$ \\
$\,\,$0.1379$\,\,$ & $\,\,$0.1789$\,\,$ & $\,\,$ 1 $\,\,$ & $\,\,$\color{red} 0.5535\color{black}  $\,\,$ \\
$\,\,$\color{red} 0.2492\color{black} $\,\,$ & $\,\,$\color{red} 0.3233\color{black} $\,\,$ & $\,\,$\color{red} 1.8068\color{black} $\,\,$ & $\,\,$ 1  $\,\,$ \\
\end{pmatrix},
\end{equation*}

\begin{equation*}
\mathbf{w}^{\prime} =
\begin{pmatrix}
0.463234\\
0.357072\\
0.063886\\
0.115808
\end{pmatrix} =
0.999622\cdot
\begin{pmatrix}
0.463409\\
0.357207\\
0.063910\\
\color{gr} 0.115852\color{black}
\end{pmatrix},
\end{equation*}
\begin{equation*}
\left[ \frac{{w}^{\prime}_i}{{w}^{\prime}_j} \right] =
\begin{pmatrix}
$\,\,$ 1 $\,\,$ & $\,\,$1.2973$\,\,$ & $\,\,$7.2509$\,\,$ & $\,\,$\color{gr} \color{blue} 4\color{black} $\,\,$ \\
$\,\,$0.7708$\,\,$ & $\,\,$ 1 $\,\,$ & $\,\,$5.5892$\,\,$ & $\,\,$\color{gr} 3.0833\color{black}   $\,\,$ \\
$\,\,$0.1379$\,\,$ & $\,\,$0.1789$\,\,$ & $\,\,$ 1 $\,\,$ & $\,\,$\color{gr} 0.5517\color{black}  $\,\,$ \\
$\,\,$\color{gr} \color{blue}  1/4\color{black} $\,\,$ & $\,\,$\color{gr} 0.3243\color{black} $\,\,$ & $\,\,$\color{gr} 1.8127\color{black} $\,\,$ & $\,\,$ 1  $\,\,$ \\
\end{pmatrix},
\end{equation*}
\end{example}
\newpage
\begin{example}
\begin{equation*}
\mathbf{A} =
\begin{pmatrix}
$\,\,$ 1 $\,\,$ & $\,\,$2$\,\,$ & $\,\,$6$\,\,$ & $\,\,$1 $\,\,$ \\
$\,\,$ 1/2$\,\,$ & $\,\,$ 1 $\,\,$ & $\,\,$4$\,\,$ & $\,\,$1 $\,\,$ \\
$\,\,$ 1/6$\,\,$ & $\,\,$ 1/4$\,\,$ & $\,\,$ 1 $\,\,$ & $\,\,$ 1/5 $\,\,$ \\
$\,\,$ 1 $\,\,$ & $\,\,$ 1 $\,\,$ & $\,\,$5$\,\,$ & $\,\,$ 1  $\,\,$ \\
\end{pmatrix},
\qquad
\lambda_{\max} =
4.0407,
\qquad
CR = 0.0153
\end{equation*}

\begin{equation*}
\mathbf{w}^{cos} =
\begin{pmatrix}
0.382043\\
0.247044\\
\color{red} 0.061611\color{black} \\
0.309302
\end{pmatrix}\end{equation*}
\begin{equation*}
\left[ \frac{{w}^{cos}_i}{{w}^{cos}_j} \right] =
\begin{pmatrix}
$\,\,$ 1 $\,\,$ & $\,\,$1.5465$\,\,$ & $\,\,$\color{red} 6.2009\color{black} $\,\,$ & $\,\,$1.2352$\,\,$ \\
$\,\,$0.6466$\,\,$ & $\,\,$ 1 $\,\,$ & $\,\,$\color{red} 4.0097\color{black} $\,\,$ & $\,\,$0.7987  $\,\,$ \\
$\,\,$\color{red} 0.1613\color{black} $\,\,$ & $\,\,$\color{red} 0.2494\color{black} $\,\,$ & $\,\,$ 1 $\,\,$ & $\,\,$\color{red} 0.1992\color{black}  $\,\,$ \\
$\,\,$0.8096$\,\,$ & $\,\,$1.2520$\,\,$ & $\,\,$\color{red} 5.0202\color{black} $\,\,$ & $\,\,$ 1  $\,\,$ \\
\end{pmatrix},
\end{equation*}

\begin{equation*}
\mathbf{w}^{\prime} =
\begin{pmatrix}
0.381986\\
0.247007\\
0.061752\\
0.309256
\end{pmatrix} =
0.999850\cdot
\begin{pmatrix}
0.382043\\
0.247044\\
\color{gr} 0.061761\color{black} \\
0.309302
\end{pmatrix},
\end{equation*}
\begin{equation*}
\left[ \frac{{w}^{\prime}_i}{{w}^{\prime}_j} \right] =
\begin{pmatrix}
$\,\,$ 1 $\,\,$ & $\,\,$1.5465$\,\,$ & $\,\,$\color{gr} 6.1858\color{black} $\,\,$ & $\,\,$1.2352$\,\,$ \\
$\,\,$0.6466$\,\,$ & $\,\,$ 1 $\,\,$ & $\,\,$\color{gr} \color{blue} 4\color{black} $\,\,$ & $\,\,$0.7987  $\,\,$ \\
$\,\,$\color{gr} 0.1617\color{black} $\,\,$ & $\,\,$\color{gr} \color{blue}  1/4\color{black} $\,\,$ & $\,\,$ 1 $\,\,$ & $\,\,$\color{gr} 0.1997\color{black}  $\,\,$ \\
$\,\,$0.8096$\,\,$ & $\,\,$1.2520$\,\,$ & $\,\,$\color{gr} 5.0081\color{black} $\,\,$ & $\,\,$ 1  $\,\,$ \\
\end{pmatrix},
\end{equation*}
\end{example}
\newpage
\begin{example}
\begin{equation*}
\mathbf{A} =
\begin{pmatrix}
$\,\,$ 1 $\,\,$ & $\,\,$2$\,\,$ & $\,\,$6$\,\,$ & $\,\,$2 $\,\,$ \\
$\,\,$ 1/2$\,\,$ & $\,\,$ 1 $\,\,$ & $\,\,$2$\,\,$ & $\,\,$2 $\,\,$ \\
$\,\,$ 1/6$\,\,$ & $\,\,$ 1/2$\,\,$ & $\,\,$ 1 $\,\,$ & $\,\,$ 1/5 $\,\,$ \\
$\,\,$ 1/2$\,\,$ & $\,\,$ 1/2$\,\,$ & $\,\,$5$\,\,$ & $\,\,$ 1  $\,\,$ \\
\end{pmatrix},
\qquad
\lambda_{\max} =
4.2277,
\qquad
CR = 0.0859
\end{equation*}

\begin{equation*}
\mathbf{w}^{cos} =
\begin{pmatrix}
\color{red} 0.443561\color{black} \\
0.252094\\
0.077793\\
0.226553
\end{pmatrix}\end{equation*}
\begin{equation*}
\left[ \frac{{w}^{cos}_i}{{w}^{cos}_j} \right] =
\begin{pmatrix}
$\,\,$ 1 $\,\,$ & $\,\,$\color{red} 1.7595\color{black} $\,\,$ & $\,\,$\color{red} 5.7018\color{black} $\,\,$ & $\,\,$\color{red} 1.9579\color{black} $\,\,$ \\
$\,\,$\color{red} 0.5683\color{black} $\,\,$ & $\,\,$ 1 $\,\,$ & $\,\,$3.2406$\,\,$ & $\,\,$1.1127  $\,\,$ \\
$\,\,$\color{red} 0.1754\color{black} $\,\,$ & $\,\,$0.3086$\,\,$ & $\,\,$ 1 $\,\,$ & $\,\,$0.3434 $\,\,$ \\
$\,\,$\color{red} 0.5108\color{black} $\,\,$ & $\,\,$0.8987$\,\,$ & $\,\,$2.9122$\,\,$ & $\,\,$ 1  $\,\,$ \\
\end{pmatrix},
\end{equation*}

\begin{equation*}
\mathbf{w}^{\prime} =
\begin{pmatrix}
0.448821\\
0.249710\\
0.077058\\
0.224411
\end{pmatrix} =
0.990546\cdot
\begin{pmatrix}
\color{gr} 0.453105\color{black} \\
0.252094\\
0.077793\\
0.226553
\end{pmatrix},
\end{equation*}
\begin{equation*}
\left[ \frac{{w}^{\prime}_i}{{w}^{\prime}_j} \right] =
\begin{pmatrix}
$\,\,$ 1 $\,\,$ & $\,\,$\color{gr} 1.7974\color{black} $\,\,$ & $\,\,$\color{gr} 5.8245\color{black} $\,\,$ & $\,\,$\color{gr} \color{blue} 2\color{black} $\,\,$ \\
$\,\,$\color{gr} 0.5564\color{black} $\,\,$ & $\,\,$ 1 $\,\,$ & $\,\,$3.2406$\,\,$ & $\,\,$1.1127  $\,\,$ \\
$\,\,$\color{gr} 0.1717\color{black} $\,\,$ & $\,\,$0.3086$\,\,$ & $\,\,$ 1 $\,\,$ & $\,\,$0.3434 $\,\,$ \\
$\,\,$\color{gr} \color{blue}  1/2\color{black} $\,\,$ & $\,\,$0.8987$\,\,$ & $\,\,$2.9122$\,\,$ & $\,\,$ 1  $\,\,$ \\
\end{pmatrix},
\end{equation*}
\end{example}
\newpage
\begin{example}
\begin{equation*}
\mathbf{A} =
\begin{pmatrix}
$\,\,$ 1 $\,\,$ & $\,\,$2$\,\,$ & $\,\,$6$\,\,$ & $\,\,$2 $\,\,$ \\
$\,\,$ 1/2$\,\,$ & $\,\,$ 1 $\,\,$ & $\,\,$4$\,\,$ & $\,\,$4 $\,\,$ \\
$\,\,$ 1/6$\,\,$ & $\,\,$ 1/4$\,\,$ & $\,\,$ 1 $\,\,$ & $\,\,$ 1/2 $\,\,$ \\
$\,\,$ 1/2$\,\,$ & $\,\,$ 1/4$\,\,$ & $\,\,$2$\,\,$ & $\,\,$ 1  $\,\,$ \\
\end{pmatrix},
\qquad
\lambda_{\max} =
4.1707,
\qquad
CR = 0.0644
\end{equation*}

\begin{equation*}
\mathbf{w}^{cos} =
\begin{pmatrix}
0.439402\\
0.337521\\
\color{red} 0.073135\color{black} \\
0.149942
\end{pmatrix}\end{equation*}
\begin{equation*}
\left[ \frac{{w}^{cos}_i}{{w}^{cos}_j} \right] =
\begin{pmatrix}
$\,\,$ 1 $\,\,$ & $\,\,$1.3019$\,\,$ & $\,\,$\color{red} 6.0081\color{black} $\,\,$ & $\,\,$2.9305$\,\,$ \\
$\,\,$0.7681$\,\,$ & $\,\,$ 1 $\,\,$ & $\,\,$\color{red} 4.6150\color{black} $\,\,$ & $\,\,$2.2510  $\,\,$ \\
$\,\,$\color{red} 0.1664\color{black} $\,\,$ & $\,\,$\color{red} 0.2167\color{black} $\,\,$ & $\,\,$ 1 $\,\,$ & $\,\,$\color{red} 0.4878\color{black}  $\,\,$ \\
$\,\,$0.3412$\,\,$ & $\,\,$0.4442$\,\,$ & $\,\,$\color{red} 2.0502\color{black} $\,\,$ & $\,\,$ 1  $\,\,$ \\
\end{pmatrix},
\end{equation*}

\begin{equation*}
\mathbf{w}^{\prime} =
\begin{pmatrix}
0.439359\\
0.337487\\
0.073226\\
0.149927
\end{pmatrix} =
0.999901\cdot
\begin{pmatrix}
0.439402\\
0.337521\\
\color{gr} 0.073234\color{black} \\
0.149942
\end{pmatrix},
\end{equation*}
\begin{equation*}
\left[ \frac{{w}^{\prime}_i}{{w}^{\prime}_j} \right] =
\begin{pmatrix}
$\,\,$ 1 $\,\,$ & $\,\,$1.3019$\,\,$ & $\,\,$\color{gr} \color{blue} 6\color{black} $\,\,$ & $\,\,$2.9305$\,\,$ \\
$\,\,$0.7681$\,\,$ & $\,\,$ 1 $\,\,$ & $\,\,$\color{gr} 4.6088\color{black} $\,\,$ & $\,\,$2.2510  $\,\,$ \\
$\,\,$\color{gr} \color{blue}  1/6\color{black} $\,\,$ & $\,\,$\color{gr} 0.2170\color{black} $\,\,$ & $\,\,$ 1 $\,\,$ & $\,\,$\color{gr} 0.4884\color{black}  $\,\,$ \\
$\,\,$0.3412$\,\,$ & $\,\,$0.4442$\,\,$ & $\,\,$\color{gr} 2.0474\color{black} $\,\,$ & $\,\,$ 1  $\,\,$ \\
\end{pmatrix},
\end{equation*}
\end{example}
\newpage
\begin{example}
\begin{equation*}
\mathbf{A} =
\begin{pmatrix}
$\,\,$ 1 $\,\,$ & $\,\,$2$\,\,$ & $\,\,$6$\,\,$ & $\,\,$2 $\,\,$ \\
$\,\,$ 1/2$\,\,$ & $\,\,$ 1 $\,\,$ & $\,\,$4$\,\,$ & $\,\,$5 $\,\,$ \\
$\,\,$ 1/6$\,\,$ & $\,\,$ 1/4$\,\,$ & $\,\,$ 1 $\,\,$ & $\,\,$ 1/2 $\,\,$ \\
$\,\,$ 1/2$\,\,$ & $\,\,$ 1/5$\,\,$ & $\,\,$2$\,\,$ & $\,\,$ 1  $\,\,$ \\
\end{pmatrix},
\qquad
\lambda_{\max} =
4.2394,
\qquad
CR = 0.0903
\end{equation*}

\begin{equation*}
\mathbf{w}^{cos} =
\begin{pmatrix}
0.435702\\
0.349155\\
\color{red} 0.071611\color{black} \\
0.143532
\end{pmatrix}\end{equation*}
\begin{equation*}
\left[ \frac{{w}^{cos}_i}{{w}^{cos}_j} \right] =
\begin{pmatrix}
$\,\,$ 1 $\,\,$ & $\,\,$1.2479$\,\,$ & $\,\,$\color{red} 6.0843\color{black} $\,\,$ & $\,\,$3.0356$\,\,$ \\
$\,\,$0.8014$\,\,$ & $\,\,$ 1 $\,\,$ & $\,\,$\color{red} 4.8757\color{black} $\,\,$ & $\,\,$2.4326  $\,\,$ \\
$\,\,$\color{red} 0.1644\color{black} $\,\,$ & $\,\,$\color{red} 0.2051\color{black} $\,\,$ & $\,\,$ 1 $\,\,$ & $\,\,$\color{red} 0.4989\color{black}  $\,\,$ \\
$\,\,$0.3294$\,\,$ & $\,\,$0.4111$\,\,$ & $\,\,$\color{red} 2.0043\color{black} $\,\,$ & $\,\,$ 1  $\,\,$ \\
\end{pmatrix},
\end{equation*}

\begin{equation*}
\mathbf{w}^{\prime} =
\begin{pmatrix}
0.435635\\
0.349101\\
0.071755\\
0.143509
\end{pmatrix} =
0.999845\cdot
\begin{pmatrix}
0.435702\\
0.349155\\
\color{gr} 0.071766\color{black} \\
0.143532
\end{pmatrix},
\end{equation*}
\begin{equation*}
\left[ \frac{{w}^{\prime}_i}{{w}^{\prime}_j} \right] =
\begin{pmatrix}
$\,\,$ 1 $\,\,$ & $\,\,$1.2479$\,\,$ & $\,\,$\color{gr} 6.0712\color{black} $\,\,$ & $\,\,$3.0356$\,\,$ \\
$\,\,$0.8014$\,\,$ & $\,\,$ 1 $\,\,$ & $\,\,$\color{gr} 4.8652\color{black} $\,\,$ & $\,\,$2.4326  $\,\,$ \\
$\,\,$\color{gr} 0.1647\color{black} $\,\,$ & $\,\,$\color{gr} 0.2055\color{black} $\,\,$ & $\,\,$ 1 $\,\,$ & $\,\,$\color{gr} \color{blue}  1/2\color{black}  $\,\,$ \\
$\,\,$0.3294$\,\,$ & $\,\,$0.4111$\,\,$ & $\,\,$\color{gr} \color{blue} 2\color{black} $\,\,$ & $\,\,$ 1  $\,\,$ \\
\end{pmatrix},
\end{equation*}
\end{example}
\newpage
\begin{example}
\begin{equation*}
\mathbf{A} =
\begin{pmatrix}
$\,\,$ 1 $\,\,$ & $\,\,$2$\,\,$ & $\,\,$6$\,\,$ & $\,\,$2 $\,\,$ \\
$\,\,$ 1/2$\,\,$ & $\,\,$ 1 $\,\,$ & $\,\,$5$\,\,$ & $\,\,$4 $\,\,$ \\
$\,\,$ 1/6$\,\,$ & $\,\,$ 1/5$\,\,$ & $\,\,$ 1 $\,\,$ & $\,\,$ 1/2 $\,\,$ \\
$\,\,$ 1/2$\,\,$ & $\,\,$ 1/4$\,\,$ & $\,\,$2$\,\,$ & $\,\,$ 1  $\,\,$ \\
\end{pmatrix},
\qquad
\lambda_{\max} =
4.1655,
\qquad
CR = 0.0624
\end{equation*}

\begin{equation*}
\mathbf{w}^{cos} =
\begin{pmatrix}
0.432373\\
0.351467\\
\color{red} 0.068613\color{black} \\
0.147547
\end{pmatrix}\end{equation*}
\begin{equation*}
\left[ \frac{{w}^{cos}_i}{{w}^{cos}_j} \right] =
\begin{pmatrix}
$\,\,$ 1 $\,\,$ & $\,\,$1.2302$\,\,$ & $\,\,$\color{red} 6.3016\color{black} $\,\,$ & $\,\,$2.9304$\,\,$ \\
$\,\,$0.8129$\,\,$ & $\,\,$ 1 $\,\,$ & $\,\,$\color{red} 5.1225\color{black} $\,\,$ & $\,\,$2.3821  $\,\,$ \\
$\,\,$\color{red} 0.1587\color{black} $\,\,$ & $\,\,$\color{red} 0.1952\color{black} $\,\,$ & $\,\,$ 1 $\,\,$ & $\,\,$\color{red} 0.4650\color{black}  $\,\,$ \\
$\,\,$0.3412$\,\,$ & $\,\,$0.4198$\,\,$ & $\,\,$\color{red} 2.1504\color{black} $\,\,$ & $\,\,$ 1  $\,\,$ \\
\end{pmatrix},
\end{equation*}

\begin{equation*}
\mathbf{w}^{\prime} =
\begin{pmatrix}
0.431648\\
0.350877\\
0.070175\\
0.147299
\end{pmatrix} =
0.998322\cdot
\begin{pmatrix}
0.432373\\
0.351467\\
\color{gr} 0.070293\color{black} \\
0.147547
\end{pmatrix},
\end{equation*}
\begin{equation*}
\left[ \frac{{w}^{\prime}_i}{{w}^{\prime}_j} \right] =
\begin{pmatrix}
$\,\,$ 1 $\,\,$ & $\,\,$1.2302$\,\,$ & $\,\,$\color{gr} 6.1510\color{black} $\,\,$ & $\,\,$2.9304$\,\,$ \\
$\,\,$0.8129$\,\,$ & $\,\,$ 1 $\,\,$ & $\,\,$\color{gr} \color{blue} 5\color{black} $\,\,$ & $\,\,$2.3821  $\,\,$ \\
$\,\,$\color{gr} 0.1626\color{black} $\,\,$ & $\,\,$\color{gr} \color{blue}  1/5\color{black} $\,\,$ & $\,\,$ 1 $\,\,$ & $\,\,$\color{gr} 0.4764\color{black}  $\,\,$ \\
$\,\,$0.3412$\,\,$ & $\,\,$0.4198$\,\,$ & $\,\,$\color{gr} 2.0990\color{black} $\,\,$ & $\,\,$ 1  $\,\,$ \\
\end{pmatrix},
\end{equation*}
\end{example}
\newpage
\begin{example}
\begin{equation*}
\mathbf{A} =
\begin{pmatrix}
$\,\,$ 1 $\,\,$ & $\,\,$2$\,\,$ & $\,\,$6$\,\,$ & $\,\,$2 $\,\,$ \\
$\,\,$ 1/2$\,\,$ & $\,\,$ 1 $\,\,$ & $\,\,$5$\,\,$ & $\,\,$5 $\,\,$ \\
$\,\,$ 1/6$\,\,$ & $\,\,$ 1/5$\,\,$ & $\,\,$ 1 $\,\,$ & $\,\,$ 1/2 $\,\,$ \\
$\,\,$ 1/2$\,\,$ & $\,\,$ 1/5$\,\,$ & $\,\,$2$\,\,$ & $\,\,$ 1  $\,\,$ \\
\end{pmatrix},
\qquad
\lambda_{\max} =
4.2277,
\qquad
CR = 0.0859
\end{equation*}

\begin{equation*}
\mathbf{w}^{cos} =
\begin{pmatrix}
0.428562\\
0.363352\\
\color{red} 0.067008\color{black} \\
0.141077
\end{pmatrix}\end{equation*}
\begin{equation*}
\left[ \frac{{w}^{cos}_i}{{w}^{cos}_j} \right] =
\begin{pmatrix}
$\,\,$ 1 $\,\,$ & $\,\,$1.1795$\,\,$ & $\,\,$\color{red} 6.3956\color{black} $\,\,$ & $\,\,$3.0378$\,\,$ \\
$\,\,$0.8478$\,\,$ & $\,\,$ 1 $\,\,$ & $\,\,$\color{red} 5.4225\color{black} $\,\,$ & $\,\,$2.5756  $\,\,$ \\
$\,\,$\color{red} 0.1564\color{black} $\,\,$ & $\,\,$\color{red} 0.1844\color{black} $\,\,$ & $\,\,$ 1 $\,\,$ & $\,\,$\color{red} 0.4750\color{black}  $\,\,$ \\
$\,\,$0.3292$\,\,$ & $\,\,$0.3883$\,\,$ & $\,\,$\color{red} 2.1054\color{black} $\,\,$ & $\,\,$ 1  $\,\,$ \\
\end{pmatrix},
\end{equation*}

\begin{equation*}
\mathbf{w}^{\prime} =
\begin{pmatrix}
0.427055\\
0.362074\\
0.070290\\
0.140581
\end{pmatrix} =
0.996482\cdot
\begin{pmatrix}
0.428562\\
0.363352\\
\color{gr} 0.070539\color{black} \\
0.141077
\end{pmatrix},
\end{equation*}
\begin{equation*}
\left[ \frac{{w}^{\prime}_i}{{w}^{\prime}_j} \right] =
\begin{pmatrix}
$\,\,$ 1 $\,\,$ & $\,\,$1.1795$\,\,$ & $\,\,$\color{gr} 6.0756\color{black} $\,\,$ & $\,\,$3.0378$\,\,$ \\
$\,\,$0.8478$\,\,$ & $\,\,$ 1 $\,\,$ & $\,\,$\color{gr} 5.1511\color{black} $\,\,$ & $\,\,$2.5756  $\,\,$ \\
$\,\,$\color{gr} 0.1646\color{black} $\,\,$ & $\,\,$\color{gr} 0.1941\color{black} $\,\,$ & $\,\,$ 1 $\,\,$ & $\,\,$\color{gr} \color{blue}  1/2\color{black}  $\,\,$ \\
$\,\,$0.3292$\,\,$ & $\,\,$0.3883$\,\,$ & $\,\,$\color{gr} \color{blue} 2\color{black} $\,\,$ & $\,\,$ 1  $\,\,$ \\
\end{pmatrix},
\end{equation*}
\end{example}
\newpage
\begin{example}
\begin{equation*}
\mathbf{A} =
\begin{pmatrix}
$\,\,$ 1 $\,\,$ & $\,\,$2$\,\,$ & $\,\,$6$\,\,$ & $\,\,$4 $\,\,$ \\
$\,\,$ 1/2$\,\,$ & $\,\,$ 1 $\,\,$ & $\,\,$9$\,\,$ & $\,\,$3 $\,\,$ \\
$\,\,$ 1/6$\,\,$ & $\,\,$ 1/9$\,\,$ & $\,\,$ 1 $\,\,$ & $\,\,$ 1/2 $\,\,$ \\
$\,\,$ 1/4$\,\,$ & $\,\,$ 1/3$\,\,$ & $\,\,$2$\,\,$ & $\,\,$ 1  $\,\,$ \\
\end{pmatrix},
\qquad
\lambda_{\max} =
4.1031,
\qquad
CR = 0.0389
\end{equation*}

\begin{equation*}
\mathbf{w}^{cos} =
\begin{pmatrix}
0.474994\\
0.351694\\
0.058951\\
\color{red} 0.114362\color{black}
\end{pmatrix}\end{equation*}
\begin{equation*}
\left[ \frac{{w}^{cos}_i}{{w}^{cos}_j} \right] =
\begin{pmatrix}
$\,\,$ 1 $\,\,$ & $\,\,$1.3506$\,\,$ & $\,\,$8.0574$\,\,$ & $\,\,$\color{red} 4.1534\color{black} $\,\,$ \\
$\,\,$0.7404$\,\,$ & $\,\,$ 1 $\,\,$ & $\,\,$5.9658$\,\,$ & $\,\,$\color{red} 3.0753\color{black}   $\,\,$ \\
$\,\,$0.1241$\,\,$ & $\,\,$0.1676$\,\,$ & $\,\,$ 1 $\,\,$ & $\,\,$\color{red} 0.5155\color{black}  $\,\,$ \\
$\,\,$\color{red} 0.2408\color{black} $\,\,$ & $\,\,$\color{red} 0.3252\color{black} $\,\,$ & $\,\,$\color{red} 1.9399\color{black} $\,\,$ & $\,\,$ 1  $\,\,$ \\
\end{pmatrix},
\end{equation*}

\begin{equation*}
\mathbf{w}^{\prime} =
\begin{pmatrix}
0.473634\\
0.350687\\
0.058782\\
0.116896
\end{pmatrix} =
0.997139\cdot
\begin{pmatrix}
0.474994\\
0.351694\\
0.058951\\
\color{gr} 0.117231\color{black}
\end{pmatrix},
\end{equation*}
\begin{equation*}
\left[ \frac{{w}^{\prime}_i}{{w}^{\prime}_j} \right] =
\begin{pmatrix}
$\,\,$ 1 $\,\,$ & $\,\,$1.3506$\,\,$ & $\,\,$8.0574$\,\,$ & $\,\,$\color{gr} 4.0518\color{black} $\,\,$ \\
$\,\,$0.7404$\,\,$ & $\,\,$ 1 $\,\,$ & $\,\,$5.9658$\,\,$ & $\,\,$\color{gr} \color{blue} 3\color{black}   $\,\,$ \\
$\,\,$0.1241$\,\,$ & $\,\,$0.1676$\,\,$ & $\,\,$ 1 $\,\,$ & $\,\,$\color{gr} 0.5029\color{black}  $\,\,$ \\
$\,\,$\color{gr} 0.2468\color{black} $\,\,$ & $\,\,$\color{gr} \color{blue}  1/3\color{black} $\,\,$ & $\,\,$\color{gr} 1.9886\color{black} $\,\,$ & $\,\,$ 1  $\,\,$ \\
\end{pmatrix},
\end{equation*}
\end{example}
\newpage
\begin{example}
\begin{equation*}
\mathbf{A} =
\begin{pmatrix}
$\,\,$ 1 $\,\,$ & $\,\,$2$\,\,$ & $\,\,$6$\,\,$ & $\,\,$5 $\,\,$ \\
$\,\,$ 1/2$\,\,$ & $\,\,$ 1 $\,\,$ & $\,\,$2$\,\,$ & $\,\,$5 $\,\,$ \\
$\,\,$ 1/6$\,\,$ & $\,\,$ 1/2$\,\,$ & $\,\,$ 1 $\,\,$ & $\,\,$ 1/2 $\,\,$ \\
$\,\,$ 1/5$\,\,$ & $\,\,$ 1/5$\,\,$ & $\,\,$2$\,\,$ & $\,\,$ 1  $\,\,$ \\
\end{pmatrix},
\qquad
\lambda_{\max} =
4.2277,
\qquad
CR = 0.0859
\end{equation*}

\begin{equation*}
\mathbf{w}^{cos} =
\begin{pmatrix}
\color{red} 0.514359\color{black} \\
0.288062\\
0.089723\\
0.107856
\end{pmatrix}\end{equation*}
\begin{equation*}
\left[ \frac{{w}^{cos}_i}{{w}^{cos}_j} \right] =
\begin{pmatrix}
$\,\,$ 1 $\,\,$ & $\,\,$\color{red} 1.7856\color{black} $\,\,$ & $\,\,$\color{red} 5.7327\color{black} $\,\,$ & $\,\,$\color{red} 4.7690\color{black} $\,\,$ \\
$\,\,$\color{red} 0.5600\color{black} $\,\,$ & $\,\,$ 1 $\,\,$ & $\,\,$3.2106$\,\,$ & $\,\,$2.6708  $\,\,$ \\
$\,\,$\color{red} 0.1744\color{black} $\,\,$ & $\,\,$0.3115$\,\,$ & $\,\,$ 1 $\,\,$ & $\,\,$0.8319 $\,\,$ \\
$\,\,$\color{red} 0.2097\color{black} $\,\,$ & $\,\,$0.3744$\,\,$ & $\,\,$1.2021$\,\,$ & $\,\,$ 1  $\,\,$ \\
\end{pmatrix},
\end{equation*}

\begin{equation*}
\mathbf{w}^{\prime} =
\begin{pmatrix}
0.525732\\
0.281316\\
0.087622\\
0.105330
\end{pmatrix} =
0.976582\cdot
\begin{pmatrix}
\color{gr} 0.538339\color{black} \\
0.288062\\
0.089723\\
0.107856
\end{pmatrix},
\end{equation*}
\begin{equation*}
\left[ \frac{{w}^{\prime}_i}{{w}^{\prime}_j} \right] =
\begin{pmatrix}
$\,\,$ 1 $\,\,$ & $\,\,$\color{gr} 1.8688\color{black} $\,\,$ & $\,\,$\color{gr} \color{blue} 6\color{black} $\,\,$ & $\,\,$\color{gr} 4.9913\color{black} $\,\,$ \\
$\,\,$\color{gr} 0.5351\color{black} $\,\,$ & $\,\,$ 1 $\,\,$ & $\,\,$3.2106$\,\,$ & $\,\,$2.6708  $\,\,$ \\
$\,\,$\color{gr} \color{blue}  1/6\color{black} $\,\,$ & $\,\,$0.3115$\,\,$ & $\,\,$ 1 $\,\,$ & $\,\,$0.8319 $\,\,$ \\
$\,\,$\color{gr} 0.2003\color{black} $\,\,$ & $\,\,$0.3744$\,\,$ & $\,\,$1.2021$\,\,$ & $\,\,$ 1  $\,\,$ \\
\end{pmatrix},
\end{equation*}
\end{example}
\newpage
\begin{example}
\begin{equation*}
\mathbf{A} =
\begin{pmatrix}
$\,\,$ 1 $\,\,$ & $\,\,$2$\,\,$ & $\,\,$7$\,\,$ & $\,\,$2 $\,\,$ \\
$\,\,$ 1/2$\,\,$ & $\,\,$ 1 $\,\,$ & $\,\,$6$\,\,$ & $\,\,$5 $\,\,$ \\
$\,\,$ 1/7$\,\,$ & $\,\,$ 1/6$\,\,$ & $\,\,$ 1 $\,\,$ & $\,\,$ 1/2 $\,\,$ \\
$\,\,$ 1/2$\,\,$ & $\,\,$ 1/5$\,\,$ & $\,\,$2$\,\,$ & $\,\,$ 1  $\,\,$ \\
\end{pmatrix},
\qquad
\lambda_{\max} =
4.2251,
\qquad
CR = 0.0849
\end{equation*}

\begin{equation*}
\mathbf{w}^{cos} =
\begin{pmatrix}
0.433219\\
0.369856\\
\color{red} 0.059735\color{black} \\
0.137189
\end{pmatrix}\end{equation*}
\begin{equation*}
\left[ \frac{{w}^{cos}_i}{{w}^{cos}_j} \right] =
\begin{pmatrix}
$\,\,$ 1 $\,\,$ & $\,\,$1.1713$\,\,$ & $\,\,$\color{red} 7.2523\color{black} $\,\,$ & $\,\,$3.1578$\,\,$ \\
$\,\,$0.8537$\,\,$ & $\,\,$ 1 $\,\,$ & $\,\,$\color{red} 6.1916\color{black} $\,\,$ & $\,\,$2.6960  $\,\,$ \\
$\,\,$\color{red} 0.1379\color{black} $\,\,$ & $\,\,$\color{red} 0.1615\color{black} $\,\,$ & $\,\,$ 1 $\,\,$ & $\,\,$\color{red} 0.4354\color{black}  $\,\,$ \\
$\,\,$0.3167$\,\,$ & $\,\,$0.3709$\,\,$ & $\,\,$\color{red} 2.2966\color{black} $\,\,$ & $\,\,$ 1  $\,\,$ \\
\end{pmatrix},
\end{equation*}

\begin{equation*}
\mathbf{w}^{\prime} =
\begin{pmatrix}
0.432395\\
0.369152\\
0.061525\\
0.136928
\end{pmatrix} =
0.998096\cdot
\begin{pmatrix}
0.433219\\
0.369856\\
\color{gr} 0.061643\color{black} \\
0.137189
\end{pmatrix},
\end{equation*}
\begin{equation*}
\left[ \frac{{w}^{\prime}_i}{{w}^{\prime}_j} \right] =
\begin{pmatrix}
$\,\,$ 1 $\,\,$ & $\,\,$1.1713$\,\,$ & $\,\,$\color{gr} 7.0279\color{black} $\,\,$ & $\,\,$3.1578$\,\,$ \\
$\,\,$0.8537$\,\,$ & $\,\,$ 1 $\,\,$ & $\,\,$\color{gr} \color{blue} 6\color{black} $\,\,$ & $\,\,$2.6960  $\,\,$ \\
$\,\,$\color{gr} 0.1423\color{black} $\,\,$ & $\,\,$\color{gr} \color{blue}  1/6\color{black} $\,\,$ & $\,\,$ 1 $\,\,$ & $\,\,$\color{gr} 0.4493\color{black}  $\,\,$ \\
$\,\,$0.3167$\,\,$ & $\,\,$0.3709$\,\,$ & $\,\,$\color{gr} 2.2256\color{black} $\,\,$ & $\,\,$ 1  $\,\,$ \\
\end{pmatrix},
\end{equation*}
\end{example}
\newpage
\begin{example}
\begin{equation*}
\mathbf{A} =
\begin{pmatrix}
$\,\,$ 1 $\,\,$ & $\,\,$2$\,\,$ & $\,\,$7$\,\,$ & $\,\,$3 $\,\,$ \\
$\,\,$ 1/2$\,\,$ & $\,\,$ 1 $\,\,$ & $\,\,$8$\,\,$ & $\,\,$2 $\,\,$ \\
$\,\,$ 1/7$\,\,$ & $\,\,$ 1/8$\,\,$ & $\,\,$ 1 $\,\,$ & $\,\,$ 1/3 $\,\,$ \\
$\,\,$ 1/3$\,\,$ & $\,\,$ 1/2$\,\,$ & $\,\,$3$\,\,$ & $\,\,$ 1  $\,\,$ \\
\end{pmatrix},
\qquad
\lambda_{\max} =
4.0576,
\qquad
CR = 0.0217
\end{equation*}

\begin{equation*}
\mathbf{w}^{cos} =
\begin{pmatrix}
0.473282\\
0.317548\\
0.053289\\
\color{red} 0.155881\color{black}
\end{pmatrix}\end{equation*}
\begin{equation*}
\left[ \frac{{w}^{cos}_i}{{w}^{cos}_j} \right] =
\begin{pmatrix}
$\,\,$ 1 $\,\,$ & $\,\,$1.4904$\,\,$ & $\,\,$8.8814$\,\,$ & $\,\,$\color{red} 3.0362\color{black} $\,\,$ \\
$\,\,$0.6709$\,\,$ & $\,\,$ 1 $\,\,$ & $\,\,$5.9589$\,\,$ & $\,\,$\color{red} 2.0371\color{black}   $\,\,$ \\
$\,\,$0.1126$\,\,$ & $\,\,$0.1678$\,\,$ & $\,\,$ 1 $\,\,$ & $\,\,$\color{red} 0.3419\color{black}  $\,\,$ \\
$\,\,$\color{red} 0.3294\color{black} $\,\,$ & $\,\,$\color{red} 0.4909\color{black} $\,\,$ & $\,\,$\color{red} 2.9252\color{black} $\,\,$ & $\,\,$ 1  $\,\,$ \\
\end{pmatrix},
\end{equation*}

\begin{equation*}
\mathbf{w}^{\prime} =
\begin{pmatrix}
0.472394\\
0.316952\\
0.053189\\
0.157465
\end{pmatrix} =
0.998124\cdot
\begin{pmatrix}
0.473282\\
0.317548\\
0.053289\\
\color{gr} 0.157761\color{black}
\end{pmatrix},
\end{equation*}
\begin{equation*}
\left[ \frac{{w}^{\prime}_i}{{w}^{\prime}_j} \right] =
\begin{pmatrix}
$\,\,$ 1 $\,\,$ & $\,\,$1.4904$\,\,$ & $\,\,$8.8814$\,\,$ & $\,\,$\color{gr} \color{blue} 3\color{black} $\,\,$ \\
$\,\,$0.6709$\,\,$ & $\,\,$ 1 $\,\,$ & $\,\,$5.9589$\,\,$ & $\,\,$\color{gr} 2.0128\color{black}   $\,\,$ \\
$\,\,$0.1126$\,\,$ & $\,\,$0.1678$\,\,$ & $\,\,$ 1 $\,\,$ & $\,\,$\color{gr} 0.3378\color{black}  $\,\,$ \\
$\,\,$\color{gr} \color{blue}  1/3\color{black} $\,\,$ & $\,\,$\color{gr} 0.4968\color{black} $\,\,$ & $\,\,$\color{gr} 2.9605\color{black} $\,\,$ & $\,\,$ 1  $\,\,$ \\
\end{pmatrix},
\end{equation*}
\end{example}
\newpage
\begin{example}
\begin{equation*}
\mathbf{A} =
\begin{pmatrix}
$\,\,$ 1 $\,\,$ & $\,\,$2$\,\,$ & $\,\,$7$\,\,$ & $\,\,$3 $\,\,$ \\
$\,\,$ 1/2$\,\,$ & $\,\,$ 1 $\,\,$ & $\,\,$9$\,\,$ & $\,\,$2 $\,\,$ \\
$\,\,$ 1/7$\,\,$ & $\,\,$ 1/9$\,\,$ & $\,\,$ 1 $\,\,$ & $\,\,$ 1/3 $\,\,$ \\
$\,\,$ 1/3$\,\,$ & $\,\,$ 1/2$\,\,$ & $\,\,$3$\,\,$ & $\,\,$ 1  $\,\,$ \\
\end{pmatrix},
\qquad
\lambda_{\max} =
4.0762,
\qquad
CR = 0.0287
\end{equation*}

\begin{equation*}
\mathbf{w}^{cos} =
\begin{pmatrix}
0.469386\\
0.324856\\
0.051755\\
\color{red} 0.154003\color{black}
\end{pmatrix}\end{equation*}
\begin{equation*}
\left[ \frac{{w}^{cos}_i}{{w}^{cos}_j} \right] =
\begin{pmatrix}
$\,\,$ 1 $\,\,$ & $\,\,$1.4449$\,\,$ & $\,\,$9.0694$\,\,$ & $\,\,$\color{red} 3.0479\color{black} $\,\,$ \\
$\,\,$0.6921$\,\,$ & $\,\,$ 1 $\,\,$ & $\,\,$6.2768$\,\,$ & $\,\,$\color{red} 2.1094\color{black}   $\,\,$ \\
$\,\,$0.1103$\,\,$ & $\,\,$0.1593$\,\,$ & $\,\,$ 1 $\,\,$ & $\,\,$\color{red} 0.3361\color{black}  $\,\,$ \\
$\,\,$\color{red} 0.3281\color{black} $\,\,$ & $\,\,$\color{red} 0.4741\color{black} $\,\,$ & $\,\,$\color{red} 2.9756\color{black} $\,\,$ & $\,\,$ 1  $\,\,$ \\
\end{pmatrix},
\end{equation*}

\begin{equation*}
\mathbf{w}^{\prime} =
\begin{pmatrix}
0.468794\\
0.324446\\
0.051690\\
0.155070
\end{pmatrix} =
0.998739\cdot
\begin{pmatrix}
0.469386\\
0.324856\\
0.051755\\
\color{gr} 0.155265\color{black}
\end{pmatrix},
\end{equation*}
\begin{equation*}
\left[ \frac{{w}^{\prime}_i}{{w}^{\prime}_j} \right] =
\begin{pmatrix}
$\,\,$ 1 $\,\,$ & $\,\,$1.4449$\,\,$ & $\,\,$9.0694$\,\,$ & $\,\,$\color{gr} 3.0231\color{black} $\,\,$ \\
$\,\,$0.6921$\,\,$ & $\,\,$ 1 $\,\,$ & $\,\,$6.2768$\,\,$ & $\,\,$\color{gr} 2.0923\color{black}   $\,\,$ \\
$\,\,$0.1103$\,\,$ & $\,\,$0.1593$\,\,$ & $\,\,$ 1 $\,\,$ & $\,\,$\color{gr} \color{blue}  1/3\color{black}  $\,\,$ \\
$\,\,$\color{gr} 0.3308\color{black} $\,\,$ & $\,\,$\color{gr} 0.4780\color{black} $\,\,$ & $\,\,$\color{gr} \color{blue} 3\color{black} $\,\,$ & $\,\,$ 1  $\,\,$ \\
\end{pmatrix},
\end{equation*}
\end{example}
\newpage
\begin{example}
\begin{equation*}
\mathbf{A} =
\begin{pmatrix}
$\,\,$ 1 $\,\,$ & $\,\,$2$\,\,$ & $\,\,$8$\,\,$ & $\,\,$2 $\,\,$ \\
$\,\,$ 1/2$\,\,$ & $\,\,$ 1 $\,\,$ & $\,\,$3$\,\,$ & $\,\,$2 $\,\,$ \\
$\,\,$ 1/8$\,\,$ & $\,\,$ 1/3$\,\,$ & $\,\,$ 1 $\,\,$ & $\,\,$ 1/7 $\,\,$ \\
$\,\,$ 1/2$\,\,$ & $\,\,$ 1/2$\,\,$ & $\,\,$7$\,\,$ & $\,\,$ 1  $\,\,$ \\
\end{pmatrix},
\qquad
\lambda_{\max} =
4.2109,
\qquad
CR = 0.0795
\end{equation*}

\begin{equation*}
\mathbf{w}^{cos} =
\begin{pmatrix}
\color{red} 0.450011\color{black} \\
0.260672\\
0.056298\\
0.233020
\end{pmatrix}\end{equation*}
\begin{equation*}
\left[ \frac{{w}^{cos}_i}{{w}^{cos}_j} \right] =
\begin{pmatrix}
$\,\,$ 1 $\,\,$ & $\,\,$\color{red} 1.7263\color{black} $\,\,$ & $\,\,$\color{red} 7.9934\color{black} $\,\,$ & $\,\,$\color{red} 1.9312\color{black} $\,\,$ \\
$\,\,$\color{red} 0.5793\color{black} $\,\,$ & $\,\,$ 1 $\,\,$ & $\,\,$4.6302$\,\,$ & $\,\,$1.1187  $\,\,$ \\
$\,\,$\color{red} 0.1251\color{black} $\,\,$ & $\,\,$0.2160$\,\,$ & $\,\,$ 1 $\,\,$ & $\,\,$0.2416 $\,\,$ \\
$\,\,$\color{red} 0.5178\color{black} $\,\,$ & $\,\,$0.8939$\,\,$ & $\,\,$4.1391$\,\,$ & $\,\,$ 1  $\,\,$ \\
\end{pmatrix},
\end{equation*}

\begin{equation*}
\mathbf{w}^{\prime} =
\begin{pmatrix}
0.450215\\
0.260575\\
0.056277\\
0.232933
\end{pmatrix} =
0.999628\cdot
\begin{pmatrix}
\color{gr} 0.450383\color{black} \\
0.260672\\
0.056298\\
0.233020
\end{pmatrix},
\end{equation*}
\begin{equation*}
\left[ \frac{{w}^{\prime}_i}{{w}^{\prime}_j} \right] =
\begin{pmatrix}
$\,\,$ 1 $\,\,$ & $\,\,$\color{gr} 1.7278\color{black} $\,\,$ & $\,\,$\color{gr} \color{blue} 8\color{black} $\,\,$ & $\,\,$\color{gr} 1.9328\color{black} $\,\,$ \\
$\,\,$\color{gr} 0.5788\color{black} $\,\,$ & $\,\,$ 1 $\,\,$ & $\,\,$4.6302$\,\,$ & $\,\,$1.1187  $\,\,$ \\
$\,\,$\color{gr} \color{blue}  1/8\color{black} $\,\,$ & $\,\,$0.2160$\,\,$ & $\,\,$ 1 $\,\,$ & $\,\,$0.2416 $\,\,$ \\
$\,\,$\color{gr} 0.5174\color{black} $\,\,$ & $\,\,$0.8939$\,\,$ & $\,\,$4.1391$\,\,$ & $\,\,$ 1  $\,\,$ \\
\end{pmatrix},
\end{equation*}
\end{example}
\newpage
\begin{example}
\begin{equation*}
\mathbf{A} =
\begin{pmatrix}
$\,\,$ 1 $\,\,$ & $\,\,$2$\,\,$ & $\,\,$8$\,\,$ & $\,\,$2 $\,\,$ \\
$\,\,$ 1/2$\,\,$ & $\,\,$ 1 $\,\,$ & $\,\,$6$\,\,$ & $\,\,$3 $\,\,$ \\
$\,\,$ 1/8$\,\,$ & $\,\,$ 1/6$\,\,$ & $\,\,$ 1 $\,\,$ & $\,\,$ 1/3 $\,\,$ \\
$\,\,$ 1/2$\,\,$ & $\,\,$ 1/3$\,\,$ & $\,\,$3$\,\,$ & $\,\,$ 1  $\,\,$ \\
\end{pmatrix},
\qquad
\lambda_{\max} =
4.1031,
\qquad
CR = 0.0389
\end{equation*}

\begin{equation*}
\mathbf{w}^{cos} =
\begin{pmatrix}
0.448083\\
0.332416\\
\color{red} 0.053826\color{black} \\
0.165675
\end{pmatrix}\end{equation*}
\begin{equation*}
\left[ \frac{{w}^{cos}_i}{{w}^{cos}_j} \right] =
\begin{pmatrix}
$\,\,$ 1 $\,\,$ & $\,\,$1.3480$\,\,$ & $\,\,$\color{red} 8.3247\color{black} $\,\,$ & $\,\,$2.7046$\,\,$ \\
$\,\,$0.7419$\,\,$ & $\,\,$ 1 $\,\,$ & $\,\,$\color{red} 6.1758\color{black} $\,\,$ & $\,\,$2.0064  $\,\,$ \\
$\,\,$\color{red} 0.1201\color{black} $\,\,$ & $\,\,$\color{red} 0.1619\color{black} $\,\,$ & $\,\,$ 1 $\,\,$ & $\,\,$\color{red} 0.3249\color{black}  $\,\,$ \\
$\,\,$0.3697$\,\,$ & $\,\,$0.4984$\,\,$ & $\,\,$\color{red} 3.0780\color{black} $\,\,$ & $\,\,$ 1  $\,\,$ \\
\end{pmatrix},
\end{equation*}

\begin{equation*}
\mathbf{w}^{\prime} =
\begin{pmatrix}
0.447456\\
0.331952\\
0.055148\\
0.165444
\end{pmatrix} =
0.998602\cdot
\begin{pmatrix}
0.448083\\
0.332416\\
\color{gr} 0.055225\color{black} \\
0.165675
\end{pmatrix},
\end{equation*}
\begin{equation*}
\left[ \frac{{w}^{\prime}_i}{{w}^{\prime}_j} \right] =
\begin{pmatrix}
$\,\,$ 1 $\,\,$ & $\,\,$1.3480$\,\,$ & $\,\,$\color{gr} 8.1137\color{black} $\,\,$ & $\,\,$2.7046$\,\,$ \\
$\,\,$0.7419$\,\,$ & $\,\,$ 1 $\,\,$ & $\,\,$\color{gr} 6.0193\color{black} $\,\,$ & $\,\,$2.0064  $\,\,$ \\
$\,\,$\color{gr} 0.1232\color{black} $\,\,$ & $\,\,$\color{gr} 0.1661\color{black} $\,\,$ & $\,\,$ 1 $\,\,$ & $\,\,$\color{gr} \color{blue}  1/3\color{black}  $\,\,$ \\
$\,\,$0.3697$\,\,$ & $\,\,$0.4984$\,\,$ & $\,\,$\color{gr} \color{blue} 3\color{black} $\,\,$ & $\,\,$ 1  $\,\,$ \\
\end{pmatrix},
\end{equation*}
\end{example}
\newpage
\begin{example}
\begin{equation*}
\mathbf{A} =
\begin{pmatrix}
$\,\,$ 1 $\,\,$ & $\,\,$2$\,\,$ & $\,\,$8$\,\,$ & $\,\,$2 $\,\,$ \\
$\,\,$ 1/2$\,\,$ & $\,\,$ 1 $\,\,$ & $\,\,$7$\,\,$ & $\,\,$4 $\,\,$ \\
$\,\,$ 1/8$\,\,$ & $\,\,$ 1/7$\,\,$ & $\,\,$ 1 $\,\,$ & $\,\,$ 1/3 $\,\,$ \\
$\,\,$ 1/2$\,\,$ & $\,\,$ 1/4$\,\,$ & $\,\,$3$\,\,$ & $\,\,$ 1  $\,\,$ \\
\end{pmatrix},
\qquad
\lambda_{\max} =
4.1681,
\qquad
CR = 0.0634
\end{equation*}

\begin{equation*}
\mathbf{w}^{cos} =
\begin{pmatrix}
0.436683\\
0.359534\\
\color{red} 0.050092\color{black} \\
0.153691
\end{pmatrix}\end{equation*}
\begin{equation*}
\left[ \frac{{w}^{cos}_i}{{w}^{cos}_j} \right] =
\begin{pmatrix}
$\,\,$ 1 $\,\,$ & $\,\,$1.2146$\,\,$ & $\,\,$\color{red} 8.7176\color{black} $\,\,$ & $\,\,$2.8413$\,\,$ \\
$\,\,$0.8233$\,\,$ & $\,\,$ 1 $\,\,$ & $\,\,$\color{red} 7.1774\color{black} $\,\,$ & $\,\,$2.3393  $\,\,$ \\
$\,\,$\color{red} 0.1147\color{black} $\,\,$ & $\,\,$\color{red} 0.1393\color{black} $\,\,$ & $\,\,$ 1 $\,\,$ & $\,\,$\color{red} 0.3259\color{black}  $\,\,$ \\
$\,\,$0.3520$\,\,$ & $\,\,$0.4275$\,\,$ & $\,\,$\color{red} 3.0682\color{black} $\,\,$ & $\,\,$ 1  $\,\,$ \\
\end{pmatrix},
\end{equation*}

\begin{equation*}
\mathbf{w}^{\prime} =
\begin{pmatrix}
0.436187\\
0.359125\\
0.051172\\
0.153516
\end{pmatrix} =
0.998863\cdot
\begin{pmatrix}
0.436683\\
0.359534\\
\color{gr} 0.051230\color{black} \\
0.153691
\end{pmatrix},
\end{equation*}
\begin{equation*}
\left[ \frac{{w}^{\prime}_i}{{w}^{\prime}_j} \right] =
\begin{pmatrix}
$\,\,$ 1 $\,\,$ & $\,\,$1.2146$\,\,$ & $\,\,$\color{gr} 8.5239\color{black} $\,\,$ & $\,\,$2.8413$\,\,$ \\
$\,\,$0.8233$\,\,$ & $\,\,$ 1 $\,\,$ & $\,\,$\color{gr} 7.0180\color{black} $\,\,$ & $\,\,$2.3393  $\,\,$ \\
$\,\,$\color{gr} 0.1173\color{black} $\,\,$ & $\,\,$\color{gr} 0.1425\color{black} $\,\,$ & $\,\,$ 1 $\,\,$ & $\,\,$\color{gr} \color{blue}  1/3\color{black}  $\,\,$ \\
$\,\,$0.3520$\,\,$ & $\,\,$0.4275$\,\,$ & $\,\,$\color{gr} \color{blue} 3\color{black} $\,\,$ & $\,\,$ 1  $\,\,$ \\
\end{pmatrix},
\end{equation*}
\end{example}
\newpage
\begin{example}
\begin{equation*}
\mathbf{A} =
\begin{pmatrix}
$\,\,$ 1 $\,\,$ & $\,\,$2$\,\,$ & $\,\,$8$\,\,$ & $\,\,$2 $\,\,$ \\
$\,\,$ 1/2$\,\,$ & $\,\,$ 1 $\,\,$ & $\,\,$8$\,\,$ & $\,\,$5 $\,\,$ \\
$\,\,$ 1/8$\,\,$ & $\,\,$ 1/8$\,\,$ & $\,\,$ 1 $\,\,$ & $\,\,$ 1/3 $\,\,$ \\
$\,\,$ 1/2$\,\,$ & $\,\,$ 1/5$\,\,$ & $\,\,$3$\,\,$ & $\,\,$ 1  $\,\,$ \\
\end{pmatrix},
\qquad
\lambda_{\max} =
4.2311,
\qquad
CR = 0.0871
\end{equation*}

\begin{equation*}
\mathbf{w}^{cos} =
\begin{pmatrix}
0.427519\\
0.380184\\
\color{red} 0.047152\color{black} \\
0.145144
\end{pmatrix}\end{equation*}
\begin{equation*}
\left[ \frac{{w}^{cos}_i}{{w}^{cos}_j} \right] =
\begin{pmatrix}
$\,\,$ 1 $\,\,$ & $\,\,$1.1245$\,\,$ & $\,\,$\color{red} 9.0669\color{black} $\,\,$ & $\,\,$2.9455$\,\,$ \\
$\,\,$0.8893$\,\,$ & $\,\,$ 1 $\,\,$ & $\,\,$\color{red} 8.0630\color{black} $\,\,$ & $\,\,$2.6194  $\,\,$ \\
$\,\,$\color{red} 0.1103\color{black} $\,\,$ & $\,\,$\color{red} 0.1240\color{black} $\,\,$ & $\,\,$ 1 $\,\,$ & $\,\,$\color{red} 0.3249\color{black}  $\,\,$ \\
$\,\,$0.3395$\,\,$ & $\,\,$0.3818$\,\,$ & $\,\,$\color{red} 3.0782\color{black} $\,\,$ & $\,\,$ 1  $\,\,$ \\
\end{pmatrix},
\end{equation*}

\begin{equation*}
\mathbf{w}^{\prime} =
\begin{pmatrix}
0.427361\\
0.380043\\
0.047505\\
0.145091
\end{pmatrix} =
0.999629\cdot
\begin{pmatrix}
0.427519\\
0.380184\\
\color{gr} 0.047523\color{black} \\
0.145144
\end{pmatrix},
\end{equation*}
\begin{equation*}
\left[ \frac{{w}^{\prime}_i}{{w}^{\prime}_j} \right] =
\begin{pmatrix}
$\,\,$ 1 $\,\,$ & $\,\,$1.1245$\,\,$ & $\,\,$\color{gr} 8.9960\color{black} $\,\,$ & $\,\,$2.9455$\,\,$ \\
$\,\,$0.8893$\,\,$ & $\,\,$ 1 $\,\,$ & $\,\,$\color{gr} \color{blue} 8\color{black} $\,\,$ & $\,\,$2.6194  $\,\,$ \\
$\,\,$\color{gr} 0.1112\color{black} $\,\,$ & $\,\,$\color{gr} \color{blue}  1/8\color{black} $\,\,$ & $\,\,$ 1 $\,\,$ & $\,\,$\color{gr} 0.3274\color{black}  $\,\,$ \\
$\,\,$0.3395$\,\,$ & $\,\,$0.3818$\,\,$ & $\,\,$\color{gr} 3.0542\color{black} $\,\,$ & $\,\,$ 1  $\,\,$ \\
\end{pmatrix},
\end{equation*}
\end{example}
\newpage
\begin{example}
\begin{equation*}
\mathbf{A} =
\begin{pmatrix}
$\,\,$ 1 $\,\,$ & $\,\,$2$\,\,$ & $\,\,$8$\,\,$ & $\,\,$3 $\,\,$ \\
$\,\,$ 1/2$\,\,$ & $\,\,$ 1 $\,\,$ & $\,\,$3$\,\,$ & $\,\,$2 $\,\,$ \\
$\,\,$ 1/8$\,\,$ & $\,\,$ 1/3$\,\,$ & $\,\,$ 1 $\,\,$ & $\,\,$ 1/4 $\,\,$ \\
$\,\,$ 1/3$\,\,$ & $\,\,$ 1/2$\,\,$ & $\,\,$4$\,\,$ & $\,\,$ 1  $\,\,$ \\
\end{pmatrix},
\qquad
\lambda_{\max} =
4.0820,
\qquad
CR = 0.0309
\end{equation*}

\begin{equation*}
\mathbf{w}^{cos} =
\begin{pmatrix}
\color{red} 0.503057\color{black} \\
0.255710\\
0.063280\\
0.177953
\end{pmatrix}\end{equation*}
\begin{equation*}
\left[ \frac{{w}^{cos}_i}{{w}^{cos}_j} \right] =
\begin{pmatrix}
$\,\,$ 1 $\,\,$ & $\,\,$\color{red} 1.9673\color{black} $\,\,$ & $\,\,$\color{red} 7.9497\color{black} $\,\,$ & $\,\,$\color{red} 2.8269\color{black} $\,\,$ \\
$\,\,$\color{red} 0.5083\color{black} $\,\,$ & $\,\,$ 1 $\,\,$ & $\,\,$4.0409$\,\,$ & $\,\,$1.4370  $\,\,$ \\
$\,\,$\color{red} 0.1258\color{black} $\,\,$ & $\,\,$0.2475$\,\,$ & $\,\,$ 1 $\,\,$ & $\,\,$0.3556 $\,\,$ \\
$\,\,$\color{red} 0.3537\color{black} $\,\,$ & $\,\,$0.6959$\,\,$ & $\,\,$2.8122$\,\,$ & $\,\,$ 1  $\,\,$ \\
\end{pmatrix},
\end{equation*}

\begin{equation*}
\mathbf{w}^{\prime} =
\begin{pmatrix}
0.504634\\
0.254899\\
0.063079\\
0.177388
\end{pmatrix} =
0.996826\cdot
\begin{pmatrix}
\color{gr} 0.506240\color{black} \\
0.255710\\
0.063280\\
0.177953
\end{pmatrix},
\end{equation*}
\begin{equation*}
\left[ \frac{{w}^{\prime}_i}{{w}^{\prime}_j} \right] =
\begin{pmatrix}
$\,\,$ 1 $\,\,$ & $\,\,$\color{gr} 1.9797\color{black} $\,\,$ & $\,\,$\color{gr} \color{blue} 8\color{black} $\,\,$ & $\,\,$\color{gr} 2.8448\color{black} $\,\,$ \\
$\,\,$\color{gr} 0.5051\color{black} $\,\,$ & $\,\,$ 1 $\,\,$ & $\,\,$4.0409$\,\,$ & $\,\,$1.4370  $\,\,$ \\
$\,\,$\color{gr} \color{blue}  1/8\color{black} $\,\,$ & $\,\,$0.2475$\,\,$ & $\,\,$ 1 $\,\,$ & $\,\,$0.3556 $\,\,$ \\
$\,\,$\color{gr} 0.3515\color{black} $\,\,$ & $\,\,$0.6959$\,\,$ & $\,\,$2.8122$\,\,$ & $\,\,$ 1  $\,\,$ \\
\end{pmatrix},
\end{equation*}
\end{example}
\newpage
\begin{example}
\begin{equation*}
\mathbf{A} =
\begin{pmatrix}
$\,\,$ 1 $\,\,$ & $\,\,$2$\,\,$ & $\,\,$8$\,\,$ & $\,\,$3 $\,\,$ \\
$\,\,$ 1/2$\,\,$ & $\,\,$ 1 $\,\,$ & $\,\,$6$\,\,$ & $\,\,$5 $\,\,$ \\
$\,\,$ 1/8$\,\,$ & $\,\,$ 1/6$\,\,$ & $\,\,$ 1 $\,\,$ & $\,\,$ 1/2 $\,\,$ \\
$\,\,$ 1/3$\,\,$ & $\,\,$ 1/5$\,\,$ & $\,\,$2$\,\,$ & $\,\,$ 1  $\,\,$ \\
\end{pmatrix},
\qquad
\lambda_{\max} =
4.1252,
\qquad
CR = 0.0472
\end{equation*}

\begin{equation*}
\mathbf{w}^{cos} =
\begin{pmatrix}
0.471181\\
0.357746\\
\color{red} 0.056412\color{black} \\
0.114661
\end{pmatrix}\end{equation*}
\begin{equation*}
\left[ \frac{{w}^{cos}_i}{{w}^{cos}_j} \right] =
\begin{pmatrix}
$\,\,$ 1 $\,\,$ & $\,\,$1.3171$\,\,$ & $\,\,$\color{red} 8.3524\color{black} $\,\,$ & $\,\,$4.1093$\,\,$ \\
$\,\,$0.7593$\,\,$ & $\,\,$ 1 $\,\,$ & $\,\,$\color{red} 6.3416\color{black} $\,\,$ & $\,\,$3.1200  $\,\,$ \\
$\,\,$\color{red} 0.1197\color{black} $\,\,$ & $\,\,$\color{red} 0.1577\color{black} $\,\,$ & $\,\,$ 1 $\,\,$ & $\,\,$\color{red} 0.4920\color{black}  $\,\,$ \\
$\,\,$0.2433$\,\,$ & $\,\,$0.3205$\,\,$ & $\,\,$\color{red} 2.0325\color{black} $\,\,$ & $\,\,$ 1  $\,\,$ \\
\end{pmatrix},
\end{equation*}

\begin{equation*}
\mathbf{w}^{\prime} =
\begin{pmatrix}
0.470749\\
0.357418\\
0.057278\\
0.114556
\end{pmatrix} =
0.999083\cdot
\begin{pmatrix}
0.471181\\
0.357746\\
\color{gr} 0.057330\color{black} \\
0.114661
\end{pmatrix},
\end{equation*}
\begin{equation*}
\left[ \frac{{w}^{\prime}_i}{{w}^{\prime}_j} \right] =
\begin{pmatrix}
$\,\,$ 1 $\,\,$ & $\,\,$1.3171$\,\,$ & $\,\,$\color{gr} 8.2187\color{black} $\,\,$ & $\,\,$4.1093$\,\,$ \\
$\,\,$0.7593$\,\,$ & $\,\,$ 1 $\,\,$ & $\,\,$\color{gr} 6.2401\color{black} $\,\,$ & $\,\,$3.1200  $\,\,$ \\
$\,\,$\color{gr} 0.1217\color{black} $\,\,$ & $\,\,$\color{gr} 0.1603\color{black} $\,\,$ & $\,\,$ 1 $\,\,$ & $\,\,$\color{gr} \color{blue}  1/2\color{black}  $\,\,$ \\
$\,\,$0.2433$\,\,$ & $\,\,$0.3205$\,\,$ & $\,\,$\color{gr} \color{blue} 2\color{black} $\,\,$ & $\,\,$ 1  $\,\,$ \\
\end{pmatrix},
\end{equation*}
\end{example}
\newpage
\begin{example}
\begin{equation*}
\mathbf{A} =
\begin{pmatrix}
$\,\,$ 1 $\,\,$ & $\,\,$2$\,\,$ & $\,\,$8$\,\,$ & $\,\,$3 $\,\,$ \\
$\,\,$ 1/2$\,\,$ & $\,\,$ 1 $\,\,$ & $\,\,$7$\,\,$ & $\,\,$6 $\,\,$ \\
$\,\,$ 1/8$\,\,$ & $\,\,$ 1/7$\,\,$ & $\,\,$ 1 $\,\,$ & $\,\,$ 1/2 $\,\,$ \\
$\,\,$ 1/3$\,\,$ & $\,\,$ 1/6$\,\,$ & $\,\,$2$\,\,$ & $\,\,$ 1  $\,\,$ \\
\end{pmatrix},
\qquad
\lambda_{\max} =
4.1681,
\qquad
CR = 0.0634
\end{equation*}

\begin{equation*}
\mathbf{w}^{cos} =
\begin{pmatrix}
0.460351\\
0.378034\\
\color{red} 0.052924\color{black} \\
0.108691
\end{pmatrix}\end{equation*}
\begin{equation*}
\left[ \frac{{w}^{cos}_i}{{w}^{cos}_j} \right] =
\begin{pmatrix}
$\,\,$ 1 $\,\,$ & $\,\,$1.2177$\,\,$ & $\,\,$\color{red} 8.6984\color{black} $\,\,$ & $\,\,$4.2354$\,\,$ \\
$\,\,$0.8212$\,\,$ & $\,\,$ 1 $\,\,$ & $\,\,$\color{red} 7.1430\color{black} $\,\,$ & $\,\,$3.4781  $\,\,$ \\
$\,\,$\color{red} 0.1150\color{black} $\,\,$ & $\,\,$\color{red} 0.1400\color{black} $\,\,$ & $\,\,$ 1 $\,\,$ & $\,\,$\color{red} 0.4869\color{black}  $\,\,$ \\
$\,\,$0.2361$\,\,$ & $\,\,$0.2875$\,\,$ & $\,\,$\color{red} 2.0537\color{black} $\,\,$ & $\,\,$ 1  $\,\,$ \\
\end{pmatrix},
\end{equation*}

\begin{equation*}
\mathbf{w}^{\prime} =
\begin{pmatrix}
0.459854\\
0.377626\\
0.053947\\
0.108574
\end{pmatrix} =
0.998920\cdot
\begin{pmatrix}
0.460351\\
0.378034\\
\color{gr} 0.054005\color{black} \\
0.108691
\end{pmatrix},
\end{equation*}
\begin{equation*}
\left[ \frac{{w}^{\prime}_i}{{w}^{\prime}_j} \right] =
\begin{pmatrix}
$\,\,$ 1 $\,\,$ & $\,\,$1.2177$\,\,$ & $\,\,$\color{gr} 8.5242\color{black} $\,\,$ & $\,\,$4.2354$\,\,$ \\
$\,\,$0.8212$\,\,$ & $\,\,$ 1 $\,\,$ & $\,\,$\color{gr} \color{blue} 7\color{black} $\,\,$ & $\,\,$3.4781  $\,\,$ \\
$\,\,$\color{gr} 0.1173\color{black} $\,\,$ & $\,\,$\color{gr} \color{blue}  1/7\color{black} $\,\,$ & $\,\,$ 1 $\,\,$ & $\,\,$\color{gr} 0.4969\color{black}  $\,\,$ \\
$\,\,$0.2361$\,\,$ & $\,\,$0.2875$\,\,$ & $\,\,$\color{gr} 2.0126\color{black} $\,\,$ & $\,\,$ 1  $\,\,$ \\
\end{pmatrix},
\end{equation*}
\end{example}
\newpage
\begin{example}
\begin{equation*}
\mathbf{A} =
\begin{pmatrix}
$\,\,$ 1 $\,\,$ & $\,\,$2$\,\,$ & $\,\,$8$\,\,$ & $\,\,$3 $\,\,$ \\
$\,\,$ 1/2$\,\,$ & $\,\,$ 1 $\,\,$ & $\,\,$7$\,\,$ & $\,\,$7 $\,\,$ \\
$\,\,$ 1/8$\,\,$ & $\,\,$ 1/7$\,\,$ & $\,\,$ 1 $\,\,$ & $\,\,$ 1/2 $\,\,$ \\
$\,\,$ 1/3$\,\,$ & $\,\,$ 1/7$\,\,$ & $\,\,$2$\,\,$ & $\,\,$ 1  $\,\,$ \\
\end{pmatrix},
\qquad
\lambda_{\max} =
4.2109,
\qquad
CR = 0.0795
\end{equation*}

\begin{equation*}
\mathbf{w}^{cos} =
\begin{pmatrix}
0.456634\\
0.385983\\
\color{red} 0.052069\color{black} \\
0.105315
\end{pmatrix}\end{equation*}
\begin{equation*}
\left[ \frac{{w}^{cos}_i}{{w}^{cos}_j} \right] =
\begin{pmatrix}
$\,\,$ 1 $\,\,$ & $\,\,$1.1830$\,\,$ & $\,\,$\color{red} 8.7698\color{black} $\,\,$ & $\,\,$4.3359$\,\,$ \\
$\,\,$0.8453$\,\,$ & $\,\,$ 1 $\,\,$ & $\,\,$\color{red} 7.4129\color{black} $\,\,$ & $\,\,$3.6651  $\,\,$ \\
$\,\,$\color{red} 0.1140\color{black} $\,\,$ & $\,\,$\color{red} 0.1349\color{black} $\,\,$ & $\,\,$ 1 $\,\,$ & $\,\,$\color{red} 0.4944\color{black}  $\,\,$ \\
$\,\,$0.2306$\,\,$ & $\,\,$0.2728$\,\,$ & $\,\,$\color{red} 2.0226\color{black} $\,\,$ & $\,\,$ 1  $\,\,$ \\
\end{pmatrix},
\end{equation*}

\begin{equation*}
\mathbf{w}^{\prime} =
\begin{pmatrix}
0.456365\\
0.385756\\
0.052626\\
0.105253
\end{pmatrix} =
0.999412\cdot
\begin{pmatrix}
0.456634\\
0.385983\\
\color{gr} 0.052657\color{black} \\
0.105315
\end{pmatrix},
\end{equation*}
\begin{equation*}
\left[ \frac{{w}^{\prime}_i}{{w}^{\prime}_j} \right] =
\begin{pmatrix}
$\,\,$ 1 $\,\,$ & $\,\,$1.1830$\,\,$ & $\,\,$\color{gr} 8.6718\color{black} $\,\,$ & $\,\,$4.3359$\,\,$ \\
$\,\,$0.8453$\,\,$ & $\,\,$ 1 $\,\,$ & $\,\,$\color{gr} 7.3301\color{black} $\,\,$ & $\,\,$3.6651  $\,\,$ \\
$\,\,$\color{gr} 0.1153\color{black} $\,\,$ & $\,\,$\color{gr} 0.1364\color{black} $\,\,$ & $\,\,$ 1 $\,\,$ & $\,\,$\color{gr} \color{blue}  1/2\color{black}  $\,\,$ \\
$\,\,$0.2306$\,\,$ & $\,\,$0.2728$\,\,$ & $\,\,$\color{gr} \color{blue} 2\color{black} $\,\,$ & $\,\,$ 1  $\,\,$ \\
\end{pmatrix},
\end{equation*}
\end{example}
\newpage
\begin{example}
\begin{equation*}
\mathbf{A} =
\begin{pmatrix}
$\,\,$ 1 $\,\,$ & $\,\,$2$\,\,$ & $\,\,$8$\,\,$ & $\,\,$3 $\,\,$ \\
$\,\,$ 1/2$\,\,$ & $\,\,$ 1 $\,\,$ & $\,\,$8$\,\,$ & $\,\,$8 $\,\,$ \\
$\,\,$ 1/8$\,\,$ & $\,\,$ 1/8$\,\,$ & $\,\,$ 1 $\,\,$ & $\,\,$ 1/2 $\,\,$ \\
$\,\,$ 1/3$\,\,$ & $\,\,$ 1/8$\,\,$ & $\,\,$2$\,\,$ & $\,\,$ 1  $\,\,$ \\
\end{pmatrix},
\qquad
\lambda_{\max} =
4.2512,
\qquad
CR = 0.0947
\end{equation*}

\begin{equation*}
\mathbf{w}^{cos} =
\begin{pmatrix}
0.448151\\
0.401226\\
\color{red} 0.049388\color{black} \\
0.101236
\end{pmatrix}\end{equation*}
\begin{equation*}
\left[ \frac{{w}^{cos}_i}{{w}^{cos}_j} \right] =
\begin{pmatrix}
$\,\,$ 1 $\,\,$ & $\,\,$1.1170$\,\,$ & $\,\,$\color{red} 9.0742\color{black} $\,\,$ & $\,\,$4.4268$\,\,$ \\
$\,\,$0.8953$\,\,$ & $\,\,$ 1 $\,\,$ & $\,\,$\color{red} 8.1240\color{black} $\,\,$ & $\,\,$3.9633  $\,\,$ \\
$\,\,$\color{red} 0.1102\color{black} $\,\,$ & $\,\,$\color{red} 0.1231\color{black} $\,\,$ & $\,\,$ 1 $\,\,$ & $\,\,$\color{red} 0.4878\color{black}  $\,\,$ \\
$\,\,$0.2259$\,\,$ & $\,\,$0.2523$\,\,$ & $\,\,$\color{red} 2.0498\color{black} $\,\,$ & $\,\,$ 1  $\,\,$ \\
\end{pmatrix},
\end{equation*}

\begin{equation*}
\mathbf{w}^{\prime} =
\begin{pmatrix}
0.447809\\
0.400919\\
0.050115\\
0.101158
\end{pmatrix} =
0.999235\cdot
\begin{pmatrix}
0.448151\\
0.401226\\
\color{gr} 0.050153\color{black} \\
0.101236
\end{pmatrix},
\end{equation*}
\begin{equation*}
\left[ \frac{{w}^{\prime}_i}{{w}^{\prime}_j} \right] =
\begin{pmatrix}
$\,\,$ 1 $\,\,$ & $\,\,$1.1170$\,\,$ & $\,\,$\color{gr} 8.9357\color{black} $\,\,$ & $\,\,$4.4268$\,\,$ \\
$\,\,$0.8953$\,\,$ & $\,\,$ 1 $\,\,$ & $\,\,$\color{gr} \color{blue} 8\color{black} $\,\,$ & $\,\,$3.9633  $\,\,$ \\
$\,\,$\color{gr} 0.1119\color{black} $\,\,$ & $\,\,$\color{gr} \color{blue}  1/8\color{black} $\,\,$ & $\,\,$ 1 $\,\,$ & $\,\,$\color{gr} 0.4954\color{black}  $\,\,$ \\
$\,\,$0.2259$\,\,$ & $\,\,$0.2523$\,\,$ & $\,\,$\color{gr} 2.0185\color{black} $\,\,$ & $\,\,$ 1  $\,\,$ \\
\end{pmatrix},
\end{equation*}
\end{example}
\newpage
\begin{example}
\begin{equation*}
\mathbf{A} =
\begin{pmatrix}
$\,\,$ 1 $\,\,$ & $\,\,$2$\,\,$ & $\,\,$8$\,\,$ & $\,\,$4 $\,\,$ \\
$\,\,$ 1/2$\,\,$ & $\,\,$ 1 $\,\,$ & $\,\,$3$\,\,$ & $\,\,$3 $\,\,$ \\
$\,\,$ 1/8$\,\,$ & $\,\,$ 1/3$\,\,$ & $\,\,$ 1 $\,\,$ & $\,\,$ 1/3 $\,\,$ \\
$\,\,$ 1/4$\,\,$ & $\,\,$ 1/3$\,\,$ & $\,\,$3$\,\,$ & $\,\,$ 1  $\,\,$ \\
\end{pmatrix},
\qquad
\lambda_{\max} =
4.1031,
\qquad
CR = 0.0389
\end{equation*}

\begin{equation*}
\mathbf{w}^{cos} =
\begin{pmatrix}
\color{red} 0.522958\color{black} \\
0.274682\\
0.065961\\
0.136398
\end{pmatrix}\end{equation*}
\begin{equation*}
\left[ \frac{{w}^{cos}_i}{{w}^{cos}_j} \right] =
\begin{pmatrix}
$\,\,$ 1 $\,\,$ & $\,\,$\color{red} 1.9039\color{black} $\,\,$ & $\,\,$\color{red} 7.9283\color{black} $\,\,$ & $\,\,$\color{red} 3.8341\color{black} $\,\,$ \\
$\,\,$\color{red} 0.5252\color{black} $\,\,$ & $\,\,$ 1 $\,\,$ & $\,\,$4.1643$\,\,$ & $\,\,$2.0138  $\,\,$ \\
$\,\,$\color{red} 0.1261\color{black} $\,\,$ & $\,\,$0.2401$\,\,$ & $\,\,$ 1 $\,\,$ & $\,\,$0.4836 $\,\,$ \\
$\,\,$\color{red} 0.2608\color{black} $\,\,$ & $\,\,$0.4966$\,\,$ & $\,\,$2.0679$\,\,$ & $\,\,$ 1  $\,\,$ \\
\end{pmatrix},
\end{equation*}

\begin{equation*}
\mathbf{w}^{\prime} =
\begin{pmatrix}
0.525205\\
0.273388\\
0.065651\\
0.135756
\end{pmatrix} =
0.995290\cdot
\begin{pmatrix}
\color{gr} 0.527690\color{black} \\
0.274682\\
0.065961\\
0.136398
\end{pmatrix},
\end{equation*}
\begin{equation*}
\left[ \frac{{w}^{\prime}_i}{{w}^{\prime}_j} \right] =
\begin{pmatrix}
$\,\,$ 1 $\,\,$ & $\,\,$\color{gr} 1.9211\color{black} $\,\,$ & $\,\,$\color{gr} \color{blue} 8\color{black} $\,\,$ & $\,\,$\color{gr} 3.8687\color{black} $\,\,$ \\
$\,\,$\color{gr} 0.5205\color{black} $\,\,$ & $\,\,$ 1 $\,\,$ & $\,\,$4.1643$\,\,$ & $\,\,$2.0138  $\,\,$ \\
$\,\,$\color{gr} \color{blue}  1/8\color{black} $\,\,$ & $\,\,$0.2401$\,\,$ & $\,\,$ 1 $\,\,$ & $\,\,$0.4836 $\,\,$ \\
$\,\,$\color{gr} 0.2585\color{black} $\,\,$ & $\,\,$0.4966$\,\,$ & $\,\,$2.0679$\,\,$ & $\,\,$ 1  $\,\,$ \\
\end{pmatrix},
\end{equation*}
\end{example}
\newpage
\begin{example}
\begin{equation*}
\mathbf{A} =
\begin{pmatrix}
$\,\,$ 1 $\,\,$ & $\,\,$2$\,\,$ & $\,\,$8$\,\,$ & $\,\,$6 $\,\,$ \\
$\,\,$ 1/2$\,\,$ & $\,\,$ 1 $\,\,$ & $\,\,$3$\,\,$ & $\,\,$4 $\,\,$ \\
$\,\,$ 1/8$\,\,$ & $\,\,$ 1/3$\,\,$ & $\,\,$ 1 $\,\,$ & $\,\,$ 1/2 $\,\,$ \\
$\,\,$ 1/6$\,\,$ & $\,\,$ 1/4$\,\,$ & $\,\,$2$\,\,$ & $\,\,$ 1  $\,\,$ \\
\end{pmatrix},
\qquad
\lambda_{\max} =
4.0820,
\qquad
CR = 0.0309
\end{equation*}

\begin{equation*}
\mathbf{w}^{cos} =
\begin{pmatrix}
\color{red} 0.552389\color{black} \\
0.279967\\
0.069425\\
0.098219
\end{pmatrix}\end{equation*}
\begin{equation*}
\left[ \frac{{w}^{cos}_i}{{w}^{cos}_j} \right] =
\begin{pmatrix}
$\,\,$ 1 $\,\,$ & $\,\,$\color{red} 1.9731\color{black} $\,\,$ & $\,\,$\color{red} 7.9566\color{black} $\,\,$ & $\,\,$\color{red} 5.6240\color{black} $\,\,$ \\
$\,\,$\color{red} 0.5068\color{black} $\,\,$ & $\,\,$ 1 $\,\,$ & $\,\,$4.0326$\,\,$ & $\,\,$2.8504  $\,\,$ \\
$\,\,$\color{red} 0.1257\color{black} $\,\,$ & $\,\,$0.2480$\,\,$ & $\,\,$ 1 $\,\,$ & $\,\,$0.7068 $\,\,$ \\
$\,\,$\color{red} 0.1778\color{black} $\,\,$ & $\,\,$0.3508$\,\,$ & $\,\,$1.4148$\,\,$ & $\,\,$ 1  $\,\,$ \\
\end{pmatrix},
\end{equation*}

\begin{equation*}
\mathbf{w}^{\prime} =
\begin{pmatrix}
0.553734\\
0.279125\\
0.069217\\
0.097924
\end{pmatrix} =
0.996995\cdot
\begin{pmatrix}
\color{gr} 0.555402\color{black} \\
0.279967\\
0.069425\\
0.098219
\end{pmatrix},
\end{equation*}
\begin{equation*}
\left[ \frac{{w}^{\prime}_i}{{w}^{\prime}_j} \right] =
\begin{pmatrix}
$\,\,$ 1 $\,\,$ & $\,\,$\color{gr} 1.9838\color{black} $\,\,$ & $\,\,$\color{gr} \color{blue} 8\color{black} $\,\,$ & $\,\,$\color{gr} 5.6547\color{black} $\,\,$ \\
$\,\,$\color{gr} 0.5041\color{black} $\,\,$ & $\,\,$ 1 $\,\,$ & $\,\,$4.0326$\,\,$ & $\,\,$2.8504  $\,\,$ \\
$\,\,$\color{gr} \color{blue}  1/8\color{black} $\,\,$ & $\,\,$0.2480$\,\,$ & $\,\,$ 1 $\,\,$ & $\,\,$0.7068 $\,\,$ \\
$\,\,$\color{gr} 0.1768\color{black} $\,\,$ & $\,\,$0.3508$\,\,$ & $\,\,$1.4148$\,\,$ & $\,\,$ 1  $\,\,$ \\
\end{pmatrix},
\end{equation*}
\end{example}
\newpage
\begin{example}
\begin{equation*}
\mathbf{A} =
\begin{pmatrix}
$\,\,$ 1 $\,\,$ & $\,\,$2$\,\,$ & $\,\,$8$\,\,$ & $\,\,$6 $\,\,$ \\
$\,\,$ 1/2$\,\,$ & $\,\,$ 1 $\,\,$ & $\,\,$3$\,\,$ & $\,\,$5 $\,\,$ \\
$\,\,$ 1/8$\,\,$ & $\,\,$ 1/3$\,\,$ & $\,\,$ 1 $\,\,$ & $\,\,$ 1/2 $\,\,$ \\
$\,\,$ 1/6$\,\,$ & $\,\,$ 1/5$\,\,$ & $\,\,$2$\,\,$ & $\,\,$ 1  $\,\,$ \\
\end{pmatrix},
\qquad
\lambda_{\max} =
4.1252,
\qquad
CR = 0.0472
\end{equation*}

\begin{equation*}
\mathbf{w}^{cos} =
\begin{pmatrix}
\color{red} 0.543789\color{black} \\
0.294152\\
0.068772\\
0.093287
\end{pmatrix}\end{equation*}
\begin{equation*}
\left[ \frac{{w}^{cos}_i}{{w}^{cos}_j} \right] =
\begin{pmatrix}
$\,\,$ 1 $\,\,$ & $\,\,$\color{red} 1.8487\color{black} $\,\,$ & $\,\,$\color{red} 7.9071\color{black} $\,\,$ & $\,\,$\color{red} 5.8292\color{black} $\,\,$ \\
$\,\,$\color{red} 0.5409\color{black} $\,\,$ & $\,\,$ 1 $\,\,$ & $\,\,$4.2772$\,\,$ & $\,\,$3.1532  $\,\,$ \\
$\,\,$\color{red} 0.1265\color{black} $\,\,$ & $\,\,$0.2338$\,\,$ & $\,\,$ 1 $\,\,$ & $\,\,$0.7372 $\,\,$ \\
$\,\,$\color{red} 0.1716\color{black} $\,\,$ & $\,\,$0.3171$\,\,$ & $\,\,$1.3565$\,\,$ & $\,\,$ 1  $\,\,$ \\
\end{pmatrix},
\end{equation*}

\begin{equation*}
\mathbf{w}^{\prime} =
\begin{pmatrix}
0.546684\\
0.292285\\
0.068335\\
0.092695
\end{pmatrix} =
0.993654\cdot
\begin{pmatrix}
\color{gr} 0.550175\color{black} \\
0.294152\\
0.068772\\
0.093287
\end{pmatrix},
\end{equation*}
\begin{equation*}
\left[ \frac{{w}^{\prime}_i}{{w}^{\prime}_j} \right] =
\begin{pmatrix}
$\,\,$ 1 $\,\,$ & $\,\,$\color{gr} 1.8704\color{black} $\,\,$ & $\,\,$\color{gr} \color{blue} 8\color{black} $\,\,$ & $\,\,$\color{gr} 5.8976\color{black} $\,\,$ \\
$\,\,$\color{gr} 0.5347\color{black} $\,\,$ & $\,\,$ 1 $\,\,$ & $\,\,$4.2772$\,\,$ & $\,\,$3.1532  $\,\,$ \\
$\,\,$\color{gr} \color{blue}  1/8\color{black} $\,\,$ & $\,\,$0.2338$\,\,$ & $\,\,$ 1 $\,\,$ & $\,\,$0.7372 $\,\,$ \\
$\,\,$\color{gr} 0.1696\color{black} $\,\,$ & $\,\,$0.3171$\,\,$ & $\,\,$1.3565$\,\,$ & $\,\,$ 1  $\,\,$ \\
\end{pmatrix},
\end{equation*}
\end{example}
\newpage
\begin{example}
\begin{equation*}
\mathbf{A} =
\begin{pmatrix}
$\,\,$ 1 $\,\,$ & $\,\,$2$\,\,$ & $\,\,$8$\,\,$ & $\,\,$6 $\,\,$ \\
$\,\,$ 1/2$\,\,$ & $\,\,$ 1 $\,\,$ & $\,\,$3$\,\,$ & $\,\,$6 $\,\,$ \\
$\,\,$ 1/8$\,\,$ & $\,\,$ 1/3$\,\,$ & $\,\,$ 1 $\,\,$ & $\,\,$ 1/2 $\,\,$ \\
$\,\,$ 1/6$\,\,$ & $\,\,$ 1/6$\,\,$ & $\,\,$2$\,\,$ & $\,\,$ 1  $\,\,$ \\
\end{pmatrix},
\qquad
\lambda_{\max} =
4.1707,
\qquad
CR = 0.0644
\end{equation*}

\begin{equation*}
\mathbf{w}^{cos} =
\begin{pmatrix}
\color{red} 0.536211\color{black} \\
0.305880\\
0.068228\\
0.089681
\end{pmatrix}\end{equation*}
\begin{equation*}
\left[ \frac{{w}^{cos}_i}{{w}^{cos}_j} \right] =
\begin{pmatrix}
$\,\,$ 1 $\,\,$ & $\,\,$\color{red} 1.7530\color{black} $\,\,$ & $\,\,$\color{red} 7.8591\color{black} $\,\,$ & $\,\,$\color{red} 5.9791\color{black} $\,\,$ \\
$\,\,$\color{red} 0.5704\color{black} $\,\,$ & $\,\,$ 1 $\,\,$ & $\,\,$4.4832$\,\,$ & $\,\,$3.4108  $\,\,$ \\
$\,\,$\color{red} 0.1272\color{black} $\,\,$ & $\,\,$0.2231$\,\,$ & $\,\,$ 1 $\,\,$ & $\,\,$0.7608 $\,\,$ \\
$\,\,$\color{red} 0.1672\color{black} $\,\,$ & $\,\,$0.2932$\,\,$ & $\,\,$1.3144$\,\,$ & $\,\,$ 1  $\,\,$ \\
\end{pmatrix},
\end{equation*}

\begin{equation*}
\mathbf{w}^{\prime} =
\begin{pmatrix}
0.537079\\
0.305307\\
0.068101\\
0.089513
\end{pmatrix} =
0.998128\cdot
\begin{pmatrix}
\color{gr} 0.538086\color{black} \\
0.305880\\
0.068228\\
0.089681
\end{pmatrix},
\end{equation*}
\begin{equation*}
\left[ \frac{{w}^{\prime}_i}{{w}^{\prime}_j} \right] =
\begin{pmatrix}
$\,\,$ 1 $\,\,$ & $\,\,$\color{gr} 1.7591\color{black} $\,\,$ & $\,\,$\color{gr} 7.8866\color{black} $\,\,$ & $\,\,$\color{gr} \color{blue} 6\color{black} $\,\,$ \\
$\,\,$\color{gr} 0.5685\color{black} $\,\,$ & $\,\,$ 1 $\,\,$ & $\,\,$4.4832$\,\,$ & $\,\,$3.4108  $\,\,$ \\
$\,\,$\color{gr} 0.1268\color{black} $\,\,$ & $\,\,$0.2231$\,\,$ & $\,\,$ 1 $\,\,$ & $\,\,$0.7608 $\,\,$ \\
$\,\,$\color{gr} \color{blue}  1/6\color{black} $\,\,$ & $\,\,$0.2932$\,\,$ & $\,\,$1.3144$\,\,$ & $\,\,$ 1  $\,\,$ \\
\end{pmatrix},
\end{equation*}
\end{example}
\newpage
\begin{example}
\begin{equation*}
\mathbf{A} =
\begin{pmatrix}
$\,\,$ 1 $\,\,$ & $\,\,$2$\,\,$ & $\,\,$8$\,\,$ & $\,\,$7 $\,\,$ \\
$\,\,$ 1/2$\,\,$ & $\,\,$ 1 $\,\,$ & $\,\,$3$\,\,$ & $\,\,$8 $\,\,$ \\
$\,\,$ 1/8$\,\,$ & $\,\,$ 1/3$\,\,$ & $\,\,$ 1 $\,\,$ & $\,\,$ 1/2 $\,\,$ \\
$\,\,$ 1/7$\,\,$ & $\,\,$ 1/8$\,\,$ & $\,\,$2$\,\,$ & $\,\,$ 1  $\,\,$ \\
\end{pmatrix},
\qquad
\lambda_{\max} =
4.2536,
\qquad
CR = 0.0956
\end{equation*}

\begin{equation*}
\mathbf{w}^{cos} =
\begin{pmatrix}
\color{red} 0.535116\color{black} \\
0.317078\\
0.067166\\
0.080640
\end{pmatrix}\end{equation*}
\begin{equation*}
\left[ \frac{{w}^{cos}_i}{{w}^{cos}_j} \right] =
\begin{pmatrix}
$\,\,$ 1 $\,\,$ & $\,\,$\color{red} 1.6876\color{black} $\,\,$ & $\,\,$\color{red} 7.9670\color{black} $\,\,$ & $\,\,$\color{red} 6.6358\color{black} $\,\,$ \\
$\,\,$\color{red} 0.5925\color{black} $\,\,$ & $\,\,$ 1 $\,\,$ & $\,\,$4.7208$\,\,$ & $\,\,$3.9320  $\,\,$ \\
$\,\,$\color{red} 0.1255\color{black} $\,\,$ & $\,\,$0.2118$\,\,$ & $\,\,$ 1 $\,\,$ & $\,\,$0.8329 $\,\,$ \\
$\,\,$\color{red} 0.1507\color{black} $\,\,$ & $\,\,$0.2543$\,\,$ & $\,\,$1.2006$\,\,$ & $\,\,$ 1  $\,\,$ \\
\end{pmatrix},
\end{equation*}

\begin{equation*}
\mathbf{w}^{\prime} =
\begin{pmatrix}
0.536143\\
0.316377\\
0.067018\\
0.080462
\end{pmatrix} =
0.997790\cdot
\begin{pmatrix}
\color{gr} 0.537331\color{black} \\
0.317078\\
0.067166\\
0.080640
\end{pmatrix},
\end{equation*}
\begin{equation*}
\left[ \frac{{w}^{\prime}_i}{{w}^{\prime}_j} \right] =
\begin{pmatrix}
$\,\,$ 1 $\,\,$ & $\,\,$\color{gr} 1.6946\color{black} $\,\,$ & $\,\,$\color{gr} \color{blue} 8\color{black} $\,\,$ & $\,\,$\color{gr} 6.6633\color{black} $\,\,$ \\
$\,\,$\color{gr} 0.5901\color{black} $\,\,$ & $\,\,$ 1 $\,\,$ & $\,\,$4.7208$\,\,$ & $\,\,$3.9320  $\,\,$ \\
$\,\,$\color{gr} \color{blue}  1/8\color{black} $\,\,$ & $\,\,$0.2118$\,\,$ & $\,\,$ 1 $\,\,$ & $\,\,$0.8329 $\,\,$ \\
$\,\,$\color{gr} 0.1501\color{black} $\,\,$ & $\,\,$0.2543$\,\,$ & $\,\,$1.2006$\,\,$ & $\,\,$ 1  $\,\,$ \\
\end{pmatrix},
\end{equation*}
\end{example}
\newpage
\begin{example}
\begin{equation*}
\mathbf{A} =
\begin{pmatrix}
$\,\,$ 1 $\,\,$ & $\,\,$2$\,\,$ & $\,\,$9$\,\,$ & $\,\,$2 $\,\,$ \\
$\,\,$ 1/2$\,\,$ & $\,\,$ 1 $\,\,$ & $\,\,$3$\,\,$ & $\,\,$2 $\,\,$ \\
$\,\,$ 1/9$\,\,$ & $\,\,$ 1/3$\,\,$ & $\,\,$ 1 $\,\,$ & $\,\,$ 1/8 $\,\,$ \\
$\,\,$ 1/2$\,\,$ & $\,\,$ 1/2$\,\,$ & $\,\,$8$\,\,$ & $\,\,$ 1  $\,\,$ \\
\end{pmatrix},
\qquad
\lambda_{\max} =
4.2469,
\qquad
CR = 0.0931
\end{equation*}

\begin{equation*}
\mathbf{w}^{cos} =
\begin{pmatrix}
\color{red} 0.453146\color{black} \\
0.257990\\
0.052657\\
0.236207
\end{pmatrix}\end{equation*}
\begin{equation*}
\left[ \frac{{w}^{cos}_i}{{w}^{cos}_j} \right] =
\begin{pmatrix}
$\,\,$ 1 $\,\,$ & $\,\,$\color{red} 1.7564\color{black} $\,\,$ & $\,\,$\color{red} 8.6057\color{black} $\,\,$ & $\,\,$\color{red} 1.9184\color{black} $\,\,$ \\
$\,\,$\color{red} 0.5693\color{black} $\,\,$ & $\,\,$ 1 $\,\,$ & $\,\,$4.8995$\,\,$ & $\,\,$1.0922  $\,\,$ \\
$\,\,$\color{red} 0.1162\color{black} $\,\,$ & $\,\,$0.2041$\,\,$ & $\,\,$ 1 $\,\,$ & $\,\,$0.2229 $\,\,$ \\
$\,\,$\color{red} 0.5213\color{black} $\,\,$ & $\,\,$0.9156$\,\,$ & $\,\,$4.4858$\,\,$ & $\,\,$ 1  $\,\,$ \\
\end{pmatrix},
\end{equation*}

\begin{equation*}
\mathbf{w}^{\prime} =
\begin{pmatrix}
0.463484\\
0.253113\\
0.051661\\
0.231742
\end{pmatrix} =
0.981096\cdot
\begin{pmatrix}
\color{gr} 0.472414\color{black} \\
0.257990\\
0.052657\\
0.236207
\end{pmatrix},
\end{equation*}
\begin{equation*}
\left[ \frac{{w}^{\prime}_i}{{w}^{\prime}_j} \right] =
\begin{pmatrix}
$\,\,$ 1 $\,\,$ & $\,\,$\color{gr} 1.8311\color{black} $\,\,$ & $\,\,$\color{gr} 8.9716\color{black} $\,\,$ & $\,\,$\color{gr} \color{blue} 2\color{black} $\,\,$ \\
$\,\,$\color{gr} 0.5461\color{black} $\,\,$ & $\,\,$ 1 $\,\,$ & $\,\,$4.8995$\,\,$ & $\,\,$1.0922  $\,\,$ \\
$\,\,$\color{gr} 0.1115\color{black} $\,\,$ & $\,\,$0.2041$\,\,$ & $\,\,$ 1 $\,\,$ & $\,\,$0.2229 $\,\,$ \\
$\,\,$\color{gr} \color{blue}  1/2\color{black} $\,\,$ & $\,\,$0.9156$\,\,$ & $\,\,$4.4858$\,\,$ & $\,\,$ 1  $\,\,$ \\
\end{pmatrix},
\end{equation*}
\end{example}
\newpage
\begin{example}
\begin{equation*}
\mathbf{A} =
\begin{pmatrix}
$\,\,$ 1 $\,\,$ & $\,\,$2$\,\,$ & $\,\,$9$\,\,$ & $\,\,$2 $\,\,$ \\
$\,\,$ 1/2$\,\,$ & $\,\,$ 1 $\,\,$ & $\,\,$6$\,\,$ & $\,\,$4 $\,\,$ \\
$\,\,$ 1/9$\,\,$ & $\,\,$ 1/6$\,\,$ & $\,\,$ 1 $\,\,$ & $\,\,$ 1/3 $\,\,$ \\
$\,\,$ 1/2$\,\,$ & $\,\,$ 1/4$\,\,$ & $\,\,$3$\,\,$ & $\,\,$ 1  $\,\,$ \\
\end{pmatrix},
\qquad
\lambda_{\max} =
4.1707,
\qquad
CR = 0.0644
\end{equation*}

\begin{equation*}
\mathbf{w}^{cos} =
\begin{pmatrix}
0.450406\\
0.345889\\
\color{red} 0.049977\color{black} \\
0.153727
\end{pmatrix}\end{equation*}
\begin{equation*}
\left[ \frac{{w}^{cos}_i}{{w}^{cos}_j} \right] =
\begin{pmatrix}
$\,\,$ 1 $\,\,$ & $\,\,$1.3022$\,\,$ & $\,\,$\color{red} 9.0122\color{black} $\,\,$ & $\,\,$2.9299$\,\,$ \\
$\,\,$0.7679$\,\,$ & $\,\,$ 1 $\,\,$ & $\,\,$\color{red} 6.9209\color{black} $\,\,$ & $\,\,$2.2500  $\,\,$ \\
$\,\,$\color{red} 0.1110\color{black} $\,\,$ & $\,\,$\color{red} 0.1445\color{black} $\,\,$ & $\,\,$ 1 $\,\,$ & $\,\,$\color{red} 0.3251\color{black}  $\,\,$ \\
$\,\,$0.3413$\,\,$ & $\,\,$0.4444$\,\,$ & $\,\,$\color{red} 3.0759\color{black} $\,\,$ & $\,\,$ 1  $\,\,$ \\
\end{pmatrix},
\end{equation*}

\begin{equation*}
\mathbf{w}^{\prime} =
\begin{pmatrix}
0.450376\\
0.345865\\
0.050042\\
0.153717
\end{pmatrix} =
0.999932\cdot
\begin{pmatrix}
0.450406\\
0.345889\\
\color{gr} 0.050045\color{black} \\
0.153727
\end{pmatrix},
\end{equation*}
\begin{equation*}
\left[ \frac{{w}^{\prime}_i}{{w}^{\prime}_j} \right] =
\begin{pmatrix}
$\,\,$ 1 $\,\,$ & $\,\,$1.3022$\,\,$ & $\,\,$\color{gr} \color{blue} 9\color{black} $\,\,$ & $\,\,$2.9299$\,\,$ \\
$\,\,$0.7679$\,\,$ & $\,\,$ 1 $\,\,$ & $\,\,$\color{gr} 6.9115\color{black} $\,\,$ & $\,\,$2.2500  $\,\,$ \\
$\,\,$\color{gr} \color{blue}  1/9\color{black} $\,\,$ & $\,\,$\color{gr} 0.1447\color{black} $\,\,$ & $\,\,$ 1 $\,\,$ & $\,\,$\color{gr} 0.3255\color{black}  $\,\,$ \\
$\,\,$0.3413$\,\,$ & $\,\,$0.4444$\,\,$ & $\,\,$\color{gr} 3.0718\color{black} $\,\,$ & $\,\,$ 1  $\,\,$ \\
\end{pmatrix},
\end{equation*}
\end{example}
\newpage
\begin{example}
\begin{equation*}
\mathbf{A} =
\begin{pmatrix}
$\,\,$ 1 $\,\,$ & $\,\,$2$\,\,$ & $\,\,$9$\,\,$ & $\,\,$2 $\,\,$ \\
$\,\,$ 1/2$\,\,$ & $\,\,$ 1 $\,\,$ & $\,\,$6$\,\,$ & $\,\,$5 $\,\,$ \\
$\,\,$ 1/9$\,\,$ & $\,\,$ 1/6$\,\,$ & $\,\,$ 1 $\,\,$ & $\,\,$ 1/3 $\,\,$ \\
$\,\,$ 1/2$\,\,$ & $\,\,$ 1/5$\,\,$ & $\,\,$3$\,\,$ & $\,\,$ 1  $\,\,$ \\
\end{pmatrix},
\qquad
\lambda_{\max} =
4.2394,
\qquad
CR = 0.0903
\end{equation*}

\begin{equation*}
\mathbf{w}^{cos} =
\begin{pmatrix}
0.446434\\
0.357560\\
\color{red} 0.048913\color{black} \\
0.147093
\end{pmatrix}\end{equation*}
\begin{equation*}
\left[ \frac{{w}^{cos}_i}{{w}^{cos}_j} \right] =
\begin{pmatrix}
$\,\,$ 1 $\,\,$ & $\,\,$1.2486$\,\,$ & $\,\,$\color{red} 9.1270\color{black} $\,\,$ & $\,\,$3.0351$\,\,$ \\
$\,\,$0.8009$\,\,$ & $\,\,$ 1 $\,\,$ & $\,\,$\color{red} 7.3101\color{black} $\,\,$ & $\,\,$2.4308  $\,\,$ \\
$\,\,$\color{red} 0.1096\color{black} $\,\,$ & $\,\,$\color{red} 0.1368\color{black} $\,\,$ & $\,\,$ 1 $\,\,$ & $\,\,$\color{red} 0.3325\color{black}  $\,\,$ \\
$\,\,$0.3295$\,\,$ & $\,\,$0.4114$\,\,$ & $\,\,$\color{red} 3.0072\color{black} $\,\,$ & $\,\,$ 1  $\,\,$ \\
\end{pmatrix},
\end{equation*}

\begin{equation*}
\mathbf{w}^{\prime} =
\begin{pmatrix}
0.446381\\
0.357518\\
0.049025\\
0.147075
\end{pmatrix} =
0.999882\cdot
\begin{pmatrix}
0.446434\\
0.357560\\
\color{gr} 0.049031\color{black} \\
0.147093
\end{pmatrix},
\end{equation*}
\begin{equation*}
\left[ \frac{{w}^{\prime}_i}{{w}^{\prime}_j} \right] =
\begin{pmatrix}
$\,\,$ 1 $\,\,$ & $\,\,$1.2486$\,\,$ & $\,\,$\color{gr} 9.1052\color{black} $\,\,$ & $\,\,$3.0351$\,\,$ \\
$\,\,$0.8009$\,\,$ & $\,\,$ 1 $\,\,$ & $\,\,$\color{gr} 7.2925\color{black} $\,\,$ & $\,\,$2.4308  $\,\,$ \\
$\,\,$\color{gr} 0.1098\color{black} $\,\,$ & $\,\,$\color{gr} 0.1371\color{black} $\,\,$ & $\,\,$ 1 $\,\,$ & $\,\,$\color{gr} \color{blue}  1/3\color{black}  $\,\,$ \\
$\,\,$0.3295$\,\,$ & $\,\,$0.4114$\,\,$ & $\,\,$\color{gr} \color{blue} 3\color{black} $\,\,$ & $\,\,$ 1  $\,\,$ \\
\end{pmatrix},
\end{equation*}
\end{example}
\newpage
\begin{example}
\begin{equation*}
\mathbf{A} =
\begin{pmatrix}
$\,\,$ 1 $\,\,$ & $\,\,$2$\,\,$ & $\,\,$9$\,\,$ & $\,\,$2 $\,\,$ \\
$\,\,$ 1/2$\,\,$ & $\,\,$ 1 $\,\,$ & $\,\,$7$\,\,$ & $\,\,$4 $\,\,$ \\
$\,\,$ 1/9$\,\,$ & $\,\,$ 1/7$\,\,$ & $\,\,$ 1 $\,\,$ & $\,\,$ 1/3 $\,\,$ \\
$\,\,$ 1/2$\,\,$ & $\,\,$ 1/4$\,\,$ & $\,\,$3$\,\,$ & $\,\,$ 1  $\,\,$ \\
\end{pmatrix},
\qquad
\lambda_{\max} =
4.1658,
\qquad
CR = 0.0625
\end{equation*}

\begin{equation*}
\mathbf{w}^{cos} =
\begin{pmatrix}
0.445024\\
0.355307\\
\color{red} 0.047754\color{black} \\
0.151916
\end{pmatrix}\end{equation*}
\begin{equation*}
\left[ \frac{{w}^{cos}_i}{{w}^{cos}_j} \right] =
\begin{pmatrix}
$\,\,$ 1 $\,\,$ & $\,\,$1.2525$\,\,$ & $\,\,$\color{red} 9.3191\color{black} $\,\,$ & $\,\,$2.9294$\,\,$ \\
$\,\,$0.7984$\,\,$ & $\,\,$ 1 $\,\,$ & $\,\,$\color{red} 7.4404\color{black} $\,\,$ & $\,\,$2.3388  $\,\,$ \\
$\,\,$\color{red} 0.1073\color{black} $\,\,$ & $\,\,$\color{red} 0.1344\color{black} $\,\,$ & $\,\,$ 1 $\,\,$ & $\,\,$\color{red} 0.3143\color{black}  $\,\,$ \\
$\,\,$0.3414$\,\,$ & $\,\,$0.4276$\,\,$ & $\,\,$\color{red} 3.1812\color{black} $\,\,$ & $\,\,$ 1  $\,\,$ \\
\end{pmatrix},
\end{equation*}

\begin{equation*}
\mathbf{w}^{\prime} =
\begin{pmatrix}
0.444272\\
0.354706\\
0.049364\\
0.151659
\end{pmatrix} =
0.998310\cdot
\begin{pmatrix}
0.445024\\
0.355307\\
\color{gr} 0.049447\color{black} \\
0.151916
\end{pmatrix},
\end{equation*}
\begin{equation*}
\left[ \frac{{w}^{\prime}_i}{{w}^{\prime}_j} \right] =
\begin{pmatrix}
$\,\,$ 1 $\,\,$ & $\,\,$1.2525$\,\,$ & $\,\,$\color{gr} \color{blue} 9\color{black} $\,\,$ & $\,\,$2.9294$\,\,$ \\
$\,\,$0.7984$\,\,$ & $\,\,$ 1 $\,\,$ & $\,\,$\color{gr} 7.1856\color{black} $\,\,$ & $\,\,$2.3388  $\,\,$ \\
$\,\,$\color{gr} \color{blue}  1/9\color{black} $\,\,$ & $\,\,$\color{gr} 0.1392\color{black} $\,\,$ & $\,\,$ 1 $\,\,$ & $\,\,$\color{gr} 0.3255\color{black}  $\,\,$ \\
$\,\,$0.3414$\,\,$ & $\,\,$0.4276$\,\,$ & $\,\,$\color{gr} 3.0723\color{black} $\,\,$ & $\,\,$ 1  $\,\,$ \\
\end{pmatrix},
\end{equation*}
\end{example}
\newpage
\begin{example}
\begin{equation*}
\mathbf{A} =
\begin{pmatrix}
$\,\,$ 1 $\,\,$ & $\,\,$2$\,\,$ & $\,\,$9$\,\,$ & $\,\,$2 $\,\,$ \\
$\,\,$ 1/2$\,\,$ & $\,\,$ 1 $\,\,$ & $\,\,$7$\,\,$ & $\,\,$5 $\,\,$ \\
$\,\,$ 1/9$\,\,$ & $\,\,$ 1/7$\,\,$ & $\,\,$ 1 $\,\,$ & $\,\,$ 1/3 $\,\,$ \\
$\,\,$ 1/2$\,\,$ & $\,\,$ 1/5$\,\,$ & $\,\,$3$\,\,$ & $\,\,$ 1  $\,\,$ \\
\end{pmatrix},
\qquad
\lambda_{\max} =
4.2300,
\qquad
CR = 0.0867
\end{equation*}

\begin{equation*}
\mathbf{w}^{cos} =
\begin{pmatrix}
0.440970\\
0.367131\\
\color{red} 0.046652\color{black} \\
0.145247
\end{pmatrix}\end{equation*}
\begin{equation*}
\left[ \frac{{w}^{cos}_i}{{w}^{cos}_j} \right] =
\begin{pmatrix}
$\,\,$ 1 $\,\,$ & $\,\,$1.2011$\,\,$ & $\,\,$\color{red} 9.4523\color{black} $\,\,$ & $\,\,$3.0360$\,\,$ \\
$\,\,$0.8326$\,\,$ & $\,\,$ 1 $\,\,$ & $\,\,$\color{red} 7.8695\color{black} $\,\,$ & $\,\,$2.5276  $\,\,$ \\
$\,\,$\color{red} 0.1058\color{black} $\,\,$ & $\,\,$\color{red} 0.1271\color{black} $\,\,$ & $\,\,$ 1 $\,\,$ & $\,\,$\color{red} 0.3212\color{black}  $\,\,$ \\
$\,\,$0.3294$\,\,$ & $\,\,$0.3956$\,\,$ & $\,\,$\color{red} 3.1134\color{black} $\,\,$ & $\,\,$ 1  $\,\,$ \\
\end{pmatrix},
\end{equation*}

\begin{equation*}
\mathbf{w}^{\prime} =
\begin{pmatrix}
0.440194\\
0.366484\\
0.048330\\
0.144991
\end{pmatrix} =
0.998240\cdot
\begin{pmatrix}
0.440970\\
0.367131\\
\color{gr} 0.048416\color{black} \\
0.145247
\end{pmatrix},
\end{equation*}
\begin{equation*}
\left[ \frac{{w}^{\prime}_i}{{w}^{\prime}_j} \right] =
\begin{pmatrix}
$\,\,$ 1 $\,\,$ & $\,\,$1.2011$\,\,$ & $\,\,$\color{gr} 9.1080\color{black} $\,\,$ & $\,\,$3.0360$\,\,$ \\
$\,\,$0.8326$\,\,$ & $\,\,$ 1 $\,\,$ & $\,\,$\color{gr} 7.5829\color{black} $\,\,$ & $\,\,$2.5276  $\,\,$ \\
$\,\,$\color{gr} 0.1098\color{black} $\,\,$ & $\,\,$\color{gr} 0.1319\color{black} $\,\,$ & $\,\,$ 1 $\,\,$ & $\,\,$\color{gr} \color{blue}  1/3\color{black}  $\,\,$ \\
$\,\,$0.3294$\,\,$ & $\,\,$0.3956$\,\,$ & $\,\,$\color{gr} \color{blue} 3\color{black} $\,\,$ & $\,\,$ 1  $\,\,$ \\
\end{pmatrix},
\end{equation*}
\end{example}
\newpage
\begin{example}
\begin{equation*}
\mathbf{A} =
\begin{pmatrix}
$\,\,$ 1 $\,\,$ & $\,\,$2$\,\,$ & $\,\,$9$\,\,$ & $\,\,$2 $\,\,$ \\
$\,\,$ 1/2$\,\,$ & $\,\,$ 1 $\,\,$ & $\,\,$8$\,\,$ & $\,\,$5 $\,\,$ \\
$\,\,$ 1/9$\,\,$ & $\,\,$ 1/8$\,\,$ & $\,\,$ 1 $\,\,$ & $\,\,$ 1/3 $\,\,$ \\
$\,\,$ 1/2$\,\,$ & $\,\,$ 1/5$\,\,$ & $\,\,$3$\,\,$ & $\,\,$ 1  $\,\,$ \\
\end{pmatrix},
\qquad
\lambda_{\max} =
4.2267,
\qquad
CR = 0.0855
\end{equation*}

\begin{equation*}
\mathbf{w}^{cos} =
\begin{pmatrix}
0.435883\\
0.375745\\
\color{red} 0.044840\color{black} \\
0.143533
\end{pmatrix}\end{equation*}
\begin{equation*}
\left[ \frac{{w}^{cos}_i}{{w}^{cos}_j} \right] =
\begin{pmatrix}
$\,\,$ 1 $\,\,$ & $\,\,$1.1601$\,\,$ & $\,\,$\color{red} 9.7209\color{black} $\,\,$ & $\,\,$3.0368$\,\,$ \\
$\,\,$0.8620$\,\,$ & $\,\,$ 1 $\,\,$ & $\,\,$\color{red} 8.3797\color{black} $\,\,$ & $\,\,$2.6178  $\,\,$ \\
$\,\,$\color{red} 0.1029\color{black} $\,\,$ & $\,\,$\color{red} 0.1193\color{black} $\,\,$ & $\,\,$ 1 $\,\,$ & $\,\,$\color{red} 0.3124\color{black}  $\,\,$ \\
$\,\,$0.3293$\,\,$ & $\,\,$0.3820$\,\,$ & $\,\,$\color{red} 3.2010\color{black} $\,\,$ & $\,\,$ 1  $\,\,$ \\
\end{pmatrix},
\end{equation*}

\begin{equation*}
\mathbf{w}^{\prime} =
\begin{pmatrix}
0.434957\\
0.374947\\
0.046868\\
0.143228
\end{pmatrix} =
0.997876\cdot
\begin{pmatrix}
0.435883\\
0.375745\\
\color{gr} 0.046968\color{black} \\
0.143533
\end{pmatrix},
\end{equation*}
\begin{equation*}
\left[ \frac{{w}^{\prime}_i}{{w}^{\prime}_j} \right] =
\begin{pmatrix}
$\,\,$ 1 $\,\,$ & $\,\,$1.1601$\,\,$ & $\,\,$\color{gr} 9.2804\color{black} $\,\,$ & $\,\,$3.0368$\,\,$ \\
$\,\,$0.8620$\,\,$ & $\,\,$ 1 $\,\,$ & $\,\,$\color{gr} \color{blue} 8\color{black} $\,\,$ & $\,\,$2.6178  $\,\,$ \\
$\,\,$\color{gr} 0.1078\color{black} $\,\,$ & $\,\,$\color{gr} \color{blue}  1/8\color{black} $\,\,$ & $\,\,$ 1 $\,\,$ & $\,\,$\color{gr} 0.3272\color{black}  $\,\,$ \\
$\,\,$0.3293$\,\,$ & $\,\,$0.3820$\,\,$ & $\,\,$\color{gr} 3.0560\color{black} $\,\,$ & $\,\,$ 1  $\,\,$ \\
\end{pmatrix},
\end{equation*}
\end{example}
\newpage
\begin{example}
\begin{equation*}
\mathbf{A} =
\begin{pmatrix}
$\,\,$ 1 $\,\,$ & $\,\,$2$\,\,$ & $\,\,$9$\,\,$ & $\,\,$3 $\,\,$ \\
$\,\,$ 1/2$\,\,$ & $\,\,$ 1 $\,\,$ & $\,\,$3$\,\,$ & $\,\,$2 $\,\,$ \\
$\,\,$ 1/9$\,\,$ & $\,\,$ 1/3$\,\,$ & $\,\,$ 1 $\,\,$ & $\,\,$ 1/6 $\,\,$ \\
$\,\,$ 1/3$\,\,$ & $\,\,$ 1/2$\,\,$ & $\,\,$6$\,\,$ & $\,\,$ 1  $\,\,$ \\
\end{pmatrix},
\qquad
\lambda_{\max} =
4.1707,
\qquad
CR = 0.0644
\end{equation*}

\begin{equation*}
\mathbf{w}^{cos} =
\begin{pmatrix}
\color{red} 0.498958\color{black} \\
0.249539\\
0.055965\\
0.195539
\end{pmatrix}\end{equation*}
\begin{equation*}
\left[ \frac{{w}^{cos}_i}{{w}^{cos}_j} \right] =
\begin{pmatrix}
$\,\,$ 1 $\,\,$ & $\,\,$\color{red} 1.9995\color{black} $\,\,$ & $\,\,$\color{red} 8.9156\color{black} $\,\,$ & $\,\,$\color{red} 2.5517\color{black} $\,\,$ \\
$\,\,$\color{red} 0.5001\color{black} $\,\,$ & $\,\,$ 1 $\,\,$ & $\,\,$4.4589$\,\,$ & $\,\,$1.2762  $\,\,$ \\
$\,\,$\color{red} 0.1122\color{black} $\,\,$ & $\,\,$0.2243$\,\,$ & $\,\,$ 1 $\,\,$ & $\,\,$0.2862 $\,\,$ \\
$\,\,$\color{red} 0.3919\color{black} $\,\,$ & $\,\,$0.7836$\,\,$ & $\,\,$3.4940$\,\,$ & $\,\,$ 1  $\,\,$ \\
\end{pmatrix},
\end{equation*}

\begin{equation*}
\mathbf{w}^{\prime} =
\begin{pmatrix}
0.499018\\
0.249509\\
0.055958\\
0.195515
\end{pmatrix} =
0.999881\cdot
\begin{pmatrix}
\color{gr} 0.499077\color{black} \\
0.249539\\
0.055965\\
0.195539
\end{pmatrix},
\end{equation*}
\begin{equation*}
\left[ \frac{{w}^{\prime}_i}{{w}^{\prime}_j} \right] =
\begin{pmatrix}
$\,\,$ 1 $\,\,$ & $\,\,$\color{gr} \color{blue} 2\color{black} $\,\,$ & $\,\,$\color{gr} 8.9177\color{black} $\,\,$ & $\,\,$\color{gr} 2.5523\color{black} $\,\,$ \\
$\,\,$\color{gr} \color{blue}  1/2\color{black} $\,\,$ & $\,\,$ 1 $\,\,$ & $\,\,$4.4589$\,\,$ & $\,\,$1.2762  $\,\,$ \\
$\,\,$\color{gr} 0.1121\color{black} $\,\,$ & $\,\,$0.2243$\,\,$ & $\,\,$ 1 $\,\,$ & $\,\,$0.2862 $\,\,$ \\
$\,\,$\color{gr} 0.3918\color{black} $\,\,$ & $\,\,$0.7836$\,\,$ & $\,\,$3.4940$\,\,$ & $\,\,$ 1  $\,\,$ \\
\end{pmatrix},
\end{equation*}
\end{example}
\newpage
\begin{example}
\begin{equation*}
\mathbf{A} =
\begin{pmatrix}
$\,\,$ 1 $\,\,$ & $\,\,$2$\,\,$ & $\,\,$9$\,\,$ & $\,\,$3 $\,\,$ \\
$\,\,$ 1/2$\,\,$ & $\,\,$ 1 $\,\,$ & $\,\,$3$\,\,$ & $\,\,$3 $\,\,$ \\
$\,\,$ 1/9$\,\,$ & $\,\,$ 1/3$\,\,$ & $\,\,$ 1 $\,\,$ & $\,\,$ 1/5 $\,\,$ \\
$\,\,$ 1/3$\,\,$ & $\,\,$ 1/3$\,\,$ & $\,\,$5$\,\,$ & $\,\,$ 1  $\,\,$ \\
\end{pmatrix},
\qquad
\lambda_{\max} =
4.2277,
\qquad
CR = 0.0859
\end{equation*}

\begin{equation*}
\mathbf{w}^{cos} =
\begin{pmatrix}
\color{red} 0.493793\color{black} \\
0.277951\\
0.057609\\
0.170646
\end{pmatrix}\end{equation*}
\begin{equation*}
\left[ \frac{{w}^{cos}_i}{{w}^{cos}_j} \right] =
\begin{pmatrix}
$\,\,$ 1 $\,\,$ & $\,\,$\color{red} 1.7766\color{black} $\,\,$ & $\,\,$\color{red} 8.5714\color{black} $\,\,$ & $\,\,$\color{red} 2.8937\color{black} $\,\,$ \\
$\,\,$\color{red} 0.5629\color{black} $\,\,$ & $\,\,$ 1 $\,\,$ & $\,\,$4.8248$\,\,$ & $\,\,$1.6288  $\,\,$ \\
$\,\,$\color{red} 0.1167\color{black} $\,\,$ & $\,\,$0.2073$\,\,$ & $\,\,$ 1 $\,\,$ & $\,\,$0.3376 $\,\,$ \\
$\,\,$\color{red} 0.3456\color{black} $\,\,$ & $\,\,$0.6139$\,\,$ & $\,\,$2.9621$\,\,$ & $\,\,$ 1  $\,\,$ \\
\end{pmatrix},
\end{equation*}

\begin{equation*}
\mathbf{w}^{\prime} =
\begin{pmatrix}
0.502815\\
0.272997\\
0.056583\\
0.167605
\end{pmatrix} =
0.982177\cdot
\begin{pmatrix}
\color{gr} 0.511939\color{black} \\
0.277951\\
0.057609\\
0.170646
\end{pmatrix},
\end{equation*}
\begin{equation*}
\left[ \frac{{w}^{\prime}_i}{{w}^{\prime}_j} \right] =
\begin{pmatrix}
$\,\,$ 1 $\,\,$ & $\,\,$\color{gr} 1.8418\color{black} $\,\,$ & $\,\,$\color{gr} 8.8864\color{black} $\,\,$ & $\,\,$\color{gr} \color{blue} 3\color{black} $\,\,$ \\
$\,\,$\color{gr} 0.5429\color{black} $\,\,$ & $\,\,$ 1 $\,\,$ & $\,\,$4.8248$\,\,$ & $\,\,$1.6288  $\,\,$ \\
$\,\,$\color{gr} 0.1125\color{black} $\,\,$ & $\,\,$0.2073$\,\,$ & $\,\,$ 1 $\,\,$ & $\,\,$0.3376 $\,\,$ \\
$\,\,$\color{gr} \color{blue}  1/3\color{black} $\,\,$ & $\,\,$0.6139$\,\,$ & $\,\,$2.9621$\,\,$ & $\,\,$ 1  $\,\,$ \\
\end{pmatrix},
\end{equation*}
\end{example}
\newpage
\begin{example}
\begin{equation*}
\mathbf{A} =
\begin{pmatrix}
$\,\,$ 1 $\,\,$ & $\,\,$2$\,\,$ & $\,\,$9$\,\,$ & $\,\,$3 $\,\,$ \\
$\,\,$ 1/2$\,\,$ & $\,\,$ 1 $\,\,$ & $\,\,$6$\,\,$ & $\,\,$6 $\,\,$ \\
$\,\,$ 1/9$\,\,$ & $\,\,$ 1/6$\,\,$ & $\,\,$ 1 $\,\,$ & $\,\,$ 1/2 $\,\,$ \\
$\,\,$ 1/3$\,\,$ & $\,\,$ 1/6$\,\,$ & $\,\,$2$\,\,$ & $\,\,$ 1  $\,\,$ \\
\end{pmatrix},
\qquad
\lambda_{\max} =
4.1707,
\qquad
CR = 0.0644
\end{equation*}

\begin{equation*}
\mathbf{w}^{cos} =
\begin{pmatrix}
0.474840\\
0.363701\\
\color{red} 0.052728\color{black} \\
0.108731
\end{pmatrix}\end{equation*}
\begin{equation*}
\left[ \frac{{w}^{cos}_i}{{w}^{cos}_j} \right] =
\begin{pmatrix}
$\,\,$ 1 $\,\,$ & $\,\,$1.3056$\,\,$ & $\,\,$\color{red} 9.0054\color{black} $\,\,$ & $\,\,$4.3671$\,\,$ \\
$\,\,$0.7659$\,\,$ & $\,\,$ 1 $\,\,$ & $\,\,$\color{red} 6.8977\color{black} $\,\,$ & $\,\,$3.3449  $\,\,$ \\
$\,\,$\color{red} 0.1110\color{black} $\,\,$ & $\,\,$\color{red} 0.1450\color{black} $\,\,$ & $\,\,$ 1 $\,\,$ & $\,\,$\color{red} 0.4849\color{black}  $\,\,$ \\
$\,\,$0.2290$\,\,$ & $\,\,$0.2990$\,\,$ & $\,\,$\color{red} 2.0621\color{black} $\,\,$ & $\,\,$ 1  $\,\,$ \\
\end{pmatrix},
\end{equation*}

\begin{equation*}
\mathbf{w}^{\prime} =
\begin{pmatrix}
0.474824\\
0.363689\\
0.052758\\
0.108728
\end{pmatrix} =
0.999968\cdot
\begin{pmatrix}
0.474840\\
0.363701\\
\color{gr} 0.052760\color{black} \\
0.108731
\end{pmatrix},
\end{equation*}
\begin{equation*}
\left[ \frac{{w}^{\prime}_i}{{w}^{\prime}_j} \right] =
\begin{pmatrix}
$\,\,$ 1 $\,\,$ & $\,\,$1.3056$\,\,$ & $\,\,$\color{gr} \color{blue} 9\color{black} $\,\,$ & $\,\,$4.3671$\,\,$ \\
$\,\,$0.7659$\,\,$ & $\,\,$ 1 $\,\,$ & $\,\,$\color{gr} 6.8935\color{black} $\,\,$ & $\,\,$3.3449  $\,\,$ \\
$\,\,$\color{gr} \color{blue}  1/9\color{black} $\,\,$ & $\,\,$\color{gr} 0.1451\color{black} $\,\,$ & $\,\,$ 1 $\,\,$ & $\,\,$\color{gr} 0.4852\color{black}  $\,\,$ \\
$\,\,$0.2290$\,\,$ & $\,\,$0.2990$\,\,$ & $\,\,$\color{gr} 2.0609\color{black} $\,\,$ & $\,\,$ 1  $\,\,$ \\
\end{pmatrix},
\end{equation*}
\end{example}
\newpage
\begin{example}
\begin{equation*}
\mathbf{A} =
\begin{pmatrix}
$\,\,$ 1 $\,\,$ & $\,\,$2$\,\,$ & $\,\,$9$\,\,$ & $\,\,$3 $\,\,$ \\
$\,\,$ 1/2$\,\,$ & $\,\,$ 1 $\,\,$ & $\,\,$6$\,\,$ & $\,\,$7 $\,\,$ \\
$\,\,$ 1/9$\,\,$ & $\,\,$ 1/6$\,\,$ & $\,\,$ 1 $\,\,$ & $\,\,$ 1/2 $\,\,$ \\
$\,\,$ 1/3$\,\,$ & $\,\,$ 1/7$\,\,$ & $\,\,$2$\,\,$ & $\,\,$ 1  $\,\,$ \\
\end{pmatrix},
\qquad
\lambda_{\max} =
4.2166,
\qquad
CR = 0.0817
\end{equation*}

\begin{equation*}
\mathbf{w}^{cos} =
\begin{pmatrix}
0.471242\\
0.371548\\
\color{red} 0.051869\color{black} \\
0.105341
\end{pmatrix}\end{equation*}
\begin{equation*}
\left[ \frac{{w}^{cos}_i}{{w}^{cos}_j} \right] =
\begin{pmatrix}
$\,\,$ 1 $\,\,$ & $\,\,$1.2683$\,\,$ & $\,\,$\color{red} 9.0853\color{black} $\,\,$ & $\,\,$4.4735$\,\,$ \\
$\,\,$0.7884$\,\,$ & $\,\,$ 1 $\,\,$ & $\,\,$\color{red} 7.1632\color{black} $\,\,$ & $\,\,$3.5271  $\,\,$ \\
$\,\,$\color{red} 0.1101\color{black} $\,\,$ & $\,\,$\color{red} 0.1396\color{black} $\,\,$ & $\,\,$ 1 $\,\,$ & $\,\,$\color{red} 0.4924\color{black}  $\,\,$ \\
$\,\,$0.2235$\,\,$ & $\,\,$0.2835$\,\,$ & $\,\,$\color{red} 2.0309\color{black} $\,\,$ & $\,\,$ 1  $\,\,$ \\
\end{pmatrix},
\end{equation*}

\begin{equation*}
\mathbf{w}^{\prime} =
\begin{pmatrix}
0.471010\\
0.371365\\
0.052334\\
0.105290
\end{pmatrix} =
0.999509\cdot
\begin{pmatrix}
0.471242\\
0.371548\\
\color{gr} 0.052360\color{black} \\
0.105341
\end{pmatrix},
\end{equation*}
\begin{equation*}
\left[ \frac{{w}^{\prime}_i}{{w}^{\prime}_j} \right] =
\begin{pmatrix}
$\,\,$ 1 $\,\,$ & $\,\,$1.2683$\,\,$ & $\,\,$\color{gr} \color{blue} 9\color{black} $\,\,$ & $\,\,$4.4735$\,\,$ \\
$\,\,$0.7884$\,\,$ & $\,\,$ 1 $\,\,$ & $\,\,$\color{gr} 7.0960\color{black} $\,\,$ & $\,\,$3.5271  $\,\,$ \\
$\,\,$\color{gr} \color{blue}  1/9\color{black} $\,\,$ & $\,\,$\color{gr} 0.1409\color{black} $\,\,$ & $\,\,$ 1 $\,\,$ & $\,\,$\color{gr} 0.4971\color{black}  $\,\,$ \\
$\,\,$0.2235$\,\,$ & $\,\,$0.2835$\,\,$ & $\,\,$\color{gr} 2.0119\color{black} $\,\,$ & $\,\,$ 1  $\,\,$ \\
\end{pmatrix},
\end{equation*}
\end{example}
\newpage
\begin{example}
\begin{equation*}
\mathbf{A} =
\begin{pmatrix}
$\,\,$ 1 $\,\,$ & $\,\,$2$\,\,$ & $\,\,$9$\,\,$ & $\,\,$3 $\,\,$ \\
$\,\,$ 1/2$\,\,$ & $\,\,$ 1 $\,\,$ & $\,\,$6$\,\,$ & $\,\,$8 $\,\,$ \\
$\,\,$ 1/9$\,\,$ & $\,\,$ 1/6$\,\,$ & $\,\,$ 1 $\,\,$ & $\,\,$ 1/2 $\,\,$ \\
$\,\,$ 1/3$\,\,$ & $\,\,$ 1/8$\,\,$ & $\,\,$2$\,\,$ & $\,\,$ 1  $\,\,$ \\
\end{pmatrix},
\qquad
\lambda_{\max} =
4.2620,
\qquad
CR = 0.0988
\end{equation*}

\begin{equation*}
\mathbf{w}^{cos} =
\begin{pmatrix}
0.468452\\
0.377677\\
\color{red} 0.051181\color{black} \\
0.102690
\end{pmatrix}\end{equation*}
\begin{equation*}
\left[ \frac{{w}^{cos}_i}{{w}^{cos}_j} \right] =
\begin{pmatrix}
$\,\,$ 1 $\,\,$ & $\,\,$1.2404$\,\,$ & $\,\,$\color{red} 9.1529\color{black} $\,\,$ & $\,\,$4.5618$\,\,$ \\
$\,\,$0.8062$\,\,$ & $\,\,$ 1 $\,\,$ & $\,\,$\color{red} 7.3793\color{black} $\,\,$ & $\,\,$3.6778  $\,\,$ \\
$\,\,$\color{red} 0.1093\color{black} $\,\,$ & $\,\,$\color{red} 0.1355\color{black} $\,\,$ & $\,\,$ 1 $\,\,$ & $\,\,$\color{red} 0.4984\color{black}  $\,\,$ \\
$\,\,$0.2192$\,\,$ & $\,\,$0.2719$\,\,$ & $\,\,$\color{red} 2.0064\color{black} $\,\,$ & $\,\,$ 1  $\,\,$ \\
\end{pmatrix},
\end{equation*}

\begin{equation*}
\mathbf{w}^{\prime} =
\begin{pmatrix}
0.468375\\
0.377615\\
0.051337\\
0.102673
\end{pmatrix} =
0.999836\cdot
\begin{pmatrix}
0.468452\\
0.377677\\
\color{gr} 0.051345\color{black} \\
0.102690
\end{pmatrix},
\end{equation*}
\begin{equation*}
\left[ \frac{{w}^{\prime}_i}{{w}^{\prime}_j} \right] =
\begin{pmatrix}
$\,\,$ 1 $\,\,$ & $\,\,$1.2404$\,\,$ & $\,\,$\color{gr} 9.1236\color{black} $\,\,$ & $\,\,$4.5618$\,\,$ \\
$\,\,$0.8062$\,\,$ & $\,\,$ 1 $\,\,$ & $\,\,$\color{gr} 7.3557\color{black} $\,\,$ & $\,\,$3.6778  $\,\,$ \\
$\,\,$\color{gr} 0.1096\color{black} $\,\,$ & $\,\,$\color{gr} 0.1359\color{black} $\,\,$ & $\,\,$ 1 $\,\,$ & $\,\,$\color{gr} \color{blue}  1/2\color{black}  $\,\,$ \\
$\,\,$0.2192$\,\,$ & $\,\,$0.2719$\,\,$ & $\,\,$\color{gr} \color{blue} 2\color{black} $\,\,$ & $\,\,$ 1  $\,\,$ \\
\end{pmatrix},
\end{equation*}
\end{example}
\newpage
\begin{example}
\begin{equation*}
\mathbf{A} =
\begin{pmatrix}
$\,\,$ 1 $\,\,$ & $\,\,$2$\,\,$ & $\,\,$9$\,\,$ & $\,\,$3 $\,\,$ \\
$\,\,$ 1/2$\,\,$ & $\,\,$ 1 $\,\,$ & $\,\,$7$\,\,$ & $\,\,$5 $\,\,$ \\
$\,\,$ 1/9$\,\,$ & $\,\,$ 1/7$\,\,$ & $\,\,$ 1 $\,\,$ & $\,\,$ 1/2 $\,\,$ \\
$\,\,$ 1/3$\,\,$ & $\,\,$ 1/5$\,\,$ & $\,\,$2$\,\,$ & $\,\,$ 1  $\,\,$ \\
\end{pmatrix},
\qquad
\lambda_{\max} =
4.1239,
\qquad
CR = 0.0467
\end{equation*}

\begin{equation*}
\mathbf{w}^{cos} =
\begin{pmatrix}
0.473666\\
0.362947\\
\color{red} 0.051500\color{black} \\
0.111887
\end{pmatrix}\end{equation*}
\begin{equation*}
\left[ \frac{{w}^{cos}_i}{{w}^{cos}_j} \right] =
\begin{pmatrix}
$\,\,$ 1 $\,\,$ & $\,\,$1.3051$\,\,$ & $\,\,$\color{red} 9.1974\color{black} $\,\,$ & $\,\,$4.2334$\,\,$ \\
$\,\,$0.7663$\,\,$ & $\,\,$ 1 $\,\,$ & $\,\,$\color{red} 7.0475\color{black} $\,\,$ & $\,\,$3.2439  $\,\,$ \\
$\,\,$\color{red} 0.1087\color{black} $\,\,$ & $\,\,$\color{red} 0.1419\color{black} $\,\,$ & $\,\,$ 1 $\,\,$ & $\,\,$\color{red} 0.4603\color{black}  $\,\,$ \\
$\,\,$0.2362$\,\,$ & $\,\,$0.3083$\,\,$ & $\,\,$\color{red} 2.1726\color{black} $\,\,$ & $\,\,$ 1  $\,\,$ \\
\end{pmatrix},
\end{equation*}

\begin{equation*}
\mathbf{w}^{\prime} =
\begin{pmatrix}
0.473501\\
0.362820\\
0.051831\\
0.111848
\end{pmatrix} =
0.999650\cdot
\begin{pmatrix}
0.473666\\
0.362947\\
\color{gr} 0.051850\color{black} \\
0.111887
\end{pmatrix},
\end{equation*}
\begin{equation*}
\left[ \frac{{w}^{\prime}_i}{{w}^{\prime}_j} \right] =
\begin{pmatrix}
$\,\,$ 1 $\,\,$ & $\,\,$1.3051$\,\,$ & $\,\,$\color{gr} 9.1354\color{black} $\,\,$ & $\,\,$4.2334$\,\,$ \\
$\,\,$0.7663$\,\,$ & $\,\,$ 1 $\,\,$ & $\,\,$\color{gr} \color{blue} 7\color{black} $\,\,$ & $\,\,$3.2439  $\,\,$ \\
$\,\,$\color{gr} 0.1095\color{black} $\,\,$ & $\,\,$\color{gr} \color{blue}  1/7\color{black} $\,\,$ & $\,\,$ 1 $\,\,$ & $\,\,$\color{gr} 0.4634\color{black}  $\,\,$ \\
$\,\,$0.2362$\,\,$ & $\,\,$0.3083$\,\,$ & $\,\,$\color{gr} 2.1579\color{black} $\,\,$ & $\,\,$ 1  $\,\,$ \\
\end{pmatrix},
\end{equation*}
\end{example}
\newpage
\begin{example}
\begin{equation*}
\mathbf{A} =
\begin{pmatrix}
$\,\,$ 1 $\,\,$ & $\,\,$2$\,\,$ & $\,\,$9$\,\,$ & $\,\,$3 $\,\,$ \\
$\,\,$ 1/2$\,\,$ & $\,\,$ 1 $\,\,$ & $\,\,$7$\,\,$ & $\,\,$6 $\,\,$ \\
$\,\,$ 1/9$\,\,$ & $\,\,$ 1/7$\,\,$ & $\,\,$ 1 $\,\,$ & $\,\,$ 1/2 $\,\,$ \\
$\,\,$ 1/3$\,\,$ & $\,\,$ 1/6$\,\,$ & $\,\,$2$\,\,$ & $\,\,$ 1  $\,\,$ \\
\end{pmatrix},
\qquad
\lambda_{\max} =
4.1658,
\qquad
CR = 0.0625
\end{equation*}

\begin{equation*}
\mathbf{w}^{cos} =
\begin{pmatrix}
0.468862\\
0.373365\\
\color{red} 0.050382\color{black} \\
0.107391
\end{pmatrix}\end{equation*}
\begin{equation*}
\left[ \frac{{w}^{cos}_i}{{w}^{cos}_j} \right] =
\begin{pmatrix}
$\,\,$ 1 $\,\,$ & $\,\,$1.2558$\,\,$ & $\,\,$\color{red} 9.3061\color{black} $\,\,$ & $\,\,$4.3659$\,\,$ \\
$\,\,$0.7963$\,\,$ & $\,\,$ 1 $\,\,$ & $\,\,$\color{red} 7.4107\color{black} $\,\,$ & $\,\,$3.4767  $\,\,$ \\
$\,\,$\color{red} 0.1075\color{black} $\,\,$ & $\,\,$\color{red} 0.1349\color{black} $\,\,$ & $\,\,$ 1 $\,\,$ & $\,\,$\color{red} 0.4691\color{black}  $\,\,$ \\
$\,\,$0.2290$\,\,$ & $\,\,$0.2876$\,\,$ & $\,\,$\color{red} 2.1315\color{black} $\,\,$ & $\,\,$ 1  $\,\,$ \\
\end{pmatrix},
\end{equation*}

\begin{equation*}
\mathbf{w}^{\prime} =
\begin{pmatrix}
0.468060\\
0.372726\\
0.052007\\
0.107207
\end{pmatrix} =
0.998289\cdot
\begin{pmatrix}
0.468862\\
0.373365\\
\color{gr} 0.052096\color{black} \\
0.107391
\end{pmatrix},
\end{equation*}
\begin{equation*}
\left[ \frac{{w}^{\prime}_i}{{w}^{\prime}_j} \right] =
\begin{pmatrix}
$\,\,$ 1 $\,\,$ & $\,\,$1.2558$\,\,$ & $\,\,$\color{gr} \color{blue} 9\color{black} $\,\,$ & $\,\,$4.3659$\,\,$ \\
$\,\,$0.7963$\,\,$ & $\,\,$ 1 $\,\,$ & $\,\,$\color{gr} 7.1669\color{black} $\,\,$ & $\,\,$3.4767  $\,\,$ \\
$\,\,$\color{gr} \color{blue}  1/9\color{black} $\,\,$ & $\,\,$\color{gr} 0.1395\color{black} $\,\,$ & $\,\,$ 1 $\,\,$ & $\,\,$\color{gr} 0.4851\color{black}  $\,\,$ \\
$\,\,$0.2290$\,\,$ & $\,\,$0.2876$\,\,$ & $\,\,$\color{gr} 2.0614\color{black} $\,\,$ & $\,\,$ 1  $\,\,$ \\
\end{pmatrix},
\end{equation*}
\end{example}
\newpage
\begin{example}
\begin{equation*}
\mathbf{A} =
\begin{pmatrix}
$\,\,$ 1 $\,\,$ & $\,\,$2$\,\,$ & $\,\,$9$\,\,$ & $\,\,$3 $\,\,$ \\
$\,\,$ 1/2$\,\,$ & $\,\,$ 1 $\,\,$ & $\,\,$7$\,\,$ & $\,\,$7 $\,\,$ \\
$\,\,$ 1/9$\,\,$ & $\,\,$ 1/7$\,\,$ & $\,\,$ 1 $\,\,$ & $\,\,$ 1/2 $\,\,$ \\
$\,\,$ 1/3$\,\,$ & $\,\,$ 1/7$\,\,$ & $\,\,$2$\,\,$ & $\,\,$ 1  $\,\,$ \\
\end{pmatrix},
\qquad
\lambda_{\max} =
4.2086,
\qquad
CR = 0.0786
\end{equation*}

\begin{equation*}
\mathbf{w}^{cos} =
\begin{pmatrix}
0.465207\\
0.381309\\
\color{red} 0.049499\color{black} \\
0.103985
\end{pmatrix}\end{equation*}
\begin{equation*}
\left[ \frac{{w}^{cos}_i}{{w}^{cos}_j} \right] =
\begin{pmatrix}
$\,\,$ 1 $\,\,$ & $\,\,$1.2200$\,\,$ & $\,\,$\color{red} 9.3983\color{black} $\,\,$ & $\,\,$4.4738$\,\,$ \\
$\,\,$0.8197$\,\,$ & $\,\,$ 1 $\,\,$ & $\,\,$\color{red} 7.7034\color{black} $\,\,$ & $\,\,$3.6670  $\,\,$ \\
$\,\,$\color{red} 0.1064\color{black} $\,\,$ & $\,\,$\color{red} 0.1298\color{black} $\,\,$ & $\,\,$ 1 $\,\,$ & $\,\,$\color{red} 0.4760\color{black}  $\,\,$ \\
$\,\,$0.2235$\,\,$ & $\,\,$0.2727$\,\,$ & $\,\,$\color{red} 2.1008\color{black} $\,\,$ & $\,\,$ 1  $\,\,$ \\
\end{pmatrix},
\end{equation*}

\begin{equation*}
\mathbf{w}^{\prime} =
\begin{pmatrix}
0.464190\\
0.380476\\
0.051577\\
0.103758
\end{pmatrix} =
0.997814\cdot
\begin{pmatrix}
0.465207\\
0.381309\\
\color{gr} 0.051690\color{black} \\
0.103985
\end{pmatrix},
\end{equation*}
\begin{equation*}
\left[ \frac{{w}^{\prime}_i}{{w}^{\prime}_j} \right] =
\begin{pmatrix}
$\,\,$ 1 $\,\,$ & $\,\,$1.2200$\,\,$ & $\,\,$\color{gr} \color{blue} 9\color{black} $\,\,$ & $\,\,$4.4738$\,\,$ \\
$\,\,$0.8197$\,\,$ & $\,\,$ 1 $\,\,$ & $\,\,$\color{gr} 7.3769\color{black} $\,\,$ & $\,\,$3.6670  $\,\,$ \\
$\,\,$\color{gr} \color{blue}  1/9\color{black} $\,\,$ & $\,\,$\color{gr} 0.1356\color{black} $\,\,$ & $\,\,$ 1 $\,\,$ & $\,\,$\color{gr} 0.4971\color{black}  $\,\,$ \\
$\,\,$0.2235$\,\,$ & $\,\,$0.2727$\,\,$ & $\,\,$\color{gr} 2.0117\color{black} $\,\,$ & $\,\,$ 1  $\,\,$ \\
\end{pmatrix},
\end{equation*}
\end{example}
\newpage
\begin{example}
\begin{equation*}
\mathbf{A} =
\begin{pmatrix}
$\,\,$ 1 $\,\,$ & $\,\,$2$\,\,$ & $\,\,$9$\,\,$ & $\,\,$3 $\,\,$ \\
$\,\,$ 1/2$\,\,$ & $\,\,$ 1 $\,\,$ & $\,\,$7$\,\,$ & $\,\,$8 $\,\,$ \\
$\,\,$ 1/9$\,\,$ & $\,\,$ 1/7$\,\,$ & $\,\,$ 1 $\,\,$ & $\,\,$ 1/2 $\,\,$ \\
$\,\,$ 1/3$\,\,$ & $\,\,$ 1/8$\,\,$ & $\,\,$2$\,\,$ & $\,\,$ 1  $\,\,$ \\
\end{pmatrix},
\qquad
\lambda_{\max} =
4.2512,
\qquad
CR = 0.0947
\end{equation*}

\begin{equation*}
\mathbf{w}^{cos} =
\begin{pmatrix}
0.462367\\
0.387523\\
\color{red} 0.048790\color{black} \\
0.101320
\end{pmatrix}\end{equation*}
\begin{equation*}
\left[ \frac{{w}^{cos}_i}{{w}^{cos}_j} \right] =
\begin{pmatrix}
$\,\,$ 1 $\,\,$ & $\,\,$1.1931$\,\,$ & $\,\,$\color{red} 9.4767\color{black} $\,\,$ & $\,\,$4.5634$\,\,$ \\
$\,\,$0.8381$\,\,$ & $\,\,$ 1 $\,\,$ & $\,\,$\color{red} 7.9427\color{black} $\,\,$ & $\,\,$3.8247  $\,\,$ \\
$\,\,$\color{red} 0.1055\color{black} $\,\,$ & $\,\,$\color{red} 0.1259\color{black} $\,\,$ & $\,\,$ 1 $\,\,$ & $\,\,$\color{red} 0.4815\color{black}  $\,\,$ \\
$\,\,$0.2191$\,\,$ & $\,\,$0.2615$\,\,$ & $\,\,$\color{red} 2.0767\color{black} $\,\,$ & $\,\,$ 1  $\,\,$ \\
\end{pmatrix},
\end{equation*}

\begin{equation*}
\mathbf{w}^{\prime} =
\begin{pmatrix}
0.461504\\
0.386800\\
0.050566\\
0.101131
\end{pmatrix} =
0.998133\cdot
\begin{pmatrix}
0.462367\\
0.387523\\
\color{gr} 0.050660\color{black} \\
0.101320
\end{pmatrix},
\end{equation*}
\begin{equation*}
\left[ \frac{{w}^{\prime}_i}{{w}^{\prime}_j} \right] =
\begin{pmatrix}
$\,\,$ 1 $\,\,$ & $\,\,$1.1931$\,\,$ & $\,\,$\color{gr} 9.1268\color{black} $\,\,$ & $\,\,$4.5634$\,\,$ \\
$\,\,$0.8381$\,\,$ & $\,\,$ 1 $\,\,$ & $\,\,$\color{gr} 7.6495\color{black} $\,\,$ & $\,\,$3.8247  $\,\,$ \\
$\,\,$\color{gr} 0.1096\color{black} $\,\,$ & $\,\,$\color{gr} 0.1307\color{black} $\,\,$ & $\,\,$ 1 $\,\,$ & $\,\,$\color{gr} \color{blue}  1/2\color{black}  $\,\,$ \\
$\,\,$0.2191$\,\,$ & $\,\,$0.2615$\,\,$ & $\,\,$\color{gr} \color{blue} 2\color{black} $\,\,$ & $\,\,$ 1  $\,\,$ \\
\end{pmatrix},
\end{equation*}
\end{example}
\newpage
\begin{example}
\begin{equation*}
\mathbf{A} =
\begin{pmatrix}
$\,\,$ 1 $\,\,$ & $\,\,$2$\,\,$ & $\,\,$9$\,\,$ & $\,\,$3 $\,\,$ \\
$\,\,$ 1/2$\,\,$ & $\,\,$ 1 $\,\,$ & $\,\,$8$\,\,$ & $\,\,$7 $\,\,$ \\
$\,\,$ 1/9$\,\,$ & $\,\,$ 1/8$\,\,$ & $\,\,$ 1 $\,\,$ & $\,\,$ 1/2 $\,\,$ \\
$\,\,$ 1/3$\,\,$ & $\,\,$ 1/7$\,\,$ & $\,\,$2$\,\,$ & $\,\,$ 1  $\,\,$ \\
\end{pmatrix},
\qquad
\lambda_{\max} =
4.2065,
\qquad
CR = 0.0779
\end{equation*}

\begin{equation*}
\mathbf{w}^{cos} =
\begin{pmatrix}
0.459617\\
0.390053\\
\color{red} 0.047597\color{black} \\
0.102734
\end{pmatrix}\end{equation*}
\begin{equation*}
\left[ \frac{{w}^{cos}_i}{{w}^{cos}_j} \right] =
\begin{pmatrix}
$\,\,$ 1 $\,\,$ & $\,\,$1.1783$\,\,$ & $\,\,$\color{red} 9.6564\color{black} $\,\,$ & $\,\,$4.4739$\,\,$ \\
$\,\,$0.8486$\,\,$ & $\,\,$ 1 $\,\,$ & $\,\,$\color{red} 8.1949\color{black} $\,\,$ & $\,\,$3.7967  $\,\,$ \\
$\,\,$\color{red} 0.1036\color{black} $\,\,$ & $\,\,$\color{red} 0.1220\color{black} $\,\,$ & $\,\,$ 1 $\,\,$ & $\,\,$\color{red} 0.4633\color{black}  $\,\,$ \\
$\,\,$0.2235$\,\,$ & $\,\,$0.2634$\,\,$ & $\,\,$\color{red} 2.1584\color{black} $\,\,$ & $\,\,$ 1  $\,\,$ \\
\end{pmatrix},
\end{equation*}

\begin{equation*}
\mathbf{w}^{\prime} =
\begin{pmatrix}
0.459085\\
0.389601\\
0.048700\\
0.102615
\end{pmatrix} =
0.998842\cdot
\begin{pmatrix}
0.459617\\
0.390053\\
\color{gr} 0.048757\color{black} \\
0.102734
\end{pmatrix},
\end{equation*}
\begin{equation*}
\left[ \frac{{w}^{\prime}_i}{{w}^{\prime}_j} \right] =
\begin{pmatrix}
$\,\,$ 1 $\,\,$ & $\,\,$1.1783$\,\,$ & $\,\,$\color{gr} 9.4268\color{black} $\,\,$ & $\,\,$4.4739$\,\,$ \\
$\,\,$0.8486$\,\,$ & $\,\,$ 1 $\,\,$ & $\,\,$\color{gr} \color{blue} 8\color{black} $\,\,$ & $\,\,$3.7967  $\,\,$ \\
$\,\,$\color{gr} 0.1061\color{black} $\,\,$ & $\,\,$\color{gr} \color{blue}  1/8\color{black} $\,\,$ & $\,\,$ 1 $\,\,$ & $\,\,$\color{gr} 0.4746\color{black}  $\,\,$ \\
$\,\,$0.2235$\,\,$ & $\,\,$0.2634$\,\,$ & $\,\,$\color{gr} 2.1071\color{black} $\,\,$ & $\,\,$ 1  $\,\,$ \\
\end{pmatrix},
\end{equation*}
\end{example}
\newpage
\begin{example}
\begin{equation*}
\mathbf{A} =
\begin{pmatrix}
$\,\,$ 1 $\,\,$ & $\,\,$2$\,\,$ & $\,\,$9$\,\,$ & $\,\,$3 $\,\,$ \\
$\,\,$ 1/2$\,\,$ & $\,\,$ 1 $\,\,$ & $\,\,$8$\,\,$ & $\,\,$8 $\,\,$ \\
$\,\,$ 1/9$\,\,$ & $\,\,$ 1/8$\,\,$ & $\,\,$ 1 $\,\,$ & $\,\,$ 1/2 $\,\,$ \\
$\,\,$ 1/3$\,\,$ & $\,\,$ 1/8$\,\,$ & $\,\,$2$\,\,$ & $\,\,$ 1  $\,\,$ \\
\end{pmatrix},
\qquad
\lambda_{\max} =
4.2469,
\qquad
CR = 0.0931
\end{equation*}

\begin{equation*}
\mathbf{w}^{cos} =
\begin{pmatrix}
0.456727\\
0.396348\\
\color{red} 0.046871\color{black} \\
0.100054
\end{pmatrix}\end{equation*}
\begin{equation*}
\left[ \frac{{w}^{cos}_i}{{w}^{cos}_j} \right] =
\begin{pmatrix}
$\,\,$ 1 $\,\,$ & $\,\,$1.1523$\,\,$ & $\,\,$\color{red} 9.7443\color{black} $\,\,$ & $\,\,$4.5648$\,\,$ \\
$\,\,$0.8678$\,\,$ & $\,\,$ 1 $\,\,$ & $\,\,$\color{red} 8.4561\color{black} $\,\,$ & $\,\,$3.9613  $\,\,$ \\
$\,\,$\color{red} 0.1026\color{black} $\,\,$ & $\,\,$\color{red} 0.1183\color{black} $\,\,$ & $\,\,$ 1 $\,\,$ & $\,\,$\color{red} 0.4685\color{black}  $\,\,$ \\
$\,\,$0.2191$\,\,$ & $\,\,$0.2524$\,\,$ & $\,\,$\color{red} 2.1347\color{black} $\,\,$ & $\,\,$ 1  $\,\,$ \\
\end{pmatrix},
\end{equation*}

\begin{equation*}
\mathbf{w}^{\prime} =
\begin{pmatrix}
0.455510\\
0.395291\\
0.049411\\
0.099787
\end{pmatrix} =
0.997335\cdot
\begin{pmatrix}
0.456727\\
0.396348\\
\color{gr} 0.049543\color{black} \\
0.100054
\end{pmatrix},
\end{equation*}
\begin{equation*}
\left[ \frac{{w}^{\prime}_i}{{w}^{\prime}_j} \right] =
\begin{pmatrix}
$\,\,$ 1 $\,\,$ & $\,\,$1.1523$\,\,$ & $\,\,$\color{gr} 9.2187\color{black} $\,\,$ & $\,\,$4.5648$\,\,$ \\
$\,\,$0.8678$\,\,$ & $\,\,$ 1 $\,\,$ & $\,\,$\color{gr} \color{blue} 8\color{black} $\,\,$ & $\,\,$3.9613  $\,\,$ \\
$\,\,$\color{gr} 0.1085\color{black} $\,\,$ & $\,\,$\color{gr} \color{blue}  1/8\color{black} $\,\,$ & $\,\,$ 1 $\,\,$ & $\,\,$\color{gr} 0.4952\color{black}  $\,\,$ \\
$\,\,$0.2191$\,\,$ & $\,\,$0.2524$\,\,$ & $\,\,$\color{gr} 2.0195\color{black} $\,\,$ & $\,\,$ 1  $\,\,$ \\
\end{pmatrix},
\end{equation*}
\end{example}
\newpage
\begin{example}
\begin{equation*}
\mathbf{A} =
\begin{pmatrix}
$\,\,$ 1 $\,\,$ & $\,\,$2$\,\,$ & $\,\,$9$\,\,$ & $\,\,$4 $\,\,$ \\
$\,\,$ 1/2$\,\,$ & $\,\,$ 1 $\,\,$ & $\,\,$3$\,\,$ & $\,\,$3 $\,\,$ \\
$\,\,$ 1/9$\,\,$ & $\,\,$ 1/3$\,\,$ & $\,\,$ 1 $\,\,$ & $\,\,$ 1/3 $\,\,$ \\
$\,\,$ 1/4$\,\,$ & $\,\,$ 1/3$\,\,$ & $\,\,$3$\,\,$ & $\,\,$ 1  $\,\,$ \\
\end{pmatrix},
\qquad
\lambda_{\max} =
4.1031,
\qquad
CR = 0.0389
\end{equation*}

\begin{equation*}
\mathbf{w}^{cos} =
\begin{pmatrix}
\color{red} 0.531097\color{black} \\
0.272631\\
0.063176\\
0.133095
\end{pmatrix}\end{equation*}
\begin{equation*}
\left[ \frac{{w}^{cos}_i}{{w}^{cos}_j} \right] =
\begin{pmatrix}
$\,\,$ 1 $\,\,$ & $\,\,$\color{red} 1.9480\color{black} $\,\,$ & $\,\,$\color{red} 8.4066\color{black} $\,\,$ & $\,\,$\color{red} 3.9904\color{black} $\,\,$ \\
$\,\,$\color{red} 0.5133\color{black} $\,\,$ & $\,\,$ 1 $\,\,$ & $\,\,$4.3154$\,\,$ & $\,\,$2.0484  $\,\,$ \\
$\,\,$\color{red} 0.1190\color{black} $\,\,$ & $\,\,$0.2317$\,\,$ & $\,\,$ 1 $\,\,$ & $\,\,$0.4747 $\,\,$ \\
$\,\,$\color{red} 0.2506\color{black} $\,\,$ & $\,\,$0.4882$\,\,$ & $\,\,$2.1067$\,\,$ & $\,\,$ 1  $\,\,$ \\
\end{pmatrix},
\end{equation*}

\begin{equation*}
\mathbf{w}^{\prime} =
\begin{pmatrix}
0.531698\\
0.272282\\
0.063096\\
0.132924
\end{pmatrix} =
0.998720\cdot
\begin{pmatrix}
\color{gr} 0.532379\color{black} \\
0.272631\\
0.063176\\
0.133095
\end{pmatrix},
\end{equation*}
\begin{equation*}
\left[ \frac{{w}^{\prime}_i}{{w}^{\prime}_j} \right] =
\begin{pmatrix}
$\,\,$ 1 $\,\,$ & $\,\,$\color{gr} 1.9527\color{black} $\,\,$ & $\,\,$\color{gr} 8.4269\color{black} $\,\,$ & $\,\,$\color{gr} \color{blue} 4\color{black} $\,\,$ \\
$\,\,$\color{gr} 0.5121\color{black} $\,\,$ & $\,\,$ 1 $\,\,$ & $\,\,$4.3154$\,\,$ & $\,\,$2.0484  $\,\,$ \\
$\,\,$\color{gr} 0.1187\color{black} $\,\,$ & $\,\,$0.2317$\,\,$ & $\,\,$ 1 $\,\,$ & $\,\,$0.4747 $\,\,$ \\
$\,\,$\color{gr} \color{blue}  1/4\color{black} $\,\,$ & $\,\,$0.4882$\,\,$ & $\,\,$2.1067$\,\,$ & $\,\,$ 1  $\,\,$ \\
\end{pmatrix},
\end{equation*}
\end{example}
\newpage
\begin{example}
\begin{equation*}
\mathbf{A} =
\begin{pmatrix}
$\,\,$ 1 $\,\,$ & $\,\,$2$\,\,$ & $\,\,$9$\,\,$ & $\,\,$4 $\,\,$ \\
$\,\,$ 1/2$\,\,$ & $\,\,$ 1 $\,\,$ & $\,\,$3$\,\,$ & $\,\,$3 $\,\,$ \\
$\,\,$ 1/9$\,\,$ & $\,\,$ 1/3$\,\,$ & $\,\,$ 1 $\,\,$ & $\,\,$ 1/4 $\,\,$ \\
$\,\,$ 1/4$\,\,$ & $\,\,$ 1/3$\,\,$ & $\,\,$4$\,\,$ & $\,\,$ 1  $\,\,$ \\
\end{pmatrix},
\qquad
\lambda_{\max} =
4.1664,
\qquad
CR = 0.0627
\end{equation*}

\begin{equation*}
\mathbf{w}^{cos} =
\begin{pmatrix}
\color{red} 0.524243\color{black} \\
0.269971\\
0.059875\\
0.145912
\end{pmatrix}\end{equation*}
\begin{equation*}
\left[ \frac{{w}^{cos}_i}{{w}^{cos}_j} \right] =
\begin{pmatrix}
$\,\,$ 1 $\,\,$ & $\,\,$\color{red} 1.9419\color{black} $\,\,$ & $\,\,$\color{red} 8.7557\color{black} $\,\,$ & $\,\,$\color{red} 3.5929\color{black} $\,\,$ \\
$\,\,$\color{red} 0.5150\color{black} $\,\,$ & $\,\,$ 1 $\,\,$ & $\,\,$4.5089$\,\,$ & $\,\,$1.8502  $\,\,$ \\
$\,\,$\color{red} 0.1142\color{black} $\,\,$ & $\,\,$0.2218$\,\,$ & $\,\,$ 1 $\,\,$ & $\,\,$0.4103 $\,\,$ \\
$\,\,$\color{red} 0.2783\color{black} $\,\,$ & $\,\,$0.5405$\,\,$ & $\,\,$2.4370$\,\,$ & $\,\,$ 1  $\,\,$ \\
\end{pmatrix},
\end{equation*}

\begin{equation*}
\mathbf{w}^{\prime} =
\begin{pmatrix}
0.531102\\
0.266079\\
0.059011\\
0.143808
\end{pmatrix} =
0.985583\cdot
\begin{pmatrix}
\color{gr} 0.538871\color{black} \\
0.269971\\
0.059875\\
0.145912
\end{pmatrix},
\end{equation*}
\begin{equation*}
\left[ \frac{{w}^{\prime}_i}{{w}^{\prime}_j} \right] =
\begin{pmatrix}
$\,\,$ 1 $\,\,$ & $\,\,$\color{gr} 1.9960\color{black} $\,\,$ & $\,\,$\color{gr} \color{blue} 9\color{black} $\,\,$ & $\,\,$\color{gr} 3.6931\color{black} $\,\,$ \\
$\,\,$\color{gr} 0.5010\color{black} $\,\,$ & $\,\,$ 1 $\,\,$ & $\,\,$4.5089$\,\,$ & $\,\,$1.8502  $\,\,$ \\
$\,\,$\color{gr} \color{blue}  1/9\color{black} $\,\,$ & $\,\,$0.2218$\,\,$ & $\,\,$ 1 $\,\,$ & $\,\,$0.4103 $\,\,$ \\
$\,\,$\color{gr} 0.2708\color{black} $\,\,$ & $\,\,$0.5405$\,\,$ & $\,\,$2.4370$\,\,$ & $\,\,$ 1  $\,\,$ \\
\end{pmatrix},
\end{equation*}
\end{example}
\newpage
\begin{example}
\begin{equation*}
\mathbf{A} =
\begin{pmatrix}
$\,\,$ 1 $\,\,$ & $\,\,$2$\,\,$ & $\,\,$9$\,\,$ & $\,\,$4 $\,\,$ \\
$\,\,$ 1/2$\,\,$ & $\,\,$ 1 $\,\,$ & $\,\,$3$\,\,$ & $\,\,$3 $\,\,$ \\
$\,\,$ 1/9$\,\,$ & $\,\,$ 1/3$\,\,$ & $\,\,$ 1 $\,\,$ & $\,\,$ 1/5 $\,\,$ \\
$\,\,$ 1/4$\,\,$ & $\,\,$ 1/3$\,\,$ & $\,\,$5$\,\,$ & $\,\,$ 1  $\,\,$ \\
\end{pmatrix},
\qquad
\lambda_{\max} =
4.2316,
\qquad
CR = 0.0873
\end{equation*}

\begin{equation*}
\mathbf{w}^{cos} =
\begin{pmatrix}
\color{red} 0.517489\color{black} \\
0.267467\\
0.057609\\
0.157435
\end{pmatrix}\end{equation*}
\begin{equation*}
\left[ \frac{{w}^{cos}_i}{{w}^{cos}_j} \right] =
\begin{pmatrix}
$\,\,$ 1 $\,\,$ & $\,\,$\color{red} 1.9348\color{black} $\,\,$ & $\,\,$\color{red} 8.9827\color{black} $\,\,$ & $\,\,$\color{red} 3.2870\color{black} $\,\,$ \\
$\,\,$\color{red} 0.5169\color{black} $\,\,$ & $\,\,$ 1 $\,\,$ & $\,\,$4.6428$\,\,$ & $\,\,$1.6989  $\,\,$ \\
$\,\,$\color{red} 0.1113\color{black} $\,\,$ & $\,\,$0.2154$\,\,$ & $\,\,$ 1 $\,\,$ & $\,\,$0.3659 $\,\,$ \\
$\,\,$\color{red} 0.3042\color{black} $\,\,$ & $\,\,$0.5886$\,\,$ & $\,\,$2.7328$\,\,$ & $\,\,$ 1  $\,\,$ \\
\end{pmatrix},
\end{equation*}

\begin{equation*}
\mathbf{w}^{\prime} =
\begin{pmatrix}
0.517968\\
0.267201\\
0.057552\\
0.157279
\end{pmatrix} =
0.999005\cdot
\begin{pmatrix}
\color{gr} 0.518484\color{black} \\
0.267467\\
0.057609\\
0.157435
\end{pmatrix},
\end{equation*}
\begin{equation*}
\left[ \frac{{w}^{\prime}_i}{{w}^{\prime}_j} \right] =
\begin{pmatrix}
$\,\,$ 1 $\,\,$ & $\,\,$\color{gr} 1.9385\color{black} $\,\,$ & $\,\,$\color{gr} \color{blue} 9\color{black} $\,\,$ & $\,\,$\color{gr} 3.2933\color{black} $\,\,$ \\
$\,\,$\color{gr} 0.5159\color{black} $\,\,$ & $\,\,$ 1 $\,\,$ & $\,\,$4.6428$\,\,$ & $\,\,$1.6989  $\,\,$ \\
$\,\,$\color{gr} \color{blue}  1/9\color{black} $\,\,$ & $\,\,$0.2154$\,\,$ & $\,\,$ 1 $\,\,$ & $\,\,$0.3659 $\,\,$ \\
$\,\,$\color{gr} 0.3036\color{black} $\,\,$ & $\,\,$0.5886$\,\,$ & $\,\,$2.7328$\,\,$ & $\,\,$ 1  $\,\,$ \\
\end{pmatrix},
\end{equation*}
\end{example}
\newpage
\begin{example}
\begin{equation*}
\mathbf{A} =
\begin{pmatrix}
$\,\,$ 1 $\,\,$ & $\,\,$2$\,\,$ & $\,\,$9$\,\,$ & $\,\,$4 $\,\,$ \\
$\,\,$ 1/2$\,\,$ & $\,\,$ 1 $\,\,$ & $\,\,$3$\,\,$ & $\,\,$4 $\,\,$ \\
$\,\,$ 1/9$\,\,$ & $\,\,$ 1/3$\,\,$ & $\,\,$ 1 $\,\,$ & $\,\,$ 1/4 $\,\,$ \\
$\,\,$ 1/4$\,\,$ & $\,\,$ 1/4$\,\,$ & $\,\,$4$\,\,$ & $\,\,$ 1  $\,\,$ \\
\end{pmatrix},
\qquad
\lambda_{\max} =
4.2469,
\qquad
CR = 0.0931
\end{equation*}

\begin{equation*}
\mathbf{w}^{cos} =
\begin{pmatrix}
\color{red} 0.514209\color{black} \\
0.288576\\
0.059467\\
0.137749
\end{pmatrix}\end{equation*}
\begin{equation*}
\left[ \frac{{w}^{cos}_i}{{w}^{cos}_j} \right] =
\begin{pmatrix}
$\,\,$ 1 $\,\,$ & $\,\,$\color{red} 1.7819\color{black} $\,\,$ & $\,\,$\color{red} 8.6470\color{black} $\,\,$ & $\,\,$\color{red} 3.7329\color{black} $\,\,$ \\
$\,\,$\color{red} 0.5612\color{black} $\,\,$ & $\,\,$ 1 $\,\,$ & $\,\,$4.8527$\,\,$ & $\,\,$2.0949  $\,\,$ \\
$\,\,$\color{red} 0.1156\color{black} $\,\,$ & $\,\,$0.2061$\,\,$ & $\,\,$ 1 $\,\,$ & $\,\,$0.4317 $\,\,$ \\
$\,\,$\color{red} 0.2679\color{black} $\,\,$ & $\,\,$0.4773$\,\,$ & $\,\,$2.3164$\,\,$ & $\,\,$ 1  $\,\,$ \\
\end{pmatrix},
\end{equation*}

\begin{equation*}
\mathbf{w}^{\prime} =
\begin{pmatrix}
0.524198\\
0.282642\\
0.058244\\
0.134916
\end{pmatrix} =
0.979437\cdot
\begin{pmatrix}
\color{gr} 0.535203\color{black} \\
0.288576\\
0.059467\\
0.137749
\end{pmatrix},
\end{equation*}
\begin{equation*}
\left[ \frac{{w}^{\prime}_i}{{w}^{\prime}_j} \right] =
\begin{pmatrix}
$\,\,$ 1 $\,\,$ & $\,\,$\color{gr} 1.8546\color{black} $\,\,$ & $\,\,$\color{gr} \color{blue} 9\color{black} $\,\,$ & $\,\,$\color{gr} 3.8854\color{black} $\,\,$ \\
$\,\,$\color{gr} 0.5392\color{black} $\,\,$ & $\,\,$ 1 $\,\,$ & $\,\,$4.8527$\,\,$ & $\,\,$2.0949  $\,\,$ \\
$\,\,$\color{gr} \color{blue}  1/9\color{black} $\,\,$ & $\,\,$0.2061$\,\,$ & $\,\,$ 1 $\,\,$ & $\,\,$0.4317 $\,\,$ \\
$\,\,$\color{gr} 0.2574\color{black} $\,\,$ & $\,\,$0.4773$\,\,$ & $\,\,$2.3164$\,\,$ & $\,\,$ 1  $\,\,$ \\
\end{pmatrix},
\end{equation*}
\end{example}
\newpage
\begin{example}
\begin{equation*}
\mathbf{A} =
\begin{pmatrix}
$\,\,$ 1 $\,\,$ & $\,\,$2$\,\,$ & $\,\,$9$\,\,$ & $\,\,$5 $\,\,$ \\
$\,\,$ 1/2$\,\,$ & $\,\,$ 1 $\,\,$ & $\,\,$3$\,\,$ & $\,\,$4 $\,\,$ \\
$\,\,$ 1/9$\,\,$ & $\,\,$ 1/3$\,\,$ & $\,\,$ 1 $\,\,$ & $\,\,$ 1/3 $\,\,$ \\
$\,\,$ 1/5$\,\,$ & $\,\,$ 1/4$\,\,$ & $\,\,$3$\,\,$ & $\,\,$ 1  $\,\,$ \\
\end{pmatrix},
\qquad
\lambda_{\max} =
4.1655,
\qquad
CR = 0.0624
\end{equation*}

\begin{equation*}
\mathbf{w}^{cos} =
\begin{pmatrix}
\color{red} 0.539131\color{black} \\
0.282335\\
0.062202\\
0.116332
\end{pmatrix}\end{equation*}
\begin{equation*}
\left[ \frac{{w}^{cos}_i}{{w}^{cos}_j} \right] =
\begin{pmatrix}
$\,\,$ 1 $\,\,$ & $\,\,$\color{red} 1.9095\color{black} $\,\,$ & $\,\,$\color{red} 8.6675\color{black} $\,\,$ & $\,\,$\color{red} 4.6344\color{black} $\,\,$ \\
$\,\,$\color{red} 0.5237\color{black} $\,\,$ & $\,\,$ 1 $\,\,$ & $\,\,$4.5390$\,\,$ & $\,\,$2.4270  $\,\,$ \\
$\,\,$\color{red} 0.1154\color{black} $\,\,$ & $\,\,$0.2203$\,\,$ & $\,\,$ 1 $\,\,$ & $\,\,$0.5347 $\,\,$ \\
$\,\,$\color{red} 0.2158\color{black} $\,\,$ & $\,\,$0.4120$\,\,$ & $\,\,$1.8702$\,\,$ & $\,\,$ 1  $\,\,$ \\
\end{pmatrix},
\end{equation*}

\begin{equation*}
\mathbf{w}^{\prime} =
\begin{pmatrix}
0.548471\\
0.276614\\
0.060941\\
0.113975
\end{pmatrix} =
0.979734\cdot
\begin{pmatrix}
\color{gr} 0.559816\color{black} \\
0.282335\\
0.062202\\
0.116332
\end{pmatrix},
\end{equation*}
\begin{equation*}
\left[ \frac{{w}^{\prime}_i}{{w}^{\prime}_j} \right] =
\begin{pmatrix}
$\,\,$ 1 $\,\,$ & $\,\,$\color{gr} 1.9828\color{black} $\,\,$ & $\,\,$\color{gr} \color{blue} 9\color{black} $\,\,$ & $\,\,$\color{gr} 4.8122\color{black} $\,\,$ \\
$\,\,$\color{gr} 0.5043\color{black} $\,\,$ & $\,\,$ 1 $\,\,$ & $\,\,$4.5390$\,\,$ & $\,\,$2.4270  $\,\,$ \\
$\,\,$\color{gr} \color{blue}  1/9\color{black} $\,\,$ & $\,\,$0.2203$\,\,$ & $\,\,$ 1 $\,\,$ & $\,\,$0.5347 $\,\,$ \\
$\,\,$\color{gr} 0.2078\color{black} $\,\,$ & $\,\,$0.4120$\,\,$ & $\,\,$1.8702$\,\,$ & $\,\,$ 1  $\,\,$ \\
\end{pmatrix},
\end{equation*}
\end{example}
\newpage
\begin{example}
\begin{equation*}
\mathbf{A} =
\begin{pmatrix}
$\,\,$ 1 $\,\,$ & $\,\,$2$\,\,$ & $\,\,$9$\,\,$ & $\,\,$5 $\,\,$ \\
$\,\,$ 1/2$\,\,$ & $\,\,$ 1 $\,\,$ & $\,\,$3$\,\,$ & $\,\,$4 $\,\,$ \\
$\,\,$ 1/9$\,\,$ & $\,\,$ 1/3$\,\,$ & $\,\,$ 1 $\,\,$ & $\,\,$ 1/4 $\,\,$ \\
$\,\,$ 1/5$\,\,$ & $\,\,$ 1/4$\,\,$ & $\,\,$4$\,\,$ & $\,\,$ 1  $\,\,$ \\
\end{pmatrix},
\qquad
\lambda_{\max} =
4.2500,
\qquad
CR = 0.0942
\end{equation*}

\begin{equation*}
\mathbf{w}^{cos} =
\begin{pmatrix}
\color{red} 0.531852\color{black} \\
0.279435\\
0.059367\\
0.129347
\end{pmatrix}\end{equation*}
\begin{equation*}
\left[ \frac{{w}^{cos}_i}{{w}^{cos}_j} \right] =
\begin{pmatrix}
$\,\,$ 1 $\,\,$ & $\,\,$\color{red} 1.9033\color{black} $\,\,$ & $\,\,$\color{red} 8.9588\color{black} $\,\,$ & $\,\,$\color{red} 4.1118\color{black} $\,\,$ \\
$\,\,$\color{red} 0.5254\color{black} $\,\,$ & $\,\,$ 1 $\,\,$ & $\,\,$4.7069$\,\,$ & $\,\,$2.1604  $\,\,$ \\
$\,\,$\color{red} 0.1116\color{black} $\,\,$ & $\,\,$0.2125$\,\,$ & $\,\,$ 1 $\,\,$ & $\,\,$0.4590 $\,\,$ \\
$\,\,$\color{red} 0.2432\color{black} $\,\,$ & $\,\,$0.4629$\,\,$ & $\,\,$2.1788$\,\,$ & $\,\,$ 1  $\,\,$ \\
\end{pmatrix},
\end{equation*}

\begin{equation*}
\mathbf{w}^{\prime} =
\begin{pmatrix}
0.532994\\
0.278753\\
0.059222\\
0.129031
\end{pmatrix} =
0.997559\cdot
\begin{pmatrix}
\color{gr} 0.534299\color{black} \\
0.279435\\
0.059367\\
0.129347
\end{pmatrix},
\end{equation*}
\begin{equation*}
\left[ \frac{{w}^{\prime}_i}{{w}^{\prime}_j} \right] =
\begin{pmatrix}
$\,\,$ 1 $\,\,$ & $\,\,$\color{gr} 1.9121\color{black} $\,\,$ & $\,\,$\color{gr} \color{blue} 9\color{black} $\,\,$ & $\,\,$\color{gr} 4.1307\color{black} $\,\,$ \\
$\,\,$\color{gr} 0.5230\color{black} $\,\,$ & $\,\,$ 1 $\,\,$ & $\,\,$4.7069$\,\,$ & $\,\,$2.1604  $\,\,$ \\
$\,\,$\color{gr} \color{blue}  1/9\color{black} $\,\,$ & $\,\,$0.2125$\,\,$ & $\,\,$ 1 $\,\,$ & $\,\,$0.4590 $\,\,$ \\
$\,\,$\color{gr} 0.2421\color{black} $\,\,$ & $\,\,$0.4629$\,\,$ & $\,\,$2.1788$\,\,$ & $\,\,$ 1  $\,\,$ \\
\end{pmatrix},
\end{equation*}
\end{example}
\newpage
\begin{example}
\begin{equation*}
\mathbf{A} =
\begin{pmatrix}
$\,\,$ 1 $\,\,$ & $\,\,$2$\,\,$ & $\,\,$9$\,\,$ & $\,\,$5 $\,\,$ \\
$\,\,$ 1/2$\,\,$ & $\,\,$ 1 $\,\,$ & $\,\,$3$\,\,$ & $\,\,$5 $\,\,$ \\
$\,\,$ 1/9$\,\,$ & $\,\,$ 1/3$\,\,$ & $\,\,$ 1 $\,\,$ & $\,\,$ 1/3 $\,\,$ \\
$\,\,$ 1/5$\,\,$ & $\,\,$ 1/5$\,\,$ & $\,\,$3$\,\,$ & $\,\,$ 1  $\,\,$ \\
\end{pmatrix},
\qquad
\lambda_{\max} =
4.2277,
\qquad
CR = 0.0859
\end{equation*}

\begin{equation*}
\mathbf{w}^{cos} =
\begin{pmatrix}
\color{red} 0.530386\color{black} \\
0.296744\\
0.061756\\
0.111114
\end{pmatrix}\end{equation*}
\begin{equation*}
\left[ \frac{{w}^{cos}_i}{{w}^{cos}_j} \right] =
\begin{pmatrix}
$\,\,$ 1 $\,\,$ & $\,\,$\color{red} 1.7874\color{black} $\,\,$ & $\,\,$\color{red} 8.5885\color{black} $\,\,$ & $\,\,$\color{red} 4.7733\color{black} $\,\,$ \\
$\,\,$\color{red} 0.5595\color{black} $\,\,$ & $\,\,$ 1 $\,\,$ & $\,\,$4.8051$\,\,$ & $\,\,$2.6706  $\,\,$ \\
$\,\,$\color{red} 0.1164\color{black} $\,\,$ & $\,\,$0.2081$\,\,$ & $\,\,$ 1 $\,\,$ & $\,\,$0.5558 $\,\,$ \\
$\,\,$\color{red} 0.2095\color{black} $\,\,$ & $\,\,$0.3744$\,\,$ & $\,\,$1.7993$\,\,$ & $\,\,$ 1  $\,\,$ \\
\end{pmatrix},
\end{equation*}

\begin{equation*}
\mathbf{w}^{\prime} =
\begin{pmatrix}
0.541923\\
0.289454\\
0.060239\\
0.108385
\end{pmatrix} =
0.975434\cdot
\begin{pmatrix}
\color{gr} 0.555571\color{black} \\
0.296744\\
0.061756\\
0.111114
\end{pmatrix},
\end{equation*}
\begin{equation*}
\left[ \frac{{w}^{\prime}_i}{{w}^{\prime}_j} \right] =
\begin{pmatrix}
$\,\,$ 1 $\,\,$ & $\,\,$\color{gr} 1.8722\color{black} $\,\,$ & $\,\,$\color{gr} 8.9963\color{black} $\,\,$ & $\,\,$\color{gr} \color{blue} 5\color{black} $\,\,$ \\
$\,\,$\color{gr} 0.5341\color{black} $\,\,$ & $\,\,$ 1 $\,\,$ & $\,\,$4.8051$\,\,$ & $\,\,$2.6706  $\,\,$ \\
$\,\,$\color{gr} 0.1112\color{black} $\,\,$ & $\,\,$0.2081$\,\,$ & $\,\,$ 1 $\,\,$ & $\,\,$0.5558 $\,\,$ \\
$\,\,$\color{gr} \color{blue}  1/5\color{black} $\,\,$ & $\,\,$0.3744$\,\,$ & $\,\,$1.7993$\,\,$ & $\,\,$ 1  $\,\,$ \\
\end{pmatrix},
\end{equation*}
\end{example}
\newpage
\begin{example}
\begin{equation*}
\mathbf{A} =
\begin{pmatrix}
$\,\,$ 1 $\,\,$ & $\,\,$2$\,\,$ & $\,\,$9$\,\,$ & $\,\,$6 $\,\,$ \\
$\,\,$ 1/2$\,\,$ & $\,\,$ 1 $\,\,$ & $\,\,$3$\,\,$ & $\,\,$5 $\,\,$ \\
$\,\,$ 1/9$\,\,$ & $\,\,$ 1/3$\,\,$ & $\,\,$ 1 $\,\,$ & $\,\,$ 1/3 $\,\,$ \\
$\,\,$ 1/6$\,\,$ & $\,\,$ 1/5$\,\,$ & $\,\,$3$\,\,$ & $\,\,$ 1  $\,\,$ \\
\end{pmatrix},
\qquad
\lambda_{\max} =
4.2277,
\qquad
CR = 0.0859
\end{equation*}

\begin{equation*}
\mathbf{w}^{cos} =
\begin{pmatrix}
\color{red} 0.544455\color{black} \\
0.288734\\
0.061554\\
0.105257
\end{pmatrix}\end{equation*}
\begin{equation*}
\left[ \frac{{w}^{cos}_i}{{w}^{cos}_j} \right] =
\begin{pmatrix}
$\,\,$ 1 $\,\,$ & $\,\,$\color{red} 1.8857\color{black} $\,\,$ & $\,\,$\color{red} 8.8451\color{black} $\,\,$ & $\,\,$\color{red} 5.1726\color{black} $\,\,$ \\
$\,\,$\color{red} 0.5303\color{black} $\,\,$ & $\,\,$ 1 $\,\,$ & $\,\,$4.6907$\,\,$ & $\,\,$2.7431  $\,\,$ \\
$\,\,$\color{red} 0.1131\color{black} $\,\,$ & $\,\,$0.2132$\,\,$ & $\,\,$ 1 $\,\,$ & $\,\,$0.5848 $\,\,$ \\
$\,\,$\color{red} 0.1933\color{black} $\,\,$ & $\,\,$0.3645$\,\,$ & $\,\,$1.7100$\,\,$ & $\,\,$ 1  $\,\,$ \\
\end{pmatrix},
\end{equation*}

\begin{equation*}
\mathbf{w}^{\prime} =
\begin{pmatrix}
0.548757\\
0.286007\\
0.060973\\
0.104263
\end{pmatrix} =
0.990555\cdot
\begin{pmatrix}
\color{gr} 0.553990\color{black} \\
0.288734\\
0.061554\\
0.105257
\end{pmatrix},
\end{equation*}
\begin{equation*}
\left[ \frac{{w}^{\prime}_i}{{w}^{\prime}_j} \right] =
\begin{pmatrix}
$\,\,$ 1 $\,\,$ & $\,\,$\color{gr} 1.9187\color{black} $\,\,$ & $\,\,$\color{gr} \color{blue} 9\color{black} $\,\,$ & $\,\,$\color{gr} 5.2632\color{black} $\,\,$ \\
$\,\,$\color{gr} 0.5212\color{black} $\,\,$ & $\,\,$ 1 $\,\,$ & $\,\,$4.6907$\,\,$ & $\,\,$2.7431  $\,\,$ \\
$\,\,$\color{gr} \color{blue}  1/9\color{black} $\,\,$ & $\,\,$0.2132$\,\,$ & $\,\,$ 1 $\,\,$ & $\,\,$0.5848 $\,\,$ \\
$\,\,$\color{gr} 0.1900\color{black} $\,\,$ & $\,\,$0.3645$\,\,$ & $\,\,$1.7100$\,\,$ & $\,\,$ 1  $\,\,$ \\
\end{pmatrix},
\end{equation*}
\end{example}
\newpage
\begin{example}
\begin{equation*}
\mathbf{A} =
\begin{pmatrix}
$\,\,$ 1 $\,\,$ & $\,\,$2$\,\,$ & $\,\,$9$\,\,$ & $\,\,$6 $\,\,$ \\
$\,\,$ 1/2$\,\,$ & $\,\,$ 1 $\,\,$ & $\,\,$6$\,\,$ & $\,\,$2 $\,\,$ \\
$\,\,$ 1/9$\,\,$ & $\,\,$ 1/6$\,\,$ & $\,\,$ 1 $\,\,$ & $\,\,$ 1/4 $\,\,$ \\
$\,\,$ 1/6$\,\,$ & $\,\,$ 1/2$\,\,$ & $\,\,$4$\,\,$ & $\,\,$ 1  $\,\,$ \\
\end{pmatrix},
\qquad
\lambda_{\max} =
4.0820,
\qquad
CR = 0.0309
\end{equation*}

\begin{equation*}
\mathbf{w}^{cos} =
\begin{pmatrix}
0.547147\\
\color{red} 0.269393\color{black} \\
0.046707\\
0.136752
\end{pmatrix}\end{equation*}
\begin{equation*}
\left[ \frac{{w}^{cos}_i}{{w}^{cos}_j} \right] =
\begin{pmatrix}
$\,\,$ 1 $\,\,$ & $\,\,$\color{red} 2.0310\color{black} $\,\,$ & $\,\,$11.7144$\,\,$ & $\,\,$4.0010$\,\,$ \\
$\,\,$\color{red} 0.4924\color{black} $\,\,$ & $\,\,$ 1 $\,\,$ & $\,\,$\color{red} 5.7677\color{black} $\,\,$ & $\,\,$\color{red} 1.9699\color{black}   $\,\,$ \\
$\,\,$0.0854$\,\,$ & $\,\,$\color{red} 0.1734\color{black} $\,\,$ & $\,\,$ 1 $\,\,$ & $\,\,$0.3415 $\,\,$ \\
$\,\,$0.2499$\,\,$ & $\,\,$\color{red} 0.5076\color{black} $\,\,$ & $\,\,$2.9279$\,\,$ & $\,\,$ 1  $\,\,$ \\
\end{pmatrix},
\end{equation*}

\begin{equation*}
\mathbf{w}^{\prime} =
\begin{pmatrix}
0.544907\\
0.272385\\
0.046516\\
0.136192
\end{pmatrix} =
0.995905\cdot
\begin{pmatrix}
0.547147\\
\color{gr} 0.273505\color{black} \\
0.046707\\
0.136752
\end{pmatrix},
\end{equation*}
\begin{equation*}
\left[ \frac{{w}^{\prime}_i}{{w}^{\prime}_j} \right] =
\begin{pmatrix}
$\,\,$ 1 $\,\,$ & $\,\,$\color{gr} 2.0005\color{black} $\,\,$ & $\,\,$11.7144$\,\,$ & $\,\,$4.0010$\,\,$ \\
$\,\,$\color{gr} 0.4999\color{black} $\,\,$ & $\,\,$ 1 $\,\,$ & $\,\,$\color{gr} 5.8557\color{black} $\,\,$ & $\,\,$\color{gr} \color{blue} 2\color{black}   $\,\,$ \\
$\,\,$0.0854$\,\,$ & $\,\,$\color{gr} 0.1708\color{black} $\,\,$ & $\,\,$ 1 $\,\,$ & $\,\,$0.3415 $\,\,$ \\
$\,\,$0.2499$\,\,$ & $\,\,$\color{gr} \color{blue}  1/2\color{black} $\,\,$ & $\,\,$2.9279$\,\,$ & $\,\,$ 1  $\,\,$ \\
\end{pmatrix},
\end{equation*}
\end{example}
\newpage
\begin{example}
\begin{equation*}
\mathbf{A} =
\begin{pmatrix}
$\,\,$ 1 $\,\,$ & $\,\,$2$\,\,$ & $\,\,$9$\,\,$ & $\,\,$6 $\,\,$ \\
$\,\,$ 1/2$\,\,$ & $\,\,$ 1 $\,\,$ & $\,\,$6$\,\,$ & $\,\,$2 $\,\,$ \\
$\,\,$ 1/9$\,\,$ & $\,\,$ 1/6$\,\,$ & $\,\,$ 1 $\,\,$ & $\,\,$ 1/5 $\,\,$ \\
$\,\,$ 1/6$\,\,$ & $\,\,$ 1/2$\,\,$ & $\,\,$5$\,\,$ & $\,\,$ 1  $\,\,$ \\
\end{pmatrix},
\qquad
\lambda_{\max} =
4.1252,
\qquad
CR = 0.0472
\end{equation*}

\begin{equation*}
\mathbf{w}^{cos} =
\begin{pmatrix}
0.541469\\
\color{red} 0.265909\color{black} \\
0.044891\\
0.147731
\end{pmatrix}\end{equation*}
\begin{equation*}
\left[ \frac{{w}^{cos}_i}{{w}^{cos}_j} \right] =
\begin{pmatrix}
$\,\,$ 1 $\,\,$ & $\,\,$\color{red} 2.0363\color{black} $\,\,$ & $\,\,$12.0617$\,\,$ & $\,\,$3.6652$\,\,$ \\
$\,\,$\color{red} 0.4911\color{black} $\,\,$ & $\,\,$ 1 $\,\,$ & $\,\,$\color{red} 5.9234\color{black} $\,\,$ & $\,\,$\color{red} 1.8000\color{black}   $\,\,$ \\
$\,\,$0.0829$\,\,$ & $\,\,$\color{red} 0.1688\color{black} $\,\,$ & $\,\,$ 1 $\,\,$ & $\,\,$0.3039 $\,\,$ \\
$\,\,$0.2728$\,\,$ & $\,\,$\color{red} 0.5556\color{black} $\,\,$ & $\,\,$3.2908$\,\,$ & $\,\,$ 1  $\,\,$ \\
\end{pmatrix},
\end{equation*}

\begin{equation*}
\mathbf{w}^{\prime} =
\begin{pmatrix}
0.539613\\
0.268425\\
0.044738\\
0.147224
\end{pmatrix} =
0.996572\cdot
\begin{pmatrix}
0.541469\\
\color{gr} 0.269349\color{black} \\
0.044891\\
0.147731
\end{pmatrix},
\end{equation*}
\begin{equation*}
\left[ \frac{{w}^{\prime}_i}{{w}^{\prime}_j} \right] =
\begin{pmatrix}
$\,\,$ 1 $\,\,$ & $\,\,$\color{gr} 2.0103\color{black} $\,\,$ & $\,\,$12.0617$\,\,$ & $\,\,$3.6652$\,\,$ \\
$\,\,$\color{gr} 0.4974\color{black} $\,\,$ & $\,\,$ 1 $\,\,$ & $\,\,$\color{gr} \color{blue} 6\color{black} $\,\,$ & $\,\,$\color{gr} 1.8232\color{black}   $\,\,$ \\
$\,\,$0.0829$\,\,$ & $\,\,$\color{gr} \color{blue}  1/6\color{black} $\,\,$ & $\,\,$ 1 $\,\,$ & $\,\,$0.3039 $\,\,$ \\
$\,\,$0.2728$\,\,$ & $\,\,$\color{gr} 0.5485\color{black} $\,\,$ & $\,\,$3.2908$\,\,$ & $\,\,$ 1  $\,\,$ \\
\end{pmatrix},
\end{equation*}
\end{example}
\newpage
\begin{example}
\begin{equation*}
\mathbf{A} =
\begin{pmatrix}
$\,\,$ 1 $\,\,$ & $\,\,$2$\,\,$ & $\,\,$9$\,\,$ & $\,\,$7 $\,\,$ \\
$\,\,$ 1/2$\,\,$ & $\,\,$ 1 $\,\,$ & $\,\,$3$\,\,$ & $\,\,$5 $\,\,$ \\
$\,\,$ 1/9$\,\,$ & $\,\,$ 1/3$\,\,$ & $\,\,$ 1 $\,\,$ & $\,\,$ 1/2 $\,\,$ \\
$\,\,$ 1/7$\,\,$ & $\,\,$ 1/5$\,\,$ & $\,\,$2$\,\,$ & $\,\,$ 1  $\,\,$ \\
\end{pmatrix},
\qquad
\lambda_{\max} =
4.1239,
\qquad
CR = 0.0467
\end{equation*}

\begin{equation*}
\mathbf{w}^{cos} =
\begin{pmatrix}
\color{red} 0.563564\color{black} \\
0.284849\\
0.065349\\
0.086237
\end{pmatrix}\end{equation*}
\begin{equation*}
\left[ \frac{{w}^{cos}_i}{{w}^{cos}_j} \right] =
\begin{pmatrix}
$\,\,$ 1 $\,\,$ & $\,\,$\color{red} 1.9785\color{black} $\,\,$ & $\,\,$\color{red} 8.6239\color{black} $\,\,$ & $\,\,$\color{red} 6.5350\color{black} $\,\,$ \\
$\,\,$\color{red} 0.5054\color{black} $\,\,$ & $\,\,$ 1 $\,\,$ & $\,\,$4.3589$\,\,$ & $\,\,$3.3031  $\,\,$ \\
$\,\,$\color{red} 0.1160\color{black} $\,\,$ & $\,\,$0.2294$\,\,$ & $\,\,$ 1 $\,\,$ & $\,\,$0.7578 $\,\,$ \\
$\,\,$\color{red} 0.1530\color{black} $\,\,$ & $\,\,$0.3027$\,\,$ & $\,\,$1.3196$\,\,$ & $\,\,$ 1  $\,\,$ \\
\end{pmatrix},
\end{equation*}

\begin{equation*}
\mathbf{w}^{\prime} =
\begin{pmatrix}
0.566225\\
0.283113\\
0.064951\\
0.085711
\end{pmatrix} =
0.993902\cdot
\begin{pmatrix}
\color{gr} 0.569699\color{black} \\
0.284849\\
0.065349\\
0.086237
\end{pmatrix},
\end{equation*}
\begin{equation*}
\left[ \frac{{w}^{\prime}_i}{{w}^{\prime}_j} \right] =
\begin{pmatrix}
$\,\,$ 1 $\,\,$ & $\,\,$\color{gr} \color{blue} 2\color{black} $\,\,$ & $\,\,$\color{gr} 8.7178\color{black} $\,\,$ & $\,\,$\color{gr} 6.6062\color{black} $\,\,$ \\
$\,\,$\color{gr} \color{blue}  1/2\color{black} $\,\,$ & $\,\,$ 1 $\,\,$ & $\,\,$4.3589$\,\,$ & $\,\,$3.3031  $\,\,$ \\
$\,\,$\color{gr} 0.1147\color{black} $\,\,$ & $\,\,$0.2294$\,\,$ & $\,\,$ 1 $\,\,$ & $\,\,$0.7578 $\,\,$ \\
$\,\,$\color{gr} 0.1514\color{black} $\,\,$ & $\,\,$0.3027$\,\,$ & $\,\,$1.3196$\,\,$ & $\,\,$ 1  $\,\,$ \\
\end{pmatrix},
\end{equation*}
\end{example}
\newpage
\begin{example}
\begin{equation*}
\mathbf{A} =
\begin{pmatrix}
$\,\,$ 1 $\,\,$ & $\,\,$2$\,\,$ & $\,\,$9$\,\,$ & $\,\,$7 $\,\,$ \\
$\,\,$ 1/2$\,\,$ & $\,\,$ 1 $\,\,$ & $\,\,$3$\,\,$ & $\,\,$6 $\,\,$ \\
$\,\,$ 1/9$\,\,$ & $\,\,$ 1/3$\,\,$ & $\,\,$ 1 $\,\,$ & $\,\,$ 1/2 $\,\,$ \\
$\,\,$ 1/7$\,\,$ & $\,\,$ 1/6$\,\,$ & $\,\,$2$\,\,$ & $\,\,$ 1  $\,\,$ \\
\end{pmatrix},
\qquad
\lambda_{\max} =
4.1658,
\qquad
CR = 0.0625
\end{equation*}

\begin{equation*}
\mathbf{w}^{cos} =
\begin{pmatrix}
\color{red} 0.555851\color{black} \\
0.296543\\
0.064846\\
0.082761
\end{pmatrix}\end{equation*}
\begin{equation*}
\left[ \frac{{w}^{cos}_i}{{w}^{cos}_j} \right] =
\begin{pmatrix}
$\,\,$ 1 $\,\,$ & $\,\,$\color{red} 1.8744\color{black} $\,\,$ & $\,\,$\color{red} 8.5718\color{black} $\,\,$ & $\,\,$\color{red} 6.7163\color{black} $\,\,$ \\
$\,\,$\color{red} 0.5335\color{black} $\,\,$ & $\,\,$ 1 $\,\,$ & $\,\,$4.5730$\,\,$ & $\,\,$3.5831  $\,\,$ \\
$\,\,$\color{red} 0.1167\color{black} $\,\,$ & $\,\,$0.2187$\,\,$ & $\,\,$ 1 $\,\,$ & $\,\,$0.7835 $\,\,$ \\
$\,\,$\color{red} 0.1489\color{black} $\,\,$ & $\,\,$0.2791$\,\,$ & $\,\,$1.2763$\,\,$ & $\,\,$ 1  $\,\,$ \\
\end{pmatrix},
\end{equation*}

\begin{equation*}
\mathbf{w}^{\prime} =
\begin{pmatrix}
0.566038\\
0.289741\\
0.063359\\
0.080863
\end{pmatrix} =
0.977063\cdot
\begin{pmatrix}
\color{gr} 0.579326\color{black} \\
0.296543\\
0.064846\\
0.082761
\end{pmatrix},
\end{equation*}
\begin{equation*}
\left[ \frac{{w}^{\prime}_i}{{w}^{\prime}_j} \right] =
\begin{pmatrix}
$\,\,$ 1 $\,\,$ & $\,\,$\color{gr} 1.9536\color{black} $\,\,$ & $\,\,$\color{gr} 8.9339\color{black} $\,\,$ & $\,\,$\color{gr} \color{blue} 7\color{black} $\,\,$ \\
$\,\,$\color{gr} 0.5119\color{black} $\,\,$ & $\,\,$ 1 $\,\,$ & $\,\,$4.5730$\,\,$ & $\,\,$3.5831  $\,\,$ \\
$\,\,$\color{gr} 0.1119\color{black} $\,\,$ & $\,\,$0.2187$\,\,$ & $\,\,$ 1 $\,\,$ & $\,\,$0.7835 $\,\,$ \\
$\,\,$\color{gr} \color{blue}  1/7\color{black} $\,\,$ & $\,\,$0.2791$\,\,$ & $\,\,$1.2763$\,\,$ & $\,\,$ 1  $\,\,$ \\
\end{pmatrix},
\end{equation*}
\end{example}
\newpage
\begin{example}
\begin{equation*}
\mathbf{A} =
\begin{pmatrix}
$\,\,$ 1 $\,\,$ & $\,\,$2$\,\,$ & $\,\,$9$\,\,$ & $\,\,$7 $\,\,$ \\
$\,\,$ 1/2$\,\,$ & $\,\,$ 1 $\,\,$ & $\,\,$3$\,\,$ & $\,\,$7 $\,\,$ \\
$\,\,$ 1/9$\,\,$ & $\,\,$ 1/3$\,\,$ & $\,\,$ 1 $\,\,$ & $\,\,$ 1/2 $\,\,$ \\
$\,\,$ 1/7$\,\,$ & $\,\,$ 1/7$\,\,$ & $\,\,$2$\,\,$ & $\,\,$ 1  $\,\,$ \\
\end{pmatrix},
\qquad
\lambda_{\max} =
4.2086,
\qquad
CR = 0.0786
\end{equation*}

\begin{equation*}
\mathbf{w}^{cos} =
\begin{pmatrix}
\color{red} 0.549020\color{black} \\
0.306466\\
0.064428\\
0.080086
\end{pmatrix}\end{equation*}
\begin{equation*}
\left[ \frac{{w}^{cos}_i}{{w}^{cos}_j} \right] =
\begin{pmatrix}
$\,\,$ 1 $\,\,$ & $\,\,$\color{red} 1.7915\color{black} $\,\,$ & $\,\,$\color{red} 8.5214\color{black} $\,\,$ & $\,\,$\color{red} 6.8554\color{black} $\,\,$ \\
$\,\,$\color{red} 0.5582\color{black} $\,\,$ & $\,\,$ 1 $\,\,$ & $\,\,$4.7567$\,\,$ & $\,\,$3.8267  $\,\,$ \\
$\,\,$\color{red} 0.1174\color{black} $\,\,$ & $\,\,$0.2102$\,\,$ & $\,\,$ 1 $\,\,$ & $\,\,$0.8045 $\,\,$ \\
$\,\,$\color{red} 0.1459\color{black} $\,\,$ & $\,\,$0.2613$\,\,$ & $\,\,$1.2430$\,\,$ & $\,\,$ 1  $\,\,$ \\
\end{pmatrix},
\end{equation*}

\begin{equation*}
\mathbf{w}^{\prime} =
\begin{pmatrix}
0.554183\\
0.302957\\
0.063691\\
0.079169
\end{pmatrix} =
0.988552\cdot
\begin{pmatrix}
\color{gr} 0.560600\color{black} \\
0.306466\\
0.064428\\
0.080086
\end{pmatrix},
\end{equation*}
\begin{equation*}
\left[ \frac{{w}^{\prime}_i}{{w}^{\prime}_j} \right] =
\begin{pmatrix}
$\,\,$ 1 $\,\,$ & $\,\,$\color{gr} 1.8292\color{black} $\,\,$ & $\,\,$\color{gr} 8.7011\color{black} $\,\,$ & $\,\,$\color{gr} \color{blue} 7\color{black} $\,\,$ \\
$\,\,$\color{gr} 0.5467\color{black} $\,\,$ & $\,\,$ 1 $\,\,$ & $\,\,$4.7567$\,\,$ & $\,\,$3.8267  $\,\,$ \\
$\,\,$\color{gr} 0.1149\color{black} $\,\,$ & $\,\,$0.2102$\,\,$ & $\,\,$ 1 $\,\,$ & $\,\,$0.8045 $\,\,$ \\
$\,\,$\color{gr} \color{blue}  1/7\color{black} $\,\,$ & $\,\,$0.2613$\,\,$ & $\,\,$1.2430$\,\,$ & $\,\,$ 1  $\,\,$ \\
\end{pmatrix},
\end{equation*}
\end{example}
\newpage
\begin{example}
\begin{equation*}
\mathbf{A} =
\begin{pmatrix}
$\,\,$ 1 $\,\,$ & $\,\,$2$\,\,$ & $\,\,$9$\,\,$ & $\,\,$7 $\,\,$ \\
$\,\,$ 1/2$\,\,$ & $\,\,$ 1 $\,\,$ & $\,\,$3$\,\,$ & $\,\,$8 $\,\,$ \\
$\,\,$ 1/9$\,\,$ & $\,\,$ 1/3$\,\,$ & $\,\,$ 1 $\,\,$ & $\,\,$ 1/2 $\,\,$ \\
$\,\,$ 1/7$\,\,$ & $\,\,$ 1/8$\,\,$ & $\,\,$2$\,\,$ & $\,\,$ 1  $\,\,$ \\
\end{pmatrix},
\qquad
\lambda_{\max} =
4.2512,
\qquad
CR = 0.0947
\end{equation*}

\begin{equation*}
\mathbf{w}^{cos} =
\begin{pmatrix}
\color{red} 0.543058\color{black} \\
0.314900\\
0.064088\\
0.077954
\end{pmatrix}\end{equation*}
\begin{equation*}
\left[ \frac{{w}^{cos}_i}{{w}^{cos}_j} \right] =
\begin{pmatrix}
$\,\,$ 1 $\,\,$ & $\,\,$\color{red} 1.7245\color{black} $\,\,$ & $\,\,$\color{red} 8.4736\color{black} $\,\,$ & $\,\,$\color{red} 6.9664\color{black} $\,\,$ \\
$\,\,$\color{red} 0.5799\color{black} $\,\,$ & $\,\,$ 1 $\,\,$ & $\,\,$4.9135$\,\,$ & $\,\,$4.0396  $\,\,$ \\
$\,\,$\color{red} 0.1180\color{black} $\,\,$ & $\,\,$0.2035$\,\,$ & $\,\,$ 1 $\,\,$ & $\,\,$0.8221 $\,\,$ \\
$\,\,$\color{red} 0.1435\color{black} $\,\,$ & $\,\,$0.2476$\,\,$ & $\,\,$1.2164$\,\,$ & $\,\,$ 1  $\,\,$ \\
\end{pmatrix},
\end{equation*}

\begin{equation*}
\mathbf{w}^{\prime} =
\begin{pmatrix}
0.544252\\
0.314077\\
0.063921\\
0.077750
\end{pmatrix} =
0.997388\cdot
\begin{pmatrix}
\color{gr} 0.545677\color{black} \\
0.314900\\
0.064088\\
0.077954
\end{pmatrix},
\end{equation*}
\begin{equation*}
\left[ \frac{{w}^{\prime}_i}{{w}^{\prime}_j} \right] =
\begin{pmatrix}
$\,\,$ 1 $\,\,$ & $\,\,$\color{gr} 1.7329\color{black} $\,\,$ & $\,\,$\color{gr} 8.5145\color{black} $\,\,$ & $\,\,$\color{gr} \color{blue} 7\color{black} $\,\,$ \\
$\,\,$\color{gr} 0.5771\color{black} $\,\,$ & $\,\,$ 1 $\,\,$ & $\,\,$4.9135$\,\,$ & $\,\,$4.0396  $\,\,$ \\
$\,\,$\color{gr} 0.1174\color{black} $\,\,$ & $\,\,$0.2035$\,\,$ & $\,\,$ 1 $\,\,$ & $\,\,$0.8221 $\,\,$ \\
$\,\,$\color{gr} \color{blue}  1/7\color{black} $\,\,$ & $\,\,$0.2476$\,\,$ & $\,\,$1.2164$\,\,$ & $\,\,$ 1  $\,\,$ \\
\end{pmatrix},
\end{equation*}
\end{example}
\newpage
\begin{example}
\begin{equation*}
\mathbf{A} =
\begin{pmatrix}
$\,\,$ 1 $\,\,$ & $\,\,$2$\,\,$ & $\,\,$9$\,\,$ & $\,\,$7 $\,\,$ \\
$\,\,$ 1/2$\,\,$ & $\,\,$ 1 $\,\,$ & $\,\,$7$\,\,$ & $\,\,$2 $\,\,$ \\
$\,\,$ 1/9$\,\,$ & $\,\,$ 1/7$\,\,$ & $\,\,$ 1 $\,\,$ & $\,\,$ 1/5 $\,\,$ \\
$\,\,$ 1/7$\,\,$ & $\,\,$ 1/2$\,\,$ & $\,\,$5$\,\,$ & $\,\,$ 1  $\,\,$ \\
\end{pmatrix},
\qquad
\lambda_{\max} =
4.1597,
\qquad
CR = 0.0602
\end{equation*}

\begin{equation*}
\mathbf{w}^{cos} =
\begin{pmatrix}
0.545308\\
\color{red} 0.272035\color{black} \\
0.042582\\
0.140075
\end{pmatrix}\end{equation*}
\begin{equation*}
\left[ \frac{{w}^{cos}_i}{{w}^{cos}_j} \right] =
\begin{pmatrix}
$\,\,$ 1 $\,\,$ & $\,\,$\color{red} 2.0045\color{black} $\,\,$ & $\,\,$12.8060$\,\,$ & $\,\,$3.8930$\,\,$ \\
$\,\,$\color{red} 0.4989\color{black} $\,\,$ & $\,\,$ 1 $\,\,$ & $\,\,$\color{red} 6.3885\color{black} $\,\,$ & $\,\,$\color{red} 1.9421\color{black}   $\,\,$ \\
$\,\,$0.0781$\,\,$ & $\,\,$\color{red} 0.1565\color{black} $\,\,$ & $\,\,$ 1 $\,\,$ & $\,\,$0.3040 $\,\,$ \\
$\,\,$0.2569$\,\,$ & $\,\,$\color{red} 0.5149\color{black} $\,\,$ & $\,\,$3.2895$\,\,$ & $\,\,$ 1  $\,\,$ \\
\end{pmatrix},
\end{equation*}

\begin{equation*}
\mathbf{w}^{\prime} =
\begin{pmatrix}
0.544971\\
0.272485\\
0.042556\\
0.139988
\end{pmatrix} =
0.999382\cdot
\begin{pmatrix}
0.545308\\
\color{gr} 0.272654\color{black} \\
0.042582\\
0.140075
\end{pmatrix},
\end{equation*}
\begin{equation*}
\left[ \frac{{w}^{\prime}_i}{{w}^{\prime}_j} \right] =
\begin{pmatrix}
$\,\,$ 1 $\,\,$ & $\,\,$\color{gr} \color{blue} 2\color{black} $\,\,$ & $\,\,$12.8060$\,\,$ & $\,\,$3.8930$\,\,$ \\
$\,\,$\color{gr} \color{blue}  1/2\color{black} $\,\,$ & $\,\,$ 1 $\,\,$ & $\,\,$\color{gr} 6.4030\color{black} $\,\,$ & $\,\,$\color{gr} 1.9465\color{black}   $\,\,$ \\
$\,\,$0.0781$\,\,$ & $\,\,$\color{gr} 0.1562\color{black} $\,\,$ & $\,\,$ 1 $\,\,$ & $\,\,$0.3040 $\,\,$ \\
$\,\,$0.2569$\,\,$ & $\,\,$\color{gr} 0.5137\color{black} $\,\,$ & $\,\,$3.2895$\,\,$ & $\,\,$ 1  $\,\,$ \\
\end{pmatrix},
\end{equation*}
\end{example}
\newpage
\begin{example}
\begin{equation*}
\mathbf{A} =
\begin{pmatrix}
$\,\,$ 1 $\,\,$ & $\,\,$2$\,\,$ & $\,\,$9$\,\,$ & $\,\,$7 $\,\,$ \\
$\,\,$ 1/2$\,\,$ & $\,\,$ 1 $\,\,$ & $\,\,$7$\,\,$ & $\,\,$2 $\,\,$ \\
$\,\,$ 1/9$\,\,$ & $\,\,$ 1/7$\,\,$ & $\,\,$ 1 $\,\,$ & $\,\,$ 1/6 $\,\,$ \\
$\,\,$ 1/7$\,\,$ & $\,\,$ 1/2$\,\,$ & $\,\,$6$\,\,$ & $\,\,$ 1  $\,\,$ \\
\end{pmatrix},
\qquad
\lambda_{\max} =
4.2059,
\qquad
CR = 0.0776
\end{equation*}

\begin{equation*}
\mathbf{w}^{cos} =
\begin{pmatrix}
0.540280\\
\color{red} 0.268455\color{black} \\
0.041342\\
0.149923
\end{pmatrix}\end{equation*}
\begin{equation*}
\left[ \frac{{w}^{cos}_i}{{w}^{cos}_j} \right] =
\begin{pmatrix}
$\,\,$ 1 $\,\,$ & $\,\,$\color{red} 2.0126\color{black} $\,\,$ & $\,\,$13.0685$\,\,$ & $\,\,$3.6037$\,\,$ \\
$\,\,$\color{red} 0.4969\color{black} $\,\,$ & $\,\,$ 1 $\,\,$ & $\,\,$\color{red} 6.4935\color{black} $\,\,$ & $\,\,$\color{red} 1.7906\color{black}   $\,\,$ \\
$\,\,$0.0765$\,\,$ & $\,\,$\color{red} 0.1540\color{black} $\,\,$ & $\,\,$ 1 $\,\,$ & $\,\,$0.2758 $\,\,$ \\
$\,\,$0.2775$\,\,$ & $\,\,$\color{red} 0.5585\color{black} $\,\,$ & $\,\,$3.6264$\,\,$ & $\,\,$ 1  $\,\,$ \\
\end{pmatrix},
\end{equation*}

\begin{equation*}
\mathbf{w}^{\prime} =
\begin{pmatrix}
0.539371\\
0.269686\\
0.041273\\
0.149671
\end{pmatrix} =
0.998317\cdot
\begin{pmatrix}
0.540280\\
\color{gr} 0.270140\color{black} \\
0.041342\\
0.149923
\end{pmatrix},
\end{equation*}
\begin{equation*}
\left[ \frac{{w}^{\prime}_i}{{w}^{\prime}_j} \right] =
\begin{pmatrix}
$\,\,$ 1 $\,\,$ & $\,\,$\color{gr} \color{blue} 2\color{black} $\,\,$ & $\,\,$13.0685$\,\,$ & $\,\,$3.6037$\,\,$ \\
$\,\,$\color{gr} \color{blue}  1/2\color{black} $\,\,$ & $\,\,$ 1 $\,\,$ & $\,\,$\color{gr} 6.5343\color{black} $\,\,$ & $\,\,$\color{gr} 1.8019\color{black}   $\,\,$ \\
$\,\,$0.0765$\,\,$ & $\,\,$\color{gr} 0.1530\color{black} $\,\,$ & $\,\,$ 1 $\,\,$ & $\,\,$0.2758 $\,\,$ \\
$\,\,$0.2775$\,\,$ & $\,\,$\color{gr} 0.5550\color{black} $\,\,$ & $\,\,$3.6264$\,\,$ & $\,\,$ 1  $\,\,$ \\
\end{pmatrix},
\end{equation*}
\end{example}
\newpage
\begin{example}
\begin{equation*}
\mathbf{A} =
\begin{pmatrix}
$\,\,$ 1 $\,\,$ & $\,\,$2$\,\,$ & $\,\,$9$\,\,$ & $\,\,$7 $\,\,$ \\
$\,\,$ 1/2$\,\,$ & $\,\,$ 1 $\,\,$ & $\,\,$7$\,\,$ & $\,\,$2 $\,\,$ \\
$\,\,$ 1/9$\,\,$ & $\,\,$ 1/7$\,\,$ & $\,\,$ 1 $\,\,$ & $\,\,$ 1/7 $\,\,$ \\
$\,\,$ 1/7$\,\,$ & $\,\,$ 1/2$\,\,$ & $\,\,$7$\,\,$ & $\,\,$ 1  $\,\,$ \\
\end{pmatrix},
\qquad
\lambda_{\max} =
4.2526,
\qquad
CR = 0.0952
\end{equation*}

\begin{equation*}
\mathbf{w}^{cos} =
\begin{pmatrix}
0.535720\\
\color{red} 0.265045\color{black} \\
0.040338\\
0.158897
\end{pmatrix}\end{equation*}
\begin{equation*}
\left[ \frac{{w}^{cos}_i}{{w}^{cos}_j} \right] =
\begin{pmatrix}
$\,\,$ 1 $\,\,$ & $\,\,$\color{red} 2.0212\color{black} $\,\,$ & $\,\,$13.2807$\,\,$ & $\,\,$3.3715$\,\,$ \\
$\,\,$\color{red} 0.4947\color{black} $\,\,$ & $\,\,$ 1 $\,\,$ & $\,\,$\color{red} 6.5706\color{black} $\,\,$ & $\,\,$\color{red} 1.6680\color{black}   $\,\,$ \\
$\,\,$0.0753$\,\,$ & $\,\,$\color{red} 0.1522\color{black} $\,\,$ & $\,\,$ 1 $\,\,$ & $\,\,$0.2539 $\,\,$ \\
$\,\,$0.2966$\,\,$ & $\,\,$\color{red} 0.5995\color{black} $\,\,$ & $\,\,$3.9391$\,\,$ & $\,\,$ 1  $\,\,$ \\
\end{pmatrix},
\end{equation*}

\begin{equation*}
\mathbf{w}^{\prime} =
\begin{pmatrix}
0.534216\\
0.267108\\
0.040225\\
0.158451
\end{pmatrix} =
0.997193\cdot
\begin{pmatrix}
0.535720\\
\color{gr} 0.267860\color{black} \\
0.040338\\
0.158897
\end{pmatrix},
\end{equation*}
\begin{equation*}
\left[ \frac{{w}^{\prime}_i}{{w}^{\prime}_j} \right] =
\begin{pmatrix}
$\,\,$ 1 $\,\,$ & $\,\,$\color{gr} \color{blue} 2\color{black} $\,\,$ & $\,\,$13.2807$\,\,$ & $\,\,$3.3715$\,\,$ \\
$\,\,$\color{gr} \color{blue}  1/2\color{black} $\,\,$ & $\,\,$ 1 $\,\,$ & $\,\,$\color{gr} 6.6404\color{black} $\,\,$ & $\,\,$\color{gr} 1.6857\color{black}   $\,\,$ \\
$\,\,$0.0753$\,\,$ & $\,\,$\color{gr} 0.1506\color{black} $\,\,$ & $\,\,$ 1 $\,\,$ & $\,\,$0.2539 $\,\,$ \\
$\,\,$0.2966$\,\,$ & $\,\,$\color{gr} 0.5932\color{black} $\,\,$ & $\,\,$3.9391$\,\,$ & $\,\,$ 1  $\,\,$ \\
\end{pmatrix},
\end{equation*}
\end{example}
\newpage
\begin{example}
\begin{equation*}
\mathbf{A} =
\begin{pmatrix}
$\,\,$ 1 $\,\,$ & $\,\,$2$\,\,$ & $\,\,$9$\,\,$ & $\,\,$8 $\,\,$ \\
$\,\,$ 1/2$\,\,$ & $\,\,$ 1 $\,\,$ & $\,\,$3$\,\,$ & $\,\,$6 $\,\,$ \\
$\,\,$ 1/9$\,\,$ & $\,\,$ 1/3$\,\,$ & $\,\,$ 1 $\,\,$ & $\,\,$ 1/2 $\,\,$ \\
$\,\,$ 1/8$\,\,$ & $\,\,$ 1/6$\,\,$ & $\,\,$2$\,\,$ & $\,\,$ 1  $\,\,$ \\
\end{pmatrix},
\qquad
\lambda_{\max} =
4.1664,
\qquad
CR = 0.0627
\end{equation*}

\begin{equation*}
\mathbf{w}^{cos} =
\begin{pmatrix}
\color{red} 0.565877\color{black} \\
0.290223\\
0.064558\\
0.079342
\end{pmatrix}\end{equation*}
\begin{equation*}
\left[ \frac{{w}^{cos}_i}{{w}^{cos}_j} \right] =
\begin{pmatrix}
$\,\,$ 1 $\,\,$ & $\,\,$\color{red} 1.9498\color{black} $\,\,$ & $\,\,$\color{red} 8.7654\color{black} $\,\,$ & $\,\,$\color{red} 7.1321\color{black} $\,\,$ \\
$\,\,$\color{red} 0.5129\color{black} $\,\,$ & $\,\,$ 1 $\,\,$ & $\,\,$4.4955$\,\,$ & $\,\,$3.6579  $\,\,$ \\
$\,\,$\color{red} 0.1141\color{black} $\,\,$ & $\,\,$0.2224$\,\,$ & $\,\,$ 1 $\,\,$ & $\,\,$0.8137 $\,\,$ \\
$\,\,$\color{red} 0.1402\color{black} $\,\,$ & $\,\,$0.2734$\,\,$ & $\,\,$1.2290$\,\,$ & $\,\,$ 1  $\,\,$ \\
\end{pmatrix},
\end{equation*}

\begin{equation*}
\mathbf{w}^{\prime} =
\begin{pmatrix}
0.572111\\
0.286055\\
0.063631\\
0.078203
\end{pmatrix} =
0.985640\cdot
\begin{pmatrix}
\color{gr} 0.580446\color{black} \\
0.290223\\
0.064558\\
0.079342
\end{pmatrix},
\end{equation*}
\begin{equation*}
\left[ \frac{{w}^{\prime}_i}{{w}^{\prime}_j} \right] =
\begin{pmatrix}
$\,\,$ 1 $\,\,$ & $\,\,$\color{gr} \color{blue} 2\color{black} $\,\,$ & $\,\,$\color{gr} 8.9910\color{black} $\,\,$ & $\,\,$\color{gr} 7.3158\color{black} $\,\,$ \\
$\,\,$\color{gr} \color{blue}  1/2\color{black} $\,\,$ & $\,\,$ 1 $\,\,$ & $\,\,$4.4955$\,\,$ & $\,\,$3.6579  $\,\,$ \\
$\,\,$\color{gr} 0.1112\color{black} $\,\,$ & $\,\,$0.2224$\,\,$ & $\,\,$ 1 $\,\,$ & $\,\,$0.8137 $\,\,$ \\
$\,\,$\color{gr} 0.1367\color{black} $\,\,$ & $\,\,$0.2734$\,\,$ & $\,\,$1.2290$\,\,$ & $\,\,$ 1  $\,\,$ \\
\end{pmatrix},
\end{equation*}
\end{example}
\newpage
\begin{example}
\begin{equation*}
\mathbf{A} =
\begin{pmatrix}
$\,\,$ 1 $\,\,$ & $\,\,$2$\,\,$ & $\,\,$9$\,\,$ & $\,\,$8 $\,\,$ \\
$\,\,$ 1/2$\,\,$ & $\,\,$ 1 $\,\,$ & $\,\,$3$\,\,$ & $\,\,$7 $\,\,$ \\
$\,\,$ 1/9$\,\,$ & $\,\,$ 1/3$\,\,$ & $\,\,$ 1 $\,\,$ & $\,\,$ 1/2 $\,\,$ \\
$\,\,$ 1/8$\,\,$ & $\,\,$ 1/7$\,\,$ & $\,\,$2$\,\,$ & $\,\,$ 1  $\,\,$ \\
\end{pmatrix},
\qquad
\lambda_{\max} =
4.2065,
\qquad
CR = 0.0779
\end{equation*}

\begin{equation*}
\mathbf{w}^{cos} =
\begin{pmatrix}
\color{red} 0.558970\color{black} \\
0.300089\\
0.064166\\
0.076775
\end{pmatrix}\end{equation*}
\begin{equation*}
\left[ \frac{{w}^{cos}_i}{{w}^{cos}_j} \right] =
\begin{pmatrix}
$\,\,$ 1 $\,\,$ & $\,\,$\color{red} 1.8627\color{black} $\,\,$ & $\,\,$\color{red} 8.7113\color{black} $\,\,$ & $\,\,$\color{red} 7.2807\color{black} $\,\,$ \\
$\,\,$\color{red} 0.5369\color{black} $\,\,$ & $\,\,$ 1 $\,\,$ & $\,\,$4.6767$\,\,$ & $\,\,$3.9087  $\,\,$ \\
$\,\,$\color{red} 0.1148\color{black} $\,\,$ & $\,\,$0.2138$\,\,$ & $\,\,$ 1 $\,\,$ & $\,\,$0.8358 $\,\,$ \\
$\,\,$\color{red} 0.1373\color{black} $\,\,$ & $\,\,$0.2558$\,\,$ & $\,\,$1.1965$\,\,$ & $\,\,$ 1  $\,\,$ \\
\end{pmatrix},
\end{equation*}

\begin{equation*}
\mathbf{w}^{\prime} =
\begin{pmatrix}
0.566993\\
0.294630\\
0.062999\\
0.075378
\end{pmatrix} =
0.981809\cdot
\begin{pmatrix}
\color{gr} 0.577498\color{black} \\
0.300089\\
0.064166\\
0.076775
\end{pmatrix},
\end{equation*}
\begin{equation*}
\left[ \frac{{w}^{\prime}_i}{{w}^{\prime}_j} \right] =
\begin{pmatrix}
$\,\,$ 1 $\,\,$ & $\,\,$\color{gr} 1.9244\color{black} $\,\,$ & $\,\,$\color{gr} \color{blue} 9\color{black} $\,\,$ & $\,\,$\color{gr} 7.5220\color{black} $\,\,$ \\
$\,\,$\color{gr} 0.5196\color{black} $\,\,$ & $\,\,$ 1 $\,\,$ & $\,\,$4.6767$\,\,$ & $\,\,$3.9087  $\,\,$ \\
$\,\,$\color{gr} \color{blue}  1/9\color{black} $\,\,$ & $\,\,$0.2138$\,\,$ & $\,\,$ 1 $\,\,$ & $\,\,$0.8358 $\,\,$ \\
$\,\,$\color{gr} 0.1329\color{black} $\,\,$ & $\,\,$0.2558$\,\,$ & $\,\,$1.1965$\,\,$ & $\,\,$ 1  $\,\,$ \\
\end{pmatrix},
\end{equation*}
\end{example}
\newpage
\begin{example}
\begin{equation*}
\mathbf{A} =
\begin{pmatrix}
$\,\,$ 1 $\,\,$ & $\,\,$2$\,\,$ & $\,\,$9$\,\,$ & $\,\,$8 $\,\,$ \\
$\,\,$ 1/2$\,\,$ & $\,\,$ 1 $\,\,$ & $\,\,$3$\,\,$ & $\,\,$8 $\,\,$ \\
$\,\,$ 1/9$\,\,$ & $\,\,$ 1/3$\,\,$ & $\,\,$ 1 $\,\,$ & $\,\,$ 1/2 $\,\,$ \\
$\,\,$ 1/8$\,\,$ & $\,\,$ 1/8$\,\,$ & $\,\,$2$\,\,$ & $\,\,$ 1  $\,\,$ \\
\end{pmatrix},
\qquad
\lambda_{\max} =
4.2469,
\qquad
CR = 0.0931
\end{equation*}

\begin{equation*}
\mathbf{w}^{cos} =
\begin{pmatrix}
\color{red} 0.552799\color{black} \\
0.308630\\
0.063843\\
0.074728
\end{pmatrix}\end{equation*}
\begin{equation*}
\left[ \frac{{w}^{cos}_i}{{w}^{cos}_j} \right] =
\begin{pmatrix}
$\,\,$ 1 $\,\,$ & $\,\,$\color{red} 1.7911\color{black} $\,\,$ & $\,\,$\color{red} 8.6588\color{black} $\,\,$ & $\,\,$\color{red} 7.3974\color{black} $\,\,$ \\
$\,\,$\color{red} 0.5583\color{black} $\,\,$ & $\,\,$ 1 $\,\,$ & $\,\,$4.8342$\,\,$ & $\,\,$4.1300  $\,\,$ \\
$\,\,$\color{red} 0.1155\color{black} $\,\,$ & $\,\,$0.2069$\,\,$ & $\,\,$ 1 $\,\,$ & $\,\,$0.8543 $\,\,$ \\
$\,\,$\color{red} 0.1352\color{black} $\,\,$ & $\,\,$0.2421$\,\,$ & $\,\,$1.1705$\,\,$ & $\,\,$ 1  $\,\,$ \\
\end{pmatrix},
\end{equation*}

\begin{equation*}
\mathbf{w}^{\prime} =
\begin{pmatrix}
0.562333\\
0.302050\\
0.062481\\
0.073135
\end{pmatrix} =
0.978679\cdot
\begin{pmatrix}
\color{gr} 0.574584\color{black} \\
0.308630\\
0.063843\\
0.074728
\end{pmatrix},
\end{equation*}
\begin{equation*}
\left[ \frac{{w}^{\prime}_i}{{w}^{\prime}_j} \right] =
\begin{pmatrix}
$\,\,$ 1 $\,\,$ & $\,\,$\color{gr} 1.8617\color{black} $\,\,$ & $\,\,$\color{gr} \color{blue} 9\color{black} $\,\,$ & $\,\,$\color{gr} 7.6890\color{black} $\,\,$ \\
$\,\,$\color{gr} 0.5371\color{black} $\,\,$ & $\,\,$ 1 $\,\,$ & $\,\,$4.8342$\,\,$ & $\,\,$4.1300  $\,\,$ \\
$\,\,$\color{gr} \color{blue}  1/9\color{black} $\,\,$ & $\,\,$0.2069$\,\,$ & $\,\,$ 1 $\,\,$ & $\,\,$0.8543 $\,\,$ \\
$\,\,$\color{gr} 0.1301\color{black} $\,\,$ & $\,\,$0.2421$\,\,$ & $\,\,$1.1705$\,\,$ & $\,\,$ 1  $\,\,$ \\
\end{pmatrix},
\end{equation*}
\end{example}
\newpage
\begin{example}
\begin{equation*}
\mathbf{A} =
\begin{pmatrix}
$\,\,$ 1 $\,\,$ & $\,\,$2$\,\,$ & $\,\,$9$\,\,$ & $\,\,$9 $\,\,$ \\
$\,\,$ 1/2$\,\,$ & $\,\,$ 1 $\,\,$ & $\,\,$6$\,\,$ & $\,\,$3 $\,\,$ \\
$\,\,$ 1/9$\,\,$ & $\,\,$ 1/6$\,\,$ & $\,\,$ 1 $\,\,$ & $\,\,$ 1/3 $\,\,$ \\
$\,\,$ 1/9$\,\,$ & $\,\,$ 1/3$\,\,$ & $\,\,$3$\,\,$ & $\,\,$ 1  $\,\,$ \\
\end{pmatrix},
\qquad
\lambda_{\max} =
4.1031,
\qquad
CR = 0.0389
\end{equation*}

\begin{equation*}
\mathbf{w}^{cos} =
\begin{pmatrix}
0.570443\\
\color{red} 0.281314\color{black} \\
0.048047\\
0.100196
\end{pmatrix}\end{equation*}
\begin{equation*}
\left[ \frac{{w}^{cos}_i}{{w}^{cos}_j} \right] =
\begin{pmatrix}
$\,\,$ 1 $\,\,$ & $\,\,$\color{red} 2.0278\color{black} $\,\,$ & $\,\,$11.8727$\,\,$ & $\,\,$5.6933$\,\,$ \\
$\,\,$\color{red} 0.4931\color{black} $\,\,$ & $\,\,$ 1 $\,\,$ & $\,\,$\color{red} 5.8550\color{black} $\,\,$ & $\,\,$\color{red} 2.8076\color{black}   $\,\,$ \\
$\,\,$0.0842$\,\,$ & $\,\,$\color{red} 0.1708\color{black} $\,\,$ & $\,\,$ 1 $\,\,$ & $\,\,$0.4795 $\,\,$ \\
$\,\,$0.1756$\,\,$ & $\,\,$\color{red} 0.3562\color{black} $\,\,$ & $\,\,$2.0854$\,\,$ & $\,\,$ 1  $\,\,$ \\
\end{pmatrix},
\end{equation*}

\begin{equation*}
\mathbf{w}^{\prime} =
\begin{pmatrix}
0.568223\\
0.284111\\
0.047860\\
0.099806
\end{pmatrix} =
0.996107\cdot
\begin{pmatrix}
0.570443\\
\color{gr} 0.285222\color{black} \\
0.048047\\
0.100196
\end{pmatrix},
\end{equation*}
\begin{equation*}
\left[ \frac{{w}^{\prime}_i}{{w}^{\prime}_j} \right] =
\begin{pmatrix}
$\,\,$ 1 $\,\,$ & $\,\,$\color{gr} \color{blue} 2\color{black} $\,\,$ & $\,\,$11.8727$\,\,$ & $\,\,$5.6933$\,\,$ \\
$\,\,$\color{gr} \color{blue}  1/2\color{black} $\,\,$ & $\,\,$ 1 $\,\,$ & $\,\,$\color{gr} 5.9364\color{black} $\,\,$ & $\,\,$\color{gr} 2.8466\color{black}   $\,\,$ \\
$\,\,$0.0842$\,\,$ & $\,\,$\color{gr} 0.1685\color{black} $\,\,$ & $\,\,$ 1 $\,\,$ & $\,\,$0.4795 $\,\,$ \\
$\,\,$0.1756$\,\,$ & $\,\,$\color{gr} 0.3513\color{black} $\,\,$ & $\,\,$2.0854$\,\,$ & $\,\,$ 1  $\,\,$ \\
\end{pmatrix},
\end{equation*}
\end{example}
\newpage
\begin{example}
\begin{equation*}
\mathbf{A} =
\begin{pmatrix}
$\,\,$ 1 $\,\,$ & $\,\,$3$\,\,$ & $\,\,$2$\,\,$ & $\,\,$5 $\,\,$ \\
$\,\,$ 1/3$\,\,$ & $\,\,$ 1 $\,\,$ & $\,\,$1$\,\,$ & $\,\,$1 $\,\,$ \\
$\,\,$ 1/2$\,\,$ & $\,\,$ 1 $\,\,$ & $\,\,$ 1 $\,\,$ & $\,\,$ 1/2 $\,\,$ \\
$\,\,$ 1/5$\,\,$ & $\,\,$ 1 $\,\,$ & $\,\,$2$\,\,$ & $\,\,$ 1  $\,\,$ \\
\end{pmatrix},
\qquad
\lambda_{\max} =
4.2277,
\qquad
CR = 0.0859
\end{equation*}

\begin{equation*}
\mathbf{w}^{cos} =
\begin{pmatrix}
0.486608\\
\color{red} 0.158915\color{black} \\
0.165552\\
0.188925
\end{pmatrix}\end{equation*}
\begin{equation*}
\left[ \frac{{w}^{cos}_i}{{w}^{cos}_j} \right] =
\begin{pmatrix}
$\,\,$ 1 $\,\,$ & $\,\,$\color{red} 3.0621\color{black} $\,\,$ & $\,\,$2.9393$\,\,$ & $\,\,$2.5757$\,\,$ \\
$\,\,$\color{red} 0.3266\color{black} $\,\,$ & $\,\,$ 1 $\,\,$ & $\,\,$\color{red} 0.9599\color{black} $\,\,$ & $\,\,$\color{red} 0.8412\color{black}   $\,\,$ \\
$\,\,$0.3402$\,\,$ & $\,\,$\color{red} 1.0418\color{black} $\,\,$ & $\,\,$ 1 $\,\,$ & $\,\,$0.8763 $\,\,$ \\
$\,\,$0.3882$\,\,$ & $\,\,$\color{red} 1.1888\color{black} $\,\,$ & $\,\,$1.1412$\,\,$ & $\,\,$ 1  $\,\,$ \\
\end{pmatrix},
\end{equation*}

\begin{equation*}
\mathbf{w}^{\prime} =
\begin{pmatrix}
0.485013\\
0.161671\\
0.165009\\
0.188306
\end{pmatrix} =
0.996723\cdot
\begin{pmatrix}
0.486608\\
\color{gr} 0.162203\color{black} \\
0.165552\\
0.188925
\end{pmatrix},
\end{equation*}
\begin{equation*}
\left[ \frac{{w}^{\prime}_i}{{w}^{\prime}_j} \right] =
\begin{pmatrix}
$\,\,$ 1 $\,\,$ & $\,\,$\color{gr} \color{blue} 3\color{black} $\,\,$ & $\,\,$2.9393$\,\,$ & $\,\,$2.5757$\,\,$ \\
$\,\,$\color{gr} \color{blue}  1/3\color{black} $\,\,$ & $\,\,$ 1 $\,\,$ & $\,\,$\color{gr} 0.9798\color{black} $\,\,$ & $\,\,$\color{gr} 0.8586\color{black}   $\,\,$ \\
$\,\,$0.3402$\,\,$ & $\,\,$\color{gr} 1.0206\color{black} $\,\,$ & $\,\,$ 1 $\,\,$ & $\,\,$0.8763 $\,\,$ \\
$\,\,$0.3882$\,\,$ & $\,\,$\color{gr} 1.1647\color{black} $\,\,$ & $\,\,$1.1412$\,\,$ & $\,\,$ 1  $\,\,$ \\
\end{pmatrix},
\end{equation*}
\end{example}
\newpage
\begin{example}
\begin{equation*}
\mathbf{A} =
\begin{pmatrix}
$\,\,$ 1 $\,\,$ & $\,\,$3$\,\,$ & $\,\,$4$\,\,$ & $\,\,$1 $\,\,$ \\
$\,\,$ 1/3$\,\,$ & $\,\,$ 1 $\,\,$ & $\,\,$2$\,\,$ & $\,\,$1 $\,\,$ \\
$\,\,$ 1/4$\,\,$ & $\,\,$ 1/2$\,\,$ & $\,\,$ 1 $\,\,$ & $\,\,$ 1/3 $\,\,$ \\
$\,\,$ 1 $\,\,$ & $\,\,$ 1 $\,\,$ & $\,\,$3$\,\,$ & $\,\,$ 1  $\,\,$ \\
\end{pmatrix},
\qquad
\lambda_{\max} =
4.1031,
\qquad
CR = 0.0389
\end{equation*}

\begin{equation*}
\mathbf{w}^{cos} =
\begin{pmatrix}
0.403788\\
0.204899\\
\color{red} 0.097107\color{black} \\
0.294206
\end{pmatrix}\end{equation*}
\begin{equation*}
\left[ \frac{{w}^{cos}_i}{{w}^{cos}_j} \right] =
\begin{pmatrix}
$\,\,$ 1 $\,\,$ & $\,\,$1.9707$\,\,$ & $\,\,$\color{red} 4.1582\color{black} $\,\,$ & $\,\,$1.3725$\,\,$ \\
$\,\,$0.5074$\,\,$ & $\,\,$ 1 $\,\,$ & $\,\,$\color{red} 2.1100\color{black} $\,\,$ & $\,\,$0.6964  $\,\,$ \\
$\,\,$\color{red} 0.2405\color{black} $\,\,$ & $\,\,$\color{red} 0.4739\color{black} $\,\,$ & $\,\,$ 1 $\,\,$ & $\,\,$\color{red} 0.3301\color{black}  $\,\,$ \\
$\,\,$0.7286$\,\,$ & $\,\,$1.4359$\,\,$ & $\,\,$\color{red} 3.0297\color{black} $\,\,$ & $\,\,$ 1  $\,\,$ \\
\end{pmatrix},
\end{equation*}

\begin{equation*}
\mathbf{w}^{\prime} =
\begin{pmatrix}
0.403400\\
0.204702\\
0.097975\\
0.293924
\end{pmatrix} =
0.999040\cdot
\begin{pmatrix}
0.403788\\
0.204899\\
\color{gr} 0.098069\color{black} \\
0.294206
\end{pmatrix},
\end{equation*}
\begin{equation*}
\left[ \frac{{w}^{\prime}_i}{{w}^{\prime}_j} \right] =
\begin{pmatrix}
$\,\,$ 1 $\,\,$ & $\,\,$1.9707$\,\,$ & $\,\,$\color{gr} 4.1174\color{black} $\,\,$ & $\,\,$1.3725$\,\,$ \\
$\,\,$0.5074$\,\,$ & $\,\,$ 1 $\,\,$ & $\,\,$\color{gr} 2.0893\color{black} $\,\,$ & $\,\,$0.6964  $\,\,$ \\
$\,\,$\color{gr} 0.2429\color{black} $\,\,$ & $\,\,$\color{gr} 0.4786\color{black} $\,\,$ & $\,\,$ 1 $\,\,$ & $\,\,$\color{gr} \color{blue}  1/3\color{black}  $\,\,$ \\
$\,\,$0.7286$\,\,$ & $\,\,$1.4359$\,\,$ & $\,\,$\color{gr} \color{blue} 3\color{black} $\,\,$ & $\,\,$ 1  $\,\,$ \\
\end{pmatrix},
\end{equation*}
\end{example}
\newpage
\begin{example}
\begin{equation*}
\mathbf{A} =
\begin{pmatrix}
$\,\,$ 1 $\,\,$ & $\,\,$3$\,\,$ & $\,\,$4$\,\,$ & $\,\,$2 $\,\,$ \\
$\,\,$ 1/3$\,\,$ & $\,\,$ 1 $\,\,$ & $\,\,$1$\,\,$ & $\,\,$1 $\,\,$ \\
$\,\,$ 1/4$\,\,$ & $\,\,$ 1 $\,\,$ & $\,\,$ 1 $\,\,$ & $\,\,$ 1/3 $\,\,$ \\
$\,\,$ 1/2$\,\,$ & $\,\,$ 1 $\,\,$ & $\,\,$3$\,\,$ & $\,\,$ 1  $\,\,$ \\
\end{pmatrix},
\qquad
\lambda_{\max} =
4.1031,
\qquad
CR = 0.0389
\end{equation*}

\begin{equation*}
\mathbf{w}^{cos} =
\begin{pmatrix}
\color{red} 0.471483\color{black} \\
0.167313\\
0.118553\\
0.242651
\end{pmatrix}\end{equation*}
\begin{equation*}
\left[ \frac{{w}^{cos}_i}{{w}^{cos}_j} \right] =
\begin{pmatrix}
$\,\,$ 1 $\,\,$ & $\,\,$\color{red} 2.8180\color{black} $\,\,$ & $\,\,$\color{red} 3.9770\color{black} $\,\,$ & $\,\,$\color{red} 1.9430\color{black} $\,\,$ \\
$\,\,$\color{red} 0.3549\color{black} $\,\,$ & $\,\,$ 1 $\,\,$ & $\,\,$1.4113$\,\,$ & $\,\,$0.6895  $\,\,$ \\
$\,\,$\color{red} 0.2514\color{black} $\,\,$ & $\,\,$0.7086$\,\,$ & $\,\,$ 1 $\,\,$ & $\,\,$0.4886 $\,\,$ \\
$\,\,$\color{red} 0.5147\color{black} $\,\,$ & $\,\,$1.4503$\,\,$ & $\,\,$2.0468$\,\,$ & $\,\,$ 1  $\,\,$ \\
\end{pmatrix},
\end{equation*}

\begin{equation*}
\mathbf{w}^{\prime} =
\begin{pmatrix}
0.472921\\
0.166858\\
0.118230\\
0.241991
\end{pmatrix} =
0.997279\cdot
\begin{pmatrix}
\color{gr} 0.474211\color{black} \\
0.167313\\
0.118553\\
0.242651
\end{pmatrix},
\end{equation*}
\begin{equation*}
\left[ \frac{{w}^{\prime}_i}{{w}^{\prime}_j} \right] =
\begin{pmatrix}
$\,\,$ 1 $\,\,$ & $\,\,$\color{gr} 2.8343\color{black} $\,\,$ & $\,\,$\color{gr} \color{blue} 4\color{black} $\,\,$ & $\,\,$\color{gr} 1.9543\color{black} $\,\,$ \\
$\,\,$\color{gr} 0.3528\color{black} $\,\,$ & $\,\,$ 1 $\,\,$ & $\,\,$1.4113$\,\,$ & $\,\,$0.6895  $\,\,$ \\
$\,\,$\color{gr} \color{blue}  1/4\color{black} $\,\,$ & $\,\,$0.7086$\,\,$ & $\,\,$ 1 $\,\,$ & $\,\,$0.4886 $\,\,$ \\
$\,\,$\color{gr} 0.5117\color{black} $\,\,$ & $\,\,$1.4503$\,\,$ & $\,\,$2.0468$\,\,$ & $\,\,$ 1  $\,\,$ \\
\end{pmatrix},
\end{equation*}
\end{example}
\newpage
\begin{example}
\begin{equation*}
\mathbf{A} =
\begin{pmatrix}
$\,\,$ 1 $\,\,$ & $\,\,$3$\,\,$ & $\,\,$5$\,\,$ & $\,\,$2 $\,\,$ \\
$\,\,$ 1/3$\,\,$ & $\,\,$ 1 $\,\,$ & $\,\,$1$\,\,$ & $\,\,$1 $\,\,$ \\
$\,\,$ 1/5$\,\,$ & $\,\,$ 1 $\,\,$ & $\,\,$ 1 $\,\,$ & $\,\,$ 1/4 $\,\,$ \\
$\,\,$ 1/2$\,\,$ & $\,\,$ 1 $\,\,$ & $\,\,$4$\,\,$ & $\,\,$ 1  $\,\,$ \\
\end{pmatrix},
\qquad
\lambda_{\max} =
4.1655,
\qquad
CR = 0.0624
\end{equation*}

\begin{equation*}
\mathbf{w}^{cos} =
\begin{pmatrix}
\color{red} 0.479412\color{black} \\
0.164723\\
0.103842\\
0.252023
\end{pmatrix}\end{equation*}
\begin{equation*}
\left[ \frac{{w}^{cos}_i}{{w}^{cos}_j} \right] =
\begin{pmatrix}
$\,\,$ 1 $\,\,$ & $\,\,$\color{red} 2.9104\color{black} $\,\,$ & $\,\,$\color{red} 4.6168\color{black} $\,\,$ & $\,\,$\color{red} 1.9023\color{black} $\,\,$ \\
$\,\,$\color{red} 0.3436\color{black} $\,\,$ & $\,\,$ 1 $\,\,$ & $\,\,$1.5863$\,\,$ & $\,\,$0.6536  $\,\,$ \\
$\,\,$\color{red} 0.2166\color{black} $\,\,$ & $\,\,$0.6304$\,\,$ & $\,\,$ 1 $\,\,$ & $\,\,$0.4120 $\,\,$ \\
$\,\,$\color{red} 0.5257\color{black} $\,\,$ & $\,\,$1.5300$\,\,$ & $\,\,$2.4270$\,\,$ & $\,\,$ 1  $\,\,$ \\
\end{pmatrix},
\end{equation*}

\begin{equation*}
\mathbf{w}^{\prime} =
\begin{pmatrix}
0.486983\\
0.162328\\
0.102332\\
0.248357
\end{pmatrix} =
0.985457\cdot
\begin{pmatrix}
\color{gr} 0.494170\color{black} \\
0.164723\\
0.103842\\
0.252023
\end{pmatrix},
\end{equation*}
\begin{equation*}
\left[ \frac{{w}^{\prime}_i}{{w}^{\prime}_j} \right] =
\begin{pmatrix}
$\,\,$ 1 $\,\,$ & $\,\,$\color{gr} \color{blue} 3\color{black} $\,\,$ & $\,\,$\color{gr} 4.7589\color{black} $\,\,$ & $\,\,$\color{gr} 1.9608\color{black} $\,\,$ \\
$\,\,$\color{gr} \color{blue}  1/3\color{black} $\,\,$ & $\,\,$ 1 $\,\,$ & $\,\,$1.5863$\,\,$ & $\,\,$0.6536  $\,\,$ \\
$\,\,$\color{gr} 0.2101\color{black} $\,\,$ & $\,\,$0.6304$\,\,$ & $\,\,$ 1 $\,\,$ & $\,\,$0.4120 $\,\,$ \\
$\,\,$\color{gr} 0.5100\color{black} $\,\,$ & $\,\,$1.5300$\,\,$ & $\,\,$2.4270$\,\,$ & $\,\,$ 1  $\,\,$ \\
\end{pmatrix},
\end{equation*}
\end{example}
\newpage
\begin{example}
\begin{equation*}
\mathbf{A} =
\begin{pmatrix}
$\,\,$ 1 $\,\,$ & $\,\,$3$\,\,$ & $\,\,$5$\,\,$ & $\,\,$2 $\,\,$ \\
$\,\,$ 1/3$\,\,$ & $\,\,$ 1 $\,\,$ & $\,\,$1$\,\,$ & $\,\,$1 $\,\,$ \\
$\,\,$ 1/5$\,\,$ & $\,\,$ 1 $\,\,$ & $\,\,$ 1 $\,\,$ & $\,\,$ 1/5 $\,\,$ \\
$\,\,$ 1/2$\,\,$ & $\,\,$ 1 $\,\,$ & $\,\,$5$\,\,$ & $\,\,$ 1  $\,\,$ \\
\end{pmatrix},
\qquad
\lambda_{\max} =
4.2277,
\qquad
CR = 0.0859
\end{equation*}

\begin{equation*}
\mathbf{w}^{cos} =
\begin{pmatrix}
\color{red} 0.471648\color{black} \\
0.163545\\
0.099326\\
0.265481
\end{pmatrix}\end{equation*}
\begin{equation*}
\left[ \frac{{w}^{cos}_i}{{w}^{cos}_j} \right] =
\begin{pmatrix}
$\,\,$ 1 $\,\,$ & $\,\,$\color{red} 2.8839\color{black} $\,\,$ & $\,\,$\color{red} 4.7485\color{black} $\,\,$ & $\,\,$\color{red} 1.7766\color{black} $\,\,$ \\
$\,\,$\color{red} 0.3468\color{black} $\,\,$ & $\,\,$ 1 $\,\,$ & $\,\,$1.6465$\,\,$ & $\,\,$0.6160  $\,\,$ \\
$\,\,$\color{red} 0.2106\color{black} $\,\,$ & $\,\,$0.6073$\,\,$ & $\,\,$ 1 $\,\,$ & $\,\,$0.3741 $\,\,$ \\
$\,\,$\color{red} 0.5629\color{black} $\,\,$ & $\,\,$1.6233$\,\,$ & $\,\,$2.6728$\,\,$ & $\,\,$ 1  $\,\,$ \\
\end{pmatrix},
\end{equation*}

\begin{equation*}
\mathbf{w}^{\prime} =
\begin{pmatrix}
0.481492\\
0.160497\\
0.097475\\
0.260535
\end{pmatrix} =
0.981367\cdot
\begin{pmatrix}
\color{gr} 0.490634\color{black} \\
0.163545\\
0.099326\\
0.265481
\end{pmatrix},
\end{equation*}
\begin{equation*}
\left[ \frac{{w}^{\prime}_i}{{w}^{\prime}_j} \right] =
\begin{pmatrix}
$\,\,$ 1 $\,\,$ & $\,\,$\color{gr} \color{blue} 3\color{black} $\,\,$ & $\,\,$\color{gr} 4.9396\color{black} $\,\,$ & $\,\,$\color{gr} 1.8481\color{black} $\,\,$ \\
$\,\,$\color{gr} \color{blue}  1/3\color{black} $\,\,$ & $\,\,$ 1 $\,\,$ & $\,\,$1.6465$\,\,$ & $\,\,$0.6160  $\,\,$ \\
$\,\,$\color{gr} 0.2024\color{black} $\,\,$ & $\,\,$0.6073$\,\,$ & $\,\,$ 1 $\,\,$ & $\,\,$0.3741 $\,\,$ \\
$\,\,$\color{gr} 0.5411\color{black} $\,\,$ & $\,\,$1.6233$\,\,$ & $\,\,$2.6728$\,\,$ & $\,\,$ 1  $\,\,$ \\
\end{pmatrix},
\end{equation*}
\end{example}
\newpage
\begin{example}
\begin{equation*}
\mathbf{A} =
\begin{pmatrix}
$\,\,$ 1 $\,\,$ & $\,\,$3$\,\,$ & $\,\,$5$\,\,$ & $\,\,$4 $\,\,$ \\
$\,\,$ 1/3$\,\,$ & $\,\,$ 1 $\,\,$ & $\,\,$1$\,\,$ & $\,\,$2 $\,\,$ \\
$\,\,$ 1/5$\,\,$ & $\,\,$ 1 $\,\,$ & $\,\,$ 1 $\,\,$ & $\,\,$ 1/2 $\,\,$ \\
$\,\,$ 1/4$\,\,$ & $\,\,$ 1/2$\,\,$ & $\,\,$2$\,\,$ & $\,\,$ 1  $\,\,$ \\
\end{pmatrix},
\qquad
\lambda_{\max} =
4.1655,
\qquad
CR = 0.0624
\end{equation*}

\begin{equation*}
\mathbf{w}^{cos} =
\begin{pmatrix}
\color{red} 0.548740\color{black} \\
0.186740\\
0.118081\\
0.146439
\end{pmatrix}\end{equation*}
\begin{equation*}
\left[ \frac{{w}^{cos}_i}{{w}^{cos}_j} \right] =
\begin{pmatrix}
$\,\,$ 1 $\,\,$ & $\,\,$\color{red} 2.9385\color{black} $\,\,$ & $\,\,$\color{red} 4.6472\color{black} $\,\,$ & $\,\,$\color{red} 3.7472\color{black} $\,\,$ \\
$\,\,$\color{red} 0.3403\color{black} $\,\,$ & $\,\,$ 1 $\,\,$ & $\,\,$1.5815$\,\,$ & $\,\,$1.2752  $\,\,$ \\
$\,\,$\color{red} 0.2152\color{black} $\,\,$ & $\,\,$0.6323$\,\,$ & $\,\,$ 1 $\,\,$ & $\,\,$0.8063 $\,\,$ \\
$\,\,$\color{red} 0.2669\color{black} $\,\,$ & $\,\,$0.7842$\,\,$ & $\,\,$1.2402$\,\,$ & $\,\,$ 1  $\,\,$ \\
\end{pmatrix},
\end{equation*}

\begin{equation*}
\mathbf{w}^{\prime} =
\begin{pmatrix}
0.553862\\
0.184621\\
0.116740\\
0.144777
\end{pmatrix} =
0.988649\cdot
\begin{pmatrix}
\color{gr} 0.560221\color{black} \\
0.186740\\
0.118081\\
0.146439
\end{pmatrix},
\end{equation*}
\begin{equation*}
\left[ \frac{{w}^{\prime}_i}{{w}^{\prime}_j} \right] =
\begin{pmatrix}
$\,\,$ 1 $\,\,$ & $\,\,$\color{gr} \color{blue} 3\color{black} $\,\,$ & $\,\,$\color{gr} 4.7444\color{black} $\,\,$ & $\,\,$\color{gr} 3.8256\color{black} $\,\,$ \\
$\,\,$\color{gr} \color{blue}  1/3\color{black} $\,\,$ & $\,\,$ 1 $\,\,$ & $\,\,$1.5815$\,\,$ & $\,\,$1.2752  $\,\,$ \\
$\,\,$\color{gr} 0.2108\color{black} $\,\,$ & $\,\,$0.6323$\,\,$ & $\,\,$ 1 $\,\,$ & $\,\,$0.8063 $\,\,$ \\
$\,\,$\color{gr} 0.2614\color{black} $\,\,$ & $\,\,$0.7842$\,\,$ & $\,\,$1.2402$\,\,$ & $\,\,$ 1  $\,\,$ \\
\end{pmatrix},
\end{equation*}
\end{example}
\newpage
\begin{example}
\begin{equation*}
\mathbf{A} =
\begin{pmatrix}
$\,\,$ 1 $\,\,$ & $\,\,$3$\,\,$ & $\,\,$6$\,\,$ & $\,\,$1 $\,\,$ \\
$\,\,$ 1/3$\,\,$ & $\,\,$ 1 $\,\,$ & $\,\,$3$\,\,$ & $\,\,$1 $\,\,$ \\
$\,\,$ 1/6$\,\,$ & $\,\,$ 1/3$\,\,$ & $\,\,$ 1 $\,\,$ & $\,\,$ 1/4 $\,\,$ \\
$\,\,$ 1 $\,\,$ & $\,\,$ 1 $\,\,$ & $\,\,$4$\,\,$ & $\,\,$ 1  $\,\,$ \\
\end{pmatrix},
\qquad
\lambda_{\max} =
4.1031,
\qquad
CR = 0.0389
\end{equation*}

\begin{equation*}
\mathbf{w}^{cos} =
\begin{pmatrix}
0.420069\\
0.213167\\
\color{red} 0.069669\color{black} \\
0.297095
\end{pmatrix}\end{equation*}
\begin{equation*}
\left[ \frac{{w}^{cos}_i}{{w}^{cos}_j} \right] =
\begin{pmatrix}
$\,\,$ 1 $\,\,$ & $\,\,$1.9706$\,\,$ & $\,\,$\color{red} 6.0295\color{black} $\,\,$ & $\,\,$1.4139$\,\,$ \\
$\,\,$0.5075$\,\,$ & $\,\,$ 1 $\,\,$ & $\,\,$\color{red} 3.0597\color{black} $\,\,$ & $\,\,$0.7175  $\,\,$ \\
$\,\,$\color{red} 0.1659\color{black} $\,\,$ & $\,\,$\color{red} 0.3268\color{black} $\,\,$ & $\,\,$ 1 $\,\,$ & $\,\,$\color{red} 0.2345\color{black}  $\,\,$ \\
$\,\,$0.7073$\,\,$ & $\,\,$1.3937$\,\,$ & $\,\,$\color{red} 4.2644\color{black} $\,\,$ & $\,\,$ 1  $\,\,$ \\
\end{pmatrix},
\end{equation*}

\begin{equation*}
\mathbf{w}^{\prime} =
\begin{pmatrix}
0.419925\\
0.213094\\
0.069988\\
0.296993
\end{pmatrix} =
0.999657\cdot
\begin{pmatrix}
0.420069\\
0.213167\\
\color{gr} 0.070012\color{black} \\
0.297095
\end{pmatrix},
\end{equation*}
\begin{equation*}
\left[ \frac{{w}^{\prime}_i}{{w}^{\prime}_j} \right] =
\begin{pmatrix}
$\,\,$ 1 $\,\,$ & $\,\,$1.9706$\,\,$ & $\,\,$\color{gr} \color{blue} 6\color{black} $\,\,$ & $\,\,$1.4139$\,\,$ \\
$\,\,$0.5075$\,\,$ & $\,\,$ 1 $\,\,$ & $\,\,$\color{gr} 3.0447\color{black} $\,\,$ & $\,\,$0.7175  $\,\,$ \\
$\,\,$\color{gr} \color{blue}  1/6\color{black} $\,\,$ & $\,\,$\color{gr} 0.3284\color{black} $\,\,$ & $\,\,$ 1 $\,\,$ & $\,\,$\color{gr} 0.2357\color{black}  $\,\,$ \\
$\,\,$0.7073$\,\,$ & $\,\,$1.3937$\,\,$ & $\,\,$\color{gr} 4.2435\color{black} $\,\,$ & $\,\,$ 1  $\,\,$ \\
\end{pmatrix},
\end{equation*}
\end{example}
\newpage
\begin{example}
\begin{equation*}
\mathbf{A} =
\begin{pmatrix}
$\,\,$ 1 $\,\,$ & $\,\,$3$\,\,$ & $\,\,$6$\,\,$ & $\,\,$2 $\,\,$ \\
$\,\,$ 1/3$\,\,$ & $\,\,$ 1 $\,\,$ & $\,\,$1$\,\,$ & $\,\,$1 $\,\,$ \\
$\,\,$ 1/6$\,\,$ & $\,\,$ 1 $\,\,$ & $\,\,$ 1 $\,\,$ & $\,\,$ 1/5 $\,\,$ \\
$\,\,$ 1/2$\,\,$ & $\,\,$ 1 $\,\,$ & $\,\,$5$\,\,$ & $\,\,$ 1  $\,\,$ \\
\end{pmatrix},
\qquad
\lambda_{\max} =
4.2277,
\qquad
CR = 0.0859
\end{equation*}

\begin{equation*}
\mathbf{w}^{cos} =
\begin{pmatrix}
\color{red} 0.484698\color{black} \\
0.162989\\
0.094078\\
0.258235
\end{pmatrix}\end{equation*}
\begin{equation*}
\left[ \frac{{w}^{cos}_i}{{w}^{cos}_j} \right] =
\begin{pmatrix}
$\,\,$ 1 $\,\,$ & $\,\,$\color{red} 2.9738\color{black} $\,\,$ & $\,\,$\color{red} 5.1521\color{black} $\,\,$ & $\,\,$\color{red} 1.8770\color{black} $\,\,$ \\
$\,\,$\color{red} 0.3363\color{black} $\,\,$ & $\,\,$ 1 $\,\,$ & $\,\,$1.7325$\,\,$ & $\,\,$0.6312  $\,\,$ \\
$\,\,$\color{red} 0.1941\color{black} $\,\,$ & $\,\,$0.5772$\,\,$ & $\,\,$ 1 $\,\,$ & $\,\,$0.3643 $\,\,$ \\
$\,\,$\color{red} 0.5328\color{black} $\,\,$ & $\,\,$1.5844$\,\,$ & $\,\,$2.7449$\,\,$ & $\,\,$ 1  $\,\,$ \\
\end{pmatrix},
\end{equation*}

\begin{equation*}
\mathbf{w}^{\prime} =
\begin{pmatrix}
0.486889\\
0.162296\\
0.093678\\
0.257137
\end{pmatrix} =
0.995749\cdot
\begin{pmatrix}
\color{gr} 0.488968\color{black} \\
0.162989\\
0.094078\\
0.258235
\end{pmatrix},
\end{equation*}
\begin{equation*}
\left[ \frac{{w}^{\prime}_i}{{w}^{\prime}_j} \right] =
\begin{pmatrix}
$\,\,$ 1 $\,\,$ & $\,\,$\color{gr} \color{blue} 3\color{black} $\,\,$ & $\,\,$\color{gr} 5.1975\color{black} $\,\,$ & $\,\,$\color{gr} 1.8935\color{black} $\,\,$ \\
$\,\,$\color{gr} \color{blue}  1/3\color{black} $\,\,$ & $\,\,$ 1 $\,\,$ & $\,\,$1.7325$\,\,$ & $\,\,$0.6312  $\,\,$ \\
$\,\,$\color{gr} 0.1924\color{black} $\,\,$ & $\,\,$0.5772$\,\,$ & $\,\,$ 1 $\,\,$ & $\,\,$0.3643 $\,\,$ \\
$\,\,$\color{gr} 0.5281\color{black} $\,\,$ & $\,\,$1.5844$\,\,$ & $\,\,$2.7449$\,\,$ & $\,\,$ 1  $\,\,$ \\
\end{pmatrix},
\end{equation*}
\end{example}
\newpage
\begin{example}
\begin{equation*}
\mathbf{A} =
\begin{pmatrix}
$\,\,$ 1 $\,\,$ & $\,\,$3$\,\,$ & $\,\,$6$\,\,$ & $\,\,$5 $\,\,$ \\
$\,\,$ 1/3$\,\,$ & $\,\,$ 1 $\,\,$ & $\,\,$9$\,\,$ & $\,\,$3 $\,\,$ \\
$\,\,$ 1/6$\,\,$ & $\,\,$ 1/9$\,\,$ & $\,\,$ 1 $\,\,$ & $\,\,$ 1/2 $\,\,$ \\
$\,\,$ 1/5$\,\,$ & $\,\,$ 1/3$\,\,$ & $\,\,$2$\,\,$ & $\,\,$ 1  $\,\,$ \\
\end{pmatrix},
\qquad
\lambda_{\max} =
4.1966,
\qquad
CR = 0.0741
\end{equation*}

\begin{equation*}
\mathbf{w}^{cos} =
\begin{pmatrix}
0.524725\\
0.313520\\
0.058649\\
\color{red} 0.103107\color{black}
\end{pmatrix}\end{equation*}
\begin{equation*}
\left[ \frac{{w}^{cos}_i}{{w}^{cos}_j} \right] =
\begin{pmatrix}
$\,\,$ 1 $\,\,$ & $\,\,$1.6737$\,\,$ & $\,\,$8.9468$\,\,$ & $\,\,$\color{red} 5.0891\color{black} $\,\,$ \\
$\,\,$0.5975$\,\,$ & $\,\,$ 1 $\,\,$ & $\,\,$5.3457$\,\,$ & $\,\,$\color{red} 3.0407\color{black}   $\,\,$ \\
$\,\,$0.1118$\,\,$ & $\,\,$0.1871$\,\,$ & $\,\,$ 1 $\,\,$ & $\,\,$\color{red} 0.5688\color{black}  $\,\,$ \\
$\,\,$\color{red} 0.1965\color{black} $\,\,$ & $\,\,$\color{red} 0.3289\color{black} $\,\,$ & $\,\,$\color{red} 1.7580\color{black} $\,\,$ & $\,\,$ 1  $\,\,$ \\
\end{pmatrix},
\end{equation*}

\begin{equation*}
\mathbf{w}^{\prime} =
\begin{pmatrix}
0.523991\\
0.313081\\
0.058567\\
0.104360
\end{pmatrix} =
0.998602\cdot
\begin{pmatrix}
0.524725\\
0.313520\\
0.058649\\
\color{gr} 0.104507\color{black}
\end{pmatrix},
\end{equation*}
\begin{equation*}
\left[ \frac{{w}^{\prime}_i}{{w}^{\prime}_j} \right] =
\begin{pmatrix}
$\,\,$ 1 $\,\,$ & $\,\,$1.6737$\,\,$ & $\,\,$8.9468$\,\,$ & $\,\,$\color{gr} 5.0210\color{black} $\,\,$ \\
$\,\,$0.5975$\,\,$ & $\,\,$ 1 $\,\,$ & $\,\,$5.3457$\,\,$ & $\,\,$\color{gr} \color{blue} 3\color{black}   $\,\,$ \\
$\,\,$0.1118$\,\,$ & $\,\,$0.1871$\,\,$ & $\,\,$ 1 $\,\,$ & $\,\,$\color{gr} 0.5612\color{black}  $\,\,$ \\
$\,\,$\color{gr} 0.1992\color{black} $\,\,$ & $\,\,$\color{gr} \color{blue}  1/3\color{black} $\,\,$ & $\,\,$\color{gr} 1.7819\color{black} $\,\,$ & $\,\,$ 1  $\,\,$ \\
\end{pmatrix},
\end{equation*}
\end{example}
\newpage
\begin{example}
\begin{equation*}
\mathbf{A} =
\begin{pmatrix}
$\,\,$ 1 $\,\,$ & $\,\,$3$\,\,$ & $\,\,$7$\,\,$ & $\,\,$5 $\,\,$ \\
$\,\,$ 1/3$\,\,$ & $\,\,$ 1 $\,\,$ & $\,\,$9$\,\,$ & $\,\,$3 $\,\,$ \\
$\,\,$ 1/7$\,\,$ & $\,\,$ 1/9$\,\,$ & $\,\,$ 1 $\,\,$ & $\,\,$ 1/2 $\,\,$ \\
$\,\,$ 1/5$\,\,$ & $\,\,$ 1/3$\,\,$ & $\,\,$2$\,\,$ & $\,\,$ 1  $\,\,$ \\
\end{pmatrix},
\qquad
\lambda_{\max} =
4.1583,
\qquad
CR = 0.0597
\end{equation*}

\begin{equation*}
\mathbf{w}^{cos} =
\begin{pmatrix}
0.535752\\
0.307749\\
0.054554\\
\color{red} 0.101945\color{black}
\end{pmatrix}\end{equation*}
\begin{equation*}
\left[ \frac{{w}^{cos}_i}{{w}^{cos}_j} \right] =
\begin{pmatrix}
$\,\,$ 1 $\,\,$ & $\,\,$1.7409$\,\,$ & $\,\,$9.8206$\,\,$ & $\,\,$\color{red} 5.2553\color{black} $\,\,$ \\
$\,\,$0.5744$\,\,$ & $\,\,$ 1 $\,\,$ & $\,\,$5.6412$\,\,$ & $\,\,$\color{red} 3.0188\color{black}   $\,\,$ \\
$\,\,$0.1018$\,\,$ & $\,\,$0.1773$\,\,$ & $\,\,$ 1 $\,\,$ & $\,\,$\color{red} 0.5351\color{black}  $\,\,$ \\
$\,\,$\color{red} 0.1903\color{black} $\,\,$ & $\,\,$\color{red} 0.3313\color{black} $\,\,$ & $\,\,$\color{red} 1.8687\color{black} $\,\,$ & $\,\,$ 1  $\,\,$ \\
\end{pmatrix},
\end{equation*}

\begin{equation*}
\mathbf{w}^{\prime} =
\begin{pmatrix}
0.535410\\
0.307553\\
0.054519\\
0.102518
\end{pmatrix} =
0.999363\cdot
\begin{pmatrix}
0.535752\\
0.307749\\
0.054554\\
\color{gr} 0.102583\color{black}
\end{pmatrix},
\end{equation*}
\begin{equation*}
\left[ \frac{{w}^{\prime}_i}{{w}^{\prime}_j} \right] =
\begin{pmatrix}
$\,\,$ 1 $\,\,$ & $\,\,$1.7409$\,\,$ & $\,\,$9.8206$\,\,$ & $\,\,$\color{gr} 5.2226\color{black} $\,\,$ \\
$\,\,$0.5744$\,\,$ & $\,\,$ 1 $\,\,$ & $\,\,$5.6412$\,\,$ & $\,\,$\color{gr} \color{blue} 3\color{black}   $\,\,$ \\
$\,\,$0.1018$\,\,$ & $\,\,$0.1773$\,\,$ & $\,\,$ 1 $\,\,$ & $\,\,$\color{gr} 0.5318\color{black}  $\,\,$ \\
$\,\,$\color{gr} 0.1915\color{black} $\,\,$ & $\,\,$\color{gr} \color{blue}  1/3\color{black} $\,\,$ & $\,\,$\color{gr} 1.8804\color{black} $\,\,$ & $\,\,$ 1  $\,\,$ \\
\end{pmatrix},
\end{equation*}
\end{example}
\newpage
\begin{example}
\begin{equation*}
\mathbf{A} =
\begin{pmatrix}
$\,\,$ 1 $\,\,$ & $\,\,$3$\,\,$ & $\,\,$8$\,\,$ & $\,\,$1 $\,\,$ \\
$\,\,$ 1/3$\,\,$ & $\,\,$ 1 $\,\,$ & $\,\,$4$\,\,$ & $\,\,$1 $\,\,$ \\
$\,\,$ 1/8$\,\,$ & $\,\,$ 1/4$\,\,$ & $\,\,$ 1 $\,\,$ & $\,\,$ 1/6 $\,\,$ \\
$\,\,$ 1 $\,\,$ & $\,\,$ 1 $\,\,$ & $\,\,$6$\,\,$ & $\,\,$ 1  $\,\,$ \\
\end{pmatrix},
\qquad
\lambda_{\max} =
4.1031,
\qquad
CR = 0.0389
\end{equation*}

\begin{equation*}
\mathbf{w}^{cos} =
\begin{pmatrix}
0.424230\\
0.215438\\
\color{red} 0.051035\color{black} \\
0.309296
\end{pmatrix}\end{equation*}
\begin{equation*}
\left[ \frac{{w}^{cos}_i}{{w}^{cos}_j} \right] =
\begin{pmatrix}
$\,\,$ 1 $\,\,$ & $\,\,$1.9691$\,\,$ & $\,\,$\color{red} 8.3125\color{black} $\,\,$ & $\,\,$1.3716$\,\,$ \\
$\,\,$0.5078$\,\,$ & $\,\,$ 1 $\,\,$ & $\,\,$\color{red} 4.2214\color{black} $\,\,$ & $\,\,$0.6965  $\,\,$ \\
$\,\,$\color{red} 0.1203\color{black} $\,\,$ & $\,\,$\color{red} 0.2369\color{black} $\,\,$ & $\,\,$ 1 $\,\,$ & $\,\,$\color{red} 0.1650\color{black}  $\,\,$ \\
$\,\,$0.7291$\,\,$ & $\,\,$1.4357$\,\,$ & $\,\,$\color{red} 6.0604\color{black} $\,\,$ & $\,\,$ 1  $\,\,$ \\
\end{pmatrix},
\end{equation*}

\begin{equation*}
\mathbf{w}^{\prime} =
\begin{pmatrix}
0.424012\\
0.215328\\
0.051523\\
0.309137
\end{pmatrix} =
0.999486\cdot
\begin{pmatrix}
0.424230\\
0.215438\\
\color{gr} 0.051549\color{black} \\
0.309296
\end{pmatrix},
\end{equation*}
\begin{equation*}
\left[ \frac{{w}^{\prime}_i}{{w}^{\prime}_j} \right] =
\begin{pmatrix}
$\,\,$ 1 $\,\,$ & $\,\,$1.9691$\,\,$ & $\,\,$\color{gr} 8.2296\color{black} $\,\,$ & $\,\,$1.3716$\,\,$ \\
$\,\,$0.5078$\,\,$ & $\,\,$ 1 $\,\,$ & $\,\,$\color{gr} 4.1793\color{black} $\,\,$ & $\,\,$0.6965  $\,\,$ \\
$\,\,$\color{gr} 0.1215\color{black} $\,\,$ & $\,\,$\color{gr} 0.2393\color{black} $\,\,$ & $\,\,$ 1 $\,\,$ & $\,\,$\color{gr} \color{blue}  1/6\color{black}  $\,\,$ \\
$\,\,$0.7291$\,\,$ & $\,\,$1.4357$\,\,$ & $\,\,$\color{gr} \color{blue} 6\color{black} $\,\,$ & $\,\,$ 1  $\,\,$ \\
\end{pmatrix},
\end{equation*}
\end{example}
\newpage
\begin{example}
\begin{equation*}
\mathbf{A} =
\begin{pmatrix}
$\,\,$ 1 $\,\,$ & $\,\,$3$\,\,$ & $\,\,$8$\,\,$ & $\,\,$3 $\,\,$ \\
$\,\,$ 1/3$\,\,$ & $\,\,$ 1 $\,\,$ & $\,\,$4$\,\,$ & $\,\,$3 $\,\,$ \\
$\,\,$ 1/8$\,\,$ & $\,\,$ 1/4$\,\,$ & $\,\,$ 1 $\,\,$ & $\,\,$ 1/2 $\,\,$ \\
$\,\,$ 1/3$\,\,$ & $\,\,$ 1/3$\,\,$ & $\,\,$2$\,\,$ & $\,\,$ 1  $\,\,$ \\
\end{pmatrix},
\qquad
\lambda_{\max} =
4.1031,
\qquad
CR = 0.0389
\end{equation*}

\begin{equation*}
\mathbf{w}^{cos} =
\begin{pmatrix}
0.531172\\
0.271119\\
\color{red} 0.064720\color{black} \\
0.132989
\end{pmatrix}\end{equation*}
\begin{equation*}
\left[ \frac{{w}^{cos}_i}{{w}^{cos}_j} \right] =
\begin{pmatrix}
$\,\,$ 1 $\,\,$ & $\,\,$1.9592$\,\,$ & $\,\,$\color{red} 8.2072\color{black} $\,\,$ & $\,\,$3.9941$\,\,$ \\
$\,\,$0.5104$\,\,$ & $\,\,$ 1 $\,\,$ & $\,\,$\color{red} 4.1891\color{black} $\,\,$ & $\,\,$2.0387  $\,\,$ \\
$\,\,$\color{red} 0.1218\color{black} $\,\,$ & $\,\,$\color{red} 0.2387\color{black} $\,\,$ & $\,\,$ 1 $\,\,$ & $\,\,$\color{red} 0.4867\color{black}  $\,\,$ \\
$\,\,$0.2504$\,\,$ & $\,\,$0.4905$\,\,$ & $\,\,$\color{red} 2.0548\color{black} $\,\,$ & $\,\,$ 1  $\,\,$ \\
\end{pmatrix},
\end{equation*}

\begin{equation*}
\mathbf{w}^{\prime} =
\begin{pmatrix}
0.530283\\
0.270665\\
0.066285\\
0.132766
\end{pmatrix} =
0.998327\cdot
\begin{pmatrix}
0.531172\\
0.271119\\
\color{gr} 0.066396\color{black} \\
0.132989
\end{pmatrix},
\end{equation*}
\begin{equation*}
\left[ \frac{{w}^{\prime}_i}{{w}^{\prime}_j} \right] =
\begin{pmatrix}
$\,\,$ 1 $\,\,$ & $\,\,$1.9592$\,\,$ & $\,\,$\color{gr} \color{blue} 8\color{black} $\,\,$ & $\,\,$3.9941$\,\,$ \\
$\,\,$0.5104$\,\,$ & $\,\,$ 1 $\,\,$ & $\,\,$\color{gr} 4.0833\color{black} $\,\,$ & $\,\,$2.0387  $\,\,$ \\
$\,\,$\color{gr} \color{blue}  1/8\color{black} $\,\,$ & $\,\,$\color{gr} 0.2449\color{black} $\,\,$ & $\,\,$ 1 $\,\,$ & $\,\,$\color{gr} 0.4993\color{black}  $\,\,$ \\
$\,\,$0.2504$\,\,$ & $\,\,$0.4905$\,\,$ & $\,\,$\color{gr} 2.0029\color{black} $\,\,$ & $\,\,$ 1  $\,\,$ \\
\end{pmatrix},
\end{equation*}
\end{example}
\newpage
\begin{example}
\begin{equation*}
\mathbf{A} =
\begin{pmatrix}
$\,\,$ 1 $\,\,$ & $\,\,$3$\,\,$ & $\,\,$9$\,\,$ & $\,\,$1 $\,\,$ \\
$\,\,$ 1/3$\,\,$ & $\,\,$ 1 $\,\,$ & $\,\,$5$\,\,$ & $\,\,$1 $\,\,$ \\
$\,\,$ 1/9$\,\,$ & $\,\,$ 1/5$\,\,$ & $\,\,$ 1 $\,\,$ & $\,\,$ 1/7 $\,\,$ \\
$\,\,$ 1 $\,\,$ & $\,\,$ 1 $\,\,$ & $\,\,$7$\,\,$ & $\,\,$ 1  $\,\,$ \\
\end{pmatrix},
\qquad
\lambda_{\max} =
4.1048,
\qquad
CR = 0.0395
\end{equation*}

\begin{equation*}
\mathbf{w}^{cos} =
\begin{pmatrix}
0.423515\\
0.220987\\
\color{red} 0.043848\color{black} \\
0.311650
\end{pmatrix}\end{equation*}
\begin{equation*}
\left[ \frac{{w}^{cos}_i}{{w}^{cos}_j} \right] =
\begin{pmatrix}
$\,\,$ 1 $\,\,$ & $\,\,$1.9165$\,\,$ & $\,\,$\color{red} 9.6587\color{black} $\,\,$ & $\,\,$1.3589$\,\,$ \\
$\,\,$0.5218$\,\,$ & $\,\,$ 1 $\,\,$ & $\,\,$\color{red} 5.0398\color{black} $\,\,$ & $\,\,$0.7091  $\,\,$ \\
$\,\,$\color{red} 0.1035\color{black} $\,\,$ & $\,\,$\color{red} 0.1984\color{black} $\,\,$ & $\,\,$ 1 $\,\,$ & $\,\,$\color{red} 0.1407\color{black}  $\,\,$ \\
$\,\,$0.7359$\,\,$ & $\,\,$1.4103$\,\,$ & $\,\,$\color{red} 7.1075\color{black} $\,\,$ & $\,\,$ 1  $\,\,$ \\
\end{pmatrix},
\end{equation*}

\begin{equation*}
\mathbf{w}^{\prime} =
\begin{pmatrix}
0.423367\\
0.220910\\
0.044182\\
0.311541
\end{pmatrix} =
0.999651\cdot
\begin{pmatrix}
0.423515\\
0.220987\\
\color{gr} 0.044197\color{black} \\
0.311650
\end{pmatrix},
\end{equation*}
\begin{equation*}
\left[ \frac{{w}^{\prime}_i}{{w}^{\prime}_j} \right] =
\begin{pmatrix}
$\,\,$ 1 $\,\,$ & $\,\,$1.9165$\,\,$ & $\,\,$\color{gr} 9.5824\color{black} $\,\,$ & $\,\,$1.3589$\,\,$ \\
$\,\,$0.5218$\,\,$ & $\,\,$ 1 $\,\,$ & $\,\,$\color{gr} \color{blue} 5\color{black} $\,\,$ & $\,\,$0.7091  $\,\,$ \\
$\,\,$\color{gr} 0.1044\color{black} $\,\,$ & $\,\,$\color{gr} \color{blue}  1/5\color{black} $\,\,$ & $\,\,$ 1 $\,\,$ & $\,\,$\color{gr} 0.1418\color{black}  $\,\,$ \\
$\,\,$0.7359$\,\,$ & $\,\,$1.4103$\,\,$ & $\,\,$\color{gr} 7.0513\color{black} $\,\,$ & $\,\,$ 1  $\,\,$ \\
\end{pmatrix},
\end{equation*}
\end{example}
\newpage
\begin{example}
\begin{equation*}
\mathbf{A} =
\begin{pmatrix}
$\,\,$ 1 $\,\,$ & $\,\,$3$\,\,$ & $\,\,$9$\,\,$ & $\,\,$6 $\,\,$ \\
$\,\,$ 1/3$\,\,$ & $\,\,$ 1 $\,\,$ & $\,\,$8$\,\,$ & $\,\,$3 $\,\,$ \\
$\,\,$ 1/9$\,\,$ & $\,\,$ 1/8$\,\,$ & $\,\,$ 1 $\,\,$ & $\,\,$ 1/2 $\,\,$ \\
$\,\,$ 1/6$\,\,$ & $\,\,$ 1/3$\,\,$ & $\,\,$2$\,\,$ & $\,\,$ 1  $\,\,$ \\
\end{pmatrix},
\qquad
\lambda_{\max} =
4.0820,
\qquad
CR = 0.0309
\end{equation*}

\begin{equation*}
\mathbf{w}^{cos} =
\begin{pmatrix}
0.574362\\
0.282792\\
0.049018\\
\color{red} 0.093827\color{black}
\end{pmatrix}\end{equation*}
\begin{equation*}
\left[ \frac{{w}^{cos}_i}{{w}^{cos}_j} \right] =
\begin{pmatrix}
$\,\,$ 1 $\,\,$ & $\,\,$2.0310$\,\,$ & $\,\,$11.7173$\,\,$ & $\,\,$\color{red} 6.1215\color{black} $\,\,$ \\
$\,\,$0.4924$\,\,$ & $\,\,$ 1 $\,\,$ & $\,\,$5.7691$\,\,$ & $\,\,$\color{red} 3.0140\color{black}   $\,\,$ \\
$\,\,$0.0853$\,\,$ & $\,\,$0.1733$\,\,$ & $\,\,$ 1 $\,\,$ & $\,\,$\color{red} 0.5224\color{black}  $\,\,$ \\
$\,\,$\color{red} 0.1634\color{black} $\,\,$ & $\,\,$\color{red} 0.3318\color{black} $\,\,$ & $\,\,$\color{red} 1.9141\color{black} $\,\,$ & $\,\,$ 1  $\,\,$ \\
\end{pmatrix},
\end{equation*}

\begin{equation*}
\mathbf{w}^{\prime} =
\begin{pmatrix}
0.574111\\
0.282669\\
0.048997\\
0.094223
\end{pmatrix} =
0.999563\cdot
\begin{pmatrix}
0.574362\\
0.282792\\
0.049018\\
\color{gr} 0.094264\color{black}
\end{pmatrix},
\end{equation*}
\begin{equation*}
\left[ \frac{{w}^{\prime}_i}{{w}^{\prime}_j} \right] =
\begin{pmatrix}
$\,\,$ 1 $\,\,$ & $\,\,$2.0310$\,\,$ & $\,\,$11.7173$\,\,$ & $\,\,$\color{gr} 6.0931\color{black} $\,\,$ \\
$\,\,$0.4924$\,\,$ & $\,\,$ 1 $\,\,$ & $\,\,$5.7691$\,\,$ & $\,\,$\color{gr} \color{blue} 3\color{black}   $\,\,$ \\
$\,\,$0.0853$\,\,$ & $\,\,$0.1733$\,\,$ & $\,\,$ 1 $\,\,$ & $\,\,$\color{gr} 0.5200\color{black}  $\,\,$ \\
$\,\,$\color{gr} 0.1641\color{black} $\,\,$ & $\,\,$\color{gr} \color{blue}  1/3\color{black} $\,\,$ & $\,\,$\color{gr} 1.9230\color{black} $\,\,$ & $\,\,$ 1  $\,\,$ \\
\end{pmatrix},
\end{equation*}
\end{example}
\newpage
\begin{example}
\begin{equation*}
\mathbf{A} =
\begin{pmatrix}
$\,\,$ 1 $\,\,$ & $\,\,$3$\,\,$ & $\,\,$9$\,\,$ & $\,\,$6 $\,\,$ \\
$\,\,$ 1/3$\,\,$ & $\,\,$ 1 $\,\,$ & $\,\,$9$\,\,$ & $\,\,$3 $\,\,$ \\
$\,\,$ 1/9$\,\,$ & $\,\,$ 1/9$\,\,$ & $\,\,$ 1 $\,\,$ & $\,\,$ 1/2 $\,\,$ \\
$\,\,$ 1/6$\,\,$ & $\,\,$ 1/3$\,\,$ & $\,\,$2$\,\,$ & $\,\,$ 1  $\,\,$ \\
\end{pmatrix},
\qquad
\lambda_{\max} =
4.1031,
\qquad
CR = 0.0389
\end{equation*}

\begin{equation*}
\mathbf{w}^{cos} =
\begin{pmatrix}
0.569203\\
0.290492\\
0.047694\\
\color{red} 0.092611\color{black}
\end{pmatrix}\end{equation*}
\begin{equation*}
\left[ \frac{{w}^{cos}_i}{{w}^{cos}_j} \right] =
\begin{pmatrix}
$\,\,$ 1 $\,\,$ & $\,\,$1.9594$\,\,$ & $\,\,$11.9344$\,\,$ & $\,\,$\color{red} 6.1462\color{black} $\,\,$ \\
$\,\,$0.5103$\,\,$ & $\,\,$ 1 $\,\,$ & $\,\,$6.0907$\,\,$ & $\,\,$\color{red} 3.1367\color{black}   $\,\,$ \\
$\,\,$0.0838$\,\,$ & $\,\,$0.1642$\,\,$ & $\,\,$ 1 $\,\,$ & $\,\,$\color{red} 0.5150\color{black}  $\,\,$ \\
$\,\,$\color{red} 0.1627\color{black} $\,\,$ & $\,\,$\color{red} 0.3188\color{black} $\,\,$ & $\,\,$\color{red} 1.9417\color{black} $\,\,$ & $\,\,$ 1  $\,\,$ \\
\end{pmatrix},
\end{equation*}

\begin{equation*}
\mathbf{w}^{\prime} =
\begin{pmatrix}
0.567921\\
0.289838\\
0.047587\\
0.094654
\end{pmatrix} =
0.997749\cdot
\begin{pmatrix}
0.569203\\
0.290492\\
0.047694\\
\color{gr} 0.094867\color{black}
\end{pmatrix},
\end{equation*}
\begin{equation*}
\left[ \frac{{w}^{\prime}_i}{{w}^{\prime}_j} \right] =
\begin{pmatrix}
$\,\,$ 1 $\,\,$ & $\,\,$1.9594$\,\,$ & $\,\,$11.9344$\,\,$ & $\,\,$\color{gr} \color{blue} 6\color{black} $\,\,$ \\
$\,\,$0.5103$\,\,$ & $\,\,$ 1 $\,\,$ & $\,\,$6.0907$\,\,$ & $\,\,$\color{gr} 3.0621\color{black}   $\,\,$ \\
$\,\,$0.0838$\,\,$ & $\,\,$0.1642$\,\,$ & $\,\,$ 1 $\,\,$ & $\,\,$\color{gr} 0.5027\color{black}  $\,\,$ \\
$\,\,$\color{gr} \color{blue}  1/6\color{black} $\,\,$ & $\,\,$\color{gr} 0.3266\color{black} $\,\,$ & $\,\,$\color{gr} 1.9891\color{black} $\,\,$ & $\,\,$ 1  $\,\,$ \\
\end{pmatrix},
\end{equation*}
\end{example}
\newpage
\begin{example}
\begin{equation*}
\mathbf{A} =
\begin{pmatrix}
$\,\,$ 1 $\,\,$ & $\,\,$4$\,\,$ & $\,\,$3$\,\,$ & $\,\,$7 $\,\,$ \\
$\,\,$ 1/4$\,\,$ & $\,\,$ 1 $\,\,$ & $\,\,$1$\,\,$ & $\,\,$1 $\,\,$ \\
$\,\,$ 1/3$\,\,$ & $\,\,$ 1 $\,\,$ & $\,\,$ 1 $\,\,$ & $\,\,$ 1/2 $\,\,$ \\
$\,\,$ 1/7$\,\,$ & $\,\,$ 1 $\,\,$ & $\,\,$2$\,\,$ & $\,\,$ 1  $\,\,$ \\
\end{pmatrix},
\qquad
\lambda_{\max} =
4.2109,
\qquad
CR = 0.0795
\end{equation*}

\begin{equation*}
\mathbf{w}^{cos} =
\begin{pmatrix}
0.567277\\
\color{red} 0.135456\color{black} \\
0.136472\\
0.160796
\end{pmatrix}\end{equation*}
\begin{equation*}
\left[ \frac{{w}^{cos}_i}{{w}^{cos}_j} \right] =
\begin{pmatrix}
$\,\,$ 1 $\,\,$ & $\,\,$\color{red} 4.1879\color{black} $\,\,$ & $\,\,$4.1567$\,\,$ & $\,\,$3.5279$\,\,$ \\
$\,\,$\color{red} 0.2388\color{black} $\,\,$ & $\,\,$ 1 $\,\,$ & $\,\,$\color{red} 0.9926\color{black} $\,\,$ & $\,\,$\color{red} 0.8424\color{black}   $\,\,$ \\
$\,\,$0.2406$\,\,$ & $\,\,$\color{red} 1.0075\color{black} $\,\,$ & $\,\,$ 1 $\,\,$ & $\,\,$0.8487 $\,\,$ \\
$\,\,$0.2835$\,\,$ & $\,\,$\color{red} 1.1871\color{black} $\,\,$ & $\,\,$1.1782$\,\,$ & $\,\,$ 1  $\,\,$ \\
\end{pmatrix},
\end{equation*}

\begin{equation*}
\mathbf{w}^{\prime} =
\begin{pmatrix}
0.566701\\
0.136333\\
0.136333\\
0.160632
\end{pmatrix} =
0.998985\cdot
\begin{pmatrix}
0.567277\\
\color{gr} 0.136472\color{black} \\
0.136472\\
0.160796
\end{pmatrix},
\end{equation*}
\begin{equation*}
\left[ \frac{{w}^{\prime}_i}{{w}^{\prime}_j} \right] =
\begin{pmatrix}
$\,\,$ 1 $\,\,$ & $\,\,$\color{gr} 4.1567\color{black} $\,\,$ & $\,\,$4.1567$\,\,$ & $\,\,$3.5279$\,\,$ \\
$\,\,$\color{gr} 0.2406\color{black} $\,\,$ & $\,\,$ 1 $\,\,$ & $\,\,$\color{gr} \color{blue} 1\color{black} $\,\,$ & $\,\,$\color{gr} 0.8487\color{black}   $\,\,$ \\
$\,\,$0.2406$\,\,$ & $\,\,$\color{gr} \color{blue} 1\color{black} $\,\,$ & $\,\,$ 1 $\,\,$ & $\,\,$0.8487 $\,\,$ \\
$\,\,$0.2835$\,\,$ & $\,\,$\color{gr} 1.1782\color{black} $\,\,$ & $\,\,$1.1782$\,\,$ & $\,\,$ 1  $\,\,$ \\
\end{pmatrix},
\end{equation*}
\end{example}
\newpage
\begin{example}
\begin{equation*}
\mathbf{A} =
\begin{pmatrix}
$\,\,$ 1 $\,\,$ & $\,\,$4$\,\,$ & $\,\,$3$\,\,$ & $\,\,$8 $\,\,$ \\
$\,\,$ 1/4$\,\,$ & $\,\,$ 1 $\,\,$ & $\,\,$1$\,\,$ & $\,\,$1 $\,\,$ \\
$\,\,$ 1/3$\,\,$ & $\,\,$ 1 $\,\,$ & $\,\,$ 1 $\,\,$ & $\,\,$ 1/2 $\,\,$ \\
$\,\,$ 1/8$\,\,$ & $\,\,$ 1 $\,\,$ & $\,\,$2$\,\,$ & $\,\,$ 1  $\,\,$ \\
\end{pmatrix},
\qquad
\lambda_{\max} =
4.2512,
\qquad
CR = 0.0947
\end{equation*}

\begin{equation*}
\mathbf{w}^{cos} =
\begin{pmatrix}
0.572920\\
\color{red} 0.133930\color{black} \\
0.136320\\
0.156829
\end{pmatrix}\end{equation*}
\begin{equation*}
\left[ \frac{{w}^{cos}_i}{{w}^{cos}_j} \right] =
\begin{pmatrix}
$\,\,$ 1 $\,\,$ & $\,\,$\color{red} 4.2778\color{black} $\,\,$ & $\,\,$4.2027$\,\,$ & $\,\,$3.6531$\,\,$ \\
$\,\,$\color{red} 0.2338\color{black} $\,\,$ & $\,\,$ 1 $\,\,$ & $\,\,$\color{red} 0.9825\color{black} $\,\,$ & $\,\,$\color{red} 0.8540\color{black}   $\,\,$ \\
$\,\,$0.2379$\,\,$ & $\,\,$\color{red} 1.0178\color{black} $\,\,$ & $\,\,$ 1 $\,\,$ & $\,\,$0.8692 $\,\,$ \\
$\,\,$0.2737$\,\,$ & $\,\,$\color{red} 1.1710\color{black} $\,\,$ & $\,\,$1.1504$\,\,$ & $\,\,$ 1  $\,\,$ \\
\end{pmatrix},
\end{equation*}

\begin{equation*}
\mathbf{w}^{\prime} =
\begin{pmatrix}
0.571554\\
0.135995\\
0.135995\\
0.156455
\end{pmatrix} =
0.997615\cdot
\begin{pmatrix}
0.572920\\
\color{gr} 0.136320\color{black} \\
0.136320\\
0.156829
\end{pmatrix},
\end{equation*}
\begin{equation*}
\left[ \frac{{w}^{\prime}_i}{{w}^{\prime}_j} \right] =
\begin{pmatrix}
$\,\,$ 1 $\,\,$ & $\,\,$\color{gr} 4.2027\color{black} $\,\,$ & $\,\,$4.2027$\,\,$ & $\,\,$3.6531$\,\,$ \\
$\,\,$\color{gr} 0.2379\color{black} $\,\,$ & $\,\,$ 1 $\,\,$ & $\,\,$\color{gr} \color{blue} 1\color{black} $\,\,$ & $\,\,$\color{gr} 0.8692\color{black}   $\,\,$ \\
$\,\,$0.2379$\,\,$ & $\,\,$\color{gr} \color{blue} 1\color{black} $\,\,$ & $\,\,$ 1 $\,\,$ & $\,\,$0.8692 $\,\,$ \\
$\,\,$0.2737$\,\,$ & $\,\,$\color{gr} 1.1504\color{black} $\,\,$ & $\,\,$1.1504$\,\,$ & $\,\,$ 1  $\,\,$ \\
\end{pmatrix},
\end{equation*}
\end{example}
\newpage
\begin{example}
\begin{equation*}
\mathbf{A} =
\begin{pmatrix}
$\,\,$ 1 $\,\,$ & $\,\,$4$\,\,$ & $\,\,$4$\,\,$ & $\,\,$1 $\,\,$ \\
$\,\,$ 1/4$\,\,$ & $\,\,$ 1 $\,\,$ & $\,\,$2$\,\,$ & $\,\,$1 $\,\,$ \\
$\,\,$ 1/4$\,\,$ & $\,\,$ 1/2$\,\,$ & $\,\,$ 1 $\,\,$ & $\,\,$ 1/3 $\,\,$ \\
$\,\,$ 1 $\,\,$ & $\,\,$ 1 $\,\,$ & $\,\,$3$\,\,$ & $\,\,$ 1  $\,\,$ \\
\end{pmatrix},
\qquad
\lambda_{\max} =
4.1707,
\qquad
CR = 0.0644
\end{equation*}

\begin{equation*}
\mathbf{w}^{cos} =
\begin{pmatrix}
0.419956\\
0.192414\\
\color{red} 0.094949\color{black} \\
0.292681
\end{pmatrix}\end{equation*}
\begin{equation*}
\left[ \frac{{w}^{cos}_i}{{w}^{cos}_j} \right] =
\begin{pmatrix}
$\,\,$ 1 $\,\,$ & $\,\,$2.1826$\,\,$ & $\,\,$\color{red} 4.4229\color{black} $\,\,$ & $\,\,$1.4349$\,\,$ \\
$\,\,$0.4582$\,\,$ & $\,\,$ 1 $\,\,$ & $\,\,$\color{red} 2.0265\color{black} $\,\,$ & $\,\,$0.6574  $\,\,$ \\
$\,\,$\color{red} 0.2261\color{black} $\,\,$ & $\,\,$\color{red} 0.4935\color{black} $\,\,$ & $\,\,$ 1 $\,\,$ & $\,\,$\color{red} 0.3244\color{black}  $\,\,$ \\
$\,\,$0.6969$\,\,$ & $\,\,$1.5211$\,\,$ & $\,\,$\color{red} 3.0825\color{black} $\,\,$ & $\,\,$ 1  $\,\,$ \\
\end{pmatrix},
\end{equation*}

\begin{equation*}
\mathbf{w}^{\prime} =
\begin{pmatrix}
0.419428\\
0.192172\\
0.096086\\
0.292314
\end{pmatrix} =
0.998744\cdot
\begin{pmatrix}
0.419956\\
0.192414\\
\color{gr} 0.096207\color{black} \\
0.292681
\end{pmatrix},
\end{equation*}
\begin{equation*}
\left[ \frac{{w}^{\prime}_i}{{w}^{\prime}_j} \right] =
\begin{pmatrix}
$\,\,$ 1 $\,\,$ & $\,\,$2.1826$\,\,$ & $\,\,$\color{gr} 4.3651\color{black} $\,\,$ & $\,\,$1.4349$\,\,$ \\
$\,\,$0.4582$\,\,$ & $\,\,$ 1 $\,\,$ & $\,\,$\color{gr} \color{blue} 2\color{black} $\,\,$ & $\,\,$0.6574  $\,\,$ \\
$\,\,$\color{gr} 0.2291\color{black} $\,\,$ & $\,\,$\color{gr} \color{blue}  1/2\color{black} $\,\,$ & $\,\,$ 1 $\,\,$ & $\,\,$\color{gr} 0.3287\color{black}  $\,\,$ \\
$\,\,$0.6969$\,\,$ & $\,\,$1.5211$\,\,$ & $\,\,$\color{gr} 3.0422\color{black} $\,\,$ & $\,\,$ 1  $\,\,$ \\
\end{pmatrix},
\end{equation*}
\end{example}
\newpage
\begin{example}
\begin{equation*}
\mathbf{A} =
\begin{pmatrix}
$\,\,$ 1 $\,\,$ & $\,\,$4$\,\,$ & $\,\,$6$\,\,$ & $\,\,$2 $\,\,$ \\
$\,\,$ 1/4$\,\,$ & $\,\,$ 1 $\,\,$ & $\,\,$1$\,\,$ & $\,\,$1 $\,\,$ \\
$\,\,$ 1/6$\,\,$ & $\,\,$ 1 $\,\,$ & $\,\,$ 1 $\,\,$ & $\,\,$ 1/5 $\,\,$ \\
$\,\,$ 1/2$\,\,$ & $\,\,$ 1 $\,\,$ & $\,\,$5$\,\,$ & $\,\,$ 1  $\,\,$ \\
\end{pmatrix},
\qquad
\lambda_{\max} =
4.2277,
\qquad
CR = 0.0859
\end{equation*}

\begin{equation*}
\mathbf{w}^{cos} =
\begin{pmatrix}
\color{red} 0.507115\color{black} \\
0.148033\\
0.087881\\
0.256971
\end{pmatrix}\end{equation*}
\begin{equation*}
\left[ \frac{{w}^{cos}_i}{{w}^{cos}_j} \right] =
\begin{pmatrix}
$\,\,$ 1 $\,\,$ & $\,\,$\color{red} 3.4257\color{black} $\,\,$ & $\,\,$\color{red} 5.7704\color{black} $\,\,$ & $\,\,$\color{red} 1.9734\color{black} $\,\,$ \\
$\,\,$\color{red} 0.2919\color{black} $\,\,$ & $\,\,$ 1 $\,\,$ & $\,\,$1.6845$\,\,$ & $\,\,$0.5761  $\,\,$ \\
$\,\,$\color{red} 0.1733\color{black} $\,\,$ & $\,\,$0.5937$\,\,$ & $\,\,$ 1 $\,\,$ & $\,\,$0.3420 $\,\,$ \\
$\,\,$\color{red} 0.5067\color{black} $\,\,$ & $\,\,$1.7359$\,\,$ & $\,\,$2.9241$\,\,$ & $\,\,$ 1  $\,\,$ \\
\end{pmatrix},
\end{equation*}

\begin{equation*}
\mathbf{w}^{\prime} =
\begin{pmatrix}
0.510457\\
0.147029\\
0.087285\\
0.255229
\end{pmatrix} =
0.993218\cdot
\begin{pmatrix}
\color{gr} 0.513943\color{black} \\
0.148033\\
0.087881\\
0.256971
\end{pmatrix},
\end{equation*}
\begin{equation*}
\left[ \frac{{w}^{\prime}_i}{{w}^{\prime}_j} \right] =
\begin{pmatrix}
$\,\,$ 1 $\,\,$ & $\,\,$\color{gr} 3.4718\color{black} $\,\,$ & $\,\,$\color{gr} 5.8481\color{black} $\,\,$ & $\,\,$\color{gr} \color{blue} 2\color{black} $\,\,$ \\
$\,\,$\color{gr} 0.2880\color{black} $\,\,$ & $\,\,$ 1 $\,\,$ & $\,\,$1.6845$\,\,$ & $\,\,$0.5761  $\,\,$ \\
$\,\,$\color{gr} 0.1710\color{black} $\,\,$ & $\,\,$0.5937$\,\,$ & $\,\,$ 1 $\,\,$ & $\,\,$0.3420 $\,\,$ \\
$\,\,$\color{gr} \color{blue}  1/2\color{black} $\,\,$ & $\,\,$1.7359$\,\,$ & $\,\,$2.9241$\,\,$ & $\,\,$ 1  $\,\,$ \\
\end{pmatrix},
\end{equation*}
\end{example}
\newpage
\begin{example}
\begin{equation*}
\mathbf{A} =
\begin{pmatrix}
$\,\,$ 1 $\,\,$ & $\,\,$4$\,\,$ & $\,\,$6$\,\,$ & $\,\,$3 $\,\,$ \\
$\,\,$ 1/4$\,\,$ & $\,\,$ 1 $\,\,$ & $\,\,$1$\,\,$ & $\,\,$1 $\,\,$ \\
$\,\,$ 1/6$\,\,$ & $\,\,$ 1 $\,\,$ & $\,\,$ 1 $\,\,$ & $\,\,$ 1/3 $\,\,$ \\
$\,\,$ 1/3$\,\,$ & $\,\,$ 1 $\,\,$ & $\,\,$3$\,\,$ & $\,\,$ 1  $\,\,$ \\
\end{pmatrix},
\qquad
\lambda_{\max} =
4.1031,
\qquad
CR = 0.0389
\end{equation*}

\begin{equation*}
\mathbf{w}^{cos} =
\begin{pmatrix}
\color{red} 0.562694\color{black} \\
0.140996\\
0.097914\\
0.198396
\end{pmatrix}\end{equation*}
\begin{equation*}
\left[ \frac{{w}^{cos}_i}{{w}^{cos}_j} \right] =
\begin{pmatrix}
$\,\,$ 1 $\,\,$ & $\,\,$\color{red} 3.9908\color{black} $\,\,$ & $\,\,$\color{red} 5.7468\color{black} $\,\,$ & $\,\,$\color{red} 2.8362\color{black} $\,\,$ \\
$\,\,$\color{red} 0.2506\color{black} $\,\,$ & $\,\,$ 1 $\,\,$ & $\,\,$1.4400$\,\,$ & $\,\,$0.7107  $\,\,$ \\
$\,\,$\color{red} 0.1740\color{black} $\,\,$ & $\,\,$0.6944$\,\,$ & $\,\,$ 1 $\,\,$ & $\,\,$0.4935 $\,\,$ \\
$\,\,$\color{red} 0.3526\color{black} $\,\,$ & $\,\,$1.4071$\,\,$ & $\,\,$2.0262$\,\,$ & $\,\,$ 1  $\,\,$ \\
\end{pmatrix},
\end{equation*}

\begin{equation*}
\mathbf{w}^{\prime} =
\begin{pmatrix}
0.563258\\
0.140814\\
0.097788\\
0.198140
\end{pmatrix} =
0.998711\cdot
\begin{pmatrix}
\color{gr} 0.563985\color{black} \\
0.140996\\
0.097914\\
0.198396
\end{pmatrix},
\end{equation*}
\begin{equation*}
\left[ \frac{{w}^{\prime}_i}{{w}^{\prime}_j} \right] =
\begin{pmatrix}
$\,\,$ 1 $\,\,$ & $\,\,$\color{gr} \color{blue} 4\color{black} $\,\,$ & $\,\,$\color{gr} 5.7600\color{black} $\,\,$ & $\,\,$\color{gr} 2.8427\color{black} $\,\,$ \\
$\,\,$\color{gr} \color{blue}  1/4\color{black} $\,\,$ & $\,\,$ 1 $\,\,$ & $\,\,$1.4400$\,\,$ & $\,\,$0.7107  $\,\,$ \\
$\,\,$\color{gr} 0.1736\color{black} $\,\,$ & $\,\,$0.6944$\,\,$ & $\,\,$ 1 $\,\,$ & $\,\,$0.4935 $\,\,$ \\
$\,\,$\color{gr} 0.3518\color{black} $\,\,$ & $\,\,$1.4071$\,\,$ & $\,\,$2.0262$\,\,$ & $\,\,$ 1  $\,\,$ \\
\end{pmatrix},
\end{equation*}
\end{example}
\newpage
\begin{example}
\begin{equation*}
\mathbf{A} =
\begin{pmatrix}
$\,\,$ 1 $\,\,$ & $\,\,$4$\,\,$ & $\,\,$6$\,\,$ & $\,\,$3 $\,\,$ \\
$\,\,$ 1/4$\,\,$ & $\,\,$ 1 $\,\,$ & $\,\,$1$\,\,$ & $\,\,$1 $\,\,$ \\
$\,\,$ 1/6$\,\,$ & $\,\,$ 1 $\,\,$ & $\,\,$ 1 $\,\,$ & $\,\,$ 1/4 $\,\,$ \\
$\,\,$ 1/3$\,\,$ & $\,\,$ 1 $\,\,$ & $\,\,$4$\,\,$ & $\,\,$ 1  $\,\,$ \\
\end{pmatrix},
\qquad
\lambda_{\max} =
4.1707,
\qquad
CR = 0.0644
\end{equation*}

\begin{equation*}
\mathbf{w}^{cos} =
\begin{pmatrix}
\color{red} 0.553243\color{black} \\
0.139421\\
0.092411\\
0.214925
\end{pmatrix}\end{equation*}
\begin{equation*}
\left[ \frac{{w}^{cos}_i}{{w}^{cos}_j} \right] =
\begin{pmatrix}
$\,\,$ 1 $\,\,$ & $\,\,$\color{red} 3.9681\color{black} $\,\,$ & $\,\,$\color{red} 5.9868\color{black} $\,\,$ & $\,\,$\color{red} 2.5741\color{black} $\,\,$ \\
$\,\,$\color{red} 0.2520\color{black} $\,\,$ & $\,\,$ 1 $\,\,$ & $\,\,$1.5087$\,\,$ & $\,\,$0.6487  $\,\,$ \\
$\,\,$\color{red} 0.1670\color{black} $\,\,$ & $\,\,$0.6628$\,\,$ & $\,\,$ 1 $\,\,$ & $\,\,$0.4300 $\,\,$ \\
$\,\,$\color{red} 0.3885\color{black} $\,\,$ & $\,\,$1.5415$\,\,$ & $\,\,$2.3258$\,\,$ & $\,\,$ 1  $\,\,$ \\
\end{pmatrix},
\end{equation*}

\begin{equation*}
\mathbf{w}^{\prime} =
\begin{pmatrix}
0.553789\\
0.139251\\
0.092298\\
0.214662
\end{pmatrix} =
0.998779\cdot
\begin{pmatrix}
\color{gr} 0.554466\color{black} \\
0.139421\\
0.092411\\
0.214925
\end{pmatrix},
\end{equation*}
\begin{equation*}
\left[ \frac{{w}^{\prime}_i}{{w}^{\prime}_j} \right] =
\begin{pmatrix}
$\,\,$ 1 $\,\,$ & $\,\,$\color{gr} 3.9769\color{black} $\,\,$ & $\,\,$\color{gr} \color{blue} 6\color{black} $\,\,$ & $\,\,$\color{gr} 2.5798\color{black} $\,\,$ \\
$\,\,$\color{gr} 0.2515\color{black} $\,\,$ & $\,\,$ 1 $\,\,$ & $\,\,$1.5087$\,\,$ & $\,\,$0.6487  $\,\,$ \\
$\,\,$\color{gr} \color{blue}  1/6\color{black} $\,\,$ & $\,\,$0.6628$\,\,$ & $\,\,$ 1 $\,\,$ & $\,\,$0.4300 $\,\,$ \\
$\,\,$\color{gr} 0.3876\color{black} $\,\,$ & $\,\,$1.5415$\,\,$ & $\,\,$2.3258$\,\,$ & $\,\,$ 1  $\,\,$ \\
\end{pmatrix},
\end{equation*}
\end{example}
\newpage
\begin{example}
\begin{equation*}
\mathbf{A} =
\begin{pmatrix}
$\,\,$ 1 $\,\,$ & $\,\,$4$\,\,$ & $\,\,$6$\,\,$ & $\,\,$5 $\,\,$ \\
$\,\,$ 1/4$\,\,$ & $\,\,$ 1 $\,\,$ & $\,\,$1$\,\,$ & $\,\,$2 $\,\,$ \\
$\,\,$ 1/6$\,\,$ & $\,\,$ 1 $\,\,$ & $\,\,$ 1 $\,\,$ & $\,\,$ 1/2 $\,\,$ \\
$\,\,$ 1/5$\,\,$ & $\,\,$ 1/2$\,\,$ & $\,\,$2$\,\,$ & $\,\,$ 1  $\,\,$ \\
\end{pmatrix},
\qquad
\lambda_{\max} =
4.1655,
\qquad
CR = 0.0624
\end{equation*}

\begin{equation*}
\mathbf{w}^{cos} =
\begin{pmatrix}
\color{red} 0.605455\color{black} \\
0.161022\\
0.103751\\
0.129772
\end{pmatrix}\end{equation*}
\begin{equation*}
\left[ \frac{{w}^{cos}_i}{{w}^{cos}_j} \right] =
\begin{pmatrix}
$\,\,$ 1 $\,\,$ & $\,\,$\color{red} 3.7601\color{black} $\,\,$ & $\,\,$\color{red} 5.8357\color{black} $\,\,$ & $\,\,$\color{red} 4.6655\color{black} $\,\,$ \\
$\,\,$\color{red} 0.2660\color{black} $\,\,$ & $\,\,$ 1 $\,\,$ & $\,\,$1.5520$\,\,$ & $\,\,$1.2408  $\,\,$ \\
$\,\,$\color{red} 0.1714\color{black} $\,\,$ & $\,\,$0.6443$\,\,$ & $\,\,$ 1 $\,\,$ & $\,\,$0.7995 $\,\,$ \\
$\,\,$\color{red} 0.2143\color{black} $\,\,$ & $\,\,$0.8059$\,\,$ & $\,\,$1.2508$\,\,$ & $\,\,$ 1  $\,\,$ \\
\end{pmatrix},
\end{equation*}

\begin{equation*}
\mathbf{w}^{\prime} =
\begin{pmatrix}
0.612070\\
0.158323\\
0.102012\\
0.127596
\end{pmatrix} =
0.983236\cdot
\begin{pmatrix}
\color{gr} 0.622505\color{black} \\
0.161022\\
0.103751\\
0.129772
\end{pmatrix},
\end{equation*}
\begin{equation*}
\left[ \frac{{w}^{\prime}_i}{{w}^{\prime}_j} \right] =
\begin{pmatrix}
$\,\,$ 1 $\,\,$ & $\,\,$\color{gr} 3.8660\color{black} $\,\,$ & $\,\,$\color{gr} \color{blue} 6\color{black} $\,\,$ & $\,\,$\color{gr} 4.7969\color{black} $\,\,$ \\
$\,\,$\color{gr} 0.2587\color{black} $\,\,$ & $\,\,$ 1 $\,\,$ & $\,\,$1.5520$\,\,$ & $\,\,$1.2408  $\,\,$ \\
$\,\,$\color{gr} \color{blue}  1/6\color{black} $\,\,$ & $\,\,$0.6443$\,\,$ & $\,\,$ 1 $\,\,$ & $\,\,$0.7995 $\,\,$ \\
$\,\,$\color{gr} 0.2085\color{black} $\,\,$ & $\,\,$0.8059$\,\,$ & $\,\,$1.2508$\,\,$ & $\,\,$ 1  $\,\,$ \\
\end{pmatrix},
\end{equation*}
\end{example}
\newpage
\begin{example}
\begin{equation*}
\mathbf{A} =
\begin{pmatrix}
$\,\,$ 1 $\,\,$ & $\,\,$4$\,\,$ & $\,\,$6$\,\,$ & $\,\,$6 $\,\,$ \\
$\,\,$ 1/4$\,\,$ & $\,\,$ 1 $\,\,$ & $\,\,$1$\,\,$ & $\,\,$2 $\,\,$ \\
$\,\,$ 1/6$\,\,$ & $\,\,$ 1 $\,\,$ & $\,\,$ 1 $\,\,$ & $\,\,$ 1/2 $\,\,$ \\
$\,\,$ 1/6$\,\,$ & $\,\,$ 1/2$\,\,$ & $\,\,$2$\,\,$ & $\,\,$ 1  $\,\,$ \\
\end{pmatrix},
\qquad
\lambda_{\max} =
4.1707,
\qquad
CR = 0.0644
\end{equation*}

\begin{equation*}
\mathbf{w}^{cos} =
\begin{pmatrix}
\color{red} 0.619434\color{black} \\
0.154974\\
0.103268\\
0.122323
\end{pmatrix}\end{equation*}
\begin{equation*}
\left[ \frac{{w}^{cos}_i}{{w}^{cos}_j} \right] =
\begin{pmatrix}
$\,\,$ 1 $\,\,$ & $\,\,$\color{red} 3.9970\color{black} $\,\,$ & $\,\,$\color{red} 5.9983\color{black} $\,\,$ & $\,\,$\color{red} 5.0639\color{black} $\,\,$ \\
$\,\,$\color{red} 0.2502\color{black} $\,\,$ & $\,\,$ 1 $\,\,$ & $\,\,$1.5007$\,\,$ & $\,\,$1.2669  $\,\,$ \\
$\,\,$\color{red} 0.1667\color{black} $\,\,$ & $\,\,$0.6664$\,\,$ & $\,\,$ 1 $\,\,$ & $\,\,$0.8442 $\,\,$ \\
$\,\,$\color{red} 0.1975\color{black} $\,\,$ & $\,\,$0.7893$\,\,$ & $\,\,$1.1845$\,\,$ & $\,\,$ 1  $\,\,$ \\
\end{pmatrix},
\end{equation*}

\begin{equation*}
\mathbf{w}^{\prime} =
\begin{pmatrix}
0.619501\\
0.154947\\
0.103250\\
0.122302
\end{pmatrix} =
0.999824\cdot
\begin{pmatrix}
\color{gr} 0.619611\color{black} \\
0.154974\\
0.103268\\
0.122323
\end{pmatrix},
\end{equation*}
\begin{equation*}
\left[ \frac{{w}^{\prime}_i}{{w}^{\prime}_j} \right] =
\begin{pmatrix}
$\,\,$ 1 $\,\,$ & $\,\,$\color{gr} 3.9982\color{black} $\,\,$ & $\,\,$\color{gr} \color{blue} 6\color{black} $\,\,$ & $\,\,$\color{gr} 5.0654\color{black} $\,\,$ \\
$\,\,$\color{gr} 0.2501\color{black} $\,\,$ & $\,\,$ 1 $\,\,$ & $\,\,$1.5007$\,\,$ & $\,\,$1.2669  $\,\,$ \\
$\,\,$\color{gr} \color{blue}  1/6\color{black} $\,\,$ & $\,\,$0.6664$\,\,$ & $\,\,$ 1 $\,\,$ & $\,\,$0.8442 $\,\,$ \\
$\,\,$\color{gr} 0.1974\color{black} $\,\,$ & $\,\,$0.7893$\,\,$ & $\,\,$1.1845$\,\,$ & $\,\,$ 1  $\,\,$ \\
\end{pmatrix},
\end{equation*}
\end{example}
\newpage
\begin{example}
\begin{equation*}
\mathbf{A} =
\begin{pmatrix}
$\,\,$ 1 $\,\,$ & $\,\,$4$\,\,$ & $\,\,$7$\,\,$ & $\,\,$1 $\,\,$ \\
$\,\,$ 1/4$\,\,$ & $\,\,$ 1 $\,\,$ & $\,\,$3$\,\,$ & $\,\,$1 $\,\,$ \\
$\,\,$ 1/7$\,\,$ & $\,\,$ 1/3$\,\,$ & $\,\,$ 1 $\,\,$ & $\,\,$ 1/5 $\,\,$ \\
$\,\,$ 1 $\,\,$ & $\,\,$ 1 $\,\,$ & $\,\,$5$\,\,$ & $\,\,$ 1  $\,\,$ \\
\end{pmatrix},
\qquad
\lambda_{\max} =
4.1667,
\qquad
CR = 0.0629
\end{equation*}

\begin{equation*}
\mathbf{w}^{cos} =
\begin{pmatrix}
0.441503\\
0.194592\\
\color{red} 0.059627\color{black} \\
0.304278
\end{pmatrix}\end{equation*}
\begin{equation*}
\left[ \frac{{w}^{cos}_i}{{w}^{cos}_j} \right] =
\begin{pmatrix}
$\,\,$ 1 $\,\,$ & $\,\,$2.2689$\,\,$ & $\,\,$\color{red} 7.4044\color{black} $\,\,$ & $\,\,$1.4510$\,\,$ \\
$\,\,$0.4407$\,\,$ & $\,\,$ 1 $\,\,$ & $\,\,$\color{red} 3.2635\color{black} $\,\,$ & $\,\,$0.6395  $\,\,$ \\
$\,\,$\color{red} 0.1351\color{black} $\,\,$ & $\,\,$\color{red} 0.3064\color{black} $\,\,$ & $\,\,$ 1 $\,\,$ & $\,\,$\color{red} 0.1960\color{black}  $\,\,$ \\
$\,\,$0.6892$\,\,$ & $\,\,$1.5637$\,\,$ & $\,\,$\color{red} 5.1030\color{black} $\,\,$ & $\,\,$ 1  $\,\,$ \\
\end{pmatrix},
\end{equation*}

\begin{equation*}
\mathbf{w}^{\prime} =
\begin{pmatrix}
0.440962\\
0.194353\\
0.060781\\
0.303904
\end{pmatrix} =
0.998773\cdot
\begin{pmatrix}
0.441503\\
0.194592\\
\color{gr} 0.060856\color{black} \\
0.304278
\end{pmatrix},
\end{equation*}
\begin{equation*}
\left[ \frac{{w}^{\prime}_i}{{w}^{\prime}_j} \right] =
\begin{pmatrix}
$\,\,$ 1 $\,\,$ & $\,\,$2.2689$\,\,$ & $\,\,$\color{gr} 7.2549\color{black} $\,\,$ & $\,\,$1.4510$\,\,$ \\
$\,\,$0.4407$\,\,$ & $\,\,$ 1 $\,\,$ & $\,\,$\color{gr} 3.1976\color{black} $\,\,$ & $\,\,$0.6395  $\,\,$ \\
$\,\,$\color{gr} 0.1378\color{black} $\,\,$ & $\,\,$\color{gr} 0.3127\color{black} $\,\,$ & $\,\,$ 1 $\,\,$ & $\,\,$\color{gr} \color{blue}  1/5\color{black}  $\,\,$ \\
$\,\,$0.6892$\,\,$ & $\,\,$1.5637$\,\,$ & $\,\,$\color{gr} \color{blue} 5\color{black} $\,\,$ & $\,\,$ 1  $\,\,$ \\
\end{pmatrix},
\end{equation*}
\end{example}
\newpage
\begin{example}
\begin{equation*}
\mathbf{A} =
\begin{pmatrix}
$\,\,$ 1 $\,\,$ & $\,\,$4$\,\,$ & $\,\,$7$\,\,$ & $\,\,$5 $\,\,$ \\
$\,\,$ 1/4$\,\,$ & $\,\,$ 1 $\,\,$ & $\,\,$1$\,\,$ & $\,\,$2 $\,\,$ \\
$\,\,$ 1/7$\,\,$ & $\,\,$ 1 $\,\,$ & $\,\,$ 1 $\,\,$ & $\,\,$ 1/2 $\,\,$ \\
$\,\,$ 1/5$\,\,$ & $\,\,$ 1/2$\,\,$ & $\,\,$2$\,\,$ & $\,\,$ 1  $\,\,$ \\
\end{pmatrix},
\qquad
\lambda_{\max} =
4.1665,
\qquad
CR = 0.0628
\end{equation*}

\begin{equation*}
\mathbf{w}^{cos} =
\begin{pmatrix}
\color{red} 0.616677\color{black} \\
0.159990\\
0.098233\\
0.125100
\end{pmatrix}\end{equation*}
\begin{equation*}
\left[ \frac{{w}^{cos}_i}{{w}^{cos}_j} \right] =
\begin{pmatrix}
$\,\,$ 1 $\,\,$ & $\,\,$\color{red} 3.8545\color{black} $\,\,$ & $\,\,$\color{red} 6.2777\color{black} $\,\,$ & $\,\,$\color{red} 4.9295\color{black} $\,\,$ \\
$\,\,$\color{red} 0.2594\color{black} $\,\,$ & $\,\,$ 1 $\,\,$ & $\,\,$1.6287$\,\,$ & $\,\,$1.2789  $\,\,$ \\
$\,\,$\color{red} 0.1593\color{black} $\,\,$ & $\,\,$0.6140$\,\,$ & $\,\,$ 1 $\,\,$ & $\,\,$0.7852 $\,\,$ \\
$\,\,$\color{red} 0.2029\color{black} $\,\,$ & $\,\,$0.7819$\,\,$ & $\,\,$1.2735$\,\,$ & $\,\,$ 1  $\,\,$ \\
\end{pmatrix},
\end{equation*}

\begin{equation*}
\mathbf{w}^{\prime} =
\begin{pmatrix}
0.620029\\
0.158591\\
0.097374\\
0.124006
\end{pmatrix} =
0.991256\cdot
\begin{pmatrix}
\color{gr} 0.625498\color{black} \\
0.159990\\
0.098233\\
0.125100
\end{pmatrix},
\end{equation*}
\begin{equation*}
\left[ \frac{{w}^{\prime}_i}{{w}^{\prime}_j} \right] =
\begin{pmatrix}
$\,\,$ 1 $\,\,$ & $\,\,$\color{gr} 3.9096\color{black} $\,\,$ & $\,\,$\color{gr} 6.3675\color{black} $\,\,$ & $\,\,$\color{gr} \color{blue} 5\color{black} $\,\,$ \\
$\,\,$\color{gr} 0.2558\color{black} $\,\,$ & $\,\,$ 1 $\,\,$ & $\,\,$1.6287$\,\,$ & $\,\,$1.2789  $\,\,$ \\
$\,\,$\color{gr} 0.1570\color{black} $\,\,$ & $\,\,$0.6140$\,\,$ & $\,\,$ 1 $\,\,$ & $\,\,$0.7852 $\,\,$ \\
$\,\,$\color{gr} \color{blue}  1/5\color{black} $\,\,$ & $\,\,$0.7819$\,\,$ & $\,\,$1.2735$\,\,$ & $\,\,$ 1  $\,\,$ \\
\end{pmatrix},
\end{equation*}
\end{example}
\newpage
\begin{example}
\begin{equation*}
\mathbf{A} =
\begin{pmatrix}
$\,\,$ 1 $\,\,$ & $\,\,$4$\,\,$ & $\,\,$7$\,\,$ & $\,\,$6 $\,\,$ \\
$\,\,$ 1/4$\,\,$ & $\,\,$ 1 $\,\,$ & $\,\,$9$\,\,$ & $\,\,$3 $\,\,$ \\
$\,\,$ 1/7$\,\,$ & $\,\,$ 1/9$\,\,$ & $\,\,$ 1 $\,\,$ & $\,\,$ 1/2 $\,\,$ \\
$\,\,$ 1/6$\,\,$ & $\,\,$ 1/3$\,\,$ & $\,\,$2$\,\,$ & $\,\,$ 1  $\,\,$ \\
\end{pmatrix},
\qquad
\lambda_{\max} =
4.2359,
\qquad
CR = 0.0890
\end{equation*}

\begin{equation*}
\mathbf{w}^{cos} =
\begin{pmatrix}
0.569105\\
0.283630\\
0.053892\\
\color{red} 0.093373\color{black}
\end{pmatrix}\end{equation*}
\begin{equation*}
\left[ \frac{{w}^{cos}_i}{{w}^{cos}_j} \right] =
\begin{pmatrix}
$\,\,$ 1 $\,\,$ & $\,\,$2.0065$\,\,$ & $\,\,$10.5600$\,\,$ & $\,\,$\color{red} 6.0950\color{black} $\,\,$ \\
$\,\,$0.4984$\,\,$ & $\,\,$ 1 $\,\,$ & $\,\,$5.2629$\,\,$ & $\,\,$\color{red} 3.0376\color{black}   $\,\,$ \\
$\,\,$0.0947$\,\,$ & $\,\,$0.1900$\,\,$ & $\,\,$ 1 $\,\,$ & $\,\,$\color{red} 0.5772\color{black}  $\,\,$ \\
$\,\,$\color{red} 0.1641\color{black} $\,\,$ & $\,\,$\color{red} 0.3292\color{black} $\,\,$ & $\,\,$\color{red} 1.7326\color{black} $\,\,$ & $\,\,$ 1  $\,\,$ \\
\end{pmatrix},
\end{equation*}

\begin{equation*}
\mathbf{w}^{\prime} =
\begin{pmatrix}
0.568439\\
0.283298\\
0.053829\\
0.094433
\end{pmatrix} =
0.998831\cdot
\begin{pmatrix}
0.569105\\
0.283630\\
0.053892\\
\color{gr} 0.094543\color{black}
\end{pmatrix},
\end{equation*}
\begin{equation*}
\left[ \frac{{w}^{\prime}_i}{{w}^{\prime}_j} \right] =
\begin{pmatrix}
$\,\,$ 1 $\,\,$ & $\,\,$2.0065$\,\,$ & $\,\,$10.5600$\,\,$ & $\,\,$\color{gr} 6.0195\color{black} $\,\,$ \\
$\,\,$0.4984$\,\,$ & $\,\,$ 1 $\,\,$ & $\,\,$5.2629$\,\,$ & $\,\,$\color{gr} \color{blue} 3\color{black}   $\,\,$ \\
$\,\,$0.0947$\,\,$ & $\,\,$0.1900$\,\,$ & $\,\,$ 1 $\,\,$ & $\,\,$\color{gr} 0.5700\color{black}  $\,\,$ \\
$\,\,$\color{gr} 0.1661\color{black} $\,\,$ & $\,\,$\color{gr} \color{blue}  1/3\color{black} $\,\,$ & $\,\,$\color{gr} 1.7543\color{black} $\,\,$ & $\,\,$ 1  $\,\,$ \\
\end{pmatrix},
\end{equation*}
\end{example}
\newpage
\begin{example}
\begin{equation*}
\mathbf{A} =
\begin{pmatrix}
$\,\,$ 1 $\,\,$ & $\,\,$4$\,\,$ & $\,\,$8$\,\,$ & $\,\,$1 $\,\,$ \\
$\,\,$ 1/4$\,\,$ & $\,\,$ 1 $\,\,$ & $\,\,$4$\,\,$ & $\,\,$1 $\,\,$ \\
$\,\,$ 1/8$\,\,$ & $\,\,$ 1/4$\,\,$ & $\,\,$ 1 $\,\,$ & $\,\,$ 1/6 $\,\,$ \\
$\,\,$ 1 $\,\,$ & $\,\,$ 1 $\,\,$ & $\,\,$6$\,\,$ & $\,\,$ 1  $\,\,$ \\
\end{pmatrix},
\qquad
\lambda_{\max} =
4.1707,
\qquad
CR = 0.0644
\end{equation*}

\begin{equation*}
\mathbf{w}^{cos} =
\begin{pmatrix}
0.440534\\
0.202137\\
\color{red} 0.049857\color{black} \\
0.307472
\end{pmatrix}\end{equation*}
\begin{equation*}
\left[ \frac{{w}^{cos}_i}{{w}^{cos}_j} \right] =
\begin{pmatrix}
$\,\,$ 1 $\,\,$ & $\,\,$2.1794$\,\,$ & $\,\,$\color{red} 8.8360\color{black} $\,\,$ & $\,\,$1.4328$\,\,$ \\
$\,\,$0.4588$\,\,$ & $\,\,$ 1 $\,\,$ & $\,\,$\color{red} 4.0544\color{black} $\,\,$ & $\,\,$0.6574  $\,\,$ \\
$\,\,$\color{red} 0.1132\color{black} $\,\,$ & $\,\,$\color{red} 0.2466\color{black} $\,\,$ & $\,\,$ 1 $\,\,$ & $\,\,$\color{red} 0.1622\color{black}  $\,\,$ \\
$\,\,$0.6980$\,\,$ & $\,\,$1.5211$\,\,$ & $\,\,$\color{red} 6.1671\color{black} $\,\,$ & $\,\,$ 1  $\,\,$ \\
\end{pmatrix},
\end{equation*}

\begin{equation*}
\mathbf{w}^{\prime} =
\begin{pmatrix}
0.440235\\
0.202000\\
0.050500\\
0.307264
\end{pmatrix} =
0.999323\cdot
\begin{pmatrix}
0.440534\\
0.202137\\
\color{gr} 0.050534\color{black} \\
0.307472
\end{pmatrix},
\end{equation*}
\begin{equation*}
\left[ \frac{{w}^{\prime}_i}{{w}^{\prime}_j} \right] =
\begin{pmatrix}
$\,\,$ 1 $\,\,$ & $\,\,$2.1794$\,\,$ & $\,\,$\color{gr} 8.7175\color{black} $\,\,$ & $\,\,$1.4328$\,\,$ \\
$\,\,$0.4588$\,\,$ & $\,\,$ 1 $\,\,$ & $\,\,$\color{gr} \color{blue} 4\color{black} $\,\,$ & $\,\,$0.6574  $\,\,$ \\
$\,\,$\color{gr} 0.1147\color{black} $\,\,$ & $\,\,$\color{gr} \color{blue}  1/4\color{black} $\,\,$ & $\,\,$ 1 $\,\,$ & $\,\,$\color{gr} 0.1644\color{black}  $\,\,$ \\
$\,\,$0.6980$\,\,$ & $\,\,$1.5211$\,\,$ & $\,\,$\color{gr} 6.0844\color{black} $\,\,$ & $\,\,$ 1  $\,\,$ \\
\end{pmatrix},
\end{equation*}
\end{example}
\newpage
\begin{example}
\begin{equation*}
\mathbf{A} =
\begin{pmatrix}
$\,\,$ 1 $\,\,$ & $\,\,$4$\,\,$ & $\,\,$8$\,\,$ & $\,\,$6 $\,\,$ \\
$\,\,$ 1/4$\,\,$ & $\,\,$ 1 $\,\,$ & $\,\,$9$\,\,$ & $\,\,$3 $\,\,$ \\
$\,\,$ 1/8$\,\,$ & $\,\,$ 1/9$\,\,$ & $\,\,$ 1 $\,\,$ & $\,\,$ 1/2 $\,\,$ \\
$\,\,$ 1/6$\,\,$ & $\,\,$ 1/3$\,\,$ & $\,\,$2$\,\,$ & $\,\,$ 1  $\,\,$ \\
\end{pmatrix},
\qquad
\lambda_{\max} =
4.1990,
\qquad
CR = 0.0750
\end{equation*}

\begin{equation*}
\mathbf{w}^{cos} =
\begin{pmatrix}
0.579496\\
0.277809\\
0.050496\\
\color{red} 0.092199\color{black}
\end{pmatrix}\end{equation*}
\begin{equation*}
\left[ \frac{{w}^{cos}_i}{{w}^{cos}_j} \right] =
\begin{pmatrix}
$\,\,$ 1 $\,\,$ & $\,\,$2.0860$\,\,$ & $\,\,$11.4761$\,\,$ & $\,\,$\color{red} 6.2853\color{black} $\,\,$ \\
$\,\,$0.4794$\,\,$ & $\,\,$ 1 $\,\,$ & $\,\,$5.5016$\,\,$ & $\,\,$\color{red} 3.0131\color{black}   $\,\,$ \\
$\,\,$0.0871$\,\,$ & $\,\,$0.1818$\,\,$ & $\,\,$ 1 $\,\,$ & $\,\,$\color{red} 0.5477\color{black}  $\,\,$ \\
$\,\,$\color{red} 0.1591\color{black} $\,\,$ & $\,\,$\color{red} 0.3319\color{black} $\,\,$ & $\,\,$\color{red} 1.8259\color{black} $\,\,$ & $\,\,$ 1  $\,\,$ \\
\end{pmatrix},
\end{equation*}

\begin{equation*}
\mathbf{w}^{\prime} =
\begin{pmatrix}
0.579262\\
0.277697\\
0.050475\\
0.092566
\end{pmatrix} =
0.999596\cdot
\begin{pmatrix}
0.579496\\
0.277809\\
0.050496\\
\color{gr} 0.092603\color{black}
\end{pmatrix},
\end{equation*}
\begin{equation*}
\left[ \frac{{w}^{\prime}_i}{{w}^{\prime}_j} \right] =
\begin{pmatrix}
$\,\,$ 1 $\,\,$ & $\,\,$2.0860$\,\,$ & $\,\,$11.4761$\,\,$ & $\,\,$\color{gr} 6.2579\color{black} $\,\,$ \\
$\,\,$0.4794$\,\,$ & $\,\,$ 1 $\,\,$ & $\,\,$5.5016$\,\,$ & $\,\,$\color{gr} \color{blue} 3\color{black}   $\,\,$ \\
$\,\,$0.0871$\,\,$ & $\,\,$0.1818$\,\,$ & $\,\,$ 1 $\,\,$ & $\,\,$\color{gr} 0.5453\color{black}  $\,\,$ \\
$\,\,$\color{gr} 0.1598\color{black} $\,\,$ & $\,\,$\color{gr} \color{blue}  1/3\color{black} $\,\,$ & $\,\,$\color{gr} 1.8339\color{black} $\,\,$ & $\,\,$ 1  $\,\,$ \\
\end{pmatrix},
\end{equation*}
\end{example}
\newpage
\begin{example}
\begin{equation*}
\mathbf{A} =
\begin{pmatrix}
$\,\,$ 1 $\,\,$ & $\,\,$4$\,\,$ & $\,\,$9$\,\,$ & $\,\,$1 $\,\,$ \\
$\,\,$ 1/4$\,\,$ & $\,\,$ 1 $\,\,$ & $\,\,$4$\,\,$ & $\,\,$1 $\,\,$ \\
$\,\,$ 1/9$\,\,$ & $\,\,$ 1/4$\,\,$ & $\,\,$ 1 $\,\,$ & $\,\,$ 1/6 $\,\,$ \\
$\,\,$ 1 $\,\,$ & $\,\,$ 1 $\,\,$ & $\,\,$6$\,\,$ & $\,\,$ 1  $\,\,$ \\
\end{pmatrix},
\qquad
\lambda_{\max} =
4.1664,
\qquad
CR = 0.0627
\end{equation*}

\begin{equation*}
\mathbf{w}^{cos} =
\begin{pmatrix}
0.448765\\
0.199643\\
\color{red} 0.047786\color{black} \\
0.303806
\end{pmatrix}\end{equation*}
\begin{equation*}
\left[ \frac{{w}^{cos}_i}{{w}^{cos}_j} \right] =
\begin{pmatrix}
$\,\,$ 1 $\,\,$ & $\,\,$2.2478$\,\,$ & $\,\,$\color{red} 9.3911\color{black} $\,\,$ & $\,\,$1.4771$\,\,$ \\
$\,\,$0.4449$\,\,$ & $\,\,$ 1 $\,\,$ & $\,\,$\color{red} 4.1778\color{black} $\,\,$ & $\,\,$0.6571  $\,\,$ \\
$\,\,$\color{red} 0.1065\color{black} $\,\,$ & $\,\,$\color{red} 0.2394\color{black} $\,\,$ & $\,\,$ 1 $\,\,$ & $\,\,$\color{red} 0.1573\color{black}  $\,\,$ \\
$\,\,$0.6770$\,\,$ & $\,\,$1.5217$\,\,$ & $\,\,$\color{red} 6.3576\color{black} $\,\,$ & $\,\,$ 1  $\,\,$ \\
\end{pmatrix},
\end{equation*}

\begin{equation*}
\mathbf{w}^{\prime} =
\begin{pmatrix}
0.447835\\
0.199229\\
0.049759\\
0.303176
\end{pmatrix} =
0.997928\cdot
\begin{pmatrix}
0.448765\\
0.199643\\
\color{gr} 0.049863\color{black} \\
0.303806
\end{pmatrix},
\end{equation*}
\begin{equation*}
\left[ \frac{{w}^{\prime}_i}{{w}^{\prime}_j} \right] =
\begin{pmatrix}
$\,\,$ 1 $\,\,$ & $\,\,$2.2478$\,\,$ & $\,\,$\color{gr} \color{blue} 9\color{black} $\,\,$ & $\,\,$1.4771$\,\,$ \\
$\,\,$0.4449$\,\,$ & $\,\,$ 1 $\,\,$ & $\,\,$\color{gr} 4.0039\color{black} $\,\,$ & $\,\,$0.6571  $\,\,$ \\
$\,\,$\color{gr} \color{blue}  1/9\color{black} $\,\,$ & $\,\,$\color{gr} 0.2498\color{black} $\,\,$ & $\,\,$ 1 $\,\,$ & $\,\,$\color{gr} 0.1641\color{black}  $\,\,$ \\
$\,\,$0.6770$\,\,$ & $\,\,$1.5217$\,\,$ & $\,\,$\color{gr} 6.0928\color{black} $\,\,$ & $\,\,$ 1  $\,\,$ \\
\end{pmatrix},
\end{equation*}
\end{example}
\newpage
\begin{example}
\begin{equation*}
\mathbf{A} =
\begin{pmatrix}
$\,\,$ 1 $\,\,$ & $\,\,$4$\,\,$ & $\,\,$9$\,\,$ & $\,\,$3 $\,\,$ \\
$\,\,$ 1/4$\,\,$ & $\,\,$ 1 $\,\,$ & $\,\,$4$\,\,$ & $\,\,$3 $\,\,$ \\
$\,\,$ 1/9$\,\,$ & $\,\,$ 1/4$\,\,$ & $\,\,$ 1 $\,\,$ & $\,\,$ 1/2 $\,\,$ \\
$\,\,$ 1/3$\,\,$ & $\,\,$ 1/3$\,\,$ & $\,\,$2$\,\,$ & $\,\,$ 1  $\,\,$ \\
\end{pmatrix},
\qquad
\lambda_{\max} =
4.1664,
\qquad
CR = 0.0627
\end{equation*}

\begin{equation*}
\mathbf{w}^{cos} =
\begin{pmatrix}
0.558084\\
0.250238\\
\color{red} 0.060568\color{black} \\
0.131110
\end{pmatrix}\end{equation*}
\begin{equation*}
\left[ \frac{{w}^{cos}_i}{{w}^{cos}_j} \right] =
\begin{pmatrix}
$\,\,$ 1 $\,\,$ & $\,\,$2.2302$\,\,$ & $\,\,$\color{red} 9.2142\color{black} $\,\,$ & $\,\,$4.2566$\,\,$ \\
$\,\,$0.4484$\,\,$ & $\,\,$ 1 $\,\,$ & $\,\,$\color{red} 4.1315\color{black} $\,\,$ & $\,\,$1.9086  $\,\,$ \\
$\,\,$\color{red} 0.1085\color{black} $\,\,$ & $\,\,$\color{red} 0.2420\color{black} $\,\,$ & $\,\,$ 1 $\,\,$ & $\,\,$\color{red} 0.4620\color{black}  $\,\,$ \\
$\,\,$0.2349$\,\,$ & $\,\,$0.5239$\,\,$ & $\,\,$\color{red} 2.1647\color{black} $\,\,$ & $\,\,$ 1  $\,\,$ \\
\end{pmatrix},
\end{equation*}

\begin{equation*}
\mathbf{w}^{\prime} =
\begin{pmatrix}
0.557281\\
0.249878\\
0.061920\\
0.130921
\end{pmatrix} =
0.998560\cdot
\begin{pmatrix}
0.558084\\
0.250238\\
\color{gr} 0.062009\color{black} \\
0.131110
\end{pmatrix},
\end{equation*}
\begin{equation*}
\left[ \frac{{w}^{\prime}_i}{{w}^{\prime}_j} \right] =
\begin{pmatrix}
$\,\,$ 1 $\,\,$ & $\,\,$2.2302$\,\,$ & $\,\,$\color{gr} \color{blue} 9\color{black} $\,\,$ & $\,\,$4.2566$\,\,$ \\
$\,\,$0.4484$\,\,$ & $\,\,$ 1 $\,\,$ & $\,\,$\color{gr} 4.0355\color{black} $\,\,$ & $\,\,$1.9086  $\,\,$ \\
$\,\,$\color{gr} \color{blue}  1/9\color{black} $\,\,$ & $\,\,$\color{gr} 0.2478\color{black} $\,\,$ & $\,\,$ 1 $\,\,$ & $\,\,$\color{gr} 0.4730\color{black}  $\,\,$ \\
$\,\,$0.2349$\,\,$ & $\,\,$0.5239$\,\,$ & $\,\,$\color{gr} 2.1144\color{black} $\,\,$ & $\,\,$ 1  $\,\,$ \\
\end{pmatrix},
\end{equation*}
\end{example}
\newpage
\begin{example}
\begin{equation*}
\mathbf{A} =
\begin{pmatrix}
$\,\,$ 1 $\,\,$ & $\,\,$4$\,\,$ & $\,\,$9$\,\,$ & $\,\,$7 $\,\,$ \\
$\,\,$ 1/4$\,\,$ & $\,\,$ 1 $\,\,$ & $\,\,$9$\,\,$ & $\,\,$3 $\,\,$ \\
$\,\,$ 1/9$\,\,$ & $\,\,$ 1/9$\,\,$ & $\,\,$ 1 $\,\,$ & $\,\,$ 1/2 $\,\,$ \\
$\,\,$ 1/7$\,\,$ & $\,\,$ 1/3$\,\,$ & $\,\,$2$\,\,$ & $\,\,$ 1  $\,\,$ \\
\end{pmatrix},
\qquad
\lambda_{\max} =
4.1658,
\qquad
CR = 0.0625
\end{equation*}

\begin{equation*}
\mathbf{w}^{cos} =
\begin{pmatrix}
0.600836\\
0.266821\\
0.046895\\
\color{red} 0.085448\color{black}
\end{pmatrix}\end{equation*}
\begin{equation*}
\left[ \frac{{w}^{cos}_i}{{w}^{cos}_j} \right] =
\begin{pmatrix}
$\,\,$ 1 $\,\,$ & $\,\,$2.2518$\,\,$ & $\,\,$12.8123$\,\,$ & $\,\,$\color{red} 7.0316\color{black} $\,\,$ \\
$\,\,$0.4441$\,\,$ & $\,\,$ 1 $\,\,$ & $\,\,$5.6897$\,\,$ & $\,\,$\color{red} 3.1226\color{black}   $\,\,$ \\
$\,\,$0.0780$\,\,$ & $\,\,$0.1758$\,\,$ & $\,\,$ 1 $\,\,$ & $\,\,$\color{red} 0.5488\color{black}  $\,\,$ \\
$\,\,$\color{red} 0.1422\color{black} $\,\,$ & $\,\,$\color{red} 0.3202\color{black} $\,\,$ & $\,\,$\color{red} 1.8221\color{black} $\,\,$ & $\,\,$ 1  $\,\,$ \\
\end{pmatrix},
\end{equation*}

\begin{equation*}
\mathbf{w}^{\prime} =
\begin{pmatrix}
0.600604\\
0.266718\\
0.046877\\
0.085801
\end{pmatrix} =
0.999614\cdot
\begin{pmatrix}
0.600836\\
0.266821\\
0.046895\\
\color{gr} 0.085834\color{black}
\end{pmatrix},
\end{equation*}
\begin{equation*}
\left[ \frac{{w}^{\prime}_i}{{w}^{\prime}_j} \right] =
\begin{pmatrix}
$\,\,$ 1 $\,\,$ & $\,\,$2.2518$\,\,$ & $\,\,$12.8123$\,\,$ & $\,\,$\color{gr} \color{blue} 7\color{black} $\,\,$ \\
$\,\,$0.4441$\,\,$ & $\,\,$ 1 $\,\,$ & $\,\,$5.6897$\,\,$ & $\,\,$\color{gr} 3.1086\color{black}   $\,\,$ \\
$\,\,$0.0780$\,\,$ & $\,\,$0.1758$\,\,$ & $\,\,$ 1 $\,\,$ & $\,\,$\color{gr} 0.5463\color{black}  $\,\,$ \\
$\,\,$\color{gr} \color{blue}  1/7\color{black} $\,\,$ & $\,\,$\color{gr} 0.3217\color{black} $\,\,$ & $\,\,$\color{gr} 1.8303\color{black} $\,\,$ & $\,\,$ 1  $\,\,$ \\
\end{pmatrix},
\end{equation*}
\end{example}
\newpage
\begin{example}
\begin{equation*}
\mathbf{A} =
\begin{pmatrix}
$\,\,$ 1 $\,\,$ & $\,\,$5$\,\,$ & $\,\,$5$\,\,$ & $\,\,$1 $\,\,$ \\
$\,\,$ 1/5$\,\,$ & $\,\,$ 1 $\,\,$ & $\,\,$2$\,\,$ & $\,\,$1 $\,\,$ \\
$\,\,$ 1/5$\,\,$ & $\,\,$ 1/2$\,\,$ & $\,\,$ 1 $\,\,$ & $\,\,$ 1/3 $\,\,$ \\
$\,\,$ 1 $\,\,$ & $\,\,$ 1 $\,\,$ & $\,\,$3$\,\,$ & $\,\,$ 1  $\,\,$ \\
\end{pmatrix},
\qquad
\lambda_{\max} =
4.2277,
\qquad
CR = 0.0859
\end{equation*}

\begin{equation*}
\mathbf{w}^{cos} =
\begin{pmatrix}
0.447091\\
0.180307\\
\color{red} 0.086428\color{black} \\
0.286175
\end{pmatrix}\end{equation*}
\begin{equation*}
\left[ \frac{{w}^{cos}_i}{{w}^{cos}_j} \right] =
\begin{pmatrix}
$\,\,$ 1 $\,\,$ & $\,\,$2.4796$\,\,$ & $\,\,$\color{red} 5.1730\color{black} $\,\,$ & $\,\,$1.5623$\,\,$ \\
$\,\,$0.4033$\,\,$ & $\,\,$ 1 $\,\,$ & $\,\,$\color{red} 2.0862\color{black} $\,\,$ & $\,\,$0.6301  $\,\,$ \\
$\,\,$\color{red} 0.1933\color{black} $\,\,$ & $\,\,$\color{red} 0.4793\color{black} $\,\,$ & $\,\,$ 1 $\,\,$ & $\,\,$\color{red} 0.3020\color{black}  $\,\,$ \\
$\,\,$0.6401$\,\,$ & $\,\,$1.5872$\,\,$ & $\,\,$\color{red} 3.3111\color{black} $\,\,$ & $\,\,$ 1  $\,\,$ \\
\end{pmatrix},
\end{equation*}

\begin{equation*}
\mathbf{w}^{\prime} =
\begin{pmatrix}
0.445758\\
0.179769\\
0.089152\\
0.285321
\end{pmatrix} =
0.997019\cdot
\begin{pmatrix}
0.447091\\
0.180307\\
\color{gr} 0.089418\color{black} \\
0.286175
\end{pmatrix},
\end{equation*}
\begin{equation*}
\left[ \frac{{w}^{\prime}_i}{{w}^{\prime}_j} \right] =
\begin{pmatrix}
$\,\,$ 1 $\,\,$ & $\,\,$2.4796$\,\,$ & $\,\,$\color{gr} \color{blue} 5\color{black} $\,\,$ & $\,\,$1.5623$\,\,$ \\
$\,\,$0.4033$\,\,$ & $\,\,$ 1 $\,\,$ & $\,\,$\color{gr} 2.0164\color{black} $\,\,$ & $\,\,$0.6301  $\,\,$ \\
$\,\,$\color{gr} \color{blue}  1/5\color{black} $\,\,$ & $\,\,$\color{gr} 0.4959\color{black} $\,\,$ & $\,\,$ 1 $\,\,$ & $\,\,$\color{gr} 0.3125\color{black}  $\,\,$ \\
$\,\,$0.6401$\,\,$ & $\,\,$1.5872$\,\,$ & $\,\,$\color{gr} 3.2004\color{black} $\,\,$ & $\,\,$ 1  $\,\,$ \\
\end{pmatrix},
\end{equation*}
\end{example}
\newpage
\begin{example}
\begin{equation*}
\mathbf{A} =
\begin{pmatrix}
$\,\,$ 1 $\,\,$ & $\,\,$5$\,\,$ & $\,\,$6$\,\,$ & $\,\,$1 $\,\,$ \\
$\,\,$ 1/5$\,\,$ & $\,\,$ 1 $\,\,$ & $\,\,$2$\,\,$ & $\,\,$1 $\,\,$ \\
$\,\,$ 1/6$\,\,$ & $\,\,$ 1/2$\,\,$ & $\,\,$ 1 $\,\,$ & $\,\,$ 1/4 $\,\,$ \\
$\,\,$ 1 $\,\,$ & $\,\,$ 1 $\,\,$ & $\,\,$4$\,\,$ & $\,\,$ 1  $\,\,$ \\
\end{pmatrix},
\qquad
\lambda_{\max} =
4.2277,
\qquad
CR = 0.0859
\end{equation*}

\begin{equation*}
\mathbf{w}^{cos} =
\begin{pmatrix}
0.453253\\
0.175181\\
\color{red} 0.073105\color{black} \\
0.298461
\end{pmatrix}\end{equation*}
\begin{equation*}
\left[ \frac{{w}^{cos}_i}{{w}^{cos}_j} \right] =
\begin{pmatrix}
$\,\,$ 1 $\,\,$ & $\,\,$2.5873$\,\,$ & $\,\,$\color{red} 6.2000\color{black} $\,\,$ & $\,\,$1.5186$\,\,$ \\
$\,\,$0.3865$\,\,$ & $\,\,$ 1 $\,\,$ & $\,\,$\color{red} 2.3963\color{black} $\,\,$ & $\,\,$0.5869  $\,\,$ \\
$\,\,$\color{red} 0.1613\color{black} $\,\,$ & $\,\,$\color{red} 0.4173\color{black} $\,\,$ & $\,\,$ 1 $\,\,$ & $\,\,$\color{red} 0.2449\color{black}  $\,\,$ \\
$\,\,$0.6585$\,\,$ & $\,\,$1.7037$\,\,$ & $\,\,$\color{red} 4.0826\color{black} $\,\,$ & $\,\,$ 1  $\,\,$ \\
\end{pmatrix},
\end{equation*}

\begin{equation*}
\mathbf{w}^{\prime} =
\begin{pmatrix}
0.452570\\
0.174917\\
0.074503\\
0.298011
\end{pmatrix} =
0.998492\cdot
\begin{pmatrix}
0.453253\\
0.175181\\
\color{gr} 0.074615\color{black} \\
0.298461
\end{pmatrix},
\end{equation*}
\begin{equation*}
\left[ \frac{{w}^{\prime}_i}{{w}^{\prime}_j} \right] =
\begin{pmatrix}
$\,\,$ 1 $\,\,$ & $\,\,$2.5873$\,\,$ & $\,\,$\color{gr} 6.0745\color{black} $\,\,$ & $\,\,$1.5186$\,\,$ \\
$\,\,$0.3865$\,\,$ & $\,\,$ 1 $\,\,$ & $\,\,$\color{gr} 2.3478\color{black} $\,\,$ & $\,\,$0.5869  $\,\,$ \\
$\,\,$\color{gr} 0.1646\color{black} $\,\,$ & $\,\,$\color{gr} 0.4259\color{black} $\,\,$ & $\,\,$ 1 $\,\,$ & $\,\,$\color{gr} \color{blue}  1/4\color{black}  $\,\,$ \\
$\,\,$0.6585$\,\,$ & $\,\,$1.7037$\,\,$ & $\,\,$\color{gr} \color{blue} 4\color{black} $\,\,$ & $\,\,$ 1  $\,\,$ \\
\end{pmatrix},
\end{equation*}
\end{example}
\newpage
\begin{example}
\begin{equation*}
\mathbf{A} =
\begin{pmatrix}
$\,\,$ 1 $\,\,$ & $\,\,$5$\,\,$ & $\,\,$7$\,\,$ & $\,\,$1 $\,\,$ \\
$\,\,$ 1/5$\,\,$ & $\,\,$ 1 $\,\,$ & $\,\,$3$\,\,$ & $\,\,$1 $\,\,$ \\
$\,\,$ 1/7$\,\,$ & $\,\,$ 1/3$\,\,$ & $\,\,$ 1 $\,\,$ & $\,\,$ 1/5 $\,\,$ \\
$\,\,$ 1 $\,\,$ & $\,\,$ 1 $\,\,$ & $\,\,$5$\,\,$ & $\,\,$ 1  $\,\,$ \\
\end{pmatrix},
\qquad
\lambda_{\max} =
4.2309,
\qquad
CR = 0.0871
\end{equation*}

\begin{equation*}
\mathbf{w}^{cos} =
\begin{pmatrix}
0.452189\\
0.186094\\
\color{red} 0.058499\color{black} \\
0.303218
\end{pmatrix}\end{equation*}
\begin{equation*}
\left[ \frac{{w}^{cos}_i}{{w}^{cos}_j} \right] =
\begin{pmatrix}
$\,\,$ 1 $\,\,$ & $\,\,$2.4299$\,\,$ & $\,\,$\color{red} 7.7299\color{black} $\,\,$ & $\,\,$1.4913$\,\,$ \\
$\,\,$0.4115$\,\,$ & $\,\,$ 1 $\,\,$ & $\,\,$\color{red} 3.1812\color{black} $\,\,$ & $\,\,$0.6137  $\,\,$ \\
$\,\,$\color{red} 0.1294\color{black} $\,\,$ & $\,\,$\color{red} 0.3143\color{black} $\,\,$ & $\,\,$ 1 $\,\,$ & $\,\,$\color{red} 0.1929\color{black}  $\,\,$ \\
$\,\,$0.6706$\,\,$ & $\,\,$1.6294$\,\,$ & $\,\,$\color{red} 5.1833\color{black} $\,\,$ & $\,\,$ 1  $\,\,$ \\
\end{pmatrix},
\end{equation*}

\begin{equation*}
\mathbf{w}^{\prime} =
\begin{pmatrix}
0.451221\\
0.185696\\
0.060514\\
0.302569
\end{pmatrix} =
0.997860\cdot
\begin{pmatrix}
0.452189\\
0.186094\\
\color{gr} 0.060644\color{black} \\
0.303218
\end{pmatrix},
\end{equation*}
\begin{equation*}
\left[ \frac{{w}^{\prime}_i}{{w}^{\prime}_j} \right] =
\begin{pmatrix}
$\,\,$ 1 $\,\,$ & $\,\,$2.4299$\,\,$ & $\,\,$\color{gr} 7.4565\color{black} $\,\,$ & $\,\,$1.4913$\,\,$ \\
$\,\,$0.4115$\,\,$ & $\,\,$ 1 $\,\,$ & $\,\,$\color{gr} 3.0687\color{black} $\,\,$ & $\,\,$0.6137  $\,\,$ \\
$\,\,$\color{gr} 0.1341\color{black} $\,\,$ & $\,\,$\color{gr} 0.3259\color{black} $\,\,$ & $\,\,$ 1 $\,\,$ & $\,\,$\color{gr} \color{blue}  1/5\color{black}  $\,\,$ \\
$\,\,$0.6706$\,\,$ & $\,\,$1.6294$\,\,$ & $\,\,$\color{gr} \color{blue} 5\color{black} $\,\,$ & $\,\,$ 1  $\,\,$ \\
\end{pmatrix},
\end{equation*}
\end{example}
\newpage
\begin{example}
\begin{equation*}
\mathbf{A} =
\begin{pmatrix}
$\,\,$ 1 $\,\,$ & $\,\,$5$\,\,$ & $\,\,$7$\,\,$ & $\,\,$3 $\,\,$ \\
$\,\,$ 1/5$\,\,$ & $\,\,$ 1 $\,\,$ & $\,\,$1$\,\,$ & $\,\,$1 $\,\,$ \\
$\,\,$ 1/7$\,\,$ & $\,\,$ 1 $\,\,$ & $\,\,$ 1 $\,\,$ & $\,\,$ 1/4 $\,\,$ \\
$\,\,$ 1/3$\,\,$ & $\,\,$ 1 $\,\,$ & $\,\,$4$\,\,$ & $\,\,$ 1  $\,\,$ \\
\end{pmatrix},
\qquad
\lambda_{\max} =
4.1667,
\qquad
CR = 0.0629
\end{equation*}

\begin{equation*}
\mathbf{w}^{cos} =
\begin{pmatrix}
\color{red} 0.582284\color{black} \\
0.127922\\
0.083233\\
0.206561
\end{pmatrix}\end{equation*}
\begin{equation*}
\left[ \frac{{w}^{cos}_i}{{w}^{cos}_j} \right] =
\begin{pmatrix}
$\,\,$ 1 $\,\,$ & $\,\,$\color{red} 4.5519\color{black} $\,\,$ & $\,\,$\color{red} 6.9959\color{black} $\,\,$ & $\,\,$\color{red} 2.8189\color{black} $\,\,$ \\
$\,\,$\color{red} 0.2197\color{black} $\,\,$ & $\,\,$ 1 $\,\,$ & $\,\,$1.5369$\,\,$ & $\,\,$0.6193  $\,\,$ \\
$\,\,$\color{red} 0.1429\color{black} $\,\,$ & $\,\,$0.6507$\,\,$ & $\,\,$ 1 $\,\,$ & $\,\,$0.4029 $\,\,$ \\
$\,\,$\color{red} 0.3547\color{black} $\,\,$ & $\,\,$1.6147$\,\,$ & $\,\,$2.4817$\,\,$ & $\,\,$ 1  $\,\,$ \\
\end{pmatrix},
\end{equation*}

\begin{equation*}
\mathbf{w}^{\prime} =
\begin{pmatrix}
0.582428\\
0.127878\\
0.083204\\
0.206490
\end{pmatrix} =
0.999656\cdot
\begin{pmatrix}
\color{gr} 0.582628\color{black} \\
0.127922\\
0.083233\\
0.206561
\end{pmatrix},
\end{equation*}
\begin{equation*}
\left[ \frac{{w}^{\prime}_i}{{w}^{\prime}_j} \right] =
\begin{pmatrix}
$\,\,$ 1 $\,\,$ & $\,\,$\color{gr} 4.5546\color{black} $\,\,$ & $\,\,$\color{gr} \color{blue} 7\color{black} $\,\,$ & $\,\,$\color{gr} 2.8206\color{black} $\,\,$ \\
$\,\,$\color{gr} 0.2196\color{black} $\,\,$ & $\,\,$ 1 $\,\,$ & $\,\,$1.5369$\,\,$ & $\,\,$0.6193  $\,\,$ \\
$\,\,$\color{gr} \color{blue}  1/7\color{black} $\,\,$ & $\,\,$0.6507$\,\,$ & $\,\,$ 1 $\,\,$ & $\,\,$0.4029 $\,\,$ \\
$\,\,$\color{gr} 0.3545\color{black} $\,\,$ & $\,\,$1.6147$\,\,$ & $\,\,$2.4817$\,\,$ & $\,\,$ 1  $\,\,$ \\
\end{pmatrix},
\end{equation*}
\end{example}
\newpage
\begin{example}
\begin{equation*}
\mathbf{A} =
\begin{pmatrix}
$\,\,$ 1 $\,\,$ & $\,\,$5$\,\,$ & $\,\,$8$\,\,$ & $\,\,$1 $\,\,$ \\
$\,\,$ 1/5$\,\,$ & $\,\,$ 1 $\,\,$ & $\,\,$3$\,\,$ & $\,\,$1 $\,\,$ \\
$\,\,$ 1/8$\,\,$ & $\,\,$ 1/3$\,\,$ & $\,\,$ 1 $\,\,$ & $\,\,$ 1/5 $\,\,$ \\
$\,\,$ 1 $\,\,$ & $\,\,$ 1 $\,\,$ & $\,\,$5$\,\,$ & $\,\,$ 1  $\,\,$ \\
\end{pmatrix},
\qquad
\lambda_{\max} =
4.2259,
\qquad
CR = 0.0852
\end{equation*}

\begin{equation*}
\mathbf{w}^{cos} =
\begin{pmatrix}
0.461721\\
0.183543\\
\color{red} 0.055738\color{black} \\
0.298998
\end{pmatrix}\end{equation*}
\begin{equation*}
\left[ \frac{{w}^{cos}_i}{{w}^{cos}_j} \right] =
\begin{pmatrix}
$\,\,$ 1 $\,\,$ & $\,\,$2.5156$\,\,$ & $\,\,$\color{red} 8.2838\color{black} $\,\,$ & $\,\,$1.5442$\,\,$ \\
$\,\,$0.3975$\,\,$ & $\,\,$ 1 $\,\,$ & $\,\,$\color{red} 3.2930\color{black} $\,\,$ & $\,\,$0.6139  $\,\,$ \\
$\,\,$\color{red} 0.1207\color{black} $\,\,$ & $\,\,$\color{red} 0.3037\color{black} $\,\,$ & $\,\,$ 1 $\,\,$ & $\,\,$\color{red} 0.1864\color{black}  $\,\,$ \\
$\,\,$0.6476$\,\,$ & $\,\,$1.6290$\,\,$ & $\,\,$\color{red} 5.3644\color{black} $\,\,$ & $\,\,$ 1  $\,\,$ \\
\end{pmatrix},
\end{equation*}

\begin{equation*}
\mathbf{w}^{\prime} =
\begin{pmatrix}
0.460810\\
0.183181\\
0.057601\\
0.298408
\end{pmatrix} =
0.998026\cdot
\begin{pmatrix}
0.461721\\
0.183543\\
\color{gr} 0.057715\color{black} \\
0.298998
\end{pmatrix},
\end{equation*}
\begin{equation*}
\left[ \frac{{w}^{\prime}_i}{{w}^{\prime}_j} \right] =
\begin{pmatrix}
$\,\,$ 1 $\,\,$ & $\,\,$2.5156$\,\,$ & $\,\,$\color{gr} \color{blue} 8\color{black} $\,\,$ & $\,\,$1.5442$\,\,$ \\
$\,\,$0.3975$\,\,$ & $\,\,$ 1 $\,\,$ & $\,\,$\color{gr} 3.1802\color{black} $\,\,$ & $\,\,$0.6139  $\,\,$ \\
$\,\,$\color{gr} \color{blue}  1/8\color{black} $\,\,$ & $\,\,$\color{gr} 0.3144\color{black} $\,\,$ & $\,\,$ 1 $\,\,$ & $\,\,$\color{gr} 0.1930\color{black}  $\,\,$ \\
$\,\,$0.6476$\,\,$ & $\,\,$1.6290$\,\,$ & $\,\,$\color{gr} 5.1806\color{black} $\,\,$ & $\,\,$ 1  $\,\,$ \\
\end{pmatrix},
\end{equation*}
\end{example}
\newpage
\begin{example}
\begin{equation*}
\mathbf{A} =
\begin{pmatrix}
$\,\,$ 1 $\,\,$ & $\,\,$5$\,\,$ & $\,\,$8$\,\,$ & $\,\,$3 $\,\,$ \\
$\,\,$ 1/5$\,\,$ & $\,\,$ 1 $\,\,$ & $\,\,$1$\,\,$ & $\,\,$1 $\,\,$ \\
$\,\,$ 1/8$\,\,$ & $\,\,$ 1 $\,\,$ & $\,\,$ 1 $\,\,$ & $\,\,$ 1/4 $\,\,$ \\
$\,\,$ 1/3$\,\,$ & $\,\,$ 1 $\,\,$ & $\,\,$4$\,\,$ & $\,\,$ 1  $\,\,$ \\
\end{pmatrix},
\qquad
\lambda_{\max} =
4.1655,
\qquad
CR = 0.0624
\end{equation*}

\begin{equation*}
\mathbf{w}^{cos} =
\begin{pmatrix}
\color{red} 0.592409\color{black} \\
0.127232\\
0.079445\\
0.200914
\end{pmatrix}\end{equation*}
\begin{equation*}
\left[ \frac{{w}^{cos}_i}{{w}^{cos}_j} \right] =
\begin{pmatrix}
$\,\,$ 1 $\,\,$ & $\,\,$\color{red} 4.6561\color{black} $\,\,$ & $\,\,$\color{red} 7.4569\color{black} $\,\,$ & $\,\,$\color{red} 2.9486\color{black} $\,\,$ \\
$\,\,$\color{red} 0.2148\color{black} $\,\,$ & $\,\,$ 1 $\,\,$ & $\,\,$1.6015$\,\,$ & $\,\,$0.6333  $\,\,$ \\
$\,\,$\color{red} 0.1341\color{black} $\,\,$ & $\,\,$0.6244$\,\,$ & $\,\,$ 1 $\,\,$ & $\,\,$0.3954 $\,\,$ \\
$\,\,$\color{red} 0.3391\color{black} $\,\,$ & $\,\,$1.5791$\,\,$ & $\,\,$2.5290$\,\,$ & $\,\,$ 1  $\,\,$ \\
\end{pmatrix},
\end{equation*}

\begin{equation*}
\mathbf{w}^{\prime} =
\begin{pmatrix}
0.596577\\
0.125931\\
0.078632\\
0.198859
\end{pmatrix} =
0.989774\cdot
\begin{pmatrix}
\color{gr} 0.602741\color{black} \\
0.127232\\
0.079445\\
0.200914
\end{pmatrix},
\end{equation*}
\begin{equation*}
\left[ \frac{{w}^{\prime}_i}{{w}^{\prime}_j} \right] =
\begin{pmatrix}
$\,\,$ 1 $\,\,$ & $\,\,$\color{gr} 4.7373\color{black} $\,\,$ & $\,\,$\color{gr} 7.5869\color{black} $\,\,$ & $\,\,$\color{gr} \color{blue} 3\color{black} $\,\,$ \\
$\,\,$\color{gr} 0.2111\color{black} $\,\,$ & $\,\,$ 1 $\,\,$ & $\,\,$1.6015$\,\,$ & $\,\,$0.6333  $\,\,$ \\
$\,\,$\color{gr} 0.1318\color{black} $\,\,$ & $\,\,$0.6244$\,\,$ & $\,\,$ 1 $\,\,$ & $\,\,$0.3954 $\,\,$ \\
$\,\,$\color{gr} \color{blue}  1/3\color{black} $\,\,$ & $\,\,$1.5791$\,\,$ & $\,\,$2.5290$\,\,$ & $\,\,$ 1  $\,\,$ \\
\end{pmatrix},
\end{equation*}
\end{example}
\newpage
\begin{example}
\begin{equation*}
\mathbf{A} =
\begin{pmatrix}
$\,\,$ 1 $\,\,$ & $\,\,$5$\,\,$ & $\,\,$8$\,\,$ & $\,\,$3 $\,\,$ \\
$\,\,$ 1/5$\,\,$ & $\,\,$ 1 $\,\,$ & $\,\,$1$\,\,$ & $\,\,$1 $\,\,$ \\
$\,\,$ 1/8$\,\,$ & $\,\,$ 1 $\,\,$ & $\,\,$ 1 $\,\,$ & $\,\,$ 1/5 $\,\,$ \\
$\,\,$ 1/3$\,\,$ & $\,\,$ 1 $\,\,$ & $\,\,$5$\,\,$ & $\,\,$ 1  $\,\,$ \\
\end{pmatrix},
\qquad
\lambda_{\max} =
4.2259,
\qquad
CR = 0.0852
\end{equation*}

\begin{equation*}
\mathbf{w}^{cos} =
\begin{pmatrix}
\color{red} 0.584070\color{black} \\
0.126110\\
0.075952\\
0.213869
\end{pmatrix}\end{equation*}
\begin{equation*}
\left[ \frac{{w}^{cos}_i}{{w}^{cos}_j} \right] =
\begin{pmatrix}
$\,\,$ 1 $\,\,$ & $\,\,$\color{red} 4.6315\color{black} $\,\,$ & $\,\,$\color{red} 7.6900\color{black} $\,\,$ & $\,\,$\color{red} 2.7310\color{black} $\,\,$ \\
$\,\,$\color{red} 0.2159\color{black} $\,\,$ & $\,\,$ 1 $\,\,$ & $\,\,$1.6604$\,\,$ & $\,\,$0.5897  $\,\,$ \\
$\,\,$\color{red} 0.1300\color{black} $\,\,$ & $\,\,$0.6023$\,\,$ & $\,\,$ 1 $\,\,$ & $\,\,$0.3551 $\,\,$ \\
$\,\,$\color{red} 0.3662\color{black} $\,\,$ & $\,\,$1.6959$\,\,$ & $\,\,$2.8159$\,\,$ & $\,\,$ 1  $\,\,$ \\
\end{pmatrix},
\end{equation*}

\begin{equation*}
\mathbf{w}^{\prime} =
\begin{pmatrix}
0.593637\\
0.123209\\
0.074205\\
0.208950
\end{pmatrix} =
0.977000\cdot
\begin{pmatrix}
\color{gr} 0.607612\color{black} \\
0.126110\\
0.075952\\
0.213869
\end{pmatrix},
\end{equation*}
\begin{equation*}
\left[ \frac{{w}^{\prime}_i}{{w}^{\prime}_j} \right] =
\begin{pmatrix}
$\,\,$ 1 $\,\,$ & $\,\,$\color{gr} 4.8181\color{black} $\,\,$ & $\,\,$\color{gr} \color{blue} 8\color{black} $\,\,$ & $\,\,$\color{gr} 2.8411\color{black} $\,\,$ \\
$\,\,$\color{gr} 0.2075\color{black} $\,\,$ & $\,\,$ 1 $\,\,$ & $\,\,$1.6604$\,\,$ & $\,\,$0.5897  $\,\,$ \\
$\,\,$\color{gr} \color{blue}  1/8\color{black} $\,\,$ & $\,\,$0.6023$\,\,$ & $\,\,$ 1 $\,\,$ & $\,\,$0.3551 $\,\,$ \\
$\,\,$\color{gr} 0.3520\color{black} $\,\,$ & $\,\,$1.6959$\,\,$ & $\,\,$2.8159$\,\,$ & $\,\,$ 1  $\,\,$ \\
\end{pmatrix},
\end{equation*}
\end{example}
\newpage
\begin{example}
\begin{equation*}
\mathbf{A} =
\begin{pmatrix}
$\,\,$ 1 $\,\,$ & $\,\,$5$\,\,$ & $\,\,$8$\,\,$ & $\,\,$3 $\,\,$ \\
$\,\,$ 1/5$\,\,$ & $\,\,$ 1 $\,\,$ & $\,\,$3$\,\,$ & $\,\,$3 $\,\,$ \\
$\,\,$ 1/8$\,\,$ & $\,\,$ 1/3$\,\,$ & $\,\,$ 1 $\,\,$ & $\,\,$ 1/2 $\,\,$ \\
$\,\,$ 1/3$\,\,$ & $\,\,$ 1/3$\,\,$ & $\,\,$2$\,\,$ & $\,\,$ 1  $\,\,$ \\
\end{pmatrix},
\qquad
\lambda_{\max} =
4.2311,
\qquad
CR = 0.0871
\end{equation*}

\begin{equation*}
\mathbf{w}^{cos} =
\begin{pmatrix}
0.569461\\
0.228954\\
\color{red} 0.066540\color{black} \\
0.135046
\end{pmatrix}\end{equation*}
\begin{equation*}
\left[ \frac{{w}^{cos}_i}{{w}^{cos}_j} \right] =
\begin{pmatrix}
$\,\,$ 1 $\,\,$ & $\,\,$2.4872$\,\,$ & $\,\,$\color{red} 8.5582\color{black} $\,\,$ & $\,\,$4.2168$\,\,$ \\
$\,\,$0.4021$\,\,$ & $\,\,$ 1 $\,\,$ & $\,\,$\color{red} 3.4408\color{black} $\,\,$ & $\,\,$1.6954  $\,\,$ \\
$\,\,$\color{red} 0.1168\color{black} $\,\,$ & $\,\,$\color{red} 0.2906\color{black} $\,\,$ & $\,\,$ 1 $\,\,$ & $\,\,$\color{red} 0.4927\color{black}  $\,\,$ \\
$\,\,$0.2371$\,\,$ & $\,\,$0.5898$\,\,$ & $\,\,$\color{red} 2.0295\color{black} $\,\,$ & $\,\,$ 1  $\,\,$ \\
\end{pmatrix},
\end{equation*}

\begin{equation*}
\mathbf{w}^{\prime} =
\begin{pmatrix}
0.568901\\
0.228729\\
0.067457\\
0.134913
\end{pmatrix} =
0.999018\cdot
\begin{pmatrix}
0.569461\\
0.228954\\
\color{gr} 0.067523\color{black} \\
0.135046
\end{pmatrix},
\end{equation*}
\begin{equation*}
\left[ \frac{{w}^{\prime}_i}{{w}^{\prime}_j} \right] =
\begin{pmatrix}
$\,\,$ 1 $\,\,$ & $\,\,$2.4872$\,\,$ & $\,\,$\color{gr} 8.4336\color{black} $\,\,$ & $\,\,$4.2168$\,\,$ \\
$\,\,$0.4021$\,\,$ & $\,\,$ 1 $\,\,$ & $\,\,$\color{gr} 3.3908\color{black} $\,\,$ & $\,\,$1.6954  $\,\,$ \\
$\,\,$\color{gr} 0.1186\color{black} $\,\,$ & $\,\,$\color{gr} 0.2949\color{black} $\,\,$ & $\,\,$ 1 $\,\,$ & $\,\,$\color{gr} \color{blue}  1/2\color{black}  $\,\,$ \\
$\,\,$0.2371$\,\,$ & $\,\,$0.5898$\,\,$ & $\,\,$\color{gr} \color{blue} 2\color{black} $\,\,$ & $\,\,$ 1  $\,\,$ \\
\end{pmatrix},
\end{equation*}
\end{example}
\newpage
\begin{example}
\begin{equation*}
\mathbf{A} =
\begin{pmatrix}
$\,\,$ 1 $\,\,$ & $\,\,$5$\,\,$ & $\,\,$8$\,\,$ & $\,\,$6 $\,\,$ \\
$\,\,$ 1/5$\,\,$ & $\,\,$ 1 $\,\,$ & $\,\,$1$\,\,$ & $\,\,$2 $\,\,$ \\
$\,\,$ 1/8$\,\,$ & $\,\,$ 1 $\,\,$ & $\,\,$ 1 $\,\,$ & $\,\,$ 1/2 $\,\,$ \\
$\,\,$ 1/6$\,\,$ & $\,\,$ 1/2$\,\,$ & $\,\,$2$\,\,$ & $\,\,$ 1  $\,\,$ \\
\end{pmatrix},
\qquad
\lambda_{\max} =
4.1655,
\qquad
CR = 0.0624
\end{equation*}

\begin{equation*}
\mathbf{w}^{cos} =
\begin{pmatrix}
\color{red} 0.658615\color{black} \\
0.140861\\
0.088018\\
0.112506
\end{pmatrix}\end{equation*}
\begin{equation*}
\left[ \frac{{w}^{cos}_i}{{w}^{cos}_j} \right] =
\begin{pmatrix}
$\,\,$ 1 $\,\,$ & $\,\,$\color{red} 4.6756\color{black} $\,\,$ & $\,\,$\color{red} 7.4827\color{black} $\,\,$ & $\,\,$\color{red} 5.8541\color{black} $\,\,$ \\
$\,\,$\color{red} 0.2139\color{black} $\,\,$ & $\,\,$ 1 $\,\,$ & $\,\,$1.6004$\,\,$ & $\,\,$1.2520  $\,\,$ \\
$\,\,$\color{red} 0.1336\color{black} $\,\,$ & $\,\,$0.6249$\,\,$ & $\,\,$ 1 $\,\,$ & $\,\,$0.7823 $\,\,$ \\
$\,\,$\color{red} 0.1708\color{black} $\,\,$ & $\,\,$0.7987$\,\,$ & $\,\,$1.2782$\,\,$ & $\,\,$ 1  $\,\,$ \\
\end{pmatrix},
\end{equation*}

\begin{equation*}
\mathbf{w}^{\prime} =
\begin{pmatrix}
0.664130\\
0.138585\\
0.086596\\
0.110688
\end{pmatrix} =
0.983845\cdot
\begin{pmatrix}
\color{gr} 0.675035\color{black} \\
0.140861\\
0.088018\\
0.112506
\end{pmatrix},
\end{equation*}
\begin{equation*}
\left[ \frac{{w}^{\prime}_i}{{w}^{\prime}_j} \right] =
\begin{pmatrix}
$\,\,$ 1 $\,\,$ & $\,\,$\color{gr} 4.7922\color{black} $\,\,$ & $\,\,$\color{gr} 7.6693\color{black} $\,\,$ & $\,\,$\color{gr} \color{blue} 6\color{black} $\,\,$ \\
$\,\,$\color{gr} 0.2087\color{black} $\,\,$ & $\,\,$ 1 $\,\,$ & $\,\,$1.6004$\,\,$ & $\,\,$1.2520  $\,\,$ \\
$\,\,$\color{gr} 0.1304\color{black} $\,\,$ & $\,\,$0.6249$\,\,$ & $\,\,$ 1 $\,\,$ & $\,\,$0.7823 $\,\,$ \\
$\,\,$\color{gr} \color{blue}  1/6\color{black} $\,\,$ & $\,\,$0.7987$\,\,$ & $\,\,$1.2782$\,\,$ & $\,\,$ 1  $\,\,$ \\
\end{pmatrix},
\end{equation*}
\end{example}
\newpage
\begin{example}
\begin{equation*}
\mathbf{A} =
\begin{pmatrix}
$\,\,$ 1 $\,\,$ & $\,\,$5$\,\,$ & $\,\,$8$\,\,$ & $\,\,$7 $\,\,$ \\
$\,\,$ 1/5$\,\,$ & $\,\,$ 1 $\,\,$ & $\,\,$1$\,\,$ & $\,\,$2 $\,\,$ \\
$\,\,$ 1/8$\,\,$ & $\,\,$ 1 $\,\,$ & $\,\,$ 1 $\,\,$ & $\,\,$ 1/2 $\,\,$ \\
$\,\,$ 1/7$\,\,$ & $\,\,$ 1/2$\,\,$ & $\,\,$2$\,\,$ & $\,\,$ 1  $\,\,$ \\
\end{pmatrix},
\qquad
\lambda_{\max} =
4.1665,
\qquad
CR = 0.0628
\end{equation*}

\begin{equation*}
\mathbf{w}^{cos} =
\begin{pmatrix}
\color{red} 0.670283\color{black} \\
0.135763\\
0.087449\\
0.106504
\end{pmatrix}\end{equation*}
\begin{equation*}
\left[ \frac{{w}^{cos}_i}{{w}^{cos}_j} \right] =
\begin{pmatrix}
$\,\,$ 1 $\,\,$ & $\,\,$\color{red} 4.9371\color{black} $\,\,$ & $\,\,$\color{red} 7.6649\color{black} $\,\,$ & $\,\,$\color{red} 6.2935\color{black} $\,\,$ \\
$\,\,$\color{red} 0.2025\color{black} $\,\,$ & $\,\,$ 1 $\,\,$ & $\,\,$1.5525$\,\,$ & $\,\,$1.2747  $\,\,$ \\
$\,\,$\color{red} 0.1305\color{black} $\,\,$ & $\,\,$0.6441$\,\,$ & $\,\,$ 1 $\,\,$ & $\,\,$0.8211 $\,\,$ \\
$\,\,$\color{red} 0.1589\color{black} $\,\,$ & $\,\,$0.7845$\,\,$ & $\,\,$1.2179$\,\,$ & $\,\,$ 1  $\,\,$ \\
\end{pmatrix},
\end{equation*}

\begin{equation*}
\mathbf{w}^{\prime} =
\begin{pmatrix}
0.673073\\
0.134615\\
0.086709\\
0.105603
\end{pmatrix} =
0.991539\cdot
\begin{pmatrix}
\color{gr} 0.678817\color{black} \\
0.135763\\
0.087449\\
0.106504
\end{pmatrix},
\end{equation*}
\begin{equation*}
\left[ \frac{{w}^{\prime}_i}{{w}^{\prime}_j} \right] =
\begin{pmatrix}
$\,\,$ 1 $\,\,$ & $\,\,$\color{gr} \color{blue} 5\color{black} $\,\,$ & $\,\,$\color{gr} 7.7624\color{black} $\,\,$ & $\,\,$\color{gr} 6.3736\color{black} $\,\,$ \\
$\,\,$\color{gr} \color{blue}  1/5\color{black} $\,\,$ & $\,\,$ 1 $\,\,$ & $\,\,$1.5525$\,\,$ & $\,\,$1.2747  $\,\,$ \\
$\,\,$\color{gr} 0.1288\color{black} $\,\,$ & $\,\,$0.6441$\,\,$ & $\,\,$ 1 $\,\,$ & $\,\,$0.8211 $\,\,$ \\
$\,\,$\color{gr} 0.1569\color{black} $\,\,$ & $\,\,$0.7845$\,\,$ & $\,\,$1.2179$\,\,$ & $\,\,$ 1  $\,\,$ \\
\end{pmatrix},
\end{equation*}
\end{example}
\newpage
\begin{example}
\begin{equation*}
\mathbf{A} =
\begin{pmatrix}
$\,\,$ 1 $\,\,$ & $\,\,$5$\,\,$ & $\,\,$8$\,\,$ & $\,\,$7 $\,\,$ \\
$\,\,$ 1/5$\,\,$ & $\,\,$ 1 $\,\,$ & $\,\,$9$\,\,$ & $\,\,$3 $\,\,$ \\
$\,\,$ 1/8$\,\,$ & $\,\,$ 1/9$\,\,$ & $\,\,$ 1 $\,\,$ & $\,\,$ 1/2 $\,\,$ \\
$\,\,$ 1/7$\,\,$ & $\,\,$ 1/3$\,\,$ & $\,\,$2$\,\,$ & $\,\,$ 1  $\,\,$ \\
\end{pmatrix},
\qquad
\lambda_{\max} =
4.2649,
\qquad
CR = 0.0999
\end{equation*}

\begin{equation*}
\mathbf{w}^{cos} =
\begin{pmatrix}
0.603985\\
0.260739\\
0.049776\\
\color{red} 0.085500\color{black}
\end{pmatrix}\end{equation*}
\begin{equation*}
\left[ \frac{{w}^{cos}_i}{{w}^{cos}_j} \right] =
\begin{pmatrix}
$\,\,$ 1 $\,\,$ & $\,\,$2.3164$\,\,$ & $\,\,$12.1340$\,\,$ & $\,\,$\color{red} 7.0642\color{black} $\,\,$ \\
$\,\,$0.4317$\,\,$ & $\,\,$ 1 $\,\,$ & $\,\,$5.2382$\,\,$ & $\,\,$\color{red} 3.0496\color{black}   $\,\,$ \\
$\,\,$0.0824$\,\,$ & $\,\,$0.1909$\,\,$ & $\,\,$ 1 $\,\,$ & $\,\,$\color{red} 0.5822\color{black}  $\,\,$ \\
$\,\,$\color{red} 0.1416\color{black} $\,\,$ & $\,\,$\color{red} 0.3279\color{black} $\,\,$ & $\,\,$\color{red} 1.7177\color{black} $\,\,$ & $\,\,$ 1  $\,\,$ \\
\end{pmatrix},
\end{equation*}

\begin{equation*}
\mathbf{w}^{\prime} =
\begin{pmatrix}
0.603512\\
0.260534\\
0.049737\\
0.086216
\end{pmatrix} =
0.999217\cdot
\begin{pmatrix}
0.603985\\
0.260739\\
0.049776\\
\color{gr} 0.086284\color{black}
\end{pmatrix},
\end{equation*}
\begin{equation*}
\left[ \frac{{w}^{\prime}_i}{{w}^{\prime}_j} \right] =
\begin{pmatrix}
$\,\,$ 1 $\,\,$ & $\,\,$2.3164$\,\,$ & $\,\,$12.1340$\,\,$ & $\,\,$\color{gr} \color{blue} 7\color{black} $\,\,$ \\
$\,\,$0.4317$\,\,$ & $\,\,$ 1 $\,\,$ & $\,\,$5.2382$\,\,$ & $\,\,$\color{gr} 3.0219\color{black}   $\,\,$ \\
$\,\,$0.0824$\,\,$ & $\,\,$0.1909$\,\,$ & $\,\,$ 1 $\,\,$ & $\,\,$\color{gr} 0.5769\color{black}  $\,\,$ \\
$\,\,$\color{gr} \color{blue}  1/7\color{black} $\,\,$ & $\,\,$\color{gr} 0.3309\color{black} $\,\,$ & $\,\,$\color{gr} 1.7334\color{black} $\,\,$ & $\,\,$ 1  $\,\,$ \\
\end{pmatrix},
\end{equation*}
\end{example}
\newpage
\begin{example}
\begin{equation*}
\mathbf{A} =
\begin{pmatrix}
$\,\,$ 1 $\,\,$ & $\,\,$5$\,\,$ & $\,\,$9$\,\,$ & $\,\,$1 $\,\,$ \\
$\,\,$ 1/5$\,\,$ & $\,\,$ 1 $\,\,$ & $\,\,$3$\,\,$ & $\,\,$1 $\,\,$ \\
$\,\,$ 1/9$\,\,$ & $\,\,$ 1/3$\,\,$ & $\,\,$ 1 $\,\,$ & $\,\,$ 1/6 $\,\,$ \\
$\,\,$ 1 $\,\,$ & $\,\,$ 1 $\,\,$ & $\,\,$6$\,\,$ & $\,\,$ 1  $\,\,$ \\
\end{pmatrix},
\qquad
\lambda_{\max} =
4.2277,
\qquad
CR = 0.0859
\end{equation*}

\begin{equation*}
\mathbf{w}^{cos} =
\begin{pmatrix}
0.464440\\
0.179635\\
\color{red} 0.049957\color{black} \\
0.305967
\end{pmatrix}\end{equation*}
\begin{equation*}
\left[ \frac{{w}^{cos}_i}{{w}^{cos}_j} \right] =
\begin{pmatrix}
$\,\,$ 1 $\,\,$ & $\,\,$2.5855$\,\,$ & $\,\,$\color{red} 9.2967\color{black} $\,\,$ & $\,\,$1.5179$\,\,$ \\
$\,\,$0.3868$\,\,$ & $\,\,$ 1 $\,\,$ & $\,\,$\color{red} 3.5958\color{black} $\,\,$ & $\,\,$0.5871  $\,\,$ \\
$\,\,$\color{red} 0.1076\color{black} $\,\,$ & $\,\,$\color{red} 0.2781\color{black} $\,\,$ & $\,\,$ 1 $\,\,$ & $\,\,$\color{red} 0.1633\color{black}  $\,\,$ \\
$\,\,$0.6588$\,\,$ & $\,\,$1.7033$\,\,$ & $\,\,$\color{red} 6.1246\color{black} $\,\,$ & $\,\,$ 1  $\,\,$ \\
\end{pmatrix},
\end{equation*}

\begin{equation*}
\mathbf{w}^{\prime} =
\begin{pmatrix}
0.463959\\
0.179449\\
0.050942\\
0.305650
\end{pmatrix} =
0.998964\cdot
\begin{pmatrix}
0.464440\\
0.179635\\
\color{gr} 0.050995\color{black} \\
0.305967
\end{pmatrix},
\end{equation*}
\begin{equation*}
\left[ \frac{{w}^{\prime}_i}{{w}^{\prime}_j} \right] =
\begin{pmatrix}
$\,\,$ 1 $\,\,$ & $\,\,$2.5855$\,\,$ & $\,\,$\color{gr} 9.1076\color{black} $\,\,$ & $\,\,$1.5179$\,\,$ \\
$\,\,$0.3868$\,\,$ & $\,\,$ 1 $\,\,$ & $\,\,$\color{gr} 3.5226\color{black} $\,\,$ & $\,\,$0.5871  $\,\,$ \\
$\,\,$\color{gr} 0.1098\color{black} $\,\,$ & $\,\,$\color{gr} 0.2839\color{black} $\,\,$ & $\,\,$ 1 $\,\,$ & $\,\,$\color{gr} \color{blue}  1/6\color{black}  $\,\,$ \\
$\,\,$0.6588$\,\,$ & $\,\,$1.7033$\,\,$ & $\,\,$\color{gr} \color{blue} 6\color{black} $\,\,$ & $\,\,$ 1  $\,\,$ \\
\end{pmatrix},
\end{equation*}
\end{example}
\newpage
\begin{example}
\begin{equation*}
\mathbf{A} =
\begin{pmatrix}
$\,\,$ 1 $\,\,$ & $\,\,$5$\,\,$ & $\,\,$9$\,\,$ & $\,\,$1 $\,\,$ \\
$\,\,$ 1/5$\,\,$ & $\,\,$ 1 $\,\,$ & $\,\,$4$\,\,$ & $\,\,$1 $\,\,$ \\
$\,\,$ 1/9$\,\,$ & $\,\,$ 1/4$\,\,$ & $\,\,$ 1 $\,\,$ & $\,\,$ 1/6 $\,\,$ \\
$\,\,$ 1 $\,\,$ & $\,\,$ 1 $\,\,$ & $\,\,$6$\,\,$ & $\,\,$ 1  $\,\,$ \\
\end{pmatrix},
\qquad
\lambda_{\max} =
4.2316,
\qquad
CR = 0.0873
\end{equation*}

\begin{equation*}
\mathbf{w}^{cos} =
\begin{pmatrix}
0.459433\\
0.191026\\
\color{red} 0.046958\color{black} \\
0.302583
\end{pmatrix}\end{equation*}
\begin{equation*}
\left[ \frac{{w}^{cos}_i}{{w}^{cos}_j} \right] =
\begin{pmatrix}
$\,\,$ 1 $\,\,$ & $\,\,$2.4051$\,\,$ & $\,\,$\color{red} 9.7839\color{black} $\,\,$ & $\,\,$1.5184$\,\,$ \\
$\,\,$0.4158$\,\,$ & $\,\,$ 1 $\,\,$ & $\,\,$\color{red} 4.0680\color{black} $\,\,$ & $\,\,$0.6313  $\,\,$ \\
$\,\,$\color{red} 0.1022\color{black} $\,\,$ & $\,\,$\color{red} 0.2458\color{black} $\,\,$ & $\,\,$ 1 $\,\,$ & $\,\,$\color{red} 0.1552\color{black}  $\,\,$ \\
$\,\,$0.6586$\,\,$ & $\,\,$1.5840$\,\,$ & $\,\,$\color{red} 6.4437\color{black} $\,\,$ & $\,\,$ 1  $\,\,$ \\
\end{pmatrix},
\end{equation*}

\begin{equation*}
\mathbf{w}^{\prime} =
\begin{pmatrix}
0.459066\\
0.190874\\
0.047718\\
0.302342
\end{pmatrix} =
0.999202\cdot
\begin{pmatrix}
0.459433\\
0.191026\\
\color{gr} 0.047756\color{black} \\
0.302583
\end{pmatrix},
\end{equation*}
\begin{equation*}
\left[ \frac{{w}^{\prime}_i}{{w}^{\prime}_j} \right] =
\begin{pmatrix}
$\,\,$ 1 $\,\,$ & $\,\,$2.4051$\,\,$ & $\,\,$\color{gr} 9.6203\color{black} $\,\,$ & $\,\,$1.5184$\,\,$ \\
$\,\,$0.4158$\,\,$ & $\,\,$ 1 $\,\,$ & $\,\,$\color{gr} \color{blue} 4\color{black} $\,\,$ & $\,\,$0.6313  $\,\,$ \\
$\,\,$\color{gr} 0.1039\color{black} $\,\,$ & $\,\,$\color{gr} \color{blue}  1/4\color{black} $\,\,$ & $\,\,$ 1 $\,\,$ & $\,\,$\color{gr} 0.1578\color{black}  $\,\,$ \\
$\,\,$0.6586$\,\,$ & $\,\,$1.5840$\,\,$ & $\,\,$\color{gr} 6.3360\color{black} $\,\,$ & $\,\,$ 1  $\,\,$ \\
\end{pmatrix},
\end{equation*}
\end{example}
\newpage
\begin{example}
\begin{equation*}
\mathbf{A} =
\begin{pmatrix}
$\,\,$ 1 $\,\,$ & $\,\,$5$\,\,$ & $\,\,$9$\,\,$ & $\,\,$1 $\,\,$ \\
$\,\,$ 1/5$\,\,$ & $\,\,$ 1 $\,\,$ & $\,\,$4$\,\,$ & $\,\,$1 $\,\,$ \\
$\,\,$ 1/9$\,\,$ & $\,\,$ 1/4$\,\,$ & $\,\,$ 1 $\,\,$ & $\,\,$ 1/7 $\,\,$ \\
$\,\,$ 1 $\,\,$ & $\,\,$ 1 $\,\,$ & $\,\,$7$\,\,$ & $\,\,$ 1  $\,\,$ \\
\end{pmatrix},
\qquad
\lambda_{\max} =
4.2346,
\qquad
CR = 0.0885
\end{equation*}

\begin{equation*}
\mathbf{w}^{cos} =
\begin{pmatrix}
0.454671\\
0.188954\\
\color{red} 0.044338\color{black} \\
0.312037
\end{pmatrix}\end{equation*}
\begin{equation*}
\left[ \frac{{w}^{cos}_i}{{w}^{cos}_j} \right] =
\begin{pmatrix}
$\,\,$ 1 $\,\,$ & $\,\,$2.4062$\,\,$ & $\,\,$\color{red} 10.2547\color{black} $\,\,$ & $\,\,$1.4571$\,\,$ \\
$\,\,$0.4156$\,\,$ & $\,\,$ 1 $\,\,$ & $\,\,$\color{red} 4.2617\color{black} $\,\,$ & $\,\,$0.6056  $\,\,$ \\
$\,\,$\color{red} 0.0975\color{black} $\,\,$ & $\,\,$\color{red} 0.2346\color{black} $\,\,$ & $\,\,$ 1 $\,\,$ & $\,\,$\color{red} 0.1421\color{black}  $\,\,$ \\
$\,\,$0.6863$\,\,$ & $\,\,$1.6514$\,\,$ & $\,\,$\color{red} 7.0377\color{black} $\,\,$ & $\,\,$ 1  $\,\,$ \\
\end{pmatrix},
\end{equation*}

\begin{equation*}
\mathbf{w}^{\prime} =
\begin{pmatrix}
0.454562\\
0.188909\\
0.044566\\
0.311962
\end{pmatrix} =
0.999761\cdot
\begin{pmatrix}
0.454671\\
0.188954\\
\color{gr} 0.044577\color{black} \\
0.312037
\end{pmatrix},
\end{equation*}
\begin{equation*}
\left[ \frac{{w}^{\prime}_i}{{w}^{\prime}_j} \right] =
\begin{pmatrix}
$\,\,$ 1 $\,\,$ & $\,\,$2.4062$\,\,$ & $\,\,$\color{gr} 10.1997\color{black} $\,\,$ & $\,\,$1.4571$\,\,$ \\
$\,\,$0.4156$\,\,$ & $\,\,$ 1 $\,\,$ & $\,\,$\color{gr} 4.2389\color{black} $\,\,$ & $\,\,$0.6056  $\,\,$ \\
$\,\,$\color{gr} 0.0980\color{black} $\,\,$ & $\,\,$\color{gr} 0.2359\color{black} $\,\,$ & $\,\,$ 1 $\,\,$ & $\,\,$\color{gr} \color{blue}  1/7\color{black}  $\,\,$ \\
$\,\,$0.6863$\,\,$ & $\,\,$1.6514$\,\,$ & $\,\,$\color{gr} \color{blue} 7\color{black} $\,\,$ & $\,\,$ 1  $\,\,$ \\
\end{pmatrix},
\end{equation*}
\end{example}
\newpage
\begin{example}
\begin{equation*}
\mathbf{A} =
\begin{pmatrix}
$\,\,$ 1 $\,\,$ & $\,\,$5$\,\,$ & $\,\,$9$\,\,$ & $\,\,$3 $\,\,$ \\
$\,\,$ 1/5$\,\,$ & $\,\,$ 1 $\,\,$ & $\,\,$1$\,\,$ & $\,\,$1 $\,\,$ \\
$\,\,$ 1/9$\,\,$ & $\,\,$ 1 $\,\,$ & $\,\,$ 1 $\,\,$ & $\,\,$ 1/5 $\,\,$ \\
$\,\,$ 1/3$\,\,$ & $\,\,$ 1 $\,\,$ & $\,\,$5$\,\,$ & $\,\,$ 1  $\,\,$ \\
\end{pmatrix},
\qquad
\lambda_{\max} =
4.2253,
\qquad
CR = 0.0849
\end{equation*}

\begin{equation*}
\mathbf{w}^{cos} =
\begin{pmatrix}
\color{red} 0.592885\color{black} \\
0.125645\\
0.073019\\
0.208451
\end{pmatrix}\end{equation*}
\begin{equation*}
\left[ \frac{{w}^{cos}_i}{{w}^{cos}_j} \right] =
\begin{pmatrix}
$\,\,$ 1 $\,\,$ & $\,\,$\color{red} 4.7187\color{black} $\,\,$ & $\,\,$\color{red} 8.1196\color{black} $\,\,$ & $\,\,$\color{red} 2.8442\color{black} $\,\,$ \\
$\,\,$\color{red} 0.2119\color{black} $\,\,$ & $\,\,$ 1 $\,\,$ & $\,\,$1.7207$\,\,$ & $\,\,$0.6028  $\,\,$ \\
$\,\,$\color{red} 0.1232\color{black} $\,\,$ & $\,\,$0.5812$\,\,$ & $\,\,$ 1 $\,\,$ & $\,\,$0.3503 $\,\,$ \\
$\,\,$\color{red} 0.3516\color{black} $\,\,$ & $\,\,$1.6590$\,\,$ & $\,\,$2.8548$\,\,$ & $\,\,$ 1  $\,\,$ \\
\end{pmatrix},
\end{equation*}

\begin{equation*}
\mathbf{w}^{\prime} =
\begin{pmatrix}
0.605688\\
0.121694\\
0.070723\\
0.201896
\end{pmatrix} =
0.968553\cdot
\begin{pmatrix}
\color{gr} 0.625353\color{black} \\
0.125645\\
0.073019\\
0.208451
\end{pmatrix},
\end{equation*}
\begin{equation*}
\left[ \frac{{w}^{\prime}_i}{{w}^{\prime}_j} \right] =
\begin{pmatrix}
$\,\,$ 1 $\,\,$ & $\,\,$\color{gr} 4.9771\color{black} $\,\,$ & $\,\,$\color{gr} 8.5643\color{black} $\,\,$ & $\,\,$\color{gr} \color{blue} 3\color{black} $\,\,$ \\
$\,\,$\color{gr} 0.2009\color{black} $\,\,$ & $\,\,$ 1 $\,\,$ & $\,\,$1.7207$\,\,$ & $\,\,$0.6028  $\,\,$ \\
$\,\,$\color{gr} 0.1168\color{black} $\,\,$ & $\,\,$0.5812$\,\,$ & $\,\,$ 1 $\,\,$ & $\,\,$0.3503 $\,\,$ \\
$\,\,$\color{gr} \color{blue}  1/3\color{black} $\,\,$ & $\,\,$1.6590$\,\,$ & $\,\,$2.8548$\,\,$ & $\,\,$ 1  $\,\,$ \\
\end{pmatrix},
\end{equation*}
\end{example}
\newpage
\begin{example}
\begin{equation*}
\mathbf{A} =
\begin{pmatrix}
$\,\,$ 1 $\,\,$ & $\,\,$5$\,\,$ & $\,\,$9$\,\,$ & $\,\,$3 $\,\,$ \\
$\,\,$ 1/5$\,\,$ & $\,\,$ 1 $\,\,$ & $\,\,$3$\,\,$ & $\,\,$3 $\,\,$ \\
$\,\,$ 1/9$\,\,$ & $\,\,$ 1/3$\,\,$ & $\,\,$ 1 $\,\,$ & $\,\,$ 1/2 $\,\,$ \\
$\,\,$ 1/3$\,\,$ & $\,\,$ 1/3$\,\,$ & $\,\,$2$\,\,$ & $\,\,$ 1  $\,\,$ \\
\end{pmatrix},
\qquad
\lambda_{\max} =
4.2277,
\qquad
CR = 0.0859
\end{equation*}

\begin{equation*}
\mathbf{w}^{cos} =
\begin{pmatrix}
0.577903\\
0.225893\\
\color{red} 0.063321\color{black} \\
0.132883
\end{pmatrix}\end{equation*}
\begin{equation*}
\left[ \frac{{w}^{cos}_i}{{w}^{cos}_j} \right] =
\begin{pmatrix}
$\,\,$ 1 $\,\,$ & $\,\,$2.5583$\,\,$ & $\,\,$\color{red} 9.1266\color{black} $\,\,$ & $\,\,$4.3489$\,\,$ \\
$\,\,$0.3909$\,\,$ & $\,\,$ 1 $\,\,$ & $\,\,$\color{red} 3.5674\color{black} $\,\,$ & $\,\,$1.6999  $\,\,$ \\
$\,\,$\color{red} 0.1096\color{black} $\,\,$ & $\,\,$\color{red} 0.2803\color{black} $\,\,$ & $\,\,$ 1 $\,\,$ & $\,\,$\color{red} 0.4765\color{black}  $\,\,$ \\
$\,\,$0.2299$\,\,$ & $\,\,$0.5883$\,\,$ & $\,\,$\color{red} 2.0986\color{black} $\,\,$ & $\,\,$ 1  $\,\,$ \\
\end{pmatrix},
\end{equation*}

\begin{equation*}
\mathbf{w}^{\prime} =
\begin{pmatrix}
0.577389\\
0.225692\\
0.064154\\
0.132765
\end{pmatrix} =
0.999110\cdot
\begin{pmatrix}
0.577903\\
0.225893\\
\color{gr} 0.064211\color{black} \\
0.132883
\end{pmatrix},
\end{equation*}
\begin{equation*}
\left[ \frac{{w}^{\prime}_i}{{w}^{\prime}_j} \right] =
\begin{pmatrix}
$\,\,$ 1 $\,\,$ & $\,\,$2.5583$\,\,$ & $\,\,$\color{gr} \color{blue} 9\color{black} $\,\,$ & $\,\,$4.3489$\,\,$ \\
$\,\,$0.3909$\,\,$ & $\,\,$ 1 $\,\,$ & $\,\,$\color{gr} 3.5180\color{black} $\,\,$ & $\,\,$1.6999  $\,\,$ \\
$\,\,$\color{gr} \color{blue}  1/9\color{black} $\,\,$ & $\,\,$\color{gr} 0.2843\color{black} $\,\,$ & $\,\,$ 1 $\,\,$ & $\,\,$\color{gr} 0.4832\color{black}  $\,\,$ \\
$\,\,$0.2299$\,\,$ & $\,\,$0.5883$\,\,$ & $\,\,$\color{gr} 2.0695\color{black} $\,\,$ & $\,\,$ 1  $\,\,$ \\
\end{pmatrix},
\end{equation*}
\end{example}
\newpage
\begin{example}
\begin{equation*}
\mathbf{A} =
\begin{pmatrix}
$\,\,$ 1 $\,\,$ & $\,\,$5$\,\,$ & $\,\,$9$\,\,$ & $\,\,$3 $\,\,$ \\
$\,\,$ 1/5$\,\,$ & $\,\,$ 1 $\,\,$ & $\,\,$4$\,\,$ & $\,\,$3 $\,\,$ \\
$\,\,$ 1/9$\,\,$ & $\,\,$ 1/4$\,\,$ & $\,\,$ 1 $\,\,$ & $\,\,$ 1/2 $\,\,$ \\
$\,\,$ 1/3$\,\,$ & $\,\,$ 1/3$\,\,$ & $\,\,$2$\,\,$ & $\,\,$ 1  $\,\,$ \\
\end{pmatrix},
\qquad
\lambda_{\max} =
4.2316,
\qquad
CR = 0.0873
\end{equation*}

\begin{equation*}
\mathbf{w}^{cos} =
\begin{pmatrix}
0.569911\\
0.239263\\
\color{red} 0.059763\color{black} \\
0.131062
\end{pmatrix}\end{equation*}
\begin{equation*}
\left[ \frac{{w}^{cos}_i}{{w}^{cos}_j} \right] =
\begin{pmatrix}
$\,\,$ 1 $\,\,$ & $\,\,$2.3819$\,\,$ & $\,\,$\color{red} 9.5362\color{black} $\,\,$ & $\,\,$4.3484$\,\,$ \\
$\,\,$0.4198$\,\,$ & $\,\,$ 1 $\,\,$ & $\,\,$\color{red} 4.0036\color{black} $\,\,$ & $\,\,$1.8256  $\,\,$ \\
$\,\,$\color{red} 0.1049\color{black} $\,\,$ & $\,\,$\color{red} 0.2498\color{black} $\,\,$ & $\,\,$ 1 $\,\,$ & $\,\,$\color{red} 0.4560\color{black}  $\,\,$ \\
$\,\,$0.2300$\,\,$ & $\,\,$0.5478$\,\,$ & $\,\,$\color{red} 2.1930\color{black} $\,\,$ & $\,\,$ 1  $\,\,$ \\
\end{pmatrix},
\end{equation*}

\begin{equation*}
\mathbf{w}^{\prime} =
\begin{pmatrix}
0.569881\\
0.239251\\
0.059813\\
0.131055
\end{pmatrix} =
0.999947\cdot
\begin{pmatrix}
0.569911\\
0.239263\\
\color{gr} 0.059816\color{black} \\
0.131062
\end{pmatrix},
\end{equation*}
\begin{equation*}
\left[ \frac{{w}^{\prime}_i}{{w}^{\prime}_j} \right] =
\begin{pmatrix}
$\,\,$ 1 $\,\,$ & $\,\,$2.3819$\,\,$ & $\,\,$\color{gr} 9.5278\color{black} $\,\,$ & $\,\,$4.3484$\,\,$ \\
$\,\,$0.4198$\,\,$ & $\,\,$ 1 $\,\,$ & $\,\,$\color{gr} \color{blue} 4\color{black} $\,\,$ & $\,\,$1.8256  $\,\,$ \\
$\,\,$\color{gr} 0.1050\color{black} $\,\,$ & $\,\,$\color{gr} \color{blue}  1/4\color{black} $\,\,$ & $\,\,$ 1 $\,\,$ & $\,\,$\color{gr} 0.4564\color{black}  $\,\,$ \\
$\,\,$0.2300$\,\,$ & $\,\,$0.5478$\,\,$ & $\,\,$\color{gr} 2.1911\color{black} $\,\,$ & $\,\,$ 1  $\,\,$ \\
\end{pmatrix},
\end{equation*}
\end{example}
\newpage
\begin{example}
\begin{equation*}
\mathbf{A} =
\begin{pmatrix}
$\,\,$ 1 $\,\,$ & $\,\,$5$\,\,$ & $\,\,$9$\,\,$ & $\,\,$7 $\,\,$ \\
$\,\,$ 1/5$\,\,$ & $\,\,$ 1 $\,\,$ & $\,\,$9$\,\,$ & $\,\,$3 $\,\,$ \\
$\,\,$ 1/9$\,\,$ & $\,\,$ 1/9$\,\,$ & $\,\,$ 1 $\,\,$ & $\,\,$ 1/2 $\,\,$ \\
$\,\,$ 1/7$\,\,$ & $\,\,$ 1/3$\,\,$ & $\,\,$2$\,\,$ & $\,\,$ 1  $\,\,$ \\
\end{pmatrix},
\qquad
\lambda_{\max} =
4.2300,
\qquad
CR = 0.0867
\end{equation*}

\begin{equation*}
\mathbf{w}^{cos} =
\begin{pmatrix}
0.613723\\
0.255015\\
0.046917\\
\color{red} 0.084344\color{black}
\end{pmatrix}\end{equation*}
\begin{equation*}
\left[ \frac{{w}^{cos}_i}{{w}^{cos}_j} \right] =
\begin{pmatrix}
$\,\,$ 1 $\,\,$ & $\,\,$2.4066$\,\,$ & $\,\,$13.0810$\,\,$ & $\,\,$\color{red} 7.2764\color{black} $\,\,$ \\
$\,\,$0.4155$\,\,$ & $\,\,$ 1 $\,\,$ & $\,\,$5.4354$\,\,$ & $\,\,$\color{red} 3.0235\color{black}   $\,\,$ \\
$\,\,$0.0764$\,\,$ & $\,\,$0.1840$\,\,$ & $\,\,$ 1 $\,\,$ & $\,\,$\color{red} 0.5563\color{black}  $\,\,$ \\
$\,\,$\color{red} 0.1374\color{black} $\,\,$ & $\,\,$\color{red} 0.3307\color{black} $\,\,$ & $\,\,$\color{red} 1.7977\color{black} $\,\,$ & $\,\,$ 1  $\,\,$ \\
\end{pmatrix},
\end{equation*}

\begin{equation*}
\mathbf{w}^{\prime} =
\begin{pmatrix}
0.613318\\
0.254847\\
0.046886\\
0.084949
\end{pmatrix} =
0.999340\cdot
\begin{pmatrix}
0.613723\\
0.255015\\
0.046917\\
\color{gr} 0.085005\color{black}
\end{pmatrix},
\end{equation*}
\begin{equation*}
\left[ \frac{{w}^{\prime}_i}{{w}^{\prime}_j} \right] =
\begin{pmatrix}
$\,\,$ 1 $\,\,$ & $\,\,$2.4066$\,\,$ & $\,\,$13.0810$\,\,$ & $\,\,$\color{gr} 7.2198\color{black} $\,\,$ \\
$\,\,$0.4155$\,\,$ & $\,\,$ 1 $\,\,$ & $\,\,$5.4354$\,\,$ & $\,\,$\color{gr} \color{blue} 3\color{black}   $\,\,$ \\
$\,\,$0.0764$\,\,$ & $\,\,$0.1840$\,\,$ & $\,\,$ 1 $\,\,$ & $\,\,$\color{gr} 0.5519\color{black}  $\,\,$ \\
$\,\,$\color{gr} 0.1385\color{black} $\,\,$ & $\,\,$\color{gr} \color{blue}  1/3\color{black} $\,\,$ & $\,\,$\color{gr} 1.8118\color{black} $\,\,$ & $\,\,$ 1  $\,\,$ \\
\end{pmatrix},
\end{equation*}
\end{example}
\newpage
\begin{example}
\begin{equation*}
\mathbf{A} =
\begin{pmatrix}
$\,\,$ 1 $\,\,$ & $\,\,$6$\,\,$ & $\,\,$8$\,\,$ & $\,\,$4 $\,\,$ \\
$\,\,$ 1/6$\,\,$ & $\,\,$ 1 $\,\,$ & $\,\,$1$\,\,$ & $\,\,$1 $\,\,$ \\
$\,\,$ 1/8$\,\,$ & $\,\,$ 1 $\,\,$ & $\,\,$ 1 $\,\,$ & $\,\,$ 1/3 $\,\,$ \\
$\,\,$ 1/4$\,\,$ & $\,\,$ 1 $\,\,$ & $\,\,$3$\,\,$ & $\,\,$ 1  $\,\,$ \\
\end{pmatrix},
\qquad
\lambda_{\max} =
4.1031,
\qquad
CR = 0.0389
\end{equation*}

\begin{equation*}
\mathbf{w}^{cos} =
\begin{pmatrix}
\color{red} 0.640276\color{black} \\
0.113469\\
0.080182\\
0.166074
\end{pmatrix}\end{equation*}
\begin{equation*}
\left[ \frac{{w}^{cos}_i}{{w}^{cos}_j} \right] =
\begin{pmatrix}
$\,\,$ 1 $\,\,$ & $\,\,$\color{red} 5.6428\color{black} $\,\,$ & $\,\,$\color{red} 7.9853\color{black} $\,\,$ & $\,\,$\color{red} 3.8554\color{black} $\,\,$ \\
$\,\,$\color{red} 0.1772\color{black} $\,\,$ & $\,\,$ 1 $\,\,$ & $\,\,$1.4151$\,\,$ & $\,\,$0.6832  $\,\,$ \\
$\,\,$\color{red} 0.1252\color{black} $\,\,$ & $\,\,$0.7066$\,\,$ & $\,\,$ 1 $\,\,$ & $\,\,$0.4828 $\,\,$ \\
$\,\,$\color{red} 0.2594\color{black} $\,\,$ & $\,\,$1.4636$\,\,$ & $\,\,$2.0712$\,\,$ & $\,\,$ 1  $\,\,$ \\
\end{pmatrix},
\end{equation*}

\begin{equation*}
\mathbf{w}^{\prime} =
\begin{pmatrix}
0.640700\\
0.113335\\
0.080087\\
0.165878
\end{pmatrix} =
0.998822\cdot
\begin{pmatrix}
\color{gr} 0.641455\color{black} \\
0.113469\\
0.080182\\
0.166074
\end{pmatrix},
\end{equation*}
\begin{equation*}
\left[ \frac{{w}^{\prime}_i}{{w}^{\prime}_j} \right] =
\begin{pmatrix}
$\,\,$ 1 $\,\,$ & $\,\,$\color{gr} 5.6532\color{black} $\,\,$ & $\,\,$\color{gr} \color{blue} 8\color{black} $\,\,$ & $\,\,$\color{gr} 3.8625\color{black} $\,\,$ \\
$\,\,$\color{gr} 0.1769\color{black} $\,\,$ & $\,\,$ 1 $\,\,$ & $\,\,$1.4151$\,\,$ & $\,\,$0.6832  $\,\,$ \\
$\,\,$\color{gr} \color{blue}  1/8\color{black} $\,\,$ & $\,\,$0.7066$\,\,$ & $\,\,$ 1 $\,\,$ & $\,\,$0.4828 $\,\,$ \\
$\,\,$\color{gr} 0.2589\color{black} $\,\,$ & $\,\,$1.4636$\,\,$ & $\,\,$2.0712$\,\,$ & $\,\,$ 1  $\,\,$ \\
\end{pmatrix},
\end{equation*}
\end{example}
\newpage
\begin{example}
\begin{equation*}
\mathbf{A} =
\begin{pmatrix}
$\,\,$ 1 $\,\,$ & $\,\,$6$\,\,$ & $\,\,$9$\,\,$ & $\,\,$3 $\,\,$ \\
$\,\,$ 1/6$\,\,$ & $\,\,$ 1 $\,\,$ & $\,\,$1$\,\,$ & $\,\,$1 $\,\,$ \\
$\,\,$ 1/9$\,\,$ & $\,\,$ 1 $\,\,$ & $\,\,$ 1 $\,\,$ & $\,\,$ 1/5 $\,\,$ \\
$\,\,$ 1/3$\,\,$ & $\,\,$ 1 $\,\,$ & $\,\,$5$\,\,$ & $\,\,$ 1  $\,\,$ \\
\end{pmatrix},
\qquad
\lambda_{\max} =
4.2277,
\qquad
CR = 0.0859
\end{equation*}

\begin{equation*}
\mathbf{w}^{cos} =
\begin{pmatrix}
\color{red} 0.605505\color{black} \\
0.117846\\
0.069469\\
0.207180
\end{pmatrix}\end{equation*}
\begin{equation*}
\left[ \frac{{w}^{cos}_i}{{w}^{cos}_j} \right] =
\begin{pmatrix}
$\,\,$ 1 $\,\,$ & $\,\,$\color{red} 5.1381\color{black} $\,\,$ & $\,\,$\color{red} 8.7162\color{black} $\,\,$ & $\,\,$\color{red} 2.9226\color{black} $\,\,$ \\
$\,\,$\color{red} 0.1946\color{black} $\,\,$ & $\,\,$ 1 $\,\,$ & $\,\,$1.6964$\,\,$ & $\,\,$0.5688  $\,\,$ \\
$\,\,$\color{red} 0.1147\color{black} $\,\,$ & $\,\,$0.5895$\,\,$ & $\,\,$ 1 $\,\,$ & $\,\,$0.3353 $\,\,$ \\
$\,\,$\color{red} 0.3422\color{black} $\,\,$ & $\,\,$1.7581$\,\,$ & $\,\,$2.9823$\,\,$ & $\,\,$ 1  $\,\,$ \\
\end{pmatrix},
\end{equation*}

\begin{equation*}
\mathbf{w}^{\prime} =
\begin{pmatrix}
0.611731\\
0.115986\\
0.068373\\
0.203910
\end{pmatrix} =
0.984219\cdot
\begin{pmatrix}
\color{gr} 0.621540\color{black} \\
0.117846\\
0.069469\\
0.207180
\end{pmatrix},
\end{equation*}
\begin{equation*}
\left[ \frac{{w}^{\prime}_i}{{w}^{\prime}_j} \right] =
\begin{pmatrix}
$\,\,$ 1 $\,\,$ & $\,\,$\color{gr} 5.2742\color{black} $\,\,$ & $\,\,$\color{gr} 8.9470\color{black} $\,\,$ & $\,\,$\color{gr} \color{blue} 3\color{black} $\,\,$ \\
$\,\,$\color{gr} 0.1896\color{black} $\,\,$ & $\,\,$ 1 $\,\,$ & $\,\,$1.6964$\,\,$ & $\,\,$0.5688  $\,\,$ \\
$\,\,$\color{gr} 0.1118\color{black} $\,\,$ & $\,\,$0.5895$\,\,$ & $\,\,$ 1 $\,\,$ & $\,\,$0.3353 $\,\,$ \\
$\,\,$\color{gr} \color{blue}  1/3\color{black} $\,\,$ & $\,\,$1.7581$\,\,$ & $\,\,$2.9823$\,\,$ & $\,\,$ 1  $\,\,$ \\
\end{pmatrix},
\end{equation*}
\end{example}
\newpage
\begin{example}
\begin{equation*}
\mathbf{A} =
\begin{pmatrix}
$\,\,$ 1 $\,\,$ & $\,\,$6$\,\,$ & $\,\,$9$\,\,$ & $\,\,$4 $\,\,$ \\
$\,\,$ 1/6$\,\,$ & $\,\,$ 1 $\,\,$ & $\,\,$1$\,\,$ & $\,\,$1 $\,\,$ \\
$\,\,$ 1/9$\,\,$ & $\,\,$ 1 $\,\,$ & $\,\,$ 1 $\,\,$ & $\,\,$ 1/4 $\,\,$ \\
$\,\,$ 1/4$\,\,$ & $\,\,$ 1 $\,\,$ & $\,\,$4$\,\,$ & $\,\,$ 1  $\,\,$ \\
\end{pmatrix},
\qquad
\lambda_{\max} =
4.1664,
\qquad
CR = 0.0627
\end{equation*}

\begin{equation*}
\mathbf{w}^{cos} =
\begin{pmatrix}
\color{red} 0.639879\color{black} \\
0.111444\\
0.072396\\
0.176281
\end{pmatrix}\end{equation*}
\begin{equation*}
\left[ \frac{{w}^{cos}_i}{{w}^{cos}_j} \right] =
\begin{pmatrix}
$\,\,$ 1 $\,\,$ & $\,\,$\color{red} 5.7417\color{black} $\,\,$ & $\,\,$\color{red} 8.8386\color{black} $\,\,$ & $\,\,$\color{red} 3.6299\color{black} $\,\,$ \\
$\,\,$\color{red} 0.1742\color{black} $\,\,$ & $\,\,$ 1 $\,\,$ & $\,\,$1.5394$\,\,$ & $\,\,$0.6322  $\,\,$ \\
$\,\,$\color{red} 0.1131\color{black} $\,\,$ & $\,\,$0.6496$\,\,$ & $\,\,$ 1 $\,\,$ & $\,\,$0.4107 $\,\,$ \\
$\,\,$\color{red} 0.2755\color{black} $\,\,$ & $\,\,$1.5818$\,\,$ & $\,\,$2.4350$\,\,$ & $\,\,$ 1  $\,\,$ \\
\end{pmatrix},
\end{equation*}

\begin{equation*}
\mathbf{w}^{\prime} =
\begin{pmatrix}
0.644038\\
0.110157\\
0.071560\\
0.174246
\end{pmatrix} =
0.988453\cdot
\begin{pmatrix}
\color{gr} 0.651561\color{black} \\
0.111444\\
0.072396\\
0.176281
\end{pmatrix},
\end{equation*}
\begin{equation*}
\left[ \frac{{w}^{\prime}_i}{{w}^{\prime}_j} \right] =
\begin{pmatrix}
$\,\,$ 1 $\,\,$ & $\,\,$\color{gr} 5.8465\color{black} $\,\,$ & $\,\,$\color{gr} \color{blue} 9\color{black} $\,\,$ & $\,\,$\color{gr} 3.6961\color{black} $\,\,$ \\
$\,\,$\color{gr} 0.1710\color{black} $\,\,$ & $\,\,$ 1 $\,\,$ & $\,\,$1.5394$\,\,$ & $\,\,$0.6322  $\,\,$ \\
$\,\,$\color{gr} \color{blue}  1/9\color{black} $\,\,$ & $\,\,$0.6496$\,\,$ & $\,\,$ 1 $\,\,$ & $\,\,$0.4107 $\,\,$ \\
$\,\,$\color{gr} 0.2706\color{black} $\,\,$ & $\,\,$1.5818$\,\,$ & $\,\,$2.4350$\,\,$ & $\,\,$ 1  $\,\,$ \\
\end{pmatrix},
\end{equation*}
\end{example}
\newpage
\begin{example}
\begin{equation*}
\mathbf{A} =
\begin{pmatrix}
$\,\,$ 1 $\,\,$ & $\,\,$6$\,\,$ & $\,\,$9$\,\,$ & $\,\,$7 $\,\,$ \\
$\,\,$ 1/6$\,\,$ & $\,\,$ 1 $\,\,$ & $\,\,$1$\,\,$ & $\,\,$2 $\,\,$ \\
$\,\,$ 1/9$\,\,$ & $\,\,$ 1 $\,\,$ & $\,\,$ 1 $\,\,$ & $\,\,$ 1/2 $\,\,$ \\
$\,\,$ 1/7$\,\,$ & $\,\,$ 1/2$\,\,$ & $\,\,$2$\,\,$ & $\,\,$ 1  $\,\,$ \\
\end{pmatrix},
\qquad
\lambda_{\max} =
4.1658,
\qquad
CR = 0.0625
\end{equation*}

\begin{equation*}
\mathbf{w}^{cos} =
\begin{pmatrix}
\color{red} 0.691969\color{black} \\
0.126011\\
0.079789\\
0.102232
\end{pmatrix}\end{equation*}
\begin{equation*}
\left[ \frac{{w}^{cos}_i}{{w}^{cos}_j} \right] =
\begin{pmatrix}
$\,\,$ 1 $\,\,$ & $\,\,$\color{red} 5.4914\color{black} $\,\,$ & $\,\,$\color{red} 8.6725\color{black} $\,\,$ & $\,\,$\color{red} 6.7686\color{black} $\,\,$ \\
$\,\,$\color{red} 0.1821\color{black} $\,\,$ & $\,\,$ 1 $\,\,$ & $\,\,$1.5793$\,\,$ & $\,\,$1.2326  $\,\,$ \\
$\,\,$\color{red} 0.1153\color{black} $\,\,$ & $\,\,$0.6332$\,\,$ & $\,\,$ 1 $\,\,$ & $\,\,$0.7805 $\,\,$ \\
$\,\,$\color{red} 0.1477\color{black} $\,\,$ & $\,\,$0.8113$\,\,$ & $\,\,$1.2813$\,\,$ & $\,\,$ 1  $\,\,$ \\
\end{pmatrix},
\end{equation*}

\begin{equation*}
\mathbf{w}^{\prime} =
\begin{pmatrix}
0.699086\\
0.123099\\
0.077945\\
0.099869
\end{pmatrix} =
0.976892\cdot
\begin{pmatrix}
\color{gr} 0.715623\color{black} \\
0.126011\\
0.079789\\
0.102232
\end{pmatrix},
\end{equation*}
\begin{equation*}
\left[ \frac{{w}^{\prime}_i}{{w}^{\prime}_j} \right] =
\begin{pmatrix}
$\,\,$ 1 $\,\,$ & $\,\,$\color{gr} 5.6791\color{black} $\,\,$ & $\,\,$\color{gr} 8.9689\color{black} $\,\,$ & $\,\,$\color{gr} \color{blue} 7\color{black} $\,\,$ \\
$\,\,$\color{gr} 0.1761\color{black} $\,\,$ & $\,\,$ 1 $\,\,$ & $\,\,$1.5793$\,\,$ & $\,\,$1.2326  $\,\,$ \\
$\,\,$\color{gr} 0.1115\color{black} $\,\,$ & $\,\,$0.6332$\,\,$ & $\,\,$ 1 $\,\,$ & $\,\,$0.7805 $\,\,$ \\
$\,\,$\color{gr} \color{blue}  1/7\color{black} $\,\,$ & $\,\,$0.8113$\,\,$ & $\,\,$1.2813$\,\,$ & $\,\,$ 1  $\,\,$ \\
\end{pmatrix},
\end{equation*}
\end{example}
\newpage
\begin{example}
\begin{equation*}
\mathbf{A} =
\begin{pmatrix}
$\,\,$ 1 $\,\,$ & $\,\,$6$\,\,$ & $\,\,$9$\,\,$ & $\,\,$8 $\,\,$ \\
$\,\,$ 1/6$\,\,$ & $\,\,$ 1 $\,\,$ & $\,\,$1$\,\,$ & $\,\,$2 $\,\,$ \\
$\,\,$ 1/9$\,\,$ & $\,\,$ 1 $\,\,$ & $\,\,$ 1 $\,\,$ & $\,\,$ 1/2 $\,\,$ \\
$\,\,$ 1/8$\,\,$ & $\,\,$ 1/2$\,\,$ & $\,\,$2$\,\,$ & $\,\,$ 1  $\,\,$ \\
\end{pmatrix},
\qquad
\lambda_{\max} =
4.1664,
\qquad
CR = 0.0627
\end{equation*}

\begin{equation*}
\mathbf{w}^{cos} =
\begin{pmatrix}
\color{red} 0.701619\color{black} \\
0.121742\\
0.079257\\
0.097382
\end{pmatrix}\end{equation*}
\begin{equation*}
\left[ \frac{{w}^{cos}_i}{{w}^{cos}_j} \right] =
\begin{pmatrix}
$\,\,$ 1 $\,\,$ & $\,\,$\color{red} 5.7631\color{black} $\,\,$ & $\,\,$\color{red} 8.8525\color{black} $\,\,$ & $\,\,$\color{red} 7.2048\color{black} $\,\,$ \\
$\,\,$\color{red} 0.1735\color{black} $\,\,$ & $\,\,$ 1 $\,\,$ & $\,\,$1.5361$\,\,$ & $\,\,$1.2502  $\,\,$ \\
$\,\,$\color{red} 0.1130\color{black} $\,\,$ & $\,\,$0.6510$\,\,$ & $\,\,$ 1 $\,\,$ & $\,\,$0.8139 $\,\,$ \\
$\,\,$\color{red} 0.1388\color{black} $\,\,$ & $\,\,$0.7999$\,\,$ & $\,\,$1.2287$\,\,$ & $\,\,$ 1  $\,\,$ \\
\end{pmatrix},
\end{equation*}

\begin{equation*}
\mathbf{w}^{\prime} =
\begin{pmatrix}
0.705067\\
0.120336\\
0.078341\\
0.096257
\end{pmatrix} =
0.988444\cdot
\begin{pmatrix}
\color{gr} 0.713310\color{black} \\
0.121742\\
0.079257\\
0.097382
\end{pmatrix},
\end{equation*}
\begin{equation*}
\left[ \frac{{w}^{\prime}_i}{{w}^{\prime}_j} \right] =
\begin{pmatrix}
$\,\,$ 1 $\,\,$ & $\,\,$\color{gr} 5.8592\color{black} $\,\,$ & $\,\,$\color{gr} \color{blue} 9\color{black} $\,\,$ & $\,\,$\color{gr} 7.3249\color{black} $\,\,$ \\
$\,\,$\color{gr} 0.1707\color{black} $\,\,$ & $\,\,$ 1 $\,\,$ & $\,\,$1.5361$\,\,$ & $\,\,$1.2502  $\,\,$ \\
$\,\,$\color{gr} \color{blue}  1/9\color{black} $\,\,$ & $\,\,$0.6510$\,\,$ & $\,\,$ 1 $\,\,$ & $\,\,$0.8139 $\,\,$ \\
$\,\,$\color{gr} 0.1365\color{black} $\,\,$ & $\,\,$0.7999$\,\,$ & $\,\,$1.2287$\,\,$ & $\,\,$ 1  $\,\,$ \\
\end{pmatrix},
\end{equation*}
\end{example}
\newpage
\begin{example}
\begin{equation*}
\mathbf{A} =
\begin{pmatrix}
$\,\,$ 1 $\,\,$ & $\,\,$6$\,\,$ & $\,\,$9$\,\,$ & $\,\,$9 $\,\,$ \\
$\,\,$ 1/6$\,\,$ & $\,\,$ 1 $\,\,$ & $\,\,$1$\,\,$ & $\,\,$2 $\,\,$ \\
$\,\,$ 1/9$\,\,$ & $\,\,$ 1 $\,\,$ & $\,\,$ 1 $\,\,$ & $\,\,$ 1/2 $\,\,$ \\
$\,\,$ 1/9$\,\,$ & $\,\,$ 1/2$\,\,$ & $\,\,$2$\,\,$ & $\,\,$ 1  $\,\,$ \\
\end{pmatrix},
\qquad
\lambda_{\max} =
4.1707,
\qquad
CR = 0.0644
\end{equation*}

\begin{equation*}
\mathbf{w}^{cos} =
\begin{pmatrix}
\color{red} 0.709393\color{black} \\
0.118275\\
0.078832\\
0.093500
\end{pmatrix}\end{equation*}
\begin{equation*}
\left[ \frac{{w}^{cos}_i}{{w}^{cos}_j} \right] =
\begin{pmatrix}
$\,\,$ 1 $\,\,$ & $\,\,$\color{red} 5.9978\color{black} $\,\,$ & $\,\,$\color{red} 8.9988\color{black} $\,\,$ & $\,\,$\color{red} 7.5871\color{black} $\,\,$ \\
$\,\,$\color{red} 0.1667\color{black} $\,\,$ & $\,\,$ 1 $\,\,$ & $\,\,$1.5003$\,\,$ & $\,\,$1.2650  $\,\,$ \\
$\,\,$\color{red} 0.1111\color{black} $\,\,$ & $\,\,$0.6665$\,\,$ & $\,\,$ 1 $\,\,$ & $\,\,$0.8431 $\,\,$ \\
$\,\,$\color{red} 0.1318\color{black} $\,\,$ & $\,\,$0.7905$\,\,$ & $\,\,$1.1861$\,\,$ & $\,\,$ 1  $\,\,$ \\
\end{pmatrix},
\end{equation*}

\begin{equation*}
\mathbf{w}^{\prime} =
\begin{pmatrix}
0.709421\\
0.118263\\
0.078825\\
0.093491
\end{pmatrix} =
0.999904\cdot
\begin{pmatrix}
\color{gr} 0.709489\color{black} \\
0.118275\\
0.078832\\
0.093500
\end{pmatrix},
\end{equation*}
\begin{equation*}
\left[ \frac{{w}^{\prime}_i}{{w}^{\prime}_j} \right] =
\begin{pmatrix}
$\,\,$ 1 $\,\,$ & $\,\,$\color{gr} 5.9987\color{black} $\,\,$ & $\,\,$\color{gr} \color{blue} 9\color{black} $\,\,$ & $\,\,$\color{gr} 7.5881\color{black} $\,\,$ \\
$\,\,$\color{gr} 0.1667\color{black} $\,\,$ & $\,\,$ 1 $\,\,$ & $\,\,$1.5003$\,\,$ & $\,\,$1.2650  $\,\,$ \\
$\,\,$\color{gr} \color{blue}  1/9\color{black} $\,\,$ & $\,\,$0.6665$\,\,$ & $\,\,$ 1 $\,\,$ & $\,\,$0.8431 $\,\,$ \\
$\,\,$\color{gr} 0.1318\color{black} $\,\,$ & $\,\,$0.7905$\,\,$ & $\,\,$1.1861$\,\,$ & $\,\,$ 1  $\,\,$ \\
\end{pmatrix},
\end{equation*}
\end{example}

\end{document}
